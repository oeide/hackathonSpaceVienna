\documentclass[11pt,twoside]{article}\makeatletter

\IfFileExists{xcolor.sty}%
  {\RequirePackage{xcolor}}%
  {\RequirePackage{color}}
\usepackage{colortbl}
      
\usepackage{fontspec}
\usepackage{xunicode}
\catcode`⃥=\active \def⃥{\textbackslash}
\catcode`❴=\active \def❴{\{}
\catcode`❵=\active \def❵{\}}
\def\textJapanese{\fontspec{Kochi Mincho}}
\def\textChinese{\fontspec{HAN NOM A}\XeTeXlinebreaklocale "zh"\XeTeXlinebreakskip = 0pt plus 1pt }
\def\textKorean{\fontspec{Baekmuk Gulim} }
\setmonofont{DejaVu Sans Mono}
\setsansfont{Arial}
\setromanfont{Times New Roman}

\DeclareTextSymbol{\textpi}{OML}{25}
\usepackage{relsize}
\def\textsubscript#1{%
  \@textsubscript{\selectfont#1}}
\def\@textsubscript#1{%
  {\m@th\ensuremath{_{\mbox{\fontsize\sf@size\z@#1}}}}}
\def\textquoted#1{‘#1’}
\def\textsmall#1{{\small #1}}
\def\textlarge#1{{\large #1}}
\def\textoverbar#1{\ensuremath{\overline{#1}}}
\def\textgothic#1{{\fontspec{Lucida Blackletter}#1}}
\def\textcal#1{{\fontspec{Lucida Calligraphhy}#1}}
\RequirePackage{array}
\def\@testpach{\@chclass
 \ifnum \@lastchclass=6 \@ne \@chnum \@ne \else
  \ifnum \@lastchclass=7 5 \else
   \ifnum \@lastchclass=8 \tw@ \else
    \ifnum \@lastchclass=9 \thr@@
   \else \z@
   \ifnum \@lastchclass = 10 \else
   \edef\@nextchar{\expandafter\string\@nextchar}%
   \@chnum
   \if \@nextchar c\z@ \else
    \if \@nextchar l\@ne \else
     \if \@nextchar r\tw@ \else
   \z@ \@chclass
   \if\@nextchar |\@ne \else
    \if \@nextchar !6 \else
     \if \@nextchar @7 \else
      \if \@nextchar (8 \else
       \if \@nextchar )9 \else
  10
  \@chnum
  \if \@nextchar m\thr@@\else
   \if \@nextchar p4 \else
    \if \@nextchar b5 \else
   \z@ \@chclass \z@ \@preamerr \z@ \fi \fi \fi \fi
   \fi \fi  \fi  \fi  \fi  \fi  \fi \fi \fi \fi \fi \fi}

\gdef\arraybackslash{\let\\=\@arraycr}
\def\Panel#1#2#3#4{\multicolumn{#3}{){\columncolor{#2}}#4}{#1}}
\gdef\arraybackslash{\let\\=\@arraycr}

\newcolumntype{L}[1]{){\raggedright\arraybackslash}p{#1}}
\newcolumntype{C}[1]{){\centering\arraybackslash}p{#1}}
\newcolumntype{R}[1]{){\raggedleft\arraybackslash}p{#1}}
\newcolumntype{P}[1]{){\arraybackslash}p{#1}}
\newcolumntype{B}[1]{){\arraybackslash}b{#1}}
\newcolumntype{M}[1]{){\arraybackslash}m{#1}}
\definecolor{label}{gray}{0.75}
\newenvironment{reflist}{%
  \begin{raggedright}\begin{list}{}
  {%
   \setlength{\topsep}{0pt}%
   \setlength{\rightmargin}{0.25in}%
   \setlength{\itemsep}{0pt}%
   \setlength{\itemindent}{0pt}%
   \setlength{\parskip}{0pt}%
   \setlength{\parsep}{2pt}%
   \def\makelabel##1{\itshape ##1}}%
  }
  {\end{list}\end{raggedright}}
\newenvironment{sansreflist}{%
  \begin{raggedright}\begin{list}{}
  {%
   \setlength{\topsep}{0pt}%
   \setlength{\rightmargin}{0.25in}%
   \setlength{\itemindent}{0pt}%
   \setlength{\parskip}{0pt}%
   \setlength{\itemsep}{0pt}%
   \setlength{\parsep}{2pt}%
   \def\makelabel##1{\upshape\sffamily ##1}}%
  }
  {\end{list}\end{raggedright}}
\newenvironment{specHead}[2]%
 {\vspace{20pt}\hrule\vspace{10pt}%
  \hypertarget{#1}{}%
  \markright{#2}%
  \pdfbookmark[2]{#2}{#1}%
  \hspace{-0.75in}{\bfseries\fontsize{16pt}{18pt}\selectfont#2}%
  }{}
\DeclareRobustCommand*{\xref}{\hyper@normalise\xref@}
\def\xref@#1#2{\hyper@linkurl{#2}{#1}}
\def\Div[#1]#2{\section*{#2}}
\catcode`\_=12\relax

\usepackage[a4paper,twoside,lmargin=1in,rmargin=1in,tmargin=1in,bmargin=1in]{geometry}
\usepackage{framed}
\definecolor{shadecolor}{gray}{0.95}
\usepackage{longtable}
\usepackage[normalem]{ulem}
\usepackage{fancyvrb}
\usepackage{fancyhdr}
\usepackage{graphicx}

\def\Gin@extensions{.pdf,.png,.jpg,.mps,.tif}

  \pagestyle{fancy} 

	 \paperwidth210mm
	 \paperheight297mm
              
\def\@pnumwidth{1.55em}
\def\@tocrmarg {2.55em}
\def\@dotsep{4.5}
\setcounter{tocdepth}{3}
\clubpenalty=8000
\emergencystretch 3em
\hbadness=4000
\hyphenpenalty=400
\pretolerance=750
\tolerance=2000
\vbadness=4000
\widowpenalty=10000

\renewcommand\section{\@startsection {section}{1}{\z@}%
     {-1.75ex \@plus -0.5ex \@minus -.2ex}%
     {0.5ex \@plus .2ex}%
     {\reset@font\Large\bfseries\sffamily}}
\renewcommand\subsection{\@startsection{subsection}{2}{\z@}%
     {-1.75ex\@plus -0.5ex \@minus- .2ex}%
     {0.5ex \@plus .2ex}%
     {\reset@font\Large\sffamily}}
\renewcommand\subsubsection{\@startsection{subsubsection}{3}{\z@}%
     {-1.5ex\@plus -0.35ex \@minus -.2ex}%
     {0.5ex \@plus .2ex}%
     {\reset@font\large\sffamily}}
\renewcommand\paragraph{\@startsection{paragraph}{4}{\z@}%
     {-1ex \@plus-0.35ex \@minus -0.2ex}%
     {0.5ex \@plus .2ex}%
     {\reset@font\normalsize\sffamily}}
\renewcommand\subparagraph{\@startsection{subparagraph}{5}{\parindent}%
     {1.5ex \@plus1ex \@minus .2ex}%
     {-1em}%
     {\reset@font\normalsize\bfseries}}


\def\l@section#1#2{\addpenalty{\@secpenalty} \addvspace{1.0em plus 1pt}
 \@tempdima 1.5em \begingroup
 \parindent \z@ \rightskip \@pnumwidth 
 \parfillskip -\@pnumwidth 
 \bfseries \leavevmode #1\hfil \hbox to\@pnumwidth{\hss #2}\par
 \endgroup}
\def\l@subsection{\@dottedtocline{2}{1.5em}{2.3em}}
\def\l@subsubsection{\@dottedtocline{3}{3.8em}{3.2em}}
\def\l@paragraph{\@dottedtocline{4}{7.0em}{4.1em}}
\def\l@subparagraph{\@dottedtocline{5}{10em}{5em}}
\@ifundefined{c@section}{\newcounter{section}}{}
\@ifundefined{c@chapter}{\newcounter{chapter}}{}
\newif\if@mainmatter 
\@mainmattertrue
\def\chaptername{Chapter}
\def\frontmatter{%
  \pagenumbering{roman}
  \def\thechapter{\@roman\c@chapter}
  \def\theHchapter{\alph{chapter}}
  \def\@chapapp{}%
}
\def\mainmatter{%
  \cleardoublepage
  \def\thechapter{\@arabic\c@chapter}
  \setcounter{chapter}{0}
  \setcounter{section}{0}
  \pagenumbering{arabic}
  \setcounter{secnumdepth}{6}
  \def\@chapapp{\chaptername}%
  \def\theHchapter{\arabic{chapter}}
}
\def\backmatter{%
  \cleardoublepage
  \setcounter{chapter}{0}
  \setcounter{section}{0}
  \setcounter{secnumdepth}{0}
  \def\@chapapp{\appendixname}%
  \def\thechapter{\@Alph\c@chapter}
  \def\theHchapter{\Alph{chapter}}
  \appendix
}
\newenvironment{bibitemlist}[1]{%
   \list{\@biblabel{\@arabic\c@enumiv}}%
       {\settowidth\labelwidth{\@biblabel{#1}}%
        \leftmargin\labelwidth
        \advance\leftmargin\labelsep
        \@openbib@code
        \usecounter{enumiv}%
        \let\p@enumiv\@empty
        \renewcommand\theenumiv{\@arabic\c@enumiv}%
	}%
  \sloppy
  \clubpenalty4000
  \@clubpenalty \clubpenalty
  \widowpenalty4000%
  \sfcode`\.\@m}%
  {\def\@noitemerr
    {\@latex@warning{Empty `bibitemlist' environment}}%
    \endlist}

\def\tableofcontents{\section*{\contentsname}\@starttoc{toc}}
\usepackage[pdftitle={Schnitlers grenseexaminasjonsprotokoller},
 pdfauthor={Peter Schnitler}]{hyperref}
\hyperbaseurl{}
\parskip0pt
\parindent1em
\@ifundefined{chapter}{%
    \def\DivI{\section}
    \def\DivII{\subsection}
    \def\DivIII{\subsubsection}
    \def\DivIV{\paragraph}
    \def\DivV{\subparagraph}
    \def\DivIStar[#1]#2{\section*{#2}}
    \def\DivIIStar[#1]#2{\subsection*{#2}}
    \def\DivIIIStar[#1]#2{\subsubsection*{#2}}
    \def\DivIVStar[#1]#2{\paragraph*{#2}}
    \def\DivVStar[#1]#2{\subparagraph*{#2}}
}{%
    \def\DivI{\chapter}
    \def\DivII{\section}
    \def\DivIII{\subsection}
    \def\DivIV{\subsubsection}
    \def\DivV{\paragraph}
    \def\DivIStar[#1]#2{\chapter*{#2}}
    \def\DivIIStar[#1]#2{\section*{#2}}
    \def\DivIIIStar[#1]#2{\subsection*{#2}}
    \def\DivIVStar[#1]#2{\subsubsection*{#2}}
    \def\DivVStar[#1]#2{\paragraph*{#2}}
}
\def\TheFullDate{2011-01-14}
\def\TheID{\makeatother }
\def\TheDate{2011-01-14}
\title{Schnitlers grenseexaminasjonsprotokoller}
\author{Peter Schnitler}\let\tabcellsep&
      \catcode`\&=12\relax \makeatletter 
\makeatletter
\thispagestyle{plain}\markright{\@title}\markboth{\@title}{\@author}
\renewcommand\small{\@setfontsize\small{9pt}{11pt}\abovedisplayskip 8.5\p@ plus3\p@ minus4\p@
   \belowdisplayskip \abovedisplayskip
   \abovedisplayshortskip \z@ plus2\p@
   \belowdisplayshortskip 4\p@ plus2\p@ minus2\p@
   \def\@listi{\leftmargin\leftmargini
               \topsep 2\p@ plus1\p@ minus1\p@
               \parsep 2\p@ plus\p@ minus\p@
               \itemsep 1pt}
}
\makeatother
\fvset{frame=single,numberblanklines=false,xleftmargin=5mm,xrightmargin=5mm}
\fancyhf{} 
\setlength{\headheight}{14pt}
\fancyhead[LE]{\bfseries\leftmark} 
\fancyhead[RO]{\bfseries\rightmark} 
\fancyfoot[RO]{\TheFullDate}
\fancyfoot[CO]{\thepage}
\fancyfoot[LO]{\TheID}
\fancyfoot[LE]{\TheFullDate}
\fancyfoot[CE]{\thepage}
\fancyfoot[RE]{\TheID}
\hypersetup{linkbordercolor=0.75 0.75 0.75,urlbordercolor=0.75 0.75 0.75,bookmarksnumbered=true}
\fancypagestyle{plain}{\fancyhead{}\renewcommand{\headrulewidth}{0pt}}\makeatother 
\begin{document}

\makeatletter
\noindent\parbox[b]{.75\textwidth}{\fontsize{14pt}{16pt}\bfseries\raggedright\sffamily\selectfont \@title}
\vskip20pt
\par\noindent{\fontsize{11pt}{13pt}\sffamily\itshape\raggedright\selectfont\@author\hfill\TheDate}
\vspace{18pt}
\makeatother

\catcode`\$=12\relax
\catcode`\^=12\relax
\catcode`\~=12\relax
\catcode`\#=12\relax
\catcode`\%=12\relax

\DivI[Forord]{Forord}\label{Schn1_8}\par
\centerline{FORORD}\par
Etter forslag, opprinnelig i promemoria 25. mai 1916, av rendriftsinspektør, senere domprost og sogneprest \textit{Kristian Nissen} tok Styret for Kjeldeskriftfondet i 1923 opp på sitt program å utgi de protokoller for major Peter Schnitlers «grenseeksaminasjoner» 1742‒45, som er oppbevart i Riksarkivets grenseregulerings-arkiv.\par
En del av disse protokoller var tidligere utgitt av den norsk-svenske \textit{Renbeitekommisjon} av 28. september 1907; de utgjorde side 1‒430 i første bind av dennes publikasjon «Dokumenter angaaende flytlapperne m. m. samlede efter renbeitekommissionens opdrag af rektor \textit{J. Qvigstad} og professor \textit{K. B. Wiklund}», Kristiania1909. Av de 6 «volumina» som Schnitlers «eksaminasjoner» (vidne-avhøringer) omfattet, var det den senere del av vol. 2, samt størsteparten av vol. 3 og 4, som var trykt av Renbeitekommisjonen. Utrykt var altså vol. 1, begynnelsen av vol. 2, noen små partier av vol. 3 og 4, samt hele vol. 5‒6.\par
Renbeitekommisjonens bind var trykt i et overflødig stort opplag, hvorav meget over 1000 eksemplarer henlå i materie i Landbruksdepartementet. Nissens tanke, som myndighetene godtok, var at 1000 eksemplarer av dette trykk, ved oppsetting av nytt titelblad og ved å fjerne s. 431‒538 og erstatte dem med en register-avdeling, skulle omgjøres til bind II i en samlet utgave av Schnitlers protokoller. Det som gikk forut for de av Renbeitekommisjonen trykte protokoller, nemlig vol. 1 og begynnelsen av vol. 2, skulle utgjøre utgavens bind I, og det som kom etterpå, skulle utgjøre bind III. For begge disse bind tilbød Nissen i 1922 å stå som utgiver og utarbeide de nødvendige registre, mens Qvigstad skulle besørge registerarbeidet for bind II.\par
Til teksten i det planlagte bind I forelå det allerede manuskript, idet Nissen1921‒22 etter oppdrag hadde skaffet Utenriksdepartementet fullstendige avskrifter av dem fra Riksarkivet, i anledning av undersøkelser som pågikk. Nissen hadde selv foreslått at gjennomslag av disse avskrifter skulle sendes Styret for Kjeldeskriftfondet med henblikk på mulig utgivelse, og Utenriksdepartementet hadde ikke bare bifalt dette, men også foranlediget særskilte stortingsbevilgninger til utgivelsen, kr. 3 000 i 1922 og samme beløp i 1923. Styret for Kjeldeskriftfondet fikk manuskriptet i slutten av 1922, og det ble satt hos Grøndahl & Søn 1923‒24.\par
Det viste seg imidlertid at Nissens tid ble så sterkt beslaglagt av andre offentlige oppdrag at han hverken fikk anledning til å ta seg av korrekturen eller til å utarbeide de planlagte registre. Da Nissen i 1926 ble domprost på Tromsø, og dermed varig avskåret fra å besøke Riksarkivet, engasjerte Styret førstearkivar i Riksarkivet, senere sogneprest H. Blom Svendsen til å lese korrekturen på det som var satt. Dette ble så rentrykt (men ikke utgitt) 1927.\hypertarget{Schn1_126}{}Forord.\par
Våren 1929 leverte Qvigstad ferdig manuskript til stedsnavnregister til bind II; og samme år ble bind II utgitt etter planen. Den opprinnelige plan for bind I ble derimot oppgitt, fordi det ved trykningen hadde vist seg at dette, når det bare skulle inneholde Schnitlers vol. 1 og delvis vol. 2, ville bli et uforholdsmessig tynt bind. Styret besluttet isteden, at \textit{alle} de volumina og fragmenter av slike, som ikke var trykt i bind II, skulle utgis \textit{samlet} som bind I. Men da man fortsatt ønsket å ha Nissen som utgiver, kunne dette først for alvor tas opp etter at Nissen i 1949 hadde tatt avskjed som embedsmann. Offentlige oppdrag la imidlertid også senere sterkt beslag på hans tid. Med Nissens samtykke ble det derfor besluttet å la vitenskapelig medarbeider ved Norsk Historisk Kjeldeskrift-Institutt, cand. theol. \textit{Ingolf Kvamen} overta en vesentlig del av det arbeide som stod igjen. I 1952 ble det tatt avskrifter i Riksarkivet, for størstedelen ved Kvamen selv og ellers under hans tilsyn, av de protokoller som ikke tidligere var avskrevet, og dette nye manuskript ble i desember 1952 levert Grøndahl & Søn til trykning. Det ble imidlertid ikke satt; Grøndahl & Søn meddelte i april 1953, at trykkeriet på grunn av mannskapsmangel ikke hadde mulighet for å utføre arbeidet. Etter lengere stans fikk man det videre trykkeriarbeide anbrakt hos A/S Dreyer i Stavanger; det lyktes her å finne en skrift som var så lik den tidligere anvendte, at skiftet av trykkeri midt i bindet ikke ble iøynefallende. Forholdet er nå det, at side 1‒144 i bind I, samt bind II er trykt hos Grøndahl & Søn, mens side 145‒478 i bind I, samt hele den romertall-paginerte del av dette bind, er trykt hos Dreyer.\par
Korrekturlesningen på den del av bindets tekst som er trykt hos Dreyer, er i sin helhet besørget av Kvamen. Videre er begge registre til bind I (stedsnavnregister og personnavnregister) utarbeidet og korrekturlest av Kvamen. Innledningen er forfattet av Nissen, som har lagt et meget stort arbeide i dette; hans opprinnelig utarbeidede manuskript, som hadde karakter av en fullstendig fremstilling av den norske riksgrenses historie, var særdeles omfangsrikt (anslagsvis 250 sider i utgavens format), og Styret beklager at det av hensyn til utgavens omfang og til bevilgningens formål ‒ utgivelse av kildeskrifter, ikke avhandlingsstoff ‒ ble funnet nødvendig å be Nissen om å foreta en meget sterk nedskjæring av innledningen. Styret har det håp at Nissens meget verdifulle grensehistoriske manuskript vil kunne bli trykt annetsteds innenfor rammen av bevilgninger til vitenskapelig forskning.\par
For bind II henviser Styret til bindets eget forord, undertegnet av J. Qvigstad og K. B. Wiklund.\hspace{1em}\par
30. desember 1961. {For Styret for Kjeldeskriftfondet \textit{Jonas Jansen}.}
\DivI[Innholdsfortegnelse]{Innholdsfortegnelse}\label{Schn1_235}\par
\hypertarget{Schn1_237}{}\footnote{\label{Schn1_237}Innholdsfortegnelsen, dvs. s. VII-XII, er utelatt. Det kan genereres ut fra kapitlenes overskrift-attributter.}
\DivI[Innledning av Kristian Nissen.]{Innledning av Kristian Nissen.}\label{Schn1_242}\par
\centerline{\textbf{INNLEDNING} AV \textit{Kristian Nissen}}\hspace{1em}
\DivI[I. Grenseforhandlingene av 1734 mellem Norge og Sverige]{I. Grenseforhandlingene av 1734 mellem Norge og Sverige}\label{Schn1_255}\par
\centerline{\textbf{I. Grenseforhandlingene av 1734 mellem Norge og Sverige.}}
\DivII[A. Forhandlingenes forhistorie]{A. Forhandlingenes forhistorie}\label{Schn1_271}\par
A. FORHANDLINGENES FORHISTORIE\par
Norges nuværende riksgrense, landegrensen på den skandinaviske halvø, har en meget lang tilblivelseshistorie.\hypertarget{Schn1_280}{}\footnote{\label{Schn1_280}Den fyldigste fremstilling av Norges grensehistorie i sin helhet er fremdeles den som er gitt i E. J. Jessen-Schardeböll: Det Kongerige Norge, I, Kbh.1763, s. 161‒263: Cap. IV. Om Norges Grændser.\par
Av betydelig interesse er også Nils Marelius' avhandlinger i Kungl. Wetenskaps-Academiens Handlingar 1771: «Om Land- och Fjällryggarne i Sverige och Norrige» og «Om Gränsen imellan Sverige och Norrige», samt hans avhandling sammesteds 1763: «Anmärkningar Rörande Herjedalens och Jämtlands Gränts emot Norrige».\par
Det samme gjelder Rudolf Kjellén: Studier öfver Sveriges politiska gränser, i Ymer 1899, s. 283‒331, jfr. samme forfatters Inledning till Sveriges geografi, Göteborg1900, spesielt avsnitt III: Det nuvarande territoriets folkrättsliga omfatning (s. 64‒89).}\par
Utviklingen var ikke tilendebragt og forholdet ikke bragt til klarhet og ro, da Norge i 1645 og 1658‒60 for siste gang i vår historie måtte avstå landområder til Sverige som følge av krig. Både sønnenfjells og enn mere nordenfjells, for ikke å tale om forholdene i Nordlandene og Finnmarken, var grensen ennu på store strekninger ubestemt, og først i 1826, efter vidtløftige undersøkelser og langvarige forhandlinger, fikk Norge hele sin landegrense i all hovedsak bestemt.\par
Ved fredsslutningen i 1326 mellem Norge og Novgorod (Russland) var man blitt enig om at de gamle grenser mellem rikene skulde respekteres. Men hvor de gamle grenseskjell var, sies der intet om i traktaten. Og den i denne uttalte forutsetning, at gesandter skulde sendes fra Novgorod til Norges konge for å fastslå de gamle grensemerker, blev ikke gjort til virkelighet.\hypertarget{Schn1_403}{}\footnote{\label{Schn1_403}Se O. A. Johnsen: Finmarkens politiske Historie, Kra. 1923, s. 24 flg. Dette helt på førstehånds arkivstudier byggede arbeide er hovedverket for Finnmarks-grensens vedkommende.}\par
Syvårskrigen 1563‒70 og Kalmarkrigen 1611‒13 sluttet begge uten landavståelser. Traktaten til Knærød20. januar 1613, hvor Sverige opgav sine krav på herredømme «ved søsiden i Norgis rige udi forschrefne Norland och Vardøehus lenne fraa Tittisfiord indtil Veranger», førte ikke til nogen eksakt grensebestemmelse mellem rikene. Selv da Norge ved Brømsebrofreden av 13. august 1645 måtte avstå til Sverige «Jemptelandh med Herredalen inclusive, så vidt befinnes af benämbde Herredalen att ligge på den Svenske sijdan om \hypertarget{Schn1_463}{}Innledning. fiällen», blev denne generelle bestemmelse ikke fulgt av en nærmere fastsettelse av den nye grenselinje.\hypertarget{Schn1_465}{}\footnote{\label{Schn1_465}For traktatenes tekst, se Laursen: Danmark-Norges Traktater; Rydberg: Sverges traktater med främmande magter.}\par
Det var først ved Roskilde-freden 26. februar 1658, da avståelsen gjaldt Trøndelagen og Bohuslen (artikel 6: «Bohus slått och lähn sampt Trundhjems gård och lähn»), at traktaten kom til å inneholde bestemmelser om grenseopgang og detaljfastsettelse av de nye riksgrenser. I artikel 10 blev bestemt at «sedan medh det förste förordnas visse fullmechtige, . . . . . . . som kunne besichtiga, åthskillia och adjustera alle gräntzer, råmärchen och tvister emillan de afträdde och inrymbde och dernäst belägne behåldne provincier och lähn . . . » I samsvar med denne bestemmelse blev grensene mellem det avståtte Trondhjems len og de tilstøtende norske landsdeler fastsatt så detaljert som man efter tidens krav og forhold fant nødvendig, ved overenskomster mai‒juli 1658.\hypertarget{Schn1_503}{}\footnote{\label{Schn1_503}Se Yngvar Nielsen: Kampen om Trondhjem1657‒1660, Trh.1897, s. 93‒104 og 167‒176.}\par
I juni‒juli blev der også opnevnt kommissærer for grensen mellem Bohuslen og Smålenene. Men innen disse fikk påbegynt arbeidet, begynte krigen på ny ‒ og med det resultat at Norge, ved freden i København27. mai 1660, fikk Trondhjems len tilbake. Avståelsen av Bohuslen blev derimot stadfestet, og i traktatens artikel 14 blev bestemmelsen om fastsettelse av grensen for de avståtte områder gjentatt. I henhold til dette blev der i juli 1660 opnevnt kommissærer, og ved overenskomst i Nasselbacka i Naverstad sogn i Bohuslen26. oktober 1661 blev riksgrensen fastsatt fra havet og så langt inn i landet som til Øvre Kornsjø, hvor Dalsland begynner. Overenskomsten var ledsaget av et offisielt kart i målestokk 1:50 000; derimot blev grenselinjen ikke avmerket ved røser eller grensegate.\par
Gyldenløvefeiden 1675‒79 sluttet uten landavståelser; men i fredstraktaten i Lund blev der i artikel 11 tatt forbehold om ransakning av den norsk-svenske grense ved opnevnte kommissærer hvis nogen av partene kom med klage over nogen formentlig uriktighet. Men nogen klage kom ikke. Det betyr dog ikke at der ikke forelå muligheter for tvil og tvist om grensene; under den store nordiske krig blev således tvistemål m. h. t. Finnmarks-grensen påberopt i den dansk-norske krigserklæring til Sverige28. oktober 1709. Og i fredstraktaten i Stockholm3. juni 1720 blev der i artikel 14, under henvisning til slike tvistemål, bestemt at kommissærer skulde opnevnes og innen 3 måneder reise til stedet for å justere grensene mellem «begge Finmarkerne», i samsvar med tidligere fredstraktater og tilhørende grensebestemmelser.\par
Slike forføininger blev ikke truffet, og saken drøiet ut. Ved promemoria 15. februar 1725 lot dog Sverige meddele i København at kongen tenkte å opnevne folk «till afgiörande af de Norrska gräntze-tvistigheterna» og henstillet til den dansk-norske konge å foreta tilsvarende opnevnelse. Tilsynelatende gjaldt henvendelsen alle tvistespørsmål både i nord og syd; men det har neppe vært tenkt på annet enn visse bestemte punkter i det sydlige Norge. Statholder Ditlef Wibe, som fikk saken forelagt, rådet til grundige arkivstudier før man innlot sig med de svenske. Og noget svar til Sverige vites ikke å være gitt. 22. november 1728 tok Sverige saken op mere spesifisert: nu gjaldt det, under henvisning til hyppige klagemål, å justere grensen «på \hypertarget{Schn1_620}{}Innledning. Wermeländska sijdan». Wibe advarte på ny, fordi det på norsk side efter hans erfaringer ikke fantes nogen bevisligheter eller adkomstbrev, mens svenskene skulde være forsynt med «tilstrekkelige ældgamle Documenter». Og regjeringens svar 15. januar 1729 gikk ut på at Vermlands-grensen fikk vente, til man i henhold til bestemmelsen fra 1720 hadde fått avgjort Finnmarks-grensen. Denne gang svarte Sverige ‒ såvidt vites ‒ ikke.\par
Det var først da Danmark-Norge og Sverige ved den defensive alliansetraktat av 1734 var trådt i positivt vennskapelig forhold til hverandre, at bestemmelsen fra 1720 blev gjort til virkelighet. Men da også i et langt videre omfang enn forutsatt i 1720-traktaten.
\DivII[B. Traktaten av 1734 og forhandlingenes igangsettelse]{B. Traktaten av 1734 og forhandlingenes igangsettelse}\label{Schn1_659}\par
B. TRAKTATEN AV 1734 OG FORHANDLINGENES IGANGSETTELSE\par
I den defensive alliansetraktat mellem Danmark-Norge og Sverige av 24. september/ 5. oktober 1734 er det i artikel 21 henvist til «adskillige Irringer» med hensyn til «Grentzeskilden imellem Wermeland, Dahl og Bahuus Leen, og den dertil stødende Deel af Norge», og avtalt å la befullmektigede kommissærer fra begge sider undersøke og avgjøre stridighetene «paa alle de Pladser, hvor nogen Tvist paa Norske Grentzen være kunde». Det forutsattes at undersøkelsene blev påbegynt så snart, at de kunde avsluttes innen høist 3 år.\par
Initiativet til defensiv-alliansen var tatt av Danmark-Norge i 1731; men det første danske traktatutkast nevner ikke riksgrensen. Denne er først omtalt i et svensk kontra-utkast, hvor det efter omtalen av tvister angående Vermlands, Dalslands  og Bohuslens norske grense forutsettes opnevnt kommissærer til å behandle «oberwehnte streitigkeiten». Som det vil sees, er det her som i 1728 bare strekningen fra Vermland og sydover, svenskene ønsker behandlet; det dansk-norske krav fra 1729 om et grenseopgjør først og fremst for Finnmarken er helt neglisjert. Dette er også tilfellet i et nytt utkast som de svenske forhandlere fremla under selve traktatforhandlingene 1734, så sent som 19. august. Når artikel 21 i sin endelige form allikevel fastsetter undersøkelser på \textit{alle} plasser «hvor nogen Tvist på Norske Grentzen være kunde», skyldes det utvilsomt krav fra dansk-norsk side om at grenseforhandlingene først og fremst måtte gjelde Finnmarken. Da den danske ordlyd her ser ut til å være en oversettelse av uttrykk i den svenske tekst, er det grunn til å tro at de svenske forhandlere ikke har gått med på uttrykkelig å nevne Finnmarks-grensen, men istedet har budt frem den mere generelle og ubestemte form, som så er akseptert fra dansk side og i oversettelse tatt inn i den danske tekst.\par
Det må fremheves ‒ hvad eldre fremstillinger overser ‒ at traktaten \textit{kun} taler om avgjørelse av grensetvister, ikke om å fastsette og avmerke den eksakte grenselinje fra Bohuslen til Finnmarken ‒ et arbeide som ingen kunde tro det mulig å fullføre innen 3 års forløp.\hspace{1em}\par
Med opnevnelsen av kommissærer forhastet man sig ikke. 21. juli 1735 blev generalmajor, friherre Gustaf Fredrik von \textit{Rosen} opnevnt til svensk kommissær. Men først 14. mai 1736 opnevnte Christian VI generalmajor Friderich Otto von \textit{Rappe}, oberst Michael \textit{Sundt} og oberstløitnant Peder \textit{Colbiørnsen} som Danmark-Norges kommissærer. I opnevnelsesbrevet for disse kommer den opfatning første gang til uttrykk, at ikke bare tvistemålene skal søkes \hypertarget{Schn1_782}{}Innledning løst, men at man også skal søke å fastslå, «hvorledes den rette Grændse-Skæl og Strækning både Norden- og Syndenfields bør være og falde». Ennu tydeligere kommer dette frem i en annen skrivelse s. d., hvori kongen pålegger kommissærene å komme med forslag på to konduktører til å «optage og forfatte et fuldkommen Cart og Tegning over Grændse-Skiællet og Strekningen imellem Norge og Sverrig paa begge sider», sønnenfra og nordover.\par
I et brev 8. juni 1736 til Rappe fremhevet oversekretæren J. L. von Holstein viktigheten av å begynne kart-arbeidet sønnenfjells, men også uten ophold å fortsette for hele grensestrekningen, da «der haves den Efterretning, at de Svensske skal meest have deris Øyemerke paa Grændse skiellet ved Wermeland, uden sig det øvrige med nogen synderlig alvorlighed at antage». Dette bekrefter jo det inntrykk de foregående forhandlinger hadde efterlatt.\par
I brev 1. august 1736 foreslo kommissærene å opnevne som konduktører overkonduktør i fortifikasjonsetaten, ingeniørkaptein Thomas Hans Henrich \textit{Knoff} og konduktør Jochum (Joachim) \textit{Wagel}, og anbefalte tillike opnevnt to underkonduktører, for å påskynde arbeidet. I svarbrev 24. august 1736 bifalt kongen forslaget om opnevnelse av Knoff og Wagel, og lovet å beordre to «Verkbaser» (en underordnet offisersgrad i fortifikasjonsetaten) til hjelp. I det siste brev blev også behandlet det viktige spørsmål om målemetoden: kommissærene hadde spurt om grensestrekningen skulde «optages Geometrice, hvor alle linjer, vinkler og Krumninger anføres udi deres netto Længde og viide» ‒ eller bare «Geographice», med notering og nedtegning bare efter kyndige folks utsagn. Kongen beordret den første metode brukt, ‒ en avgjørelse som er karakteristisk for den fullstendighet og nøiaktighet som den dansk-norske regjering tilstrebte å gi grenseverket.\par
I brev 9. november 1736 meldte kongen til kommissærene, at deres møte med de svenske, til påbegynnelse av arbeidet, efter svensk forslag var utsatt til begynnelsen av mai 1737. Men det fant heller ikke sted da. Dette synes å måtte skyldes nølende holdning fra svensk side til selve kravet om at grenseforhandlingene skulde omfatte kartlegning og detaljfastsettelse av riksgrensen i sin helhet. Først ved resolusjon 28. mai 1737 (g. st.) bifalt Sverige ‒ «ehuru förut snarast böjd för att endast afgöra redan uppkomna tvister», sier B. Boëthius\hypertarget{Schn1_847}{}\footnote{\label{Schn1_847}Sverges traktater, VIII, 2, s. 261.} ‒ Danmark-Norges i resolusjon 3. november 1736 (n. st.) fremførte ønske om «en fullständig gränsregulering». Derefter blev det 13. juni 1737 ‒ fremdeles ifølge Boëthius ‒ utferdiget instruksjon for «ordinarie ingeniören och landtmätaren i VärmlandLorens \textit{Gillberg}». Og i brev 27. juni 1737 (g. st.) meldte von Rosen til de dansk-norske kommissærer at han var opnevnt som svensk kommissær, og ba dem foreslå tidspunkt for det første møte, som han under henvisning til traktatens artikel 21 mente, måtte finne sted ved Svinesund. Skjønt de dansk-norske kommissærer var av den mening (skrivelse til von Holstein3. august 1737) at kartlegningen burde begynne ved Kornsjøen, hvor den i 1661 fastsatte grenselinje sluttet ‒ et syn som siden fikk medhold av kongen ‒, søkte de kontakt med Gillberg ved Svinesund. Men på grunn av Gillbergs forsømmelighet blev ingen kontakt opnådd og arbeidet ikke påbegynt i 1737.\par
I brev til de norske kommissærer 17. februar 1738 meddelte von Rosen at Gillberg var satt under tiltale, og at ingeniør og landmåler Lorents \textit{Lindgren} var opnevnt i hans sted, ‒ \hypertarget{Schn1_923}{}Innledning. videre at han selv vilde innfinne sig ved Svinesund6. april, hvis det passet. Under 27. februar utferdiget den svenske konge instrukser så vel for von Rosen som for Lindgren. Kort efter blev imidlertid von Rosen syk, og 30. mars 1738 blev generalmajor, landshøvding i GöteborgBengt \textit{Ribbing} utnevnt til svensk kommissær i hans sted; dette forsinket arbeidet endda nogen uker, så det første møte ved Svinesund først fant sted 29. april 1738.\par
I de svenske kommissærers instruksjon var det i ingressen slått fast at opgaven både var å undersøke og å avgjøre stridigheter «uppå alla de ställen, hvarest någon twist på Norska gränsen wara kan», og tillike, slik som kongene var kommet overens om, få «en gränselinea emellan Swerige och Norrige . . . . . befordrad och aftagen». I instruksjonen for Lindgren var det gitt bestemmelser om opgang og måling av grenselinjen der hvor ingen tvist fantes; dernæst var det for «de orter och ställen som giöres stridiga» påbudt at han skulde «aftaga» dem «med behöriga instrumenter på det nogaste» og samtidig notere «alla Intagor» av norske eller svenske undersåtter, at dernæst begge parters pretensjons-linjer skulde «afmätas och föras till papperet» av de fra begge sider beordrede landmålere, som så skulde «connectera» det som forelå og på grunnlag derav utarbeide «en ricktig Carta» over hvert enkelt omstridt punkt. Ved siden av sin åpne instruksjon hadde Lindgren også fått en spesialinstruks som skulde holdes hemmelig for de dansk-norske deltagere; her fikk han ordre om ikke uten nødvendighet å la danskene se de karter og dokumenter han hadde fått utlånt, om på forhånd i stillhet å drøfte grensemerker osv. med den lokale almue, samt om å gjøre iakttagelser angående skogenes art, beskaffenhet og mulige nytte, om lettvinte passasjer over grensen og om muligheten av å anlegge befestninger ved disse.\hspace{1em}\par
Av de i 1736 opnevnte dansk-norske kommissærer var Peder Colbiørnsen død 17. mars 1738. Rappe foreslo i hans sted opnevnt broren oberstløitnant Hans Colbiørnsen, men kongen fant ikke dette nødvendig. Det blev derfor bare Rappe og Sundt som 29. april 1738 møttes med Ribbing ved Svinesund. Ribbing fikk ved loddtrekning første utspill, og stillet da forslag om at forhandlingene skulde begynne straks, enten på Fredrikshald eller i Strömstad (efter loddtrekning), og at ingeniørene og landmålerne også straks skulde begynne sitt arbeide ‒ «yderst ved Svinesund». De norske svarte at de måtte forebringe dette for sin konge ‒ forslaget stred jo avgjort mot instruksjonen om å begynne der hvor man slapp i 1661. I en utførlig skrivelse 30. mai 1738 understreket kongen igjen dette punkt, og fremholdt at der ikke måtte røres ved den i 1661 fastsatte grense (han insinuerte, ikke uten grunn, at svenskene kanskje var mest interessert i å få denne revidert). Samtidig ansatte kongen en sekretær for kommissærene, regimentskvartermester og auditør Peter \textit{Smith}, og påla kommissærene å levere utkast til endelige instrukser, såvel for dem selv som for grenseingeniørene og sekretæren. Slike utkast blev sendt 27. juni 1738, og 14. juli undertegnet kongen instruksene. I begge instrukser er det på ny bestemt at arbeidet skal begynne hvor grensen av 1661 slutter. Det står intet om at denne bestemmelsen er truffet efter avtale med den svenske regjering; men da den svenske kommissær ikke sees å ha protestert, må man gå ut fra at hans regjering hadde gitt ham ordre til å la kravet om at man skulde begynne ved Svinesund, falle.\par
I skrivelsen 14. juli påla kongen ‒ efter forslag fra den svenske konge ‒ kommissærene \hypertarget{Schn1_1055}{}Innledning. å ha et møte med den svenske kommissær før ingeniørene begynte arbeidet. Men Ribbing var blitt utålmodig over den lange ventetid; han var reist til Göteborg, og en anmodning fra kommissærene om et møte på Fredrikshald før ingeniørene 1. august skulle ta fatt ved Kornsjøen, nådde ham for sent. De norske kommissærer lot da 31. juli ingeniørene reise i retning av Kornsjøen, hvor de 1. august møtte sine svenske kolleger. Sammen med ingeniørene reiste også sekretæren Smith, som ved kongens instruks var pålagt å følge ingeniørene, føre protokoll over deres arbeide, opta vidneforklaringer, nedtegne oplysninger m. v.; Smith blev derved en betydningsfull medarbeider i hele grenseverket.\par
Noget nytt møte mellem selve kommissærene, de norske og de svenske, fant ikke sted før i 1749. Og da var det nye menn på begge sider. Sundt blev fritatt for hvervet 28/8 1739, og det blev da bestemt at det også på norsk side kun skulde være én grensekommissær. Rappe fungerte til 10/2 1741. Hans efterfølger oberst Jobst Conrad von \textit{Rømeling} (d. e.) blev 20/9 1743 avløst av oberst Johan von \textit{Mangelsen}, som fra norsk side ledet arbeidet og undertegnet grensetraktaten i 1751. På svensk side blev Ribbing fritatt for hvervet 3/12 1739; derefter kom landshøvding, friherre Nils E. \textit{Reuterholm}1739‒40 og generalmajor L. Chr. \textit{Stobée}1740‒47. Så fulgte oberst Johan Mauritz \textit{Klinckowström}, som kom til å føre de avsluttende forhandlinger med Mangelsen.\hspace{1em}\par
Det grensearbeide som begynte ved Søndre Kornsjø1. (eller rettere 2.) august 1738, blev utført på norsk side av de opnevnte ingeniørkaptein Th. H. H. Knoff og sekretær Smith, assistert av Knoffs yngre bror underkonduktør Friedrich Christian \textit{Knoff}, og på svensk side av den opnevnte premieringeniør L. Lindgren assistert av landmåleren Kilian \textit{Ratkind} (mens den svenske sekretær ikke deltok).\par
Av arbeidets gang gir Absalon Taranger en summarisk fremstilling i Lappekommisjonsinnstillingen av 1904 (s. 15):\hspace{1em}\par
«Forretningsgangen var følgende: Grændsebetjentene (ɔ: Ingeniørerne og Sekretæren) rapporterede alt, hvad der foregik i Marken, til Kommissæren, der besørgede Rapporten til Kongen eller til Oversekretæren i det danske Cancelli. Under Maalingen i Marken korresponderede de norske og svenske Ingeniører og Sekretærer direkte med hinanden, uden at det altid lod sig gjøre at indhente Kommissærernes end sige Hoffernes Approbation paa de udvexlede Skrifter. Denne Korrespondance antog til sine Tider en temmelig skarp Karakter, hvilket atter førte til yderligere Drøftelser, Forsvar eller Undskyldninger mellem Kommissærerne.\par
Kommissærerne korresponderede direkte med hinanden, dog saaledes, at alle vigtige Skrivelser først var approberede af Hoffet.\par
Grændsens Gang bestemtes efter Undersøgelser eller Vidneførsel paa Aastedet, og Sekretæren var udstyret med Myndighed til at tage Vidnerne i Ed.»\hspace{1em}\par
Arbeidet nådde i 1738 til grensemerket «Reise Manen», der hvor grensen mellem Aremark (Øymark) og Rødenes støter til riksgrensen. Man var da ferdig med Idde og Marker i Norge og var kommet et stykke inn i Vermland for Sveriges vedkommende. Som resultat fremla de to brødre Knoff et vakkert utført kart (nr. 49 i Norges Riksarkiv), hvor den grenselinje som man efter de norske vidners utsagn fant måtte være den rette, er avlagt kraftig. Hvor svenskene holdt på en annen linje, er også denne avlagt, og området mellem de to pretensjonslinjer er \hypertarget{Schn1_1247}{}Innledning. markert med en særlig farve. Hvert avsnitt med slike avvikende pretensjoner er i protokollen betegnet som en «tvist», og disse tvister er gitt fortløpende numre fra syd mot nord.\par
I 1739 blev man ferdig med Vermlands-grensen og avsluttet årets arbeide på Kildegraven i det sydlige Trysil, nettop der hvor grensen mellem Vermland og Dalarne møter riksgrensen.\par
I 1740 kom arbeidet slett ikke i gang, vesentlig på grunn av sen vår. Men i 1741 kom man videre, så langt som til Brekken øst for Røros. Mens tvistene, dissensene, mellem de norske og svenske grensemålere de foregående år hadde gjeldt mindre, tildels helt ubetydelige områder, var man nu i 1741 nådd op til en av de store, nemlig den nesten 100 år gamle tvisten om Idre og Særna.\par
Disse vidstrakte, men folkefattige gamle annekssognene under Elverum var jo blitt okkupert av svenskene i 1644. Efter Brømsebro-freden i 1645 blev de ikke gitt tilbake til Norge, idet svenskene helt uriktig påstod, at de hørte med til Herjedalen. Fra norsk side blev denne påstand møtt med energisk protest. Men å ta dem igjen med makt så man sig ikke istand til. Og spørsmålet var derfor blitt stående åpent og uavgjort, idet svenskene fortsatt utøvet geistlig og verdslig jurisdiksjon i disse sognene. De lå jo også i den øvre delen av den svenske ÖsterDalälvens nedslagsdistrikt. Så rent geografisk kunde det forsåvidt synes naturlig, at de blev forenet med Sverige.\par
Efter gammel betraktning, stadfestet ved en dom av 15. september 1581, gikk bygdegrensen mellem Idre sogn av Elverum prestegjeld og Rendalen sogn av Tønset prestegjeld midt efter Femunden i omtrent hele dens lengde. Efter sitt syn på Idre-Særna-spørsmålet kartla derfor de svenske grensemålere en linje midt efter Femunden som riksgrense. På norsk side var man nok på det rene med, at det nu vilde være håpløst å gjøre krav på å få både Idre og Særna i sin helhet tilbake. Uten definitivt å opgi det prinsipielle norske krav kartla derfor de norske grensemålere, i forståelse med grensekommissærene, som et kompromissforslag en grenselinje omtrent rettlinjet fra syd mot nord østenfor Femunden i en avstand av 20-25 kilometer fra denne store fjellsjøen.\par
For Norge var det jo selvsagt noget av et prestisjespørsmål å vinne iallfall en del av Idre tilbake. Men det bestemmende har utvilsomt vært Rørosverkets interesser. Innen grensemålerne i 1741 var nådd frem til sydenden av Femunden, hadde verkets direktør, Leonhard Borchgrevink, efter opdrag av partisipantene satt sig i forbindelse med grensekommisjonen og på det sterkeste fremholdt verkets behov for Femundskogene. Og nogen uker senere mottok kommisjonen («grensebetjentene») fra direktør Borchgrevink også «et fra hr. Major Snitler til hannem indløben brev, Grendsens Gang betreffende, hvilket be.te hr. Directeur forlangede at maatte tages ad acta, saa blev samme brev dat. Trundhiem den 7 aug. sidstleden, imodtagen og commissionens øvrige papirer vedlagt».\par
Dette er første gang man i de norske grensebetjenters protokoll støter på Peter \textit{Schnitlers} navn. Men allerede innen grensebetjentene det følgende år, 27. juni 1742, innfant sig på Røros for å fortsette arbeidet videre nordover fra Brekken av, var Schnitler engasjert som fast medarbeider i grenseverket, ja hadde allerede på vinterføre foretatt en befaring av grensetraktene fra Røros til Snåsa og påbegynt de grenseeksaminasjoner, som har gitt hans navn og virke en central plass i våre riksgrensers tilblivelseshistorie.\hypertarget{Schn1_1415}{}Innledning.
\DivI[II. Peter Schnitler og hans deltagelse i grensearbeidet]{II. Peter Schnitler og hans deltagelse i grensearbeidet}\label{Schn1_1417}\par
\centerline{\textbf{II. Peter Schnitler og hans deltagelse i grensearbeidet.}}\par
Peter Schnitler, «grensemajoren», stamfaren til den store og ansette norske slekt Schnitler, var født på Christianshavn i København17. januar 1690 som sønn av den tyskfødte kjøbmann Lorentz Schnitler og hustru Dorothea Hansdatter Nobel.\hypertarget{Schn1_1449}{}\footnote{\label{Schn1_1449}Se E. A. Thomle: Forskjellige Slægtebogs-Optegnelser. I. Familien Schnitler, Personalhist. Tidsskr. 2 Række 2 Bind, Kbh.1887, side 169 flg. Dette er en hovedkilde til Schnitlers personlige livshistorie. Se også Kr. Nissens artikel om Schnitler i Norsk biogr. leks. XII, Oslo1954, s. 490‒498.}\par
Moren var datter av en kjøbmann i Christianopel i Blekinge, Hans Mortensen Nobel, som efter at Blekinge ved Roskildefreden 1658 var avstått til Sverige, hadde flyttet til København. Innen hun blev gift med Lorentz Schnitler, hadde hun endel år vært gift med presten Peder Hoffgaard, som døde på St. Thomas 1684.\hypertarget{Schn1_1511}{}\footnote{\label{Schn1_1511}Smlgn. E. A. Thomle: Familien Hofgaard i Norge, Kra.1911, s. 2‒3, note 3.}\par
Peter Schnitler var altså av tysk-dansk slekt. Allerede under gutteårene i København blev han av sin far holdt til studeringer, først under tyske «præceptorer» (huslærere), senere under magister Jørgen Hulbek, som da var kapellan til Vor Frelsers kirke på Christianshavn og som i 1707 dimitterte denne sin elev til universitetet i København. At Schnitler ‒ som man kan finne anført ‒ det følgende år tok examen philosophicum, er vel mulig. Det bekreftes dog ikke av universitetsprotokollen. Men dette året ‒ altså 1708 ‒ blev han av sin far sendt til universitetet i Rostock, hvor hans oldefar Johannes Schnitler i sin tid hadde vært kjøbmann og kirkeverge. Og det er iallfall mulig, at Peter Schnitler hadde forholdsvis nære slektninger bosatt der.\par
Det studium han slog inn på var jus, et studium som han må ha tatt overmåte grundig, å dømme ikke bare efter slektsoptegnelsene, som utvilsomt skyldes førstehånds meddelelser fra ham selv, men ‒ helt avgjørende ‒ efter det storverk han senere utførte i Norge. I slektsoptegnelsene er det da også gjort detaljert rede for, hvilke særskilte fag han studerte og hvem hans lærere var i de forskjellige discipliner. Det blev sikkert også av stor fremtidig betydning for ham at han, som det heter, «øvede mellemstunder Exercicerne og Sprogene», hvilket vel vil si, at han deltok i fekteøvelser og annen datidig sport og leste moderne sprog. Å dømme efter den ferdighet, med hvilken han senere skrev ikke bare dansk og tysk og latin, men også fransk, må det sannsynligvis være her i Rostock han erhvervet sig kunnskaper også i det franske sprog. Det vites iallfall ikke, at han nogensinne har besøkt Frankrike eller at han idetheletatt efter studenterårene i Rostock foretok nogen utenlandsreise.\par
Hans studieophold i Rostock varte dog bare tre år eller vel det. Det har sannsynligvis vært avsluttet med en juridisk eksamen. At så var tilfelle er dog ikke uttrykkelig oplyst. Men med de utvilsomt solide juridiske kunnskaper han hadde erhvervet sig, søkte iallfall den nu 21-årige Peter Schnitler sig en civil-militær stilling i den danske hær og blev 11. desember 1711 ansatt som auditør og regimentskvartermester ved det Viborgske regiment.\par
Det var jo nu midt under den store nordiske krig. Og i stillings medfør førte Schnitler derfor i de følgende årene en meget omskiftelig tilværelse, dels i Nord-Tyskland, dels i Holsten, \hypertarget{Schn1_1603}{}Innledning. dels i Jylland og tilslutt fra juni 1716 til våren 1717 i Smålenene (Larkollen, Glomma, Svinesund og Fredrikstad).\par
Under et siste ophold i Danmarkvåren 1717 gjorde Schnitler tjeneste ved andre regimenter. Men da hans tidligere chef ved Viborgske regiment, oberst Baltzer Meitzner, som 16. mars s. å. var utnevnt til chef for 1. Trondhjemske nationale infanteriregiment\hypertarget{Schn1_1647}{}\footnote{\label{Schn1_1647}Ovenstad: Militærbiografier, II, s. 165.}, ønsket å få Schnitler med sig til Norge, tok Schnitler imot et tilbud om ansettelse som auditør og regimentskvartermester ved Meitzners nye regiment. 4. august s. å. kom Schnitler til Trondhjem og 11. september s. å. blev han utnevnt i den tilbudte civil-militære stilling.\hypertarget{Schn1_1676}{}\footnote{\label{Schn1_1676}Sst., s. 372.}\par
Fra da av og helt til sin død i 1751, altså i 34 år, var Schnitler ved sit virke helt knyttet til Norge.\par
Den 11. november 1718, altså umiddelbart før den svenske hær under Armfelt innesluttet Trondhjem, blev Schnitler forflyttet som auditør og regimentskvartermester til 3. Trondhjemske nationale infanteriregiment, som også deltok i forsvaret av byen. Det som både innenfor og utenfor voldene voldte størst mannefall var sygdom. Og blandt de mange som døde på sotteseng, var også Schnitlers tidligere chef oberst Meitzner, som døde 22. (el. 23.) november. «Dog havde [han],» heter det i de slektsoptegnelser, hvorfra oplysningene om Schnitlers ungdom er hentet, «i sit levende Live tilforn givet sit Samtykke til et Christeligt Ægte imellem sin jomfrue Dotter og Regimentsqvarteermester Peter Schnitler.»\par
Dette, at oberst Meitzner hadde godkjent Schnitler som sin vordende svigersønn, kan vel ha vært medvirkende til, at Schnitler allerede innen årets utgang, 30. desember, altså umiddelbart før Armfelt tiltrådte sitt tragiske tilbaketog til Sverige over Tydalsfjellene, blev ansatt som kaptein og chef for det Størenske kompani av samme 3. Trondhjemske regiment. Nogen egentlig offisersutdannelse hadde han jo ikke. Men det tør vel hende, at all sygdommen under beleiringen hadde tæret så sterkt også på offiserskorpset, at man så sig nødsaget til å ansette også menn uten offisers- eller underoffisersutdannelse i offisersstillinger.\par
Noen uker efter avansementet til kaptein falt også Schnitler «i den da i Svang gaaende hidsige Sygdom og holdtes derved i nogle Maaneder ved Sængen». Men 23. mai 1719 var han i vigør igjen og kunde holde «Brøllup med Jomfrue Sophia Christine Meitzner, salig Obriste Meitzners Dotter i Trondhjem».\par
Til å begynne med bodde ekteparret Schnitler i Trondhjem. Men 20. august 1720 flyttet de til gården Høysæt i Horg sogn av Støren prestegjeld, altså midt i Schnitlers kompanidistrikt. Her blev de boende til 14. mars 1729. Men da flyttet de tilbake igjen til Trondhjem, hvor Schnitler hadde bygget sig sitt eget hus på Kalvskindet, «for at have der bæder Leilighed til at reducere [ɔ: opdra] og lade oplære sine Børn». I dette huset, som ennu står og brukes av Dahls Bryggeri som kontorbygning, hadde da Schnitler og hustru sitt hjem resten av sin levetid. Og der vokste deres store barneflokk op.\par
Da Schnitler23. juni 1734 blev utnevnt til major, bevirket ikke dette nogen endring i bosted eller embedsvirke. Han fikk nemlig bare majors titel og rang eller «karakter», blev «ka\hypertarget{Schn1_1811}{}Innledning. rakterisert major», som det gjerne kaltes. Han vedblev nemlig å være chef for Størenske kompani og for det alene.\par
Det samme var forholdet, da han 28. oktober 1749 blev oberstløitnant, fikk «oberstløitnants karakter». Denne utnevnelsen skjedde utvilsomt for å yde ham en fortjent honnør for det overordentlig store og verdifulle arbeide han spesielt i årene 1742‒45 hadde utført med sine «grenseeksaminasjoner», som nedenfor skal bli nærmere omtalt og som i vår historie har gitt ham navnet «grensemajoren».\par
Heller ikke hans utnevnelse i 1750 til amtmann i Finnmarken bevirket nogen endring i hans embedsgjerning og livsforhold. Han frasa sig nemlig straks embedet. Hvilket må bety, at han ikke hadde søkt det, men uspurt var blitt utnevnt av Kong Frederik V, som naturlig nok forsøkte å sikre sig en mann med Schnitlers personlighet og inngående kjennskap spesielt til Finnmarkens utenrikspolitiske forhold som landsdelens overøvrighet istedetfor den tidligere samme år avsatte amtmann Rasmus Kjeldsøn. Men at Schnitler frabad sig den tiltenkte ære er ikke så underlig. Han var jo nå en 60 års mann og hadde sitt gode hjem i Trondhjem.\par
Hvad som derimot kan synes noget underlig er, at de forberedende grenseundersøkelser fra Røros og nordover i 1742 var blitt overdradd til den danskfødte offiser Schnitler. Utenfor sitt kompanidistrikt Guldalen hadde han i de fjorten årene han hadde vært bosatt i Norge ikke hatt anledning til å lære grensetraktene mot Sverige å kjenne, hverken i Nord-Trøndelagen, ennsi videre nordover gjennem Nordlandene og Finnmarken helt til russegrensen.\par
Forklaringen til at dette viktige opdraget blev gitt Schnitler, ligger dels deri, at Schnitler jo ikke bare var en vanlig offiser, men først og fremst jurist med full akademisk utdannelse som sådan og med den formelle og reelle erfaring i å opta vidneforklaringer, som syv års tjeneste, som militær auditør hadde gitt ham. Uten å ha eller behøve nogen prokuratorbevilgning har han sannsynligvis, iallfall efter at han i 1729 påny tok bolig i Trondhjem, i nogen utstrekning også virket som juridisk konsulent. Et typisk eksempel herpå har man i en lengere utredning, som Schnitler i 1738 avgav til konferenseråd J. W. Schrøder i København i en sjøassuransetvist mellem avdøde lagmann Dreyers bo og en kommisjonær for et hollandsk assuranseselskap. Og av Røros Kobberverks vel bevarte og velordnede arkiv fremgår, at Schnitler iallfall i årene 1740‒42 bistod verkets partisipanter, spesielt Thomas Angell, og daværende oberst, senere general og grenseforhandler Johan von Mangelsen, som konsulent og sekretær med utredninger og korrespondanse vedrørende grenseopgjøret med Sverige.\par
Rørosverket var nemlig, som allerede antydet (s. XIX), i høi grad interessert i en for Norge mest mulig gunstig grenselinje, særlig i Femundtraktene, dette særlig av hensyn til tilgangen på tremateriale og trekull også fra skogene på østsiden av Femunden til den nye smeltehytten som i årene 1739‒42, særlig på Mangelsens initiativ, blev opført på vestsiden av sjøen.\par
Uten å være engasjert fra høieste hold til å uttale sig om grensespørsmålene hadde Schnitler2. desember 1738 sendt en lengere skrivelse til sin chef oberst Emahusen, hvori han gav et sterkt uttrykk for aktiv interesse for grensespørsmålene og for en for vårt land ‒ og for Rørosverket ‒ fordelaktig løsning av dem. Foruten å gjøre rede for grenseforholdene i sitt kompanidistrikt Guldalen, hvortil Rørostraktene da som nu blev henregnet, fremkom han nemlig med et begrunnet forslag om, hvorledes grenseundersøkelsene burde settes iverk, med vidneforkla\hypertarget{Schn1_1969}{}Innledning. ringer overfor sorenskriverne samtidig med kartlegning og undersøkelser i marken ved offiserer og landmålere. Nettop på denne måten var jo undersøkelsene foregått allerede tidligere samme år fra Hisøy i Nordre Kornsjø, der hvor den i 1661 efter avståelsen av Båhuslen fastsatte korte grensestrekning sluttet. Men dette kan neppe Schnitler ha kjent til og heller ikke Mangelsen, som først i 1743 avløste oberst Jobst Conrad Rømeling som grenseforhandler. Og heller ikke Emahusen. Det er dog mulig, ja kanskje sannsynlig, at Schnitler var tilskyndet av sin regimentschef til å fremkomme med denne redegjørelsen. For Emahusen hadde selv som major i 1727 hatt i opdrag å rekognoscere grensetraktene mot Jemtland og Herjedalen og derefter i desember s. å. avgitt til sin chef, generalmajor Wincentz Budde, en «Omstændelig Forklaring» angående grenseforholdene i Finnliene.\par
Heller ikke Emahusen kan dog ha kjent til at grenseundersøkelsene allerede sommeren 1738 var satt iverk fra Kornsjø nordover i hovedsaken på samme måte som Schnitler nu foreslog. Praktiske vanskeligheter hadde snart bevirket, at det blev grenseingeniørenes sekretær, juristen regimentskvartermester Peter Smith, som måtte ta op vidneforklaringene istedenfor sorenskriverne, helt beslaglagt som disse var av sine vanlige embedsforretninger. Og ved et kongelig brev av 17. juli 1739 var denne ordning blitt godkjent.\par
Imidlertid hadde Emahusen20. desember 1738 sendt Schnitlers redegjørelse videre til den kommanderende general, generalløitnant Hans Jacob Arnold. Og han lot den gå videre til generalmajor Friderich Otto von Rappe, som på dette tidspunkt ‒ inntil 10. februar 1741 ‒ var en av de norske grensekommissærer. Med ham kom Schnitler i korrespondanse i april 1739. Og da Rappe i februar‒mars 1740 opholdt sig i Trondhjem, benyttet han anledningen til å sette sig i direkte forbindelse med Schnitler, som derefter med følgeskrivelse av 11. mars s. å. sendte Rappe en lengere «Underdanig-uforgribelige Deduction over Grændse-Skiellet Nordenfields imellem Norge og Sverrig».\par
Denne «deductionen» var datert Trundhiem29. februar 1740 og gav en videregående utredning av de spørsmål Schnitler allerede hadde behandlet i ovennevnte skrivelse til Emahusen. Av følgeskrivelsen fremgikk forøvrig, at Rappe under opholdet i Trondhjem tillike hadde gjort bruk av Schnitler til å utarbeide konsepter til en hel rekke skrivelser vedrørende grenseforhandlingene til overordnede instanser i København og Christiania, konsepter som legger for dagen, at Schnitler allerede da satt inne med et omfattende kjennskap til grensespørsmålene og deres forhistorie.\par
Da den nye grensekommissær oberst Jobst Conrad Rømeling i februar 1742 opholdt sig i Trondhjem, hadde også han, i likhet med Rappe et par år tidligere, gjort bruk av Schnitler som sekretær. Og med det kjennskap han herunder vant til Schnitlers kvalifikasjoner, foreslog han i skrivelse av 17. februar 1742 til Kanselliet i København, at Schnitler skulde knyttes til grenseverket, og det således, at han ikke samtidig med, men forut for grensebetjentenes grenseopgang i marken, skulde foreta undersøkelser og la opta vidneforklaringer ved sorenskriverne. Dog skulde Schnitler være bemyndiget til selv å opta forklaringene, om sorenskriverne ikke fant tid og anledning til det.\par
Rømelings forslag blev vel mottatt i Kanselliet. Allerede 9. mars s. å. blev saken behandlet i konseilet på Christiansborg slott. 16. s. m. undertegnet kong Christian VI brev til alle vedkom\hypertarget{Schn1_2178}{}Innledning. mende, deriblandt til Rømeling og til Schnitler personlig, hvori han godkjente Rømelings forslag.\hypertarget{Schn1_2189}{}\footnote{\label{Schn1_2189}Bilag 1, s. XXVI.} Og 31. s. m. utferdiget Rømeling en detaljert instruks for Schnitler, hvem han påla «at begynde og udi mueligste hast fuldende sin Forretning».\hypertarget{Schn1_2204}{}\footnote{\label{Schn1_2204}Bilag 2, s. XXVIII.} Hvad Schnitler ikke lot sig si to ganger. Allerede 7. april s. å. var han på vei til Røros og var dermed gått igang med det intense arbeidsliv under ofte slitsomme reiser, som med korte avbrytelser varte til hans definitive tilbakekomst til Trondhjem29. november 1745.\hspace{1em}\par
Nogen detaljert redegjørelse for Schnitlers reiser, grenseeksaminasjoner og arbeide forøvrig i henhold til det ansvarsfulle opdrag som var ham gitt ved det kongelige brev av 16. mars 1742, kan ikke her bli gitt. Det kan kun bli en kort oversikt under henvisning såvel til den i næste avsnitt av denne innledning gitte oversikt over Schnitlers eksaminasjonsprotokoller, relasjoner, kartarbeider, korrespondanse og efterlatte papirer forøvrig, som til nærværende trykte utgave av hans grenseeksaminasjonsprotokoller 1742‒1745. I utgavens detaljerte \textit{innholdsfortegnelse} (foran, pag. VII‒XII) vil reisenes detaljer og arbeidets nøiaktige forløp kunne følges.\par
Grenseeksaminasjonsreisene var tre i tall av stadig økende varighet. Den første var ovennevnte vinterreise i Trøndelagens grensetrakter Røros‒Snåsa7. april‒10. juni 1742. Den andre var sommerreisen i Nord-Trøndelag og den søndre del av Nordlandene, vesentlig Helgeland, i tiden 13. juli‒12. november 1742. Og tilslutt kom den store reisen til Salten, Senjen og Tromsen i Nordlandene samt hele Finnmarken i tiden 4. mai 1743‒29. november 1745, iverksatt overensstemmende med den spesielle utførlige instruksjon som blev gitt Schnitler gjennem kongens skrivelse til oberst Rømeling av 22. mars 1743.\hypertarget{Schn1_2302}{}\footnote{\label{Schn1_2302}Bilag 3, s. XXX.}\par
Dette var de egentlige undersøkelsesreiser, hvorunder han personlig optok og protokollerte vidneforklaringer av både lapper og fastboende. Det blev ikke til noe med forhør ved noen av sorenskriverne. Dertil var disse for optatt. Schnitler greide det hele selv ved siden av alle de andre opgaver som hvilte på ham: lange konferanser med lensmenn og embedsmenn, utarbeidelse av sammenfattende relasjoner, en meget omfattende korrespondanse og det ikke bare på norsk eller dansk, men også på fransk, ja endog på latin. Og i tillegg til alt dette tegnet han en mengde kartskisser til sine eksaminasjonsprotokoller, ja hele amtskarter både over Nordlanddene og over Finnmarken, karter som ganske visst ikke står for nøieregnende geodetisk kritikk, men som var og er gode hjelpemidler til forståelse av protokollenes tekster.\par
Men foruten disse undersøkelsesreisene måtte Schnitler i disse årene også foreta flere andre av kortere varighet: En snartur fra Trondhjem til Røros18.‒30. juni 1742 for å konferere med grensebetjentene (Knoff'ene og Smith). En reise fra Trondhjem til Christiania7. februar‒ 18. april 1743 til konferanser med grensekommissæren, oberst Rømeling. Og så blev den store Nordland‒Finnmark-reisen avbrutt i 1744 ved en reise til Trondhjem til konferanser med oberst Mangelsen, som 20. juni 1743 var beordret til å avløse Rømeling som grensekommissær. \hypertarget{Schn1_2368}{}Innledning.Schnitler reiste sydover fra Alta18. april 1744. Og da han under tilbakereisen nordover efter anmodning overvar en rekke «lappeting» i Nordlandene, kom han først i august s. å. tilbake til fortsatt arbeide i Finnmarken.\par
Detaljert kjennskap til Schnitlers reiser og virksomhet i disse årene får vi først og fremst fra hans konsise, innholdsrike og gjennemført saklige grenseeksaminasjonsprotokoller, som nu utgis in extenso i trykken, samlet i to bind.\par
Men Schnitler førte også en journal over alle sine reiser, og over sin virksomhet idetheletatt under og mellem reisene og i det første halve året efter tilbakekomsten fra den tredje og siste reisen. Man er derfor i stand til å følge ham i hans arbeide fra dag til dag helt fra 7. april 1742 til 17. juni 1746.\par
For å få rede på hans virksomhet for grenseverket helt til høsten 1750, altså til kort før hans død i Trondhjem21. januar 1751, er man derimot henvist til å søke oplysninger fra andre dokumenter og da først og fremst fra hans kopibok og fra hans korrespondanse, som likeledes finnes bevart i stor utstrekning i grensereguleringsarkivet i Riksarkivet i Oslo. Nærmere oplysninger om alt dette materialet blir gitt i nedenstående avsnitt III av denne innledning.\par
I disse siste årene efter avslutningen av grenseeksaminasjonsreisene var Schnitler nærmest å betrakte som en konsulent, på den ene side for grensebetjentene (Knoff'ene og Smith) og på den annen for grensekommissæren oberst, senere generalmajor Mangelsen.\par
Under de endelige grenseforhandlinger i Strömstad mellem Mangelsen og den svenske kommissær oberst Klinckowström fra mai 1749 til oktober 1751 var den eldste av grensebetjentene, Th. H. H. Knoff, en tid til stede og gikk Mangelsen tilhånde. Når det ikke blev gjort bruk av Schnitler, henger det vel til dels sammen med, at han ikke hadde personlig kjennskap til grensetraktene sønnenfor Røros, og at han ikke som Knoff på en fagmessig måte kunde bistå Mangelsen med kartfremstillinger til bruk under forhandlingene. Men det tør også ha berodd på, at man fra svensk side ikke hadde nogen å ta med til forhandlingene, som satt inne med et så inngående og omfattende kjennskap til både de historiske, folkerettslige og geografiske spørsmål som Schnitler. Mulig er det også, at Schnitler, som ganske visst ikke ennå var nogen gammel mann, allikevel var slitt og sykelig efter de i aller høieste grad forserte og slitsomme reiseårene, hvorunder han både kvantitativt og kvalitativt utførte et så stort og fremragende arbeide i og for vårt land, at det må minnes med beundring i dyp takknemlighet.\par
Men som nevnt døde Schnitler allerede 21. januar 1751, altså vel åtte måneder før Strömstadtraktaten blev undertegnet 2. oktober s. å. Han fikk derfor ikke den glede å vite, at kronen var satt på det grenseverk, han på en så verdifull måte hadde vært med om å underbygge og reise.\hypertarget{Schn1_2492}{}Innledning.
\DivII[Bilag (Schnitlers instruksjoner)]{Bilag (Schnitlers instruksjoner)}\label{Schn1_2494}\par
\centerline{BILAG (SCHNITLERS INSTRUKSJONER) \textit{Bilag 1. Kongens skrivelse til Schnitler16. mars 1742.}}\par
{Christian den Siette, Af Guds Naade Konge til Danmark og Norge, de Venders og Gothers, Hertug udi Slesvig,  Holsten, Stormarn og Dytmersken, Greve udi Oldenburg og Delmenhorst,}\par
Vor Naade tilforn; Da Vi ere komne i Erfaring, at ved den imellem Vort Rige Norge og Sverrig anordnede Grændse Commission, paa de Svendskes side bruges den Methode, at de Svendske Grændse Bønder for Herretshøfdingerne paa vedkommende Tingsteder forud aflegger deres Eedlige Deposition, om Grændsens gang, hvilke Forretninger de Svendske Betientere ved Grændse-Commissionen til stilles, der Sagen siden viidere ved Opmaalingen in loco undersøger, og Vi, i henseende at Grændserne Nordenfields, for den største deel, ere meget langt fra Bøygderne beliggende, saaog paa endeel steder af saa liden betydning for Bøygde Folkene, at det kunde formoedes, at de, saameget mueligt er, ville undslaae sig fra, at indfinde sig paa Grændserne, deres Vidnesbyrd at aflegge, foruden at det ogsaa kunde hænde sig, at mange, formedelst alderdom skrøbelighed, eller af mangel paa Livs Ophold ikke paa Grændserne kunde møde, hvorved Vores Interesse ved Commissionen kunde komme til at liide, allernaadigst have funden for got, at ovenmelte af den Svenskes Sides Commissions Betientere brugende Methode og Orden skal paa denne siide ligeledes observeres, saaat Bøygde-Folkene først hiemme ved Tinget, hvor de ere bekiendte, og de beqvemmelig kand comparere, og siden, som tilforn, ved Opmaalingen in loco, hvor deres ydermeere Forklaring paa adskillige poster, behøves, skal afhøres; Men, som det er at befrygte, at vedkommende Sorenskrivere deels ikke kunde have den Kundskab, som til saadan examination behøves, deels ogsaa formedelst andre Forretningers skyld, ikke kunde have den tiid, som dertil fornødiges; Og Vi have fornummen, at du haver forhvervet Dig een god connoissance om Grændse Commissionens Væsen, og Vi saaledes giøre Os allernaadigst forsikrede, at du herudinden udviiser Os din allerunderdanigste troe Tieneste, Saa er hermed Vores allernaadigste Villie og Befaling, at du retter din Leylighed efter, at bievaane forbemelte Sorenskrivernes Forretninger paa Grændse Commissionens vegne og det fornødne til Sagens oplysning observerer, da Vi allernaadigst have tillagt dig Toe Rigsdaler i diæt Penge daglig, og frie Skydtz for dig samt medhavende Bagage og Tienere, saalenge, og naar du udi denne Forretning og paa Reisen virkelig er begreben, hvilke Penge Vi allernaadigst have befalet Os Elskelig Christian Ulrich Nissen, Vores Stiftbefalingsmand over Trundhiems Stift Etatz Raad og Amtmand over Trundhiems Amt, at foranstalte, at af Vores Cassa i Trundhiem, til dig bliver betalte, ligesom Vores Grændse Commissions Betientere deres diæt Penge paa Requisition af Os Elskelig Jobst Conrad von Rømeling, Vores Oberst ved et Geworben Regiment i Vort Rige Norge, hos Vores Casserer i Christiania opbære, Og paa det at denne Forretning saameget desto bedre kand vorde befordret, saa have Vi allernaadigst rescriberet Stiftamtmand Nissen 1, At beordre Sorenskriverne, at indhente Almuens Eedlige Forklaring, angaaende Grændsens Gang og de derved behøvende omstendigheder paa saadanne \hypertarget{Schn1_2574}{}Innledning. tiider og stæder, som du det requirerer, paa det du alle steder saadanne Forretninger kand bievaane, da vi allernaadigst have ladet befale, at dig udi alt skal beviises ald Velvillighed, Føyelighed og Reedebondhed. 2, Ville Vi allernaadigst, at saadanne Forretninger skal gives dig, uden betaling og ufortøvet beskrevne, paa det du samme til fornevnte Oberste Rømeling, saa hastig, som behøves, kunde, lade befordre. 3, Skal alle Vidnerne, paa dit forlangende, examineres over alle de poster, som du til Sagens oplysning, finder fornødne, hvorom du til deels forud af Obriste Rømeling skal blive instrueret, til deels og ved Correspondence videre, om behøves, vorde underrettet. 4, Eragte Vi det at være nok, at 2de Mænd overværer og underskriver med Sorenskriveren paa de Stæder, hvor Otte Laugrettes Mænd ey beqvemmelig kand haves. 5. Saafremt det skulle hænde sig, at Sorenskriverne til saadan tiid og sted, som du det forlanger, ikke kunde foretage sig saadan Forretning, enten formedelst Svaghed, eller virkelig occupation paa Reyser og i andre Forretninger, da haver Du Selv at fuldføre samme Forretning, i steden for Sorenskriveren, hvortil du af Vedkommende skal nyde ald muelig assistence, alt paa det at de fornødne Oplysninger, til rette tiid kand indløbe, og Vores Tieneste ved Grændse Commissionen i sin tiid ikke skal opholdes eller hindres; Udi det øvrige have Vi ligeledes allernaadigst ladet befale Stiftbefalingsmand Nissen, om dette forskrevne at avertere vedkommende vore Betientere, paa det de forud herom kunde være vidende, og saavidt mueligt indrette deres Forretninger og Leylighed saaledes, at de til den tiid, som du Stiftbefalingsmand Nissen forud haver at tilkiendegive, kand være tilstæde; Da Vi ogsaa have ladet befale Os Elskelig Eyler Hagerup, Biskop over Trundhiems Stift, at giøre den anstalt i vedkommende Sogner paa Grændserne, at denne Vores allernaadigste Villie og Intention vorder giort Almuen bekiendt, og at Lap Finnerne af Missionairerne derom bliver betiimelig advarede, paa det de fornødne Vidner til rette tiid kunde være ved haanden, da bemelte Missionairer ogsaa skal være pligtige, Sagen med ald muelig fliid og omhyggelighed at befordre; Derefter du dig allerunderdanigst haver at rette, og at see denne Vores allernaadigste befaling, den du i fornødne tilfælde kand foreviise, med allerunderdanigst hørsommelighed efterlevet, forsaavidt den dig vedkommer; Ellers indberettes Os af dig allerunderdanigst, at du samme Vores allernaadigste ordre haver bekommet; Befalendis dig Gud; Skrevet paa Vort Slot Christiansborg i Vores Kongelig Residentz Stad Kiøbenhavn den 16 Martii 1742.\hspace{1em}\par
\centerline{Under Vor Kongelig Haand og Signet (sign.) Christian R.}\par
{_____________ (sign.) J. L. Holstein.}\par
Til\par
Major Peter Snitler, at bievaane Sorenskrivernes Forretning paa Grændse Commissionens vegne og det fornødne til Sagens oplysning at observere, da hannem i diæt Penge daglig skal gives 2 Rdr og frie Skydtz for ham og medhavende Bagage og tienere, saalenge og naar hand i denne Forretning og paa Reysen virkelig er begreben med videre.\hypertarget{Schn1_2629}{}Innledning.\par
\centerline{\textit{Bilag 2. Oberst Rømelings skrivelse til Schnitler31. mars 1742.}}\par
Welbaarne Høitærede H:r Major\par
Da H:s Maj:t under 16 Marty sidst:l allernaadigst haver committeret Mhr Major at indtage paa Grendse Commissionens Veigne en Edelig Forklaring fra samt:l Jndbygerne ved Grendserne udi Trundhiems Stift, angaaende hvad de om Grendsens Gang imellem bemelte Stift og de anstødende Svenske Provintzer kand være vidende, til den Ende, at Grendse Commissionen, førend den til Stæderne self med Opmaalingen avancerer, af saadan Mhr Majors Forretning kand have en tilstrekkelig Præcognition om hvad som forud bør vides, og H:s Maj:t derved tillige allernaadigst haver anbefalet Mhr Major forud at indhente min Underretning om hvad som efter mine Tanker, til bemelte Commissions Oplysning og Tienestens desbedre Befordring kand være tienligt, saa haver ieg saadant herneden under postviis forfattet, saaledes som følger:\hspace{1em}\par
1) At Mhr Majors Forretning begynder ved Trundhiems Stifts Sydre Ende og continuerer videre Nordpaa, saa langt som Sverrig strekker sig imod Norge, og indtil Rusland udi Finmarken enten for sig self allene eller tillige med os og de Svenske begynder at prætendere nogen Jurisdiction, da ieg, forinden H:r Major med sin Forretning saa vidt avancerer, agter at indhente H:s Majt:s allernaadigste Resolution, hvorledes man sig i Henseende til saadanne Districter, hvorunder Russisk Jnteresse verserer, haver at forholde.\par
2) At saa som Grendse Commissionen med forledet Aars Forretning allerede til det Nordenfieldske er avanceret, og udi Slutningen af nestanstundende Juny Maaned formodentlig skal continuere videre Nordefter, følgelig betimelig behøver ovenberørte Underretning, Mhr Major da, saa snart som dette indløber, ufortøvet ville behage at begynde og udi mueligste Hast fuldende sin Forretning, saa som der dog formodentlig vil indfalde et og andet, hvorom man betimelig behøver at underrettes.\par
3) At de Underretninger, som H:r Major inden neste Juny Maaneds Udgang indhenter, sikkerlig og uden Ophold maatte adresseres til H:r Directeur Borchgrevink paa Røraas, der samme til Commissions Secretereren H:r Regimentsqvartermester Smith paa Forlangende lader være følgagtige. De andre Underretninger som i den øvrige deel af nestanstundende Sommer erholdes, kand beqvemmeligst hos vedkommende Bøigdelehnsmænd deponeres, med Ordre, at De samme til Grendse Commissionen, naar den til Sognets Grendser arriverer, sikkerlig befordres, hvilket alt om Trundhiems Amt er at forstaae, saa som Commissionen udi dette Aar formodentlig ikke videre kand avancere. Hvad Underretninger som Mhr Major ellers erholder udi bemelte nestforestaaende Sommer og senere paa Aaret om Nordlands Amt og ellers siden videre om Finmarken Aaret forud, førend Commissionen sammesteds hen med Opmaalingen kommer efter, behager De at adressere immediaté til mig self, naar de ellers kand være vis paa, at disse documenter saa betimelig kand indløbe, at vedkommende om Foraaret, førend de sig paa Reisen begiver, dermed af mig kand blive forsiunede.\hypertarget{Schn1_2708}{}Innledning.\par
4) Da H:s Maj:t under 16 Marty sidstleden allernaadigst har resolveret, at det i henseende til Jemtelands Grendser skal forblive ved Almuens Udsagn og de forhen vedtagne Sognemerker, saa tienner Mhr Major saadant til Efterretning, paa det De angaaende bemelte Grendser ikke skal give sig nogen videre Møde, end fornødiges. Angaaende Grendserne imod Herjedalen forbliver det derimod efter den Kong. Resolution af 8:de December sidstafvigt saaledes, at denne Sides Paastand i anledning af den Broemsebroeske Tractat og Cessions Act skal indrettes, hvilket sidste forud er Mhr Major bekiendt.\par
5) Angaaende Grendsens Gang i Almindelighed behager Mhr Major at indtage saa nøjagtigen Jnformation som muelig er, baade om de tvistige og utvistige Stæders Beskaffenhed samt alle de øvrige Omstendigheder, som enten i Henseende til H:s Majt:s Tienneste og Sagen i sig self eller Commissionens Reises og Opmaalningens Befordring kand eragtes tienlige.\par
6) Om et hvært Grendsemerkes, samt det imellem Grendsemerkerne og ved Siderne nærmest derved beligende Landskabs Beskaffenhed, samt Grendsemerkernes distance fra hværandre efter Norske maalte Miile, og i hvad Strekning de fra hværandre ere beligende, samt ved hvilken Gaards prætenderede Grund saa vel paa Svenske som paa Norske Side, ville H:r Major paa det nøjeste lade sig informere, hvadenten Stæderne af Almuen angives for tvistige eller ikke, hvorved de Folkes Navne, Alder og Boepæl, af hvilke saadan Underretning erholdes, tillige beskrives, samt hvilke de ere, som saadanne Grendsemerker kand udvise.\par
7) Angaaende de tvistige Stæders Bonité, Importance og videre Beskaffenhed vil fornemmelig en omstendelig Underretning indhentes, og Tvisten paa det beste med alle sine Omstendigheder samt hvad derved merkværdigt enten i forige eller senere Tider imellem Prætendenterne paa begge Sider i Henseende til deres tvistige Ejendele, kand være passeret, klarlig oplyses. Saa vil og af vedkommende æskes og imodtages alle de gamle Documenter, som saadan Tvistighed paa nogen Maade pro eller contra kand opdage, paa det Sandhed og Sagens rette Beskaffenhed kand uden partialitet komme for Lyset, og man i sin tid af Mangel for Underretning ikke skal paastaae mere, end man er berettiget til, hvorved Sagen i sin tid bliver vanskeligere, end som fornødiges, og H:s Maj:t, som vil have disse Tvistigheder bragt til en venlig Composition og Endskab, i Lengden ikke er tient.\par
8) For at befordre Commissions Betienternes Reise fra og til Grendserne, samt fra et Grendsemerke til et andet, var det nøttigt, om man tillige kunde faae at vide, hvor langt Grendsemerkerne fra Bøigderne eller Landevejen ere beligende, samt hvorledes Landskabet derimellem i saadan Hensigt er beskaffet, om og hvor Underholdning for Folk og Heste, Vand og Brendeved, er at bekomme, og paa hvad Maade de Vande, som ligge i Vejen, kand passeres, om nær derved nogen Grov Skoug til Flaader befindes, eller om det er fornødent og mueligt at forskaffe sig Baader, samt paa hvad Maade saadant best kunde skee, hvortil H:r Majoren i saa Fald provisionaliter kunde føje de beste Anstalter, hvilke Folk ere de kyndigste paa saadanne Stæder og Vejen best kand udvise, med videre, som til denne Ende kand eragtes tienligt.\hypertarget{Schn1_2738}{}Innledning.\par
9) Alt øvrigt, som af Mhr Major maatte befindes nøttigt og nu forud af mig ikke kand sees, overlades til H:r Majorens beste Skiønsomhed og Omhue, lige som ieg og efter Haanden forventer deres gode Underretning i Almindelighed om hvad som kand være mig nøttigt at vide.\par
Ieg forbliver med ald Consideration og nest beste Ønske til en lykkelig Reise.\par
Christiania d. 31 Marty 1742.\par
{Welbaarne Høitærede H:r Majors Ergivenste tiener J. C. Rømeling (Sign.)}\par
Til H:r Major Snitler.\begin{figure}[htbp]
\noindent\par
*
\caption{\label{Schn1_2767}}\end{figure}
\par
\textit{Bilag 3. Kongens skrivelse til oberst Rømeling22. mars 1743}.\par
{Christian den Siette, af Guds Naade Konge til Danmark og Norge, de Venders og Gothers, Hertug udi Slesvig, Holsten, Stormarn og Dytmersken, Greve udi Oldenborg og Delmenhorst,}\par
Vor Gunst tilforn, Af den ved Din til Vores Geheime Conferentz Raad, Geheime Raad i Vores Geheime Conseil og Ober Secreterer i Vores Danske Cancellie, Os Elskelig Iohan Ludvig von Holstein, Ridder etc., under den 2:den hujus afladte Skrivelse, nedsendte Forestilling fra Os Elskelig Peder Schnitler, Vores Major ved det 3-die Trundhiemske National Regiment til Foeds, have Vi ladet Os allerunderdanigst referere, at Hand, saasnart Hans kommer til Vor Kiøbstad Trundhiem, vil foretage sin Reyse til Nordlandene og Saltens Fogderie, nærmest det Sted, hvor Hand afvigte Høst endte, for atter at begynde med Vidnernes Examination, og dermed continuere Nordlandene igiennem, indtil i Finmarken, hvor Hand formeener at vil møde Ham adskillige dubieuse Omstendigheder 1) Jdet der i Vest Finmarken ere Lappebyer til Fields, som Kaudiken, Adevar og Arisbye, hvor Sverrig har privative og allenne den verdslige og Geistlige Iurisdiction, dog, som disse Lapfinner om Sommeren, opholder sig ved Fiordene i Norge, reyser Vores Foget engang om Aaret til disse Lapbyer og Skatlegger forskrevne Byers Lap-Finner. 2) Forekommer i Øst Finmarken Lappebyen Indiager eller Enarrabye, hvor Vi saavelsom Russerne og de Svenske concurrere, i at opbære Skatterne, men Sverrig allenne skal have den Geistlige og Verdslige Iurisdiction og 3) Findes der fremdeles i Øst Finmarken Lappebyer, saasom Neiden, Pasvig og Peisen, hvoraf Lapfinnerne om Sommeren sidde ved Fiordene, men om Vinteren til Fields i forbemelte Byer, hvor Russerne allenne øve den Geistlige og verdslige Iurisdiction, og dog betales Skat af Dem til Os og Rusland, da Vores Foget maa reyse til Cola, for af den der værende Russiske Commandant at requirere assistence til Skatternes Jnddrivelse paa disse Steder, af hvilke Omstændigheder Major Schnitler finder sig beføyet, at forespørge sig, om hand kand fare did til disse tvende slags differente Lappe\hypertarget{Schn1_2884}{}Innledning. byer (hvor Vort Rige Norge med forberørte tvende Puissancer i Skatternes Oppebørsel concurrere) og holde noget Vidne-Examen, om Grændsegangen i forskrevne Lappebyer, da Hand i saa Fald er af de Tanker, at det var best, at der blev ladt Ham Frihed, at conferere med Vore Norske Betientere og Geistlige, om Examinations Stedets Berammelse, og at Hand maatte sette Retten der, hvor Hand kunde finde det beleiligste og beste Sted, menende i sær, at Hand ey kand sette nogen Examinations Ret i Kaudiken, Adevar og Arisbye, samt om fleere saadanne findes, siden de Svenske alleene der have ald baade Geistlig og Verdslig Iurisdiction, og Hand der hverken kand faae Assessores eller Vidner, uden de Svenske Betienteres tilladelse og foranstaltning, det og derforuden kunde formodes, at Vidnerne ey ville udsige andet, end i Faveur for de Svenske, da derforuden de Svenske Undersaatters Vidnesbyrd faaes i sin tiid ved Opmaalingen, hvorimod Major Schnitler formeener, at disse Lapfinner, naar de om Sommeren fare need til de Norske Fiorder, Deres Udsagn kunde tages under formelig examination, saafremt Hand, under Samtale, finder Dem overeensstemmende med de Norske Vidner. Samme Beskaffenhed skal det og have paa de Steder, hvor baade Vi, Russerne og de Svenske opbære Skat, der Hand ey kand foretage nogen judical Forretning, Og, som Indiager skal være det første Sted, hvor Rusland interesserer i Skatterne, forespørger Hand sig, om Hand paa samme maade skal gaae til Verks, som forhen er meldet, i fald de Finner om Sommeren, kom need til de Norske Fiorder eller Søekanten, det Hand troer, ey at lade sig giøre uden det Russiske Hofs foreviidende, og uden hvilket saadan Forretning ey kunde blive gyldig, indstillende derfor, om med Examinationen skulle fortfares længer, end til Arisbye eller den sidste Lappebye i Finmarken, hvor Vi og Sverrig alleene interessere, men at med viidere Examen maatte opsettes, indtil der blev sluttet Fred imellem Rusland og Sverrig, dog at Hand hos Vore civile Betientere, Præster og Missionairer samt Norske Undersaatter ved Grændserne under Haanden indhentede ald den underretning Hand kunde faae, udbedende sig Instrux, om Hand i et ordentlig Vidnes Forhør skal examinere Vore egne og Vort Rige Norge privative tilhørende Undersaatter over Grændserne imellem Norge og Rusland Nord i Finmarken paa Norsk Grund paa beleilige Steder, Ey heller holder Hand det giørligt, at holde nogen Iudical Act udi Neiden, Pasvig og Peisen, hvor Vi med Rusland tage deel i Skatterne, men bemelte Rusland privative eene haver ald Iurisdiction, Betreffende ellers den af Major Schnitler beskrevne Reyse til Cola, hvor Hand holder for, at en Officeer ey kand udrette det ringeste til Grændse Commissionens nytte, haaber Hand, i det Sted, allerbest af Finmarkens Archiv og af de der værende Norske Betientere, saavel verdslige som Geistlige, og af de Norske Undersaatter, at kand vorde underrettet, om Beskaffenheden af den paa den Norske side giørende Protestation i Cola, og om alle andre Omstændigheder saavel paa den Russiske som Svenske side, naar Ham maatte lades fri Hænder, at erkyndige sig derom paa beste maade, Til den ende hand haver begieret 1) At der maatte gaae Ordres til Amtmanden, Fogden og andre Vore Betientere i Finmarken, til Commandanten paa Wardøhuus, ligeledes til de Geistlige, Præsterne og Missionairerne, samt Norske Bønder og Lapfinner, at vedkommende ey alleene af det Finmarkske Archiv skulle give Ham copielige Extracter af de Documenter, som høre til Finmarkens Grændse Væsen, baade ad Rusland og Sverrig, men endog, at alle og enhver maatte gaae Ham til Haande og giøre det, Hand, til Grændse Forretningens Fuldførelse, kunde \hypertarget{Schn1_2980}{}Innledning. requirere. 2) At Ham maatte meddeeles Copie af de Fordrage og Transactioner, som angaar Grændse-Skiellet imellem Finmarken og Rusland. 3) At Hand maatte faae Copie af den Journal over Salig og Høylovlig i Hukommelse, Konge Christiani Qvarti Reyse i Norden, og i sær over det Støkke, den Høystsalige Konge tog fra Wardøhuus viidere i Øster og 4) At Ham iligemaade maatte gives Copie af det A:o 1595 imellem Rusland og Sverrig passerede Fordrag, hvorved Rusland har afstaaet til Sverrig sin Deel af Skatterne i Finmarken fra Malanger Fiord til Waranger Fiord, hvilken Deel tilligemed meere Konge Gustav har cederet til Høystbemelte Fierde Christian den Fierde ved den Kærøeske (sic) Fredsslutning A:o 1613. Endelig, som Major Schnitler formoder, tilkommende Vinter at blive over i Nordland eller Finmarken halv andet Aar fra sit Hiem, for at til ende bringe den Ham anbefalede Forretning paa engang, beder Hand til at settes i Stand med erforderlig Provision, Equipage og andre fornødenheder, at maa optage af Vores Casse i Trundhiems Stiftamtstue forud, til sig 600 Rd:r, og til Tolken 50 Rd:r. Da, som Vi af Din ved fornevnte Major Schnitlers Forestilling til Vores Geheime Conferentz Raad von Holstein afladte Skrivelse have fornummen, at Du imod fornevnte Major Schnitlers Forestilling aldeles intet haver at erindre, Saa give Vi Dig hermed tilkiende, at Vi allernaadigst have funden for got, at med Examinationen ey Længer skal fortfares, end til Arisbye, eller den sidste Lappebye i Finmarken, hvori Norge og Sverrig alleene interesserer, Dog at Hand paa Vores side hos Vore Civile Betientere, Præster, Missionarier samt Vore Undersaatter ved Grændserne under Haanden effterspørger og indtager ald den underretning og Kundskab, Ham mueligt bliver at faae, hvori Vi, effter dine Tanker, ey ville foreskrive Ham noget vist, saasom mand ey forud kand viide, hvad Ham in Loco kand møde, og Du derhos beretter, at Hand er en habil, forstandig, flittig, varsom og nidkier Mand; Udi det øvrige overlade Vi oftbemelte Major Schnitler fri Hænder naar Lapfinnerne, om Sommeren, til de Norske Fiorder needfarer, og der opholder Dem med Deres Reensdyr, paa den tiid at rette sin Leilighed effter at være hos Dem, og da, om Hand, under particulier Samtale, finder dem nogenlunde overeensstemmende med de Norske Vidners Udsagn, at tage Dem tillige, som Vidner, under formelig examination, siden De paa den tiid ere paa Norsk Grund, og følgelig under Norsk Iurisdiction. Endelig maa ligeledes Major Schnitler i almindelighed overlades Frihed, med Vore Norske Betientere og Stedernes Geistlige, særdeles Missionairene, at beslutte om Examinations Stedets berammelse, saa at, hvor og naar Hand med Dennem kand finde det beste og beleiligste Sted, til den anbefalede Forretnings Fuldbyrdelse, maa Hand der sette Retten; Betreffende ellers de af Dig begierte Ordres til Amtmanden i Finmarken og Commandanten paa Wardøehuus, da have Vi ey alleene ladet Dem befale, at være Major Schnitler assisterlige udi alt, hvad Hand til denne Ham anbefalede Forretnings Befordring forlanger, men end og erindret Biskopen over Trundhiems Stift, Os Elskelig Doctor Eiler Hagerup, allerunderdanigst at effterleve Vores til Hannem den 16:de Martii udi nestafvigte Aar ergangne Rescript, angaaende den Oplysning, der af Vore Undersaatter skal gives, om Grændsens Gang imellem Norge og Sverrig, ligesom Vi og allernaadigst have ladet befale Stiftamtskriveren i Trundhiem, paa Major Schnitlers anfordring og imod Hans Qvitering at betale til Hannem de begierte 600 Rd:r til den forestaaende Reyse til Nordlandene og Finmarken og 50 Rd:r til den Tolk, som skal følge med Ham paa Reysen; Dereffter Du dig allerunderdanigst haver at rette, og Major Schnitler af dette \hypertarget{Schn1_3106}{}Innledning. Vores allernaadigste Rescript den fornødne Efterretning at meddeele; Sluttelig indberettes Os af Dig allerunderdanigst, at dette er Dig til Hænde kommen, Befalendes Dig Gud, Skrevet paa Vort Slot Christiansborg i Vores Kongelig Residentz Stad Kiøbenhavn den 22:de Martii A:o 1743.\par
\centerline{Under Vor Kongelig Haand Og Signet Christian R (Sign.)}{L. S. _______________ J. L. v. Holstein (Sign.)}\par
Til\par
Oberste Iobst Conrad von Rømling, at med Vidnernes Examination, ang. Grændse Skiellet imellem Norge og Sverrig, ey længere skal fortfares, end til Arisbye eller den sidste Lappebye i Finmarken, hvori Norge og Sverrig allenne interesserer; Dog at Major Schnitler under Haanden indtager ald muelig underretning og Kundskab af Hans May:ts civile og Geistlige Betientere og Undersaatter, hvori Major Schnitler ey noget vist foreskrives, men hannem overlades fri Hænder med Lapfinnernes Examination, naar De om Sommeren needfarer til de Norske Fiorder, med viidere.\hspace{1em}
\DivI[III. Arkivmaterialet efter Schnitlers grenseeksaminasjoner]{III. Arkivmaterialet efter Schnitlers grenseeksaminasjoner}\label{Schn1_3166}\par
\centerline{\textbf{III. Arkivmaterialet efter Schnitlers grenseeksaminasjoner.}}
\DivII[A. Eksaminasjonsprotokollene]{A. Eksaminasjonsprotokollene}\label{Schn1_3176}\par
A. EKSAMINASJONSPROTOKOLLENE \textit{1. Originalprotokollene}.\par
Iberegnet den av Schnitler selv som «7de volum» betegnede «Beskrivelse af VaardehusAmt eller Finmarken», utgjør de originale grenseeksaminasjonsprotokoller som efterhvert blev tilstillet grensekommissærene, syv i tall.\par
De var i eldre tid registrert i Riksarkivets grensereguleringsarkiv som vol. XXI‒XXVI b. I den nuværende arkivordning er de ført sammen i pakkene 10 (XXI‒XXII), 11 a (XXIII‒ XXVI a) og 11 b (XXVI b).\par
Bare de egentlige eksaminasjonsprotokoller er trykt i nærværende utgave Bind I‒II, altså bare vol. 1‒6. Og av vol. 6 er sløifet den del som bare er avskrift fra vol. 5. (Herom nærmere nedenfor.) Ellers er protokollenes tekst på enkelte undtagelser nær gjengitt in extenso med originalens stil og ortografi. De få undtagelsene er det på vedkommende sted gjort opmerksom på.\par
Publikasjonen åpnes da med «Widners Examination over Grendserne imellem Norge Nordenfields og Sverrige Anno 1742. Paa Vinterføret 1 Volumen».\hypertarget{Schn1_3215}{}Innledning.\par
Iberegnet bilagene optar dette 1ste volumen s. 1‒137 i nærværende Bind I. Vidneforklaringene er ført i pennen av Schnitlers skriver på denne reisen, «capitaine des armes» ved det Størenske kompani Peter Jacob Røyem. Bilagene er skrevet dels med Schnitlers, dels med Røyems og dels med andres hånd.\par
Det næste volumen: «Widners Examination over Grendserne imellem Norge Nordenfields og Sverrige Anno 1742 paa Sommerføret. 2 Volum», er som vol. 1 ført i pennen av Røyem. Den første delen av dette volumen med bilag er trykt i nærværende Bind I, s. 138‒178, og det øvrige i Bind II, s. 1‒104.\par
Av de to følgende volumer er «Vidners Examination over Grendserne Ymellem Norge Nordenfields og Sverrig A° 1743 om Sommeren, 3die Volumen» trykt i efterfølgende Bind II, s. 105‒249, og «Vidners Examination over Grendserne imellem Norge Nordenfields og Sverrig A° 1743 om Høsten 4de Volumen. Senjens og Tromsøens Fogderier udi Nordland» trykt i Bind II, s. 250‒430. Noen korte avsnitt av 3dje og 4de vol., som var utelatt ved trykningen av Bind II, finnes trykt i nærværende Bind I, s. 179‒206. ‒ Bortsett fra dem av bilagene, som er skrivelser eller erklæringer fra andre, er begge disse volumene i sin helhet skrevet med Schnitlers hånd.\par
Det 512 sider store vol. 5 er nu uten noget titelblad. Det gjelder grenseeksaminasjonene i Vardøhus amt eller Finnmarken og er i sin helhet ført i pennen av Schnitler. Brev eller lignende, skrevet av andre, finnes ikke som bilag i vol. 5. Det finnes nu trykt i Bind I, s. 207‒415.\par
Det lille, bare 100 sider store, vol. 6 står i en særskilt stilling. Utenpå omslaget har Schnitler skrevet: «Grændse-Examinations Protocolls 6te volumen. Ang. Grændserne ad Russisk Lapmark.» Men ellers er det i sin helhet skrevet med andre hender. Side 1‒42 er en «Extract af den 5te Examinations Protocoll over Waardehuus Amt» forsåvidt angår de norsk- svenske fellesdistrikter Kautokeino, Avjovarre (Karasjok) og Aritsby (Utsjok), det norsk- svensk-russiske fellesdistrikt Indjager (Enare) og spesielt de norsk-russiske fellesdistrikter Neiden, Pasvig og Peisen (Petsjenga). Denne delen er skrevet med Schnitlers ledsager Erik Helsets\hypertarget{Schn1_3353}{}\footnote{\label{Schn1_3353}Helset (skrives også Hælset eller Helsæt) var en tidligere lærer og ikke ordinert finnemisjonær, som fulgte Schnitler helt fra Namdalen til Varanger og tilbake til Trondhjem. Hans opgave var først og fremst å være tolk overfor samene, men han fungerte også som skriver og karttegner. Han blev senere tollbetjent i Tønsberg.} hånd uten spor av rettelser eller tilføielser med Schnitlers. Da denne delen av vol. 6 allerede er trykt som en del av vol. 5 (s. 345‒361), er den ikke medtatt pånytt under vol. 6.\par
Men resten av vol. 6, s. 43‒100 er vidneforklaringer vedkommende grensene mot Russland, som bare kjennes fra vol. 6, og som derfor nu er trykt i nærværende Bind I, s. 416‒435. De er skrevet med en fremmed hånd, som vel også har skrevet en større del av nedennevnte kopi av vol. 5. Riktigheten av hans protokollasjon i vol. 6 er ikke in optima forma bekreftet av Schnitler. Men det er Schnitler som har skrevet titelen på omslaget og som har gjort endel mindre tilføielser til denne delen s. 43‒100, så han må vel dermed sies å ha godkjent den.\par
Vel å merke bare som kopi. Utenpå omslaget står også skrevet ‒ visstnok med den ukjente skrivers hånd ‒ ordet «Copie». Forholdet synes da å være det, at den egentlige av Schnitler selv førte protokoll over de vidneforklaringer som er nye i vol. 6, er gått tapt, men først efter at avskriften av dem var tatt og ført inn i en protokoll, som egentlig bare var be\hypertarget{Schn1_3404}{}Innledning. regnet som kopiprotokoll. Og så har Schnitler sett sig nødsaget til å godkjenne denne som original uten dog å gjøre uttrykkelig opmerksom på forholdet. I den nedenfor omhandlede serie kopier av eksaminasjonsprotokollene finnes da heller ikke nogen kopi av vol. 6.\par
\textit{2. Konsepter til eksaminasjonsprotokollene}.\par
Som ovenfor nevnt synes originalprotokollene, hvad selve eksaminasjonene angår, å være ført direkte i pennen, for 1ste og 2det vol.s vedkommende av capitaine des armes Røyem, for 3dje til og med 5te vol.s vedkommende av Schnitler. Hvad 6te vol. angår synes dette, som nevnt, i virkeligheten å være et kopieksemplar. Det er i sin helhet skrevet med andre hender enn Schnitlers eller Helsets. Men sannsynligvis har selve originalprotokollen, som nu er ukjent, vært skrevet av Schnitler selv.\par
De mange i eksaminasjonsprotokollene innskutte redegjørelser og beskrivelser kan dog neppe være skrevet i den foreliggende form uten konsept. Men noen samling av slike konsepter finnes ikke. Sannsynligvis har Schnitler tilintetgjort dem efterhvert. Hist og her i hans papirer finnes dog stumper av papirer, endog hele ark, skrevet med hans hånd og med et innhold som tyder på, at de kan ha vært deler av konsepter. Nogen særskilt registrering og granskning av disse er dog ikke foretatt.\par
\textit{3. Kopier og utskrifter av eksaminasjonsprotokollene}.\par
I Riksarkivets registratur til grensereguleringsarkivet er pk. 15, vol. XXXII‒XXXVI, betegnet som inneholdende «Kopier eller Koncepter til de foran (Vol. XXII‒XXVI) anførte Grænse-Examinationsprotokoller». Denne innholdsangivelsen går tilbake til den i Det danske kanselli førte fortegnelse. Men den er misvisende. Pakken inneholder nemlig bare avskrifter av de fem første eksaminasjonsprotokollene, ikke konsepter til disse.\par
Hvad de enkelte volumina angår, kan bemerkes:\par
ad \textbf{Vol. 1:} Både selve eksaminasjonene og de fleste av de innføiede relasjoner m. v. er skrevet med samme hånd som i originalprotokollen, altså Røyems. Avskriften er ikke uttrykkelig merket som kopi, men viser sig på alle måter å være det. Og tilslutt har Schnitler med sin egenhendige underskrift godkjent den.\par
Den efterfølgende «Designation over 1) de Østligste fielde ved Herjedalen og 2) over de gamle Lande-Merker imellem Norge og Jemteland osv.» er i sin helhet skrevet av Schnitler selv. Med hans hånd er også teksten til det her i kopieksemplaret utelatte kartriss skrevet.\par
ad \textbf{Vol. 2:} Til forskjell fra vol. 1 er vol. 2 på titelsiden uttrykkelig betegnet som «copie». Selve protokollen er skrevet med Røyems hånd. Men med sin egenhendige underskrift har Schnitler godkjent den. En større del av kolumnetitlene over sidene er også skrevet med Schnitlers hånd, bare s. 222‒44 er de skrevet med Røyems.\par
Av de mange og vidløftige bilagene er det alfabetiske stedsregistret (s. 251‒77) påbegynt av Schnitler, men fortsettelsen s. 252‒71 er vistnok skrevet av Helset. Den derpå følgende redegjørelsen for de bevidnede grensemerker er i sin helhet skrevet av Schnitler, likeså det\hypertarget{Schn1_3508}{}Innledning.\par
overmåde detaljerte «Register af Vidnernes Examinations Protocol 2det Volumen over de derudi benævnte og forklarede Navne af Landemerker, Fielde, Søer, Elver, Gaarder osv.»\par
Avskriftene av bilagene A‒R, s. 281‒309, er derimot med Helsets hånd. Blandt disse er også (s. 297) en kopi av Aron Nordmands «schiographie» (kart) over Rana, som var bilag til Nordmands skrivelse til Schnitler25/8 1742.\par
ad \textbf{Vol. 3:} Dette er i sin helhet med registre og bilag skrevet med Helsets hånd. Det er ikke påtegnet «copie» og er heller ikke undertegnet av Schnitler personlig. Men hist og her støter man på små rettelser og tilføielser med Schnitlers hånd. Dette viser, at kopien er gjennemgått og dermed også i realiteten godkjent av ham.\par
ad \textbf{Vol. 4:} M. h. t. dette gjelder i alt vesentlig de samme bemerkninger som ad. vol. 3. Det er skrevet med Helsets hånd, hist og her med små rettelser med Schnitlers. Og så er hele baksiden av den store, i nærværende trykte utgave ikke medtatte «Summarisk Tabell», altså alle oplysningene om Lofotens og Vesterålens fogderier m. v., tilføiet av Schnitler personlig.\par
ad \textbf{Vol. 5:} Dette er en fullstendig kopi av originalens 5te volumen. Men den er skrevet med forskjellige hender. Fra og med s. 1 til næsten ned s. 44 er det Helsets hånd, men derfra helt til midt på s. 396 er det en fremmed hånd, visstnok den samme som har skrevet den ovenomhandlede andre delen av vol. 6.\par
Reiseberetningen s. 396‒419 er skrevet av Schnitler personlig, likeså de siste 4 sidene av det upaginerte bilaget «Underretning til den provisionelle Grendse-Befaring» og titelsiden til «Alphabethisk Register over Navnene i forestaaende Protokolls 5te Volumen». Resten av registret er skrevet med Helsets hånd. Men forøvrig er det den fremmede hånd vi møter i dette kopieksemplaret av vol. 5.\par
Som allerede nevnt finnes ikke her nogen kopi av vol. 6. Og heller ikke av den store innholdsrike Finnmarksbeskrivelsen, der som foran nevnt noget misvisende er blitt arkivert som eksaminasjonsprotokoll vol. 7.
\DivII[B. Schnitlers journaler]{B. Schnitlers journaler}\label{Schn1_3588}\par
B. SCHNITLERS JOURNALER\par
I Grensereguleringsarkivets pakke 17 finnes bl. a. de gamle volumina XLII og XLIII, anført i registranten som: «Schnitlers Journal over Grændsekommissionsakten 1742‒1745 (også i koncept).»\par
Den foreligger altså både i renskrift og konsept, begge innheftet i grått papiromslag. Renskriften, som i sin helhet er skrevet med Schnitlers egen hånd, har følgende titel på omslaget, skrevet av Schnitler:\par
«Journal over Grændse-Commissions Acten fra 1742 til 1745.»\par
Med andre hender er notert oventil «42de Volumen» og nedentil «No. 11».\par
Første blad i bindet er benyttet til en hel del senere overstrøkne kladdeoptegnelser.\par
På det annet blads forside er skrevet følgende titel:\par
«Journal over min Reise og Forretning ved den anordnede Grændse-Commission imellem Norge og Sverrig af mig tiltrædet, og begyndt i Tronhiems Amt d. 7 April: 1742».\par
Før Schnitler side 14 begynner den dagbokmessige redegjørelse for reisenes og arbeidets \hypertarget{Schn1_3653}{}Innledning. gang, har han på de foregående sider 1‒13 innført kopier av de for hans mandat og arbeide til grunn liggende dokumenter, nemlig\par
1) den kongelige ordre av 16/3‒1742 og\par
2) oberst Rømelings instruksjonsskrivelse av 31/3‒1742 med et utdrag av den av F. O. v. Rappe og M. v. Sundt utfærdigede og av kongen 14/7‒1738 approberte «Instruction for Hs. May'ts. Betiente ved Grændse-Commissionen.»\par
Redegjørelsen for hvad Schnitler fra dag til dag har foretatt sig, hvorhen han har reist, til hvem han har skrevet brever og fra hvem han har mottatt sådanne, går dog ikke bare til tilbakekomsten til Trondhjem efter avsluttede reiser 29/11‒1745. Side 134 fortsetter han umiddelbart med en lignende redegjørelse for hvad han efter 29/11‒1745 har foretatt sig, hvem han har skrevet til eller fått brever fra i anledning av grensekommisjonens arbeide. Denne redegjørelsen stopper dog (side 143) med 26/4‒1746, men er av betydning for kjennskapet til alt det efterarbeide Schnitler fikk efter at reisene til Nordland og Finnmarken var avsluttet.\par
Journalkonseptet har på det fettede og makulerte gråpapirs omslag følgende titel med Schnitlers hånd: «Journalens(!)-Concept over Grændse-Commissions Acten fra 1742 til 1745» og med andre hender tilføiet: «43de Volumen» og «No. 12».\par
Også i dette bind eller hefte er det første blad benyttet til kladdeoptegnelser, som for størstedelen senere er overstrøket. Blandt disse notatene finnes flere kvitteringer fra Helset, som dels har skrevet sitt navn «Erich Hælsætt» og dels «Erich Helsætt», ennvidere fra Røyem og (i 1743) en Lochnæs.\par
Også journalkonseptet er for størstedelen skrevet med Schnitlers hånd. Men man finner også andre hender i boken. De dokumentavskrifter, som Schnitler egenhendig førte inn i journalen (renskriften), er i journalkonseptet skrevet med en fremmed hånd (en barnehånd? monn en av sønnene?).\par
Man støter også på Helsets hånd, men det er ikke i de egentlige dagboksoptegnelser, men i et legg på et par ark, som er innlagt i journalkonseptet ved 30/11‒1744. Det inneholder endel brevavskrifter og synes egentlig å ha hørt hjemme i kopiboken, hvor der dog ikke finnes nogen tilsvarende lakune. Blandt brevavskriftene i dette løse legget finnes også et av Schnitler selv innført brev på latin til kanselliråd og lagmann Povel Dons i Trondhjem, avsendt fra Polmak («Bolma ved Tana Elv») 9/11‒1744.\par
Ved journalkonseptet er forøvrig å merke, at det ikke som renskriften stopper 26/4‒1746, men går videre til 17/6 s. å.\par
Nogen brev- eller arbeidsjournal for Schnitlers virke efter denne dato er derimot ikke kjent.
\DivII[C. Schnitlers kopibøker]{C. Schnitlers kopibøker}\label{Schn1_3793}\par
C. SCHNITLERS KOPIBØKER\par
Grensereguleringsarkivets pakke 16 inneholder de gamle volumina XXXII‒XXXVI, i registranten betegnet som «Oberstltnant Schnitlers Kopibøger vedkommende Grændsebefaringen».\par
Kopibøkene er fem i tall, alle innheftet i grått papiromslag i likhet med journalene og de ikke innbundne eksaminasjonsprotokoller.\hypertarget{Schn1_3808}{}Innledning.\par
1) Den første av bøkene bærer på omslaget følgende påskrift med Schnitlers hånd: «Copie-Bog af Brevene ang. Grændse-Commissionen. 1te Bind fra Ao 1741 til Ao 1743». Med andre hender er tilføiet «37te Volumen» og «No. 6».\par
Boken har også et indre, defekt og makulert omslag, som vel er det oprindelige. På dette står skrevet: «Copie-Bog over Grentze-Forretningen. P. S. ‒ ‒ ‒ 1741, 1742, 1743.»\par
Boken er ikke paginert. Den inneholder kun kopier av utgående brever (og av bilag til disse) og er for størstedelen skrevet med Schnitlers hånd, delvis også med Røyems, Helsets og enkelte andre hender, heriblandt en barnehånd (jfr. foregående side), som kun påtreffes i brever avsendt fra Trondhjem og som jeg har antatt er et av Schnitlers barn.\par
Kopiboken begynner ikke først i 1742, efter at Schnitler hadde fått i opdrag å foreta sine grenseeksaminasjoner, men allerede 12/8‒1741 med et brev på tysk til den daværende grensekommissær generalmajor og stiftamtmann Rappe, Christiania, hvori Schnitler på vegne av Rørosverkets direktør og participanter fremholder behovet for en gunstig riksgrenselinje ved Femunden.\par
Også adskillige andre brever til forskjellige mottagere finnes innført i tiden før Schnitler i april 1742 efter mottatt opdrag begynte sine reiser. Blandt disse brevene finnes flere til Rappes efterfølger som grensekommissær, oberst Rømeling. Disse er skrevet på fransk, det sprog som Schnitler og Rømeling som oftest korresponderte på.\par
Leilighetsvis finnes også annet enn brevkopier innført i boken, således under 1/2‒1743 en kvittering fra Helset for å ha mottatt et pengebeløp av Schnitler.\par
Tildels er brevkopiene ikke egentlig kopier av, men kladder til utgående brever, hvilket gir sig tilkjenne ved betydelige overstrykninger og rettelser.\par
Boken slutter 4/5‒1743 med et av Schnitler selv innført brev på fransk til Rømeling, hvori Schnitler melder ham, at han s. d. sammen med tolken Helset går ombord i en nordfarbåt, som er den første som begir sig til Vefsen, hvor Schnitler skal gjenopta sitt arbeide.\hspace{1em}\par
2) Den andre boken har på det defekte omslag med Schnitlers hånd titlen: «Copie-Bog af Brevene af 1743 og 1744 angaaende Grendse-Commissionen. 2det Bind». Under dette omslaget sees restene av et eldre.\par
Boken begynner 4/5‒1743 med et notat om det brev av s. d. til Rømeling, som var innført sist i første bind, og slutter 6/3‒1744 med et av Helset innført brev fra Schnitler til «Velbaarne Hr. Obriste» (Rømeling? Mangelsen?), efterfulgt av et par dokumentavskrifter, likeledes med Helsets hånd.\par
Også denne boken er dog for størstedelen ført med Schnitlers egen hånd, men dessuten påtreffes foruten Helsets den ovenfor under omtalen av eksaminasjonsprotokollenes 5te og 6te volumina omtalte skriverhånd. «Barnehånden», som i første bind av kopiboken dukket op i brevkopier skrevet i Trondhjem, finnes derimot ikke i dette bind.\par
Eiendommelig for dette 2net bind av kopiboken er derimot to karter, som er innheftet i boken, nemlig over «Helgelands Fogderies Grendser» og over «Sennien og Tromsøens Fogderiers Grendser».\hypertarget{Schn1_4002}{}Innledning.\par
3) Det tredje bind har sålydende påskrift på omslaget med Schnitlers hånd: «Brev-Copiebog fra 12te Martij 1744 til ultimo December samme Aar ang. Grændse-Commissionen. 3die Bind». Med andre hender er tilføiet «39te Volumen» og «No. 8». Under dette omslaget sees restene av et eldre, sterkt makulert.\par
Innen denne kopiboken egentlig begynner 12/3‒1744 med et brev fra Schnitler til prost Falck i Talvik, finnes innheftet to legg med avskrifter som mere tilfeldig synes å ha fått sin plass her, det ene med brev- og dokumentavskrifter fra 1742, det annet inneholdende avskrift av et brev av 25/3‒1744 fra Schnitler til amtmann Kjeldsøn med dennes svar av 4/4‒1744 og av Schnitlers brev til general Arnoldt15/4 s. å.\par
M. h. t. håndskrifter er dette bindet like med det foregående. Håndskriften fra 5te og 6te eksaminasjonsprotokoll møter man dog her i større omfang. Likeså støter man på ovenomtalte «barnehånd». Men denne møter man kun i to legg, som ved nærmere eftersyn viser sig å være av noget mindre format enn de øvrige og tydeligvis er skrevet under Schnitlers korte ophold i Trondhjem i mai 1744.\par
Efter den siste ordinære brevavskrift ‒ et brev fra Schnitler fra Vadsø30/12‒1744, innført med skriverhånden fra 5te og 6te eksaminasjonsprotokoll ‒ følger endel spredte innførsler med Schnitlers hånd fra 1744 og 1745, delvis av notatmessig og kladdemessig art.\hspace{1em}\par
4) Det fjerde bind har følgende påskrift med Schnitlers hånd på det sterkt makulerte omslag: «Brevers Copiebog ang. Grændse Commissionen. 4 Volumen fra 1745de Aars Begyndelse». Med andre hender er tilføiet «40de Volumen», «Copiebog» og «No. 9».\par
Innenfor dette ytre omslaget finnes et velbevart indre omslag, kun med ordet «Copie» i øvre venstre hjørne. Dette indre omslaget er vel i virkeligheten nyere enn det ytre, men er ved innheftningen kommet innenfor istedetfor utenfor det ældre.\par
Boken begynner med en skrivelse av 7/1‒1745 til foged Wedege og tolken Helset, innført med den fra 5te og 6te eksaminasjonsprotokoll kjente skriverhånd, som også har skrevet det meste inntil 2/3‒1745. Derefter er boken for det meste skrevet med Schnitlers egen hånd og har da ofte karakteren av kladd eller konsept.\par
Midt inne i boken påtreffes «barnehånden», i brevavskrifter som efter påtegningen av Schnitler viser sig å være skrevet først efter hans tilbakekomst til Trondhjem.\par
Brevene fra reisen slutter med et av Schnitler selv innført brev av 5/11‒1745 fra Vinnesund i Helgeland til oberst Mangelsen. Men efter 1 1/2 blanke sider fortsetter kopiboken med et brev av 30/11‒1745 til stiftamtmann Stockfleth. Dette er innført med barnehånden, hvormed bl. a. også et latinsk brev av 13/12‒1745 til sønnen Lorenz («Mi fili Laurentz»), som da utvilsomt har vært i København, er innført. Foruten barnehånden og Schnitlers og Helsets hender påtreffes også en eller et par andre hender i bokens siste del, som slutter med et av en fremmed hånd innført brev av 30/12‒1745 fra Schnitler til sorenskriver Kiergaard på Gibostad i Senja.\hspace{1em}\par
5) Det femte bind bærer på omslaget følgende påskrift med Schnitlers hånd: «Brev- Copie-Bog angaaende Grændse-Commissionen 5te Volumen fra 1746de Aars Begyndelse». Med andre hender er tilføiet «41de Volumen» og «No. 10». I motsetning til de fire første \hypertarget{Schn1_4192}{}Innledning. bindene har dette femte ikke noget eldre omslag innenfor det ytre, hvilket utvilsomt henger sammen med, at denne femte kopiboken ikke som de fire første har vært med på reisene og derfor har kunnet bevares bedre.\par
Boken begynner med et dels av en skriverhånd og dels av «barnehånden» innført brev fra Schnitler av 10/1‒1746 til geheimeråd von Holstein og slutter med et av en fremmed hånd innført brev av 21/2‒1750 fra Schnitler til generalmajor Mangelsen i Strömstad.\par
Boken er kun i mindre utstrekning skrevet med Schnitlers hånd, da ofte på en kladdemessig måte.\par
Barnehånden treffes ofte, senest i et brev av 16/11‒1748 til amtmann Kjeldsøn.\par
Spredt ut gjennem boken påtreffes bl. a. også en ny, meget jevn og klar håndskrift. Med denne hånd er også skrevet en bakerst i boken løst innlagt «Beskrivelse af Landskabernes Beskaffenhed paa begge Sider af Grændsene imellem Norge Nordenfields og Sverrig saavelsom af Vejene og Passagene ud af det Nordenfieldske ind ad Sverrig». Beskrivelsen er med Schnitlers hånd underskrevet «Tronhjem d. 8 Aug. 1750. P. S.» og må derfor være den kopi Schnitler  har latt ta av beskrivelsen, hvis original efter innholdet å dømme må være sendt til general Arnoldt. Denne nye hånd er utvilsomt den samme som den, hvormed nedenomhandlede grensekart nr. 121 er skrevet. Men hvem skriveren er, er ennu ikke konstatert.\par
Innlagt i denne beretningsavskriften ligger et originalbrev til Schnitler fra J. C. Bay\hypertarget{Schn1_4271}{}\footnote{\label{Schn1_4271}Se Ovenstad: Militærbiografier I, s. 65.}, dat. Østvold7/12‒1750.\par
Lengere foran i boken ligger løst innlagt et originaldokument fra Fr. Hammond\hypertarget{Schn1_4288}{}\footnote{\label{Schn1_4288}Se Norsk biografisk leksikon V, s. 324 flg.}, dat. Trondhiem21/2‒1749, hvori Hammond erkjenner å ha mottatt tilbake fra Schnitler 7 nærmere spesifiserte gamle dokumenter og kartkonsepter, som Schnitler i sin tid hadde mottatt tillåns av den daværende grensekommissær generalmajor Rømeling.\par
Likeledes finner man, dels innheftet i boken og dels løst innlagt, avskrifter av brever til Schnitler, således (innheftet) kopi av Morten Lunds brev til Schnitler av 5/4‒1746 ang.Masi kapell mellem Alta og Kautokeino og (løst innlagt) av Mangelsens brev av 14/5‒1746 til Schnitler, med påført konsept til svarbrev av 11/6 s. å. Innlagt i dette ligger Schnitlers egenhendige kladdekonsept til et latinsk brev av 2/3‒1744 til ...... (uleselig).
\DivII[D. Schnitler-kartene]{D. Schnitler-kartene}\label{Schn1_4352}\par
D. SCHNITLER-KARTENE \textit{1. Den kartografiske situasjon i Norge, da Schnitler tok fatt}.\par
Da Schnitler i februar 1742 påbegynte sine grenseeksaminasjoner, var det smått stell med karter som kunde gi ham oversikt over de grensetrakter han hadde fått til opgave å befare.\par
Datidens trykte atlaskarter over Skandinavien i sin helhet eller særskilt over Norge var alle utgitt i utlandet, i Holland, Tyskland, Frankrike eller England og var praktisk talt uforandrede kopier efter de svenske Bureus-karter av 1626 eller den noget endrede, men ikke vesentlig forbedrede hollandske utgaven av Bureus fra 1635.\hypertarget{Schn1_4406}{}Innledning.\par
For Sydvest-Norges vedkommende med tillegg av Hallingdal og Valdres bygget disse kartene av Bureus-typen på de forholdsvis meget gode Scavenius-kartene over det gamle Stavanger stift fra ca. 1620, som kom i flere, innholdsmessig sett i hovedsaken uforandrede, utgaver i de følgende årtier utover i 1600-årene.\par
Men for de norsk-svenske grensetrakters vedkommende er hele denne på Bureus-kartene grunnede internasjonale kartografi ytterst mangelfull og misvisende, særlig da for Trøndelagen og Nordland. Meget bedre er fremstillingen av Finnmarken. Dette kommer av, at svenskene i en årrekke før Kalmarkrigen hadde drevet en ganske intens spionasje i Finnmarken, med fri adgang til det åpne verdenshav, Nordishavet, som det eftertraktede mål. Denne spionasjevirksomheten hadde gitt sig et klart uttrykk allerede i det av den samme \textit{Anders Bure}, latinisert: Andreas Bureus, utarbeidede «Lapponia-kart», som forelå trykt i 1611, altså i samme år som Kalmarkrigen brøt ut.\par
Ved Knærødfreden 1613 måtte dog Sverige som bekjent avstå fra alle krav på eiendoms- eller beskatningsrett til Finnmarkskysten. Men ikke bare Lapponia-kartets topografi, men også dets med fine, prikkede linjer avmerkede grensekrav gikk for Nord-Norges vedkommende over i de senere Bureus-karter av 1626 og 1635 og i mange av disses efterfølgere i den internasjonale kartografi. Såvidt vites blev de dog ikke direkte påberopt under de undersøkelser og forhandlinger som gikk forut for Strömstadtraktaten av 1751.\par
Anderledes så det ut på de karter over de svenske lappmarker, som blev optatt av den svenske landmåler \textit{Olof Tresk} i 1640-årene efter offentlig opdrag og detaljerte feltundersøkelser. Grensen mellem Sverige (inclusive Finnland) på den ene side og Norge og Russland på den andre trakk Tresk op efter det forbløffende nøiaktig bestemte hovedvannskill mellem Den botniske bukt på den indre side og Atlanterhavet, Nordishavet og Kvitsjøen på den ytre.\par
Følgen herav var øiensynlig, at disse kartene blev lagt ad acta. De blev ikke trykt eller kopiert eller gjort bruk av. Og først i 1928 er de blitt offentliggjort i trykken, i et festskrift til professor K. B. Wiklund, med en historisk innledning av professor Nils Ahnlund.\par
Lignende skjebne fikk et av det svenske Generallantmäteri i 1688 under ledelse av \textit{Carl Gripenhjelm} utarbeidet anonymt og titelløst kart over Sverige og Finnland med Nord-Norge og deler av Kolahalvøen og russisk Karelen m. v.\par
På et vis kan dette kartet betegnes som en revidert og ajourført gjengivelse av Bureuskartet av 1626. De tidligere danske landskaper Gotland, Blekinge, Skåne og Halland er inndradd under Sverige og likeså de gamle norske landskapene Bohuslen, Idre-Særna, Herjedalen og Jemtland. Men nordpå er grensen mellem Norge og Sverige ført ned til Malangen uten antydning til nogen fortsettelse nordover gjennem Finnmarken. En grense mellem Västerbotten og Torne lappmark i vest og Österbotten og Kemi lappmark i øst er trukket fra nordenden av Den botniske bukt (mellem Torneå og Kemi) omtrent rett nordover næsten helt frem til Altafjordens indre ende, mens grensen mellem Österbotten og Kemi lappmark i vest og de østenforliggende russiske landskaper stanser et godt stykke sønnenfor Kolajärvis sydvestre ende uten nogen fortsettelse til Nordishavet østenfor.\par
Var et kart med en sådan fremstilling av grenseforholdene nordpå blitt offentliggjort i trykken og kjent i Danmark-Norge og i Russland, vilde det lett kunne ha medvirket til en ny \hypertarget{Schn1_4642}{}Innledning. Kalmarkrig. Heldigvis for alle parter blev kartet dog liggende i arkivet og er først i 1907 blitt publisert, av Sven Lönborg i en minnepublikasjon til Carl von Friesen i Swedish Maps, First Series.\par
«Det er vel helst bare et tilfelle, men det ser ut som en tanke», at det nettop var høsten 1688\textit{Melchior Ramus} tilbød kong Christian V å kartlegge hele Norge, et tilbud som pr. omgående blev mottatt og som resulterte i, at han høsten 1692 kunde avlevere til kongen sitt store, detaljrike kart over vårt land, utført helt og holdent av Ramus selv uten på noget punkt for Norges vedkommende å være kopiert efter eldre karter. Hverken hans store Norgeskart eller noget av de karter over visse deler av landet, som han helt eller delvis fikk utarbeidet innen sin tidlige død i 1693, var datert eller signert. Og bortsett fra at en av U. Fr. Aagaard i 1706 utført kopi av «salig Melchior Ramus»' kart over Finnmarken ga til kjenne, hvem originalen skrev sig fra, er det først ute i 1750-årene, at vi støter på en del karter som ‒ delvis med urette ‒ blev angitt å være kopier efter Melchior Ramus. Og noget kart fra hans hånd er ennu ikke blitt utgitt i trykken, bortsett fra at nogen av dem er blitt gjengitt i meget sterkt forminsket skala, spesielt i mitt Bidrag til Norges karthistorie, III, i Norsk geografisk tidsskrift 1943.\par
At Schnitler har hatt noget kjennskap til Melchior Ramus' kartografi er derfor litet sannsynlig. Men fra 1719 av forelå et helt nytt trykt kart over Norge, som Schnitler iallfall kan ha sett og funnet nogen veiledning i. Det er den ganske merkelige «Delineatio Norwegiæ novissima», som kun er trykt som bilag til den av \textit{Jonas Ramus}, Melchior R.s yngre bror, utgitte «Norriges Kongers Historie».\par
Dette kartet er tegnet i liten målestokk for på ett blad å kunne rekke helt fra Götaelvens munding i syd til Varanger i nord. Det medtar også Idre og Særna, Herjedalen og Jemtland, har i stor utstrekning gammelnorske stedsnavn, er uten grenselinjer og fremtrer i det hele tatt som et historisk kart.\par
Kildene til dette kartet skriver sig fra flere hold. For det sydvestre Norges vedkommende er det en forenklet kopi efter Scavenius. For Bohuslen, Østlandet og Trøndelagen er det efter Geelkerck (1644‒57) og dennes militære efterfølgere i annen halvdel av 1600-årene. Og hele Nord-Norge skyldes skipperen og sjøkartografen Johan Heitman.\par
Men for Nord-Norges vedkommende finnes ingen topografiske detaljer eller stedsnavn innenfor kyst- og fjordkonturene. Og så liten målestokk som dette kartet var tegnet i, kunde det ikke ha gitt Schnitler annet eller mere enn en smule oversikt over det veldige område, han hadde fått i opdrag å befare og undersøke m. h. t. grensespørsmålene, fra Femunden og Røros-traktene i syd til Varanger lengst i nord.\par
Hvad som derimot er mulig, ja sannsynlig, er at Schnitler kan ha kjent og gjort bruk av en kopi av det håndtegnede Norges-kartet i stor målestokk, som Jonas Ramus' trykte, historiske kart i topografisk henseende er en forminsket gjengivelse av. Selve originalen til dette store Norges-kartet er dessverre forsvunnet. Men kopier av det kjennes fra første halvdel av det 18de århundre. Vel er det beheftet med mange feil. Vesterålen og øyverdenen omkring Tromsø er f. eks. meget misvisende fremstillet. Men på den annen side er f. eks. fjordene og øyene i Finnmarken alt ialt ganske vel fremstillet, riktigere enn på andre karter fra den tid. Og her synes dette kartet å ha vært til adskillig hjelp for Schnitler.\hypertarget{Schn1_4828}{}Innledning.\par
Vi skal nu se, hvad den kartografiske autodidakt Schnitler også ved sine karter bidrog til løsning av sin og grenseingeniørenes felles opgave.\par
\textit{2. Schnitlers egne kartarbeider}. a. {Vedrørende Trøndelagen}. (Eksaminasjonsprot. Vol. 1.)\par
Sammen med hver av de følgende fire eksaminasjonsprotokoller vol. 2‒5 innsendte Schnitler også et av ham utarbeidet kart over vedkommende landsdel med tilstøtende grenseområde. Sammen med hans egne konsepteksemplarer ligger disse nu i Riksarkivets grensearkiv nummerert og katalogisert. Men nummereringen er hverken skjedd efter nogen geografisk eller kronologisk rekkefølge. Med brudd på den uheldige nummerorden må her den geografiske rekkefølge bli fulgt. I hovedsaken iallfall faller den sammen med den kronologiske.\par
Serien begynner da med \textbf{Kart nr. 135 a og b,} katalogisert som «2 konseptkarter over grensen mellem a) Trondhjems stift og Jemtland, b) Østerdalen og Herdalen, (udat.). Usign.»\par
Hvert av dem er på et utbrettet folioarks størrelse og som notert i katalogen hverken datert eller signert. Men skriften røber, at de er et par i all hast av Schnitler nedrablede kopier efter et eldre foreliggende kart over de norsk-svenske grensetrakter fra Mjøsen, Trysil og Siljan i Dalarne i syd til «Ingermanland» (d. e. Ångermanland), Overhalla og Namdalsfjordene i nord. Det avkopierte kart må være et håndtegnet kart, nu i Kgl. Bibliotek i København, på hvis bakside står skrevet: «Reise carte von Christiania nach Drontheim1724.»\par
Hvem tegneren av dette reisekartet fra 1724 har vært, vites ikke. Men sannsynligvis har det vært en tyskfødt eller tysktalende offiser, av hvem Schnitler har fått det tillåns til kopiering i all hast, innen han i februar 1742 tiltrådte sin første grenseeksaminasjonsreise.\par
Og hvad som kan hevdes med sikkerhet er, at dette noget tyskpåvirkede kartet er en direkte kopi efter størstedelen av det «Græntse Carte imellem Norge og Swerrige», som daværende feltprest, senere stiftsprost og tilslutt biskop i ChristiansandJens Christian Spidberg hadde utarbeidet efter høiere ordre og i 1714 avlevert til kong Frederik IV.\par
Dette Spidbergs kart finnes i Norges geografiske opmåling, katalogisert under A, I, nr. 38. Det gjelder ikke bare grenseområdet i snevrere forstand, men rekker langt både østover og vestover, bygget som det er på Geelkercks og andre senere militære kartografers arbeider fra tiden mellem Hannibalsfeiden og Den store nordiske krig.\par
For Schnitler var det derfor overmåde verdifullt å kjenne til dette kartet under sine grenseeksaminasjoner i Trøndelagen i vårhalvåret 1724. Og innbundet i selve vol. 1 finner vi to skissemessige karter, som er bygget på hans hastverkskopier indirekte efter Spidberg og de av ham selv innhentede oplysninger. Det første av disse to finnes ad pag. 82 og gjelder strekningen fra Idre («Eirebøyd») til fjellene mellem Stjørdalen og Værdalen. Det andre finnes ad pag. 152 og gjelder traktene fra og med Stjørdalen i syd til og med Frostviken og Namdalen i nord. Begge er som nevnt innbundet i eksaminasjonsprot. Vol. 1. Men det vil være fullt berettiget å la gode kopier av dem inngå i kartsamlingen enten under egne numre eller under nr. 135 som litra c og d.\hypertarget{Schn1_4994}{}Innledning.\par
Det kan føies til, at der i Schnitlers «Collectanea» i Grenseregul.-pk. 17 i Riksarkivet i Oslo finnes noen få anonyme og udaterte kladderiss til karter, som synes å være tegnet med Schnitlers hånd. Blandt disse er ett som gjelder Trondhjemsleden fra Akerø og Molde i Romsdalen og Trondhjemsfjorden inn til Stjørdalen og Leksviken. Det er bare et tarvelig hastverksarbeide uten interesse for grensespørsmålene og fortjener ikke å bli gitt plass i den nummererte rekken av Schnitler-karter.\par
b. {Vedrørende den nordligste del av Trøndelagen og hele Nordlands amt}. (Eksaminasjonsprot. Vol. 2‒4.)\par
Blandt Schnitlers ovennevnte «Collectanea» ligger et ark med kladderiss til karter på begge sider. På den ene siden finnes litt av øvre og ytre Namdalen, Bindalen og kysten videre nordover til og med Velfjorden samt Svenningdalen m. m. øverst i Vefsen. På den andre siden finner man riss fra Vistenfjorden og Tjøtta nordover til Nesna. Skriften er Schnitlers. Og disse små kladderissene er forsåvidt av interesse som de gir uttrykk for, hvorledes Schnitler samlet materiale til sine karter. Men det er iallfall tvilsomt, om de bør tas op i og gis nummer i serien av Schnitler-karter.\par
Det første ferdigtegnede og innsendte Schnitler-kart er \textbf{kart nr. 144 a}. Det er meget stort, 158,5 cm. i retning nord-syd, 93,5 cm. øst-vest. Det er udatert og usignert. Men med Schnitlers hånd bærer det følgende titeltekst: «Ohngefærlig Delineation af Grendserne, og derved paa begge Sider beliggende Landskabers Situation med benævnelse af Stæderne som i VidneActens 2det Volum, ere beskrevne: Som dog ikke udgives at være geometrice accurat, men kun at tiene til en Oplysning for de Kongelige Ingenieurer til at forstaae Protocollen des bedre; den røde øvre Stræg bemerker Grendse-gangen, som den er bevidnet, og den nedre min Reise-Route fra Søer i Nord».\par
Det er med andre ord nettop det kartet, som Schnitler sendte inn fra Sakshaug i Inderøy31/10-1742 under tilbakereisen fra Helgeland, se nærværende utgave, Bind I, s. 171.\par
Sammen med kart 144 a er arkivert kart 144 b, som må være et konsept til 144 a.\par
Som kart nr. 144 var kartet til eksaminasjonsprot. vol. 2, så er i geografisk og kronologisk rekkefølge \textbf{kart nr. 123 a} kartbilag til vol. 3 og er katalogisert som «Konsept Kart over Saltens fogderi med anmerket grenselinje mod Sverrige».\par
I likhet med nr. 144 er det nordlig orientert. Størrelsen nord‒syd er 82 cm., øst‒vest 64 cm. Det er ‒ som også nr. 144 ‒ tegnet og tekstet av Schnitlers assistent Helset. Dette gjelder også titlen, skrevet i havet utenfor Folden: «Saltens-Fogderie med sine Grenser ad Sverrige med rød Betejgnelse». Med en rød strek er denne grensen avmerket gjennem alle de navngitte grensemerker fra «Hekke field» i syd til «Torne Vand» i nord. Men «Saltens-Fogderie»s grenser mot «Helgelands Fogderie» i syd og «Senniens-Fogderie» i nord er i det hele tatt ikke avmerket på kartet.\par
Kartet er ikke datert. Men det må utvilsomt være kartbilaget til den medio august 1743 avsluttede grenseeksaminasjonsprotokoll vol. 3. Denne slutter nemlig med følgende, som er utelatt i den trykte utgaven, Bind II, men som der burde ha funnet plass på side 249:\hypertarget{Schn1_5173}{}Innledning.\par
«Herefter følger (1) Alphabetisk Register (2) Bilagerne (3) en Ohngefærlig Delineation over Grendsernes Gang ved derved angrendsende Lande.»\par
Når dette kartet i Riksarkivet er katalogisert som et «konseptkart», tør denne betegnelsen være begrunnet i Schnitlers eget uttrykk om det: «en Ohngefærlig Delineation». Men at det allikevel er det til overordnet instans innsendte, fremgår av følgende noget vanskelig leselige ord med en fremmed hånd under lerredet på baksiden av kartet: «Et Kart af» ‒ eller «Et Udkast af ‒ Schnitler over Saltens Fogderie».\par
Konseptet til nr. 123 a må søkes i det mere omfattende \textbf{kart nr. 123 b,} som ligger innlagt i nr. 123 a uten å være nummerert og katalogisert. Også dette er nordlig orientert, er nesten 100 cm. langt nord‒syd, mens bredden er 64 cm., altså som for nr. 123 a anført. Med sin større lengde i nord‒sydlig retning rekker dette kartet fra og med «Sør‒Raen» i syd til et stykke innpå «Senniøen» i nord. Og litt av Lofoten og Vesterålen er derfor også kommet med.\par
Ingen av fogderinavnene er anført på kartet som sådanne. Og av landskapsnavn finnes foruten «Senniøen», «Westeraalen», «Hinden» og «Lofod» bare «Lula Lapmark» og «Pita Lapmark» på svensk side.\par
Under lerredet på baksiden av kartet finnes påklebet en med en helt fremmed hånd skrevet tekst, som ikke har noe spesielt med dette kartet å bestille (den handler om sneskred).\par
Schnitlers grenseeksaminasjoner i de to nordligste fogderiene i Nordland, Senjen og Tromsø, fant sted i tiden 16/8‒1743‒2/3‒1744 og er protokollert i vol. 4. Som anført i den trykte utgaven, Bind II, side 405, blev denne protokollen innsendt til oberst Mangelsen i Trondhjem med brev av 2/3‒1744. Og foruten de øvrige bilagene blev medsendt «et ohngefærlig Carte over Grendsegangen for Senniens og Tromsøens Fogderier, med Landskabet på begge Sider, til nogen Forstaaelse af denne Protocoll».\par
Nettop dette kartet finnes i Riksarkivet som \textbf{grensekart nr. 136 a.} Det er ikke signert eller datert. Men følgende tekst, skrevet på forsiden med Schnitlers hånd, er tydelig nok: «Delineation af Grændse-Gangen imellem de Norske Fogderier Sennien og Tromsøe, og Sverriges Torne-Lapmark, som begynder fra Tornevattenets Søndre Side i Sønden indtil Paresoive, et Grendsefield af Finmarken, i Øster, med Landets Situation paa begge Sider af Grendsen; ... alt til nogenlunde forstaaelse af Examinations Protocollens 4de Volumen». Og hertil er føiet i katalogen: «Usignert. (Helset ‒ Schnitler) [1743?]».\par
En eiendommelighet ved denne «Delineation af Grændsegangen» er, at den er ført betydelig lenger østover enn til det punkt, hvor grensen mellem Tromsø fogderi og Finnmarken tar av fra riksgrensen, høifjells- eller vannskillsryggen, og går nordover til havet. Den er ført videre i en bue sønnenom Kautokeino helt til Karasjok-elvens kilder nær det i ovenanførte tekst nevnte fjell «Pares-oive», der ligger mellem de nuværende grenserøiser nr. 334 og 335. Dette blir en strekning på over 150 km., som hverken den gang eller nu har noget å gjøre med de områder som utgjorde Senjen og Tromsø fogderier. Men forklaringen til at denne grensestrekningen er tatt med allerede her, er den, at Schnitler på rettsmøter i Kvenangen og Lyngen i november og desember 1743 fikk klare og bestemte vidnesbyrd om denne grensestrekningen fra svenske og fra tidligere svenske, nu til Norge overflyttede «østlapper», som han fikk anledning til å avhøre. Se nærværende utgave, Bind II, s. 381‒84, 387‒89 m. fl. st. Og så verdifulle \hypertarget{Schn1_5345}{}Innledning. som disse vidnesbyrdene var for den norske opfatning, har Schnitler da heller ikke villet nøle med å gi dem uttrykk på kartet.\par
Foruten det rentegnede og med ovenciterte påtegning med Schnitlers hånd innsendte eksemplar, nr. 134 a, finnes yderligere to eksemplarer av dette kartet i Riksarkivet. Om disse kan nu bare sies, at de er katalogisert således: «\textbf{Nr. 136 b.} Konsept til eller kopi av grensekart nr. 136 a [1743?]. Usignert. (Helset ‒ Schnitler)», og «\textbf{Nr. 136 c.} Konsept til eller kopi av grensekart nr. 136 a [1743?]. Usignert. (Helset ‒ Schnitler).»\par
Hvad angår den for alle de tre kartene under nr. 136 benyttede forfatterangivelse «(Helset ‒ Schnitler)», er å bemerke, at den i og for sig er meget riktig. Schnitler er nok ansvarshavende redaktør for alle Schnitler-kartene. Men hans ledsager Helset var hans karttegner. Det er bare på en del av de som konsepter betegnede eksemplarer, at vi i nogen grad finner Schnitlers egen hånd i konturer og stedsnavn. På de rentegnede og innsendte eksemplarer finner vi den bare i titeltekstene.\hspace{1em}\par
Sist i rekken av Schnitlers Nordlandskarter er \textbf{Nr. 120:} «Geographisk Charte over Nordlands amt». Dette gjelder hele det daværende Nordlands amt, omfatter altså begge de nuværende fylker Nordland og Troms, og er meget stort, nord‒syd 151,5 cm., øst‒vest 134 cm.\par
På baksiden er med en fremmed hånd skrevet: «Et Geographisk Carte af Schnitler over Nordlands Amt» og på forsiden med Schnitlers egen hånd: «Geographisk Carte over Nordlands Amt hvorpaa den røde‒gule lange Stræg fra Sør t. Nord til Øster betyder den bevidnede Grændse Gang imellem Norge og Sverrig, de grønne og røde Linier Raaganger eller Skiellet imellem Tronhiems ‒ Nordlands ‒ og Finmarkens Amt samt imellem Nordlands Fogderier i sær».\par
Kartet er altså i hovedsaken et sammendrag i noget forminsket målestokk av de tidligere med eksaminasjonsprotokollene vol. 2, 3 og 4 innsendte karter nr. 144 a, 123 a og 136 a.\par
Men uaktet det efter påskriften på baksiden å dømme skulde synes å måtte være et til overordnet myndighet innsendt kart, bærer det i flere henseender preg av å være et ikke helt rentegnet konsept. De farvede grenselinjer er f. eks. temmelig slurvet anlagt, likeså de med rød blyant avmerkede tegn for kirker og kapeller.\par
Og kartet har da også i første hånd tjent som konsept for et annet, nu noget slitt, men oprindelig meget pent og nitid utført kart over Nordlands amt i det danske Rigsarkiv i København. På små kopieringsunøiaktigheter nær viser dette kartet i København sig å være helt ut en kopi av R.A. nr. 120, tegnet og tekstet som det er med Helsets hånd. Og den ovenciterte titeltekst er på dette eksemplaret ikke bare skrevet med Schnitlers hånd, men også undertegnet av ham: Peter Schnitler mpr.\par
Det prinsipale formål med dette kartet har dog ikke vært å gi de norske grenseforhandlere et samlet kartografisk overblikk over grensespørsmålene i Nordland, men å gi et fyldig bidrag til den i 1743 iverksatte innsamling av topografiske oplysninger om Norge. Dette fremgår av en skrivelse fra Schnitler til oversekretæren i det kgl. danske Kanselli, dat. «Hammerfest paa Hvaløen i Vest-Finmarken den 7de September», hvori han meddeler, at han hadde vært «angelegen at forfatte en Samling af de anbefalede Puncter over Nordland» efter at han våren 1743 gjennem stiftsbefalingsmannen i Trondhjem var gjort bekjent med, at det skulde geråde \hypertarget{Schn1_5515}{}Innledning. kongen til allernådigste behag «at erholde en forklarlig Kundskab om hvert Districts Beskaffenhed, Art og Eegenskab». Og til «des bedre Forstaaelse» av de skrevne oplysninger sendte han også inn «et geographisk Carte, saa godt, og saameget som mig paa denne min Nordlandske Reise under min Forretning har været mueligt».\par
Om kvaliteten av dette sitt Nordlandskart tilføier Schnitler følgende: «Kan jeg ei give det ud for noget fuldkommen; thi at beskrive Nordlands Amt Geometrice og saa omstændelig, som desidereret, dertil vil vel behøves meere end Een Mands Værk og Een Mands Alder. Dog kan jeg derom sige, at jeg har gjort alt, hvad jeg har formaaet, meener og allerunderdanigst, siiden jeg har haft Leilighed, med Fjeld-Lapperne igjennem Tolk at handle, saa vil mit Carte og Beskrivelse derover, særdeles angaaendes Fjeldtracten, blive accuratere, end nogen andens hidendtil har kunnet være.»\par
Denne Schnitlers saklige og nøkterne vurdering av sitt (og Helsets) Nordlandskart var fullt berettiget. Senere kartografer var også enige i dette. C. J. Pontoppidans store, ansette kart over Det nordlige Norge (1795) er, som Pontoppidan selv sier, bygget på Schnitlers. Og den del av Nordland som er kommet med på friherre Hermelins store «Charta öfver Wästerbotten och Svenske Lappmarchen» (1797) er igjen en direkte kopi efter Pontoppidan.\hypertarget{Schn1_5565}{}\footnote{\label{Schn1_5565}I Riksarkivets grensearkiv er i nyere tid også optatt som kart nr. 143(?) et «Kart over fjordene fra Beyern til Kvænangen. [P. Schnitler?] (uttatt fra «Alm. Saml. b. Amtskarter» nr. 33». Dette er dog bare et ytterst tarvelig riss over fjordene i Nordland, har ikke noget med Schnitler å gjøre og bør fjernes fra grensearkivet.}\par
c. {Vedrørende Finnmarken}. (Eksaminasjonsprot. Vol. 5.)\par
Begge fogderiene i Finnmarken er behandlet i eksaminasjonsprotokoll vol. 5, og er i grensearkivets kartsamling representert med ialt 5 karter. Et av dem, \textbf{nr. 122}, er bare en påbegynt kopi av en del av et av de fire kartene over hele Finnmarken, som alle i hovedsaken er helt like, nemlig nr. 121, nr. 134 a, nr. 134 b og nr. 142. De er alle tegnet i samme målestokk, er alle anonyme og udaterte, alle er sydlig orientert og alle har i nedre venstre hjørne nøiaktig samme titel, nemlig «Geographisk Carte over Vaardehuus Amt til Forstaaelse af 5te Examinations Protocoll».\par
Også den til titlen føiede forklaring over, hvad de forskjellig farvede grenselinjer på kartet betegner, er helt den samme på alle de fire kartene.\par
Forskjellen mellem dem består kun i, at de ikke alle har tatt med like meget av det helt russiske område østenfor de norsk-russiske fellesdistrikter, Neiden, Pasvig og Peisen og det norsk-svensk-russiske fellesdistrikt Enare. Kartenes ytre størrelse er derfor noget forskjellig. Og hertil kommer at ett av dem, nr. 121, ved nøiere undersøkelse viser sig å være tegnet og tekstet av en annen hånd enn Helsets og Schnitlers. Og nettop på baksiden av dette står skrevet: «Et Kart af Oberst Lieutenant Schnitler til et geographisk Carte over Finmarkens Amt».\par
Nogen lignende påskrift finnes ikke på noget av de andre heromhandlede Finnmarkskarter. Og da det således på en måte synes å være sluttstenen på Schnitlers arbeide med Finn\hypertarget{Schn1_5642}{}Innledning. markens kartografi, vil det her bli omtalt tilslutt, efter at noen få oplysninger er gitt om de tre andre, som på en måte kan betegnes som forarbeider til nr. 121. Iallfall synes de alle å være tegnet på et tidligere tidspunkt enn nr. 121.\par
\textbf{Nr. 134 a} er det største, 145 cm. nord‒syd, 135 cm. øst‒vest. Det rekker helt til «Hvide Søes Fiord», som vel er ment å være identisk med «Kandalax», som også er nevnt på kartet. Både disse to navnene og endel andre fra de østlige traktene i Kemi lappmark er vistnok hentet fra Isak Olsens bekjente «relation» av 1717, som Helset må antas å ha kjent. Den finnes nu i grensearkivet. Kartet har mange rettelser, delvis med Schnitlers egen hånd. Det er sterkt slitt og må betegnes som et konsept for de følgende.\par
\textbf{Nr. 134 b} er i hovedsaken som foregående nr. 134 a. Men hverken sydover eller østover rekker det fullt så langt som 134 a, østover bare til Kolafjorden og Kildin. Dets størrelse er: nord‒syd 121,5 cm., øst‒vest 135 cm. Det kan se ut som en rentegning. Men enkelte ting, som f. eks. at Tanafjorden og Nordkynhalvøen er blårandet, men ikke kysten forøvrig, tyder på, at 134 b må betegnes som et nytt, forbedret konsept til det følgende, nemlig \textbf{nr. 142}, som i alt vesentlig er maken til de to ovennevnte, spesielt 134 b. Det går dog ikke fullt så langt syd som dette. Dets størrelse nord‒syd er bare 109 cm., men øst‒vest hele 140 cm. Det bærer preg av å være en for overordnede instanser beregnet rentegning. Nogen påtegning på baksiden, som tyder på dette, finnes dog ikke. Men det som allikevel peker på dette, er at kartet tidligere har ligget i grensereguleringsarkivet, pakke 27, hvorfra det først i nyere tid er tatt frem.\par
Som bebudet ovenfor gjenstår det å gi noen nærmere oplysninger om kart \textbf{nr. 121}, der som nevnt er tegnet og tekstet av en fremmed hånd. Dets størrelse øst‒vest er omtrent som for nr. 134 b, 136 cm., men nord‒syd omtrent som for nr. 142, 108,5 cm. Men om tegneren av nr. 121 muligens kan ha sett og i nogen monn gjort bruk av både nr. 134 b og nr. 142, så er det dog en eiendommelighet ved nr. 121, som viser at det kartet som han i hovedsaken har kopiert, det er hverken nr. 134 b eller nr. 142, men nr. 134 a. For mens titlen og forklaringen på nr. 134 b og nr. 142 er skrevet av Schnitler som vanlig på en lang rekke ‒ in casu 31 ‒ like lange linjer, så har han på nr. 134 a skrevet titlen og forklaringen på et mindre antall linjer, men som blir lengre og lengre nedover. Og denne linjerekken er begrenset på høire side av en bølgelinje. Nøiaktig på samme måten er det gjort av den ukjente tegner av nr. 121. Det kan bare bety direkte kopiering.\par
Men hvem er denne ukjente tegner eller kopist?\par
Sannsynligvis er det en anonym offiser i Trondheim.\par
Som meddelt i nærværende utgave, Bind I, s. 391, hadde Schnitler13/9‒1745 under et ophold i Straumfjord i Skjervøy «faaet Bilagene med et ungefærlig Geographisk Carte til dette 5te Volumen af Grænse-Examinations Protocollen ferdig».\par
Da han 16/9 var nådd til Sørvik i Trondenes, lot han «Det 5te Volumen ved Tolken Hælset levere til Secretairen Regts qv. Mester Smith» og hadde den 17de selv en samtale med ham.\par
Med det til Knoff'ene gjennem Smith leverte eksemplar av protokollen vol. 5 har utvilsomt også fulgt et eksemplar av Finnmarks-kartet. Om det i tilfelle er et av dem som nu finnes i grensearkivet og er omtalt her ovenfor, lar sig ikke si med visshet.\hypertarget{Schn1_5758}{}Innledning.\par
Efter sin hjemkomst til Trondhjem29/11‒1745 gikk Schnitler snart igang med fortsatt protokoll- og kartarbeide. I sin journal har han bl. a. notert: «til Janu. 10de [1746] bleven fuldfærdig med 6te og 7de Volumen over Vaardehuus Amt af min Examinations Protocoll og sendt dem til Grændse-Commissionen, Obriste Mangelsen à Købhavn.» Dette meddelte han samtidig til oversekretæren von Holstein, gjennem hvem oversendelsen av protokollene til Mangelsen skjedde.\par
Ifølge Schnitlers kopibok begynte skrivelsen til von Holstein således: «. . . . . . . . . . tilstiller Jeg underdanigst det 6te volumen af vidim. Examinations Protocollen over Finmarkens Grændser ad Rusland, saa og det 7de volumen, indeholdende een særdeles Beskrivelse over Vaardehuus Amt; med et Geografisk Carte derover til Protocollens des bedre Forstaaelse». Det er sannsynlig eller iallfall mulig, at dette til von Holstein oversendte eksemplar av kartet over Vardøhus amt nettop er det ovenfor som kart nr. 121 omtalte.\par
Og det var ikke nok med dette. Arbeidet med karter fortsatte iallfall utover i februar. I sin journal skriver Schnitler: «fra 16de janu. til 16de febr. ladet forarbeide Geographiske Carter over Nordlands Lapmark, saavidt det henhører til Seniens og Tromsøens Fogderier, og sendt dem til de Kongelige Betientere samme steds, for deraf at faae en nøyere og omstændelig Underretning om dets Fieldtract, og derefter bedre at kiende Østlappernes Field Leier i dette Norske Lapmark; og Skatten af dem at kunde annamme».\par
Dette eller rettere disse nye kartene over Senjen og Tromsø fogderier må være utarbeidet på grunnlag av ovennevnte med vol. 4 i mars 1744 innsendte kart over disse fogderiene, altså grensekart nr. 136 a. Men nu har han kunnet supplere og korrigere fremstillingen med de mange nye oplysninger han hadde skaffet sig under reisen sydover fra Finnmarken i sommerhalvåret 1745.\par
Om noget eksemplar av dette nye og forbedrede kart over Senjen og Tromsø fogderier fremdeles eksisterer, er ukjent.\par
\textit{3. Schnitler-kartenes verdi for grenseundersøkelsene}.\par
Man kan vanskelig overvurdere Schnitlers betydning for det store grenseopgjør mellem Norge og Sverige, som resulterte i den territorielt ennu uforandret bestående Strömstadtraktat av 21. september/2. oktober 1751.\par
Dette gjelder selvfølgelig først og fremst selve de inngående, helt saklige og ytterst detaljerte grenseeksaminasjoner med de mange spesielle utredninger og oplysninger han hadde samlet. Men det gjelder også alle de mange for forståelsen av grenseprotokollenes tusener av stedsnavn nødvendige \textit{karter}, som han med et forbløffende godt resultat fikk utarbeidet vesentlig på grunnlag av de utallige mileopgaver og topografiske oplysninger i vidneforklaringene. Vel hadde han en god medhjelper i sin tolk i lappenes eller samenes sprog, den tidligere misjonær i Finnemisjonen Erik Helset. Men om Helset kan tillegges nogen andel i selve konstruksjonen av kartene, er iallfall tvilsomt. Vi vet bare, at kyst- og elve- og innsjøkonturene m. m. som regel er tegnet av Helset og at alle eller de fleste stedsnavn er påført kartene med hans håndskrift.\hypertarget{Schn1_5892}{}Innledning.\par
På første hånd var det de norske grenseingeniører, brødrene Knoff og deres sekretær Smith, som fikk nyde godt av Schnitlers karter. Alle de av Schnitler bevidnede grensemerker går igjen på Knoff'enes. Og at disse, som inntil Lyngen i nord arbeidet i fellesskap med de svenske grenseingeniører som virkelige landmålere, kunde avlegge selve den norske pretensjons-grenselinje med langt større nøiaktighet enn Schnitler, som bare kunde bygge på muntlige vidneforklaringer, er fullt forståelig. Men utenfor selve grenseområdet bygger også Knoff'enes karter praktisk talt helt og holdent på Schnitlers.\par
Og det blev ikke bare Knoff'ene som gjorde det. Ovenfor (s. XLVII) er nevnt, at den danske kartograf C. J. Pontoppidans kart over Det nordlige Norge, utgitt 1795, bygget i hovedsaken på Schnitlers store Nordlands-kart og på hans Finnmarks-kart. Og spor efter Schnitlers karter kan merkes på karter over Nord-Norge til langt ut i det 19de århundre.\par
_______
\DivI[Rettelser]{Rettelser}\label{Schn1_5953}\par
\hypertarget{Schn1_5955}{}\footnote{\label{Schn1_5955}Her sto rettelsene trykt. Disse er lagt inn i teksten.}
\DivI[1. volumen: 1742, på vinterføret.]{1. volumen: 1742, på vinterføret.}\label{Schn1_5964}\par
\centerline{\textbf{WIDNERS EXAMINATION OVER GRENDSERNE IMELLEM NORGE NORDENFIELDS OG SVERRIGE ANNO ‒ 1742 PAA VINTERFØRET 1 VOLUMEN.}}\begin{figure}[htbp]
\noindent\par
_______
\caption{\label{Schn1_5979}}\end{figure}

\DivI[I Gauldal fogderi: 12 vidner.]{I Gauldal fogderi: 12 vidner.}\label{Schn1_5981}\par
\centerline{\textbf{GRÆNDSE COMMISSIONS PROTOKOLL Over De Tagne Viidner Begyndt Paa Røraas Dend 16de Apriilis 1742{ve} Continueret d: 17de ibdm. paa Koyen ved Ferragen dend 18 og 19de ibm, i Bræchen d: 21de Apriil: I GULDALS Fogderie.}}\begin{figure}[htbp]
\noindent\par
_______
\caption{\label{Schn1_6020}}\end{figure}

\DivII[April 7. Schnitler reiser fra Trondheim]{April 7. Schnitler reiser fra Trondheim}\label{Schn1_6022}\par
Efter deris Kongl: Mays{ts} allernaadigste befalning til \textit{Major Peter Schnitler} av 16 \textit{Martj} Sistleeden og der paa Grundede \textit{Jnstrux} fra Hr. \textit{Oberste} og Kongelige grændze \textit{Comissaire Romling} til bemelte \textit{Major Schnitler} af 31: \textit{ejusdem}, hvilcke begge \textit{Documenter} bielegges under \textit{Lit:} a: og B: har \textit{Major Schnitler} begivet sig paa Reisen fra \textit{Trondhiem} til  Grændzerne d: 7 \textit{Apriil} nest efter og efter foregangne \textit{Reqvisition} til vedckommende øfrighed og betiendtere beckyndt paa \textit{Røraas} først at sætte Rætten dend 16 \textit{Apriilis} Nest efter.
\DivII[April 16.-17. Rettsmøte på Røros]{April 16.-17. Rettsmøte på Røros}\label{Schn1_6096}\par
Soerenskriiveren i Guldahlen, Som av Hr. Stiftbefalnings-Mand \textit{von Nissen} var befallet at føre \textit{Protocollen} her, hørte man ej at kunde hidckomme, formedelst af nogle dagers Reignveæir \textit{Dragaas} jisen i Guldahlen Skal være opgaaet, og til Lands ej er at fremckomme.\par
Til biesiddere i denne Rætt Mødte som efter \textit{directeur Borgrevings} foranstaltning vare tilsagde og tilforn havde Siddet Som laugRættis Mænd i Rætten, 2de Mænd, boendes her paa \textit{Røraas} Platz Nafnl: \textit{Joen Andersen} 54 aar gammel og \textit{Elling Svendsen bælgmager} 60 aar gammel.\par
Af Viidner, som skulde være de Kyndigste paa grændserne, Fandtes og fremstillede Sig her for Rætten 1: Ole Larsen \textit{Riise}, 2: Peder Andersen Dahlen og 3: Anders Pedersen Dahl.\par
For dennem blef høyst bem{te} Kongl: \textit{Ordre} lydelig oplæst og dernest viidnerne formanet, at udsiige, hvad som Rætt og Riigtigt var, og dennem beckiendt Om grændze Gangen og Fieldene imellem dette Riige Norge og det Riige Sværig paa denne Kant; hvor paa dem og Eedens Forklaring af \textit{Loven} blef forelæst, og de aflagde deris \textit{Corporlig} Eed ‒ Da og \textit{directeuren} paa Værcket \textit{Leonhardt Borgreving} indfandt sig for Rætten 4: viidne.\par
Spørsmaal: 1: Hvad hans nafn er? \textit{Resp: Ole Larsen Riise}.\par
2: hvor hand er Føed og Af hvad Forældre? \textit{Resp:} hand er Føed her paa \textit{Røraas} Platz og var hands Fader en bergsmand her paa værcket.\hypertarget{Schn1_6202}{}Schnitlers Protokoller I.\par
3: Hvor gammel hand er? om er Gift? hvor hand boer eller tilholder ? hvad hands Næring eller Handteering? \textit{Resp:} Hand er 71 aar gammel; er gift; hand boer her paa Platzen; og er een bergsmand ved Værcket her.\par
4: om hand er beckiendt paa Grænserne her i mellem Guldahls Fogderie i \textit{Trondhiems} Stift og \textit{Herjedalen} Paa dend Anden Svenske Siide, og Viidere i Søer? \textit{Resp:} noget lidt kand hand være beckiendt.\par
5: hvor langt efter Norske Nye Maalte Miile, og i hvad Stræckning denne \textit{Røraas} Platz ligger fra Grændsen, og i hvad Fogderie? \textit{Resp:}\textit{Røraas} Platz ligger fra høyeste \textit{Rutt} Fieldet, Som holdes for et Grændze field, 4{re} Nye Maalte Norske Miile, og er bem{te}\textit{Rutten} fra \textit{Røraas} Platz i Nord oest beliggende i \textit{Guldahls} Fogderie.\par
6: hvad landets beskaffenhed er imellem Røraas Platz og Grænse Fieldet, at forstaa; om der er Skoug, vande, Elfve, Fielde, Myhr el: \textit{Morast}, dyrcket og bebygget, eller øede u-frugtbar Land?\par
\textit{Resp:} Det bestaar mest af Myhrer og \textit{Morast} en hoben vande og Søer, og adskillige Smaa Fielde og Vohler; dog ligger der hidts og her nogle Smaa bønder-gaarder og Platzer, Som have deris Næring af Værckets brug, og her er saa godt, Som ingen Skaug i denne Eign.\par
7: Hvilcke ere de nærmeste Gaarder ved Grændse Fieldene?\par
\textit{Resp:} Synden i fra at reigne, Saa har et par husmænd for nogle Aar Satt Sig need ved \textit{Elgaaen}, som kommer Synden fra det Field \textit{Elgaa-Hogna} øesten for \textit{Femund-Søen}, (et par husmænd), Men om de ere der endnu, veed hand icke; 3 Miile fra denne \textit{Elgaaen} i Nord boer nærmist en bonde paa en liiden Skatte platz ved den Nordere \textit{Femunds} viig paa den øestere Siide, væsten for det Field viggelskaftets Syndere Ende, ved Nafn \textit{Lasse Jensen Femund}, Som holder een hæst og een oxe. ‒ Halfanden Miil i Nord fra denne \textit{Lasse Femund} ligger een Gammel \textit{Hytte}-Platz ved dend Nordre Ende af \textit{Ferragen}-Søe, 1 Miil omtrent i Væster fra høyeste \textit{Wigel}-Field, bestaaende af en 8te opsiddere, Som have Smaa Platzer, og holde Gemeenligen hver een el: Toe øxene at Kiøre om Vinteren med. Herfra \textit{Ferragens} Platz 1 Miil i Nord ere de nærmeste \textit{Bræche} gaardene, liggende i Væster nær under \textit{Rutt}-Fieldet bestaaende omtrent af 13{ten} Smaa Gaards parter, hvilcke holde hver omtrent een hæst el: Kiør-øxene. ‒ Disse alle opReignede Mænd fra \textit{Elgaaen} til \textit{Bræchen inclusive Sorterer} under \textit{Røraas} Kiercke-Sogn. ‒\par
8: Hvilcket er det første Field i Syder, viidnet Kiender, Som ligger østligst el: mest østlig need til \textit{Herjedallen} ad den østre Svenske Siide?\par
\textit{Resp:} Det første i Syder, som hand Kiender mest øestlig at være, er Lang-Fieldet, Som hører \textit{Eire} bøygd til. Hvad Fielde Siden i Nord eller Nordvæst ligge fra dette \textit{Lang}Field, der veed hand ingen Særdehlis Nafn paa, førend Man Kommer en 4: Miile i Nord til \textit{Siebru}-Field og \textit{Brat Riie-Field}. Dog er her af \textit{Brat Rie}-Fieldet det østerste, som dahler ned til \textit{Herjedalens} Dahl, og er ingen Field yderligere i øster der fra uden nogle Smaa \textit{Wohler} (Som er nogle Smaa berg høyder) ved Foeden af \textit{Brat Rie}-Fieldet. ‒\par
9: Hvor langt er dette \textit{Brat Rie-Field} at reigne fra dets Syndere Foeds begyndelse over til dets Nordre Foeds Ende, og i hvad Stræckning det ligger?\hypertarget{Schn1_6432}{}1 Vidne i Guldals Fogderi.\par
\textit{Resp:} Hand Meener det Kand være en 1/2 Miil ofver fra Søer i Nord. ‒\par
10: Hvor breedt det er fra væster i øster at forstaa fra dets væstre Foed til dets østre Foed?\par
\textit{Resp:} Hand meener 1: Miil.\par
11: Hvordanne er landskabet væsten for dette \textit{Brat Rie}-Field?\par
\textit{Resp:} Væster til hænger det sammen med \textit{Sie Brut}-Field; Væsten her for igien ere de Søer \textit{Roassen} og \textit{Mug}-Søen Som begge ligge en 1/2 Miil i øster fra \textit{Femund}-Søen. Fra \textit{Roassens}-Søe Rinder dend Elfv \textit{Røen} Elfv i \textit{Femu[n]den} og Norden derfor af \textit{Mug}-Søen Rinder \textit{Mug}-Elfven i Samme \textit{Femund}. Græsgang er her iche uden nogle Faa Myhr Striiber.\par
12: Hvordan Landskabet er paa dend østre Siide ad \textit{Herjedahlen} af dette \textit{Brat Rie}-Field?\par
\textit{Resp:} Man kommer da need i Gran- og Furru-Skoug og Myhr-Land og boer ingen Folck i øester fra \textit{Brat Rie}-Fieldet førend en 1 1/2 Miil derfra, hvilcke boe paa \textit{Tennes} gaardene, Som er et annex til \textit{Heede} Præstegield.\par
13: Hvilcket Field i Nord følger paa \textit{Brat Rie} Field som ligger østerst til \textit{Herjedalen?}\par
\textit{Resp:}\textit{Raug} Fieldet, østen for hvilcket indtet andet Field ligger, dog er dette \textit{Raug} Field noget i Nordvæst beliggendis, dette \textit{Raug} Field hænger Sammen med \textit{Brat Rie} Fieldet i Sydost, med \textit{Sie Brue} Field i Syder og med \textit{Wigel}-Skaftet i Væster.\par
14: Hvor langt det er fra dette \textit{Raug} Fields Syndere Foed fra Søer i Nord ofver til dets Nordre Foed?\par
\textit{Resp:} Egentlig veed hand det icke, dog Meener hand, det Kand blive 1: Miil ofver fra Søer i Nord.\par
15: Hvor breedt dette \textit{Raug}-Field er fra væster i oster at Reigne fra dets væstre til østre Foed?\par
\textit{Resp:} Hand meener det Kand være en 1/2 Miil.\par
16: Hvordan er Landskabet væsten for \textit{Raug} Field?\par
\textit{Resp:} Landskabet Kand være det Samme som væsten for \textit{Brat Rie} Field; ellers ligger Væsten for \textit{Raug} Fieldet: det Field \textit{Wigel}-Skaftet, i mellem hvilcke bare en liden bæckdahl gaar, og den bæck derj heeder \textit{Mugen}. I Sydvæst fra \textit{Wigel} Skaftet ligger en liiden Field-Stødt, heeder \textit{Mug-Ruven} en 1/2 Miil, og fra \textit{Mug-Ruven} i væster er en 1/2 Miil neer i \textit{Femunds} Nordre viig; græsgang er i denne Egn Saa godt som ingen, og \textit{Raug}-Fieldet i Sig Selfv er Slet og bahrt uden Skoug og Græs, men Væsten for disse Fielde, hvor de Dahler need til \textit{Femunden}, er nogen Furru Skoug, hvor af Grændze-Maalerne Skulde Kunde betiene sig til at giøre Flotter eller Far-Koster af. Dog ere og \textit{Røe}- og \textit{Mugen} Elfve iche Større end at man Kand vade der over om Sommeren.\par
17: Hvordan er Landskabet paa dend østere Side ad \textit{Herjedalen} af dette \textit{Raug}-Field?\par
\textit{Resp:} Land-Skabet paa den østere Siide af \textit{Raug} Fieldet er paa Samme viis, Som om \textit{Brat Rie} Fields østere Siide er meldt: Dog ligger i Nord ost der fra nogle Sæter-Marcker, Som de \textit{Funnæs}-Dahlinger i \textit{Herjedalen} bruge til bue-\textit{Havn} og høe-Slott om Sommeren, og er det \textit{Funnesdals} bøyden Som boe østen for dette \textit{Raug}-Fields høyeste Topp en 1 1/2 Miil. \hypertarget{Schn1_6725}{}Schnitlers Protokoller I. Denne bøyd \textit{Funnesdalen} i \textit{Herjedalen} bestaar af en 14 bønder Gaarder foruden en 12: til 16 platzer, en 1/2 Miil omtrent i Nord derfra, Som beboes af bergs Folck af det Nye \textit{Liusendals} (ϕ) Kaabberværck, denne bøyd \textit{Funnesdalen} er Svagt til Kornland, thj Kornet der gemeenligen affryser, men Græsgang har de nock at holde \textit{Creaturer} af, og ellers have god hielp af Skytterie- og Fiskerie.\par
18: Hvad Field der følger i Nord paa \textit{Raug}-Fieldet, imellem \textit{Guldals} Fogderie og \textit{Herjedalen?}\par
\textit{Resp:} viidere i Nord er hand iche beckiendt at give beskeed om.\par
19: om viidnet veed, hvem dend væstre Siide af disse opreignede Fielde fra \textit{Brat Rie} Fieldet til \textit{Raug} Fieldet\textit{inclusive} tilhører og Ved hvis Gaarders Grund den ere beliggende?\par
\textit{Resp:} Paa dend væstre Siide af disse Fielde have \textit{Herdalingerne} og \textit{Eire}-Mænd haft deris brug ved Skytterie og Fiskerie i Fieldckynnene- eller- Kierner og have de Norske ogsaa gaaet paa denne væster Siide med deris Skytterie og Fiskerie, dog Saa at de Svenske have villet til-Eignet Sig alleene dend væstere Siide af Fieldene, og Kand viidnet icke nægte: at joe de Norske have Skyet for de Svenske: men neer til \textit{Femund}-Søen ere de Svenske icke kommet til at Fiske, ej heller have de Svenske giort den Norske bonde, der Sidder østen for \textit{Femunds} Nordre-Viig: Nafnl: \textit{Lasse Femund}, i dend Stræckning til Fieldene, Som hand af dend Kongl: Norske Foged har bøxlet, nogen indpas.\par
20: om der har været nogen tvistighed imellem de Norske og \textit{Herjedalinger}, om disse Fielde?\par
\textit{Resp:} hand veed icke at der har værit nogen tvistighed om disse Fielde, ej heller veed hand, hvad Rætt eller adckomst de Svenske eller de Norske Dertil Kand have: Dog synes ham: at Siiden Vandene rinder fra Fieldene i væster, at dend væstere Siide af Fieldene Skulde tilkomme de Norske: vel Mindis hand, at for en 50 aar: var \textit{General Wibe} i \textit{Bræche}-gaardene og skulde have trædet Sammen med den Svenske \textit{General} angaaende Grændse gangen imellem \textit{Norge} og \textit{Sværrig}, og laag bem{te}\textit{General Wibe} en heel Maanet der for i bræcken; hvilcken \textit{Commission} Reiste sig der af, at de Norske havde optaget \textit{Dagvolds-grube}: Nord ost fra \textit{Bræchen} og de Svenske havde ladet borttage Malmen: Men de Svenske Kom den gang iche frem til \textit{General Wibe}, og hvorledis den Sag er udfaldet, veed viidnet icke.\par
21: Hvad for Mænd hand Meener, der Kand være Kyndigst og beqvæmmest til at udviise de østerste Fielde, Som ligger Nærmist \textit{Herjedalen}?\par
\textit{Resp:} Nafngive Kand hand ingen: men de nærmeste i Syder ere de Mænd i \textit{Engeldalen} i \textit{Rindalen}; der efter i Nord ere de nærmeste: \textit{Lasse Femunden}, de Mænd paa \textit{Dahls}- Gaarderne i \textit{Røraas} Sogn, de Mænd ved \textit{Færrag}-Søen og omsider \textit{Bræche}-Mændene;\par
22: Hvad Nytte godhed betydelighed og herlighed er der ved dend væstre Siide af Fieldene, Som de Svenske vil tileigne Sig?\par
\textit{Resp:} Det er Skougene, hvor af mange \textit{Familier} maae leve og giøre brug til \textit{Røraas}værck, Saa og Fiskerie i Field vandene, Skytterie i Fieldene og nogen \textit{Bæver} Fangst; men Græsgang er der indtet eller gandske lidet.\par
23: Hvor langt de østligste Fielde fra bøidene og fra Lande vejene er beliggende?\hypertarget{Schn1_6931}{}1-2 Vidne i Guldals Fogderi.\par
\textit{Resp:}\textit{Brat Rie} Field, hvor fra dette viidne har begyndt at forcklare, ligger fra \textit{Femunds} Nordre Ende en 2 1/2 Miil, og her fra viidere til \textit{Røraas} Platz 3 Miile: for deris Nembl: Grænse Maalernes Persohner, Kand de iche faa nærmere undsætning af \textit{proviant} og underholdning end fra \textit{Røraas}: thj indbyggerne af bønder Folck i dend Egn har indtet af Sligt at bortlade. Underholdning for hæsterne Bliver ingen anden Raad for, end dend faar tages i de Dahlene imellem de Syndere Fielde, men naar Grænse Maalerne Komme noget Frem i Nord til \textit{Ferrag}-Søen og \textit{Bræchen}, findes der vel dend fornødne Græsgang, og Som dette viidne Nafnl: \textit{Ole Larsen Riise} iche vidste meere at Siige blef hand dimittered.\hspace{1em}\par
2{det} Viidne: \textit{Peder Andersen Dahlen}.\par
Spørsmaal 1: hvad hands Nafn?\par
\textit{Resp:}Peder Andersen Dahlen.\par
Til det 2: \textit{Resp:} er Føed paa dend gaard Dahlen i Røraas Sogn og Guldahls Fogderie; og var hans Fader bonde paa Samme Gaard.\par
Til det 3{die}: \textit{Resp:} er 59 aar gammel; er gift og har 8{te} levendis børn; hand boer og er bonde paa Forbem{te} Gaard Dahl; Ernærer Sig med Kuld og bergsveeds \textit{leverance} til \textit{Røraas} værck.\par
Til det 4: \textit{Resp:} Hand er beckiendt noget liidet til \textit{SvuKue}-Field, viidere iche beckiendt i Syder;\par
Viidnet tilspørges da hvor Stoert \textit{SvuKue} Fieldet er Fra Syder i Nord og fra øst i væst?\par
\textit{Resp:} hand har vel værit paa den eene væstre Siide af \textit{SvuKue} Fieldet, men iche Faret der over, hvor fore hand iche egentlig Kand Siige, hvor Stoert det er. Men det veed hand at \textit{SvuKue} Field, dets Foed ligger i øster omtrent 1/4 Miil-veigs fra \textit{Femund}-Søen.\par
Til det 5: \textit{Resp:} dend vej fra Røraas til \textit{Rutt}-Fieldet er ham icke beckiendt; men fra \textit{Røraas} forbie \textit{Femunden} til \textit{Raug} Fieldet og der over har hand vel faret; og Skal dette \textit{Raug} Field vel ligge beent i øster fra \textit{Røraas} 5 1/2 Miil.\par
Til det 6{te}: \textit{Resp:} Det Samme Som første viidne.\par
Til det 7: \textit{Resp:} Svarer som første viidne undtagen der i, at hand icke Egentlig veed Tallet af opsidderne ved \textit{Ferragen} Søe, og at der er eendeehl af dem som holder 1: hæst og Kiørøxene:\par
Til det 8: \textit{Resp:} hand har iche befaret andet Field end forbesagde \textit{Raug}-Field.\par
9{de} 10{de} 11{te} 12{te} 13{de} Spørsmaale \textit{Cessere}.\par
14: hvor langt dette \textit{Raug} Field er efter Nye Maalte Miile, fra dets Syndere Foeds begyndelse, over til dets Nordre Foeds Ende, og i hvad Stræckning det ligger?\par
\textit{Resp:} Det Samme som første viidne, og forcklarer der hos, at ingen Field, Som man kan Sige af, ligger østen for dette \textit{Raug} Field men Strax Skoug land begynder.\par
Til det 15{de} 16{de} og 17{de} Svarer hand det Samme, som første viidne, uden det at han iche til visse Kand siige hvor mange Gaarder i \textit{Funnæsdalens} bøyd, og hvor mange Platzer ved det Nye optagne \textit{Liusendals} Kobber værck ere; j det øfrige udsiiger det Samme om \textit{Raug}-Fields leje og \textit{Situation}, som det første viidne ved 13 Spørsmaal forcklaret haver. Viidere om Grændse Fieldene viiste viidnet icke, hvercken i Nord eller Søer.\hypertarget{Schn1_7211}{}Schnitlers Protokoller I.\par
18: Spørsmaal \textit{Cesserer}.\par
19: Om Viidnet veed, hvem den væstre Siide her af Fieldene Nembl: fra \textit{Raug} Fieldet neer til \textit{Femunds} Nordre viig, tilhører, og ved hvis Gaarders Grund den er beliggende?\par
\textit{Resp:} Dend Norske bonde ved \textit{Femunds} Søens Nordre Viig, Nafnl: \textit{Lasse Femund}, Skal have Som Viidnet har hørt, af dend Kongl: Norske Foged i Sin bøxel Sæddel det Field \textit{Mug Ruven} op til \textit{Wigel}-Skaftet i øster til det Svenske Mærcke og det Field \textit{Røvola} i Syder; de Svenske der imod fra \textit{Herjedalen} gaar over \textit{Raug} Fieldet og \textit{Wigel}-Skaftets Syndere Ende, og har hand iche hørt at der har værit nogen tvistighed imellem tidt bem{te}\textit{Lasse Femunden} og de \textit{Herjedalinger}, hvilcke Sidste i bem{te} Fielde have deris brug med Skytterie og Fiskerie.\par
Viidnet tilspørges: om hand veed der er noget grændse-Mærcke Satt imellem \textit{Lasse Femunds} Grund og de \textit{Herjedalingers} Grund, eller paa andre Stæder i Fieldene?\par
\textit{Resp:} Nei Som hand iche veed; ‒\par
Viidnet fremdeelis tilspørges: hvem den Grund østen for \textit{Raug}-Fieldet nærmest er, og ved hvis Gaarder i \textit{Herjedalen} den Stræckning er beliggende?\par
\textit{Resp:} Dend ligger til \textit{Fundnes} Dahls Gaardene, som kand være 1 1/2 Miil fra \textit{Raug} Fields høyeste Topp.\par
20: Spørsmaal \textit{Cesserer}.\par
Til det 21{de} Svarer det Samme som første viidne.\par
Til det 22{de} Svarer det Samme som første viidne har forcklaret.\par
Til det 23{de} har hand til forne Svaret ved det 5{te} Spørsmaal at det er fra \textit{Raug} Fieldet til \textit{Røraas} en 5 1/2 Miil; i det øfrige Kommer hand over ett med Første viidne, udsagt ved det 23{de} Spørsmaal.\hspace{1em}\par
3{die} Viidnet:\par
1: Hvad hans Nafn er?\par
\textit{Resp: Anders Pedersen Dahl}.\par
Til det 2{det} Spørsmaal Svarer: Hand er Føed paa dend Gaard \textit{Dahl}, Som ligger i Syd ost 1 1/2 Miil fra \textit{Røraas} Platz: og leever hans Fader endnu bonde paa Samme Gaard.\par
Til det 3{die} Svarer: Hand er 30 aar Gammel, gift og har 3 børn; han holder til hos hands Fader paa Gaarden, og er i ett Brød med hannem.\par
Til det 4{de}: Svarer hand Kand viide Nafn paa nogle Fielde.\par
Til det 5{te}: Svarer: at hand ei har værit til \textit{Rutt} Fieldet, og der for ej veed \textit{distencen} der imellem j det øfrige Kommer over et med det første viidne.\par
Til det 6{te} Spørsmaal Svarer Som første viidne.\par
Til det 7{de} Svarer som det første viidne: undtagen at hand icke veed, hvor mange opsiddere ved \textit{Ferragen} egentlig ere, og at der er een Mand i \textit{Bræchen} Som holder een hæst og 7: til 8{te} Kiør øxene: de andre have hver een hæst og een Kiør oxe eller toe; ‒\par
Til det 8{de} Svarer: at det Synderste Field, hand kiender, der ligger østerst til \textit{Herjedalen}, er \textit{Grøtt Hongna}.\hypertarget{Schn1_7463}{}2-3 Vidne i Guldals Fogderi.\par
9: Hvor langt er dette \textit{Grøtt Hongna} Field at reigne fra dets Syndere Foeds begyndelse over til dets Nordre Foeds Ende og i hvad Stræckning det ligger?\par
\textit{Resp:} hand har iche faret over dette \textit{Grøtt Hongna} fra Søer i Nord, men et par gange har hand gaaet Nord fra \textit{Grøttoadalen} op til høyeste \textit{Grøtt Hongna}, Som hand fandt at være qvast og Toppet oven til, saa at hand Kunde derfra See need paa Alle Siider, og er dette Field bahrt uden Skoug og Græs. Veien fra denne \textit{Grøttaadalen} til høyeste \textit{Grøtt Hongna} meente hand at være 1/4 Miil.\par
10: hvor breedt dette \textit{Grøtt Hongna} kand være fra øster i væster?\par
\textit{Resp:} Paa dend østere Siide har hand iche faret: men paa dend væstere Siide er der fra høyeste Topp ned 1 1/4 Fierding Miil og hænger det der i Nordvæst Sammen med \textit{SvuKue}-Field.\par
11: hvordan er Landskabet væsten for dette \textit{Grøtt Hongna} Field?\par
\textit{Resp:} Som mældt er, hænger det Sammen i Nordvæst med \textit{SvuKue} Fieldet, i væster har det den Søe \textit{Revling} Søen, hvoraf \textit{Revling} Elfven i væster Rinder need i \textit{Femund} Søen. J Syd Væst har \textit{Grøtt Hongna} det Field \textit{Elgaa Hongna}, dog ligger der imellem \textit{Revling} Søen. Paa dend sydlige Siide er Viidnet icke beckiendt. Paa den Nordlige Siide ligger \textit{Grøttaadalen} en Trang Dahl, hvor i giennem Rinder en bæck Kaldes \textit{Grøttaaen} som Nord væst løeber ind i \textit{Roassen} Søe, og der af Siiden med \textit{Røen} Elfv væster udj \textit{Femund} Søen.\par
Til det 12{te} Svarer at hand iche veed af Landskabets beskaffenhed paa den østere Siide af \textit{Grøtt-Hongna}.\par
13{de}: Hvilcket Field i Nordost følger paa \textit{Grøtt-Hongna}, Som ligger østerst til \textit{Herjedalen?}\par
\textit{Resp: Wonvola} Som og ellers Kaldes \textit{Kraltvola}.\par
14{de}: Hvor langt det er fra dette \textit{Wonvola} eller \textit{Kraltvolas} Syndere Foed fra Søer i nord over til dets Nordre Foed?\par
\textit{Resp:} hand meener det Kand være en 1/4ding Miils vejs ofver.\par
15: hvor breedt det er over fra væster i øster?\par
\textit{Resp:} hand meener det er ligeleedis 1/4 Miil over.\par
16: hvordan er Landskabet væsten for \textit{Wonvola} eller \textit{Kraltvol} Fieldet?\par
\textit{Resp:} væster og Synder for er \textit{Grottaaedalen} hvor \textit{Grøttaaen} Rinder igiennem, og her Kand være beete for nogle hæste i nogle dage. Norden der for er Furru Skoug, og her fra igien i Nord og øst ligger \textit{Rogen}-Søe.\par
17: Hvordan er Landskabet paa dend østere Siide ad \textit{Herjedalen} af dette \textit{Wonvola} Field?\par
\textit{Resp:} viidnet er der icke beckiendt, ‒ vel har hand i Nordost Fra \textit{Wonvola} Seed nogle Fieldstødter Synden for \textit{Rogen} Søe: men hvad dem heeder, veed hand iche.\par
18: hvad Field der følger i Nord paa \textit{Wonvola} Fieldet imellem \textit{Guldals} Fogderie og \textit{Herjedalen}?\par
\textit{Resp.} Norden for \textit{Wonvola} ligger først en Furru Skoug 1: Miil lang eller der over i een Dahl, hvor \textit{Røa} Elfv falder af \textit{Rogen} Søe og Rinder væstlig i \textit{Roassen} hvilcken Søe \textit{Roassen} Stræcker sig i Nord væst, og der af Rinder dend forbem{te}\textit{Røen} Elfv Væstlig i \hypertarget{Schn1_7753}{}Schnitlers Protokoller I.\textit{Femund} Søen, Siiden Norden for denne Furru Skoug ligger \textit{Siebrud} Field, ‒ dette \textit{Siebrud} Field hvor langt det er fra Søer i Nord eller fra væster i øster, veed hand icke, Saa som hand har Kun faret over dends væstre Foed: der hvor Skougen Endis i Nord; ellers ligger Synden for dette Field Som mælt: \textit{Røa}-Elfv og \textit{Rogen} Søe; denne \textit{Røa} Elfv fra det Stæd hun kommer af \textit{Rogen} Søe til det hun fallder i \textit{Roassen} Søe, er en 1/2 Miil. Hvor lang og breed denne \textit{Rogen} Søe er, Kand han iche viide; men Som hand har Seet ham, er den Stoer og lang; og Syntes den at Stræcke Sig i øster.\par
Den væstere Siide af \textit{Siebrud} Field bestaar af 2{de} Dahle, Som Stræcker sig omtrent 1 1/2 Miil i væster need ad \textit{Femund} Søen; dend ene Syndere dahl er \textit{Røe}-dahlen, hvor, af \textit{Roassen}, \textit{Røen} Elfv Rinder; dend Nordre er \textit{Mugdalen}: hvor i \textit{Mug} Søen og \textit{Mug} Elfven gaar i væster ind i \textit{Femunds} Nordre viig, paa denne \textit{Mug} Elfvens Syndere Siide, der hvor den falder ind i \textit{Femunden}, boer den Norske bonde \textit{Lasse Jensen Femund}; j disse 2{de} Dahle er nogen liden Græsgang for Hæste. Imellem de Toe Sidst bem{te} Elfve bestaar Landskabet af nogle Aase med nogen Skoug paa Kaldes \textit{Rø-aasene}. Distancen mellem \textit{Røen} og \textit{Muggen} Elfv er ved \textit{Femund} Søen Saa vel som og imellem begge Søene Nemblig \textit{Mug} Søen og \textit{Roassen} 1: Miil; hvordan \textit{Situation} er paa den Nordre og østere Siide af \textit{Siebrud} Field, veed hand iche.\par
19: Om viidnet veed, hvem den Væstre Siide af disse opreignede Fielde fra \textit{Grøtt Hongna} til \textit{Siebrud} Fieldet\textit{inclusive} tilhører og ved hvis Gaarders Grund den ligger?\par
\textit{Resp:} Til dette svarer viidnet det Samme som det første viidne, med forcklaring der hos: at hand med mange fleere fra Dalsbøyden og \textit{Røraas} har hugget af Skougene paa dend væstere Siide af Fieldene østen for \textit{Femunden} til \textit{Røraas} værckets brug for 10 à 12 aar Siiden og det uden nogen insiigelse eller paa anche fra den Svenske Siide. Men til hvilcke gaarders grund denne væstre Field Siide langs østen for \textit{Femund} Søen ligger, veed hand iche, dog har hand hørt, at den Norske bonde \textit{Lasse Femund}: og hans formænd paa gaarden have af dend Norske Foged bøxlet den væstre Field Stræckning fra \textit{Wigel} Skaftet i \textit{Mug Ruven} til \textit{Røvola} Som er Foeden af \textit{Svu Kue} Field; ‒\par
Til det 20{de} Svarer, at hand icke har hørt eller veed der af.\par
Viidnet tilspørges om hand veed: at der er nogen vahre eller grænse-Mærcke i disse Field, imellem Sverrig og Norrig?\par
\textit{Resp:} hand veed icke af nogen Grænse Mærcker i Fieldene i mellem Riigerne; hand har vel Seet i \textit{Svu Kue} Field paa dends væstre Kant imellem dets høyeste Spits og Field Foden een opReyst Træ-Stolpe, Som Saae ud at være dreiet med 9 indhollede dreijnger paa en 3 1/2 allen høy, fæsted paa Fieldet med omlagte Steene; gammel og med Maase begroed, Staaendes østen for \textit{Femund} Søen en 1 1/2 Fierding Miil omtrent; hvilcken Stolpe vidnet har Seet for 8{te} aar Siden at være needfalden, dog Siiden igien at være bleven, og nu at være opReyst: mens hvad denne Stolpe eller Vahre skal betyde, det veed vidnet icke, Som dend iche heller veed, af hvem? at være opReyst. Vel er det Saa at den Staar omtrent i liige \textit{linie} og Stræckning med de grænse Mærcker, som Skiller Synden fields fra Norden fields i Norge: men om denne Stolpe til den Ende er hensatt paa \textit{Svu Ku} Fieldet, det veed viidnet icke. Siiden, fornemmes at viidnets gaard \textit{Dahlen} ligger nærmest de Syn\hypertarget{Schn1_8013}{}3-4 Vidne i Guldals Fogderi. denfields grænser: Saa til Spørges ham, hvor og hvilcke de Skille Mærcker ere i mellem den Syndenfieldske og Nordenfieldske \textit{district} i Norge?\par
\textit{Resp:} i \textit{Femunds} Kløven Staaer een Steen-vahre med aabne glugger eller huller igiennem: for det 2{det} 1 Miil derfra findis Steen Fieldet for de 3{die} en Miil her fra \textit{Ravn} berget; og 4. atter her fra 1 1/2 Miil i Nord væst \textit{Haa}-Steenen, Som hand har hørt Skal være en høy Steen midt i Elfven \textit{Glommen}.\par
Til det 21: Svarer det Samme som første viidne.\par
Til det 22{de} Svarer Samme Som første viidne.\par
Til det 23{de} hvor fra hand har begyndt. Nembl: fra det Field \textit{Grøtt Hongna}, hvor langt det er til \textit{Femunds} Nordere Viig, og der fra Siiden til \textit{Røraas?} Svarer: fra \textit{Grøtt Hongna} til \textit{Femunds} Nordre viig, er 3 miile efter hands formeening, og der fra til \textit{Røraas} Platz en 3 1/2 Miil. I det øfrige kommer over ett med dett første viidne. Og Som Viidnet icke viidste mer: blefv hand bortforlovet.\hspace{1em}\par
Siiden det var nu Saa Silde paa aftenen blef Rætten opsatt: til i morgen tiilig. ‒ dend 17{de} Apriilis.\hspace{1em}\par
Dend 17{de}\textit{Apriilis} 1742{ve} blef \textit{examinationen} her paa \textit{Røraas} igien foretaget, overværende Samme Laug Rættesmænd.\par
Der fremstellede Sig da til Viidner \textit{Lasse Jenssen Femund}, boendes ved \textit{Femundens} Nordre Viig; og nu denne \textit{Friederich Jensen Ferragen} ‒ for hvilcke Eedens forcklaring af lov-bogen lydelig blef forelæst, at Sige deris Sandhed: om h[v]is de angaaende dend Rætte grænsens gang paa denne Kant ere viidende. Og de aflagde deris \textit{Eed}: ‒\hspace{1em}\par
4{de} Viidne.\par
Til 1{te} Spørsmaal:\par
\textit{Resp:} hand heeder: \textit{Lasse Jensen Femund}.\par
Til det 2{det}: Hvor hand er Føed, og af hvad Forældre?\par
\textit{Resp:} Føed i Nordfiord\textit{Bergenhuss} Stift paa gaarden \textit{Støfreede}; hands Fader een bonde der.\par
Til det 3{die}\textit{Resp:} hand er 47{ve} aar gammel; er gift og har 5: børn; boer ved \textit{Femunds} Nordre Viig i \textit{Røraas} Sogn; Giør Arbeide og brug til \textit{Røraas} værck.\par
Til det 4{de}\textit{Resp:} Noget Liidet.\par
Til det 5{te} Svarer lige det Samme, som det Første viidne.\par
Til det 6{te} Svarer som Første viidne.\par
Til det 7{de} Svarer Som de forrige: dog veed hand, at der var een igien for 8{te} dage Siiden ved Nafn \textit{Jonas Svensche}. Som er husmand ved \textit{Elgaaen}.\par
Til det 8{de}\textit{Resp:} Det Sydligste Field, Som viidnet har værit paa, er \textit{Elgaa Hongna:} hand har vel her fra Kundet See Flere Fielde i øster, men ej værit der, og veed iche deris Nafne.\hypertarget{Schn1_8272}{}Schnitlers Protokoller I.\par
9: hvor langt er dette \textit{Elgaa Hongna} Field at Reigne fra dets Syndere Fods begyndelse, over til dets Nordre Foeds Ende og i hvad Stræckning det ligger?\par
\textit{Resp:} hand Siunis det Kand være 1/4: miil vejs ofver, og vel det, og er det Field J Sin Stræckning længst fra Søer i Nord væst.\par
10: hvor breedt det er fra Væster i øster, at forstaa fra dets væstre Foed til dets østre Foed?\par
\textit{Resp:} hand Meener at det Kand være een god 1/4 Fierdingveis øfver. Ellers er dette Field bart uden Skoug.\par
11: hvordan er Landskabet væsten for dette \textit{Elgaa-Hongna?}\par
\textit{Resp:} der Ligger nogle Field Rabber med nogen Tøerr Skoug paa; Fra Foden af dette \textit{Elgaa Hongna}, need til \textit{Femund} Søen Kand reignes 1/4 Fierding Miil, og er der ingen græsgang paa den Siide.\par
12: hvordan Landskabet er paa dend østere Siide ad \textit{Herjedalen} af dette \textit{Elgaa Hongna} Field?\par
\textit{Resp:} i Syd ost er een dahl omtrent, Som hand Synis, een 1/2 Fierding lang til Foeden af \textit{Sal} Fieldet hvilcket \textit{Salfield} Skiær Sig i Syd ost imod \textit{Eire}-bøyd, og Skal ligge der fra meer end 2 miile; j Sig Selv er dette Field, Som ham Synis, fra Væster i Sydoest Streckendes en Miil lang omtrent, men tverts over Synes det Kun kort. Fra \textit{Elgaa Hongna} i Nord ost er een Dahl grubbe, hvor i ligger \textit{Revling} Søen, Som er kun liiden, i Nord ost; fra denne \textit{Revling} Søen ligger en liden berg-Stødt Kaldes Lille \textit{SvuKue}, som hænger sammen med \textit{SvuKue} Field, og gaar did i nordvæst. J Nord ost fra dette til \textit{SvuKue} ligger een Søe, Kaldes \textit{Rund Søen} som er Smal over, men efter Sin længde gaar Kroget fra Nord i Søer og Siiden i Sydost og Synes at være 1/4 Miil lang. ‒ Udaf denne Søe udckommer en liden Elfv, som gaar i Syd ost Kroget ind i \textit{Gruvel} Søen og er 1/4 miil lang ‒ i Nord ost fra denne \textit{Rund} Søe ligger i Norst øst et Field som er ganske bratt og Steil paa dend Siide som er imod \textit{Run Søen}; dette \textit{Grøtt Hongna} Field, Som hand Synes, Skal være Fra Væster i Syd oest en 1/2 Miil lang men hvor breedt det er i Nord ost over, Kand hand iche Siige; Paa dend Nord ostlige Siide af dette Field, som hand forstaar maa være \textit{Grøtt Hongna}, ligger en dahl, heeder \textit{Grøttaadalen}, hvilcken \textit{Grøttaadal} paa dend østlige Siide er ganske Smal, men Nord efter udbreeder Sig; i denne \textit{Grøttaadal} Rinder en bæck, heder \textit{Grøttaaen} Som hand Meener, maa Komme fra \textit{Grøttvalds} Søen, hvilcken Søe ligger i Syd ost ved et lidet Field \textit{Grøttvals} Stødten, og denne \textit{Grøttvals} Stødten er i Syd ost fra \textit{Grøtt Hongnens} østlige Ende beliggende. Oven bem{te} bæck \textit{Grøttaaen} Rinder Kroget dog Nordlig først i \textit{Røa} Elfv, og med den følger i Søen \textit{Roassen}, og Kand være Lang fra hendis udspring til \textit{Røa} Elfv en 1 1/2 Miil, viidere i øster har hand vel Seet der ligger Fielde, men hvad Nafn de have, veed hand icke. Som viidnet i begyndelsen af Sit Svar til dette Spørsmaal har Næfnt \textit{Sal}-Fieldet; Saa forcklarer hand viidere, at paa dend Nord østlige Siide af dette Field ligger \textit{Gruvel Søen}, som Stræcker sig fra væster i øster, og Synes ham at være en 1: miil Veis Lang; men derhos Gandske Smal, denne Søe Efter vejen Som gemeenligen Fahres, ligger fra \textit{Eire} bøyd vel 3: maalte miile, og have \textit{Eire}-mænd deri megen Skiøn Fiske fangst. Af denne \textit{Gruvel} Søe udkommer een Elfv, heeder \textit{Gruvla}, Som Rinder Syd ost og kommer \hypertarget{Schn1_8484}{}4 Vidne i Guldals Fogderi. Sammen med dend anden Elfv \textit{Fosscha}, der har Sit udspring Nord ost fremmenfor \textit{Lang}- Fieldet, og løber med den til \textit{Eire} bøyd og Siiden viidere.\par
J Nord ost fra \textit{Gruvel} Søen ligger det Field \textit{Lang-Fieldet}, hvis Væstere Ende Kaldes \textit{Brotene}, og hænger vel sammen med andre Smaa Fielde ved mellemliggende Tange i Nord Væst, men hvad de Smaa Fielde have for Nafne, veed hand icke: Dog meener hand at væster Enden af \textit{Brotene} eller \textit{Lang} Fieldet hænger Sammen ved een Tang med forbem{te}\textit{Grøttvals} Stødten i Væster. Dette \textit{Lang} Field Skal være fra Søer i Nord ost, hvor det er bredest, 1: Miil ofver. Ellers er dette \textit{Lang} Field bart og uden Skoug, og bruges af \textit{Eire}mænd til Skytterie af \textit{Reen}-Dyer. Nord ost for dette \textit{Lang} Field ligger Søer, hvor \textit{Eire} mænd alleene Fiske j ‒ og Kaldes Nord-Field vandene. Viidere i øster er denne Mand iche beckiendt. Dette viidne har for nogle aar værit hos en \textit{Hollands} Falche-Fanger paa forbem{te}\textit{Lang} Field, og taget fra Sit hiem i \textit{Femunds} Nordre viig Følgende vej om Sommeren, Nembl: Fra Nordre viig til \textit{Røen} Elfv 1: Miil, fra \textit{Røen} til Foeden eller Tangen af \textit{SvuKue} Field 1: Miil, her fra igiennem \textit{Revling}-dahlen over til \textit{SvuKue} Fieldet til \textit{Sal}- Fieldet 1: miil, fra \textit{Sal}-Fieldets Nordre Ende til \textit{Gruvels} Søens østere Ende 1 miil omtrent efter hands giissning, og der fra til mitten af \textit{Lang} Fieldet 1 Miil.\par
Landskabet paa dend Syndere Siide af \textit{Elgaae Hongna} er \textit{Elgaadalen} Som er en breed Dahl, igiennem den: \textit{Elgaaen} Rinder need i \textit{Femund} Søen, og kommer den fra \textit{Elgaa} Søen: men hvor denne Søe ligger har hand iche Værit; Een 1/4 miil Veis i Nord fra denne \textit{Elgaaen} har for nogle aar Siiden Satt Sig need een husmand ved Nafn \textit{Jonas}.\par
13: hvilcket Field i Nord følger paa \textit{Elgaa Hongna?}\par
\textit{Resp}: det Field \textit{SvuKue}, og ligger imellem disse Toe Fielde een Dahl, heeder \textit{Revlingdalen}, Som fra dend ene Fields Fod til dend anden Kand være en1/2 Miil breed, og er der Græsgang om Sommeren for nogle hæste en liden tiid; igiennem denne dahl til \textit{Femund} Søen Rinder en Elfv heeder \textit{Revling}, Som kommer af \textit{Revling} Søen Nord ost fra \textit{Elgaahongna;} denne \textit{Revling} Søen er kun liden, dog meere lang end breed, og er dend i Mitten ganske trang og liigesom et Sund imellem, og Som hand Synes, ligger den i Sydvæst fra \textit{LillSvuKue} Fieldet. \textit{Revlings} Elfven Synes Ham, fra hendis udsprang af \textit{Revling} Søen need til \textit{Femund} Søen i væster, hvorudj dend Løber, at være en1/2 Miil lang. Fra \textit{SvuKue} Field i væster følger Smaa vohler og Berg hyller, der efter en liden Smal Furru Skoug need til Søen; og Kand fra høyeste \textit{SvuKue} Field til \textit{Femund} Søen være omtrent 1/2 Miil, Landskabet østen for \textit{SvuKue} Field er en dahl bestaaend af Myhr og \textit{Morast}, Nafnlig \textit{Grøttaadalen}, Som hand Meener, Skal være imod en1/4 Fierdings vej breed til næste østlige Field. J denne \textit{Grøttaadal} Rinder den \textit{Grøttaaen} Nordlig i \textit{Røa} Elfv, Som forhen er melt. Østen for denne \textit{Grøttaadal} har hand og Seet Fielde, Som Synis høye men icke viide, dog veed hand iche at Nafngive dennem. Norden for \textit{SvuKue} Field ligger et mindere Field, heeder \textit{Røvola}: Som hænger ved en liiden Sløgd eller Field dahl, Sammen med \textit{SvuKue} Field. J: denne Sløgd eller Field dahl ligger et par: Smaa kynner eller Kierner, hvor af en liiden bæck ved Nafn \textit{Røvol}-Bæcken udrinder need i \textit{Femund} Søen: dog ligger dette \textit{Røvola} noget i Nord-væst fra \textit{SvuKue} Fieldet. Dend Væstre Siide af dette \textit{Røvola} Field bestaar Af nogle Smaa Field \textit{Hamre} og een liiden Smal Furru Skoug ved bredden af \textit{Femund} Søen. \hypertarget{Schn1_8809}{}Schnitlers Protokoller I. Dend østere Siide af \textit{Røvola} bestaar af Myhr land og Smaa Kierner. Nord ost fra \textit{Røvola} ligger dend Søe \textit{Roassen} Som Stræcker Sig efter Sin længde fra Nord væst i Syd ost efter hands giissning en god 1/2 miil lang. J denne Søe Fanges dend Fisk øret, Røe, gietter, \textit{Abbor}. J denne \textit{Roassen}-Søe Rinder de Tvende Elfve \textit{Røa} Elfv Østlig fra, og \textit{Grøttaaen} fra Sydost, hvor om før er meldet: \textit{Røa}-Elfv kommer fra øster udaf \textit{Rogen}-Søe, og Skal fra dends udspring udaf \textit{Rogen}-Søe, til dends jndløb i \textit{Roassen} være en1/2 miil lang. Hvor Stor \textit{Rogen}-Søe er, veed hand icke dog er det en Fiske-riig Søe, hvor \textit{Herjedalingerne} fra \textit{Funnes} dahl og \textit{Tennes} bøyden Plejer at Fiske. Lige som og Samme Folck Fiske i for bem{te}\textit{Roassen} Søe. \textit{Rogen} Søe liger imellem Fieldene, mens hand veed iche Nafn paa de Fielde, ej heller hvor langt det Kand være fra \textit{Rogen} Søes østere Ende neer til dahlen i \textit{Herjedalen}.\par
Landeds beskaffenhed Østen for \textit{Roassens} Søe er een Dahl af Myhr land og Smaa Kynner, efter hands Mening 1: Miil til Væstere Fod af \textit{Siebrue} Field. J Nord Nord ost ligger \textit{Siebrue} Field; hvor langt Og hvor breedt det er, kand hand iche Siige dog Synes det at være et høyt Stoert Field. Østen for dette \textit{Siebrue} Field har hand vel Seed en Rund høyde af Field, men veed iche hvad dett heeder.\par
For bem{te} Field \textit{Røvola} er saa godt som Rund, og kand være en1/2 Fierding lang paa begge Siider.\par
Paa dend Nordre Siide af \textit{Røvola} ligger een Dahl omtrent en1/4 Miil vejs breed beStaaende af Smaa houger og liiden Furru Skoug; igiennem hvilcken Dahl rinder \textit{Røen}-Elfv, en1/2 Miil i Væster udaf \textit{Roassen}-Søe need i \textit{Femund}Søen. J Nord fra denne Dahl følger nogle Smaa Field-aasener, Kaldes \textit{Rø-aasene}, bestaaende af nogen Furru Skoug og Smaa Kierner; Disse \textit{Røaaser} ere 1: Miil lang omtrent fra Syder i Nord. Men fra Væster i øster at Reigne fra \textit{Femund} Søes breedde til \textit{Siebru} Fields væstere Foed kand der være omtrent 1 1/4 miil vejs.\par
Disse \textit{Røaase} Stræcker Sig i væster need til \textit{Femund} Søen.\par
J øster hænge de Sammen ved Smaa houge med \textit{Siebrue} Field, og kand der imellem være noget lidet Myhr Græss. ‒ J Nord øst fra \textit{Røaasene} ligger \textit{Mug} Søen Som Stræcker Sig efter sin længde fra Nord Væst i Syd-ost omtrent en1/4 miil veis, og Kand være breed i mod en 1/2 Fierding. J denne Søe Fiske og de Norske med \textit{Funnes} Dahlingerne af \textit{Herjedalen}, og Fange de der j Siig, harr, øret, og abbor; Af denne \textit{Mug} Søe, Rinder en Elfv, Kaldes \textit{Mugen}, en 3/4: miil væis i væster need i \textit{Femunds} Nordre Viig. ‒ Landskabet Norden for \textit{Røaasene} er Myhr land og nogle Tørre Steen houger med Furru Skoug iblandt, og her er der at Viidnet boer paa den Syndere Siide af \textit{Mugen}, hvor den falder ud j \textit{Femunden}. Hands gaard Nordre viig hører Værckets \textit{participantere} til og er i \textit{
     \par \bgroup\itshape 5800010matriculn \egroup\vskip6pt\par
   } lagt for 2 marcklaug; hand holder een hæst og een Kiør oxe hvor med hand giør Sit brug.\par
Siiden denne Mand boer Saa nær \textit{Femunden}; Spørges hand hvor lang \textit{Femund}-Søen er Fra Søer i Nord og hvad \textit{Situation} dend har? \textit{Resp}: hand har aldriig faret dend heel igiennem; men hørt Siige, at den Skal være 9: Miile lang; breed er hun \textit{ordinair} fra væster i øster en 1/2 miil omtrent: men hvor \textit{Soler-øen} ligger i \textit{Femunden} der Skal hun være breedere over. J: \textit{Femundens} Nordre Ende gaar en jord Klimp eller: jord Klyft ud næsten \hypertarget{Schn1_9077}{}4 Vidne i Guldals Fogderi. midt i Søen et Støcke, og giør Toe Viiger, hvor af den østere Kaldes Nordre Viigen, og dend væstre Synderviigen, dog er Synderviigen noget breedere over end Nordrviigen thj: Norder viigen er 1/12{de} deehl Miil og Synderviigen 1/8: dehl Miil omtrændt breed, og Kaldes dend østereviig Nordenviigen, fordi den gaar længgere i Nord, end Synderviigen; \textit{Femund}- Søen i Sig Selfv Stræcker sig beent fra Nord til Søer indtil den \textit{øe- Soller-øen;} men fra \textit{øen} gaar den fra Nordvæst i Sydoest. Viidere Viste hand icke at give beskeeden om denne Søe saasom hand iche har igiennem faret den. Fisk Fanges der i; Siig, harr, øret, Abbor, Giedde, Røe og Lache.\par
Norden for dend Elfv \textit{Mugen}: ligger det Field \textit{Mug-Ruven}, som er lidet omtrent en 1/2 Fierding over, Rundagtig, dog gaar een Tang der af i Syd oest.\par
Væsten for denne \textit{Mug Ruven} ligger een Kyn eller Kiern dahl med Myhr land, og nogen Furru Skoug.\par
Østen for \textit{Mug Ruven} ligger een dahl med Smaa Kynner udj, og i Søed ost ligger den forhen omskrevne \textit{Mug} Søe; dahlen kand Være i øster breed, til Det Field \textit{Wigel} Skaftets væstre Foed 1: Miil: og er mest u-frugtbar: dog kand der findis noget hæst beede i Myhr-landet Sammestæds.\par
Dette Field \textit{Wigel}-Skaftet, Stræcker Sig efter længden fra Syd ost i Nord væst en 1/2 Miil, og er gandske bart; breed er det over Skaftet fra Foed til Foed en 1/2 Miil, paa denn Syndere Siide af \textit{Wigel} Skaftet er en Dahl, hvor i man Seer adskillige Smaa Vande, hvori \textit{Funnæs} Dahlingerne fiske og Skal være i Søer til \textit{Siebrue} Fieldet: omtrent en 1/4 Fierdings vej.\par
Østen for dette \textit{Wigelschaft} ligger det Field \textit{Raug} Fieldet: om hvilcket hand udsiiger det Samme som første viidne fra det 13{de} til det 17{de} Spørsmaal \textit{inclusive}.\par
Landskabet paa dend Nordre Siide af \textit{Mug-Ruven} er Smaa Field Rabber, vandkynner, iblandt hvilcke een er, der Fiskes i og heder \textit{Halvors-Kynnen} liggendis en 1/4 Fierding Miil veis fra \textit{Mug Ruven}.\par
Norden for \textit{Wigel} Skaftet lidet i væster, ligger det høye lange Field \textit{Wigeln}, som hænger med \textit{Wigel} Skaftet Samme i Syd oest og er der med Eett Field, dog høyere og Større end \textit{Wigel}skaftet, Saa at \textit{Wigel} Skaftet Synes Kun at være een Tang eller Foed udgaaende af \textit{Wigeln} i Syd ost; hvor langt dette \textit{Wigeln} er og hvordanne den Stræckning gaar, veed hand iche, som hand iche har værit der.\par
Til det 20{de}: Spørsmaal S[v]arer: Hand har ingen Stridighed haft med \textit{Fundnesdalingerne}, om hands bøxlede Gaards Stræckning; vel have endeehl af dem Kiæret derpaa, at hand bruger nogen liiden høe-Slott væsten for \textit{Mug}-Søen, og derfor begiert af ham nogeslags Villighed; Mens hand har Stoelet Paa Sin bøxel Sæddel, hand har Fra dend Kongl: Norske Foged \textit{Hans Jbsen} og leveerede her i Rætten, og icke givet de \textit{Herjedalinger} noget.\par
Viidnet Tilspørges om hand veed at Siige af nogle gamle bøyde eller Sogne Mærcker imellem \textit{Herjedalingerne} og vore?\par
\textit{Resp}: Jmellem de Fielde \textit{Røvola} og \textit{SvuKue} Field er een Sløgt eller liiden Fielddahl; J denne Fielddahl Nord for \textit{SvuKue} Field dog mod dends væstre Kant, har hand for 9 a 10: aar Siiden Seed een Gammel otteckandtet Træ-Stolpe omtrent 3 allen høy, hvor \hypertarget{Schn1_9295}{}Schnitlers Protokoller I. i hand Synis at kunne have \textit{observeret} Een 4 eller 5: Ringe at være omckring Skaarene, Ligesom man Seer een Foed paa een træ-Stoel omdreiet, om der paa var Skrevet noget har hand iche Kundet Seed, Saasom dend var gandske overVoxen med Maase, ej heller Kunde hand See, om der have værit nogen Knapp eller hoved paa Stolpen, Siiger der hos at Samme Stolpe Staar ved Vejen som man farer fra \textit{Røraas} til \textit{Eire} bøyd i Sværrig; Nogle gange har hand Seed dend at være needfalden: og nogle gange igien at være opreyst; Men af hvem denne Stolpe er først opsadt og paa hvad Tiid jtem: til hvad Ende og i hvad betydning den er opReyst, Veed hand iche. Fleere vahrer eller mærcker veed hand iche af at Sige.\label{Schn1_9313} \par 
\begin{longtable}{P{0.1176923076923077\textwidth}P{0.07846153846153846\textwidth}P{0.6538461538461539\textwidth}}
 \hline\endfoot\hline\endlastfoot Til det\tabcellsep 21:\tabcellsep Spørsmaal Svarer hand det Samme Som Første Viidne.\\
og\tabcellsep 22:\end{longtable} \par
 \par
Til det 23{de} Svarer: Fra det Field \textit{Elgaahongna}, hvor hand har begyndt sin forcklaring, er til Norderviigen hvor hand boer, 3: Miile og der fra til \textit{Røraas} er 3 miile og noget meere. J det øfrige angaaendes folckenes og hæstenes underholdning; udsiger han Det Samme som første Vidnet.\par
Og Som dette 4{de} Viidne icke havde meere at udsiige: end at hands formænd har brugt Skogen og Fieldene efter deris bøxel Sæddel, hvor af hand een af 18 decb{r} 1707: til grænse Commissionen i fjor har indleveeret, u-paa anchet af de S[v]enske fra gammel tiid, Saa blefv hand dimitteret.\hspace{1em}\par
5{te}: Viidne.\par
Til 1{te} Spørsmaal Svarer? Hand heeder \textit{Friderich Jensen Ferragen}.\par
2: \textit{Resp}: Føed paa \textit{Ferragen} i \textit{Røraas} Sogn: Hans Fader en bonde paa benefn{te}Ferragen ‒\par
3{die}\textit{Resp:}? hand er 25 aar gammel: tilholder ved \textit{Ferragen}, har bergs arbeide her paa \textit{Gruberne} ‒\par
det 4{de}? om hand er beckiendt paa Grænserne her imellem Guldahls Fogderie i \textit{Trondhiems} Stift og \textit{Herjedalen} paa den anden Svenske Siid[e], og viidere i Søer ‒\par
\textit{Resp:} ja noget.\label{Schn1_9435} \par 
\begin{longtable}{P{0.09842105263157895\textwidth}P{0.07157894736842105\textwidth}P{0.68\textwidth}}
 \hline\endfoot\hline\endlastfoot Til det\tabcellsep 5{te}\tabcellsep Spørsmaal? Svarer det Samme Som 1{te} viidne ‒ Det Samme Som første viidne ‒\\
og\tabcellsep 6\\
og\tabcellsep 7:\end{longtable} \par
 \par
det 8{de} Spørsmaal: hvilcket er det første Field i Syder viidnet Kiender som ligger østligst eller mest østlig need til \textit{Herjedalen} paa den østere Svenske Siide?\par
\textit{Resp}: hand har iche længer værit Søer i Fieldene, end til \textit{Elgaahongna}.\par
Til 9{de}: 10{de}, 11{de}: 12{te} Spørsmaale angaaend \textit{Elgaahongnes} Stræckning og Landets \textit{Situation} der omkring Siger hand det Samme Som 4{de} viidne Nemblig \textit{Lasse Femund}; Undtagen der j, at hand ej har værit paa dend Syd østere Siide fra \textit{Elgaahongna} hen til \textit{Sal} Fieldet ej heller ved \textit{Gruvel} Søen eller \textit{Lang}-Fieldet; og der fore ej Kiender dets Landskabs \textit{Situation}; Ligesaa har hand ej heller Seed det Lille Field \textit{Grøtvals} Stødten, Som hand veed af, ved hvis Syndere og østere Siide \textit{Grøttvalds} Søen skulde ligge.\hypertarget{Schn1_9542}{}4-5 Vidne i Guldals Fogderi.\par
Derimod veed dette viidne meere end forbem{te}\textit{Lasse Femunden:} Nembl: at Elfven \textit{Elgaaen} Kommer fra \textit{Elgaa} Søen og rinder der fra beent i væster omtrent en 1/2 miil lang ud i \textit{Femund} Søen; Ligesaa har hand og værit paa det Field \textit{Kraltvola} eller \textit{Wonvola} Kaldet: Som Skal ligge fra det Field \textit{Grøtt Hongna} og \textit{Grøttaadalen} i Nord ost.\par
Paa 13{de} 14{de} 15{de} 16{de} 17{de} og 18{de} Siiger hand om \textit{Kraltvola} eller \textit{Wonvola} Saa og om \textit{Siebrue} Field det Samme som 3{die} viidne Anders Pedersen Dahl. Dette endnu tilleggendes; at østen for det Field \textit{Kraltvola} i Syd ost ligger en Field Stødt Kaldes \textit{Wonsiøgusten}, hvilcket er Rund, og Kand være 1/4 Fierding Miil over fra Foed til foed.\par
Jmellem dette \textit{Wonsiøgusten} og \textit{Kraltvola} ligger een liiden \textit{Wohle} som hand meener er egentlig det Som heeder \textit{Wonvola;} fra dette \textit{Wonsiøgusten} i Nord væst til \textit{Kraltvola} kand der være en1/4 Miil vejs; liige i øster fra \textit{Kraltvola} ligger Toe Smaa \textit{Vohler}: dend eene efter dend anden i øster, Som hand ej veed at Nafngive; ellers er Landskabet østen for \textit{Kraltvola} Steenet og u-lændt; Viidere Field beent i øster fra \textit{Kraltvola:} findes icke uden nogle Smaa \textit{Vohler} hvor efter Landet dahler neer til Skougdahlen ad \textit{Tennes} bøyd i \textit{Herjedahlen;} Nord ost fra forbem{te}\textit{Wonsiøgusten} ligger Een Søe Kaldet \textit{vonsøen}, hvilcken Søe er Rund og kand være en 1/2 Fierding Stoer og er Fiskeriig, Som bruges af \textit{Tennes} Mænd i \textit{Herjedahlen;}\par
Beent i øster fra det Field \textit{Wonsiøgusten} 1: Miil omtrent ligger \textit{Sluggu} Field Som hand vel har Seed; men icke værit derpaa; Saa hand kand ej Siige, hvor danne dets Stræckning og Størelse er. Om dette \textit{Sluggu} Field har hand hørt at det Skal giøre Skillet imellem \textit{Eire} bøyd og \textit{Tennes} bøyd\textit{Herjedalen;} ud af forbem{te}\textit{Wonsiø} Rinder een bæck j Nord en 1/4 miil vejs i \textit{Rogen} Søe; og ligger Jmellem disse 2{de} Søer, de Toe forberørte Smaa \textit{Vohler} væsten for \textit{Wohn}-bæcken.\par
\textit{Rogen} Søe Stræcker sig fra Væster i Syd øst 3 maalte miile, og Kand være fra Søer i Nord hvor den er viidest 3/4 miil vejs breed.\par
Fra denne \textit{Rogen} Søe udriinder dend Elfv \textit{Røa} Een 1/2 miil i væster i dend Søe \textit{Roassen}, og Siiden viidere.\par
Viidere viiste dette viidne at forcklare om \textit{Siebrue} Fieldet, at det Stræcker Sig fra Nord væst i Syd ost 3/4 vejs Maalte Miile. Fra Syd væst over i Nord ost fra Foed til Foed er det en 1/2 miil ‒\par
Fremdeelis viste hand at forcklare Land-Skabet paa dend Nordre og østere Siid af \textit{Siebrue} Field; Nembl: Paa den Nordre Siide er det Steen Rabbet med Smaa Kynner og hængger ved En Sløgt eller liden Field dahl Sammen med \textit{Wigel} Skaftet og \textit{Raug} Fieldet.\par
Dend østre Siide er og houget og har Smaa Kynner, hvor af Een Elfv ved Nafn \textit{Mysgla} udriinder i Syd ost og vænder Sig øster ad \textit{Tennes}dahlen i \textit{Herjedahlen;} ‒\par
Om \textit{Brat Rie} Field Siiger hand det samme Som første viidne; og legger der til: at det j Sig Selv er ganske Rundt, men der ligger Een Tang eller Foed der fra i øster og væster hvor af det Kand Siiges at have Sin Stræckning fra væster i øester.\par
Om \textit{Raug} Fieldet og \textit{Wigel} Skaftet Stadfæster han det første viidnes udsagn i alle Ting, tilføyendes det, at det Stræcker sig fra Syd ost i Nord væst omtrent 1: miil; Saa at hand kommer over eett med det første viidne fra det 8{de} til 17{de} Spørsmaal \textit{inclusive}.\hypertarget{Schn1_9860}{}Schnitlers Protokoller I.\par
Om det Field \textit{Wigeln} Stadfæster hand det Samme som næste: 4{de} viidne \textit{Lasse Femunden} har udsagt; og tillegger dette, at det Stræcker Sig i Nord til Nordvæst 1 1/2 miil lang: er breed fra væster til øster foed en 3/4 vej.\par
Landskabet paa dend væstre Siide er Myhrland, hvor Knapp Fæe-beede er; Stræckende Sig i væster til Ferrag Platzen en 1/4 Miil.\par
Paa dend Sydvæstlige Siide gaar det med Sin Tang el: Foed en 1/2 Fierdingsvej need ad Ferrag Søen.\par
Landskabet østen for \textit{Wigeln} er een Søe Kaldet \textit{Field Baalagen}, liggendis i Foden af \textit{Wigeln} og Stræcker Sig fra Søer i Nord en god 1/4 vejs: og iche fuld 1/8 miil breed. Strax fra Søen i øster Staar op et Field ved Nafn \textit{Waattaa} Fieldet, Stræckende Sig fra Syd ost til Nordvæst 3/4 vejs; og kand være breed fra væster over til østre Foed 1 1/2 Fierding miil.\par
Dette \textit{Waattaa} Field hænger paa dend Syndere Siide Sammen med \textit{Raug} Fieldet.\par
Landskabet paa dets østre Siide hælder need til \textit{Tenna} Elfv i \textit{Tendalen}, og har \textit{Funnes}dalingerne der deris buehafn.\par
Landskabet paa dend Nordre Siide dahler need til \textit{Tenn} Dahlen, og er der de \textit{Funnesdallingers} bue hafn.\par
Norden for \textit{Wigeln} forcklarer hand, ligger \textit{Baalagen} Søe, Stræckende Sig fra Syd ost omtrent een 1/2 miil i Nord væst; breed er dend Søe over en 1/2 Fierding og er Fiskerie af Øret der i; og er denne Søe at forstaa om Field-\textit{Baalagen} Søe; thj Nordvæst fra \textit{Rutten} ligger \textit{Bræch-Baalagen} Søe.\par
Paa denne Siste Søe i Nord følger een Dahl ved Nafn \textit{Waaldahlen} Stræckende Sig fra øster i væster 1 1/2 miil.\par
J: denne dahl ligger een Kyn: \textit{Waul}-Kynden Kaldet, hvor af dend bæck \textit{Waula} udrinder 1 1/2 Miil i \textit{Borgen} Elfv.\par
Liigesom og af \textit{Baalagen} Søe udkommer Dend Elfv \textit{Borga}, og Rinder omtrent en 3/4 Miil ud i \textit{øresund}.\par
De Spørsmaale fra 19{ten} til 23{ve} Svarer hand det Samme til, som det første viidne giort haver.\par
Hvad det 4{de} viidne \textit{Lasse Femunden} ved det 13{de} Spørsmaal i Enden om \textit{Femund} Søen har udsagt, det beckræfter nærværend Viidne og føyer der til, at den \textit{øe: Soller øe}, ligger i midt-diubet af \textit{Femund} Søen; dog længere bort fra dend Nordre Ende til dend Syndre, og Som hand efter den Nye Maalning giætter og reiigner \textit{Femunden} kun 7 miile lang: Saa Kand der blive fra dend Nordre Ende til \textit{Soller øen} 4: miil, og derfra til dend Syndere Ende 3: miil.\par
\textit{Femunds} Søndere Ende er iche Kløvet Som Nordre Ende, men Rundagtig;\par
Fra \textit{Femunds} Syndere Ende 1/4 Miil der fra paa dend væstre Siie; udløber 2{de} Elfve, eett Spøsse Skud fra hinanden, liigeleedes ett Bøsse-Skud Lang, ind i en liiden Søe Nafnl: \textit{Gloten}, Som er Rund og omtrent Toe Bøsse Skud breed.\par
Viidnet havde iche meere at Siige, og blef Saa bortladt, og denne Forrætning Paa dette Stæd Sluttet, og af Laug-Rættet tillige med underskrevet og Forseiglet.\par
\textit{Peter Schnitler}\centerline{\textit{Elling Svendsen}. BelgMager (L. S.)}\centerline{\textit{Joen Andersen}tolgen. (L. S.)}\hypertarget{Schn1_10109}{}5-6 Vidne i Guldals Fogderi.\par
Siiden ingen af Fogderiets Kongel: betiendter formedelst paakomne u-føere her har værit tilstæde, Saa blef \textit{Directeuren} her paa værcket S{r}\textit{Borgreving} her fra Rætten tilskreven at hand Som nærmest Kongel: betiendt ville foranstalte, at ved tilstundende Junii maanets udgang, da de Kongl: Norske grændse maalere formeentl: paa \textit{Wonvola} eller \textit{Kraltvola} Field østen for \textit{Grøthongna} begynde deris Samling eller der omckring nogenstæds, hos dennem Sig da indfinde først lensmanden eller anden beqvæm Mand med de af hørte og her\label{Schn1_10139} \par 
\begin{longtable}{P{0.8134969325153374\textwidth}P{0.036503067484662574\textwidth}}
 \hline\endfoot\hline\endlastfoot efter i denne Eign ved Færragen og i bræcken afhørende Viidner, 2: De Kyndigste\tabcellsep (L. S.)\\
mænd paa hver Stæd at vej viise grændse maalerne omtrent de østligste Grændse\end{longtable} \par
 \par
Fielde ad \textit{Herjedalen}. 3: arbejdere Til at drage Kieden og forfærdige Broer Klopper Flotter eller om haves kand baader, at komme paa over Søer og Elfver og ellers være dennem behielpelig til at faa for betalning underholdning paa de afliggende øede Field. ‒ og det til deris Kongl: May{ts} tieniste og Grændze Commissions befordring \textit{Røraas} dend 17 Apriilis 1742. \centerline{\textit{Peter Schnitler}.}\hspace{1em}
\DivII[April 18. Rettsmøte på Koyen ved Røros]{April 18. Rettsmøte på Koyen ved Røros}\label{Schn1_10177}\par
\textbf{Koyen,} Een Liiden bonde gaard Liggendes i øster fra \textit{Røraas} bergstad 2: Miile og fra Færragen gamble \textit{Hytte}-Platz ved Samme Færragen Søes Nordre Ende en 1/2 Miil i Nord væst er det andet Stæd hvor A{o}1742{ve} d: 18{de}\textit{Apriil} Viidners \textit{Examen} blefv holdet, angaaendes Grændse Fieldene imellem Norge og Sverrig paa denne Kandt.\par
Ved \textit{Rætten assisterede} Tvænde opnæfnte mænd Som Laug-Rættes Mænd, Nafnl: \textit{{Steen Andersen Ferragen}}: 62 aar gammel; og \textit{{Jacob Andersen Sund}} ‒ 59 aar gammel.\par
Kongl: Betiendtere vare ej tilstæde Nembl: Fogd og Soeren Skriiver, som for u-Føere og Nyelig opgangne Elfve ej heller have kundet fremkomme.\par
Som Viidner, Fremckom for Rætten hvilcke \textit{Directeuren} paa \textit{Røraas} fremskaffede \textit{{Knudt Olsen Ferragen}}, \textit{{Bent Thøresen}}, ibm. og \textit{{Peder Jonsen}} ibm.\par
For dennem blef den Kongl: \textit{ordre} Til denne Rættes holdelse \textit{Publiceret}; der næst Eedens Forcklaring af \textit{Lov}-bogen oplæst; og viidnerne derpaa i \textit{Corporlig} Eed Tagen.\par
Af dennem blef da foretagen ‒\hspace{1em}\par
6{te} Viidne? \textit{Knudt Olsen Færragen}, Føed i \textit{Størdahlens} Præstegield og Fogderie j \textit{Trondhiems} Stift; hands Fader var een bonde Sammestæds; Er 74 aar gammel; er gift og har 9 børn; har een Platz ved dend forrige gamble \textit{Hytte} Paa dend Nordre Siide af \textit{Ferragen} Søe; Nærer Sig med bærgs arbeide til \textit{Røraas} Kobber værck; ‒\par
1 Spørsmaal, hvor langt denne gaard \textit{Koyen} ligger fra Grændserne? i hvad Sogn og Fogderie?\par
\textit{Resp:} Denne gaard \textit{Koyen} ligger j \textit{Røraas} Præste Sogn\textit{Guldahls} Fogderie, 2 miile i væster fra høyeste \textit{Rutt} Fieldet; og Folckene her paa gaarden Nærer sig ligeleedes af bergsbrug til værcket, noget Foerland have de vel, men Korn voxer her iche.\par
2: hvordanne Landets beskaffenheder, imellem Dette Stæd og grændse Fieldet: at \hypertarget{Schn1_10354}{}Schnitlers Protokoller I forstaa om der er Skoug, vande, Elfve, Fielde, Myhr, eller \textit{Morast}, Dyrcket og bebygget, eller øede u-frugtbar Land?\par
\textit{Resp}: Om Landets beskaffenhed Siiger hand det Samme Som første viidne: ved 6{te} Spørsmaal. Folck som boer her imellem ere en 9 Mand paa Gamble \textit{Ferragens Hytte} Platz i Syd ost fra denne Gaard \textit{Koyen} Som har nogle Smaa Platze, de Sidde paa, Dernest Een liden gaard \textit{Walset} i øster her fra, og viidere i øester Tædt under \textit{Rutt} Fieldet 13 Smaa \textit{Bræche} gaards parter; Foeder til hæster og \textit{Creaturer} have de vel, mens Faar iche Korn, og leeve alleene af brug til \textit{Røraas} værcket.\par
Siiden dette Viidne er fra \textit{Ferragens} Platz; saa forcklarer hand om \textit{Ferragen} Søe, at dend Stræcker sig fra Nord i Syder dog lidet østlig imod 1 1/2 Miil; fra væster i øster er dend Største dehls een 1/4 Miil breed; mens ved dend Syndere Ende gaar een Smal viig ud een 1/4 vejs lang og et Bøsse Skud breed, hvilcken viig kaldes \textit{Haaen}; og af denne Ferragens Søes\textit{Haae} udRinder \textit{Haae} Elf, Som Rinder først een 1/2 Miil i Syd væst i en Søe Kaldes \textit{Haae}-Søen; den Stræcker Sig i væster en 1/2 Miil og er een 1/8 miil breed; der efter gaar bem{te}\textit{Haae} Elfv et Bøsse Skud i væster ind i \textit{Ramberg}-Søen der er liidet mindere end \textit{Haae} Søen; om Siider gaar den 1: miil vejs i Nord væst op til \textit{Røraas} Platz. \textit{Ferragens} Søes Syndere Ende ligger Fra \textit{Femunds} Søens Nordre viig, en 1/2 miil; dog er der fra i Søer een Kynn: Kaldes Lang Kierna, hvor af een bæck Riinder i Nord ind i Ferragens Søe; Fra denne Lang kiern til \textit{Femunds} Nordre Viig Kand være et Smalt Støcke land af et par Bøsse Skud lang; og dette er det Støcke land, Som \textit{participanterne} af \textit{Røraas} Kaabber værck forehafde at vilde lade igiennem grave, for at føre deres Tømmer og Veed der igiennem til vands ud af \textit{Femund} Søen liige op til \textit{Røraas} platz. J denne Ferragens og de øfrige Smaa Søer fanges Fiske; Harr, øret, giædder, Abborr, og Lache.\par
Fra denne \textit{Ferragens} Søes Nordre Ende i Nord-Nordvæst een Stor Fierding Miil der fra ved \textit{øresunds} Syndere \textit{Botten} har i Forrige Kriig, med Sverrig; i Kong \textit{Christiani V:} tiid Staaed en liiden \textit{Schandze} af Træe, hvor man nu iche meere Seer Spoeret af.\par
Denne \textit{Ferragens} Platz er og een af de Nærmeste gaarder imod dend Syndenfieldske \textit{district} af \textit{Aggerhuus} Stift, og have een 4: Miile til \textit{Tufsing-dalen} dend første bøyd i \textit{Tolgens} Præstegield Syndenfields; Landskabet her imellem er u-lændet med Steen, \textit{Morast}, vande, Field vaahle med liiden Skoug, udyrcket og øede.\par
3: Hvilcke ere de Nærmeste Gaarder her i Norge ved grændse Fieldet, at reigne fra Søer i Nord, Paa denne Kandt, og af hvad beskaffenhed Landskabet der af er, og hvad Næring bønderne bruger?\par
\textit{Resp:} Liige Som første viidne ved 7{de} Spørsmaal har forcklaret; Landskabet er haard og Slett, Som det før er beskreeven, noget liidt til Foeder kand de have, men Korn Saar de iche.\par
Som det var Saa Siilde paa aftenen blef forrættningen opsatt til følgende Morgen Nembl: Dend 19{de}\textit{Apriilis} 1742{ve}; Da i forbenæfnte LaugRættes Mænds overværelse; Samme Viidne Knudt Olsen Ferragen, viidere blef \textit{examineret}.\par
4: hvilcket er det første Field i Syder imellem Norge og Sverrig, Som viidnet Kiender, at ligge mest østlig need til \textit{Herjedalen} ad den Svenske øestere Siide?\hypertarget{Schn1_10598}{}6 Vidne i Guldals Fogderi.\par
\textit{Resp:} det Syderste Field, hand kiender er først \textit{Elgaahaangna;} dog ligger nogle Fielde der fra i Nord ost, Som Siiden skal forcklares. ‒ Dette \textit{Elgaahaangna} Stræcker Sig fra Syder i Nord væst; dets Størelse og landskab paa den væstere Siide beskriiver han liige som det 4{de} viidne ved 9{de} l0{de} og 11{te} Spørsmaal; paa den Syndere og beent østlige Siide er hand iche beckiendt; men paa dend Nord østlige Siide har Hand Seet det Field \textit{Grøtthaangna}, hvor om hand giver Samme forcklaring Som det 3{die} viidne \textit{Anders Dahl}, fra det 8{de} til 12{te} Spørsmaal \textit{inclusive;} der til leggendis at dets Stræckning gaar fra Syd ost i Nord væst.\par
J Nord ost fra Grøthaangna har hand Seet det Field \textit{Kraltvola} Som og ellers kaldes \textit{Wonvola}, der om giiver hand dend Samme beskriivelse som forrige 3{die} viidne fra det 13{de} til 16{de} Spørsmaal, og legger til dend forcklaring om \textit{Kraltvolas} østere Siide, at landskabet fra Fieldet dahler need til \textit{Rogen} Søe en 3/4 miil veis, Steenet og Myhret. Denne Søe Siger hand begynder østen for \textit{Sluggu} Fieldet, gaar først Krumagtig Nord væst, Siiden beent i væster og Endes imellem de Fielde \textit{Kraltvola} og \textit{Siebrue} Field, og Skal dend Søe være 3 miile lang, og midtpaa Nembl: østen for \textit{Kraltvola} een 3/4 Miil breed, og østen for denne Søe er dahlen af \textit{Herjedalen}. Stræckningen af \textit{Kraltvola} Field, Siiger hand, at gaa fra Syd ost i Nord væst. ‒\par
J Sydøst ligger \textit{Wonsiøgusten} fra \textit{Kraltvola}, hvorom hand giver Samme forcklaring som 5{te} viidne Friderich Ferragen ved hands 18{de} Spørsmaal.\par
Det Field \textit{Sluggu} Field ligger liige i øster fra \textit{Wonsiøgusten} 1: miil, hvor dette viidne Selv har værit paa Dette Sluggu Field Stræcker sig fra væster i øster og er høyt og Rundt, dog mest høy paa dend væstere Siide og Sluter neer paa dend østere Siide; det kand være Stort fra Dend væstre til østere Siide 1 Miil. ‒\par
Og dette \textit{Sluggu} Field Siger hand at være et grændse mærcke imellem \textit{Eire} bøyd i Søer og \textit{Herjedahlen} i Nord.\par
5: Om hand veed hvad viidere Skilnet giør imellem \textit{Eire} og \textit{Zerne} bøyder paa dend eene og Herjedalen paa den anden Siide?\par
\textit{Resp:} Hand har veld hørt at det yderste mærcke i øster Skal være Kors-kiilden men har ej været der; der paa i væster Følger \textit{Wætta}-Field, som hand vel har Seet at være Som en liden \textit{vohle} men iche værit der paa; j væster der paa ligger \textit{Næss} Field som er vel Saa Stort Som \textit{Wætta} Fieldet; dernæst i væster \textit{Lill}-Fieldet, og Siden i væster \textit{Sluggu} Fieldet, Som før er beskreeven; viidere have de Svenske fortælt ham, at \textit{Limitten} i mellem \textit{Eire} og \textit{Herjedalen} gaar fra \textit{Sluggu} Fieldet i væster til \textit{Wonsiøgusten}, der fra i væster til \textit{Grøttaa}- dalen, Siiden østen og Nord om \textit{SvuKue} Fieldet i en Sløgd imellem \textit{SvuKue} Field og \textit{Røvola}, hvor een Kyn Nafnl: Stoer-\textit{øre} Kiern ligger om denne Kiern har hand hørt af een Fisker fra Jemteland, der fordum tiendte hos \textit{Berg-Meister}\textit{Jrgens}, at dend var Skille Mærcke imellem \textit{Eire, Herjedalen, Aggerhuus}, og \textit{Trondhiem}: men mærcker eller Vahrer paa disse Fielde veed hand iche nogenstæds at have Seed. Efter denne viidnets forcklaring er da \textit{Kraltvola} eller \textit{Wonvola} de første Field i Nord fra \textit{Eire} bøjds Skille-mærcke, hvor fra man østerligst need gaar til \textit{Herjedalen}. ‒\par
6{te} hvad viidere i Nord fra dette \textit{Kraltvola} ligger for Fielde østerst ad \textit{Herjedalen?}\hypertarget{Schn1_10907}{}Schnitlers Protokoller I.\par
\textit{Resp:} Norden for dette \textit{Kraltvola} ligger dend væstre Ende af \textit{Rogen} Søe, Som før er beskreven. ‒ Norden for denne Søes væstre Ende ligger \textit{BratRie} Fieldet, hvilcket Field hand beskriiver med Sine omstændigheder ligesom første viidne: fra 8{de} til 12{te} Spørsmaal, dette tilføjendes, at det Stræcker Sig fra Sydvæst i Nord ost. ‒\par
Paa dette \textit{BratRie} Field følger i Nord væst \textit{Raug}-Fieldet, hvor om hand giver Samme Forcklaring Som forbm{te} 1{te} Viidne fra 13{de} til 17: Spørsmaal \textit{inclusive;} Tilleggendes: at dett Stræcker Sig fra Væster i øester.\par
J Nord fra dette \textit{Raug} Field ligger \textit{Waattaa} Fieldet, Som har væsten fra Sig liggendes det Field \textit{Wiggeln}; om disse Fielde udsiiger hand det Samme Som 5{te} Viidne Friderich Ferragen ved det 18{de} Spørsmaal mod Enden. ‒\par
J Nord væst fra \textit{Waattaa} Fieldet og Norden for \textit{Wigeln} ligger dett Field \textit{Rutten:} Dog ligger imellem \textit{Waattaa} Field og \textit{Rutten}\textit{Fieldbaalagens} Søe, hvis Størelse og Stræckning 5{te} viidne ved Enden af det 18{de} Spørsmaal har beskrevet. ‒\par
Dette \textit{Rutten} er en Rund Klimp, og bradtest paa den væstre Siide; Stræckende Sig fra Nord i Syd oest; Viider herom foor man oplysning om i bræche gaardene. ‒\par
7: Om hand veed, hvad Skilnet er i Nord imellem de Svænske Landskaber \textit{Herjedalen} og \textit{Jemteland} ved de Norske Grændser i Væster?\par
\textit{Resp:} Nej.\par
8: Om hand veed, hvem den væstre Siide af de forhen opReignede Fielde imellem Norge og \textit{Herjedalen} tilhører, og veed hvis Gaarders grund de ere beliggende?\par
\textit{Resp:} Svarer det Samme som 1{te} Viidne ved dett 19{de} Spørsmaal; dette tilleggendes at dend væstere Siide er Kongens \textit{Alminding}. ‒ Efter \textit{Directeur Brogrevings} forlangende, blef viidnet tilspurt\par
(A) om icke hæstbeed aasen og Røaasene Som ligge paa dend væstere Siide af \textit{Wigel} Skaftet, og paa dend østere Siide af \textit{Femund} Søen, ligger Jnden \textit{Røraas} værckets\textit{Circumference?} ‒\par
\textit{Resp:} Ja. ‒ viidere\par
(B) Om \textit{Røraas} værck længe har haft brug udj hæst beed aasen, og viidere paa dend østere Siide af \textit{Femund} Søen, paa væstre Siide af \textit{Wigeln} og \textit{Wigel} Skaftet?\par
\textit{Resp:} Ja, hand meener det kand være en 30{ve} aar: meer eller mindre, ‒ fremdelis\par
(C) Om hand veed at \textit{Lars Olsen}, Sal: Tørris olsen, og Sal: Ole Joensen Jembt \textit{Ferragen} har brugt og bøxlet af dend Kongl: Norske Foged, Sæterbolig og Engesletter udj Svartviig dahlen, ved heestbeed aasen under \textit{Wigel} Skaftets væstre Side?\par
\textit{Resp}: Ja, De har brugt og bøxlet. ‒\par
(D) Liigeleedis om Bendt \textit{Ferragen} og Ole Johansen \textit{Ferragen} har bøxlet Sæterboliger og Enge Sletter ved \textit{Liusenaaen} og op i \textit{Liusenaadalen}, hvor af de Svenske har indtaget een Stoer deel med Sin \textit{Linie?}\par
\textit{Resp}: hand veed iche andet end at de 2{de} mænd har bøxlet: men om de Svenske Maalere har indtaget eendehl der af, veed hand iche.\hypertarget{Schn1_11201}{}6-7 Vidne i Guldals Fogderi.\par
9: Hvad Nytte, godhed, og Herlighed, er der, ved disse Fielde, Som de Svenske Maalere har Villet tilEigne Sig?\par
\textit{Resp:} Liigesom 1{te} viidne veed det 22{ve} Spørsmaal har Svaret. ‒\par
10: om der har værit nogen tvistighet imellem vore og de Svænske om disse Fielde, jtem: hvad Rætt og adckomst enten vore eller de Svenske foregive at have?\par
\textit{Resp:} Svarer det Samme Som første viidne, dette tilleggendes,: Som det Synis ham maa de være Kongens \textit{Alminding}.\par
11: hvad Mænd hand meener beqvæmmist og Kyndigst at være, til at udviise de østerste Fielde. Som ligge Nærmist ad \textit{Herjedalen}? Svarer det Samme Som 1{te} viidne ved 21{de} Spørsmaal, der hos Siigendes, at \textit{Friderich Jensen Ferragen} Som een ung Rask og Fieldckyndig Karl, vel der til kand være beqvem. ‒\par
12: hvor Langt de østlige Fielde ere fra bøjdene og Lande vejen beliggende?\par
\textit{Resp:} Liigesom Fieldene ligger langt borte og Nær til; Saaleedis: fra \textit{Kralvola} til \textit{Lasse Femund} kand være 3 1/2 miil lidt meer el: mindre fra ham til \textit{Røraas} 3: miil, derimod fra \textit{Rutten} til \textit{Bræchen} 1: miil og her fra: til \textit{Røraas} 3 miil. ‒\par
13: hvor underholdning for Folck, og beete for hæstene paa eller veed Grændse Fieldene er at faa?\par
\textit{Resp}: Det Samme som 1{te} viidnet, ved det 23 Spørsmaal j Enden.\par
Som hand iche videre havde at Siige blef hand \textit{dimitteret}. ‒\par
7{de} Viidne ‒\par
Heeder \textit{Bendt Tørresen Ferragen}, Føed paa \textit{Ferragen} hvor hands Fader og boede og arbeidede til værcket; er 72 aar gammel og gift, haver 4{re} børn, boer paa dend gamble Ferragens hytte Platz? ‒\par
Til det 1 Spørsmaal Svarer hand det Samme som det 6{te} viidne Knudt Olsen Ferragen ved dette 1{te} Spørsmaal har Svaret ‒\par
Til det 2{det} Liigesom 6{te} viidne ved dette 2{det} Spørsmaal.\par
Til 3{die} Svarer det Samme som 6{te} viidne ved dette 3{die} Spørsmaal.\par
Til det 4{de}: Svarede hand at hand iche har værit længere i Fieldene end til \textit{Røen} Elfv, og een gang om viinteren faret over \textit{Rutten} need til \textit{Fundnes-dalen} og \textit{Tennes} i \textit{Herjedahlen} Saa hand iche er kyndig paa de østlige Fielde.\par
Til 5{te} 6{te} 7{de} og 8{de} veed indtet at Svare.\par
Til \textit{Directeurens} Spørsmaal Lit: (A)? Svarer Ja, Som hand iche andet veed, efter dj værckets Folck have haft Stædz deris brug der, over en 30 aar. Til Lit: (B)? Svarer Ja. og \textit{Referrer} Sig til tiiden paa Sit nest forrige Svar ‒ Til Lit: (C:): at Ole LarsenTørris Ericksen og Ole Joensen have brugt og bøxlet Sammestæder af dend Norske Foget. ‒\par
Til Lit: (D)? Svarer Ja. ‒\par
Til 9{de} Spørsmaal Svarer det Samme Som 6{te} Viidne ved dette 9{de} Spørsmaal har forcklaret.\par
Til 10 og 11{te} Svarer ligeleedes Som 6{te} Viidne. ‒\par
Til det 12{te} Svarer at Siiden hand iche har befaret Fieldene uden \textit{Rutten} Saa veed \hypertarget{Schn1_11447}{}Schnitlers Protokoller I. hand icke viidere herom at give beskeeden: end om \textit{Rutten}, hvilcken er den Samme som 6{te} viidne har udsagt.\par
Til det 13{de} Svarer ligesom det 6{te} viidne, hvor paa hand blefv bortladt. ‒\par
8{de} Viidne. ‒\par
Heeder \textit{Peder Jonsen Ferragen} Føed i \textit{Fundnes dalen} af hans Forældre Som vare bønder folck, er 53 aar gammel, gift og har 4 børn, boer paa dend gamble Ferragens hytte Platz. ‒\par
Til 1{te} Spørsmaal Svarer Som 6{te} Viidne ved dette Samme Spørsmaal. ‒\par
Til det 2{det} og 3{die} Svarer ligesom 6{te} viidne til Samme Spørsmaale har Svaret. ‒\par
Til 4: Spørsmaal Svarer hand som det 6{te} viidne: angaaende \textit{ElgaahongnaGrøtthaangna} og \textit{Kraltvola}, Saaleedis at østen for \textit{Kraltvola} daler Landskabet neer til \textit{Rogen} Søe og der fra Siiden i øster viidere need til dahlen i \textit{Herjedahlen}. Om Landskabet paa den væstre Siide af \textit{Elgaahaangna} forcklarer hand Sig, at Der kand være noget liidet Myhr Græs til beede: J Syd ost fra \textit{Kraltvola} kand han vel have Seed nogle Fielde, men veed iche Nafn derpaa. ‒ hvor fore hand ingen Forcklaring veed at giive om \textit{Wonsiø gusten} eller \textit{Sluggu} Fieldene, ligesaa veed hand og iche breeden og Længden af \textit{Rogen} Søe: dog har hand hørt, at den skal være 3 miile lang.\par
Til det 5{te} Spørsmaal veed hand indtet at Svare. ‒\par
Til det 6{te} Spørsmaal Svarer? Ligesom det 6{te} Vidne paa dette Samme 6{te} Spørsmaal ‒\par
Til det 7{de} Svarede Nej. ‒\par
Til det 8{de} Svarer Som 6{te} Viidne. ‒\par
Paa \textit{Directeurens} 4{re} Spørsmaale Svarer hand det Samme Som 6{te} Viidne. ‒\par
Til det 9{de} 10{de} 11{te} 12{te} og 13{de} Spørsmaale, Svarer det Samme som 6{te} Viidne, har forcklaret. ‒ Og Som hand indtet Viidere havde at viidne, blef hand \textit{dimitteret}. ‒ Og Forrætning paa dette Stæd Endt. ‒\par
Og af de 2{de} LaugRættes Mænd underskrevet og forzeiglet ‒ \hspace{1em}\centerline{Peter Schnitler (L. S.)}\centerline{Steen andersen Ferragen. (L. S.)}\centerline{Jacob andersen Sund. (L. S)}\hspace{1em}
\DivII[April 21. Rettsmøte i Brekken]{April 21. Rettsmøte i Brekken}\label{Schn1_11654}\par
\textbf{Bræchen} dend østerligste bøyd i Guldahlen Nærmist Til \textit{Herjedahlen} er det 3{die} Stæd hvor A{o}1742{ve} dend 21{de}\textit{Apriil}, viidners Forhør over grændse Fieldene blef foretaget. ‒\par
Ved Rætten vare, efter Joen Henningsen bræcken hands anstaldt, Som LaugRættes Mænd \textit{Jørgen Olsen Bræchen} 66 aar gammel, og \textit{Anders Jørgensen Bræchen} 34 aar gammel. Kongl: May{ts} Foged og Sorenskriiver vare ej her, Saa som de ved u-Vejr der fra ere blefven forhindrede. ‒\par
Efter bem{te}\textit{Jon Henningsens} forandstaltning Mødte Som viidner Gunder Ingbrigtsen, Johannes Nilsen, og Anders Thoresen bræcken, for hvilcke dend Kongl: \textit{Ordre} til denne \hypertarget{Schn1_11725}{}8-9 Vidne i Guldals Fogderi. forrættning dernest Eedens Forcklaring af \textit{Lov} bogen blef forelæst og de aflagde deris \textit{Corporlig} Eed ‒\par
9: viidne.\par
Er Gunder Ingbrigtsen Søevold ved bræcken, Føed i \textit{Funnesdalen} i \textit{Herjedalen}, af Forældre Som vare der bønder Folck, 76 aar gammel, gift, har 4 Sønner er husmand under gaarden Søevold en 1/4 Miil i Sydvæst fra bræckegaardene.\par
1: Spørsmaal? hvor langt ligge disse \textit{Bræche} gaarder fra Grænserne? i hvad Sogn og Fogderie\par
\textit{Resp:} de ligge fra høyeste \textit{Rutt} Klimpene en god Miil elle 5 Fierding vejs, i \textit{Røraas} Kierche Sogn og guldahls Fogderie. ‒ Og nærer sig disse Folck af brug til Røraas værck.\par
2: Hvordan er Landskabets beskaffenhed imellem disse \textit{Bræche} gaarderne og benæfnte \textit{Rutt} Field?\par
\textit{Resp:} Landskabet er Steenet og Myhret, har og 2{de} Kynner og 1: Søe som Kaldes \textit{Baalagen}; dend \textit{Bræche-Baalagen} Søe forklarer hand, ligesom 5{te} viidne Friderich Jensen ved det 18. Spørsmaal mod Enden; ellers er her nogen Smaa biercke Skoug førend man kommer op paa \textit{Rutt}-Fieldet. Paa disse bræche gaarder faaes eller Saaes indtet korn men det er bare Foerland Som Haves. Boesiddendis bønder her i bræcken ere 14 i Tallet; at dette viidnes udsagn \textit{diferrerer} fra 6{te} viidne ved det 2{det} Spørsmaal, der til er aarSag, at 2{de} Mænd have Skiftet een gaardspart imellem Sig i Toe.\par
3{die} h[v]ilcke ere de Nærmiste gaarder her i Norge ved Grændse Fieldet, at reigne fra Søer i Nord paa denne kandt, og af hvad beskaffenhed Landskabet der af er og hvad Næring bønderne bruger?\par
\textit{Resp}: Liigesom første viidne ved 7{de} Spørsmaal. ‒\par
4: Spørsmaal? Hvilcket er det Første Field i Søer imellem Norge og Sverrig, Som viidnet kiender og ligge mest østlig need til \textit{Herjedalen} ad dend Svænske østere Siide?\par
\textit{Resp:} De østerste Fielde, hand i Sønden kiender, ere \textit{Brat Rie} Field og \textit{Raug} Field; om hvi[l]cke Fielde hand udsiiger det Samme Som Friderich Jensen ved det 18 Spørsmaal; Ligesom hand og om det Field \textit{Wigeln} og \textit{Wottaa} Fieldet og deris Landskaber lige saa om Field- og Bræck-\textit{Baalagen} Søer og \textit{Waaldalen} i alle poster kommer over ett med bem{te}Friderich Jensens viidne: dog forcklarer hand at imellem \textit{Wottaa} Fieldet og \textit{Waaldalen} ligger det Field \textit{Rutten}, i Nord væst fra \textit{Waattaa} Fieldet og beent i Nord fra \textit{Wigeln;} Dette \textit{Rutt} Field beskriiver hand at Stræcke Sig fra Nord i Syd ost, og At være Rundagtig med 3{de} opstiigende Klimper, hvor af de Toe yderste Nembl dend Syndere og Nordre ere Smaa, men den mellemste høy; imellem dend Midtelste høye, og dend Syndere Klimp gaar alfarevejen fra guldalen i \textit{Trondhiems} Stift til \textit{Herjedalen} og viidere ind i Sværrig, Som er Steenet Myhret og Trang, Saa at man ej kand kiøre den om Sommeren med \textit{Chaise}. ‒\par
Landskabet paa dend væstere Siide af dette \textit{Rutt} Field er Steenet og Myhret, og 5 Fierding vejs der fra ligge Bræche gaardene i Dahlen der needen under; Paa dend Søndre Siide er det Steenet og dend Søe Field-Baalagen Søe, og Synden for denne Søe er det Field \textit{Wigeln}; Paa dend Nordvæstlige Siide er Rutten bradtest og der under ligger \textit{Bræck-Baalagen} Søe, men beent i Nord ligger dend for omtalte \textit{Waaldalen}. Paa dend østerlige Siide af \hypertarget{Schn1_12000}{}Schnitlers Protokoller I. dette Rutten ligger indtet andet Field, men Landskabet dahler fra de høyeste klimper Strax neer til \textit{Herjedalen}. Dette Rutt Field kand være fra Søer i Nord omtrent en 1/2 miil og fra øster i væster 3/4 miil Stoer. Fra høyeste Rutt Klimpene i øster gaar det een 1/2 Miil neer til \textit{Malmagen} Søe, Som hører til Funnæsdahlen i \textit{Herjedalen}; denne \textit{Malmagen} Søe Stræcker Sig fra Nord væst i Syd ost een 1/2 Miil lang og en 1/2 Fierding Miil breed, og er Fiske-riig af Røe, øret og harr.\par
Udaf Malmagen Søe Rinder een Elfv kaldes \textit{Tenna} Elfv 1: Miil i Syd ost ind i \textit{Tenna} Søen, hvilcken \textit{Tenna} Søe Stræcker Sig Fra Nord væst i Syd ost een Fierding Miil lang og 1/8 miil breed, Siiden Rinder denne \textit{Tenna} Elfv udaf \textit{Tenna} Søe i Syd ost i een Søe der kaldes Øster-Søen og er 3/4 miil lang og 1/4 breed, derefter ind i \textit{Laassen} Søe, hvilken \textit{distance} fra \textit{Tenna} Søen til øster Søen er 1/4 miil og fra øster-Søens Syndere Ende til \textit{Laassen}-Søen er 1 1/2 Miil. Denne \textit{Laasen} Søe Stræcker Sig fra øster i væster 1: miil lang og 1 1/2 Fierding breed; af denne \textit{Laassen} Søes østere Ende, udRinder \textit{Liusna} Elfv; fra forbem{te}\textit{Tenna} Søe er Landskabet ondt af Steen og Myhr, og fra Søen 1 Miil i øster ligger dend bøyd, \textit{Funnes} bøyden, som har Sit Nafn af \textit{Funnesdals}-Søen der Stræcker Sig fra Nord væst i Syd ost en 1/2 Miil lang og er een 1/8 miil breed, liggendis Synden for gaardene; udaf denne Rinder en Elf i øster forbje \textit{Liusendals} Kaabber værcks brug ind i \textit{Liusna} Elfv, omtrendt en 1/4 miil lang. \textit{Fundnes} Dals bøyden ere 14{ten} bønder gaarder; 1/4 vejs og vel det i nord ligger omtrent een 16 platzer, hvor bergs Folck af det Svenske nye optagne Kaabber værcks: det \textit{Liusendalsche} kaldet, tilholde; dette \textit{Liusendalsche} Kaabber værch er første gang omtrent for een 60 aar optagen, men formedelst dend Slette malm skyld Snart igien blefv needlagt, nu kand det for een 4 aar Siiden være igien anfangett, Mens det maa iche være af nogen betydenhed helst efterdj der arbeider kuns ved \textit{Gruben} og hytten een Snees-Mand. Dette \textit{Liusendals}-værck har Sit Nafn af \textit{Liusna} Elfv, paa h[v]is væstre Siide det ligger; denne \textit{Liusna} Elfv udSpringer Som hand har hørt af det Field \textit{Helag}-Stodten i Nord, hvor fra dend mest Sydlig Rinder ind i \textit{Lossen} Søe i ved dends østere Ende, og imellem denne \textit{Liusne} Elfv og forbm{te}\textit{Tenna} Elfv der hvor de løbe ind i \textit{Lossen} Søe, ligger \textit{Tennes Anex} Kiercke en 1 Miil i nord fra \textit{Lossen} Søe, der efter beholder \textit{Liusna} Sit Nafn naar hun gaar udaf \textit{Lossen} Søe og rinder Syd ostlig, Siiden igiennem \textit{Herjedalen} og \textit{Helsing} Land ud i dend \textit{Bottnische} Søe. ‒\par
Norden for Rutt Field ligger \textit{Waaldalen} Som for er omtalt 1 1/2 Fierding miil breed hvor Myhr græss er; Jgiennem denne dahl Rinder een liiden Elfv kaldes \textit{Wol} Elfven, og kommer fra \textit{Gruv}-Søen eller dagvolds gruv-Søen, mens viidere i Nord var hand iche beckiendt.\par
5: Hvilcket Field i Nord følger Som ligger østerst til \textit{Heriedalen?}\par
\textit{Resp:} veed iche. ‒\par
6: Om hand veed hvad Skillet er i Nord imellem de Svænske Landskaber \textit{Heriedalen} og \textit{Jemteland} ved de Norske Grændser i væster?\par
\textit{Resp:} Nej.\par
7: Om viidnet veed til hvis gaard de grænse Fielde, Som hand har forcklaret om, anstøde og under hvis gaarders grund de skulle ligge?\par
\textit{Resp:} De af ham forcklarede Rutt Field og hos liggende \textit{Woaldal} og \textit{Voul}-Kynn for\hypertarget{Schn1_12298}{}9-10 Vidne i Guldals Fogderi. meener hand at ligge under \textit{Bræche} gaardenes grund, thi de Mænd have fra \textit{Arilds} tiid brugt den Stræckning Nembl: i at Sanche Maase paa \textit{Rutt} Fieldet, at Slaa græs i \textit{Waaldalen} og at fiske i \textit{voul} og andre omliggende kynner ‒ ubehindret Af de Svenske og nogen anden. ‒\par
Efter \textit{Directeur Borgrevings} forlan[gen]de bliver viidnet tilspurt, om hannem bekiendt, at Bendt, og Ole Johansen Ferragen af dend Norske Foged have bøxlet Sæterboeliger og Enge Sletter ved \textit{Liøsenaaen} og op i \textit{Liøsenaadalen}, hvor af de Svenske Skal have indtaget een Stoer deehl med Sin \textit{linie?}\par
\textit{Resp:} Saa længe hand kan miindes har de Mænd brugt bem{te} Engesletter, ubehindret af de Svenske: men om de af Fogden have bøxlet den, eller om de Svenske Grændse Maalere eendehl der af have indtaget, det veed hand iche. ‒\par
De øfrige \textit{directeurens} Spørsmaale veed hand iche at Svare paa. ‒\par
8: Hvad Nytte, godhed, og herlighed, er der veed Disse Fielde, om de Svenske maalere Skulle ville tilEigne Sig i neste Sommer? \textit{Resp:} De Svænske grændze Maalere hvor langt de ere i fior gaaet, eller hvorleedis de i næste Sommer vil gaa,? veed hand iche. ‒ Men i Fieldet er hviid \textit{Maasse}-Foer til Kiørne, i \textit{Woaldalen} er høSlott og i Søene og kiønnene er Fiskerie hvor af Bræcke bønderne til deels have deris underholdning. ‒\par
9: Om der haver værit nogen tvistighed imellem vore og de Svænske om disse Fielde, jtem hvad rætt og adckomst enten de Svænske eller vore fore give at have? \textit{Resp:} hand veed iche om nogen tvistighed imellem bræcke mænd og de Svenske at Siige herom. ‒\par
10: Hvad Mænd hand Meener beqvæmmest og kyndigst at være, til at udviise de østerste Fielde Som ligger Nærmist ad \textit{Herjedalen?} Hand Svarer Som det 1{te} Viidne ved 21{de} Spørsmaal.\par
11: Hvor langt de østligste Fielde, ere fra bøydene og Lande vejene beliggende? Det er 5: Fierding fra bræcke bøyden. ‒\par
12: Hvor underholdning for Folck og beete for hæstene paa eller ved grændse fieldene er at faa? Svarer Som første viidne, veed 23 Spørsmaal mod Enden. Og Som hand iche meere viste at Siig fich hand \textit{dimission}.\hspace{1em}\par
10: Viidne ‒\par
Hans Nafn er \textit{Johannes Nilsen Bræchen}, er Føed i bræcken, af bønder folck, i Røraas Sogn i Guldals Fogderie, 80 aar, gift, har 8 børn boer i een af bræcke gaardene, Nærer sig af gaarden, og Værckets brug. ‒\par
Til 1{te} Spørsmaal Svarer det Samme Som 9{de} viidne. ‒\par
Til 2{det} og 3{die} Spørsmaal Svarer ligeleedis Som første viidne. ‒\par
Til 4{de} Spørsmaal Svarer: Det første Field i Søer hand kiender er det Field \textit{Wigeln}, hvilcket hand efter Sin Stræckning beskriiver ligesom 4 viidne \textit{Lasse Femund} ved det 13 Spørsmaal i Enden. Dets Længde og breede med Landskab forcklarer hand liigeleedis Som 4{de} viidne; Mens hvad angaar \textit{Waattaa}-Fieldet og dets Landskab Saa og Field- og bræckeBaalagens Søer, jtem \textit{Rutt} Fieldet og dets Landskab, Saa og om dend dahl \textit{Wauldalen} Jnd\hypertarget{Schn1_12484}{}Schnitlers Protokoller I. til \textit{Malmagen} Søe\textit{inclusive} deri og iche viidere kommer hand over eett med Siste og 9{de} viidne: ved det 4{de} Spørsmaal: Dette tilleggendis, at i nordost fra \textit{Wouldalen} ligger een liden Søe, Nafnl: \textit{Glee}-Søen Stræckende Sig fra Nord i Søer en 1/4 Miil hvor af \textit{Tenna} Elfv udSpringer og løeber i Soer 1/4 Miil lang, først i \textit{Malmagen} Søe, førend dend Siiden udRinder ad \textit{Lossen} Søe.\par
Til det 5{te} Spørsmaal Svarer hand, at i Nord Væst fra det field Rutten og \textit{Waaldalen} ligger det Field \textit{Dahlvola}, Stræckende Sig fra Sydost i Nordvæst meenes en 1/2 miil lang, og liige Saa breed. ‒\par
Dette \textit{Dahlvola} Field hænger i øster Sammen med \textit{Glee} Fieldet ved een Sløgd eller Fielddahl, dette \textit{Glee} Field er flat oven paa, og Stræcker Sig fra Nord væst i Syd ost Som ham Synis vel en 1/2 miil lang, og en 1/2 miil breed. Østen for dette \textit{Glee} Field er der ingen Field, men der fra dahler Landskabet neer ad \textit{Liusendahlen}, Som hører \textit{Fonnesdalen} til og bruges af dets indbyggere til høe Slott. Paa dend væstere Siide af \textit{Dahlvols} Fieldet ligger \textit{Gruv}-Søen eller \textit{Dahlvaalsgruv}-Søen, som Stræcker Sig fra Nord væst i Syd ost 1/4 miil lang og fra væster i øster 1/8 miil breed, i denne Søe Fanges Smaa øret, og ud af dend rinder een Elfv Nafnl: \textit{Gruv} Elfven, een 1/2 Fierding lang i Søer ind i \textit{Wohla}-Elfven, der falder i \textit{Boarga} Elfv, og med dend Siiden ud i \textit{øresund}-Søe. Ved dend Nord væstlige Ende af denne \textit{Gruv} Søe, ligger dend gamle \textit{Dahlvohls grube}; om denne \textit{Dahlvols} grube erindrer Viidnet at nogle \textit{participantere} i \textit{Trondhiem}, de Samme Som vare \textit{interessentere} i \textit{Røraas} Kaabber værck, for meer end 50 aar Siiden havde ladet her paa dette Stæd \textit{Schurfe}, og eendeel Kaaber \textit{Malm} deraf optage: Men de Svænske vare viinteren der efter komne og havde borttaget dend \textit{Malm} og ført dend øst over; Over dette Samme var \textit{general Wibe} Kort der efter kommet hid til bræcken og Skulde have holdet en grændse \textit{Commission} med de Svenske: men der blev indtet af viidere end at der gick bud imellem dem og hvorleedis det er afløben, veed hand icke.\par
Ved dette erindrede viidnet Sig at forcklare at østen for dette forbem{te}\textit{Glee} Field omtrent 1 Miil der fra, er det, at de Svenske have deris Nye optagne \textit{Liusendals} værck, Som hidindtil har kun værit af liiden betydenhed. ‒\par
Landskabet af dend væstere Siide af \textit{Dahlvaalds} Fieldet, naar man er kommet over \textit{Gruv}-Søen, bestaar af Field \textit{vohler} og viidere naar de hælde need til dahlen af nogen liiden biercke Skoug og derpaa følger bræcken gaardene, Som fra Dahlvaalds Field kand ligge i væster 1 Miil; Skille mærcke paa disse Fielde, Nembl: \textit{Rutten Dahlvaald} og \textit{Glee}-Fieldet imellem begge Riigerne findes ingen opsatt; Vel Staar der een Svensk Miil Stolpe paa \textit{Rutt}-Fieldet, 1/4 Miil væsten for de 3 høye \textit{Rutt} klimper: men den er af de Svenske alleene, uden overværelse af Kongel: Norske betiendtere, bleven opRejst Bestaaende af een Fiirckandtet Træe Stolpe, og det er af de Svenske giort, efter at høybem{te}\textit{general Wibe} havde værit her i bræcken ved grændserne; og har viidnet aldriig hørt andet fra gammel Tiid, end at de høye Rutt Klimper Skulde være ett bøyde Mærcke i mellem Guldahlen i Norge og \textit{Herjedalen} i Sværrig, og Siiden i Nor ost \textit{Haftor} Stødten. ‒\par
Til 5{te} Spørsmaal Svarer hand,: er icke længer kyndig i Nord. ‒ end som ved dette samme Sp. før er sagt.\hypertarget{Schn1_12753}{}10-12 Vidne i Guldals Fogderi.\par
Til det 6{te}\textit{Resp:} hand veed det icke.\par
Til det 7{de} Svarer hand angaaendes RuttenDalvollen og de der i liggend vand kynner, lige det Samme Som nest forrige viidne: men angaaende \textit{Dahlvold}- og \textit{Glee}-Fieldene forcklarer hand Sig Saaleedis at \textit{Bræche}-mænd fra gammel tiid have brugt og endnu bruge \textit{Dahlvolds} fieldet med detz Syndere og Væstere Siide: men de Svænske have tilEignet Sig og brugt \textit{Glee} Fieldet.\par
Til det 8{de} og 9{de} Svarer det Samme Som næst forrige 9{de} viidne: ved disse 8{de} og 9{de} Spørsmaale, dette tilleggendis at omtrent for 14: aar Siiden i Sal: Laugmand \textit{Dreiers} Tiid, er een \textit{lovlig} Marckegangs Forrettning af \textit{Rætten} holden, og der efter Skulde \textit{Bræche} gaardene have \textit{Waahldalen} og dens vand kierner med \textit{Dahlvaals} Fieldet og viidere i Nord ost indtil \textit{Liusen Vohla} som ellers og kaldis \textit{Haftor}-Stødten. ‒\par
\textit{Rætten} Edskede da \textit{Bræche Mænd}, som blefve indckaldede, denne paaberaabte Marcke gangs Forretning og \textit{Bræche Mænd} Svarede at de iche havde den nu men Skulde See at Faa den fra forrige Eiere eller øfrigheden af \textit{Justise Protocoln}, og der af da til Sænde \textit{Major Schnitler} een \textit{Copie} der af, eller leveere dend til \textit{Commissions Secretairen}. ‒\par
Til 10{de} 11{te} og 12{te} Svarer Som nest forige 9 viidne og der paa blef \textit{dimitteret}. ‒\hspace{1em}\par
11{de} Viidne.‒\par
Hans Nafn er \textit{Anders Thorsen Bræchen} er Føed i Bræcken, af bønder folck, i Røraas Sogn i guldahls Fogderie, 46 aar, gift, har 2 børn, boer i een af bræckegaardene, Nærer Sig af Gaarden og Værckets brug ‒.\par
Fra 1{te} Til 12{te} Spørsmaal \textit{Jnclusive} Svarer hand i alle Poster lige som nest forige viidne \textit{Johannes Nilsen Bræchen}. Og Som hand indtet viidere i Nord vidste at Forcklare blef hand \textit{dimitteret}. ‒\par
Efter at dette var forrættet fremckom, endnu ett viidne, der vidste noget om grændse Fieldene, og var borte i Skougen i hands Arbejde i Morgis dend gang Rætten begyntis; hvor fore hannem af Rætten Skeede forcklaring om Eeden og een formaning at Siige Sin Sandhed om dett hand kunde blive tilspurt, for Saa vidt ham der om vitterligt var og derpaa aflagde sin \textit{corporlig} Eed.\hspace{1em}\par
12{te} Viidne ‒\par
Hands Nafn er \textit{Anders Jversen Borgoesen} Føed paa \textit{Røraas Hytte}-Platz, af een bergs Mand Sammesteds, 55 aar gammel, gift, uden børn, boer i Borgoesen ved Bræchen, næhrer Sig af gaarden og Værckets brug. ‒\par
Til 1{te} 2{det} og 3{die} Spørsmaal Svarer det Samme som det 10{de} Viidne Johannis Nilsen Bræchen. ‒\par
Til det 4{de} igien tager hand det Samme Som bem{te} 4{de} viidne, og legger der til, at Norden for \textit{Glee}-Fieldet og \textit{Dahlvaalds} Fieldet ligger een Dahl ved Nafn \textit{Hydd Krochen}; Denne dahl begynder i øster, øst for Dahlvaals-Fieldet, hvor dend grændser til dend Svenske Dahl, Kaldet \textit{Grøndahl}, og Stræcker Sig væster, imellem Dahlvaahls Fieldet og \textit{Liusen vohla}, eller Saa kaldet \textit{Haftor}-Stødten, og naar den er kommet forbi haftor-Stødten, udbree\hypertarget{Schn1_13036}{}Schnitlers Protokoller I. der dend Sig j Nord Nembl: paa det Stæd, hvor \textit{Hydd}-Elfven Kroger Sig fra Søer i væster, og Siiden gaar Samme dahl Saa breed neer i væster til øre-Sund; i denne dahl er Myhr græs Somme Stætz, Som Slaaes om Aaret; Jgiennem denne dahl \textit{Hydd Krochen}, Rinder een Elfv kaldes \textit{Hydd}-Elfven, hvilcken \textit{Hydd}-Elfv, udSpringer af een Søe, \textit{Hydd}-Søe kaldet, der er een 1/4 lang og breed, der af Rinder denne Elfv 1/4 Miil først i Søer, Siiden kroger dend Sig i væster og løeber 1/4 Fierding vejs i dend Søe, \textit{Hyllingen} der er 1 miil lang i væster, og eett bøsse Skud Breed, der af gaar Siiden denne \textit{Hydd}-Elfv 1 miil vejs omtrent i væster ud i \textit{øresund} Søe. Deraf, at denne \textit{Hydd}-Elfv Kroger Sig fra Søer i væster, har denne Dahl Sit Nafn, at den heeder \textit{Hydd Krochen}. breeden af denne dahl jmellem Dahlvaals Field og Haftor-Stødten, kand være omtrent 1: Miil; Lang kand dend være fra \textit{Grøndalen} neer til øresund vel en 3 Miile; ‒\par
Norden for denne Dahl \textit{Hydd Krochen}, ligger det Field \textit{Liusenvohla} eller Saa kaldet \textit{Haftor}-Stødten; dette \textit{Liusenvola} er Rund, og paa dend Syndre og væstere Sambt øestere kandt er dend høy, men paa dend Nordre kandt Sludter dend need ad \textit{Hydd}-Søen og Siges dend at være 1 Miil Rundten omkring; Landskabet paa dend væstere Siide ad \textit{Liusenvola} er, som Sagt \textit{Hydd} Elfven, der paa følger i væster nogle Smaa Fjeld\textit{vohler}, og omsiider \textit{Hydd Kroch}-Dahlen. ‒\par
Landskabet østen for \textit{Liusenvola} er een Dahl, kaldes \textit{Grøndal}, Som hører de Svenske til, denne \textit{Grøndal} ligger imellem \textit{Haftor} Stødten og \textit{Glee}-Fieldet en 1/2 miil breed og Stræcker Sig østen for det haftor Stødt. Østen for denne \textit{Grøndal} ligger de \textit{Herjedalingers Liusendal}, Som er een lang dahl og Stræcker Sig fra Nord i Syd ost een 2 Miile need til det \textit{Liusendalsche} Kaabber værck [.] Norden for dette \textit{Liusenvohla} ligger \textit{Hyll} Søen og længer i Nord er hand iche beckiendt. ‒\par
5{te} Spørsmaal Cessat: ‒\par
Til det 6{te} Svarer de høyeste \textit{Rutt} Klimper og \textit{Haftor}-Stødten, har hand hørt at være Rætte Skille-Mærcker ‒\par
Til det 7{de} Svarer Som næst forrige viidner, derhos Siigendes, at hand og hands formænd har brugt dend Dahl \textit{Hydd Krochen} til at Slaa i, og dend Søe \textit{Hyllingen} at Fiske i, ubehindret af de Svenske eller af nogen anden; ligesom de Svenske derimod have brugt, og bruge \textit{Grøndalen} og \textit{Liusendalen} i øster. ‒\par
Til 8{de} Svarer det Samme Som nest forrige.\par
Til 9{de} 10{de} 11{te} og 12{te} Svarer Som nest forrige viidner: og derpaa blefv hand bortladet; og Forrætningen paa dette Stæd for denne Gang Sluttet. ‒ \hspace{1em}\centerline{Peter Schnitler (L. S.)}\centerline{Jørgen Olsen hougen Bræcken. (L. S.)}\centerline{Anders Jørgensen Bræcken. (L. S.)}\hspace{1em}\par
Siiden her ingen kongelig Civil betiendt for uføre Skyld har kundet til dette Stæd hidckomme, Saa i Kraft af deris Kongl: May{ts} allernaadigste befalning har \textit{major Schnitler}\hypertarget{Schn1_13293}{}Relation om bygdene Eire, Heen og Zerne. betydet bræcke bønderne Saa mange Som hos hand vare forsamblede, og efter ladt een Skriftlig \textit{ordre} til Joen Henningsen bræcken Der Er een af de beqvæmmiste her, at ved til Stundende Junj Maanets udgang, da de kongel: Norske Grændse Maalere formeentlig bebegynde deris forrætning, hand da om ingen Lænsmand Skulde her møde, med de her afhørte viidner Skal indfinde Sig hos grændse Maalerne, Saa Snart hand hører de nærmer sig hiid imod \textit{Rutten} paa \textit{Wigeln} eller \textit{Waattaa} Fieldet ‒\par
2: Maa hand have med Sig diid de kyndigste Mænd af hvert Stæd, for at veiviise grænse-maalerne og at underrette dem hvor omtrent de østlige grændse Fielde, og hvilcke de ere\par
3: Hvor der er Elfver eller Søer hvor grændse Maalerne Maa over, skal hand være betænckt paa at- og hvorleedis af de nærmiste Skouge broer, klopper eller Flotter forfærdiges eller baater, hvor de ere, tilReede holdes, at de kand være i tiide færdig og veed haanden, uden ophold at over sætte de andkommende grændse Maalere paa, og med have arbejds Folck til at rødde hvor det giøres fornøden til grændse gangen, og at hielpe at drage kieden, og i det øfrige være dem \textit{assisterlige} og give dem anviisning: hvor de paa Saa vidt afliggende øede Stæder beqvæmmeligst kand faae for betalning underhold for deris Persohner og beedte for hæstene fra, og det alt til deris Kongl: May{ts} tieniste og grændse \textit{Commiss}: Befordring. befallendes dig gud \textit{Bræche Gaard} d: 21{de} Apriil 1742{ve}. {Peter Schnitler.}\par
P:S: Dersom Lænsmanden kommer leveerer du ham denne \textit{ordre} til efterleevelse ‒\hspace{1em}
\DivII[Liste over bilag]{Liste over bilag}\label{Schn1_13356}\par
Følger nu Bielagerne\label{Schn1_13360} \par 
\begin{longtable}{P{0.22285902503293809\textwidth}P{0.6271409749670619\textwidth}}
 \hline\endfoot\hline\endlastfoot \hypertarget{Schn1_13363}{}\footnote{\label{Schn1_13363}Trykket i forordet}Lit: A:\tabcellsep Dend Kongel: \textit{ordre} til \textit{Major Schnitler} af 16 \textit{martj} Sidstleeden, er før bekiendt.\\
\hypertarget{Schn1_13383}{}\footnote{\label{Schn1_13383}Trykket i forordet}Lit: B:\tabcellsep Hr. \textit{oberste Rømlings Jnstrux} til \textit{major Schnitler} af 31 \textit{dito} nest efter, grundes derpaa og vides før.\\
Lit: C:\tabcellsep Jndhændtet \textit{Relation} om de 3{de} bøjder \textit{Eire, Heen} og \textit{Zerne}\\
Lit: D:\tabcellsep jndhændtet \textit{Relation} om dend Svenske \textit{Provintz Herjedalen}\\
Lit: E:\tabcellsep om Guldahls Fogderie i Norge, grændsende til \textit{Herjedalen}\\
Lit: F:\tabcellsep Om \textit{Røraas} Kaabber Værck.\\
Lit: G:\tabcellsep Skille-mærcker imellem Trondhiems og Aggerhuus Stift paa denne Kandt. ‒\end{longtable} \par
 
\DivII[Bilag A og B, se innledning, s. XXVI-XXXIII]{Bilag A og B, se innledning, s. XXVI-XXXIII}\label{Schn1_13475}\par
[Bilag A: s. XXVI f.]\par
[Bilag B: s. XXVIII f.]
\DivII[Bilag C-G]{Bilag C-G}\label{Schn1_13489}\par
Lit: C: \centerline{\textbf{Relation.}}\par
Om De 3: Nu Svenske Bøjder \textit{Eire, Jrre} eller \textit{Jdre} (2) \textit{Heen} og (3) \textit{Zerne}, Som ligge Synden for \textit{Herjedalen}, Stræckende Sig væster til imod \textit{Femund} Søen j Norge, hvor paa Kong \textit{Frederik} den III i Sit \textit{Manifest} af A{°} 1658 giorde \textit{Prætention}.\hypertarget{Schn1_13550}{}Schnitlers Protokoller I.\label{Schn1_13552} \par 
\begin{longtable}{P{0.6833333333333333\textwidth}P{0.044444444444444446\textwidth}P{0.044444444444444446\textwidth}P{0.07777777777777777\textwidth}}
 \hline\endfoot\hline\endlastfoot \textit{Eire, Irre} eller \textit{Jdre} Skal bestaa af \textit{Familier} ‒\tabcellsep een\tabcellsep 15\tabcellsep bønder\\
\textit{Heen} bøyd\tabcellsep af\tabcellsep 14\tabcellsep »\\
\textit{Zerne}\tabcellsep af\tabcellsep 28\tabcellsep »\\
\multicolumn{4}{l}{____________}\\
i alt\tabcellsep =\tabcellsep 57\tabcellsep bønder\end{longtable} \par
 \hspace{1em}\par
De ere haarde u-frugtbare Field-bøyder: j \textit{Eire} voxer indtet og i \textit{Heen} og \textit{Zerne}, som ligger østerligere undertiiden korn, og maa Folcket leeve mest av Skytterie i Fieldene og Fiskerie i Field Søene, hvor af Landet er overflødigt.\par
Grændserne af disse 3{de} bøjde Lauger til deris Naboer gaae ligesom i en Triangel: østerst i dend Saa kaldede Tolfv Miil Skoug er een Kors-kiilde, saa kaldet af nogle Træckors, som i kiilden ere nedStuckne; og Her er det hvor \textit{Zerne} bøjd i øster begynder; Denne Kors-kiilde er Odden eller Spidzen af \textit{Triangeln}, hvor fra disse bøjders grændzer udviide Sig i væster, Og Skal denne Kiilde giøre Skilnet fra \textit{Herjedalen} i øster og Nord, og fra \textit{Siljesdahlen} i \textit{Dahle-Karlien} i Sønder, grændserne paa dend Nord[r]e Siide til \textit{Herjedalen} giør (1) denne Kors-kiilde (2) længere i Væster \textit{Weiing-} eller efter Norsk Tale Maade Veidefieldet, liggendis imellem \textit{Zerne} og \textit{Herjedalen} og (3): \textit{Fiedt}-Siøe Volan, et bierg, liggendis imellem Fiedt-Søerne, Som Skille \textit{Eire} fra \textit{Herjedalen;} Og kunde af \textit{Referentens} Sagn Sluttes, at \textit{Linien} fra denne Fiedt-Siøe-Volan Skulle gaa over \textit{Gruveln} i væster need til Soller øen i \textit{Femund}-Søen: men \textit{Linien} til Lands Stræcke Sig meere i Nord væst ad \textit{Svuckue} Fieldet.\par
Grændserne paa den Søndre Siide af disse 3 bøjder til \textit{Dale Karlien} have værit (1) forbem{te}Kors-kiilde (2:) Troulle-grav, ell: Tiuru-grav, en 3 Miil fra Kors-kiilden beliggendis, Som har Sit Nafn af een Stoer Dyb grav, hvor Tiære fordum skal være brændt; (3) \textit{Ranolle}-bæck i væster, (4) \textit{FulluMielje} i væster (5) \textit{Waadals} Field j væster: dog alle Mercker Stræcke Sig liidet i Søer, Liige Som bøydernis \textit{teritorium} udbreder Sig i væster; Dette \textit{Waadals}-Field skal ligge nær \textit{Fullu} Field Syndenfields og Saa godt Som hænge dermed Sammen. ‒\par
Dette blev af een \textit{Zerne}-bonde paa \textit{Røraas} hos \textit{Directeur Borgrevink discursive} berettet. ‒\par
Det 6{te} Norske Viidne Paa \textit{Koyen}, \textit{Knudt Olsen Ferragen}, \textit{examineret} d: 18de \textit{Apriil:} 1742: har herom tildeels og viidnet, at Skilnet imellem disse bøjder og \textit{Herjedalen} Skal være Kors-kiilden i øster, derpaa i væster \textit{Vætta} Field (hvilcket Nafn Synis at komme overeens med \textit{Veiings}- eller \textit{Weide}-Fieldet) der efter i væster \textit{Næs-Fiæld}, \textit{Lill}-Field og \textit{Sluggu} Field, her fra i væster \textit{Wonsiø-gusten, grøttaadalen} i Nordvæst, Siden øst og Nord om \textit{SvugKue}-Fieldet en Sløgd, eller Fielddahl imellem bem{te}\textit{SvuKue} Field og \textit{Røvola}, hvor een kiern, Stoer \textit{øre}-kiern ligger, hvilcket viidnet har af en Svænsk karls Sigende, og man viidere af \textit{Acten} Samme Stædz fornemme kan. ‒\par
Een karl fra \textit{Eire}, der tienner hos Klockeren paa \textit{Røraas}, Joen Povelsen, Sagde mig \textit{Relative}; at i Syd ost fra \textit{Kraltvola} ligger ett Field \textit{Brunsdahls Kletten}, 1/4: miil fra Nord \hypertarget{Schn1_13902}{}Relation om bygdene Eire, Heen og Zerne. til Sønder foed. Dette Skal i væster henge Sammen med dett Field \textit{grøthaangna;} Fra \textit{Brunsdals}-Kletten i Syd oest følgge de Fielde \textit{Brottene-} ellers og kaldet \textit{GruvelSiøe}-Fieldene eller \textit{LangField}, Som hænge Sammen med Brunsdahls-klætten;\par
Disse Stræcke Sig fra Nord væst i Syd ost 2 mille over til det Field \textit{StorWitterhogna} der er fra Nord i Søer 1 1/2 Fierding-miil over. Østen for Witterhogna er een Dahl 3: miile lang, bestaaende af Myhrer, Jord houger med Furru og Gran bevoxene. ‒\par
Sydvæst fra \textit{Brottene} Skal ligge \textit{Gruvel}-Siøn, langs ved \textit{Brottene}, og Stræcker Sig i Syd ost 5/4 miile lang, hvor Røe og øret fiskes; Af denne Søe \textit{Gruvel} Søe Riinder een Elfv, kaldis \textit{Gruvla}, beent i Syd ost til \textit{Eire} bøjd, og løber viidere i øster igiennem \textit{Zerne} bøjd til \textit{Elfdalens} kircke-Sogn, og derfra igiennem \textit{Mora}-Sogn i Søer ind i Siøen \textit{Silien} i \textit{Siliensdalen}, hvor af \textit{Siliesdalen} i \textit{DaleKarlien} har Sit Nafn; Disse Fielde \textit{Stor Witterhaangna, Brottene, BrunsdalsKletten} og halften af \textit{Kraltvola} høre \textit{Eire} bøjd til, hvor de have deris Skytterie og Fiskerie. J Skougen ligger \textit{Steien}, som er et liidet Field mest i øster fra Storvitterhongna. Nordre Sommer-vej gaar fra \textit{Eire}, igiennem Skoug Land 1 1/2 miil i Nord væst op ad et Field Storvitterhongna, ett Slet maaset Field, 1 1/2 fierding miil over fra Søer i Nord; Dette Storvitterhongna hører \textit{Eire} mænd til, som have østen der for deris Sæter boliger. Fra Storvitterhongna gaar Vejen langs efter \textit{Lang} Fieldet eller \textit{Brottene} 1: Miil lang; der fra igiennem \textit{grottaadalen} 1 1/2 miil lang; dend halve Søndere deel, som denne \textit{Eire}- karl berettede, af denne \textit{grøttaadalen} hører \textit{Eire}, og den anden halfve Nordre deel hører \textit{Funnesdalen} j \textit{Herjedalen} til, og er Skille Mærcket i mellem \textit{Eire} bøjd og \textit{Funnesdalen} det Field \textit{Kraltvola}, liggendis østen for \textit{Grøttaadall}, fra \textit{grøttaadal} fare de over \textit{Røa}-Elfv til dend Nordre \textit{Femunds} viig, hvor dend Norske bonde \textit{Lasse Femund} boer; Jmellem \textit{Røen} og Nordre viig er noget lidet, dog Slet hæste-beete, og fra Nordre viig viidere Rejse de til \textit{Røraas}. ‒\par
Den anden Søndre Sommer-vej fra \textit{Eire} fares igiennem Skougen til \textit{Lang} Fieldet, Siiden til \textit{Sal}-Fieldet Sydvæsten for \textit{Gruvel} Søen, dernest igiennem en Dahl, \textit{Ruvling} dahl, frem ved \textit{Elgaahongna} paa dend østere Siide, hvor græsgang er, over \textit{SvuKue} field, til \textit{Røa} Elfv og viider til Nordre \textit{Femunds} viigen.\par
Viinter vejen fra \textit{Eire}, gaar meere Sydlig i væster til \textit{Gutu} Søen og Field-gutu Søen, \textit{Elgaa} Søen, og \textit{Elgaa} dahlen, Siiden langs \textit{Femund} Søen til \textit{Femunds} Syndere Viig; og derfra til \textit{Røraas}; eller og fra \textit{Elgaaen} til \textit{Tufsingdalen} i \textit{Tolgens} gield og viidere til \textit{Røraas}. ‒\par
Denne vej fra \textit{Eire} bøyd til Nordre \textit{Femunds} viig hvor \textit{Lasse Femund} i Trondhiems Stift boer, er en 8 Nye Maalte Miile lang, og der imellem boer ingen Folck, og fra Lasse Femund til første 4 Dahls gaarder Hvor forbie Sommer-vejen gaar til Røraas, er 2 1/2 miil Saa dend vej er meget vanskelig, og Saa godt Som \textit{upassable} for en hob Folck Særdelis med hæste, for mangel af \textit{Proviant} og \textit{fourage}, og for Landskabets vanskeligheds Skyld. ‒\par
J \textit{Zerne} bøjd, Som dend østerligste, og af de bøjder dend beste, boer Præsten, og er der fra til \textit{Heede} 1 miil fra \textit{Hede} til \textit{Eire} er 2 miil, fra \textit{Eire} til \textit{Femunds} Søen 6 maalte miile, og berettes, at Kongel: Norske grændse maalere i forleeden Sommer med deris Maaling have gaaet fra \textit{Fuluguta} Field Søndenfields i Nord igiennem \textit{Eire}-mændz formeenende \hypertarget{Schn1_14302}{}Schnitlers Protokoller I\textit{district} til \textit{Fulugutavola}, Som Skal være en 4{re} miile østen for \textit{Femund} Søen og 2 miile væsten for \textit{Eire} bøjd. ‒\par
Betydenheden af disse 3: bøjder er i henseende til jndkomsterne og Landets beskaffenhed kun Slet og Ringe, Thj (1) voxer der i \textit{Eire} indtet og i \textit{Heede} og \textit{Zerne} Sielden Korn, (2) Jndbyggerne have fordum ydet deris Skatt i dyr-Skiind \textit{in Natura:} dog nu i penge, som er ringe, Saa at Sverrigs Crones jndkomst af disse bøjder er mindre, end deris beckostning paa betiendtere; (3) Thj Cronen lader aarlig til Præstens underholdning give 50 tønder byg, Som leveres fra Nærmeste \textit{mora}-Sogn i \textit{Silies} dahlen, der Skal være et korn land; Der som nu fra Norges Siide Saadant korn til Præstens ophold endelig \textit{in Natura} maatte gives, Saa ere de nærmiste Norske Præstgield \textit{Tolgen} og \textit{Tønset} i væster, og \textit{Holtaalens} gield i Nord væst Lande, der trænger Selv til korn at kiøbe, og Skulle da guldbrandsdalen, Som derfra er Saa langt afliggende, vel være det Nærmiste Korn-land, at \textit{fournere} Præsten fra med korn. Disse 3 bøyder \textit{Sorterer} under \textit{Dalernis}, eller \textit{Dalekarliens} Landshøfding, der \textit{Residerer} i \textit{Fahlun}. \textit{Referenterne} af Forestaaende have værit mest \textit{Eire} og \textit{Zerne}-mænd, jeg har treffet paa \textit{Røraas}.\par
Lit D: \centerline{\textbf{Relation} om \textbf{Herjedalen}, en Svensk \textbf{Provintz}.}\par
Har Sit Nafn af een Aae kaldis \textit{Herje-aaen} der løber i dend Elfv \textit{Lussnen} igiennem det østerligste \textit{Annex, Lille Herdal}; paa dens østere Side ligger \textit{Helsingland;} paa dend væstere guldahlen i \textit{Trondhiems} Stift, paa den Søndere \textit{Zerne, Heede} og \textit{Eire}, og paa dend Nordre Siide \textit{Jemteland}. Landet Stræcker Sig i Syd oest. ‒\par
Det bestaar af 2{de} Præstegield, \textit{Heede} og \textit{Sveeg}, under \textit{Heede Sorterer} (1) \textit{Heede} hoved Sogn (2) Tennes \textit{Anex} væsten derfor, nærmist de Norske grænser, og (3) \textit{Wimdalens Annex} østen for \textit{Heede}; Under \textit{Sveeg} Svarer \textit{Sveeg}-hovet kierche, og østen der for \textit{Lille Herdals, Elveros} og \textit{Hodals Annexer}[.] Dette nu nærmere at forcklare: saa er fra de yderste Norske gaarder i guldahlen\textit{Bræche} gaardene 14 i Tallet, i øster op ad \textit{Rutt} Fieldet til dend første af de Svenske \textit{unilateraliter} opSatte Miil-Stolper imod 1 Miil; Fra denne Miil Stolpe i øster er til første bøyd i \textit{Herdalen, Funnesdalen} een {3 Miile}\par
Denne bøjd har Sit Nafn af een liiden færsk vands Søe kaldes \textit{Fonnen}, paa hvis Nordre Siide gaardene ligge, og bestaar den af een 14 bønder gaarder, foruden en 16{ten} Platzer, 1/4 Miil der fra i Nordost beliggende Som hører til det Svænske igien optagne \textit{Liusendals} kaabberværck, fra denne Svænske Miil Stolpe paa \textit{Rutt}-fieldet begynder Fieldet Strax at Dahle need til een Søe kaldet \textit{Malmagen}, Som kand være en 1/2 miil Stoer, og ligger fra Miil Stolpen paa Rutten en 1/2 miil vejs i øster; fra denne \textit{Malmagen} Søe i øster ligger \textit{Funes} bøjd en 1 1/2 Miil. Fra Funnæsdahlens bøjd til første \textit{annex} Kircke Som er kun liiden og Som et \textit{Capell} kaldet \textit{Tennes} Kiercke er {2 1/2 miil.}\hypertarget{Schn1_14602}{}Bilag D: Om bygder i Herjedalen.\par
Bøjden derved kaldes \textit{Tennes} bøjd og ligger fra \textit{Fonnesdalen} i Syd Syd ost, bestaaendes af een 11 bønder gaarder, Som ligge vel adspreed, dog hinanden Tæmmelig nær.\par
Denne \textit{Tennes} bøjd og kiercke har Sitt Nafn af \textit{Tenna} Elf, der har Sit første udSpring af \textit{Glee}Søen ved \textit{glee}Field, østen for det Field \textit{Dahlvohla}, og Rinder i Søen i \textit{Malmagen} Søe østen for \textit{Rutt} Fieldet, og der fra i \textit{Lossen} Søe paa dens væstere Siide, forbj \textit{Tennes} Kiercke paa dend væstre Siide liigesom en anden, nembl: \textit{Lusna} Elfv, der opckommer ved \textit{Helax} Fieldet, rinder i Syd Sydost, paa dend østere Siide forbie Samme \textit{Tennes} Kircke, faldendis udj \textit{Lossen} Søe, ved dends østre Ende jmellem hvilcke 2{de} Elfver \textit{Tennes} Kircke ligger omtrent 1: Miil i Nord fra \textit{Lossen} Søe, Naar Elfven af bem{te}\textit{Lossen} Søe igien udløber, har den kun det Eene Nafn \textit{Liusna} og rinder derefter østlig igiennem heele \textit{Herjedalen} og \textit{Helsingland} ud i dend \textit{Bothniske} Søe.\par
J dette \textit{Tennæs annex}, Nordostlig l/4 miil fra den bøjd \textit{Funnesdalen} ligger ett kaabber værck, Som for en 60 aar var først bearbejdet, men for \textit{Malmens} Sledthed blev needlagt; Nu for en 4 aar er det igien optaget, og driives kun af een Snees Mand baade \textit{Hytten} og \textit{Gruben}, liggendis ved \textit{Liusna} Elfv dette ny værck er iche af nogen betydenhed, thj først er \textit{Malmen} Slet og \textit{Ejsen-Sküssig}, der næst er \textit{Transporten} af \textit{proviant} fra \textit{Fahlund} i \textit{Siliesdalen} eller fra \textit{gevle} i \textit{gestrike} land didhen, og af Kaabberen der fra til \textit{gevle}, Som Skaligge derfra en 45 miile, meget kostbar og Vanskelig. ‒\par
J dette \textit{Tennæs Annex} voxer Sielden korn, men Folcket nærer Sig af deris Foederland, Skytterie og Fiskerie, af øret, Røe, harr, Siig, Abborr og Gedder i Field vandene.\par
Fra \textit{Tennes} Kircke i øster gaar vejen forbj \textit{Langaa} Skandseen{3 1/2 mil}\par
At herimellem ingen kiercke er, der til er Aarsag, at Landskabet der, er u-frugtbar, Myhre og ubeboet; dog skal der være en god Furru Skoug. ‒\par
Denne \textit{Langaa} Skandse var til i kriigens tiid, byget af Tiimmer, havendes paa dend Søndere Siide \textit{Liusna} Elfv, og paa dend Nordre Siide Var een Dam ved een Bæck opckastet, Saa der var vand paa begge Siider: Dog Skal paa dend Nordre Siide af Skandsen nær der ved ligge ett højd berg, fuld af \textit{Moratz}, hvorunder Skandzen er beliggende. ‒\par
J Kriigens tiid laag ett \textit{Compagnie} Soldatter der; i \textit{gvarnison} Som tiid efter anden blef afløst af \textit{Comanderte} Folck, thj \textit{Herjedalen} holder ingen udskreeven \textit{Militz}: dog nu er denne Skandse forfalden og ingen Folck derj ‒\par
Østen for \textit{Langaa} Skandze begynder \textit{Heede} hoved Sogn og er fra Skandzen til denne hoved kircke 1 1/2 Miil: Dog bestaar Landskabet her af, Skouge og heeder, og er iche Synderlig bebygget, førend man kommer nær til hovet Kircken \textit{Heede}, hvor een Samblet bonde bøjd er, Nafnlig \textit{Heede}; Dette Sogn haver omtrent en 40 bønder gaarder, mest paa dend Søndre Siide af \textit{Liusna} Elfv, og \textit{Sorterer}\textit{Wiigen} bøjd, mest Norden for \textit{Liusna} 1: miil fra kircken liggendes, herunder, Som har kun en 4: eller 5: bønder gaarder; ‒\par
J Nord væst fra \textit{Heede} hovet kircke Norden for \textit{Liu(s)na} Elfv, vejen til Jemteland under \textit{Klefsiø}-Field ligger \textit{Heedes} andet \textit{Annex} kaldet \textit{Wimdalen}, hvori en 50 bønder gaarder Skal være men Som dend \textit{ordinaire} landevej, gaar igiennem \textit{Wiigen}, Langs efter \textit{Liusna} Elf, Saa reignes fra \textit{Heede} Kircke til Wiigen{1 miil.}\par
Fra \textit{Wiigen} kommer man til een liiden bøjd, kaldet \textit{Ransiø}, af nogle gaarder 2 1/2 miil.\hypertarget{Schn1_14943}{}Schnitlers Protokoller I.\par
Dette \textit{Ransiø Sorterer} under ett andet Præstegield, nembl: \textit{Sveeg} hovet kircke. ‒\par
Fra \textit{Ransiø} reises til en bonde bye, kaldet \textit{Glisseberg}, af en 10 bønder gaarder 1: miil, herfra til \textit{S(v)eeg} hovet kirche, hvor vel Saa mange bønder gaarder, Som i \textit{Wimdalen} findis, er {1 1/2 miil;}\par
Fra \textit{Sveeg} hovet kircke reises i Sydost til dets \textit{Annex lille Herjedal}, hvor en 16 a 18 bønder gaarder ere {2 miile,} Fra \textit{Sveeg} i Nordost ligger et \textit{Annex} af en 18 gaarder, \textit{Elveros}, og der fra i øst \textit{Hodals Annex} liige Saa Stoer.\par
Jmellem \textit{lille Herjedal} og \textit{Helsingland} ligger en Skoug {1 1/2 miil.}\par
Midt i denne Skoug ligge 2{de} bønder gaarder \textit{Wensiø} kaldet, og dette er Skillnet imellem \textit{Herjedalen} og \textit{Helsingland}.\par
Fra \textit{lille Herjedals Annex} i Søer gaar vejen til \textit{Siliesdalen} i \textit{Dalekarlien}, og der imellem ligger en Skoug af 6 miile lang: men fra \textit{Sveeg} hovet Sogn gaar vejen Nordlig til \textit{Helsingland}. ‒\par
J \textit{lille Herjedahl} har i gammel tiid ligget en Skandze imod \textit{Helsing} land, Som nu gandske er borte. ‒\par
Dette land \textit{Herjedalen} er Knægt-frii, og have ingen Soldatter eller udskriving, fordj det er et Svagt land paa korn: dog maa de derfor yde Penge, Som Siges, 900: daler Sølvermynt aarlig Deris \textit{Gouverneur} er Landshøfding \textit{Odelstrøm, Residerendis} i \textit{Gevle}, Som har under Sin befaling \textit{Gestriche}-Land, \textit{Helsingland, Jemteland} og dette \textit{Herjedalen}. ‒\par
Vejene fra \textit{Herjedalen} til Norge ere (1) den \textit{ordinaire} Landsvej over \textit{Rutt} Fieldet til de første \textit{Bræche}-gaarder i guldalen: dog er den om Sommeren iche at køre i \textit{Chaise}, eller med vogner, men bare at riide og gaa; hvilcken vej og de Svenske med Smaa \textit{Partier} i Kriigs tiider taget have, og A{o} 1718 et Partie Svenske ved deris udgang ud af Norge beholden \textit{passerede:} mens at noget Stort \textit{Corps} fra enten af Siiderne Skulle tage denne \textit{passage}, er iche vendteligt; thi for det første er vejen Steenede, Myrede, af een hoben Søer, Elver og \textit{Præcipicer}, besværlig, saa at det vilde falde meget vanskeligt og koste megen tiid og arbejde at føre med Sig tilstræckelig \textit{Proviant} eller tung \textit{Bagage}, for det andet er \textit{Herjedalen} i Sverrig, og det Nærmiste Norske Præstegield i guldalen, (naar \textit{Røraas} værckets\textit{proviant} huus i tiide bortskaffes, eller er udtømmet) Svage korn lande, hvor ingen Madvahrer er at faae til \textit{subsistance}. ‒\par
Dend 2{den} vej meere nordlig fra \textit{Herjedalen} er igiennem \textit{Liusendalen} i \textit{Herjedalen} over det Norske field \textit{Skardøren} need i den Norske bøjd Stuedahlen i \textit{Tydalen}, dog Som dette Field er en 5 Nye Miile over, og haardt, Saa fares den vej Sielden, og det kun om Someren med Enckelte heste, icke om viinteren for det haarde vejer der tager. ‒\par
Dend 3{die} Vej mest Nordlig tages fra \textit{Jongdalen}, 2 gaarder i \textit{Herjedalen}, liggendis fra bem{te} Norske Field \textit{Skardøren} i øster en 5 miil Nye Reigning; denne \textit{passage} tages nu Sielden, og det kun om Sommern; Thj til denne \textit{Passage} at fare, ere kun forbenæfnte 2 gaarder, og der fra i øster 1 1/2 Miil, Stoer Siø-bøjden, af en 5 gaarder, hvilcke \textit{Herdalsche} gaarder ligge mest nordlig hen ad Jemtelands grændser, og hører til \textit{Heede} hoved Sogn, hvor til de første, Nembl: \textit{Jongdalens} gaarder have en 8 gamle, eller 4 Nye Stærcke Miile. ‒\hypertarget{Schn1_15269}{}Bilag E: Om Guldals Fogderi.\par
\textit{Referenterne} heraf om \textit{Herjedalen} have værit Saa vel Norske bønder ved grændserne boende, hvilcke did have Faret, Saa og \textit{Herdalingerne} Selv, Som jefnlig fare hid væst over at kiøbe Sild og Fisk, dem ieg har Treffet og Talt med. ‒\par
Lit: E: \centerline{Om Det Fogderie \textit{Guldahlen} i \textit{Trondhiems} Stift.}\par
Dette Fogderie har Sig Nafn af \textit{Gul}-Elfven Som løber derimellem;\par
Det Stræcker Sig fra øster i væster, og Sist Nordlig; Er lang at reigne fra øster:\hspace{1em}\label{Schn1_15310} \par 
\begin{longtable}{P{0.3670061099796334\textwidth}P{0.3929735234215886\textwidth}P{0.051934826883910386\textwidth}P{0.03808553971486761\textwidth}}
 \hline\endfoot\hline\endlastfoot Fra høyeste\tabcellsep \textit{Rutten} Field til bræcke gaard\tabcellsep 1\tabcellsep Miil\\
her fra til\tabcellsep \textit{Røraas} borgstad\tabcellsep 3\tabcellsep »\\
til\tabcellsep hof \textit{annex} eller \textit{aalens annex}\tabcellsep 3 3/8\tabcellsep »\\
til\tabcellsep \textit{holtaalens} hoved Kircke\tabcellsep 1 1/2\tabcellsep »\\
til\tabcellsep Singsaas \textit{annex}\tabcellsep 3\tabcellsep »\\
til\tabcellsep Størens hoved Kircke\tabcellsep 3\tabcellsep »\\
til\tabcellsep \textit{Horrigs annex}\tabcellsep 1\tabcellsep »\\
til\tabcellsep Meehl-hus hovet Kircke\tabcellsep 2\tabcellsep »\\
\multicolumn{4}{l}{___________}\\
\tabcellsep \tabcellsep 17 7/8\tabcellsep Miil\\
\multicolumn{2}{l}{herfra Siiden gaar vejen igiennem et andet det \textit{Strindensche} Fogderie til Solberg i Liinstrandens \textit{Annex}}\tabcellsep 1\tabcellsep »\\
\multicolumn{2}{l}{her fra til \textit{Trondhiem}}\tabcellsep 1\tabcellsep »\\
\multicolumn{4}{l}{___________}\\
\tabcellsep \tabcellsep 19 7/8\tabcellsep Miil.\end{longtable} \par
 \hspace{1em}\par
Dette Guldahls Fogderie omtrent fra Nord i Sør er en 4 Miile, meer og mindere breed; og næsten paa alle Siider med Fielde omgiivet. har 4: Præstegield (1) \textit{Røraas} bergstadz Kircke (2) \textit{Holtaalens} ‒ Som har \textit{Aalens}- og \textit{Singsaas Annexer} (3) Størens ‒ Som har Sognedalens og Horrigs \textit{Annexer}, og (4) Meehlhuus Præstegield, som har \textit{Flaa- Hølands} og \textit{Liinstrandens Annexer}. ‒\par
Landskabet i dette Fogderie bestaar mest af Fielde Elfver, Søer, bæcke, undtagen den trange Dahl hvor \textit{Gul}-Elfven igiennem, og bønder gaarder paa begge Siider ere. ‒\par
Omckring \textit{Røraas} berstad og i første \textit{Aalens annex} Som ligge til Fiels, voxer Snart aldrig korn; j \textit{Holtaalens} hoved Sogn og \textit{Singsaas Annex} forfryser Kornet ofte. ‒\par
J Størens og Meehlhuus Præstegield, er jorden Stærckere til Ager dyrckning: dog hvor gaardene af begge gield Særdeelis i Størens- Sognedahlens og Høelands Sogner fra dahlen, ligge afSiides op til Fieldz, der have de Samme Skiebne at kornet der tidlig affryser.\par
Med Eng og Græsgang er den \textit{district}, sæhr de saa kaldede Marcke- eller til Fieldz liggende Gaarder forsynede.\par
Skoug nær omkring \textit{Røraas} bergstad er mest udhugget og til værcket forbrugt; og \hypertarget{Schn1_15566}{}Schnitlers Protokoller I. maa jordarten ej være goed til Skoug voxter, Siiden den afhuggne Skoug ej voxer op igien. ‒\par
Fiskerie af øret og Lax faaes af Guld-Elfven og andre der i løbende \textit{Rivirer} og bæcke; J Field-Siøerne fanges øret, Røe, og Siig \textit{etc:} ‒\par
Guldalens jndbyggere nære Sig, Særdeelis \textit{Røraas} Sogn og \textit{Holtaalens} gield, af deris arbeide og \textit{leverance}, af Kuuld og bergs Veed til værcket, hvor til dennem deris havende Eng-land treflig til Nytte kommer, j det at de deraf holde hæste og kiør øxene, at bringe \textit{Materialierne} frem med.\par
J \textit{Størens} og \textit{Meelhus} gield, giøris og brug til \textit{Røraas}, dog mest med Førsel af \textit{proviant} til og af Kaabber fra Røraas til Trondhiem; desuden haves j Størens gield Saug brug med boord Kiørsel, og Sognedahlen (et \textit{annex} af Støren) giøre forbem{te} Førsel til og Fra Qvickne værck; Høeland (\textit{annexet} af Meelhus) Pleier at giøre Kuldbrug til \textit{Meldals} Kaabber værck, naar dette iche er i aftagende, som det nu er ‒ Foruden disse leiligheder har Størens Sogn og horrig \textit{annex} Sambt Meehlhus Præstegield, saa meget der af ligger need i Dahlene temmelig god Ager-land, naar Frosten dem iche overjiler, Som i afvigte 1741{de} aar skeede. Størens Præstegield bruger og at brænde Tiære, og Meelhus gield at køre Saug obord til ladestæderne. ‒\par
Elfver i guldalen ere de fornemste (1) GulElfven, dend udRiinder af een Søe kaldes Gulhaaen, 1/4 Miil Sønden for det Field \textit{grønwolen}, først i Sydvæst forbie dend den gaard \textit{Reuten} til Engen j \textit{aalen}, Siiden vender Sig i væster, indtager Mangfoldige andre Elfver og bæcke, og har ved \textit{Gul-oesen} Sit udløb i \textit{Trondhiems} Fiord.\par
Jblandt andre er Sogn-elf temmelig Stor, som opckommer af \textit{Jil-Kiøn} i Renneboe i ørchedahls Fogderie, hvor fra den efter en kort Krumning rinder fra Søer i Nord igiennem Sognedalen et \textit{annex} af Størens Præstegield Ved \textit{Størens} Præstegaard ud i \textit{gul} Elfven, og giver det \textit{annex}Sognedalen Sit Nafn. ‒\par
2: Den Stoere Elf \textit{Glommen} Rinder og igiennem dette Guldahlen: dog opckommer den længer i øster i Tydahlens \textit{annex} af en liden Siøe kaldet \textit{Eflingen} ved det Field \textit{Eflingvola} 1/2 miil østen for bem{te}\textit{grønvola} og en 1/2 miil Syd væsten for Stuedahls gaardene i Tydahlen; af denne Efling-Søe, løber den nu først i Sydvæst i den Søe \textit{Riiaassen}, derfra i øster i \textit{Riien}-Søe, liggende i \textit{Riidalen} (hvilcke Siiden nærmere beskriives) og udaf \textit{Riien} Søe vænder dend Sig i Søer, faldendis i Søe \textit{Øresund} dens Norde viigs østere Ende, hvor af den ved dend gaard \textit{Kuraasen} igien udgaar i Sønder, løbendes mest Sydlig forbi \textit{Røraas} gaard, 1/8 miil væsten forbj \textit{Røraas} bergstad, viidere forbj \textit{aas Annex} igiennem \textit{Tønset} bøjden, \textit{Lille} og Stoer-Elfdalen, \textit{Aamots} gield, og det øfrige af østerdalens Fogderie, og Falders over \textit{Sarps} Fossen forbj \textit{Friderichstadt} ud i Nord Siøen. ‒\par
J forrige Kriigens tiid, har østen for \textit{Røraas} Platz et par bøsse Skud været een liiden Schandtze opbygget af jord og Steen, hvor \textit{Comanderte} Folck laag i, for at afholde Fiendtlige Partier, dog nu er den forfalden; Tædt derved er Slagg-hougen af hytten Saa høy opvoxet At dend er høyere end Skandtzen; og kan den allestetz paa alle Siider omkring gaaes.\par
Kriigs Folck i Guldahlen, ere De nærmiste til grændserne: 76 mand Skiiløber-Sol\hypertarget{Schn1_15883}{}Bilag F: Om Røros Kobberverk. datter af \textit{Hottaalens} gield; og 2{de} nembl: de Størenske og Meelhusiske \textit{Compagnie}, for uden \textit{Dragounerne}.\par
Dette har ieg mest af eegen forfarenhed, Som ieg have mit \textit{Compagniets district} i \textit{Størens} gield. ‒\par
Lit: F: \centerline{Om \textbf{Røraas} Kaabber Værck, Liggendes fra Græntze Fieldet \textit{Rutten} 4{re} Miil i væster;}\par
Ved \textit{Røraas} bergsted ligger een liiden bondegaard \textit{Aas} kaldet, Som Skilles der fra kun ved en liden hage; j væst nord væst 3/4 miil fra \textit{Røraas} berg-Stad eller Platz ligger een Siø, nafnl: \textit{Røe}-Søe, hvor af en \textit{Røe} Elfv først i Søer Siiden krummer Sig i øster og løber ud i dend Stoere Elf \textit{Glommen}, hvor fra \textit{Røe} Siøe kan ligge en 1/2 Miil;\par
Paa dend Nordre Siide af denne \textit{Røe} Elfv, der hvor den falder ud i \textit{Glommen}, er en gammel bonde gaard, \textit{Røeros}, eller af dend højde dend gaard føst var bygget paa \textit{Røer-aas} kaldet; Til denne gaard blef nu i førstningen, da \textit{Gruben} blef optaget, \textit{provianten} henført og needlagt, og der af er det, at \textit{Røraas} Platz og værck har Sit Nafn: endog \textit{Røraas} Platz ej ved \textit{Røen} Elfv, men 1/4 miil derfra østen for \textit{Glommen} Elfv ligger. ‒\par
\textit{Røraas} bergstad eller platz have et par hundrede huuse, foruden \textit{participanternes} byggninger, en Smuck Kiercke, for en 92 aar bygget, hvortil henhører een Meenighed af Een 1400 \textit{communicantere} ‒\par
Værcket har een Smelte hytte paa \textit{Røraas} Platz af en 8. ovner, hvor imod 100 arbejdere holdes; den anden er \textit{Tolgens} hytte af 3 ofner i \textit{Tolgens} Præstegieldøsterdalens Fogderie Synden Fields, noget over 3 miile i Syd væst fra \textit{Røraas} Platz. ‒\par
Den 3{die} er \textit{Eede}-Smelte-Hytte i \textit{Aalens annex}, \textit{Holtaalens} gieldTrondhiems Ambt een 4 miile Nordvest i fra \textit{Røraas} Platz, hvor arbejdere \textit{proportionaliteter} efter ofnenes thal holdes. den 4 hytte omlegges i Sydost ved \textit{Femund}-Søen. ‒\par
\textit{Malm Gruber} har værcket 5. hvilcke yde Ertzen med Saa Stoer guds velsignelse, at efter Menniskelig \textit{apparence} vil for Skougen, end \textit{Malmen} komme til at \textit{desidereres}, helst dette haarde Landskab ej er beqvæm til Skoug vexter. ‒\par
Første \textit{grube} er Aar 1644: opfunden af een Skoug-bonde, boendes paa forbem{te} liden gaard \textit{Aas}: ved Nafn \textit{Hans olsen Aas} til hvis Erindring i kircken følgende \textit{inscription} paa een Tafle læses: med hans billede:\hspace{1em}\par
\centerline{Staa læser See den Mand, Som \textit{Storvardts G(r)ube} først opfand, der var een Aarsag med næst gud, At her nu læris \textit{Christi} bud, at disse deris Føede har her, Som tilforn een ørcken var, Priis gud, og ønsk dend gamle graa, \hypertarget{Schn1_16135}{}Schnitlers Protokoller I. at hands been Rolig hviile maae, at Som dets finder hundret aar, og Sexten Dertil leeved har, Saa efterckomme i Mange leed, kan fødis her og glædis ved; \hspace{1em}\textbf{Hans Aasen} føed \textit{A{o}}1557: død \textit{A{o}}1673. ‒ ‒ ‒ ‒ ‒ ‒ ‒ ‒ ‒ ‒ ‒ ‒ ‒ ‒ ‒ ‒ ‒ ‒ ‒ ‒ ‒ ‒ ‒ ‒ ‒ ‒ ‒}\par
Man hører og denne opfinderis afkom Som mangfoldig har formeeret Sig, boendes paa Samme deris Fædris gaard \textit{Aasen}, paa \textit{Stuedalens} gaard i Tydalen, og paa \textit{Røraas} Platzen, nyde Fortriin for andre og fordeelagtige Vilckaar i værckets Arbejde. ‒\par
\textit{Major Schnitler} har, ved et \textit{particulier} Bilag givet Hr. Obriste \textit{Rømeling} sin u-forgribelige Betænkning tilkiende; hvilket bilage Hr. Obriste til deres \textit{Excellence}, Hr. Geheime \textit{Conference-Raad von Holstein} haver indsendt, om dette Røraas-Verkes Skoug vedligeholdelse og opElskning til Verkets \textit{conservation} for Efterkommerne.\par
Lit G: \centerline{Skille-merker jmellem Trondhiems og \textit{Aggerhuus} Stift paa denne kandt:}\par
Fra øster i væster\par
(1) \textit{Femund} Søens Nordre kløv, hvor en Vare Staar.\par
(2) Derfra omtrent 1: Miil \textit{Steenfieldet}, hvor en liden vare er. ‒\par
(3) Derfra 1 miil til \textit{Ravn}berget. ‒\par
(4) Her fra 1 1/2 miil til \textit{Haasteenen} midt i \textit{Glommen}. ‒\par
(5) Der paa Skal følge \textit{Meeliie} kiønnene vel 1 1/2 miil fra haasteenen ‒\par
(6) Her fra gaar \textit{Limiten} til \textit{Farolshongnen;} som fra Meelie-kiønnene i væster ligger 2 1/2 miile, og findes i mellem Qvickne kaabber \textit{grube}, i det Syndenfieldske, og buedahlens bøjd i Størens Sogn i det Nordenfieldske. ‒\par
(7) Herfra gaaes til Kongs vare, som er, hvor \textit{ørchedals} Fogderie Endes, og Guldahls Fogderi begynder.\par
\textit{Autoritet} herfor haves: Græntze \textit{Commissariernes} af Dannemarch og Sverrig deris Forretning, da \textit{A{o}}1659: efter den Roskilske Freed Tronhiems lehn til Sverrig Skulle overgives, og fra \textit{Aggerhuus} Stift adskilles, Saa og det øfrige af \textit{directeur Borgrevinks} og Fleeres \textit{Relation}. ‒\hypertarget{Schn1_16345}{}Retten sættes i Selbo Fogderi.
\DivI[I Selbu fogderi: 6 vidner.]{I Selbu fogderi: 6 vidner.}\label{Schn1_16347}\par
\centerline{J \textbf{Selboe} Fogderie\textbf{Tydalens}\textit{Annex} af \textbf{Selboe Præste}-\textit{gield}}\par
Begyndt i \textit{Stuedalen}, 2{de} gaarder liggendes Tædt under \textit{Skar-døren} d. 25{de}\textit{Apriil} 1742{ve} og 26 \textit{ejusdem}\textit{Continuered} ‒\par
Paa \textit{Aasgaard} i \textit{Tydalen} d. 27{de} og 28{de}\textit{apriil} næst følgende ‒\hspace{1em}
\DivII[April 23. Fra Brekken til Stuedal]{April 23. Fra Brekken til Stuedal}\label{Schn1_16431}\par
A{o}1742{ve} d: 23 \textit{Apriil} Reiste ieg fra \textit{Bræche} gaardene i Guldalen, Vejen ad Stuedahlens gaarder i Sælboe Præste gieldTydalens \textit{annex}; Paa hvilcken vej begyndte at bruge Tryer under hæstene, Saasom Sneen, hvor vand var under, ej bahr hæsten, J Nord væst over dend Nordre viig af Søen \textit{øresund}, til en liiden Gaard \textit{Brynnildsvold} eller Nord øesund, fra \textit{Bræche} gaardene 1: Miil. ‒\par
Denne Søe øresund Stræcker sig fra øster i væster en 2: Nye Miile, er breed fra Søer i Nord en 1/4 miil, og Some Stætz meere, Derj fiskes øret, Røe og Lache. ‒\par
Fra \textit{Brynnildsvold} et Støcke begynder \textit{Rii-dalen}, lang fra Søer i Nord, en 2 1/2 miil, breed en 1/2 miil, og miindere; I denne dahl ligger Søen \textit{Riien}, Stræckende Sig fra Syd ost j Nord væst 1: Miil lang og næsten lige saa breed, hvorj øret Røe giedde og Abborre fanges; Da ieg var kommet over den og den backe \textit{Langsvola}, fuldte en anden dahl ved Nafn \textit{Miøsødal}, derpaa, Stræckende Sig fra Sydost i Nordvæst en 1/2 miil ned til Stuedahlen, 2{de} gaarder i Tydalens \textit{Annex}\textit{Selboe} Præstegield og Fogderie; J denne dahl er een Søe \textit{Miø}-Siøe, en 1/2 Fierding Miil lang, og liige Saa breed.\par
Denne \textit{Passage} Skeede Tædt under, og langs de Fielde \textit{Fiind Floa} og Stoer Fieldet hen til \textit{Schars} Fieldet imod \textit{Scharsdøren} hvilcket Findfloa ligger nærmist i væster til de Fielde Dahlsvola, Haftor-Stødten, og \textit{Liusenvola;} og for bem{te}Stoer Field er paa Sin Syndere Siide Fieldfast med Fiindfloa, haftor Stødten og \textit{Liusenvola}, og var paa denne vej i Fielddahlene kun nogen Tønd- og Smaa bircke Skoug at See.\par
Jmellem bræckegaardene til disse Stuedahls Gaardene er een \textit{distance} af 4{re} Miile, hvor ingen uden Tøe Smaa Field gaarder nembl: \textit{Brynnilds-vold} af en 9 mrk: og \textit{Miøsødal}, 2 miile der fra i nord, af 1 marcklaugs Skatte skyld, Midt for disse Stuedahls gaarder kan man klarlig See, \textit{Schandøren} en god 1/4 miil derfra, og det Field \textit{Sylen} Norden for det ligge\par
Dend 24{de}\textit{Apriil} maadte man her bie, for at Samble LaugRættes Mænd og viidnerne og næst følgende Dag derefter begyndtez Forrætningen.\hspace{1em}
\DivII[April 25.-26. Rettsmøte i Stuedal]{April 25.-26. Rettsmøte i Stuedal}\label{Schn1_16628}\par
\textbf{A{°}}\textbf{1742{ve}} Dend 25de \textit{April} Satte Rætten J \textit{Stuedalen}, Overværendis de 2 LaugRættes mænd \textit{Lars Olsen} Stuedahlen og \textit{Ingebrigt Halvorsen} ibm.\par
Af Kongel: Betiendtere Mødte ingen Som ej heller paa denne Aarsens tiid kunde hiidkomme formedelst Elfvene vare opgangene og Landevejen u-kørlig.\hypertarget{Schn1_16663}{}Schnitlers Protokoller I.\par
Viidner Som fremckom, vare \textit{Erich Halvorsen Aas}, \textit{Peder Ellefsen} Stuedahl, og \textit{Anders Henningsen} Miøsødal, hvilcke, Efter at Eedens forcklaring af Lovbogen havde hørt oplæse, blefve tagne i \textit{Corporlig} Eed, at Siige Sandhed og hvad dennem beckiendt er om Grændsens Gang imellem Norge og Sværge paa denne kandt. ‒\par
1: Viidne i Sælboe Fogderie hans nafn er \textit{Erich Halvorsen Aas}, er Føed paa dend gaard \textit{Aas} i Tydahlens \textit{annex}, \textit{Selboe} Præstegield, af halvor Erichsen en bonde Samestetz er 54 aar gammel, ugift og har vanchet paa fieldene og øster i Landet, holder til hos hands broder her i Stuedahlen, og arbeider der.\par
1 Spørsmaal? hvor langt denne gaard Stuedahlen ligger Fra de nærmiste Grændserne? i hvad Sogn og Fogderie? ‒\par
\textit{Resp:} Det nærmiste grændze Mærche her til gaarden Stuedahlen, er en bæck, kaldet \textit{biskopsaaen}, Som opckommer af adskillige Smaakiønner omtrendt midt imellem det Field haftor Stødten og det Field helagsStødten, omtrent 2 nye Maalte Miile østen fra denne gaard Stuedahl, Som ligger i Tydahls \textit{annex}Sælboe Præstegield.\par
2: Hvad er Landets beskaffenhed imellem bem{te} grændse Mærcke og Stuedahlen, Nembl: om der er Skoug, vande, Elfve, Field, Myhr eller Moratz, dyrcket og bebygget, eller øde, u-frugtbar land? ‒\par
\textit{Resp:} Det er bare Field uden Skoug og græs u-dyrcket og u-bebygget, dog kand her nær ved Gaarden, hvor Fieldet dahler neer, være nogen tynd smaa biercke Skoug.\par
3: Hvilcke ere de nærmiste gaarder paa denne Norske Siide nærmist ved grændse Fieldene, fra Synden at reigne, af hvad beskaffenhed er landet og hvad næring bønderne bruge?\par
\textit{Resp:} Fra Rutt Fieldet i Guldahlen at reigne, Saa ere de nærmeste, \textit{Bræche}gaardene, der ligge under Rutt Fieldet; Der nest beent i nord er ingen Gaard, førend 2 1/2 miil derfra, Nafnl: \textit{Miøsødal} af Eett Marcks Skyld, der ligger tædt under Stoer-Fieldet; ‒ der fra i nord en 1/2 miil ere De nærmiste 2{de}Stuedahls gaarder, som ligge under \textit{Schardøren}, Saa kaldet af een huulet vej, en 1/2 miil lang i øster, i det Field \textit{Schars} Fieldet, h(v)or igiennem \textit{passagen} gaar, fra Tydalen østover og tilbage, viidere i Nord er der ingen Gaarder nær ved Grændse-Fieldene i dette Tydals \textit{annex} eller Fogderie; Angaaendes landskabet fra Fieldet til \textit{Bræche} gaardene, Siiger hand det Samme ud, Som forrige viidner. Landskabet til \textit{Miøsødal} og Stuedahl er bare viilde øede Field, uden hvor de dahler neer til gaardene da der er en liiden tønd biercke Skoug, og nogen liiden Slodt af Myhr-høe; nogle Søer kan og her ligge ved gaardene, mens der er liiden lejlighed at Fiske i; korn Saaes her aldrig, Saa at Folcket have alleene deris Næring af Qvæg og Arbejde til \textit{Røraas} værck. ‒\par
4: Hvilcke ere de Nærmiste gaarder paa dend østlige Svænske Siide nærmist til grændse Fieldene, fra Synden at reigne, og af beskaffenhed Landet er og hvad næring indbyggerne bruge?\par
\textit{Resp:} Paa dend østlige Svenske Siide i Sydost fra helagstødten ere de nærmiste, 2{de}\textit{Jongdals} gaarder hørende til \textit{Herjedalen}, liggende fra helagstødten 4{re} gamble miile, Som nu kan reignes for 2{de} Nye døgtige miile; disse \textit{Jongdals} gaarder ere og Field gaarder, hvor Sielden korn voxer; Folcket der leeve af deris Qvæg, Fiskerie og Skytterie, J Nord ost \hypertarget{Schn1_16880}{}1 Vidne i Selbo Fogderi. fra \textit{Helagstødten} 6 miile derfra ligge 2{de}\textit{Handøls} gaarder, hørende til \textit{Aare Annex} i \textit{Undersogs} Præstegield i Jemteland; Paa disse Gaarder, Som ligge Saa nær Fieldet, voxer og Sielden korn, og have de indtet andet at leeve af end som forbm{te}\textit{Jong} gaarder.\par
5: Hvilcket er det første Field eller Stæd imellem Norge og Sverge, Som grændzer til de østlige Svenske Lande, det viidnet kiender paa denne kandt?\par
\textit{Resp:} De Syderligste grændse Fielde Som hand har hørt fra gammel tiid Sige, at vær Skille mærcker imellem Norge og Sverrig, ere de højeste klimper paa \textit{Rutt} Fieldet og det Field Haftor Stødten, østen for hvilcket haftor-Stødt det Field \textit{Liusenvola} ligger, og disse Fielde have østen for sig det Svenske Landskab \textit{Herjedalen} ‒\par
6: Hvilcket Field i Nord kiender viidnet at ligge nærmist de Sist opReignede Fielde?\par
\textit{Resp:} Paa \textit{Liusenvola} og haftor Stødten følge i nord, Som med dennem ere Field faste eller Sammen hængig, Stoere viide Fielde, Som Stræcke sig fra Søer i Nord, Nafnlig (1) Stoer Fieldet, hvor af Helag Stødten kiendelig opstaae (2) Det Field \textit{Sylen} (3) det Field \textit{Remmen}(4) dett Field \textit{Langvola}.\par
\textit{Rætten} forholdte viidnet at forcklare og beskriive hvert et af de opreignede Fielde i Sæhrdelished efter hinanden!\par
Viidnet Forcklarede først om \textit{Liusenvola} og haftor Stødten og dets Landskab paa alle Siider omckring, lige Som det 12{te} viidne ved 4 Spørsmaal giordt haver: dog der om nærmere oplyser, at de Fielde \textit{liusenvola} og \textit{Haftor}-Stødten ere 2{de} adski(l)dte Stødter, ved een mellem løbende bæck, og at hind ligger i Nordost fra denne, følgelig at \textit{Hyd} Elfven Rinder Væsten baade for \textit{Liusenvola} og haftor Stødten.\par
Angaaendes for det (l:) det der paa følgende Stoer-Field; beskriiver hand det Saaleedis: Det Stræcker Sig fra Søer i Nord fra dets Syndere Fod over til \textit{Neea} Elfv i Nord en 3 Nye Miile; Fra væster, hvor hun opstiiger fra \textit{Miøsøe}- og Stuedahlen, i øster til \textit{Helag Stødten} er reignet for 4{r} gamle og nu kan være 3 nye miile; Videre i øster gaar dette Field fra \textit{Helag} Stødten need til dends østlige Foed, Som er holden fo{r} 4{re} gamble, og kan være 3 Nye Miile; Længere i øster fra denne \textit{Helag Stødtens} Fod, har han iche andet kundet Seed, end Fielde og Berge. Landskabet paa dend væstere Siide af dette \textit{Stoer}-Field, ere \textit{Miøsødalen} og \textit{Stuedalen} i Norge Som ligge Tædt under dette \textit{Stoer}-Field; De gaarder, Som i disse Dahle ligge, har han forcklaret før ved 3{die} Spørsmaal; om Dahlene giver hand følgende beskriivelse:\par
\textit{Miøsødalen} Stræcker Sig fra Søeyd-ost i Nord væst en 1/2 miil, den der i værende \textit{Miø} Søe er en 1/2 Fierding Miil lang og liige Saa breed; Dahlen i Sig Selv er og iche breedere. \textit{Stuedahlen} Stræcker Sig fra Søer i Nord 1/4 miil lang, og en 1/2 Miil breed fra væster i øster; j denne Dahl ligger een Søe kaldet \textit{Stue}-Søen, Stræckende Sig fra Sydost {i} Nord væst en 1/2 Miil lang, og Som en 1/4 miil breed; j denne Søe Fanges øret.\par
Norden for Stue Søen er Landet Steenet Myhret og u-frugtbar paa væstere Siide af \textit{Stoer} Fieldet. ‒\par
Paa dend Syndere Siide er \textit{Stoer} Fieldet Fielfast med \textit{Haftor} Stødten og \textit{Liusenvola:} dog ligger der imellem i Fieldene \textit{Hyd} Søen, hvor af \textit{Hyd} Elfven Rinder i Søer, og giør en Krog der fra i væster med viidere, Som før er beSkreven. \textit{Stoer} Fieldet Sammen hæn\hypertarget{Schn1_17159}{}Schnitlers Protokoller. ger og med Fiindfloa paa denne Siide. Een 1/2 miil j Nord fra dend \textit{Hyd} Søen, jmellem Fieldene ligger een Saa kaldet biskops \textit{aae} ‒. Synden for \textit{Helags} Stødten ‒\par
Østen for denne biskops \textit{aaen} paa dend Sydostlige Siide af \textit{Helag}-Stødten ligger een liden Søe kaldes \textit{Nea}-Søe, af hvilcken Søe udkommer \textit{Nea} Elv, og rinder imellem Fieldene væsten for \textit{Helags} Stødten i Nord væst, jmellem \textit{Schars}-Fieldet og \textit{Sylene}, Siiden til gaarden \textit{Aas} i Tydalen en 6 Miil vejs, der hun Tya Elfv til Sig Tager; Viidere fra \textit{Aas} løber denne \textit{Nea} Elfv i væster 6 1/2 gamble, Som kand være 5 Nye Miile, i Sælboe Søen; hvilcken Sælboe Søe Stræcker Sig j væster 2 nye Miile, og er en 1/2 miil breed; udaf denne Sælboe Søe riinder \textit{Nea} Elfv i Nord væst 2 1/8 maalte miile ved \textit{Trondhiems} bye udj \textit{Trondhiems Fiord}, og giver der Sit udløb i fiorden det Nafn \textit{Nederos}, eller \textit{Nideros} hvor af og byen kaldes \textit{Niderosia}. Biskops \textit{aaen} rinder i Nord strax ind i \textit{Nea} Elfv. Paa dend Nordre Siide riinder dend Elfv \textit{Nea} forbj, (Som meldt) det Field \textit{Stoer}-Fieldet; Ellers har dette \textit{Stoer} Field, efter dets visse deele, Sine Sæhrdeelis Nafne: paa dets Norderlige Siide, en 1/2 Miil omtrendt fra \textit{Nea} Elfv, gaar fra væster i øster en 1/2 miil lang een huul vej omtrendt 1/8 miil el: mindre breed denne huule vej kaldes \textit{Schardøren}, og hvad Som ligger paa begge Siider af denne \textit{Schardøren}, Som er lige Som en Rygg af Fieldet, det Stycke og liige til \textit{Nea} Elfv i nord heeder \textit{Schars} Fieldet: men det Støcke af Fieldet fra dend Syndere foed op imod Rychen af \textit{Schardøren}, og det Som Ligger østen for \textit{Schardøren} liige over til \textit{Nea}-Elfv, det alt kaldes i Sæhrdeelished Stoer fieldet.\par
2 Miil i øster fra denne \textit{Schardørs} østere Ende opstaar af dette Stoer Field en høy Rundagtig Field klimpe, og denne klimpe kaldes \textit{Helags} Stødten.\par
(2) Det Field \textit{Sylen}, Som følger paa \textit{Stor}-Fieldet i Nord, beskriiver viidnet Saaleedis, \textit{Sylen} er Fieldfast med \textit{Stor} Fieldet eller \textit{Schars} Fieldet, Kun at \textit{Nea} Elfv Rinder der imellem, Som meldt er. Det stræcker Sig fra Syd ost i Nord væst en 2 miile lang, og ligger fra \textit{Helag} Stødten i Nord nord væst, værende det høyeste Field imellem Tydalen og \textit{Jemteland}, og meener at det fra væster i øster Skal være 1 1/2 miil over, breed, Paa dend østere Siide af det Field \textit{Sylen} ligge endnu fleere, dog mindere Fielde, Som ere Sammenhængige Med Sylen, og ligge nærmest dertil; Een Miil fra Sylen øst over de mindere Flade Fielde Rinder \textit{Handøl} Elfven Nord efter: dog kun imellem Fieldene; thi østen for \textit{Handøl} Elfven Seer man indtet andet end høye, viide Fielde ind ad \textit{Jemteland} Saa lang man øyene kand. ‒\par
Denne Elfv \textit{Handøl} udkommer fra den Nordre Siide af \textit{Helag} Stødten og løeber først i Nord 2 Nye Miile lang i den Søe \textit{Ahn,} der Stræcker Sig fra Søer i Nord 2 nye Miile og er meere end 1: Miil breed, der fra løber Elfven 1: miil lang til \textit{Giøv}-Søen, Som er mest rund en 1/2 Miil Stoer, Siiden gaar Elfven i Nord ost; der efter dend kroger Sig og Rinder een Stund i Sydvæst, siden vænder Sig i Søer ad dend Søe \textit{Aarsøe}, der Stræcker Sig fra Nord i Søer 1 1/2 miil, og er en god Fierding Miil breed, der af rinder Samme Elfv i dend Søe \textit{Liiten}, 1 miil omtrent lang og breed, Synden for \textit{Gierpe} Skandtze, der fra flyder Elfven i dend Søe \textit{Okke}, og Siiden i \textit{Stoer} Søen, hvor i \textit{Fraas-øyen} ligger. ‒\par
Paa dend væstre Siide af det Field \textit{Sylen} ligger dend Søe \textit{Eesand}, Stræckende Sig fra Søer i Nord 1: miil lang, og liige Saa breed fra øster i væster; der i fiskes øret og \hypertarget{Schn1_17475}{}1 Vidne i Selbo Fogderi. Røe. væsten for \textit{Eesand} ligger \textit{øy}-Field ett Rundt Field 1: miil over: Dog Rinder een Elfv kaldes \textit{Esna} af den Søe \textit{Eesand}, østen forbi \textit{øy}-Fieldet, i Søer ind i \textit{Nea} Elfv.\par
Væsten for \textit{øy}-Fieldet følger En Furru og gran Skoug en 1/2 miil lang indtil Østbye gaardene i Tydalen, Saa der er 3 miile fra \textit{Sylens} væstere Ende til østbye,\par
J Nord hænger det Field \textit{Sylen} Sammen med det Field \textit{Remmen}. ‒\par
(3) Dette \textit{Remmen} Stræcker Sig i Nord, og liidet i nord væst, Som hand meener, en god miil Lang;\par
østen for dette \textit{Remmen} Rinder \textit{Eina} Elfv, der udkommer af dend nordre Ende af det Field \textit{Sylen}, og løber først i Nord Østen forbi det Field \textit{Remmen} en 2 miile imod det Field \textit{Langvola} dets Syndere Ende, hvor fra dend giør een bugt paa Sig ad øster, Siiden vænder dend Sig i Søer og løeber ind i \textit{Handøl} Elfven et bøsse Skud, førend denne \textit{Handøl} gaar ind i \textit{Ahn}-Søen; der hvor de 2{de} Elfver løbe Sammen, der ligger de 2{de}\textit{Handøls} gaardene, nembl: paa dend Syndere Siide af \textit{Eina} Elfven. ‒\par
østen for \textit{Eina} Elfv, er et Flat Field 2 Miile lang i øester indtil det høye Field \textit{Snasahougen}, Som Stræcker Sig fra Søer i nord 1: miil lang, og er fra det Flate Field over i øster til \textit{Handøl} gaardene 1 1/2 miil breed; østen for disse \textit{Snasahougen} ligge de 2{de}\textit{Handøl}- gaarder paa denne Siide af \textit{Handøl} Elfven,\par
østen for \textit{Handøls} gaardene ligger \textit{Ahn} Søe og een stoer Furru og Granskoug deromckring, 1 Miil lang; 1 Miil fra \textit{Handøl} gaardene J Nordost ligger de 2{de} gaarder \textit{Wollan}, hvis leilighed er liigedan Som \textit{Handøls} gaardenes, fra \textit{Wollan} gaardene i øster er een Stoer Skoug af Furru og Gran 1 Miil lang, hvor efter den gaard \textit{Tongbøl} ligger, hvilcke gaarder alle henhøre til \textit{Aare Annex} i \textit{Undersogs} Præstegield, og haver \textit{Handøl} bønderne til deris \textit{Aares annex} Kircke 5 Miile at fare.\par
Dend væster Siide af det Field Remmen bestaar af luther øede Field vohler, hvor ingen Skoug eller Gres er, 2 Miile indtil biønEggen, een field vohle, hvor under paa dend Nordre Siide Mærrager have deris Sæterbolig og Slott; nord væsten for denne Sæter vold ligger een Stor Skoug dahl af Furru og gran, kaldes \textit{Stordalen}, Mærrager bøjd tilhørende ‒\par
Efter dette erindrede viidnet Sig, at forcklare nærmere om \textit{Remmens} østere Siide, at der hvor \textit{Ein}-Elfven giør en bugt paa Sig fra Nord i øster, den bugt kaldes Ejnbogen, og der om har hand Stædtze hørt fra gammel tiid, at det Skulde være et grændse-mærcke imellem Tydalen i Norge og \textit{Jemteland} i Sverrig; ‒\par
Om det Flate Field østen for \textit{Eina} Elfv, som han har talt om før paa nest foregaaende Siide forcklarer hand nærmere, at næst ved \textit{Eina} Elfv i øster, og wæsterst af det Flate Field, ligger een Berg \textit{Hammer} kaldet \textit{Blaahammer}-klimpen, et kort Støcke rætt i Søer fra \textit{Einbogen}. Fremdeelis berætter hand at Syndenfor \textit{Snasahougene} liger ett Flatt Field, kaldes \textit{OlvaaKiølen}, fieldfast med \textit{Snasahougene;} og over denne \textit{OlvaaKiølen} gaar dend \textit{ordinaire} vej udaf Tydahlen til \textit{Handøl}.\par
(4) Angaaendes det liidet Field \textit{Langvola} forcklarer hand, at det Stræcker Sig fra Sydost i nord væst; hvor langt og breedt det kand være erindrer hand iche; ‒\par
Paa dend østere Siide af dette \textit{Langvola} er Stor Skoug af Furru og Gran med nogen \hypertarget{Schn1_17802}{}Schnitlers Protokoller I. bierck, og Stræcker Sig denne Skoug fra Fieldet i øster, og til \textit{Handøl} gaardene i Sydost en 2 Miile, ligesom \textit{Eina} Elfv gaar did. ‒\par
Dend væstre Siide til Norge erindrer hand iche rættere end at den bestaar af Smaa Field vohler og Skouge, Som hælder til Mærrager bøjd. ‒\par
paa Dend Syndere Siide af \textit{Langvola} er imellem det og \textit{Remmen} en liiden Dahl med biercke Skoug i, og oven til er \textit{Einbogen} Synden for \textit{Langvola} Fieldet. ‒\par
Norden for dette \textit{Langvola}, veed hand ingen beskeed om at give: dog veed hand at have Al tiider hørt, at dette \textit{Langvola} er ett grænse Mærcke imellem Mærrager i Norge og \textit{Jemteland} ‒\par
7: Om hand veed, til hvis gaarder disse beskrevne Fielde henhøre, eller om de under nogen hvis gaards grund ligge? eller er Alminding ‒.\par
\textit{Resp:} Der om kand han ingen Egentlig Svar giive Saa som de ere blefne anseede for udørckene: Dog veed han at mændene i \textit{Stuedalen} have altiid haft deris høe Slott Synden og Norden for \textit{Nea} Elv væsten for \textit{Schars} Fieldet. og Folck af Tydahlen have u-behindret brugt deris Fiskerie i de Smaa kiønne Synden og væsten for Biskops \textit{Aaen}. ‒\par
8: Om der har værit nogen tvistighed {i}mellem vore og de Svenske undersaattere om disse Fielde? \textit{Resp:} Nej aldriig. ‒\par
9: Hvad Nytte Godhed og Herlighed er der ved disse grændze Fielde?\par
\textit{Resp:} ingen, thj hvercken er der \textit{Maase} eller Græs, Fiskeriet i de Smaa Field-vande og Elfver, er iche af nogen Synderlig betydenhed Skytteriet i Fieldene er og Ringe; thj Lapp Findnerne iidelig der jage. ‒\par
10: Om viidnet veed, hvad Skillnet giør imellem De Svenske Landskaber \textit{Herjedalen} og \textit{Jemteland} ved de Norske grænser i Væster? ‒\par
\textit{Resp:} Hand Synis, at Skilnet imellem \textit{Herjedalen} og \textit{Jemteland} ved de Norske grænser maa være \textit{Helag} Stødten, med de andre der fra i øster liggende Fielde; Thj (1) Hvad Som ligger paa Søndere Siide af disse Fielde, det alt hører \textit{Herdalen} til, og hvad Som ligger paa dend Nordre Siide, er \textit{Jemtelands}.\par
(2) Naar man kommer øster ud af \textit{Schardøren}, hvor fra \textit{Helag} Stødten ligger en Toe Miil beent i øster, Saa vænder vejen Sig Strax i Syd-ost need ad \textit{Liusendalen} og \textit{Herjedalen}; Naar man vil rejse til \textit{Jongdalen}, Som ere de yderligste og Norderligste gaarder i \textit{Herjedalen}, Saa gaar vejen fra dend østere Ende af \textit{Schardøren} liige i øster mod \textit{Helag} Stødten, og derpaa dends Syndere Siide gaar da vejen viidere i øster til \textit{Jongdalen;} og disse \textit{Jongdals} gaarder Som ere de Norderligste i \textit{Herjedalen}, ligge paa dend Sønder Siide af de Fielde, Som fra \textit{Helag} Stødten gaar ud i øster. ‒\par
Derimod paa dend Nordre Siide af \textit{Helags} Stødten Naar man skulde Reise neer i øster, kommer man neer i \textit{Jemteland;} vel ligger ingen \textit{ordinaire} vej fra dend øster Enne av \textit{Schardøren} paa vænstre haand i Nord ost, thi der er det baade Steenet og Myret; mens dog naar man vil fare i Nord ost fra \textit{Schardøren}, eller Norder for \textit{Helag} Stødten og de østlige Fielde need j landet, Saa kommer man need i \textit{Jemteland} til dets første \textit{Otsiø} Sogn, Som ere de yderste og Sydligste gaarder i \textit{Jemteland}. ‒ Angaaendes \textit{Jongdals} gaardene i \textit{Herjedalen} erindrer viidnet sig dette at tillegge, at de have deris Nafn af \textit{Jong}-Elfven, Som \hypertarget{Schn1_18064}{}1 og 2 Vidne i Selbo Fogderi. udrinder fra dend østere Siide af \textit{Helags} Fieldet, og løber en 3 Nye Miile i ost Sydost til \textit{Jongdals} gaardene, og vænder Sig Derfra i Nord ind ad \textit{Jemteland}.\par
11: Hvilcke ere de Rætte grændse Mærcker, Som viidnet kiender, eller har fra gammel tiid hørt, at være imellem Norge og Sverrig paa denne kandt? og om der findis nogen Vare eller Andet Kiende tegn Nogen Stædz opsadt?\par
\textit{Resp:} Han har hørt fra gammel tiid, at (1) det Field \textit{Haftor} Stødten, der nest (2) \textit{Biskopsaaen} Som hand meener at ligge 1 Miil vejs j Nord Fra \textit{Haftor} Stødten og 1 1/2 miil i Søer fra \textit{Helag} Stødten, at forstaa; hvor \textit{Biskops-aaen} har Sit udSpring; Dernest (3) \textit{Einbogen}, som er dend bugt, før er omtalt, hvilcken \textit{Eina}-Elfven henimod \textit{Langvola} Fieldet fra Nord j øster paa Sig giør; hvilcken \textit{Einbogen} ham Synes, at ligge liige i Nord fra biskops-\textit{aaen} een 7 nye Miile og (4) Endelig \textit{Langvola}.\par
12: Hvor langt disse grændse Fielde og Stæder ligge fra bøydene og Landevejen?\par
\textit{Resp:} Fra Biskops-\textit{aaen}, Som det nærmiste grændse Stæd, er til Stuedahlens 2{de} gaarder vel 2 goede Miile; og til Tydahls bøjden 4 1/2 Miil. ‒\par
13: Hvor underholdning for Folck og beete for hæstene paa eller ved grænse-Fieldene bliver at faa?\par
\textit{Resp:} J denne Tydahls bøjd er liiden underholdning for Folck at vændte, thj bønder(n)e have n(e)ppe Selv levnets midler: j næste Selboe Sogn er tilstanden iche bedre, hvor for det nærmeste Stæd for dennem vil blive \textit{Røraas} eller \textit{Trondhiem}, hvilcken Sidste Stad fra biskobs \textit{aaen} bliver en 16 Miil. beete for hæstene er nærmist at tage i Fieldene Selv, saa som imellem Stoer fieldet og \textit{Haftor}-Stødten kan være noget liidet, omckring \textit{Nea} Elfv Synden for \textit{Sylen} kan være meere Ligesom om \textit{Eina} Elfven ved \textit{Langvola} god Lejlighed der til kan være ‒\par
14: Hvad Mænd han kiender at være beqvæmest og kyndigst til at udviise grændse Stæderne til Grændse Maalerne?\par
\textit{Resp:} Hand veed ingen beqvæmmere, end dend \textit{Lap-Fiind}\textit{Morten Nillsen}, Som med Sit Sælskab holder til i disse Fielde og kand tale godt Norsk.\par
Hvor paa dette viidne blef \textit{dimittered;} og Siiden det var Saa Siilde paa aftenen, Rætten opsatt til i Morgen. ‒\hspace{1em}\par
Dend 26 Apriil blef Rætten \textit{Reasumeret} og fremkaldet\par
2{det} viidne: \textit{Peder Ellefsen} Stuedahlen, Føed paa \textit{Hølandet} i \textit{Meelhus} Præstegield af bønder-Forældre, er 79 aar gammel, gift har 1 barn, boer paa gaarden Stuedahl nærer Sig af gaards brug og arbejde til \textit{Røraas} værck.\par
Til 1: Spørsmaal Svarer det Samme, som første viidne\par
Til det 2{det} 3{die} og 4{de} Spørsmaale Svarer det Samme, Som første viidne\par
Til det 5{te} Svarer det Samme, som første viidne.\par
Til det 6: Svarede, at hand vel kiendte de 3 første, Nembl: \textit{Stor}-Fieldet, det Field \textit{Sylen} og \textit{Remmen} men iche \textit{Langvola}\par
Om det første nembl: \textit{Stoer}-Fieldet giver hand Samme Forcklaring, som 1 viidne.\par
Det 2{det} Field \textit{Sylen} med dets Landskab paa Siiderne beskriiver hand, ligesom første viidne undtagen dend Elfv \textit{Handøl}, hvis løb og vænninger hand iche Saa nøjagtig kiender. ‒\hypertarget{Schn1_18343}{}Schnitlers Protokoller I.\par
Om det 3{die} field \textit{Remmen} forcklarer hand sig, ligesom første viidne. ‒\par
Om det 4{de} Field \textit{Langvola} veed han intet at viidne.\par
Til det 7: Spørsmaal? Sambt 8{de} 9{de} 10{de} og 11{te} viidner hand det Samme, Som første viidne: undtagen \textit{Langvola}, der hand iche er beckiendt\par
Til 12{te} 13{de} og 14{de} Svarte det Samme Som Første viidne, hvor paa hand blef \textit{dimittered}.\hspace{1em}\par
3{die} Viidne blef da fremckaldet ‒\par
heeder \textit{Anders Henningsen Miøsødal} Er Fød paa denne gaard Stuedahlen, af bønderForældre, er 27 aar gammel, u-gift, boer paa dend gaard \textit{Miøsødal} af 1 mrk: leje ‒\par
Til 1{te} 2{det} 3{die} og 4{de} Svarer hand, Som 1{te} viidne undtagen at hand indtet veed om \textit{Handøl} at Sige;\par
Til 5{te} Spørsmaal Svarer, at det første field, hand kiender, er \textit{Haftor} Stødten, hvor om hand viidner Som 1{te} Viidne. ‒\par
Til det 6{te} Svarer: Norden for \textit{Hoftor} Stødten kiender hand \textit{Stoer}-Fieldet, viidere i Nord er hand iche beckiendt. ‒\par
Om dette \textit{Stoer} Field viidner hand det Samme Som de 2{de} næst forrige viidner.\par
Til det 7: Svarer, Som næst forrige viidner, undtagen at hand ej veed, at Tydalinger have fisket ved biskops \textit{aaen}; ‒\par
Til det 8{de} og 9{de} Svarer hand Som det første viidne;\par
10 Spørsmaal \textit{Cesserer} Siiden hand iche har værit i \textit{Jongdalen} og \textit{Handøl} ‒\par
Til 11{te} Spørsmaal Svarer, at hand har vel hørt \textit{Haftor} Stødten at være eett Grændze Skiil: men længere i de Norderlige Fielde har hand iche vanchet, og der for indtet hørt om dennem. ‒\par
12: \textit{Cesserer} ‒\par
Til det 13{de} Svarer, Som forrige Viidner ‒\par
og Til det 14{de} Svarer det Samme ‒\par
Hvor paa han blef \textit{dimittered}; og Rætten Sluttet. \hspace{1em}\centerline{(sign.) Peter Schnitler (L. S.)}\centerline{(sign.) Lars Olsen Stuedahl (L. S.)}\centerline{(sign.) Jngebrigt Halvorsen Stuedal (L. S.)}\hspace{1em}\par
Rætten Syndtes, første viidne i \textit{Selboe}\textit{Erich Halvorsen Aas} at være beqvæm og kyndigst at vejviise grænse Maalerne i Fieldene paa denne kandt, dog kunde Lap Fiindner og der til tages. Rætten betydede og Lars Stuedahlen, at lade Strax Lensmanden i Tydahls bøjden viide, saa Snart grændse-Maalerne Sig hid nærme, at hand kan Straxen opckomme, og gaae dem tilhaande med de fornødne anstalter. {(sign.) Peter Schnitler.}\hypertarget{Schn1_18597}{}4 Vidne i Selbo Fogderi.
\DivII[April 26. Fra Stuedal til Ås i Tydal]{April 26. Fra Stuedal til Ås i Tydal}\label{Schn1_18599}\par
For Føerefaldets Skyld Skyndede man Sig, at \textit{Continuere} Forrætningen paa Nestfølgende Grændse-Gaard, og Reiste derfor Samme\par
Dend 26{de}\textit{Apriil} 1742{ve} fra Stuedahlen til \textit{Aas} gaard i Tydalen, Som er Reignet for 3 1/2 gamle Miil, og nu kand være omtrendt 1 3/4 nye Miil i væst Nord væst der fra beliggendes, hvilcken vei man og maa tage, naar mand vil herfra til \textit{Mærrager Annex} i Nord, een Grændse bøjd til \textit{Jemteland} liggende. 
\DivII[April 27.-28. Rettsmøte på Ås]{April 27.-28. Rettsmøte på Ås}\label{Schn1_18639}\par
Paa denne Gaard \textit{Aas}: Dend Næst paa følgende 27 \textit{Apriil} blef Da Rætten Satt, overværendes bøjde Lensmand Joen Biørnsen Aas, og bøjde Skolemæster \textit{Svend Hellein}, Som af dette \textit{Selbo} Præstegields\textit{Pastor} Hr. \textit{Peter Haasbøl Rosenvinge} er beskicket og bruges til at underviise og oplære Tydals Fieldenes Lapp Findnene ‒\par
Kongel: Civiil: betiendtere vare her iche, Som Elfvene Rundten om vare opgangne og ingen der af vel kunde fremkomme. ‒\par
J denne Rætt vare efter Lensmandens foranstaltning Som Rættens bie-siddere de 2{de} LaugRættes Mænd: \textit{Jngbrigt Biørnsen Aas;} og \textit{Johannes Olsen ibm}.\par
Som Viidner fremstillede Lensmanden paa dette Stæd: \textit{Bendt Simensen loføyen} og \textit{Ole Pedersen Østbye} ‒\par
For dennemSambtlig blef dend Kongel: \textit{order} til \textit{Major Schnitler} af 16 \textit{martj} Sidstleeden forckyndt, og Eedens Forklaring af Lov bogen Viidnerne fore læst, til at Siige deris Sandhed, og hvad dennem vitterligt var om Grændsernis Rættegang og Skilnet imellem det Riige Norge og Sverriig; hvorpaa de og aflagde deris \textit{Corporlig} Eed. ‒ Hvor paa blef foretagen\hspace{1em}\par
4 Viidne i dette \textit{Selbo} Fogderie og Gield. ‒\par
Hans Nafn er \textit{bendt Simensen Løføyen}, Er Føed i \textit{Handøl} gaardene i \textit{Jemteland}, Nærmist liggende her til Tydahlen i Norge, af bønder Folck, Og var hands Fader Klocker til det nærmeste \textit{Annex} kaldet \textit{Aare} Sogn i \textit{Undersagers} Præstegield i \textit{Jemteland}, Er 76 aar gammel, er gift, og har ingen børn, har værit her i Norge over 50 aar i førstningen tiendte, og Arbejdede hand her i Tydals bøjden og paa \textit{Røraas}, Siiden fick hand her een gaard bøxlet i Tydalen nembl: \textit{Løføyen}, hvor hand nu boer og er bonde.\par
Der efter blef Viidnet tilSpurt ved Samme Spørsmaal Som viidnerne tilforn i Stuedahlen, dend 25{de}\textit{Apriil} 1742{ve} Nembl:\par
1: Spørsmaal. Hvor langt denne Gaard \textit{Aas} ligger fra grændserne? J hvad Sogn og Fogderie.\par
\textit{Resp:} Denne Gaard \textit{Aas} ligger Fra Stuedahl i Væst Nord væst 3 1/2 gamle Miile, Som kand vær en 2 nye Miile Meer eller Mindre; \textit{distancen} fra \textit{Stuedalen} til grændse Mærcket \textit{Biskops aaen} beskriiver hand ligesom 1{te} viidne i Stuedahlen ‒\par
2 Spørsmaal. Hvad er Landets beskaffenhed imellem bem{te} grændse Mærcke \textit{Biskops aaen} og denne gaard \textit{Aas}, Nembl: om der er Skoug, vande, Elfve, Field, Myhr, dyrcket og bebygget, eller øede ufrugtbar land?\par
\textit{Resp:} Angaaendes det Landskab imellem biskops-\textit{aaen} og Stuedahlen Siiger hand det Samme ud, Som bem{te} 1{te} viidne.\hypertarget{Schn1_18883}{}Schnitlers Protokoller I.\par
Landskabet imellem Stuedahl og disse \textit{Aas} gaarder er een trang dahl, Stræckende Sig fra øster i væst Nord væst 5 1/2 gammel Miil, og kand være omtrent 3 1/8 Nye Miil.\par
Denne Dahl kand være fra Søer i Nord en 1/2 miil og mindere breed; Og haver paa begge Siider Fielde, nembl: paa dend Syndere Siide ad \textit{Aalen Annex} et Sammenhængig Field, kaldet Største deel \textit{Løfunden;} Paa dend Nordre Siide ad Mærrager \textit{Annex} har det de Fielde \textit{Oye} Field og \textit{Fongen} med mellem løbende Dahle, ‒\par
J Dette Tydalen ligger dend Søe Stuesøen, om hvilcken hand gir Samme forcklaring, Som 1{te} viidne. Elfver i dette Tydalen er først: Tya Elfv, hvor af denne bøjd har Sit Nafn; Denne Tya Elfv udSpringer i førstningen fra dend Søe \textit{Miøe}Søen i \textit{Miøsødalen} hvilcken Søe er Rund 1/8 miil lang og breed, der af Rinder denne (og har da det Nafn \textit{Miøaaen},) fra Søer i Nord en 1/2 miil vejs i Søen Stue Søen, Norden for Stuegaardene, hvilcken Stue Søe før er beskreeven, naar nu Elfven Rinder af denne Stue Søe, faar dend det Nafn Tya og løeber da i Nordvæst een god 1/4 miil i \textit{Moen} Søe, Som er Rund en 1/8 miil stor, og der fra fortsætter \textit{Tya} sit løb i nordvæst 1 1/2 miil til under denne gaard \textit{Aas} hvor dend løber j \textit{Nea}-Elfv, og Taber der Sitt Nafn. ‒\par
Dend anden Elfv i denne Dahl er \textit{Nea}, hvilcken 1{te} viidne ved 6{te} Spørsmaal før har forcklaret. ‒\par
Jmellem Stuedahl og første gaard \textit{Løføyen} Er indtet andet end nogen Tyndt biercke Skoug og nogle Myhr-Slotter, og kand vejen imellem begge gaardene være en 1 1/2 miil vejs; Fra Løføyen begynder dend tædtere bøjd, hvor gaardene ligge 1/8, 1/4: eller meere af een Miil fra hinanden; j denne Tædtere beboede bøjds gaarder Saaes nu vel, men Sielden høstes Moet-Korn; Nogen Furru og Gran Skoug Sambt Enge Sletter ligger til disse Gaarder: Saa at deris Næring bestaaer af deris Qvæg og liidet Fiskerie i Field-vandene, Sambt nogen Tømmer hugster til Saug brug: dog som dette brug med Skougene formindskes, Saa maa de Søge over Fieldet, at giøre arbejde til \textit{Røraas} værk\par
Til det 3{die} Spørsmaal Som var giort til 1{te} Viidne i Studahlen, Svarer hand liigesom Samme 1{te} Viidne. ‒\par
Til det 4: Spørsmaal Svarer hand det Samme som bem{te} 1{te} viidne.\par
Til det 5{te} Svarer det Samme, Liigesom 1{te} viidne\par
Til det 6{te} Angaaendes de Fielde Dahlvol og \textit{Liusenvola} Sambt \textit{Haftor} Stødten med deris omliggende Landskaber giver hand Samme forcklaring, Som tidt bem{te} 1{te} Viidne. Angaaendes det derpaa følgende (1:) \textit{Stoer}-Field Og dets omliggende \textit{Situation} Sambt den Saa kaldede Biskops-\textit{aae} kommer hand ligeleedis over eet med 1{te} Viidne Sammestæds: Dog legger dend forcklaring der om til, at denne biskops-\textit{aae} ligger vel i Søer fra \textit{Helag}- Stødten, men dog noget i væster der fra, Som hand meener 1: Miil, at forstaa om \textit{aaens} begyndelse: fremdeelis Siiger hand, at der ere 2{de}biskops \textit{aaer}, begge opckommendes af \textit{Stoer}-Eieldet, og begge Rindendes fra Søer i Nord, i begyndelsen er der en \textit{distance} af 1/8 miil omtrent imellem dem, men imod det at de falde begge ind i \textit{Nea} Elf, er der kun et liidet Rum imellem dem, Som hand icke rættere mindes; af disse Toe biskops \textit{aaer} er dend væstligere mindre, end dend østlige, og denne østlige biskops \textit{aae} er det, Som er ett beckiendt grændse Mærcke.\hypertarget{Schn1_19137}{}4 Vidne i Selbo Fogderi.\par
Angaaendes det (2) Field \textit{Sylen}, og de (3) Field \textit{Remmen} giver hand Samme forcklaring Som tidt bem{te} 1{te} viidne: dog om dend Elfv \textit{Handol} tilføjer hand dend underrætning nøjere, end Samme viidne giort haver, at naar \textit{Handøl} Elfven kommer af \textit{Giøv} Søen, gaar dend i nord væst en god 1/2 miil lang, indtil dend falder i en Søe kaldet \textit{Bos}-Søe, hvilcken Søe Stræcker Sig fra oster i væster 3/8 Miil Lang og er en 1/4 miil breed, Siiden Riinder Elfven i Nord ost 1/8 miil i \textit{Tæn} Søen, Som er 1/8 Miil Stoer paa alle Siider; fra denne \textit{Ten}-Søe gaar Elfven fremdelis 1/4 miil i dend Søe \textit{Noeren}, Som Stræcker Sig fra væst i øster en 1/2 miil, og er 1/4 miil breed, paa hvis Nordre Siide 4{re} bønder gaarder av Nafn \textit{Norhaalla} ligge; fra \textit{Noer}-Søen gaar Elfven viidere i Syd ost 1: miil at forstaa Nye Miil, i \textit{Aar}-Søe, fra hvilcken Søe hand Siiden beskriiver Elfvens løb, liige Som bem{te} 1{te} Viidne giordt haver; Angaaendes det (4{de}) Field \textit{Langvola}, viidner hand liigesom forbem{te} 1{te} viidne Sammestæds, Dette tillæggendis, at paa dend væstre Siide af \textit{Langvola} og de der ved hængende Smaa Field \textit{voler} er Skiøn Furru og gran Skoug, Som fra øster i væster til Mærrags bøjden er en 3 gamle Miil lang og kaldes \textit{Tevel-dalen}; ‒\par
Rætten foreholdt viidnet at forcklare, om hand viste, hvad Fielde eller Landskab viidere i nord paa \textit{Langvola} Field følger? ‒\par
\textit{Resp:} Norden for \textit{Langvola} følger een Dahl med nogen Smaa biercke Riis j, imod 1/4 miil breed, men hvor lang dend dahl er fra øster i væster, veed hand iche. ‒\par
Paa denne bierche Skoug-dahl, følger i Nord Een Jord houg med nogen bierche Skoug paa, Nafnl: \textit{StoerLie}:, Som Stræcker Sig fra Søer i Nord 1/4 Miil over, og Meener at dend er liige Saa breed fra øster i væster, at reigne fra Foed til foed; ‒\par
Dend østere Siide af denne \textit{Storlien}, bestaar af een Furru, gran og bierche Skoug kaldet \textit{Lil Teveldalen}, Som Stræcker Sig fra \textit{Storlien} i Sydost først til \textit{Eina} Elfv, Siiden til \textit{Handøl} gaardene og hører \textit{Handøl} Mændene til i \textit{Jemteland}, hvorhendtil Skougen er 3 Miile lang. J anleedning af denne \textit{Handølingernis} Skoug erindrer viidnet Sig at berætte, at liige op Fra \textit{Handøl} gaardene i Søer i det Field \textit{Snasahougen} er omtrendt for en 6 aar Siiden af de Svenske et nytt kaabber værck anlagt, og her \textit{gruben} optaget og 6 miile her fra ved \textit{Gierpe} Skandze er Smælte hytten hertil opbygget.\par
J begyndelsen har det tæignet Sig vel med denne \textit{Grubes Malm}, men nu Siiges dend at have taget af. ‒\par
Paa dend væstere Siid af \textit{StorLien} ligger dend før omtalte Skoug \textit{Stoer Teveldalen} 3 miile lang i væster og hører Mærrags bøjden til. Og dette \textit{Stoer Lien} er et beckiendt Lande Mærcke imellem \textit{Jemteland} og Mærrags bøjden, i Norge ‒\par
Paa \textit{Storlien} følger i Nord \textit{Kiørkgaals}-Fieldet, dog ligger der imellem een liiden Sløgt eller Fielddahl med liidet græs, med jngen Skoug i; Dette \textit{Kiørkgaals} Field ligger fra Stoerljen j Nord Nord ost; hvor langt og breedt det er, kand hand iche Siige.\par
Dend østerlig Siide af dette \textit{Kiørkgaals} Field bestaar af bare Myhr land, og Skal være dend Svenske \textit{Crones} Alminding, Siiden det ligger Saa langt i fra bøjdene; 2 miile i øster for denne Myhr er een Skoug 1: Miil lang og hører de 2{de} gaarder \textit{Wollan} til, Som ligge de nærmeste i øster 3 Miile fra dette \textit{Kiørkgaals} Field. Paa dette Field Plejer de \textit{Hollandsche} Falche Fængere om Midt Sommer tiid at være, og fange Falche, hvilcke fra \hypertarget{Schn1_19396}{}Schnitlers Protokoller I.\textit{Handøls} gaardene lade Sig forsyne med underholdning, Som de fra fieldet ligge 3 Miile i Sydost.\par
Hvad Landskabet er paa dend væstere Siide ad Norge, veed hand iche: men det er ham beckiendt at det høyeste af dette \textit{Kiørkgaals} Fieldet, Der hvor Falche Fængere have deris Falche hytte, er ett Lande Mærche jmellem \textit{Jemteland} og dend Norske Mærrager bøjd. Paa \textit{Kiørkgaals} Fieldet i Nord følger \textit{Halsiø-Ruva}, ett liidet Field; hvordanne Landskabet er paa dets Siider? eller hvordanne det Stræcker Sig, Det erindrer hand iche, men det veed hand, at det er ett grændse Skill imellem \textit{Jemteland} og Mærrager; Siiden gaar Fieldet jidelig Sammenhængig i Nord, men veed ingen Sæhrdeelis Nafn derpaa at giive ‒\par
Til det 7: Spørsmaal Svarer hand, at de af ham opreignede Fielde ere Skille Mærckerne imellem begge Riigerne: men om de ligger under nogle Gaarder, veed hand iche.\par
Til det 8{de} Svarer: Nej. ‒\par
Til det 9{de} Svarer, at hand iche veed, der kand være nogen Nytte ved de haarde Steen Fielde, uden nogen liiden Fiskerie i Fieldvandene, og nogen ringe Myhr-Slott i Fielddahlene, for de angrændsende Gaarder.\par
Til det 10{de} Svarer, Liigesom det 1{t}e viidne. ‒\par
Til det 11: \textit{Resp}: \textit{Haftor} Stødten og \textit{Bischopsaaen} er imellem \textit{Herjedalen} og \textit{Tydalen}; \textit{Einbogen} af Elfven \textit{Eina, Langvola} paa Sitt høyeste, \textit{StoerLie}, \textit{Kiørkgaals} Field, og \textit{Halsiø Ruva} ere Skille Mærcker imellem \textit{Jemteland} paa dend østlige og \textit{Tydalen} Sambt \textit{Mærrager} paa dend Væstere Norske Siide: men hvad \textit{distance} der er imellem hver af disse Mærcker, veed hand ej Egentl: at Siige ‒\par
Til det 12{te} Svarer om vejen fra Biskops\textit{aaen} til Stuedahl Siiger hand det Samme, Som 1{te} Viidne ‒ Fra \textit{Einbogen} og \textit{Stoerlien} forcklarer hand, at være 3 Miil til \textit{Mærrager} bøjd; Hvilcken vej hand meener, det og bliver Samme Stæds hæn fra \textit{Langvola; distancen} fra \textit{Kiørkgaals} Field og \textit{HalsiøRuva} veed hand iche, til \textit{Mærrager} eller næste bøjd. ‒\par
Til 13{de} Svarer, Som 1{t}e Viidne ‒\par
Til det 14{de} Svarer, at det 1{te} viidne \textit{Erich Halvorsen Aas} i Stuedahlen, dernæst dend Lap \textit{Fiind} i di \textit{Tydalsche} Fielde\textit{Morten Nilsen} kand være beqvæmme og kyndige til at vejviise grændse Maalerne.\par
Hvor paa viidnet blef \textit{dimitteret}, og Rætten til i Morgen opsatt Siiden det var Saa Siilde paa Aftenen ‒\hspace{1em}\par
\textit{Anno}1742{ve} dend 28 \textit{Apriil}\textit{Continuerede} Rætten med at \textit{examinere}\par
Det 5{te} Viidne i Tydalen ‒\par
Hans Nafn er \textit{Ole Pedersen Østbye} ‒ er føed paa gaarden \textit{Aas} i Tydalen, af bønder Folch her Samme Stæds, 46 aar gammel, gift, har 3 Sønner, er bonde paa dend gaard Østbye ett par bøsse Skud liggendes fra denne gaard \textit{Aas} i Nord ost.\par
Til det l{te} Spørsmaal, Svarer ligesom 4{de} viidne denne gaard \textit{Aas:} at ligge fra Stuedal, viidere i øster paa dend kandten er hand ej beckiendt ‒\par
Til det 2{det} Spørsmaal forcklarer hand Landskabet Med viidere fra Stuedahlen hiid til \textit{Aas}, ligesom 4{de} Viidne.\hypertarget{Schn1_19699}{}5 Vidne i Selbo Fogderi.\par
3 Spørsmaal Svarer, Som 4{de} Viidne ‒\par
Til det 4{de} Spørsmaal Siiger hand om \textit{Jongdals} gaardene i \textit{Herjedalen} at have hørt det Samme, som 4 viidne har udsagt: men j \textit{Handøl} har hand været i \textit{Jemteland}, og Stadfæster derom det 4{de} viidnes udsagn. ‒\par
Til det 5{te} Spørsmaal, Svarer: Det Sydligst(e) Fie(l)d, hand har værit paa, er \textit{Remmen}, hvor over hand har faret vejen fra Tydalen til \textit{Handøl} i \textit{Jemteland}, og giver om dette \textit{Remmen} Samme forcklaring, Som 1{te} viidne: dog om dend Elfv, \textit{Eina} østen for \textit{Remmen}, Rindendis, veed hand ej at Siige, hvor fra dend har Sit udsprang. Dette legger hand til: at hand fra det Field \textit{Remmen} har Seed det Field \textit{Sylen} i Søer, og dend Field Rygg af \textit{Sylen} i øster kaldet \textit{Stoersola}, Som er Fieldfast med \textit{Sylen;} tædt under denne \textit{Stoersola} i øster kunde hand iche andet forstaa, end \textit{Eina} Elfv maatte flyde.\par
Viidere enten i Søer eller Nord har hand iche værit.\par
De 6{te} Spørsmaal \textit{Cesserer} ‒.\par
Til det 7{de} Svarer? Fieldene i Sig Selv ere u-frugtbare og ansees for ud-ørckener; hvad Fiskerie og nogen liiden høe-Slott imellem Fieldene kand være, det ligger til Tydalens bøjd. ‒\par
Til det 8{de} Svarer Nej. ‒\par
Til det 9{de} Svarer, Som 1{te} Viidne ‒\par
Til det 10{de} veed hand indtet at Svare ‒\par
Til det 11{te} Selv har hand iche værit viidere, end at hand har faret dend vej imellem \textit{Handøl} j \textit{Jemteland} og Tydalen: men af hans Fader og Gamble Folck har hand hørt følgende Grændse Mærcker, imellem Tydalen og \textit{Jemteland} fra Søer at reigne, Nembl: (1) \textit{Biskopsaaen}, for det (2) \textit{Helags} Stødten, for det (3) \textit{Storsola} østen for det Field \textit{Sylen}, hvor af det er een Ryck og Sammenhængig, for det (4) det Field \textit{Hammer}, kaldet blaa- \textit{Hammer}, Som Sammenhænger i væster med \textit{Snasahougene}, og for det (5) \textit{Einbogen} men om der har værit nogen vahrer eller Mærcker nogen Stædtz opsatt, veed hand iche ‒\par
J anleedning af dette Svar blef næst forrige 4{de} viidne indkaldet, og ført til gemytte: om hand kunde erindre Siig, og i kraft af hands Eed med Sandhed Siige: om hand har hørt, at foruden de af ham opReigned grændse Stæder, følgende Fielde Skulde og være Lande Mærcker Nembl: (1) \textit{Helag} Stødten (2) \textit{Stoersola}, østen for det Field \textit{Sylen}, Fieldfast dermed og for det (3) \textit{Blaa Hammer?} hvortil hand Svarede ja, det har hand hørt fra hands barndom. ‒\par
Til det 12{te} Svarer fra \textit{Ein} Elfven hid til Østbye gaard ved \textit{Aas} er 4{re} gamle Miile, ligesom der og 4{re} gamle Miile er fra denne \textit{Eina} Elfv til \textit{Handøl} gaardene i \textit{Jemteland} ‒\par
Til 13{de} og 14{de} Svarer, Som 4{de} viidne, og hand blef \textit{dimiteret}. ‒\hspace{1em}\par
Der paa fremckaldede Rætten een Lapp-\textit{Fiind}, Som efter Lensmandens \textit{Jon Biørnsen Aas} hands foranstaltning efter \textit{Major Snitlers} forlangende med 2{de} andre yngere Lapp Findner fra \textit{Tydals} Fieldene her neer til \textit{Aas} gaardene, anckom; Samme \textit{Fiind}, Som den ælste af dennem, efter foregangne \textit{examination} og forcklaring om Eedens betydenhed i bøjde- \hypertarget{Schn1_20009}{}Schnitlers Protokoller I. Skolemesterens Svend \textit{Helleins} overværelse blef tagen i \textit{Corporlig} Eed, og der efter paa følgend Maade \textit{examineret}\par
6{te} viidne ‒\par
1: Spørsmaal, hvad hand heeder? hvor hand er Føed? af hvad Forældre? hvor gammel? og om gift? og hvor hand nu til holder?\par
\textit{Resp:} hand heeder \textit{Morten Nilsen}; Er Føed i et \textit{Jemte} Field, Som ligger til \textit{oviigens} Præstegield; af \textit{Finde}-Folck og var hands Fader lensmand iblandt \textit{Findnerne}; hand er 62 aar gammel; ugift; hand vancher fra ett Stæd til det andet paa Fieldene; Stundum har hand værit paa dem ved \textit{Røraas:} men mæste tiiden paa Fieldene hen ved det \textit{Liusendalsche} kaabber værck i \textit{Herjedalen}, og Sal: \textit{general Budde} har meget brugt ham til Skytterie ‒\par
Denne \textit{Find} viste vel at forstaa Norsk og at Svare paa Norsk ‒\par
2 Sp: Om hand viste: om hand var Døbt? og hvor? om hand har gaaet til gudsbord? naar ‒ og hvor Sist? om hand veed, hvad en Eed betyder? og om giort Eed før?\par
\textit{Resp:} Joe; hand er Døbt i \textit{Heede} Kierche i \textit{Herjedalen}, hand gaar til guds bord 3 gange om Aaret, og Sist har hand gaaet ved det \textit{liusendalsche} brug i \textit{Herjedalen}; hand har og giort Eed før paa \textit{Herjedals} Tiingstæd\par
3: Om Viidnet veed, hvad Skilnet giør imellem \textit{Herjedalen} og \textit{Jemteland} ved de Norske grændser i væster ‒\par
\textit{Resp:} hand Stadfæster det Samme Som 1{te} viidne ved 10 Spørsmaal har \textit{deponered}, og Sagde det var Saa ‒\par
4: Hvilcket ere de rætte grændse Mærcker, Som viidnet kiender eller har for gammel tiid hørt, at være imellem Norge og Sværrig paa denne kandt, og om der findes nogen vare eller andet kiendeteign nogen Stædts opsatt? ‒\par
\textit{Resp:} (1) \textit{Rutten} og der af de høyeste klimper, (2) \textit{Haftor}-Stødten, Som ligger fra \textit{Rutten} 3 gamle miil, og Rætten meener, kand være 1 1/2 Nye goede Miile (3) Biskopsaae, paa hvis Sydere Siide een Steen-vahre, hand har hørt af Sl: \textit{general Wiibe} Skal være opreyst og af Smaa Steene Sammen Sadt, hvor af noget Skal være affalden, og det øfrige, Som igien Staar, betydede hand, at være Som en 1 1/2 allen høy, dend hand og i Sidst afvigte Sommer har Seed med viidere som 4 viidne. Denne biskops-\textit{aaen} skal ligge i nord fra \textit{Haftor} Stødten Een god eller Nye Miil. for de (4) \textit{Stoersola} et Field; hvilcket hand Siiger, ligger fra \textit{Nea} Elfv 1: Miil, og denne \textit{Nea} Elfv at ligge fra biskops-\textit{aae} en 1/2 Miil; paa dette Fie(l)d, Sagde hand, ingen Vahre at være opsadt. (5) \textit{Einbogen}, hvor og ingen Vahre Skal være opsatt. og Som hand iche viidere i Nord var beckiendt, blef hand \textit{dimitteret}, og Rætten paa dette Stæd Sluttet. og af LougRættet tilligemed underskreevet og forseiglet.\hspace{1em}\par
Der efter blef og i Rættens overværelse Lænsmanden af Tydalen\textit{Jon Aas} betydet, at imod grændse Maalernis anckomst, Som vil blive omtrent 14 dage efter tilstundende \textit{Sanct Hans} tiid maa hand See der hen, at i tiide klopper, Flotter, broer eller hvor haves kand, baader færdig haves, over Søer, Elfver, og vande, at komme over paa, saa maa hand, og Saa Snart de anckomme paa Tydals grænser, indfinde Sig med de viidner, Som i Stuedahlen og her paa \textit{Aas} ere afhørte, Saa og med nogle de kyndigste mænd til at viise maalerne \hypertarget{Schn1_20222}{}Bilag A: Om Tydalen. grændsens rætte gang, Saa og nogle Arbejds folck, til at drage kieden, og at giøre anden forefaldende arbejde. ‒\par
\centerline{(sign.) Peter Schnitler. (L. S.)}\centerline{(sign.) Jngbrigt BiørnsenAas (L. S.)}\centerline{(sign.)Johannes OlsenAas (L. S.) }
\DivII[Liste over bilag]{Liste over bilag}\label{Schn1_20255}\par
\centerline{Følger nu \textit{Bielagerne} For \textit{Selboe Fogderie}. ‒}\par
Lit: A: Om \textit{Tydalen} en Norsk bøjd ad \textit{Jemteland} ‒\par
Lit: B: Dend Svenske-Finlandske Armees \textit{Fatale} udgang fra dette Tydalen i Januarij: 1719.\par
Lit: C: \textit{Relation} om \textit{Lap Finnerne}, Som begyndes i de Tydalske Fielde, med u-forgribelig Anmærchning ‒
\DivII[Bilag A-C]{Bilag A-C}\label{Schn1_20300}\par
Lit: A: \centerline{Om \textit{Tydalen} grændtzende i øster til dend Svenske \textit{Provintz Jemteland} ‒}\par
Dette Tydalen har Sit Nafn af \textit{Tya} Elfv der fra begyndelsen ud af \textit{Miøsiø} under \textit{Stor} Fieldet udkommer og igiennem Stue-Siø Siiden \textit{Møen} Søe Rinder i \textit{Nea} Elv under den gaard \textit{Aas} i \textit{Tydalen} og her taber Sit Nafn. Det er \textit{Annex} af \textit{Selboe} Præstegield i \textit{Selboe} Præstegield i \textit{Selboe} Fogderie, hvor 4 gange om Aaret Præckes, og bestaar af 20 bøndergaarder ‒\par
Der er en trang dahl, bestaaende af Søer Elfver Skouge og bacher, og paa Siiderne omgivet med Fielde nembl: imod øster ad \textit{Jemteland} med \textit{Schar}fieldet og \textit{Sylen} Field paa Søndere Siide ad \textit{Aalens Annex} med \textit{Løfunden} Field i Nord med \textit{Øye} Field og \textit{Fongen} ad \textit{Mærrager} bøjd, og paa dend væstere Siide hænger det med \textit{Selboe} hovet Sogn Sammen ved en Skoug;\par
Det Stræcker Sig fra øster liige i Væster, Saaleedis, naar det øede u-frugtbare grændze field fra \textit{Bischopsaaen} eller \textit{Helag}-Stødten til Stuegaardene ej reignes ‒\label{Schn1_20432} \par 
\begin{longtable}{P{0.4209905660377358\textwidth}P{0.16639150943396228\textwidth}P{0.1623820754716981\textwidth}P{0.05613207547169811\textwidth}P{0.04209905660377359\textwidth}P{0.0020047169811320755\textwidth}}
 \hline\endfoot\hline\endlastfoot \multicolumn{5}{l}{fra første østlige gaard \textit{Stuedalen} til gaarden \textit{Løføyen} ‒}\\
\multicolumn{3}{l}{1: gaard 2 1/4 gamble kand være Nye}\tabcellsep 1 1/2\tabcellsep Miil\\
her fra\tabcellsep til \textit{Fossum}\tabcellsep 1 gaard\tabcellsep 1/4\tabcellsep miil\\
her fra\tabcellsep til \textit{Kierchvold}\tabcellsep 1 gaard\tabcellsep 1/8\tabcellsep »\\
her fra\tabcellsep til \textit{Østbye} ‒\tabcellsep 4 gaarder\tabcellsep 1/8\tabcellsep »\\
 - - \tabcellsep til \textit{Aas} ‒ ‒\tabcellsep \multicolumn{3}{l}{6 gaarder ‒ ligger Strax derved}\\
\multicolumn{2}{l}{Til \textit{Oune} ‒ ‒ ‒}\tabcellsep 1 gaard\tabcellsep 1/4\tabcellsep »\\
\multicolumn{2}{l}{til \textit{græslien}‒ ‒ ‒}\tabcellsep 3 g:\tabcellsep 3/4\tabcellsep »\\
\multicolumn{2}{l}{til \textit{Hilmoen} ‒ ‒ ‒}\tabcellsep 1 gaard\tabcellsep 1/3\tabcellsep miil\\
\tabcellsep \tabcellsep \multicolumn{3}{l}{_________}\\
\tabcellsep \tabcellsep \tabcellsep 3 1/8\tabcellsep Miil.\tabcellsep ‒\end{longtable} \par
 \hypertarget{Schn1_20571}{}Schnitlers Protokoller I.\par
J de østlige gaarder Saaes aldriig Korn og neere i bøjden voxer det Sielden, at det bliver moenet, Folckene leever af deris qvæg og nogen Fiskerie, og bruger paa Timmer hugster til Saugene. Dend Stoer Elfv \textit{Glommen} opckommer i dette Tydalen, og \textit{Nea} Elfv Riinder her igiennem ved \textit{Trondhiem} i Havet, om hvilcke begge i \textit{Protocollen} er talt nembl: \textit{Stuedahlen} af 1{te} viidne ved 6{te} Sp: og paa \textit{Aas} ved 2 Sp:\hspace{1em}\label{Schn1_20611} \par 
\begin{longtable}{P{0.7777011494252873\textwidth}P{0.07229885057471264\textwidth}}
 \hline\endfoot\hline\endlastfoot Vejen til \textit{Jemteland} den \textit{ordinaire} gaar fra \textit{Østbye} gaarder til Øye-field kand Reignis\tabcellsep 1 Nye Miil,\\
der fra igiennem en liiden biercke Skoug ved dend væstere Ende forbie dend Sieø \textit{Eesand} i øster til \textit{Remmen}\tabcellsep 1 Miil\\
der fra igiennem een liiden bircke Skoug til \textit{Olvaa-Køl}\tabcellsep 1 Miil\\
her fra igiennem Skougen til \textit{Handøl} de 2{de} første gaarder i \textit{Jemteland}\tabcellsep 2 Miil\\
\multicolumn{2}{l}{_________}\\
Nye\tabcellsep 5 Miil ‒\end{longtable} \par
 \par
hvilcken vej de Svenske A{o}1719: toge her af Landet, dog til Krogs 1 mil længere, hvor om næste bielage giiver forcklaring. ‒\par
Een anden vej fra Tydalen til \textit{Herjedalen}, tages om Sommeren, men ej om Viinteren for det haarde Field skyld, fares nembl:\label{Schn1_20699} \par 
\begin{longtable}{P{0.7268551236749116\textwidth}P{0.12314487632508833\textwidth}P{0\textwidth}}
 \hline\endfoot\hline\endlastfoot fra Stuedahlen til \textit{Schardøeren}\tabcellsep 1/2 Miil\\
fra Skardøeren igiennem Skardøeren\tabcellsep 1/2 Miil\\
Siiden over det bare viilde field i Sydost til voldalen øverst i \textit{liusendalen} i \textit{Herdalen} ‒ reignis\tabcellsep 3 Miil\\
Fra \textit{Woldalen} til \textit{liusendals} brug\tabcellsep 3 »\\
\tabcellsep \multicolumn{2}{l}{_________}\\
giør\tabcellsep 7 Miile\end{longtable} \par
 \par
Dend 3{die} vej herfra Tydalen til \textit{Jongdalen} i \textit{Herdalen} Skeer Saaleedis:\label{Schn1_20779} \par 
\begin{longtable}{P{0.6725917431192661\textwidth}P{0.14621559633027523\textwidth}P{0.031192660550458717\textwidth}}
 \hline\endfoot\hline\endlastfoot fra Stuedalen igiennem Skardøren\tabcellsep 1\tabcellsep Miil\\
Derfra tædt paa dend Syndere Siide af \textit{Helag} Stødten\tabcellsep 1 1/2\tabcellsep Miil\\
Der fra til \textit{Grøndøren} en \textit{grubbe} i fieldet Siiden til \textit{Jongdal}\tabcellsep 1/2\tabcellsep Miil\\
\tabcellsep \multicolumn{2}{l}{_____________}\\
Er Reignet for 8 gamle og nu for nye\tabcellsep 4{re} 1/2\tabcellsep miil\\
Der fra viidere til \textit{Storsiø} 5 gaarder i \textit{Herdalen}, 3 gamle, nu Nye\tabcellsep \multicolumn{2}{l}{1 1/2 Miil}\\
der fra til \textit{Heede} Sogn 3 gl: nu nye\tabcellsep \multicolumn{2}{l}{1 1/2 Miil}\\
\tabcellsep \multicolumn{2}{l}{_____________}\\
Nye\tabcellsep 7 1/2 Miile\end{longtable} \par
 \hspace{1em}\par
Paa denne Samme Vej fra Stoer Siø i \textit{Herdalen} kan og igiennem en Skoug til \textit{Oviga} gield i \textit{Jemteland} fares: dog besværligen og om Sommeren kun. ‒\par
Vel kan fra østbye i Tydalen om viinteren Sidst paa Skarre-føerret til \textit{Handøl} reises dog er det 2{de} Dages Reise, og ingen huuse undervejs, men man maa ligge ude om Natten.\par
Dend beste vej til \textit{Herdalen} om viinteren og Sommeren er fra Stuedalen øvest i Ty\hypertarget{Schn1_20923}{}Bilag B: Armfeldts tog 1719. dalen igiennem \textit{Miøssdal} og \textit{Riidalen} i Søer til \textit{Bræche} gaardene over Rutt fieldet til \textit{Funnesdalen;}\par
Endelig ved disse Tydals fielde og der omckring er at mærcke, at de udenlandske Falche-fængere komme her viisse tiider om Sommeren og fange Falche; og have her Samme visse bønder, som bringe dem Fersk Mad til tiid efter anden til deris Underholdning, medens de paa Fieldene i deris hytter haver deris tilhold. ‒\par
J Tydalen er 10 Mand Skiiløber Soldatre og i næste Væstere \textit{Selboe} og \textit{Kleboe} Sogner 10 Skiiløbere og 1 a 2: Staaendes \textit{Compagnier} ‒\par
Mærckværdigt er fra denne \textit{Tydals} bøjd 1 1/2 mill i væster i \textit{Selboe} hovet Sogn ett kaabber værck, Som for meere end 30 aar først blef begyndt, og for \textit{Malmens} Slethed needlagt; Nu er det for en 6 aar igien optaget, og drives \textit{gruben} ohngefehr af en 40: mand, og \textit{Hytten} naar dend gaar, af een 20 mand: dog \textit{Malmen} har værit Ringe og bliver Ringe. ‒\hspace{1em}\par
Lit: B:\par
Den Svenske-\textit{Finnsche} Armees \textit{Fatale} udgang, under \textit{Commando} af \textit{General Lieutenant Arnfeldt}, her fra Tydalen ud af \textit{Trondhiems} Stift in Januarij 1719{ten}.\hspace{1em}\par
Den Svenske-Finnske Arme laae i begyndelsen af def 1719{de} Aar i Guldalen\textit{indqvarteredt} for at være, Som hørtes, nær ved det Syndenfieldske, om deris \textit{general Lieutenant Arenfeldt} af hans \textit{Konge} Kong \textit{Carl} dend XII, Skulde blive beordret at Støde til dend Svenske hoved \textit{Armee} Synden Fields. Der nu bem{te}\textit{general Lieutenant} fick dend Tiidende at høyst bem{te} deris konge for \textit{Friderichshald} var falden, og dend \textit{ordre} at træcke Sig udaf \textit{Trondhiems} Stift til Sverrige tilbage; loed hand Sine under havende \textit{Troupper} Neederst fra Guldalen Strax SammenRycke til ud \textit{march}, Denne ud \textit{March} havde nu lættest kunde Skeed paa dend \textit{ordinaire} vej fra guldalen over Rutt fieldet ad \textit{Herjedalen}, men til deris u-lykke opfangede \textit{GeneralLieutenanten} ett brefv Skrevet fra østerdalen Syndenfields, at de Synden fieldske Norske Troupper vare i \textit{anMarch} imod dennem, for at \textit{delogere} de Svenske udaf de \textit{Trondhiemsche}, det kan og have verit \textit{general Lieutenantens} aarsag at i \textit{Herdalen} ingen underholdning for hands folck var, men at hand viste, at der af bedre Forraad i \textit{Jemteland} var at finde;\par
Hvor fore hand Satte Sin \textit{Armee} i \textit{march} nord efter fra dend gaard \textit{Grøt} i \textit{Holtaalens} hoved Sogn over det field \textit{Bokhammer}, at komme need til de yderste gaarder \textit{Floren} i Sælboe Præstegield helst hand ej havde berørt dette \textit{Se(l)boe} Præstegield før med Sine Troupper og derfore tænchte der fra at forSyne Sig med \textit{fourage} og \textit{proviant} paa vejen til Sit hiem. Der bleve og \textit{partier detacherede} need i Selboe gaardene at Sammen bringe, hvad høe og \textit{provission} der var: men da de fick See de Norske Skiiløbere at anckomme imod dem, \textit{Retirerede} de Svenske \textit{partier} Sig op til hovet \textit{armeen} og efter loede alt. ‒\par
Før omtalte første Field \textit{Bokhammer} som de Svenske fra \textit{Holtaalen} overgick need til Selboe, er ett haardt bart Field 2 Miile over need til de første benæfnte gaarder, og der \hypertarget{Schn1_21207}{}Schnitlers Protokoller I. begyndte de af kuulden at falde, og bleeve nogle hundrede der igien liggendes døede; hvor i blandt een Præst med kalchen i hænderne fandtes.\par
De Svenskes \textit{March} gick da østerlig ud af det Selboeske ind ad dette Tydalen een field bøjd af een 20 Smaa bønder gaarder, Som før er meldt; her var nu liiden eller ingen ophold for ham at faae, thi maatte hand Strax fortsætte \textit{Marchen} viidere i øster; om Morgenen Saae det ud til godt vejr, thi opbrøed hand fra Østbye gaard i Tydalen, Som ellers er dend Vanlige vej til \textit{Handøl} i \textit{Jemteland}, til \textit{Oy} fieldet, et haardt og Strængt field, Een Miil over af Størelse liggendes fra berørte østbye gaarder een 1/2 Miil, førend Folcket kom til dette \textit{Øye} Field begyndte u-vejret af Streng kulde og Snee-drev at de ej Saae hvor, eller hvor hen de gick; over denne Tunge \textit{March} i Snee og Dreev, faldt folcket af kuld og afmagt need i haabe Tall; \textit{Armeen ava(n)cerede} da fra dette øyfield need i een liiden biercke Skoug, hvor de omlagde Varme een Stund, dog fandtes og her mange Døede 10 a 12 Mand hos hinanden ligge paa hvert Stæd ‒\par
Denne Skoug Dahl var kun 1/2 Miil over, og da maatte de til det field \textit{Remmen}, liigeleedis ett haardt field 1/2 miil over; Der Styrdtede Folckene af kuld og heste af Suldt; fra \textit{Remmen} kom de igienleevende i een Dahl, kaldet \textit{Eendalen} 1/8 miil breed, hvor de fleeste Folck og heste laa døede, hine liggendis paa hinanden dynge Viis; her var det at Folcket hafde Slaget Skiæfterne af deris gevæhr og Pistoler, og optændt deraf Varme, hvilcken naar den var Slucknet, var alle folckene der om liggendis døede. ‒ Af denne Dahl, skulle nu \textit{Marchen} Efter dend \textit{ordinaire} Vej have fremgaaet over det field \textit{Olvaa} Kiølen need ad de første gaarder \textit{Handøl} i \textit{Jemteland}; men \textit{general lieutenanten} Torde ej vove at gaa fleere fielde; thj forfuldte hand \textit{Marchen} i denne Dahl langs \textit{Eina} Elfven een goed Miil til krogs forbie fieldet, hvor For-Troupperne maatte Natten over ligge og i denne Dahl fandtes u-tallige døede, Saa tygt Strøede paa Jorden Som med Slaget græs; og Ræsten af \textit{Armeen}, som igien leevede vandt derfra til \textit{Handøls} 2{de} gaarder. ‒\par
Norske ved grændserne boende bønder fienge her godt bytte af klæder og andet: Skiiløberne toge de Svenske efter ladte \textit{Metallene} feldt-Støcker med Gevæhr, Saa meget de kunde føre, førend Sneen Skiulte kropperne. og det kom bønderne vel tilpas, at kiøbe der for korn og føede, Som de Svenske havde fra dennem fortæred; om Vaaren efter, var til de Stæder ej fremckommendes for Stanch men der Saaes mangfoldig Ræve og viilde Rof-Fugle.\par
Dette er en \textit{notoirement} bekiendt Sag, som Folk leve endnu, der have været og seet det.\hspace{1em}\par
Lit: C:\par
\centerline{\textbf{Relation} om \textbf{Lap Finnerne!} Her i \textbf{Tydalen} forefandt ieg de første \textit{Lap Finner} Som og vel ere de Sydligste her \textbf{Nordenfields;}}\hspace{1em}\par
Det er et Folch liiden og Mager af vexter, og lech-øyet af dend jidelig Røeg i deris Køyer el: Field hytter hvor de boe i, Siunis Eenfoldig u-skyldig, Som indtet ondt giør no\hypertarget{Schn1_21352}{}Bilag C: Om Lap-finnerne. gen, og er lydigt naar de af bøjde betiendterne faa bud at møde. De have haft deris Egen Særdeelis \textit{Religion} førend de ved dend høyloflig \textit{Missions} anstalter er bragte til dend Christelige Kircke; de har endnu deris eget Sprog, Sæhrdelis Sæder og leevemaade og gifte sig iche uden med deris Egne; Paa deris anseende ere de og kiendelige fra andre Normænd ‒\par
Deris Gudstieniste angaaende, Saa holdes de af de beSkichede \textit{Missionaires} til kierckens Samfund ved Daab- og Alterens \textit{Sacramenters} brug Sambt anden Kierckelig tieniste, mens for de tiider have de værit meget blinde og til overtroe genegne; det er iche mange aar siiden at een bonde her af Tydahlen har paa fieldene dog paa de Svenske grændser, fundet et Træe billede Som af een bierck var tilhugget, og Skaaret efter ett Menniskes liignelse med hovet og Fødder, hveed hvilcket mange Aadselebeene af Creaturer ere Seede, Saasom af hæster gedder \textit{etc}: hvilcke de i bøjdene maa have kiøbt og til dette Træbillede fastbunden, ladet Sulte ihiel; Paa hvilcken Maade de skal have ofret til deris af-gud, den de have kaldet \textit{Harkel-gud}: dog er det længe Siiden Skeed, thi da dend bonde har Seed dette billede, har det værit næsten igiennem raadnet, og fornemmer man nu omstunder icke til Saadant noget hos dennem.\par
Med deris Rund-bommer have de og haft megen overtroe i fordum tiider, i det de Sig indtet foretog førend de liigesom Hafde adspurdt dette deris \textit{oracul}: De bommer vare Runagtige og huule, Som omtrent ett \textit{Jnstrument, Luth}, dog uden Saa lang hals, betrekket paa dend eene Siide med Reenkalvskiind Som een Tromme, malet med adskillige \textit{Figurer} og behengt med riinge og nogle fugle fiere, naar de nu med en liiden hammer eller Sticke Sloge paa Skiindet, da, efter som Ringen faldt i ett godt eller ondt Tegn af \textit{Figurene}, efter deris indbilding, der efter vedblev, eller efter loede de deris forehavende, dog fornemmes nu omstunder ej til, at de have eller bruge Saadanne bommer, i det minste veed man det iche. Dend overtroe skal de have endnu at, naar Døeren paa deris Køyer er paa dend eene Siide, Saa have de tvært der imod paa dend anden Siide een Glugge, eller ett Fiirckandtet hull, omtrendt imod een allen-stoer; Naar de nu vil bruge deris gevæhr til Skytteric eller de have Skudt eller Slagtet noget Dyr, Saa maa alt det icke igiennem dend \textit{ordinaire} Døer, men igiennem denne \textit{glugge} Stickes og indføeres. ‒ Angaaenee Rætten og \textit{d(i)ciplinen} imellem Sig; Saa fornemmer man ej, at de nogen tiid komme til bøjde-Ting med nogen Trædte eller \textit{process} imod hinanden; De have Som en Lensmand, eller oldermand iblandt Sig hvilcken med fleere ælste træder Sammen, og, naar uskickelighed med Rapperie eller andet kand være foregaaet, finder dend Skyldige efter forseelsens beskaffenhed, enten til at bød dyer, eller om iche har at Pidskes paa bare krop, eller bagbunden at kastes paa jorden, eller bindes til et Træe der af Myggene og utøyg at Stickes og plages een tiid lang, ellers erfaris iche meget i bøjdene, hvad, eller om u-dyd af Findnene i fieldene begaaes, thj de ere for langt fra dem, og det folck tier den Eene med dend anden ‒\par
Een Find fortalte mig, at een af de Svenske Finner paa \textit{Liusendals} fielde, havde Skudt Elsdyer i forbudne tiid, hvor for, da det blefv beckiendt, hand ved Tinget blev Dømbt, og maatte løbe gade-løb eller Spitsrod i \textit{Herdalen}, igiennem 50 mand nogle gange. De skifte og Selv imellem Sig arven efter deris forældre: dog naar nogen formuendes \textit{Finn} i de Svenske fielde døer, kommer Lænsmand og mænd af bøjden, op at Skifte boet efter \hypertarget{Schn1_21421}{}Schnitlers Protokoller I.\textit{Finden} ‒ angaaende deris \textit{øconomie} og Næring: Saa boe de i een liiden trang Køye, eller hytte, midt i hvilcken paa Jorden een, omtrendt Fiirckandtet af Steene Sammen Muuret Jild-Stæd dog uden Røer op ad, er Satt, fra denne steen gaar Røegen op af Køyen, igiennem een glugge eller hull op i Spitzen af Taget; Køyens bygning bestaar af biercke- eller gran Stænger paa hver Siide, hvilcke oven til krum- og nær Sammenføjede, fæstes, Saa at det er viidt needenfor og gaar Spitz Sammen oven til, der hvor Røg gluggen Skal være, omtrent dannet Som ett Soldatter Tælt, dog liidet Større, efter Som Finners \textit{Familie} kand være Stoer eller liiden til; Om disse 4{re} bierche Træer paa begge Siider, Som er \textit{Fundamentet} og heele Køyens Timmerage, Slaaes nu vadmel, eller og om de bygge noget tiilig om høsten, becklæde de disse biercke Træer eller Stænger, med Næver-barck af bierck, og legge her paa Siiden Jord-Torv rundt omckring; J denne Køye holder nu een \textit{Findes Familie} til, Nembl: Mand og Kone, børn og tienere, om de nogen have, de bruge ingen Stoele, thi de Sidde med deris korsviis over hinanden lagde fødder, paa \textit{posteriora;} de Sidde Saaleedis, de ligge, de Spiise de koge og arbejde Sammen derj\par
Senge-Rommerne ere paa begge Siider, der legge de noget bierche Riis paa jorden, Siiden Reen Skiind at have under Sig, og Reenskiind oven paa at dæcke Sig med; Her i disse fielde er mændernis gierning, at gaa paa Skytterie, fiske med Angel (thi garn have de iche,) at jæte eller vogte dyerene, jtem at koge, og Skifte Maden: Qviind folckenes arbejde bestaar mest i at Sye Fiind-Madter, eller kioler, Støvler, Skoe, og handsker af Reenskiin, her hielpe de og mændene med at Mælcke Dyerene, men i Nord, befatte Sig iche med noget, Som hører til Maden ‒ Naar Mændene Skyde et Els- eller Reens dyer, beholde de kiødet Selv at Spiise, og Sælge Skiindet i bøjderne. De klæde Sig med Reenskiind, og Vadmel Som de kiøbe i Bøjderne; thi lærit paa deris krop, eller lagen i deris Sænge ihvor Riige de ere, viide de ej af, Som de ej viid af hamp, el: liin, eller at Spiinde; Vadmel maa de have for Reignens Skyld, hvilcken deris Fin Mudter og Skiincklæder ej taale. Deris føede have de af Reenene, bestaaende af kiødet Melck Smør ost, Saa af Fugle de Skyde, og noget liidet Fiskerie af Field vandene. Mælck-Maden have de om Sommeren, og om Viinteren naar dend ophører, kiødet Som de Tædt Sammen Snøre om høsten liigesom i ett Meise greie med Næver omlagt og \textit{Conserverer} viintern igiennem uden Salt, Siiden Skiære de det i Tynde Skiiver, og viind-tørcke; Marcken eller Smaa orme kan vel Sætte Sig uden paa, men icke æde Sig igiennem, fordi det er Saa fast Snørt; thi naar det imod Vaaren fremtages, er det inden til frisk, naar kiødet dem mangler, komme de i bøjden at kiøbe mel, af nød, ellers icke af nogen lyst, deris Drick er vand, og om Viinteren Smælted Snee: ellers naar de have koget kiødet, dricker de vandet og Sausen, hvori kiødet er kogt; Til det feedeste de æde, er kiødet i stæden for brød, og anden brød have de iche. ‒\par
Om Sommern kan de have Visse bøjde Fielde, hvor de \textit{ordinair} tilholde, dog, naar et Stæd eller Myhr græs Plats er afbeedet, fløtte de med Dyerene til andet: om vinteren naar Sneen paa de fielde bliver for dyb, eller kuld frossen, at Reenen ej kan Sparcke den af, for at faae Maassen af fieldet, Som er det dyers eeniste føede om viinteren, Saa maa \textit{Finnen} flødte anden Stædz hen til Skougen hvor Mennisker have \textit{luer} og Varme, og hvor Sneen er løsere, eller iche Saa tyck; thi \textit{Finnen} Slaar eller høster aldriig høe ind om Som\hypertarget{Schn1_21466}{}Bilag C: Om Lap-finnerne. mern, at have foeder Viinteren over, der til har hand og ingen Lade- eller huuse, men Reenen gaar ude viinter og Sommer, og føeder Sig Selv om Sommeren med Græs, og om Viinteren med Maase paa fieldene, hvilcke derfor \textit{Finden} maa jæte eller vogte imod Ulfven aleene Som er efter dem, men biørnen tage dem icke; thi hand kan iche opløbe dem; ulven derimod meere underfundig, Søger Dyrene iche paa dend vej, hvor Vejret staar fra ham (ulven) til dem, thi da kunde de af lugten kiende hands komme: men hvor Vejret Staar fra Sig til Dyerene. Der fra Stiæler hand Sig ind paa dem, og naar hand har Sneeget Sig Saa nær, giør hand iblandt hoben et Spring, og Snapper den, hand Slaar i. Folckene ere meget haardføre til at taale kuulden, og enskiønt gamle, dog Myg og Raske; det er Sæhrdeelis hos dem, at Mand folckene, enskiøndt gamble, have kun liidet Tyndt Skiæg om Munden, \textit{comme du poel Folet} at, om man der af maatte Dømme om deris \textit{Naturel}, Saa maa de være blødagtige og \textit{Femenins} af hierte; thi de efter deris leevemaade ej Som andere Bønderfolck, ved Arbejde eller lang Fære og March udslides, de ere heller iche af deris \textit{Natur}, (Som er vel optræckelsen at tilskriive) genegne til Stadighed og Arbejdsomhed, Efterdi man veed en Præst i Selboe havde taget ett ungt fattigt Findbarn til Sig at opfostere, det hand loed lære læse, Skriive og arbejde: men Saa Snart det voxte til dend alder af en 15 aar, undviigede det udaf bøjden til Sit \textit{Finne}-folck; vel fare de meget og langt, men det Skeer paa Skier, eller i deris \textit{Kieriester} med Reen-dyr for; paa Skierne Som er en 4 alen lange Tynne, foed breede, lætte Træe Skier, kan de Rænde i godt føere over Sneen, Saa Stærckt Som een hæst, deris kørsel om viinteren gjøre de liigeleedis over Sneen i \textit{Kieriester} med Reen for, hvilcke løbe dermed langt Snellere, end at dend Raskeste hund kan følge dem, diss \textit{Kierister} ere een Slags kør Sleder af Tynne Træbord bygde flade Needen under, og Rund op ad Siiderne, Som een liiden baad dog uden køel og Mejer, og icke længere eller breedere, end att ett Menniske paa bonden med udstrachte fødder kan Sidde derj, oven til aaben, J dend kierist, breede de nu ett Reenskiind, hvorpaa Persohnen Setter Sig need og ett Reenskiind oven paa, hvormed hand deccer Sig til op under Armene; Saa Snøres hand fast i kieristen med ett band eller Smalt Reeb, Som drages igiennem hullerne paa begge Siider af \textit{kieriesten} Tvært over det øfre Skiin decce, Saa at hand ej kand falde ud, naar \textit{Kieriesten} i dend fastefart kom til at omvælte. Persohnen faar til dend Ende een kort tyk Stvav i haanden at Styre og Rætte denne \textit{Kierist} med, naar den paa nogen Siide vil hælde eller omckante, thi den er ett ganske lætt fartøj; Dyret at Regere, har dend kørendes een Tømme kun i haanden paa den eene Siide af Dyret liggendis, hvilcket først om hornet bunden, holdes af dend kørende; naar han nu vil vænde Rechte om, kaster hand dend eene Tømme paa højere Siide, og Drager Dyret did, naar hand vil Linchts om, kaster hand det paa vændstere Siide; Og Saaleedis farer da Persohnen ved god føere een 2 ja 3 miile i hver Tiime fort; Saa at \textit{Finnen} føedes, klædes og føeres af Reenen, og deris formue og riigdom bestaar i Reen-Dyer. ‒ Saadanne \textit{Finne Familier} er nu 3 i disse \textit{Tydals} fielde til, Som holde Sig til denne kiercke og \textit{Misionaire}, ej reignendes, at andre løse \textit{Finner} kand Som Snareste der komme og fare bort igien, Fleere \textit{Finner} have gerne Som jeg fornemmer, villet needsette Sig; men deris dyer komme bøndernis Eng Slott for nær, Som de have for deris Egne \textit{Cratuere;} hvor fore de have afviist dem; bønderne fortælle og; førend \textit{Finnerne}\hypertarget{Schn1_21534}{}Schnitlers Protokoller I. kom hid til Tydals fieldene, hafde de viild Reen nock, men efter at disse \textit{Finner} var kommen, er ingen viild Reen at See; Disse Tydalske\textit{Lap-Finner} Skal nu kun være efter deris viis maadelig riig, at Skatteres for at eje hver een 80 Reen meere og mindere. ‒ De \textit{Liusendalsche Finner} paa de field til \textit{Liusendals} brug i \textit{Herjedalen}, Skal og være gemeenlig en 3 \textit{familier} dog riigere, hver af nogle hundrede Reen; Det viidne for denne \textit{Commission}\textit{Morten Nilsen}\textit{Fiin} har Sagt mig: hand har haft een Farbroder, paa Samme \textit{Liusendalsche} Svendske fielde, \textit{Jonas}, der har ejet een 3000 Reendyer, andere Siiger u-tallige, naar man nu vil voerdere et Reendyr over hovet for 3 rd{r} Saa var det een Skickckelig formue for een \textit{Fiin}; Samme \textit{Jonas Fiin}, Skal og have faaet et ære-Nafn af de Svenske over de andere \textit{Fiiner}, hvor af hand har giordt Sig Saa til, at naar hand i Sin Stadz er kommet need i bøjden, har hand haft Fløjels kiol, med \textit{Laass} foeret paa Sig, med een \textit{Sviite} af nogle tiennere. Der giives og Fattige \textit{Fiiner} Som tienner de riigere, og de ere de, Som gerne Strippe omckring i bøjdene og Tigge, og disse \textit{incommmoderer} bønderne ej alleene med deris tryglerie, men og naar de kand komme til, med at Rappe, dog kun det Som kan være Mad og tiene til føde, thi arbejde giør de iche, endskiøndt de ere friske og Sunde; det bønderne her have Sagt mig, Dersom \textit{Lap Finnerne} have meere end Reen, Saa bestaar dend Riigdom i heele penge eller Dalere, thi der af Skal de være Elskere, og goede Forvarere; og dem kan dem fortienne med at Sælge Dyer, Fiin-Mudter, eller løyede Reenskiin Kiorteler, Som ere lætte og varme. Støvler og handsker af Reenskiind, meget varme. pengene have de nu iche at udgiive, uden til vadmel at kiøbe, hvorj de gemeenlige gaae Klædde. Til Kongen Skatte de indtet ej heller giive noget til bonden til hvis gaarde Fieldene grændse Som jeg har hørt: men forbem{te}\textit{Fiinn Morten Nilsen} Sagde mig, at de \textit{Liusendalsche Lap- Fiinner} Som ere nu riigere, maa ærlig Skatte hver 2 daler til dend Svenske Foged, bonderne have her Sagt mig af deris forældre at have hørt, at \textit{Lap-Fiinnerne} tilforn iche har værit Saa langt her neer i Søer, men holdet Sig alt i Nord, og at i itzige bønders forældres tiid, ere de første \textit{Lap Fiinner} komne hiid Saa langt i Sønden paa de Tydalske fielde, Saa at det er tegn til at de maa formeere Sig. Det maa være ett gammelt folck, een afkom af det gamle Nordligste land, kaldet \textit{Biarmeland; Ole Mog(e)ns} en fordum Erke Bisp til \textit{Upsal} deeler \textit{Biarmeland} i det øvere og needre; i det øvere biarmeland (Som er de, Norden for field kølen) Skriiver hand, ere folck af een Særdeelis forunderlig anseende, hvor til Vejen er \textit{upassable}, og med u-overviindelige farer omgiiven og for Mennisker iche læt at komme til; thi den Største deel af \textit{passagen}, er med meget høy Snee Stetze belagt, hvilcken om nogen vil fare over, Spender hand field dyer (det er Reen) for Sleden, ved hvis u-trolig Snelhed hand kan køre over Snee Fonnene naar de ere Stiv-frosne; Saxo fortæller: at een meget riig Skoug Mand ved Nafn \textit{Memming}, Skal vel kanskee være: \textit{Henning}, thi dette Nafn er endnu brugelig blandt \textit{Finnerne}, der skal have haft Sit tilhold; Til denne da Kong \textit{Hother} af Sverrig med dette Slags Dyer er kommen kørendes, har hand faaed kostelig bytte og Gods, og bleven der over riig og een lycksalig herre. \textit{Olai Magni gothi Archi-Episcopi upsaliensis de gentibus Septentrionalibus libr:} I\textit{: Biarmiæ descriptio:}\par
\textit{Jn ulteriore vero Biarmia sunt qvidam monstrosæ novitatis populi, adqvos aditus invius et insuperabilibus periculis obfusus est, nec facile mortalibus patere potest; major siqvidem\hypertarget{Schn1_21695}{}Bilag C: Om Lap-Finnerne. itineris pars præaltis nivibus perenniter obsidetur, qvas si qvis superare velit, Cervis Jugalibus currum instruit. cujus incredibile Celeritate, eximio gelu rigentia Juga Transcendat. Meminit Saxo: Memmingum qvendam Sylvarum Satyrum insignibus divitiis præditum, illio mansionem tenuisse: ad qvem Hotherus Sveciæ Rex domitorum Cervorum cursu perveniens, maximorum spoliorum opumque adeptione ditatas et felix evasit.}\par
\centerline{U-forgribelig betænckning.}\par
Det er omtrent en Mandz Alder, Siiden at disse \textit{Lap-Finner} ere fra Nord til Tydals fieldene hidkomne de have og villet udbreede Sig længere i Søer til bræcke og \textit{Femunds} østere Fielde; men bønderne paa begge Siider have formeent dem det, deels for Dyer weiden, og Fiskeriets skyld Som de betage bønderne, deels og fordj \textit{Finnernis} dyer stundum kand komme bøndernis Enge-Sletter i fieldene for nær, hvorfore \textit{Finnerne} der fra maattet begiive Sig: Dog Sees Saa meget heraf, at bønderne i bøjdene ej have Saa ganske kundet forviise \textit{Finnerne} udaf de fielde Som de have i besiddelse, og er uden for bøndernis Græs- eller Fægang. Thi Siunis den Stræckning, \textit{Finnerne} nu indehave af landet til fields, det alt at være kongens Almiinding; hvor om de kongl: Norske betiendtere ej Saa meget har kundet beckymret Sig, deels fordj de ere andseede for øede og ufrugtbare, samt utilckomelige og langt afliggende, der Snart af det eene Snart af Det andet \textit{Fiinne partie} ere brugte, deels for uvisse, Saa lenge Som grændse-Skillet imellem Norge og Sverrig ej har værit fast Satt. Det er og noget fremmed og underligt: De \textit{Lap Finner} komme og Sætte Sig i fieldene, hvor de lyste, need uden forlov, de leeve og nære Sig af fieldene i hobe Tall, og betage de Norske bønder mangen god Lejlighed af Skytterie, Fiskerie, og EngeSlet, Som disse kunde nyde godt af, og til hielp i deris Skatters batalning, Det er der for min uforgriibelige Meening: naar fieldene imellem Riigerne eengang bliver deelte, at da\par
(1) Een ordentlig Marckegang af kongl: Committerede med hver bøjds betiendtere og kyndige bønder Saavelsom Skickckelig Lap Finner i disse Fielde anstilles, og enhver bøjds Fielde i visse Field-lejer legges: dog Saaleedis at dend Norske bonde ingen afgang eller \textit{præjudice} Skeer paa de herligheder af Dyer vejderie, fiskerie, Fægang, Enge-Slett eller andet han til Sin gaard fra \textit{arilds} tiid haver haft og have bør, dernest at eftersom det øfrige af fieldene findis frugtbar til af græsgang fiskevande og Skytterie om Sommeren, og af Maasse og Dyer vejde om viinteren, derefter kunde fieldene til een eller fleere \textit{Finne Familiers} fornødne underholdning udviises og fæstes eller bøxles, med J agtagende at dend udviisning Skeer heller Rum og viid til jtzige \textit{Finne Familiers} Rundelig ophold, end knap og trang, thi man vil forestille Sig, at disse Folck formeere Sig med tiiden, og naar alle fielde Saaleedis \textit{Successive} det eene efter det andet til andere visse \textit{Familiers} egen \textit{privative} brug og Nytte bleve overgiivne, Saa hafde de jtzige \textit{Finners} børn naar de bleve mange, ingen udveje til at føede Sig paa! og har ieg vel hørt af \textit{Finnerne}, at de foruden de bare fielde endogsaa i haardeste viinter tiid behøve Skoug at være i, for at have luun imod det haarde vejr at varme Sig veed, thi alle viinter Natte maa de være ude, og vogte deris Dyer imod Ulfven Som er deris tyngeste og besværligste bestilling i deris Stand. Naar nu fieldene Saaleedis i viise fieldlejer var inddeelte, var nest guds kundskab een \textit{Politic} og \textit{moralitet} der fornøden \hypertarget{Schn1_21785}{}Schnitlers Protokoller I. at indføre; Thi Som de nu leeve, ere de lige som uden for verden, følgelig uden for lovene, derfore ville een Skole Mæster hos dennem holdes, at oplære børnene troelig i Guds ord hvoraf naar de fienge dend Sande oplysning, Saa vilde ald den overtroe og gøglerie, Som de Mistænckes for, og i Saadanne afliggende Stæder kan iche andet, end være Skiult for øfrigheden, af Sig Selv forsviinde; Een af de ælste og vittigste \textit{Lap Finner} kunde og Som Kongl: lænsmand \textit{authoriseres}, i ett vist qvartier j fieldlejerne og over dem een \textit{Civiil} betiendt eller høvitz Mand settes, hvilcken med Skole Mæsteren for det første maatte have enslags Myndighed, til at kalde Fieldlaugerne Sammen paa visse beqvæmme tiider og Stæder, at underviise dend Meenighed i Guds ord, forckynde dem de Kongl: befalninger, have indseende med at de efterleeves, anmelde for bøjde-øfrighed, hvorunder de \textit{sorterede}, hvad betydeligt forefaldt af grove Misgierninger, af andre fremmede \textit{Finners} Jndtrængseler, af dødsfald og Arve-Skifter, hvor ved øfrigheden kunde komme til nogen erfaring om deris tilstand, Som de nu holde dulgt. eller andre mærckelige \textit{Fataliteter;} i det øfrige See derhen, at de børn, Som ved fieldlejen efter \textit{loven} bør blive vorder veed haandthævede, og de andere Som ej deraf kan fødes, Sættes hos bønder folck til arbejde og i tieniste, hvor ved den for \textit{Publico} besværlige om-Strippen og Tryglen kunde hemmes, hvor efter Saadanne unge folck til øede Stæder, hvortil lejlighed noch er i Norge, og Sæhrdeelis i FindMarcken, kunde anviises at rødde og bygge; (3) Det var da Naturligt, at Saadanne Lap Finner naar de bleve forsickrede og beskiermede ved deris Næring i kongens lande, de liigesom andere under Danere med tiiden maatte Skatte, af deris field lejer efter disse deris frugtbarhed og \textit{Finnernis} formue til Saadan een \textit{Civiil} betiendters beholdning; Saaleedis hører man de \textit{Liusendalske} field- \textit{Finner} og andere Som formaae noget, til dend Svenske \textit{Krone} aarlig hver 1 a 2 rdr. maa betale; og er i Sig Selv efter Folcke Rætten, at een undersaat, Som nyder en herres Jord og beskiermelse, maa Skatte efter taalelig lejlighed til Sin lands herre. Dog hvad udgift kund vel være billigere end naar disse Norske Lap \textit{Finner} aflagde noget taaleligt til deris geistlige og verdzlige betienteres underholdning og belønning; ‒\par
Dette er de geistlige og vertzlige øfrigheds Persohner og betiendteres tilsiun og nærværelse, der kan erfare og See deris vildfarelser, og Ryggesløsheder, om noget der af hos dem Skulle findes, Rætte, som geraader til at indføre een god Skick og orden næst gudsfrygten til de folckes eget beste; Ej at tale om andre goede \textit{Conseqvenser}, at under tilsyen af nærværende kongelige betiendtere kunde de bringes til een Slags Stadighed og Sædelighed, i den Stæd de nu som ingen Roe have, fra et til andet Stæd omstrippe, J Særdeelished i kriigs tiider kunde de være de beqvæmmiste kundskabs Mænd; viidere kunde Reen- og Elsdyer Skiind til ingen fremmed men til vore Egne i landet, efter Sit værdj, Selges og \textit{Fabriqveres:} da \textit{Finnen} nu drager dermed bort, man veed iche hvor; Fremdeelis dend heele fieldtract fra Nordenfieldske Stift begyndelse indtil Nord østligst i \textit{Finmarcken} Kunde med tiiden blive en ordentlig indrætted Sammenhengende \textit{Colonie} eller Folcke Slag imellem Nordske bøjdelauger og Sverriges\textit{Provintzier}, hvilcke dog her under Sit bøjdelaugs geistlig og vertzlig \textit{Juristiction} maatte blive \textit{sorterende}. Af eendel bønder i de Nordske bøjder har ieg forstaaed, at de heller Saae disse Lapfinner bort fra fieldene ganske \textit{elimineredt}, end Saa nær hos dem der \textit{logered}, thi Som troligt er, \textit{Finnerne} Som ingen gaardsbrug have at tage \hypertarget{Schn1_21874}{}Bilag C: Om Lap-finnerne. vahre paa, bortskyde de viilde Dyr og Fugle paa fieldene fra bønderne, de fiske i hvor liidet det end er, fra dem, og hvad fæebete eller Maasse paa fieldene og i fielddalene er eller blive kunde, det fortærer \textit{Finnerne} fra dennem: men naar man vilde forud Sætte (1) at kongen har een Alminding, adskildt fra bøjdenis Fælleds March, eller Saa kaldet bøjde alminding, Som hører bøjdelaugene til, Saa Staar det kongen fridt for at \textit{disponere} allernaadigst over Sin Alminding til hvem hand vil, (2) de Lap Finner Som et gammelt Norskt Folck og Christene Mennisker maa dog lades nogen Stedz at være til og de maa have ett Rum og Stæd at Sidde paa, man kan ej viise dem ud af landet, naar de ej have Syndet eller forbrudt noget, og bliver det en \textit{glorieux} Sag; om de med tiid og Stund \textit{redigeres} til en Slags gudelig Skikkelig og ordentlig leevemaade, Som ieg meener, nogenlunde er giørlig, for saavidt man kan bringe dem under vedkommende geistlig og Vertzlig øfrighedz Tilsiun. ‒ Efter at ieg havde hørt bondernis kære maal over Lap Finnerne, Nembl: at disse med deris dyr komme undertiiden bøndernis Enge-Slætt og deris afliggend Sæter- Marcker for nær; kom ieg til at Reise, Siiden bøjde vejene baade vare \textit{upassables} og for lange omckring at fare, fra østbye gaard i Tydalen Fieldleedz en 6 miile over til næste grændse bøjd \textit{Merrager} og det i Fiinnernis \textit{Kieriester} med Reens dyer for; Paa denne vej havde ieg nu Lap \textit{Finner} 3: Som skydzede mig frem; disse Nafnl: \textit{Nils Larsen} Anders og Hendrich Nilsen, brødere af Tydals fielde, og Særdelis dend første klagede der imod til mig over eendeel af de Norske bønder i Tydalen: at hvad Lap \textit{Finnen} tilforn af Field og Sæhrdeelis af Skouge nær ved Fieldene i det Tydalske haver paa visse tiider om Aaret besiddet og brugt det til deels, og i Sæhr Skougen faar \textit{Finnen} af nogle bønder ej lov til nu at bruge: da hand dog i haardeste viinters tiid ej kan undvære Skougen, for at have der luun og liise imod det haarde vejr paa fieldene, og varme for Sig, naar de maa være ude, og vogte deris Dyr Natt og Dag; Siigendes der hos; hand kan umuelig leeve paa bare haarde field, hvor ingen Skoug eller varme er at faae, om viinteren, men maa flødte did need i Dalen, enten til Tydalen i Norge, eller til \textit{Handøls} Skougene i \textit{Jemteland;} hand tilstoed og, at det undertiiden hænder Sig, at deris Dyr omckringvanchendes, kan komme bøndernis Marcke Sletter for nær, hvor fore og bønderne ere Vreede paa dem, men det Skeer kun Sielden, og det kun af nogle Enckelte Dyr; fortællendes derhos: naar Finnerne i Fieldalene nogle aar kan have haft Sit tilhold med deris Reendyer, og dermed frødig-giordt, eller bemøeget det Støkke græs-land nogen Stædz imellem fieldene, Saa derefter voxer godt Græs; Saa vil eendel bønder tilEigne Sig det Støcke græs-land til deris Egne Creaturer, og viiser \textit{Finnen} der fra; Fremdeelis at der har været fleere Lapfinner, Saasom førstbem{te}\textit{Nils Larsens} farbroder, der har villet need Sette Sig i Tydals fielde: men af eendeel bønder ej faaed lov der til, og er der fra bort viist. ‒\par
De nu værende 3 Tydalske \textit{Lap Finner} paa mit tilspørgende, Svarede: De gerne vilde til kongen Svare aarlig Skatt, Som deris Slegtninge paa de Svenske fielde giøre, Een Rdr. eller saa om Aaret; dersom de maatte nyde Roelig besiddelse af field lejene, hvor af de kunde leeve;\par
Hvilcket atter er een omstændighed, Som viiser, hvor tienlig een Marcke gang imel\hypertarget{Schn1_21966}{}Schnitlers Protokoller I. lem Norske bøjde Lauger og \textit{Finners} field lejer j fremtiiden, naar Skillet af grænzerne imellem begge Riigerne eengang er bleven \textit{adjousteret} og til Rigtighed bragt. ‒\par
Liigesom jeg \textit{avancerer} Nord efter, Skal ieg erckyndige mig om disse field lejers liggend og \textit{Lap Finnernes} tal til hvert bøjdelaug, forestillendis mig, jo længere Nord joe fleere der af vil findis; dend heele \textit{Tract} fra Søer at reigne til forbie \textit{Wardehus}, af Finnernis field lejer kunde uforgribeligen, deeles i viss qvartier, og over hvert et qvartier en \textit{Civiil} øfrigheds beckiendt med Nafn af høveds Mand, fogd eller befalnings Mand Settes, dermed de Folches \textit{politiske} Stand skulle have indseende, og efter beckommende \textit{Jnstrux} bringe dem med tiiden under en \textit{disciplin} og orden, der og i behøvende tilfælde kunde haandhæve og \textit{Maintineere} dem ved de Rættigheder, Som dennem af deris Kongl: May{t} Allernaadigst ville \textit{accorderes}, thi Som ieg har forstaaed af \textit{Finnernis} tale til mig, og vændtelig være kan, Saa er de folck øfrighed- følgelig og forsvar-løes; Og dertil kunde ret findes beqvæmme \textit{Subjecta}, der kunde Rende paa Skiie, og gerne for en Ringe belønning, uden besvær for deris Kongl: May{ts}\textit{Cassa} (at forstaae, om hver een \textit{Finne}-huusfader til de kongel: Fogder kom til at Svare noget liidet i aarlig Skadt, det de gerne vill giøre) og uden besvær med Skyts for bøjdene skulle ville paatage Slig en \textit{Civiil} betienning over \textit{Finnerne}, naar hand kun kunde erholde det allernaadigste løfte til viidere \textit{emploje} anden Stedz hen i fremtiiden; Og Saadanne \textit{Civiile} betiendtere kunde under hver \textit{districts} Ambtmand eller dend deris Kongel: May{tet} dertil allernaadigst ville beskikke, vorde \textit{Subordineredt}. ‒\par
\textit{Referenterne} heraf have været til deels Finnernes Skolemester, dels bønderne, deels Finnerne selv.\hspace{1em}\par
\centerline{Følger nu en \textit{Extract} af Grendse-\textit{Protocollen}, over Grendsens Gang fra \textit{Herjedalens} begyndelse i Søer indtil det Grendse Field \textit{Haalsio-Ruva} i Nord imod \textit{Jemteland}, saaviit som jeg med Vidners \textit{Examen} til 28 April: 1742 er \textit{avanceret}.}\hspace{1em}\par
\textit{Concept} af et ohngefærlig \textit{Plan} til Grendsens Gang imellem det Svenske Fogderie \textit{Herjedalen} og de i Vester dertil anstødende Norske Landskager, \textit{extrahered} af det Norske Grendse-Protocoll over de tagne Tings-Vidner, at reigne fra \textit{Slugu}-field, som giør Skielnet imellem den Svenske \textit{Eire}-bøyd (hvis Jndmaaling \textit{ex suspens} er udsat) og \textit{Herjedalen} i Søer, indtil \textit{Helags-Stødten} i Nord, som skiller \textit{Herjedalen} fra \textit{Jemteland}, indretted efter den \textit{Brømsebroiske} Fredz \textit{Tractat} af A{o} 1645. dens 25{de}\textit{Artic:} og det Kongl: danske \textit{Cessions} Brev til \textit{Sverrigs} Crone paa \textit{Herjedalen} af 8{de} Sept: bem{te} Aar.\par
Forud maa jeg sige, at \textit{Slugu}-field østen for \textit{Vonsiøgusten} giør Skielnet imellem \textit{Eire}- bøygd og \textit{Herjedalen} efter 5{te} Vidne i Guldalen Sp: 13. og 6. Vidne ibid. Sp: 4. derfra i Vester ligger nu det Field \textit{Vonsiø-Gusten}, som ligeledes efter bem{te} 6. Vidne ved Sp. 5. skiller \textit{Eire} fra \textit{Herjedalen:} Men som deris Skielle-merker efter samme Vidne gaae endnu længere i Vester til igiennem \textit{Grøtaadalen}, saa veed man ikke, om den \textit{allegverede raison} af Skielle-Merke kan fritage \textit{Vonsiø-gusten} fra, at blive \textit{terminus} a Qvo, hvorfra Maalingen \hypertarget{Schn1_22227}{}Ekstrakt av Grenseprotokollen. vil begyndes; Thi (1) har det Norske Vidne kun vidnet, eftersom han af de Svenske har hørt, (2) \textit{cesserer} her i denne \textit{Casu} bøyde-Merker; Uforgribeligen meenes, at af de H{rer}\textit{Jngenieurs} paa Stædet best kan tages i Øyesiun, og skiønnes paa: Om fra \textit{Vonsiø-Gusten}, eller næste Field i Vester \textit{Kraltvola}, \textit{Linien} i Nord efter Fieldenes Foed best og ligest kan drages, nemlig derfra til \textit{Rogensiø}, \textit{Bratriefieldz} og \textit{Raufields} Foed; Jmidlertid indstiller til hr. obriste \textit{Rømelings Ordination:} Om \textit{Vonsiøgusten}, eller \textit{Kraltvola} skal blive \textit{terminus a Qvo}, hvorfra Maalingen begyndes skal? Jndtil det her anføres \textit{Vonsiø-Gusten}, og fra Søer i Nord gaaes: ←⊕\hspace{1em}\label{Schn1_22288} \par 
\begin{longtable}{P{0.27056338028169014\textwidth}P{0.09577464788732394\textwidth}P{0.05267605633802817\textwidth}P{0.040704225352112676\textwidth}P{0.07422535211267604\textwidth}P{0.05028169014084507\textwidth}P{0.04788732394366197\textwidth}P{0.05985915492957747\textwidth}P{0.04788732394366197\textwidth}P{0.05028169014084507\textwidth}P{0.055070422535211265\textwidth}P{0.004788732394366197\textwidth}}
 \hline\endfoot\hline\endlastfoot \textit{Biscops- Aae}, see herom \textit{dubium} paa følgblad.\tabcellsep 11.\tabcellsep 10.\tabcellsep 9.\tabcellsep 8.\tabcellsep 7.\tabcellsep 6.\tabcellsep 5.\tabcellsep 4.\tabcellsep 3.\tabcellsep 2.\tabcellsep 1.\\
\textit{Liusenvola} øst for \textit{Haftor Stødten}\tabcellsep \textit{HydKrogens} Østre Ende en dal\tabcellsep \textit{Gleefield}.\tabcellsep \textit{Ruten}.\tabcellsep Field- \textit{Baalagens Siø}\tabcellsep \textit{Vaatafield}\tabcellsep \textit{Raufield}.\tabcellsep \textit{Bratrie-} field\tabcellsep \textit{Rogensiø}.\tabcellsep \textit{Kraltvola}.\tabcellsep \textit{Vonsiøgusta}.\end{longtable} \par
 \hspace{1em}\par
Følger nu af \textit{Protocollen extrahered} Vidnernes bekreftelse, at ovenstaaende ere de østligste Fielde, og betyder det første \textit{Numer}-Tal: \textit{Fieldet}, som oven benævnt er; det Andet \textit{Num:} Hvad det er for et Vidne? og af hvad Fogderie? det 3{die} Tal betyder hans Svar til Spørsmaalet, som \textit{alleqveres:}\hspace{1em}\label{Schn1_22404} \par 
\begin{longtable}{P{0.031057692307692307\textwidth}P{0.013076923076923078\textwidth}P{0.8058653846153846\textwidth}}
 \hline\endfoot\hline\endlastfoot I.\tabcellsep Efter\tabcellsep 5 Vid: i Guldal til Sp: 13. \textit{item} 6. Vid: til 4.\\
II.\tabcellsep »\tabcellsep 3 V: \textit{ibid.} til 13. S. \textit{item} 5 V: til 12. 13 S: \textit{item} 6. V: til 4. 5.\\
III.\tabcellsep »\tabcellsep 6. V: \textit{ibid.} til 4. dens vestre ende efter 6 V: til 6.\\
IV.\tabcellsep »\tabcellsep 1 V: \textit{ibid.} til 8. \textit{item} 6 V: til 6.\\
V.\tabcellsep »\tabcellsep 1 V: \textit{ibid.} til 13. \textit{item} 6. V: til 6.\\
VI.\tabcellsep »\tabcellsep 5 V: \textit{ibid.} til 13. S: \textit{it}: 6. V: til 6.\\
VII.\tabcellsep »\tabcellsep 6 V: \textit{ibid.} til 6. \textit{item} 9. V: 4.\\
VIII.\tabcellsep »\tabcellsep 9 V: \textit{ibid.} til 4. \textit{item} 10 V: 4.\\
IX.\tabcellsep »\tabcellsep 10 V: \textit{ibid.} til 5. \textit{item} 10 V: 4.\\
X.\tabcellsep »\tabcellsep 12 V: \textit{ibid.} til 4.\\
XI.\tabcellsep »\tabcellsep 12 V: \textit{ib}: til 4. \textit{item} 1. V: i \textit{Selboe} til 5 og 6.\\
XII.\tabcellsep »\tabcellsep 1 V: i \textit{Selboe} til 6. N: 1. og 2. V: \textit{ibid}: at ligge Sønden for \textit{Helags Stødten:} mens efter 4. V. i \textit{Selboe} til 6. N. 1. at ligge i Sydvest 1 Miil fra \textit{Helags}; Stødten; og herover opstaar et \textit{dubium}, som paa følgende blad videre udføres. Jmidlertid er dette \textit{Helags} Stødt Enden paa \textit{Herjedal} i Nord.\end{longtable} \par
 \hspace{1em}\par
Plan af Grendse-Gangen imellem den Svenske \textit{province}\textit{Jemteland} og Landskaberne i \textit{Trondhiems} Stift fra \textit{Tydalens Annex} i \textit{Selboe} Fogderie af, til forbi \textit{Merager} bøygden af \textit{Stiørdalens} Fogderie, saaviit man dermed er \textit{avanceret} til 28 April 1741 efter Vidners Udsagn, \textit{extraheret} af Grendse-\textit{Protocollen} derover holden:\par
Forudsettes: at efter 1. 2. og 4. Vidne i \textit{Selboe} Fogderie til Sp: 10. og efter 6. V. til \hypertarget{Schn1_22657}{}Schnitlers Protokoller I. Sp. 3. giør \textit{Helags} Stødten Skielnet imellem \textit{Herjedalen} og \textit{Jemteland} ved de Norske Grendser; Paa hvilken Kant der da vil gaaes fra de østlige fieldes foed i \textit{Herjedalen} op til den grendse-gang paa Fieldene, saaledes som Skielnet imellem \textit{Jemteland} og de \textit{Trondhiemske} Lande efter Vidners udsagn fra Arildz Tiid har været, dog maa forudberette, at et \textit{dubium} her strax møder om Grendse-gangens begyndelse: 4. Vidne i \textit{Selboe} til Sp: 6. N. 1. har udsagt, at BiscopsAaen ligger i Sydvest fra \textit{Helags} Stødten 1 Miil: derimod har 1. Vidne \textit{ibid.} til Sp: 6. N. 1. og 2 V: \textit{ibid.} forklaret, at for \textit{Helags} Stødten ligger i Sønden Biscops Aaen, fremdeles 5 Vidne i \textit{Selboe} ved Sp: 11. \textit{item} 4. V. \textit{ibid.} ved samme 11. Sp. \textit{in fin}: have hørt, at \textit{Helags Stødten} skal være et Skiellemerke: derimod de andre øfrige viide det ikke, \textit{au contraire} have det for mig benegtet; bem{te} 4. V., som efter \textit{confrontation} med 5. Vidne, derom blev tilspurdt, kunde vel ikke negte, at have hørt det: men sagde mig dog i særdeleshed, som den der var bekiendt paa Stædet, at siden Biscops Aaen ligger 1. Miil fra \textit{Helags} Stødten i Sydvest, og rinder der forbi saa langt i Vester, saa siuntes ham ikke, at være riimelig, at efter \textit{linien} fra Biscops-Aaen i Nord \textit{Helags} Stødten kunde blive et lande-merke; den Lap-Finn Morten Nielsen, 6{te} Vidne i \textit{Selboe}, en meget bekiendt Mand der, har og udeladt \textit{Helags Stødten} af sin \textit{Recite}, og \textit{particulierement} for mig sagt, at det ei kunde være Lande-merke, af den \textit{raison}, som 4. Vidne forklaret har; her spørges nu: Om Grendse-Maalerne fra \textit{Herjedalen} i Nord skal gaae lige til \textit{Helags Stødten}, hvorved formeentig Biscops Aaen paa Vores Side vil vorde indbegreben, og det paa \textit{fundament} af de 2{de} aflagde Vidners Eed, eller om man vil gaae fra \textit{Herjedalen} lige op til Biscops Aaen, som af alle \textit{u-disputerlig} tilstaaet Lande-Merke: dog kan imod denne sidste gang siges, at Biscops Aaen, som Sydvest fra \textit{Helags Stødten} liggendes, imod \textit{Herjedalen}, er paa Eet af de Fielde, ved hvis Foed kan gaaes i Øster; Uforgribelig siunes, at det meget vil ankomme paa \textit{Jngenieurenes} øyensiunlige befindende paa Stædet; Thi om de finde Biscops Aaen beent i Søer fra \textit{Helags Stødten}, har det liden Vanskelighed: men ligger Biscops Aaen 1. Miil i Vester fra \textit{Helags Stødten} kunde eftersees, hvorledes \textit{linien} Nordefter fra Biscops Aaen til næstanden Merke gaar, om \textit{Helags Stødten} meget af Veien ligger uden for \textit{Linien}, derefter at tage \textit{Messurene}. Over alt er Sagen i sig selv af liden eller ingen Betydenhed; thi den \textit{Distance} er øde og u-frugtbar, og indstilles til H{r} Obristens befindende, herj at føye sin \textit{ordination}. Følge nu de bevidnede Landemerker fra Søer i Nord, saaviit som jeg er kommet 28. Apr: 1742.\hspace{1em}\label{Schn1_22868} \par 
\begin{longtable}{P{0.09093023255813953\textwidth}P{0.11069767441860465\textwidth}P{0.0869767441860465\textwidth}P{0.0869767441860465\textwidth}P{0.08302325581395348\textwidth}P{0.09093023255813953\textwidth}P{0.08302325581395348\textwidth}P{0.11069767441860465\textwidth}P{0.09883720930232558\textwidth}P{0.007906976744186046\textwidth}}
 \hline\endfoot\hline\endlastfoot ←⊕\tabcellsep IX.\tabcellsep VIII.\tabcellsep VII.\tabcellsep VI.\tabcellsep V.\tabcellsep IV.\tabcellsep III.\tabcellsep II.\tabcellsep I.\\
\textit{Haalsiøruva}.\tabcellsep \textit{Kiørkgaalsfield}.\tabcellsep \textit{Storlie}.\tabcellsep \textit{Langvola}.\tabcellsep \textit{Einbogen}.\tabcellsep \textit{Blaa hammer}.\tabcellsep \textit{Storsola}.\tabcellsep \textit{Helags- Stødten}\tabcellsep \textit{Biscops Aaen}.\end{longtable} \par
 \hspace{1em}\par
Følger nu af Grendse-\textit{Protocollen} Vidnernes Forklaring over hvert Sted, og bekreftelse at de ere Lande-Merker til \textit{Jemteland:}\hypertarget{Schn1_22950}{}Retten sættes for Stjørdals Fogderi.\par
\centerline{Grendse-Steder:}\par
I. Beskrives af 1 Vid: i \textit{Selboe} til Sp: 6. og af 4. V. \textit{ibid. item} af 6 V. til Sp. 4. ‒ at være et Grendse-Merke af 1 V. \textit{ibid}. ved Sp. 11 af 2 V. til Sp. 11. af 4. \textbf{V.} til 6. og 11.\par
II. Beskrives af 1 Vid: i \textit{Selboe} til 6. N. 1. og 4 \textbf{V.}\textit{ibid} til 6 N. 1. ‒ et Grendse-Merke af 5. V. \textit{ibid.} til 11. og af 4. V. ved samme 11. Sp. \textit{in fine.}\par
III. beskrives af 5. V. i \textit{Selboe} til Sp. 5. ‒ et Grendse-Merke af 5. V. \textit{ibid.} til 11. og af 6. V. \textit{ib:} til 4.\par
IV. beskrives af 1 V. \textit{ibid.} til 6. N. 3. ‒ et Grendse-Merke af 5. V. \textit{ibid.} til 11. og af 4 V. \textit{ibid.} til samme Sp. \textit{in fine.}\par
V. beskrives af 1. V. i \textit{Selboe} til 6. N. 3. ‒ et Grendse-Merke af den samme \textit{ibid.} og til 11. \textit{item} af 2 V. \textit{ibid.} til 11. og af 4 V. til 11. saavel som af 6. V. t{i}l 4.\par
VI. beskrives af 1 V. i \textit{Selboe} til 6. N. 4. ‒ et Grendse-Merke af samme \textit{ibid.} og til 11. \textit{item} af 4. V. til 11.\par
VII. beskrives af 4. V. i \textit{Selboe} til 6. ‒ et Grendse-Merke \textit{ibid.} og til 11.\label{Schn1_23059} \par 
\begin{longtable}{P{0.4421965317919075\textwidth}P{0.40780346820809243\textwidth}}
 \hline\endfoot\hline\endlastfoot VIII. forklares af 4. V. \textit{ibid} til 6. og 11.\tabcellsep Grendse-Merker at være; hvorom meere Stadfæstelse kan ventes fra Merager og videre.\\
IX. ‒ ‒ ‒ af \textit{dito} sammesteds.\end{longtable} \par
 \hspace{1em}
\DivI[I Stjørdal fogderi: 3 vidner.]{I Stjørdal fogderi: 3 vidner.}\label{Schn1_23081}
\DivII[April 30. Fra Tydal til Meråker]{April 30. Fra Tydal til Meråker}\label{Schn1_23083}\par
Om \textbf{Mandagen} nest efter dend \textbf{30 Apriil 1742{ve}} Da \textbf{Major Snitler} Skulde fortSætte Reisen til næste grændsebøjd \textit{Merrager}, forestillede bønderne i \textit{Tydalen}, at landevejene igiennem bøjdene af \textit{Selboe} og \textit{Støerdalens} gield op til Mærrager var næsten iche farendes, formedelst at Elvene vare opgangne, og een ond vej af en 17{ten} Miile, og foresloeg da, at tage vejen over Fieldene lige til Mærrager i \textit{Kierester} med Reendyer for, Som er \textit{Lap Fiinnernes voitur} om viinteren, og da var denne vej fra \textit{Tydalen} til Mærrager Field-Leedz kun en 6 miile; \textit{majoren Resolverede} da til denne Reise, og toeg \textit{Kierestene} med Reen-forspend 7 i Tallet for Sig og følge Sambt \textit{Bagage}, hvormed \textit{Fiinnerne} fuldte paa Skiie; det gick paa denne Aarsens tiid kun langsom, thi Sneen var blød, Saa at Dyrene slog derigiennem, og Dyrene ere paa denne tiid Som magerst; Dog komme vel til \textit{Merrager}, dend Dag, farendes væsten forbie \textit{øye}field, \textit{Sylen}, og \textit{Stoer Sola}, hvilcke man paa vejen havde paa høiere haand og kunde See, kommendes Tædt forbje biørnEggen, en Tang af \textit{Remmen}; Saasom Viidner vare borte i Skougen, Som best om grændserne viste beskeeden, og man giorde bud efter \textit{merragers Lap Fiinner} en 3 Miil fra bøjden, for at tale med dem liigeleedis om grændserne og viidere Skytz, blev man opholdet derfor til Løverdagen dend 5{te}\textit{may} næst efter.\hspace{1em}
\DivII[Mai 5.-6. Rettsmøte på Tømmerås i Meråker.]{Mai 5.-6. Rettsmøte på Tømmerås i Meråker.}\label{Schn1_23191}\par
\textbf{A{o} 1742} Dend \textbf{5{te} May} blev paa Gaarden \textbf{Tømmeraas} i \textbf{Mærrager} bøjd, Som nærmest liggende til \textit{Jemteland} paa denne Kandt, Rætten Satt, j overværelse af de 2{de} laugRættes \hypertarget{Schn1_23223}{}Schnitlers Protokoller I. Mænd, \textit{Ole Larsen Nøstad} og \textit{Bæss Jonsen Mærrager;} tilstæde værende \textit{Jon Jversen Mærragernes}, Som forrettede nu dend fraværendes \textit{vice}-Lænsmands \textit{Jver Halvorsen Mærragers} tieneste her i bøjden; af Kongl: \textit{Civiil} betiendtere mødte ingen, formoedentligen for dette uføere skyld paa denne aarsens tiid. ‒ Viidner Som Skulle være kyndigste om grændse gangen i denne Ejgn, vare nu fremckomne og for Rætten fremstilledes \textit{Erich Olsen Riisvold} og \textit{Peter Olsen Nøstad}.\par
J Sambtlige deris paahør blef dend Kongl: \textit{order} til \textit{major Snitler} af 16 martj Sistleeden forckyndt, der nest for viidnerne Eedens forcklaring af Lov bogen oplæst, og de derpaa tagen i \textit{Eed}; at Siige deris Sandhed, om hvis dennem beckiendt er angaaendes grændsens gang. ‒\hspace{1em}\par
1{te} Viidne i \textit{Størdalens} Fogderie\textit{Mærrager Annex} ‒\par
Heeder \textit{Erich Olsen Riisvold}, er fød paa dend gaard \textit{Brænde} i Mærrager-bøjd af bønder-folck Sammestædz, 60 aar gammel, gift, har 7 børn, er boende paa dend gaard Riisvold.\par
1: Spørsmaal ‒\par
hvor langt ligger dend østligste gaard i denne Mærrager-bøjd fra de nærmiste grændser? i hvad Sogn og Fogderie?\par
\textit{Resp:} Fra den gaard \textit{Dalenæs} til det nærmiste grændse-mærcke i øster Nembl: Det field \textit{Langvola} er een 3 gamble field Miile, Som man nu kan Reigne for 2 Stoere Maalte, gaarden ligger i Mærrager \textit{Annex}, \textit{Størdalens} Fogderie og Præstegield. ‒\par
2: hvad er Landets beskaffenhed imellem bem{te} grændse-mærcke og denne Mærrager bøjd, at forstaa: om der er Skoug, vande, Elfv, field, Myhr, dyrcket og bebygget, eller øede ufrugtbar Land?\par
\textit{Resp:} her fra bøjden er nærmist een Gran-Skoug 1: miil lang, Som er eendeel af den Saa kaldede Stoer-\textit{Tæveldal} Skougen; Der paa i øster følger bare Field \textit{vohler}, Som ere øde, u-frugtbare, optil \textit{Langvola}, og boer ingen Folck her imellem, Saa det Land er u-dyrket og u-bebygget. ‒\par
3: hvilcke ere de nærmiste gaarder paa denne Norske Siide nærmist ved grændse Fieldene, fra Søer At reigne? Af hvad beskaffenhed er Landet? og hvad Næring bønderne bruge? ‒\par
\textit{Resp:} De nærmiste gaarder til denne bøjd, nær ved Grændserne paa dend Syndere Siide, ere de 2{de}Stuedahls gaarder i Tydalen, 6 maalte miile herfra liggende; Paa dend Nordre Siide ere 4{re} Smaa \textit{Suul}-gaarder i \textit{Werdals} gield og Fogderie, En 5 gamle miile herfra, Som hand meener at blive en 4 Nye maalte Miile. Landskabet imellem Stuedalen og dette Mærrager er bare øede viilde fielde, hvor ingen boer eller kand boe; dog kand der være noget Myhr Lend og nogle faa Seterboeliger for Fæed væst for Fieldene, hvor de need dale; Landskabet imellem \textit{Suul} og Mærrager bestaar paa Samme Maade af Fielde og nogle Smaa Sætervolde. Søer imellem Stuedahlen og Mærrager er \textit{Eesand} Siø, hvor af dend Elf \textit{Esna} Rinder i \textit{Nea} Elfv (hvilcke før ere beskrevne.) Søer imellem Mærrager og \textit{Suul} ere\par
(1) \textit{Fund} Søen, liggendes her fra Mærrager i Nord een 1/2 Miil, denne \textit{Fund} Søe\hypertarget{Schn1_23453}{}1 Vidne i Stjørdals Fogderi. Stræcker Sig fra øster i væster een 1/2 Miil lang, og er 1/3 miil breed; Af denne Søe udriinder den liiden Elfv \textit{Funna} Elv een 1/2 Miil i Søer ind i \textit{Skiørrens} Elv. Norden for denne liiden \textit{Fund} Søe en l/2 miil, er dend Søe\par
(2) \textit{Færen}, Stræckende Sig fra øst i væster 2 Miile lang og 1 miil i breeden; af denne \textit{Færen} Søe kommer dend Elv \textit{Faarra}, Som flyder først i Søer, Siiden i væster igiennem \textit{Faar}-bøjden af \textit{Hægre Annex} omtrendt 3 miile, ind i Skiøra-Elfv i \textit{Hægre annex}. Een liiden Miil i Nord-ost fra Mærrager bøjd, ligger dend 3{die} Søe \textit{Fierjen}, Norden for \textit{Stoer Tæveldals} Skougen, Stræckende Sig fra øster i væster, 1 Miil lang, og 1/8 Miil og mindere breed; Fra denne \textit{Fiergen}-Søe udckommer een bæck, kaldet kaabber-\textit{aae}, løbendis først i Søer 1 Miil, Siiden i væster, dog kort tilforn tager dend til Sig \textit{Skiørra}-Elfv, Som kommer udaf \textit{Schaarra} Søe, 2 Miile der fra i øster; Derfra nu, at denne kaabber\textit{aae} foreenes med \textit{Skiørra} Elv, taber dend Sit Nafn og heeder Siiden \textit{Skiørra}-Elv, Rindendes fort i væster igiennem Mærrager-bøjden og \textit{Hegre Annex} 5 gamle, eller formeenende 4 Nye Miile, ud i \textit{Striind}- eller \textit{Trondhiems} fiorden; naar dendne Skiørra-Elfv fremløeber lidt i væster i \textit{Hegre Annex}, kaldes dend, med forandred bogstav \textit{Stiørra}-Elv, og giiver Præstegieldet Samt Fogderiet det Nafn \textit{Stiørdalen}. I oven beskrevne 3 Søer fiskes indtet andet end øret. Een 1/2 miil østen for \textit{Fiergen} Søe ligger dend Søe \textit{Hol} Søe mest rund 1/4 miil over; ud af denne \textit{Hol} Søe Rinder dend bæck, \textit{HolSøeaaen} 1/4 miil vejs i væster ind i \textit{Fierjens} Søe; disse for beskrevne 4 Søer Siiger hand at ligge til Mærrags bøjden, dogsaa at Merragerne med Werdalingerne har \textit{Færen} Søe til fælles, at disse Fiske paa dend Nordre Siide, og hiine paa dend Søndere Siide i bem{te}\textit{Færen}. Skouge er der noget liidet af Gran paa dend Syndere og Nordre Siide til denne bøjd i de Dahle væsten for Fieldene. ‒ Bønderne her i Mærrager nære Sig af deris gaarder, Som dog ere frostnæmte, og af Tømmer hugster til Saugbrug ‒\par
4: hvilcke ere de nærmiste gaarder paa dend østlige Svenske Siide nærmist til grændse fieldene, fra Sønden at reigne? og af hvad beskaffenhed er landet? og hvad Næring bønderne der bruge?\par
\textit{Resp:} De nærmeste gaarder paa dend østlige Svenske Siide, fra Søer at reigne ere de 2{de}\textit{Handøls} gaarder, Som ligge nærmist østen for det field \textit{Remmen}. 1 Miil Norden for disse \textit{Handøls} gaarder ligge de 2{de}\textit{Waallan} gaarder, nærmest østen for \textit{Langvola} field 3 miile fra Samme field efter gammel reigning. Norden for disse \textit{Wollan} gaarder 3 miile ligger dend gaard \textit{Schals Stuen}, omtrendt Som hand meener 1 miil østen for det Lande Mærche \textit{Stor} Søe Sundet. Angaaendes landskabet af di ovenskreevne Gaarder, Saa er det fieldet, dog have de alle meget god Eng-land, \textit{Handøl} i Sæhr har god Skoug i \textit{Liill Tæveldalen}, liigesaa har \textit{Wollan} gaarder god Skoug af Gran, Som de og have goed Fiskerie og noget korn land, Som dog er Frostnæmt: men \textit{Schalstuen} har kun liiden Skoug og liidet til fiskerie, og Saaes og her aldrig korn, Saa de Folck leeve mest af deris qvæg og Fiskerie Samt liidet Skytterie. ‒\par
5: hvilcket er det første field eller Stæd, Som holdes for at være ett Lande mærcke imellem Norge og Sverig paa denne Kandt; det Viidnet har hørt eller kiender Sønden i fra at Reigne?\hypertarget{Schn1_23705}{}Schnitlers Protokoller I.\par
\textit{Resp:} Hand har hørt at blaa hammer klimper, Som ligger nest ved \textit{Eina} Elfv Paa dens øestere Siide og ligger beent i Søer fra \textit{Einbogen} omtrendt 1/4 miil vejs, er ett Lande mærcke; hand har veel Seed dend Nordre Siide af dette blaa hammer, at værn bradt og Steilt: men de andere Siider kiender hand iche, Siiden hand iche har værit derpaa: dog har han kundet Seed, at det Stræcker Sig i øster ad det Stoere field \textit{Snasahougan} og er der med fieldfast.\par
Det andet Lande Mærcke, viidnet kiender er \textit{Einbogen} 1/4 vejs, Som meldt, fra \textit{Blaahammern}; Landskabet af dette \textit{Einbogen} paa dend østere Siide er nogen gran Skoug, Som kommer Sammen i Nord og øster med \textit{Lill Tæveldal} Skougen.\par
Paa dend væstere Siide er een trang Skougdal med grantræer, hvilcken Skoug dahl Stræcker Sig i væster en 3 miile vejs need ad Merrager bøjden; Fremdelis er væsten for dette \textit{Einbogen} og Synden for dend bem{te} Trange Skougdal een fieldklimp, kaldes \textit{Klutkiæn}, Som er een klimp paa den Nordre Foed af \textit{Remmens} field.\par
Paa dend Syndere Siide af \textit{Einbogen} er dend østere Ende af \textit{Rem}fieldet; Paa dend Nordere Siide ligger det field \textit{Langvola};\par
Hvilcket \textit{Langvola} er det 3{die} Lande Mærcke, hand kiender: Det Stræcker Sig fra øster i væster, dog liidet fra Sydost i Nord væst, vel en 1/2 miil lang, er ellers et Slet Flatt Field, breed kan det være over omtrendt 1/4 miil vejs, Paa dette \textit{Langvolas} østere og Nordere Siide liigger \textit{Lill Tæveldals} gran Skoug Som gaar langs efter \textit{Eina}-Elven hend til \textit{Handøl} gaardene i Sydost, Paa dends Syndere Siide er den trange Skoug Dahl, Som før om talt, gaar need ad Merrags bøjden. ‒\par
Dette \textit{Langvola} ligger Tædt ved \textit{Einbogen} i Nord, Saa at \textit{Einbogen} løeber der under Rætt midt paa dend Syndere Siides foed, og der af Siiger hand er dend høyeste Midtte af \textit{Langvola} oven paa er Lande Mærcke, Saa at dend østere halfve deel tilhører Sverriig og dend væstere halfve deel Norge; ellers findes derpaa ingen Vahre, eller Mærcketegn opsatt. Det 4{de} Lande Mærcke i Nord fra \textit{Langvola} er \textit{Storljen}, Een houg, Som hænger Sammen i Nord med \textit{Kiørkgaals} fieldet: Dog som dette Stoerlje er mindere end Kiørckgaals fieldet, og ligger meere væstlig ad Norge, Saa er dette Stoerljes østere foed grændse-Mærcket imellem Sverrig og Norrig, og kand herfra til næst forrige lande mærcke \textit{Langvola} være Som een 1/2 Miill vejs, og paa dend Maade kommer heele Stoerljehougen at høre Norge til; ‒\par
Om dette \textit{Storljens} landskab paa Siidene Siiger hand det Samme ud, Som 4{de} Viidne i Fogderie \textit{Sælboe}, dette bare tilleggendes, at paa \textit{Storljens} Syndere Siide er der vel nogen Smaa, birke Riis, men og Saa gran Skoug.\par
Om dette \textit{Storljens} Størrelse og Stræckning giiver hand denne forcklaring, at det ligger fra øster i væster l/4 miil vejs lang, og kand det være 1/8 mil over fra Søer i Nord. paa dette \textit{Stoerljen} voxer græs, Som fra \textit{arilds} tiid er Slagit og høsted af Mærrager Mænd, og har viidnet Selv værit der; ellers findes paa dette Stoerljen ingen Vahrer eller andet Mærcketegn opsatt. ‒\par
Det 5{te} Lande-Mærcke bliver \textit{Kiørkgaals} fieldet Som er ett Slett Flatt field oven paa, og Stræcker Sig fra øster i væster 1: miil, og er fra Søer i Nord 1 1/2 miil over; i Søer \hypertarget{Schn1_23938}{}1 Vidne i Stjørdals Fogderi. hænger det Sammen med nest forrige beskrevne \textit{Storlje} ved een field Sløgd, 1/8 miil bred, hvor i ingen Skoug eller Græs er.\par
Jngen Vare findes paa dette field, men hand Siiger, at dend Største deel af \textit{Kiørkgaals} fieldet tilckommer Sverrig, og den minste deel deraf Norge, og det af dend \textit{Raisong}, at kiøn-vandene fra dend Største deel af fieldet løebe need i øster ad Sverrig, og Kiønvandene fra dend minste deel af fieldet falde need i Væster til Norge; Det mellemRum nu, hvor fra vandene falde baade i øster og i væster er det høyeste af \textit{Kiørkgaals} fieldet, og findes paa dend væstere Deel af Fieldet; hvor af det da er, at dend miindere væstere Deel der af tilckommer Norge, og dend Større østere Deel er Sverrigs. Om dette \textit{Kiørkgaals} landskab paa den østere Siide kommer hand over ett med 4 viidne j Sælboe, undtagen derj, at de \textit{Hollandsche} Falcke fængere Pleje at holde til paa dend østere Svenske deel af Fieldet ‒\par
Landskabet paa dends Væstere Siide, beskriiver hand, at være Steenet og Stygt een 1/2 Miil lang: Derpaa følger Myhr og Gran skoug, Som gaar til \textit{Stor Tæveldals} Skougen, Desuden ligger 1/8 miil væsten for \textit{Kiørkgaals} fieldet dend Søe \textit{Skiorra}-Søe, Som Stræcker Sig fra Søer i Nord en 1/2 Miil lang og er ett parr Bøsse Skud breed; Af denne Søe gaar i Væster \textit{Skiorra} Elv 2 Miile til Mærrager bøjd, og Strax derved løeber kaabber-\textit{aaen} derj, Som før er beskreven;\par
Paa dend Nordre Siide er dette \textit{Kiørkgaals} field Sammenhængendes med \textit{HolsiøRuva} ved et liidet field Sløgd der imellem.\par
Og dette \textit{Holsiøruva} er det\par
6{te} Lande Mærcke, Det er ett Slet flat fjeld oven paa; det Stræcker Sig fra Søer i Nord 1/2 miil lang, og er 1/8 deel miil breed over fra øster i væster, Vahre findes der iche: mens midt der oven paa er det høyest, Saa vandene fra dend halve Siide falde i øster til Sverrig, og vandene af dend anden væstere Siide falde til Norrig, hvorfore denne middelste høyde af Fieldet er lande-mærcket jmellem Riigerne;\par
Landskabet paa dend østere Siide bestaar af Smaa field Ruver og der paa i øster følger een Søe, kaldes \textit{Reen}-Søen, og ligger fra \textit{Holsiøruven} 1/4 miil vejs i øster; Af denne Reen Søe udkommer \textit{Reen-aaen} og gaar i Søer omtrendt 2 miile ind i \textit{Eina}-Elfven, førend denne \textit{Eina} falder i \textit{Handøl} Elfven; Østen for denne \textit{Reen} Siøe er udyrckelig Myhr-land med nogen bierche-Skoug paa. ‒\par
Paa dend væstere Siide af dette \textit{HolsiøRuva} ligger 1/8 miil der fra, \textit{Hold} Søe, og der fra 1/2 miil i væster \textit{Fiærjen} Søe, med kaabber-aaen, Som alle før er beskreeven; imellem \textit{Holdsiø} og \textit{Fiærjen} er der granskoug og Sæter-boliger for Mærrags Mændene; ved \textit{Fund} Søen findes og nogen Gran Skoug med Sæter boeliger for Samme bøjds bønder. Paa \textit{HoldsiøRuva} i Nord følger een liiden field dahl, Som hand meener er et par bøsse-Skud breed, uden Skoug og Græs; derhos ligger i Nord det Field \textit{Kiølhougene} ett bart Slet field, hvor vel nogen maase men iche Skoug eller Græs er: og dette er det\par
7{de} Lande Nærcke imellem Riigerne Nembl: Midt oven paa, hvor det er høyest;\par
Dette \textit{Kiølhougen} Stræcker sig fra Søer i Nord 1: miil lang, og er fra øster i væster 1/4 Miil breed over ‒\hypertarget{Schn1_24147}{}Schnitlers Protokoller I.\par
Det østere Landskab af dette \textit{Kiølhougen} er Slet Myrland og nogen biercke Skoug ved, u-beboet og udyrcket. 1/2 miil fra \textit{Kiølhougen} ligger i øster een Søe kaldes \textit{Schalsvatnet}, Som Stræcker Sig fra øster i væster 1 Miil lang og er een 1/4 miil breed, hvor i fanges øret; længer i øster er hand iche beckiendt.\par
Dend væstere Siide af \textit{Kiølhougen} bestaar af en field Dahl med liidet græs i; væsten der for ligger en Skoug-Dahl med gran i, kaldes \textit{Kulkiendalen}, hørendes til Mærrager bøjden, hvis bue- og Sæter-havn det er; længer j Nord var viidnet iche beckiendt ‒\par
6: Til hvis gaarder hører disse beskrevne fielde, eller under hviis gaards grund ligge de? eller om de ere Kongens Alminding?\par
\textit{Resp:} Hvad Sig angaar \textit{BlaahammerKlimpen} og \textit{Einbogen}, Som ligger mod Tydals bøjden, der om veed hand ingen viss beskeeden at give. De andre Fielde fra \textit{Langvola} til \textit{Kiølhougene} i Nord \textit{inclusive}, Som ligge ad Merragers bøjden, veed hand iche at de ligge under nogen viss gaard eller grund, men hvad Eng land i dalene findes, og Fiskeriet, det haver Mærrager bønder fra \textit{arilds} tiid bruget; Skougene har hidindtil været u-deelt og bruges af Mærragers bøjde Laug, Som den de mest maa leeve af ‒\par
7: Om der har værit nogen tvistighed imellem de Norske og Svenske undersaatter om disse Fielde?\par
\textit{Resp:} Nei, icke det han hørt haver ‒\par
8: hvad Nytte, godhed, og herlighed, er der ved disse grændse Fielde?\par
\textit{Resp:} Der er ingen herlighed i fieldene for bonden, førend de komme i Skougdahlen, at de kand benytte Sig af Skougen: men Lap Finnene have hidindtil brugt fieldene, og de nærmest underliggende Skoug Dahle ‒\par
9: hvor langt ligge disse grændse-fielde og Stæder fra bøjdene og lande-vejene?\par
\textit{Re[s]p: Langvola} med de andere fielde til \textit{Kiølhougen inclusive}, kan ligge her fra Mærrager bøjden og lande vejen 3 Miile vejs, meer og mindre,\par
10: hvor underholdning for Folck, og beete for hæstene, paa eller ved disse grænseFielde ere at faae?\par
\textit{Resp:} her i bøjden bliver liidet til underholdning for dem at faae, og ved hand ingen nærmere Raad for dennem at faae, end i \textit{Trondhiem}, Som ligger fra de op-Reignede Fielde 11: Miil ‒\par
For hæstene bliver ingen Mangel paa beete, Som findes noget hvert Stæd i field- og Skoug-Dahlene. ‒\par
11{te} hvad Mænd hand kiender at være beqvæmmest og kyndigst, til at vei-viise grænse-maalerne dend rætte grændsens gang paa denne kandt? ‒\par
\textit{Resp:} han meener de beqvæmmeste at være \textit{Ole Andersen Brænde}, og \textit{Hemming Jngvalssen Efien}, Saa og dend \textit{Lap Fin, Ole Nilsen}, tilholdende i de \textit{Mærragersche} Fielde ‒\par
12: J anleedning af det 8{de} Spørsmaal, tilSpørges Viidnet hvor mange \textit{Finne-Familier} Sig i \textit{Mærragers} Fielde opholde? og af hvad tilstand de ere?\par
\textit{Resp:} hand meente: der var 6: \textit{Familier}; de fleste have kun liidet, og gaa mest paa bøjden med Tryglerje, ‒\par
Hvor paa hand blev \textit{dimitteret} ‒\hypertarget{Schn1_24324}{}2 og 3 Vidne i Stjørdals Fogderi.\par
det 2{de} Viidne i Mærrager ‒\par
heeder \textbf{Peter Olsen Nøstad} ‒ Er føed paa \textit{Nøstdad} i Mærrager, af bønder Folck paa Samme gaard i \textit{Mærrager Annex}\textit{Stiørdals} Fogderie, er 60 aar gammel, gift, uden børn, boer og er bonde paa bem{te} gaard, Nærer Sig af gaarden og Tiimmerhugst til Saugbrug ‒\par
Spørsmaalene fra 1{te} til 12{te}\textit{inclusive}, hvilcke nest forrige l{te} viidne af Mærrager haft haver, blev for hannem hvær i Særdeelished efter hinanden oplæst, og hand Svarede der til til det Samme, Som næst forrige viidne uden Ringeste Forandring; hvorpaa hand blev \textit{dimiteret}.\hspace{1em}\par
Som det 3{die} Viidne der var kyndig paa grændserne Nembl: \textit{Lap Finnen Ole Nilsen}, ej var fremckommet endnu, blef Rætten opsatt til viidere. ‒\hspace{1em}\par
Dend 6{te} May om aftenen indfandt Sig Samme Lap Finn, og der paa blef Rætten Satt neste morgen derefter dend \textbf{7}{de}\textbf{May 1742}{ve} overværendes forbenæfnte Mand paa Lænsandens Vejgne og Samme Laug Rættes Mænd; J deris Paahør blef Viidnet Lap \textit{Fiinnen} deris Kongl: May{ts} allernaadigste befalning om denne Rættes holdelse beckiendt giordt og efter Som man hørte, at hand kunde giøre goed Reede baade for Sin \textit{Christendom} og for andet, mand Spurte ham om og det paa godt Norsk, Saa blev Eedens forcklaring af lovbogen hannem forelæst, og hand tagen i \textit{Corporlig Eed:} at Siige Sandhed, hvad ham om grændsens gang imellem Norge og Sverrig var beckiendt og der paa tilSpurt ‒\par
1 Spørsmaal til 3{die}\textit{Vidne i Merager}:\par
hvad hand heeder? hvor hand er føed? af hvad forældre? hvor gammel? om gift? Og hvor hand Nu tilholder?\par
\textit{Resp:} Heeder \textit{Ole Nilsen}, er føed j Tydals fieldene, Som er Norges, af \textit{Finne} forældre, er 50 aar gammel, gift, og har ett barn; har Sin meste tiid holdt til i Tydals fieldene, og nu tilholder i de Norske Mærragers fielde ‒\par
2: om hand veed, om hand er døbt? og hvor? om hand har gaaet til guds bord? naar? og hvor Sist? om hand veed, hvad een Eed betyr? \textit{Resp:} hand er døbt i Tydals kiercke j Selboe Præste gield, han har gaaed mange gange til gudsbord og Siste gang i høst nest afvigt i Mærragers kierckeStørdalens Præstegield, og Siiger at viide, hvad Een Eed betyder.\par
3: Om Viidnet veed, hvad Skilnet giør imellem \textit{Herjedalen} og \textit{Jemteland} ved de Norske grendser i væster? \textit{Resp}: \textit{Helags} Stødten og de derfra i øster udgaaende flere fielde. ‒\par
4: hvilcke ere de rætte grændse Mærcker, Som viidnet kiender, eller fra gammel tiid har hørt at være imellem Norge og Sverrig, og om der findes nogen Vahre eller kiende tegn nogen Stæds opsatt?\par
\textit{Resp:} hand forcklarer først de østligste Fielde nærmest \textit{Herdalen} at ligge Fra \textit{Raug} fieldet af, hvor hand først er beckiendt Nembl: (1) \textit{Raug} field, (2) \textit{Waattaa} field, (3) \textit{Field Baalagen} Søe, (4) \textit{Rutten}, (5) \textit{Glee-Fieldet}, (6) Dend Dahl \textit{Hyd Krocken}, (7) \textit{Liusenvola}, indtil dend Bæck, (8) \textit{Bischopsaaen;} ‒\par
Hernest beviidner hand følgende grændse Mærcker at være imellem Norge paa denne \hypertarget{Schn1_24575}{}Schnitlers Protokoller I. kandt og \textit{Jemteland;} og begynder (1) fra Biskops\textit{aaen}, hvilcken hand Siiger, at begyndes i Sydvæst fra \textit{Helags} Stødten, og Rinder 1 miil væsten for \textit{Helags} Stødten ind i \textit{Nea} Elfv, og har Samme \textit{aae} 1 miil vej, Som dend løeber, fra dens udSprang indtil dend falder i \textit{Nea} Elfv; fremdeelis forcklarer hand, liige som det 6{te} Viidne i Sælboe fogderie at Synden for denne Biskops \textit{aae}, omtrendt eett bøsse Skud, Staar een gammel Steen vahre Som nu meget er forfalden, dog der af Staar igien ett Stycke omtrendt 1 1/2 allen høj; men hvem denne Steen-vahre har opsatt, veed hand iche; det (2) lande-mærcke Siiger hand at være det høyeste af \textit{Stoer Sola} Som er Eett Field med \textit{Sylen}.\par
Rætten tilSpurtdte viidnet: om hand har hørdt, \textit{Helags} Stødten at være ett grændse mærcke? hvor til hand Svarede: Nej det har hand aldriig hørt. ‒\par
for (3) \textit{Blaahammer Klimpen}, nærmest østen for \textit{Eina} elfv liggendes ‒\par
for det (4) \textit{Einbogen} hvilcken \textit{Einbogen} hand Siiger, at ligge beent i Nord for \textit{Blaahammer} Liige Som dette \textit{Blaahammer} ligger liige i Nord for \textit{StorSola}, og \textit{StorSola} liigeleedes beent i Nord for Biskops\textit{aaen}. ‒\par
For det (5) Dett Midtelste og høyeste af \textit{Langvola} Som er liige i Nord fra \textit{Einbogen} beliggende. ‒\par
(6) Det østerste eller dend østere Foed af \textit{Storlje}, Som ligger liige j Nord fra \textit{Langvola}.\par
(7) \textit{Kiørkgaals} fieldet, der hvor det er høyest, at vandene falde der fra til begge Siider i øster og Væster.\par
(8) Dend høyeste Midte af \textit{HolsiøRuva}, som ligger i Nord fra \textit{Kiørkgaals} Fieldet ‒\par
For det (9{de}) \textit{Kiøllehougen}, der fra liige i Nord liggende ‒ viidere i Nord var viidnet icke beckiendt. ‒\par
Paa Rættens til Spørgende forcklarede hand, at der er, i disse Mærragers fielde 6 Mad Lauger eller \textit{Familier}, foruden Eett Mad Laug Som Nys er tilckommet; hvorpaa hand blev \textit{dimittered}, og Rætten paa dette Stæd Slutted:\hspace{1em}\par
\centerline{Ole Larsen Nøstad. (L. S.) bæss Joensen Mærrager. (L. S.)}\centerline{P. Schnitler (L. S.)}\hspace{1em}
\DivII[Bilag: Om Meråker anneks]{Bilag: Om Meråker anneks}\label{Schn1_24768}\par
\centerline{\textbf{Bielage} for \textit{Stiørdalens Fogderie}.}\par
\textbf{Mærrager}. Dend nærmiste Norske grændse-bøjd til \textit{Jemteland}, er ett \textit{Annex} under \textit{Stiørdalens} Præstegield, i hvis Kiercke om aaret 3 gange Præddickes.\par
Dette \textit{Mærrager} er een trang Dahl, Een Fierdendeel miil breed meere og mindere imellem Fielde, Strækkende Sig fra øster i væster fra dend øverste til Neederste Gaard i bøjden ad \textit{Hegre Annex} Een Miil lang, omgiivet med Fielde, undtagen paa dend væstere Siide, hvor dend anstøeder til bem{te}\textit{Hegre Annex}, Nembl: paa dend østere Siide ad \textit{Jemteland}, med det Field \textit{Langvola}, 3 miile fra Mærrager bøjden liggendes med fleere, imellem hvilcke Fielde og bøjden fleere andere fielde og eendeel af \textit{StorTæveldals} Gran-Skoug ere; \hypertarget{Schn1_24831}{}Retten sættes i Verdals Fogderi. Paa dend Søndere Siide ad \textit{Tydalen} med \textit{Øye}-Field og \textit{Fongen}, i hvilcken Stræckning bønderne have nogle høe-Slotter og Sæterboeliger med nogen liiden Gran-Skoug; Paa dend Nordre Siide til \textit{Werdalen} med ett Stoert field \textit{Hermanssnosen} imellem hvilcket field og bøjden nogle Søer og Elfve ere, Som i \textit{Protocolln} er beskreevne. Denne bøjds \textit{district} i øster ved grændserne Reigne bønderne fra \textit{Einbogen} i Søer til \textit{Kiølehougen} i Nord omtrent 6 Miile breed. bøjden bestaar af en 50 bønder Som leeve af deris gaarder og Skoughougst til Saug-brug jtem liidet Fiskerie, dog mest af deris qvæg, Siiden denne bøjd er Frostnembt; De brænde og Jern af Myhr Jord, og Fabriqvere der af Jern-Fang. ‒\par
Jmellem denne bøjd og \textit{Jemteland} er ingen alfarevej, men om viintern fares der undertiiden af nogle enckelte Folck over \textit{Hol} Søen, og \textit{HolsiøRuven} need i \textit{Jemteland} over \textit{Reen} Søen til \textit{Wolland} gaardene Samestædz; Om Sommern er Vejen meget vanskelig, Myhret og Steenet, Saa der fares kun af Enckelt folck med een hæst, Som maa kroge Sig frem over Myhrene og Fieldene forbie Vandene, Som hand bæst kand. J denne Mærrager bøjd ere 10 Mand Skiiløber-Soldattere og i næste \textit{Hegre Annex} en 14{ten} Mand \textit{dito}, foruden de Staaende Soldattere, hvor af i dette \textit{Størdalens} Præstegield er omtrendt 2: \textit{Compagnier}. ‒\par
Forestaaende til deels selv Seet, til deels af bønderne erfaret.\hspace{1em}
\DivI[I Verdal fogderi: 4 vidner.]{I Verdal fogderi: 4 vidner.}\label{Schn1_24925}
\DivII[Mai 8. Fra Meråker til Sul i Verdal]{Mai 8. Fra Meråker til Sul i Verdal}\label{Schn1_24927}\par
\textbf{A{o} 1742{ve}} dend \textbf{8{de} May} har \textit{major Schnitler} giivet Sig paa Reisen fra Mærrager Fieldleetz til \textit{Suul} gaardene i \textit{Werdalens} Præstegield, Som holldes for en 5 field Miile, derpaa brugendes dend \textit{Voitür} af \textit{Finnernes Keerester} med Rendyr for, Som ellers dend \textit{ordinaire} land-vej igiennem \textit{Stiørdalen}, \textit{Schongen} og \textit{Werdalen} hiid til \textit{Suul}-gaardene var omtrendt en 15{ten} Miil, og paa denne tiid meget vanskelig at \textit{passere}. ‒\hspace{1em}
\DivII[Mai 9. Rettsmøte på Sul]{Mai 9. Rettsmøte på Sul}\label{Schn1_24998}\par
\textbf{A{o} 1742{ve}} Dend 9 may blef her paa \textit{Suul} i \textit{Werdalens} Præstegield\textit{examinations} Rætten holden; Tilstæde værende \textit{Peder Henrichsen} og \textit{Peder Jørgensen Suul}, Som Laug Rættes Mænd. ‒\par
Viidner fremstilledes \textit{Tørris Nilsen} og \textit{Anders Olsen Suul}. Af Kongl: \textit{Ciiviil}-betiendtere mødte jngen, Som formeentlig denne tiids u-føere er at tilskriive. ‒\par
Dend Kongel: \textit{order} af 16{de} Martj 1742: blef da oplæst med Eedens forcklaring for viidnerne, Som derpaa bleve tagne i Eed. ‒\hspace{1em}\par
1 Viidne i \textbf{Werdalen} ‒\par
heeder \textit{Tørres Nilsen Suul}, er føed paa gaarden \textit{Suul} i værdalen af bønder folck, 55 aar gammel gift har 5 børn, er bonde her paa \textit{Suul} gaard, Som ligger i \textit{Wuckue Annex} under \textit{Werdalen} Præstegield og Fogderie. ‒\par
1: Spørsmaal ‒ hvor langt ligger denne gaard fra de nærmiste grændser ad \textit{Jemteland?}\par
\textit{Resp:} Dend ligger fra nærmeste lande-mærcke \textit{Storsiø sundet} 1 1/2 miil, som kand være efter Nye Reigning 1 1/4 Miil.\par
2: Hvad er Landets beskaffenhed imellem bem{te} grændse Stæd og disse gaarder?\par
\textit{Resp:} Først paa \textit{Suul} gaardene, er een gran Skoug, 1/2 miil breed i øster, og fra Søer i Nord 1 god mil lang; og udbreeder Sig denne Skoug i Nord ost 3 mile lang. Paa denne \hypertarget{Schn1_25127}{}Schnitlers Protokoller I. Skoug i øster følger een Søe, kaldes \textit{Krog}-Søen, mest rundagtig, der kand være imod 1/4 miil Stoer paa alle Siider. Af denne \textit{Krog}-Søe udfalder een bæck, ved nafn \textit{Krogka}, der rinder i Nord væst 3/4 miil vejs lang, ind i \textit{Suul}-Elven, væsten for \textit{Suul} gaardene. Paa denne \textit{Krog}-Søe i øster følger ett field, kaldet \textit{Suul} fieldet; dette field Stræcker Sig fra Sodost til Nordvæst 2 1/2 mil lang, og har adskillige berg-klimper paa Sig, bart, Skallet, uden Skoug, græs og Maase; breed er det over fra øster i væster 1 mil vejs; Efter nærmere til Spørgende, forcklarede viidnet, at Skougen øst efter er 1/2 Miil breed med \textit{Krog} Søen indberegnet; Miidt paa dette \textit{Suul} field ligge 3 berg-klimper fra Sydvæst i Nord ost efter hinanden; De 2{de} yderste Klimper paa begge Siider ere høyere, end dend Mellemste, og denne Mellemste klimp kaldes \textit{KochSteen} Som er kun Saa høy, at dend Rækker een karl under armen, naar dend er udstrackt, og er mesten treeckandtet af Sin dannelse, Flat oven paa; Denne \textit{KochSteen}, har fra Sig i øster ‒ 1/8 miil til \textit{Stoer}Søens væstere Endes begyndelse, og her fra til StoerSøe \textit{Sundet} i øster 3/8 miil; her ved denne \textit{Koch} Steen er det, at de Svenske gierne vil have grændse-Mærcket imellem \textit{Werdalen} i Norge og \textit{Jemteland}: Da det dog er en beckiendt Sag for alle her boende, at \textit{StoerSøeSundet}, en half miil fra \textit{Koch Steen} i øster liggendes, er ett lande-mærcke imellem Norge og Sverrig; om dette \textit{Koch} Steen fortæles, at een gammel Skick der veed har værit og endnu er, der holdes Saavel af de Norske, Som Svenske, at, hvo Som første gang farer denne \textit{Koch} Steen forbj, bliver handslet, det er, maa giive til Sitt følge ett Dricke-Laug eller penge dertil, i viidrig fald hand med det øye- eller Vedder-Nafn: Guuelbeen, spotteligen belegges, hvilcket Nafn den beholder, indtil den faar løeset Sig der fra med handsel-pengene; ‒\par
3: hvilcke er de nærmeste gaarder, paa denne Norske Siide, nærmist ved grændse fieldene, fra Søer at Reigne? af hvad beskaffenhed er landet? og hvad Næring bønderne bruge?\par
\textit{Resp:} De nærmiste gaarder i Søer, ere Mærrage bøjden, her fra \textit{Suul} 5 field miile, og fra grændse Mærcket i væster omtrendt 3 Miile beliggende; om landskabet derimellem udsiiger hand det Samme Som Mærrager-Mændene ved dette Spørsmaal tilforn forcklaret have, undtagen det, at hand veed, at bønder af \textit{Schongens} gield have dend Søe \textit{Færen} under deris gaarder bøxlet. Disse \textit{Suul} gaarders, beskaffenhed er nu Saa, at de Saae liidet, omtrendt hver bonde et bismær pund, eller 1/2 Vog-Korn, men nu i 4{re} aar icke høsted det Ringeste der efter, Saasom det fortiilig er affrossen; Nogen liidet Eng og liidet fiskerie i \textit{Krog}- og \textit{Jnd}-Søerne haves; Gran-Skoug er her vel nock, men \textit{emploieres} nu iche til noget Saugbrug, til deris fortieniste; Skytterie er her kun Ringe, fordj \textit{Lap-Finnerne} drage af med det viildt, om noget findes. ‒\par
J Nord her fra ere de nærmiste gaarder, de Saa kaldede Nyebyggere i \textit{Helgaae} dalen ved \textit{Wærens-aae}, 3 field Miile herfra beliggende; Denne \textit{Wærensaae} udckommer fra \textit{Wærens}-Søe og løeber i væster først 3 miile til \textit{WugKue} Kiercke; Paa dends Søndere Siide dend foreener Sig med \textit{Suul}Elfven; Der efter Rinder denne \textit{Wærensaae}, i begyndelsen i væster, Siiden i Nord-væst 1 1/2 miil paa dend Søndere Siide af \textit{Wærdals-øeren} ind i \textit{Schonæs} Fiord, Som er dend Same med \textit{Strind}- og \textit{Trondhiems}-Fiorden. Eet Støcke fra \textit{Suul} Elfven, \hypertarget{Schn1_25338}{}1 Vidne i Verdals Fogderi. opckommer af fieldene og Myhrene een liiden Elfv, kaldet \textit{Tromsdals} Elfven, Som løeber 1/2 miil vejs i Nordost ind i \textit{Suul} Elfven, Kort førend denne falder ind i \textit{Wærensaa};\par
Tædt ved \textit{Suul}-Elv i væster der fra, og 1/2 Miil i Søer fra \textit{Wærens aae} ligger dend forrige Norske grændse-Skandze, af Tømmer opbygget, Som nu er gandske forfalden. ‒ ved nafn \textit{Steene} Skandze. ‒ 1 1/2 miil i væster fra denne \textit{Steene} Skandze, og 1/4 miil i Syd væst fra \textit{Werdals øeren} ligger \textit{Schonæs} Skandze, bestaaende af Jord Voller, hvor endnu \textit{Commendant} er inde, hvilcken \textit{Schonæs} Skandze ligger j \textit{Schongens} Præstegield, og i Nord ost 1/2 miil fra \textit{Levangers} Platz, hvor aarlig i martj maanet et betydelig Marcket holdes, og af \textit{Jemterne} Stærck did Søeges. ‒\par
Dend Søe, hvor af denne \textit{Wærens aae} udspringer, er Rund og 1/2 miil Stoer, og fiskes der i, øret, Røe, og Lache, hørende under dend Gaard \textit{Stickstad} i \textit{Wærdalens} hoved-Sogn; og denne \textit{Wærens} Søe og Elfv giiver \textit{Wærdalens} hoved Sogn, Præstegield, og Fogderie deris Nafn. ‒\par
Synden for denne \textit{Wærens} Søe og Norden for \textit{Stoersiø-sundet} Miidt der imellem ligger dend Søe \textit{Jnd}, som Stræcker Sig fra øster i væster 1 Miil, og fra Søer i Nord er 1/8 Miil breed, Fiske-Riig paa øret, Røe, og Lache. ‒ Fra denne Søe udckommer \textit{Suul} Elven, løbendes i væster 4 1/2 miil i \textit{Wærensaae}.\par
Ellers er Landskaber imellem disse \textit{Suul} gaarder og Nye-byggerne Myhret, med Skoug og nogle Smaa Fielde, udyrckelig og u-beboet, undtagen hits og her Nogle Smaa Sæterboeliger kand findes. Nye-byggerne faa undertiiden i goede Aaringer Korn, have Eng-land til deris qvæg og Timmer hougster til Saug brug, hvor af de nære Sig. ‒\par
4{de} hvilcke ere de nærmiste gaarder paa dend østlige Svenske Siide nærmist til grændse-fieldene, fra Sønden at reigne? og af hvad beskaffenhed er landet? og hvad Næring bønderne der bruge?\par
\textit{Resp:} Nærmist er 2{de}\textit{Wollan}-gaarder i \textit{Jemteland} Som før er beskreeven i Mærrager, ligger i Nord der fra 1 Miil \textit{Tongbøl}, een gaard, der fra i Nord 1 miil \textit{Rein}berg, een gaard, ‒ endnu viidere her fra 1/4 miil vejs er dend gaard \textit{Sta}; Disse gaarder ligge omtrendt 3 eller 4{re} miile i øster fra de Norske lande Mærcher, og er der imellem, indtet uden Skoug og Myhr Sletter ubeboed og u dyrckelig, Saa at man Neppe kand komme frem der imellem med een hæst om Sommern.\par
Østen for dend gaard \textit{Sta}, ligger een Gaard \textit{Sundet} og derved over \textit{Sundet} Strax \textit{Due} Skandze Norden for Sundet, Som var opbygget af jord og Riis-\textit{Fachiner}, men nu er gandske forfalden; østen for denne Skandze har i Kriigens tiid endog været en anden nembl: \textit{Jerpe} Skandz, hvor langt derfra veed hand iche; Som liigeleedes nu Skal være forfalden: og Skal der ved Som hand har hørt, være opbygt een Smelte-hytte, hørendes til det \textit{Liusendalsche} kaabber værch. ‒\par
Som bem{te} gaard \textit{Sta} ligger fra \textit{Suul} gaardene i Norge omtrendt 5 1/2 Svenske Miile, Saa have de Svenske For \textit{pasagens} Skyld til Norge, Sæhrdeelis til \textit{Levangers} Marcked efter Kriigens tiid igien opbygged nogle Stuer under vejs, hvor de Reisende kand have deris tilhold i, Saasom efter di fra disse \textit{Suul} gaarder til grændse-Mærcket \textit{Stoer} Siø Sund er 1 1/2 Miil, Saa er dend første nembl: \textit{Schalstue} 1 miil der fra i øster, dernest herfra 1 miil i øster \hypertarget{Schn1_25583}{}Schnitlers Protokoller I. dend anden gaard Midtstue, og her fra 1 Miil i øster dend 3{die}\textit{Stald}kier Stue opsatt, hvor fra til gaarden \textit{Sta}, er 1 1/2 miil i øster; alle liggendes efter \textit{Stoeraaen}, der udkommer af dend Søe \textit{Schalsvatn}, 1/8 miil østen for \textit{Stoersiøens} østere Ende, hvilcket \textit{Schals}vatten Stræcker Sig fra Søer i Nord 1 miil og fra øster i væster en 1/4. ‒\par
\textit{Norden} for disse gaarder ligge vel andere j Samme nembl: \textit{Undersaagers} Præste gield nær grændserne østen for de Norske \textit{Nyebyggere}, men hand har iche værit der, og er der for ej der beckient. ‒\par
For beskrevne gaarder ernærer Sig Saaleedis: i \textit{Schalstuen, Midtstuen} og \textit{Stalkierstuen}, voxer indtet korn, men der imod have de godt Engeland, Fiskerie og noget Skytterie; de øfrige gaarder have foruden dette dend beleilighed, at der voxer korn, naar frosten iche overjiler det; Saa have de og, brug med at hugge Tælg Steene eller grøtt Steene, af ett berg ved \textit{Handøl}, Som kaldes Telg berget, hvor af blødagtige Steene hugges og dannes til ofner, gryder, Pander og Tørch-hylder; De bruge og Smaa handel med jern Smiefang at kiøbe i \textit{Fahlun}, og føre det over til Norge, og der for at tilforhandle Sig Siild og Fiskevahrer. ‒\par
5: Hvilcket er det første field eller Stæd, Som holdes for at være ett lande mærcke imellem Norge og Sverrig paa denne kandt, det viidnet har hørt eller kiender, Sønden i fra at reigne?\par
\textit{Resp:} Dett første field, viidnet er beckiendt, er \textit{Kiølhougen}, hvis leje og landskab paa Siiderne see 1 Vidne i Størdalen 5. Sp: 7. N: Dog forcklarer hand grændsens gang derveed Saaleedis, at dette Field er oven til Steilt og Spitz, og at øverst paa denne Spitz lande-Mærchet gaar, anderleedis hand det aldriig hørt haver; givende derfor dend \textit{Raison}, at fra Mitten af \textit{Holsiø Ruven}, hvor nest forrige Mærcke er, gaar \textit{Linien} liige i Nord til det østerste af \textit{Kiølhougens} Spitze, og fra dette østerste Støcke, gaar \textit{linien} fremdeelis liige i Nord til \textit{Stoer} Siø Sundet; Thj dend østerste deel af dette \textit{Kiølhougen}, gaa liige neer i Nord til \textit{StoerSiøn}, og giør dette Sund paa Samme \textit{Stor-Siøn}. ‒\par
Og Er da dette \textit{Storsiøsundedt} det andet lande-Mærcke Viidnet kiender at være imellem \textit{Jemteland} og \textit{Wærdalens} Fogderie i Norge. ‒\par
Denne \textit{Stoer} Søe beskriiver hand nu Saaleedis, at den Stræcker Sig fra væster i øster 1/2 Miil lang, og hvor hun er breedest paa begge Ender, kand det vær 1/16 deel miil; Midt paa denne \textit{Storsiø} er ett Sund icke breedere End ett Pistol-Skudt over, og dette Sund \textit{formeeres} el: Dannes af \textit{Kiølhougens} østere deel, Som gaar Saa langt ud i StoerSiøn; Dend Elfv, Som af denne Stor-Søe udgaar i øster, kaldes \textit{Stoer} Søe \textit{aaen}, og rinder i \textit{Schals} vatnet, og der fra Viidere i øster, Som før er beskreevet. Landskabet østen for dette \textit{Storsiøsund} er Myhret og blødt med nogen Smaa biercke-Skoug, indtil forbi \textit{Schals}-vatnet der hvor gran-Skoug tager paa til mod \textit{Schals-Stuen;} ‒\par
Paa dend væstere Siide af denne \textit{Stor}Siø ligger \textit{Suul}-fieldet, Som ved 2{det} Spørsmaal her, er beskreeven. ‒\par
Landskabet paa dend Nordre Siide af dette \textit{Storsiøsund} er Stycht land af berge, Myhr, med nogen biercke- Furru- og Granskoug, hvor igiennem iche er om Vintern vel fremckommendes med hæst og Slede ej heller om Sommern med hæst, uden med Stoer Møje og \hypertarget{Schn1_25801}{}1 Vidne i Verdals Fogderi. Krogning; Det Stygge land i Nord, er 2 gamle miile, Som hand meener, at være 1 1/2 Nye Miile lang; hvorpaa i Nordost følger det field \textit{Finvohla}, som er det ‒\par
3{die} Lande Mærcke Viidnet kiender. ‒\par
Dette \textit{Finvohla} er høyt og Flatt oven paa, det Stræcker Sig fra Sydvæst i Nordost, imod 1/2 miil, og mest liige Saa breedt; midt over \textit{Finvohla} gaar lande Mærchet; landskabet paa dend østere Siide er Slett og Myhret med nogen Furru- og Gran-Skoug; Ellers u-frugtbar og u-beboet. ‒ Dend væstere Siide af dette \textit{Finvola} bestaar af Gran og Furru Skoug, Stræckende Sig til \textit{Jnd}-Søen, og tilhører \textit{Suul} gaardene. ‒\par
Landskabet paa dend Nordre Siid er liigesom paa dend østere Siide, af Myhr og Skoug, dog meer grubbet og dahlet, omtrendt 1/2 Miil i Nord væst til \textit{Skiærrevandet;} og midt over dette \textit{Schiærrevand}, har viidnet hørt, at LandeMærcket gaar, Saa at denne Midte er det\par
4{de} grændse-Mærcke, Viidnet kiender ‒\par
Dette \textit{Schiærrevand} Stræcker Sig fra Søer i Nord 1/2 miil lang, er breed halfv Fierding, har i Sig Fiske, Røe og Lache. ‒\par
Landskabet østen og væsten for \textit{Schiærvandet} er liige som af \textit{Finvola}: dog er Flatere og Slettere; Dend Nordre Siide af \textit{Schiærrevandet}, er Skouget og Myhret 1/4 miil vejs indtil \textit{Schiærrevandshougen. Dette Schiærrevandshoygen} er een Jord-houg eller Aas med granSkoug oven paa, Rund og Spitz op i vejret, dog kun liiden, omtrendt 1/16 deel miil over fra foed til foed. Fra Midten af \textit{Schiærrevandet} til dend Spitz af \textit{Schiærrevandshougen} gaar \textit{Linien} beent i Nord, Saa at denne \textit{Schiærrevandshougen} er det ‒\par
5{te} Lande Mærcke. ‒\par
Her ved erindrede Viidnet, af gamle Folck og Sine Forældre og Farfader at have hørt, Tædt Norden for \textit{Schiærre}-vandet, der hvor dets midte er, Skal have Staaet een Mærcke Steen, Flad needen under og Spitzet oven til, Som ett hus-Tag; i Steenen Selv, skal der have værit Skreeven bogstaver, og needen derunder ligget kuld og kriidt; hvilcken Steen af Mennisker har værit henlagt, og iche Større, end at een karl kundet løftet den; Dette Siiger nu viidnet, Som mældt at have hørt, mens da hand har værit didhenne, at See efter Steenen, har hand ej fundet dend. ‒\par
Landskabet af denne \textit{Schiærrevandshougen}, er det Samme paa østere og væstere Siide, som ved \textit{Schiærrevandet} er beskreeven. ‒\par
Dends Nordre Siide er liigeleedes Myhrland med Skoug: dog drubbet med berge iblandt, 3/4 Miil i Nord liige til \textit{Straadals} Fossen, Som er et vand-fald over ett berg kaldes \textit{Straadals} berget, kommendes næst østen for dette berg af enn Kyn; Som der \textit{formeeres} og Samles af adskillige Smaa tilløbende bæcke.\par
Denne \textit{Straadals} Fos, naar dend har Styrdtet Sig over Sit berg i een Kyn needen derunder, Sambler dend Sig til een Elfv, og løber 1/2 Miil i væster ind i \textit{Væren} Søe, Som før er beskreven. Denne Fos ligger liige i Nord fra \textit{Schiærrevandshougen}, øg er beckiendt det 6{te} Lande-mærcke. Dend østere og væstere Siide Af denne Fos, er liige Saa Styg, Som ved nest beskreevne landeMærcke, af Myhr, berge Skoug og Dybe Grubbe. Længere i Nord er viidnet iche beckiendt. ‒\hypertarget{Schn1_25963}{}Schnitlers Protokoller I.\par
6: Spørsmaal: Til hvis gaarder hører disse beskrevne fielde? eller under h(v)is gaarders grund ligge de? eller om de ere kongens Alminding?\par
\textit{Resp:} han veed iche andet, end at de ere Kongens Alminding, Stræckende Sig fra \textit{Kiølhougen} til \textit{Fossen} mod \textit{Werdalen}.\par
7{de} Om der har værit nogen tvistighed imellem De Norske og Svenske undersaattere om disse Fielde? ‒\par
\textit{Resp:} Nej! det hand iche har hørt: undtagen \textit{Storsiøsundet}, hvilcket de Svenske vil tilEigne Sig, indtil \textit{Koch} Steen i øster der fra; thi af hans forfædre har hand hørt, at naar de Svenske har Satt een Mærcke-Stolpe af Træ ved \textit{Koch} Steen, har dend Norske øfrighed ladet dend fløtte til \textit{StorsiøSundet} i øster; Saaleedes er denne fløtning af MærckeStolpen nogle gange Skeed, af begge parter, indtil dend omsiider er bleven borte, efter hvis \textit{order} veed hand iche. ‒ Dog har hand i hands liivets tiid iche hørt til nogen tvistighed om Dette grændse Stæd fra nogen af Siiderne. ‒ hand erindrer, af hans Farfader at have hørt, at naar de Svenske skulde Forsvare Skille-mærcket ved \textit{Koch} Steen, var deris hensigt, at gaa ind paa Eendel af Røraas i Søer, og paa \textit{Finliene} af \textit{Sneaassens} gield i Nord. ‒\par
8{de} Hvad, Nytte, godhed, og herligheder er der ved disse grændze-fielde?\par
\textit{Resp:} Slett ingen; Det mellem Rum af \textit{Koch}-Steen, og \textit{Storsiøsundet} er og i Sig Selv af ingen værdie, med minder de Svenske der efter ville træcke deris \textit{linie} og \textit{prætention} baade i Søer og Nord, Som Sagt er. ‒\par
9: hvor langt liigge disse grændse-fielde og Stæder fra bøjdene og lande-vejene?\par
\textit{Resp:} Det nærmiste Mærcke-Stæd \textit{Storsiøsundet} er fra disse \textit{Suul} gaarder 1 1/2 miil, hvor fra landevejen viidere gaar need ad \textit{Werdalens} bøjd. ‒\par
10: hvor underholdning for Folck, og beete for hæstene paa eller ved disse grændseFielde ere at faa?\par
\textit{Resp:} hand ved ingen nærmere Stæd, end \textit{Wærdalsøeren}, 7 miile fra \textit{Storsiøsundet} eller paa \textit{Levanger} 8 miile der fra; beedte til hæstene, Siiger hand, vil findes noget hvert Stæd i Field-Dahlene. ‒\par
11{te} Hvad mænd hand kiender at være beqvæmmest og kyndigst til at vejviise grændsemaalerne dend Rætte Grændsens Gang paa denne kandt?\par
\textit{Resp}: her i bøjden ved hand ingen at Nafngiive; thi de Som kan være noget kyndige, ere gamble og Skrøbelige: Dog foreSlaar hand dertil een \textit{Lap Fin}Zacharias \textit{Nilsen}. ‒\par
12{te} Angaaende \textit{Finnerne} forcklarer hand, at de Som nu ere i Mærrager fielde have undertiden deris tilhold i disse \textit{Werdals}-fielde; hvorpaa viidnet blev \textit{dimittered}. ‒\hspace{1em}\par
2 Viidne i \textbf{Werdalen}\par
heeder \textit{Anders Olsen Suul}, føed paa gaarden \textit{Suul} i \textit{Werdalen} af bønder folck, 60 aar gammel, gift, har 7 børn, er bonde her paa \textit{Suul}-gaard, Som ligger i \textit{Wuekue Annex} udj \textit{Werdalens} Præstegield og Fogderie ‒\par
Fra 1{te} Til 11{te} Spørsmaale \textit{inclusive}, Svarer uden miinste og Ringeste \textit{difference}, ligesom 1{te} viidne udsagt haver; hvorpaa blef \textit{dimittered:} og Rætten paa dette Stæd Sluttet. ‒\hypertarget{Schn1_26200}{}3 Vidne i Verdals Fogderi.\par
J kraft af dend kongel: ordre betydede til Slutning \textit{major Schnidtler} laugRættes Mændene, at Saa Snart de faar høre grændse-Maalerne, at Nærme Sig her hiid, til \textit{Wærdalens} grændse Stæder, maa de Strax lade Lensmanden det viide, og Siige ham til forn, at hand maa møde grændse-Maalerne ved første grændse-Stæd i Søer, uden Ringeste ophold, med een kyndig vejvisere og fornødne arbejdere, Sambt (om de Skulle forlange det), viidnerne, i fald de fandt fornøden at tage deris førcklaring over ett eller andet meere end Som nu udsagt er; Liigeleedes maa Lensmanden forud være betænckt paa, at have Flotter, broer eller baater tilreede for grænse Maalerne og deris følge, at komme over Sunde, Søer, Elfve, eller Myhr-vand derpaa. ‒\par
Grændse Maalernes andkomst vændtes at blive Sist i Julio. ‒\hspace{1em}\par
(sign.) Peter Schnitler. (L. S.) {Peder Hendriksen Suul. (L. S.)}\centerline{Peder Jørgensen Suul. (L. S.)}\hspace{1em}
\DivII[Mai 10. Fra Sul til Helgådalen i Verdal]{Mai 10. Fra Sul til Helgådalen i Verdal}\label{Schn1_26239}\par
A{o}1742{ve} dend 10 may begav \textit{Major Schnitler} Sig paa Reisen Field-leedz fra \textit{Suul}- bøjden til Nye-byggerne i \textit{Helgaaedalen}\textit{Werdalens} Præstegield, 3 Miil der fra liggendes, brugendes \textit{Finnernes Kiereester}. ‒\hspace{1em}
\DivII[Mai Rettsmøte på Julneset i Helgådalen]{Mai Rettsmøte på Julneset i Helgådalen}\label{Schn1_26276}\par
1742 Dend 11{te} may blev her i \textit{Helgaadalen} giordt Anstalt til laugrættes Mænd og Viidner at faa til denne forrætning, Sambt bud Skikket til \textit{Lap-Fiinnerne}, 3 miile her fra i \textit{Werdals} fieldene, for at faa nogen af dennem herhiid, til \textit{examination}, om forgodt Skulle findes, Sambt til viidere Skydz ved deris Reendyr til Nærmiste grændse-bøjd i Nord; Siiden man hørte af bønderne her, at Vejen til \textit{Werdals} bøjden for hæste var \textit{u-passabel}; Næstfølgende 12. maij blev da Rætten Satt paa dend gaard \textit{Julnæsset} i dend bøjd \textit{Helgaaedalen}, Som ellers og kaldes \textit{Nybyggere}, tilstæde værende de 2{de} Laugrettes mænd \textit{Christopher Larsen Snekkermoen} nogle og 40 aar gammel; og \textit{Anders Olsen Kleppen} Omtrendt 50 aar gammel, begge boendes her i \textit{Helgaadalen}, Som Viidner fremckom \textit{Tomes Jacobsen}, Een gammel Døv-man, Som besværlig kunde høre og \textit{Otto Lassen}, begge her boendes i \textit{Helgaadalen}; andere viidner man ej paa dette Stæd kunde erholde; af \textit{Ciivele} betiendtere var ingen tilstæde, for hvilcke, bønderne Sagde, at det var ugiørlig at komme fra bøjdene hiid op til fields. J Sambtlige deris paahør blef dend kongel: \textit{ordre} til denne Rættes holdelse forckyndt, der næst Eedens Forcklaring af lov:bogen viidnerne forelæst og betydet, hvilcke derpaa aflagde deris \textit{Corporlig} Eed, tii at Siige Sandhed om grændsens gang paa denne Kandt. ‒\hspace{1em}\par
3{die} Viidne i \textit{Werdalen} hvilcken Saasom hand var tung-høert, Saa Stoed hands Søn Arrendt hos ham og forcklarede ham Rættens Spørsmaal. Dette Viidnes Nafn er \textit{Tomes Jacobsen Helgaasen}, er Føed paa gaarden \textit{Suul} i \textit{Werdalen}, af bønder folck, er over 80 aar gammel, gift, har 6 børn, er boende Paa \textit{Helgaasen} her i \textit{Helgaadalen}. ‒\par
Til 1{te} Spørsmaal, Som skeed er til 1{te} Viidne i \textit{Werdalen} Svarer: Denne bøjd \textit{Helgaadalen} ligger fra nærmiste lande-Mærcke i øster Nembl: fra \textit{Schiærre}vandet 3 1/2 gaml: Miile, \hypertarget{Schn1_26415}{}Schnitlers Protokoller I. Som man kan reigne, efter Laugrættets formeenende omtrendt 3 nye Miile, i \textit{Wukue Annex}\textit{Werdalens} Præstegield og Fogderie. ‒\par
Til 2{det} Spørsmaal, Svarer: Nærmist fra \textit{Helgaadalen} i Sydost ligger Een Skoug-aass kaldet \textit{Snecher-aass}, med Gran- og Furru-Skoug paa; Denne \textit{Snecheraass}, Strækker Sig fra Søer i Nord, at forstaae, fra \textit{Juldalen} til \textit{Fagerlje}-Fossen 1: Miil lang, og fra væster i øster, at forstaa fra \textit{Snechermoen} til Stoerlje 1 Miil; Paa denne \textit{Snecheraassen} fra \textit{Storlie} at reigne følger nogle Myhr-dale med nogen Tynd-Skoug i, 1 miil breed til \textit{Billings}-Kynnen j øster, og imod 2 miile lang fra \textit{Juldalen} til over \textit{Tværaadalen} i fieldet; Paa disse Skougdale i øster kommer ett field, ved Nafn \textit{Billings} fieldet, myhret og med nogen Tynd Skoug paa; Dette \textit{Billings}-field fra \textit{Billings}-Kynnen til \textit{Schiærre}-vandet er 1: miil over i øster, og Stræcker Sig fra Søer i Nord, Nembl: fra \textit{Billingsdalen} til \textit{Wæren} Søe 1 Miil lang eller noget miindere; ved disse \textit{Specificerede} 3 miile forstaar Viidnet dend Nye maaling. J disse forbem{te} Skoug-Dale ligger adskillige Vand-kynner eller kierner, hvor af bæcke riinde. i Nord i \textit{Wærens-aae}. Dette Landskab imellem \textit{Helgaadalen} og \textit{Sehiærrevandet} er ganske ufrugtbar, u-beboet, og udyrckelig, hvor og ingen haver høe-Slott eller Sæterboeliger engang, ej heller kand boe og tilholde, uden \textit{Lap-Fiinnerne}, Som der Stundum Sidde;\par
Til det 3{die} Sp: Svarer: De nærmiste gaarder ved grændse fieldene til dennem i Søer ere 3 field-Miile hvorfra liggende \textit{Suul} gaardene i dette \textit{Werdalens} Præstegield; hvad nu det 1{te} viidne paa disse \textit{Suul} gaarder i \textit{Werdalen} ved dette 3{die} Spørsmaal har udsagt, om Landskabet imellem Sig, og Nybyggernis deris Næring, jtem om \textit{Wærens}-Søe og \textit{Wærens-aae}, om \textit{Suul} Elven og \textit{Jnd}-Søen Sambt \textit{Tromsdals} Elven, fremdeelis om \textit{Steene}- og \textit{Schaanæs}-Skandzer, det Stadfæster hand, undtagen der i, at hand Siiger, at \textit{Wærens-aae} fra \textit{Wærens}-Søe til \textit{Wuekue} Kierche her i væster 4{re} miile at løbe. De nærmiste gaarder til denne \textit{Helgaadals} bøjden i nord 3 field-miile, Som vel kand reignes for 3 Nye Miile herfra, er \textit{Aangdals}-bøjden bestaaende af 5 gaarder, i \textit{Scheiis annex}, \textit{Sparboe} Præstegield, \textit{Jnderøens} Fogderie; hvor fra lande Mærcket imellem \textit{Sparboe} og \textit{Jemteland}, Nafnl: \textit{Gaundalsvudu} ligger i øster 3 field-miile. ‒\par
Angaaendes landskabet imellem \textit{Helgaadalen} og \textit{Aangdals}-bøjden i Nord, Siiger han, at nærmist i Nord fra \textit{Helgaadalen} ligger ett Stoert langt field, kaldet \textit{Halbach}-fieldet, Som oven til er Rundagtig, bart og Skallet uden Skoug og græs, Stræckende Sig fra væster i øster, at Reigne fra \textit{Schiæchermoe} gaarden i \textit{Helgaadalen}, hvor fieldet opstiiger i Væster, indtil det field i \textit{Jemteland}, kaldet \textit{Manshougan}, i øster, og i Sæhr indtil Samme bem{te}\textit{Manshougans} Norderste haug omtrendt 5 gode Nye Miile lang; breedden af dette \textit{Halbacs}-field, paa det Stæd hvor \textit{Straadals} Fossen er, bliver fra denne \textit{Straadals} Fossen 2 field-miile i Nord til \textit{Gaundalsvudu}: mens breedden af dette Field længere hiid i væster, at reigne her fra \textit{Helgaadalen} over til \textit{Schiæcheraadalen} er 1 miil Stoer; ‒ Denne \textit{Schiæcheraadalen} er Een Skoug-dahl af gran og Furru-Træer, Stræckende Sig fra væster i øster, Nembl: fra den gaard i \textit{HelgaadalenSchiæchermoen} til \textit{Schiæchervandet} 1 1/2 nye miile lang, og gaar fra Søer i Nord 2 Miile til Maadz dend første gaard i \textit{Aangdals} bøjden. ‒\par
Foruden forbeskrevne \textit{Weren} Søe og \textit{Wæren Aae} ligger imellem denne bøjd og \textit{Aangdals}bøjden, 2 mile i Nord ost fra hin dend \textit{Søe Schiæchern}, hvori fiskes øret, dend Strækker \hypertarget{Schn1_26760}{}3 Vidne i Verdals Fogderi. Sig fra væster i øster 1: liiden miil lang, og er 1/4 breed; og bruges som en Alminding, af \textit{Wærdalingerne} og \textit{Sneasingerne} tilffælles; Af denne \textit{Schiæcher}-Søes væstere Ende Riinder \textit{Schiæcher}-Elven Stedze i Væster 2 Miile ind i \textit{Wærens}-Elv ved dend gaard \textit{Schiæchermoens} Syndere Siide i \textit{Helgaadalen}, og er dette Landskab imellem \textit{Helgaadalen}, og \textit{Aangdalen} udyrcket og u-bebygget, undtagen nogle høeSletter, Som \textit{Werdalingerne} kand have væsterst i Skaugdalen. ‒\par
Til det 4 Sp: Svarede Først i øster fra \textit{Straadals} Fossen er nogen Skoug 1/4 miil Stoer i øster, med nogle Smaa Fiske-Kynner; derpaa i øster følger en Søe ved Nafn \textit{Ahnin}- Søe, der Stræcker Sig fra væster i øster 3 miile lang og fra Søer i Nord, hvor dend er Viidest, 1 Miil breed, Fiskeriig paa øret, Røe og Lache; Af denne \textit{Ahnin}-Søes østere Ende gaar Een Elv kaldes \textit{Ahnin}-Elv 1/4 miil vejs i øster i \textit{Kald}-Søen, hvilcken Stræcker Sig fra Søer i Nord 3 Miile lang, og 1 miil breed fra øster i væster; Der af udrinder Elven Viidere i Nordost 1 Miil i den Søe \textit{Liit} og derfra Siiden igiennem \textit{offerdals} Præstegield hvor dend faar det Nafn af \textit{offerdals}-Elv, ind i \textit{Jemtelands Stor}-Søe; Tædt ved forbem{te}\textit{Ahnin} Elv imellem \textit{Ahnin} Søe og \textit{Kald} Søen, dog nærmere \textit{Kald} Søen, ligger Een Gaard af Nafn \textit{Sundet}, Som er frostnembt, Saa at bonden der leever Mæst af Sit qvæg og Fiskerie; østen for \textit{Kald}-Søen ligger Een bøjd, hørende til \textit{Kald}-Sogn; Dendne \textit{Kald}bøyd er og Svagt til korn-land og har ingen anden Vilckaar af deris gaarder, end Som mælt er om \textit{Sunds} gaarden; ellers driive de folck der hos nogen liiden handel, med at føre \textit{Liaar} hid over til Norge, og ckiøbe der fore igien Sild og Fisk, Som de tilbage føere; ‒\par
J Nord ost fra \textit{Straadals} Fossen og i Nord fra den ommældte nogen Skoug (Som ligger nærmist i øster fra \textit{Straadals} Fossen) følger det Stoere \textit{Jemtelands} field, Som Stræcker Sig fra Søer i Nord 2 Miile og fra væster i øster 3 Miile Stoer, ved Nafn \textit{Manshougan}; Som er mest Rundagtig oven paa med 4{re} kiendelige berg-Klimper, Som Stiige der af op i væjret, det er ellers bart og Skallet uden Skoug, græs og Maasse. Østen for dette \textit{Mandshougan} og Een 1/2 miil liige i Nord fra \textit{Sunds} gaarden ligger en gaard, kaldes \textit{Graasiøn}, og viidere 1 miil fra denne gaard i Nord een anden gaard ved Nafn \textit{Kuhlaasen}, hvor 3 bønder boe, hvilcke gaarder have ingen bedre lejlighed, end dend før omtalte gaard \textit{Sundet}; østen for disse benæfnte gaarder er Skoug liige hen til \textit{Kald}-Søen. Norden for disse ligger ingen fleere gaarder, Som hand veed, men det er bare gran-aaser og Furru-Moer ‒\par
Til det 5{te} Svarer: Saavidt hand i hands høye Alder kand erindre, Saa er \textit{Storsiøsundet} østen for \textit{Kochsteen}, dernæst \textit{Finvola}, Saa Midten af \textit{Schierrevandet}, der efter \textit{Schierrevandshougen}, og her fra 1/4 miil vejs i Nord ost fra \textit{Straadals} Fossen, een gammel MærcheSteen med Skrift indhachet og der fra 2 Miile i Nord det Field \textit{Høysætta}.\par
Denne angiivne Gamble Mærcke-steen beskriiver hand efter tilspørgende Saaleedis: i Sidste u-freeds tiider var hand i følge med Een \textit{General} paa denne kandt, og da Saag hand denne Steen, Som af et berg Steeg op i vejret af \textit{Natueren}, Rundagtig, vel af een Mands højde Stoer, flat oven paa, og viid Rundten om, Som een goed Skaarsten Stoer; Skriiftet der paa, miindes hand ikke Rættere end at det Stoed paa dend Søndere Siide: men hvad enten det var bogstaver eller anden Mærcke der i hachet, kand hand iche Siige, og Skal denne Steen have Staaet paa ett Stæd, Som var myhret og berget med nogle fleere \hypertarget{Schn1_27001}{}Schnitlers protokoller I. houger derved; Der Rætten tilspurte ham, om hand trøsted Sig til at finde denne Steen op jgien? Svarede hand, hand meente det mueligens, om hans Alderdom og legems Skrøbelighed ville tillade ham at fare der efter; ‒\par
Om Viidere omstændigheder af lande Mærckernis Strækkning og \textit{Situation}, fandt Rætten iche forgodt, at Spørge Viidnet, Saasom hand undskyldtes dermed, at hands huckommelse der i icke ville staa ham bie, og hand desuden var meget tunghørt.\par
Til 6{te} Sp: Svarer: at det er Kongens Alminding ‒\par
Til 7{de} Sp: Svarer Hand ikke har høert ‒\par
Til det 8{de} Svarer: De ere icke tienlige til nogen uden til \textit{Finnerne}. ‒\par
Til 9: Svarer: Fra \textit{Schiærrevandet, Schiærrevandshougen} og dend ommældte mærcke Steen hiid til \textit{Helgaadalen} og Lande vejen omtrændt 3 miile vejs. ‒\par
Til 10: Svarer, Som første Viidne i \textit{Werdalen}. ‒\par
Til 11{te}: Dend angiivne Mærckesteen med skriift paa kunde vel ingen uden hand Selv maatte See at opfinde den,\par
De andere Mærcker kunde vel nogen hver Som er yngere og \textit{disposere}, end hand, udviise. ‒\par
Til det 12{te} Svarede: undertiiden kand her være 2 el: 3: \textit{Finne-Familier}, undertiiden jngen, Saasom de jefnl: fløtte. ‒ hvor paa Viidnet blef \textit{dimittered}.\hspace{1em}\par
4{de} Viidne i \textbf{Werdalen}. ‒\par
heeder \textit{Otte Lassen Brataas}, føed paa gaarden \textit{Houan} i \textit{Werdals} bøjden, af bønderfolck, 60 aar gammel, gift, har 5 barn, er bonde paa dend gaard \textit{Brataas} i \textit{Helgaadalen}. ‒\par
Til 1{te} 2{det} og 3{die} Sp: Svarer hand liigesom det 3{de} viidne her i \textit{Werdalen}. ‒\par
Til det 4{de} Svarer: hand icke er kiendt østen for Mærckerne i \textit{Jemteland}, uden \textit{Mans}- fieldene eller \textit{hougene}, hvilcke hand beskriiver liigesom 3{die} Viidne her næsttilforn.\par
Til det 5{te} Svarer: hand kiender indtet andet grændse Stæd, end \textit{Straadals}-Fossen, Som hand har hørt at være ett Lande Mærcke imellem \textit{Werdalens} Præstegield i \textit{Trondhiems} Stift og \textit{Jemteland} i Sverrig ‒ denne \textit{Straadals} Fos med Sitt Landskab paa østere væstere, Sambt Syndere Siide, Liigesom hand og dend Nordre Siide forcklarer at være af Samme Stygge beskafenhed, Som de andere ere af 1{te} Viidne i \textit{Werdalen} ved 5{te} Spørsmaal under dets 5{te}\textit{N{o} in fine} beskreevne. Viidere i Nord Sagde hand iche at være beckiendt. ‒\par
Fra 6{te} til 12{te} Spørsmaale \textit{inclusive} Svarer det Samme Som næst forrige 3{de} Viidne, hvorpaa hand hiemlovedes.\par
Siiden \textit{Major Snitler} ej har faaet Saa tilstræckelig forcklaring om denne Ejgns grændsemærcker med deris landskabs omstændelige beskriivelse, Som hand havde ønsked; Saa tilSpurde hand Laugrettet: om de Vidste, at Navngiive ham nogen kyndigere Mand, til Viidne, at give ham nøjere underRættning om grændse-Skillet paa dendne Kandt? Laug Rættet Svarede Nej; De vidste ingen bedre at Nafngiive; efterdi de Mænd, her nu boe i \textit{Helgaadalen}, alle ere komne hiid op fra Needre-bøjden i \textit{Werdalen}. \textit{Majoren} der paa forkyndede for \textit{Helgaamend} det Samme, Som paa \textit{Suul}, efter Forrætningens Slutning, er indført, og \hypertarget{Schn1_27244}{}Bilag: Om Verdalen Prestegjeld. derpaa blev Rætten paa det Stæd Endet; af Laug Rættet tillige med underskreevet og for Seiglet. ‒\par
\centerline{(sign.) Peter Schnitler. (L. S.) Christopher Larsen Sneckermo (L. S.) Anders Olsen Kleppen. (L. S.)}
\DivII[Bilag: Om Verdal prestegjeld]{Bilag: Om Verdal prestegjeld}\label{Schn1_27260}\par
\centerline{\textbf{Bielager} ved \textbf{Werdalens} Præstegield.}\par
(1) Over dend field-bøjd nærmest grændsen ad \textit{Jemteland}, \textit{Suul}gaardene: \textit{Suul}gaardene, dend nærmeste bøjd ad \textit{Jemteland} 1 1/2 miil i væster fra lande-Mærcket, \textit{Storsiø}-Sundet, bestaar af 6 Smaa bønder Gaarder i \textit{Wuku Annex} Kircke, hvorhen de have imod 4: miile, i \textit{Werdalens} Præstegield\textit{Trondhiems} Stift. ‒\par
De ligge for Sig Selv til Fields i een Skoug dahl adskilte fra bøjde lauget, omtrendt 3 Miile østen for de nærmeste Steene gaarder, og dend Saa kaldete Norske \textit{Steene}-Skandze. De have deris Nafn af \textit{Suul}-Elven, Som kommer øst fra \textit{Jnd}-Søen, ved det Stoere \textit{Suul}- field, og Riinder tædt væsten forbie disse \textit{Suul}-gaarder. ‒\par
Paa disse gaarder Saaes gandske liidet Korn, og det fryser gemeenligen af. Her er goed Skoug, men Siiden Saugen for nogen tiid er need lagd, har bønderne jngen fortieniste deraf, uden at de der af giøre Smaa Kiørler og Træfang til bøjde-Mænd; deris mæste næring bestaa af deris Qvæg, jtem af at Skydze og huuse de Reisende, og ved \textit{Levanger} Marckets tiider at Sælge dem liidet Foder og kanskee øl, naar de det have. Disse gaarder ere lagd til Skiiløeber Lægd. ‒\par
J dette \textit{Werdalens} Præstegield, hvorunder disse \textit{Suul} gaarder\textit{Sortere}, merckes 3{de} Ting: (1) Dend alfare og beste Lande-vej fra- og til Norge til og udaf Sverrig, ved disse gaarder forbie dend beckiendte Field klimp, \textit{Kogsteen}, over \textit{Suul} Fieldet; Til dend Ende og de Svenske paa deris Siide have opbygged 3 field-Stue, eller boeliger, \textit{Stalkierstuen, Mitstuen} og \textit{Skarstuen}, 1: miil fra hindanden liggende fra øster i væster, hvor de Reisend have deris hviile og tilhold paa; Af hvilcke field-Stuer de Svenske i Sidste kriig Selv have opbrendt 2{de} for at giøre de Norske overgangen til Sig diss Vandskeligere. Paa diss StueGaarder Voxes nu indtet korn, og der for Nyde bønderne derpaa af de Svenske bøjder Korn-told \textit{in Natura} til deris \textit{soutien}; liigesom Med de Norske field gaarder paa Dofrefield imellem Sønden- og Nordenfields her i Norge holdes. ‒\par
Denne almindelige \textit{Passage} fra- og til Norge at decke her paa dend Norske Siide i \textit{Werdalens} Gield dend Saa kaldede \textit{Steene}-Skandze af Tømmer været anlagt tædt vesten for \textit{Suul} Elven, i væster fra \textit{Storsiø}-Sundet 1 1/2 Miile, ved denne alfare vej, og 1 1/2 miil fra denne Skandze i væster er dend anden \textit{Schaanes} Skandse af Jord voller ved Schaanæsfiord (der gaar i \textit{Trondhiems} fiord) j Skongens Præstegield; hin Steene Skandze er nu forfalden, ligger ogsaa til, at dend kand forbiegaaes, og af een \textit{Souperieur} Magt \textit{Couperes} fra \textit{Communication}; Som Saaes i Siste kriig, da de Svenske gick over \textit{Tromsdals}-Elven, og fat\hypertarget{Schn1_27470}{}Schnitlers Protokoller I. tede \textit{Posto}, imellem Skandzen og dend Norske \textit{Postering}; Denne Skognæs Skandze endskiøndt 1/8 Miil fra landevejen værendes, ligger dog Saa til at dend er ved Fiorden hvorhved dend kand have \textit{Communication} med \textit{Trondhiems} bye. Paa dend østlige Siide, en 5 miil vejs østen for Lande-Mærcket Stoersiø-Sundet have de Svenske ved Alfarvejen \textit{Due}-Skandse, af Tømmer, nu forfalden, og Nordost derfra \textit{Jerpe}-Skandze af Jordvoller med en Muuret Rund deel i. Denne Alfare vej gick og dend Svendske Finske \textit{Aarmee} under \textit{general Lieutenant Arnfelds Commando}, ‒ da den d: 12 Sept{r} 1718 i \textit{Trondhiems} Stift indfaldt.\par
(2) Denne Samme vej fare og de Svenske og Sæhrdeelis \textit{Jemterne} i Freeds tiider, en 800 til 1000 hæste Stærck, meere og miindere i \textit{Martij} Maaned om aaret igiennem \textit{Werdalens} Præste gield til det \textit{Fameuse}\textit{Levanger}-Marcket i Skongens Præstegield. ‒\par
(3) J \textit{Werdalens} Præstegield ved dets hoved- eller Saakaldede \textit{Stikkelsstads}-Kircke, ett par Bøsse Skud herfra i ost Sydost, Staar en Muret Steen-Støtte, omtrent halfanden Mand-høy, med ett Jern-kors paa en houg under Stikkelstads gaards Grund, med et Skrift paa en Fiirekandtet grøtt-Steen, Fæsted paa dend eene Siide af Stødten, der viiser at paa dette Steed er fordum \textit{Sanct Ola} ihiel Slaget. (hvilcket Skeede i ett FeltSlag, der holdtes A{o}1030 d: 29 Julj og er hans liig i \textit{Trondhiems} Domckircke Siiden bleeven begraven.)\hspace{1em}\par
(2) Om Nybyggerne eller \textit{Helgaadalen} Nærmest liggende til lande-Mercket \textit{Straadals} fossen 3 1/2 gammel Miil; Nybyggerne kaldes de, fordi de i afdøde Laugmand \textit{Peter Dreiers} Tiid, Som kand være en god Mands Alder herfra, først have bebygget denne \textit{Helgaadalen}; \textit{Helgaadalen} har Sit Nafn af \textit{Helgaaen}, Som Riinder igiennem denne Dahl og er dend Samme Som heeder i Almindelighed \textit{Werens-aae} ligger fra Sin \textit{Annex} kircke i \textit{Wuku} 3 miile i øster i \textit{Werdalens} Præstegield; Stræckker Sig fra øster i væster fra dends østere til de Sidste væstere gaarder imod 1/2 Miil lang, breed imellem fieldene fra Søer i Nord 1/8 Miil. ‒ bønder gaarder er der 6. Som nære Sig af Deris Qvæg og liidet Tømmer brug af Skougen, naar de kand have lejlighed at driive den til Saug brug; til korn er og Jorden frostnembt.\par
Vej fra \textit{Helgaadalen} til \textit{Jemteland} er ingen, Som bruges; thj Jemterne ere forbødne at fare anden vej, end forbie \textit{Due}-Skandze, hvor toldstæden er anlagt, og der alfare vej ved \textit{Suul} gaardene forbi gaar. Vejen derfor her fra til \textit{Jemteland} er meget Vanskelig for vande og Myhrer Skyld, og Skulle det om Sommern Skee, at nogen forbi \textit{Stradals} Fossen over- eller Søndenfor det \textit{Jemte}-field\textit{Manshougen} ville fare, maatte det Skee med nogen besvær og omckrogning. ‒\par
Disse Nybyggere med Suulgaardene giøre lægd ud til Skiiløber Soldattere. ‒
\DivI[I Inderøy fogderi: 14 vidner.]{I Inderøy fogderi: 14 vidner.}\label{Schn1_27673}
\DivII[Mai 13.-15. Fra Helgådalen til Ogndal]{Mai 13.-15. Fra Helgådalen til Ogndal}\label{Schn1_27674}\par
\centerline{\textbf{A{o}1742.}}\par
Dend 13de Maij mod Natten (Jngen anden tiid paa Dagen kunde fares, Som da Sneen noget Fryser, at bære Reenen) tiltrædede \textit{Major Snitler} Reisen; Dog Som vejen ud af Dalen gick temmelig brat op ad Fieldet, maatte man bruge Folche-magt 1/2 Miil vejs, at træde Sneen ned for Reenen, Som ellers ej havde kundet draget Kieriistterne med Perso\hypertarget{Schn1_27697}{}Retten sættes i Inderøen Fogderi. nerne i, did op; Dend Natt vandt man imod 2 Miile til Finnens \textit{Koye} i fieldet, hvor man dagen efter, dend 14{de} Maij indtil mod aftenen maatte fortøve; J denne Køy, Som var af Smalle Træestammer opsatt, becklædt med Vadmel uden paa, viid needen, og Spitz oven til med ett aabent Røg-huull midt øverst i; Dannet Som ett \textit{Musqveteer-Tælt}, dog noget Viidere needen under, kunde man ikke anderleedis rømmes, end at ligge ved Siiden omckring Jildsteenen, og det endnu langs efter, ikke imod Jildsteenen med Fødderne udstracht, thi ellers brændte man dem; J denne \textit{Finn-Køy} tilholdede 4 Madlauger eller \textit{Finne-Familier}, bestaaende af 12 Persohner, naar de ere Samblede; her fra om Aften Reiste viidere, og Komme om Natten til Nœste Grændze-bøid \textit{Ongdalen} af \textit{Sparboe} Præstegield; Dend 15{de} næst efter blef rætten i dette \textit{Ongdalen ordineret} til næste følgende Dag der efter. ‒\hspace{1em}
\DivII[Mai 16.-17. Rettsmøte i Ogndal]{Mai 16.-17. Rettsmøte i Ogndal}\label{Schn1_27740}\par
\textbf{1742: d: 16 maij} blev Retten i forbem{te}\textit{Ongdalen} i \textit{Scheiis Annex}\textit{Sparboe} Præstegield\textit{Jnderøens} Fogderie foretaget, hvor veed vare, Som LaugRættes-Mænd, overværende \textit{Baard Tomesen Lusstad}, og \textit{Lars Olsen Gulstad}; Som Viidner mødte \textit{Tarrald Olsen Schieldstad} og \textit{Erich Jensen Jedsaassen} Sambt \textit{Lap:Finnen Lars Larssen}; j hvis paahøer dend kongel: \textit{order} til \textit{Major Schnitler}, Sambt Eedens forcklaring for Viidnerne af Laugbogen blev oplæst; Derpaa og Viidnerne aflagde deris \textit{Corporlig} Eed; af de Kongel: \textit{Civiile} betiendtere var ingen tilstæde, Saasom man hørte, at de paa denne Aarsens tiid, fra bøjdene ej kunde komme op til denne field-bøjd. Rætten foretoeg da under \textit{examen}\par
1{te} Viidne i \textit{Ongdalen}\textit{Sparboe} Præstegield\textit{Jnderøens} Fogderie ‒\par
heeder \textit{Tarrald Olsen Schieldstad}, føed i \textit{Sneaasens} hoved-Sogn, af bønder forældre, er 64: aar gammel, gift, har 5 børn, boer og er bonde paa \textit{Schieldstad} her i \textit{Ongdalen}. ‒\par
1 Spørsmaal ‒\par
hvor langt ligger denne bøjd fra de nærmeste grændser ad \textit{Jemteland} og i hvad Præstegield og Fogderie?\par
\textit{Resp:} Fra de østerste gaarder her i \textit{Ongdalen} til nærmeste grændse-Mærcke i øster ad \textit{Jemteland} er omtrendt 7: field Miile, hvilcket grænse-Mærcke er \textit{HøySætte} een fieldklimp, og Reigner viidnet de 7 field Miile, efter hands formeening, at være 5 1/2 Nye Maalte Miile. Dette \textit{ongdalen} ligger i \textit{Scheiis Annex}\textit{Sparboe} Præstegield\textit{Jnderøens} fogderie; og denne Mercke-bøjd for Sig alleene, bestaar af 8{te} Smaa bønder-gaarder, de nære Sig mest af deris \textit{Creaturer}, og Som jorden her er meget frost-nembt, at de Sielden faar Korn høsted, Saa de maa behielpe Sig med brød af Furrubarck; liidet Fiskerie kand de have, dog kun af Elvene og bæckene; Tilforne har de haft nogen fortieniste af Skougen, mens Som dend er udhuggen, Saa \textit{Cesserer} det og ‒\par
2 Sp: Hvad er Landets beskaffenhed imellem bemelte grændse-Mærcke og denne Marcke-bøjd?\par
\textit{Resp:} Nærmest i øster fra \textit{Ongdalen} er Skoug- og Myr- Samt Maase-land og berg, hvor Slett Knastret og misvoxen Skoug er, Som indtet andet duer til, end till brænde-fang, 1: miil, Siiden er der nogle Schnaug, det er Skallede Field \textit{Wohler}, 1: miil; der paa farer man over een Elv, Som kommer af \textit{Aas}-vandet, og Riinder i \textit{Schiecker}-vandet; Efter denne Elv begynder i øster \textit{Schiecker}-fieldet, fra dets væstre Fod til høyeste \textit{Schiecker}-Stødten 1: \hypertarget{Schn1_27935}{}Schnitlers Protokoller I. miil, hvor fra til dends østere foed kand være 1: miil, der fra Synis hand følger i øster field-\textit{vohler} og berge, omtrent 2 Miile, og Sist een Skoug-Dahl, Som er og kun Slett, til Lande Mærcket \textit{Høysætta}, 1: miil. ‒ Forbem{te}\textit{aas}-vand ligger her fra \textit{ongdalen} 2 Miile i øster, Stræckker Sig fra Nordost i Syd-væst imod 1/2 miil lang, og er omtrent 1/2 Fierding breed, hvor \textit{Sneaase}-Mænd fiske Smaa øret j; Af denne aas Søe, udriinder forberørte Elv, kaldes \textit{aas}-vats-Elven, og løeber fra Nord i Søer ett par bøsse-Skud lang i \textit{Schiecker}-vandet; Dette \textit{Schiecker}-vandet Strækker Sig fra ost-Nord-ost til væst-Syd-væst efter hands formeening 1: liiden Miil lang, og er 1/4 mill breed, hvor i øret Fanges af \textit{Sneaasingerne}; Om \textit{Schiecker}-Elven eller \textit{aaen}, Som Riinder af dette \textit{Schiecker}-vand, Saa og om \textit{Schieckeraadalen}, hvor igiennem denne \textit{Schiecker}-Elv Løeber, Siiger hand det Samme ud, Som 3{die} Viidne i \textit{Werdalen} til det 3{die} Spørsmaal; Dette \textit{Schiecker}-vand giiver viidnet dend Anmærckning om, at det \textit{Schiecker}, fordj i dets væstere Ende Skiær Sig af, eller Skiller Sig ad 3{de} Nembl: \textit{Werdalens- Sparboens- Sneaasens}- Præstegield. ‒ For berørte \textit{Schiecker}-field beskrii(ver) hand Saaleedis, at det Strækker Sig fra Nord-ost, 2 Miile væsten for \textit{Gaundals Wuddu} i Sydvæst hen til \textit{Halbachs}fieldet omtrendt 3 Miile langt, og er fra væster til øster, at reigne fra foed til foed 2 miile breed; Østen for dette \textit{Schiecker}-field, hvor der tilforn her er talt om field\textit{vohler} og berge, Saa Skal disse \textit{vohler} være at forstaa Saaledes, at de ere eendeel af \textit{Halbachs}-fieldet, hvor om det 3{die} Viidne i \textit{Werdalen}, ved det 3{die} Spørsmaal har forcklaret, at breeden af \textit{Halbachs}-fieldet paa det Stæd, hvor \textit{Straadals}-Fossen er, bliver fra denne \textit{Straadals}-Fossen 2 field-Miile i Nord til \textit{gaundals-Wuddu}, Saa at heraf fornemmes, at \textit{Halbachs} fieldet med Sine Deele og \textit{vohler} Stræckker Sig fra \textit{Straadals} Fossen tædt Øesten forbie \textit{Schiecker} fieldet hen til \textit{gaundals Wuddu}; Væsten for \textit{Schiecker}fieldet ligger \textit{Aas}-vandet, \textit{Schieckervandet} med \textit{Schiecker}-Elven, Sambt \textit{Schieckeraa} Skoug-Dahl. Foruden forbeskrevne Elve forcklarer hand endnu Een Elv, ved Nafn \textit{Gauna}, at være, hvilcken Elv opckommer af Nogle Kynner østen for \textit{Schiecker}-fieldet ved dets Nordre Ende, og Riinder fra væster i øster 3 Miile lang ind i een Søe i \textit{Jemteland}, kaldes \textit{Tørøyen;} Saa er og een anden Elv ved Nafn \textit{Ongna}, hvor af denne \textit{Ongdalens} bøjd har Sitt Nafn; Denne \textit{Ongna}-Elv kommer af almaas Søe, og Rinder fra dends udSpring imellem \textit{ongdals} gaardene, og Siiden igiennem dette \textit{Scheiis Annex} fra øster i væster 3 1/2 gamle-, Som kan være 3 Nye- Miile, ved \textit{SteenKier}-gaard Paa dennes Nordre Siide, i \textit{Steen-Kier}- eller \textit{Beedstads}-fiorden, der er Eett Med \textit{Trondhiems} Fiord; \textit{Almaas}-Søen, hvor af denne \textit{ongnaa}-Elv udRiinder, Stræcker Sig fra øster i væster, og i Sæhr med Sin væstere Ende ligger imellem \textit{Gulstad} og \textit{Lustad}, de første østlige \textit{ongdals} Gaarder, 1 miil lang, og er omtrent 1/8 miil breed; J hvilcken Almaas Søe af dette \textit{Annexes} bønder, øret tilfælles Fiskes. 1/2 miil Sønden for denne \textit{almaas} Søe ligger een Anden Søe, kaldes \textit{Maga}-Søen; denne \textit{Maga} Søe Stræckker Sig fra Søer i Nord, imod Midten af \textit{Almaas} Søen, paa 1/2 Miil Nær, Siiden giør Samme \textit{Maga} Søe en vænding fra Nord i Sydvæst, og er in alles 1/2 miil lang; Af denne Søes Nordre Ende udSpringer dend Elv, \textit{Maga}-Elv, og løeber først i Nord, Siiden i væster 3/4 Miil vejs, jnd i \textit{Ongna} Elv, 1/8 Miil vejs fra dendne \textit{ongnas} udfald af \textit{Almaas} Søen; Denne \textit{Maga}-Søe tilhører \textit{Mock}-gaard i \textit{Ongdalen} her, og fiskes Smaa øret der j. ‒ Her efter erindrede viidnet Sig nærmere at forcklare om \textit{Schiecker}-fieldet, at det er høyt, og har mange høye bergstødter \hypertarget{Schn1_28288}{}1 Vidne i Inderøen Fogderi. og Klimper oven paa Sig, er ellers bart og Skallet uden Skoug, græs og Maasse. Høe- Sletter eller Sæter-boeliger findes icke imellem Lande-Mærcket og dette \textit{Ongdalen}, uden \textit{gaundals}-Sæter, Som hører til \textit{Sneaasens} Præstegield, Saa det land er øede, u-bebygget og u-dyrckelig, og tienner icke til nogen, uden at \textit{Lap-Finnerne} til deels kand benytte Sig deraf om Vaaren. Ellers forcklarede viidnet, at dette \textit{ongdals Annex} ej Stræckker Sig længer i øster, end til dend væstere Ende af \textit{Schieckervandet}, østen der for have de intet; mens \textit{Sneaasens} Præstegield gaar fra Nord i Søer- og \textit{Werdalens} Præstegield fra Søer i Nord, og eje det østere Landskab Jmellem grændsemærcket og dette \textit{Schieckervandet}, hvor fra denne \textit{ongdalens} bøjd aldeelis er udelugt. ‒\par
3: Spør: Hvilcke ere de Nærmeste gaarder paa Denne Norske Siide, nærmist ved grændse-fieldene fra Søer at reigne? af hvad beskaffenhed er landet? og hvad Næring bønderne bruge? ‒\par
\textit{Resp:} De nærmeste gaarder ved grændse fieldene, er til dette \textit{ongdalen, Nyebyggerne} i \textit{Helgaadalen}, omtrent 3 miile her fra i Søer; Disse Nybyggere deris Næring har hand hørt at være, liigesom 1{te} Viidne i \textit{Werdalen} til 3{de} Sp: dend forcklaret har. Landskabet imellem Nybyggerne og dette \textit{ongdalen} beskriiver hand, liigesom 3{de} Viidne i \textit{Werdalen} ved 3{de} Spørsmaal, undtagen at her i \textit{ongdals} Marcke-bøjd ere vel 8{te} Smaa gaarder, mens 5 der af ere øede; fremdeelis her fra \textit{ongdalen} til Lande-Mærcket \textit{gaundalsvuddu} er 7 field eller 5 1/2 Ny- Miile, og at i \textit{Schiecker} Søen alleene af \textit{Sneaasingerne} fiskes. Disse 8 \textit{Ongdals} gaarder, hvor af kun igien er 3 beboede, ernærer Sig Kummerlig, Som ved 1{te} Spørsmaal er forcklaret. ‒ Denne Marckebøjd er fra bøjde-laget i \textit{Scheiis Annex}, hvor under det \textit{Sorterer}, i \textit{Sparboe}-Præstegield 1 1/2 Nye Miile adskildt. Fra østerste gaard \textit{Maack} i væster Need til Sidste af disse 8{te} Marckegaarder, Nafnl: Rok(s)tad er 1 Miil, og fra bem{te}\textit{Maack} til Sidste gaard i dette \textit{annex}, Nembl: \textit{Steenkier} ved \textit{Steenkier}-fiorden er omtrendt 3 1/2 miil; Breed er denne Marckebøjd fra Sønder i Nord, at Reigne fra \textit{Maack} til \textit{Schielstad} 1/2 Miil. ‒ Landskabet paa dend væstere Siide af denne Marcke-bøjd er ufrugtbar, bestaaende Af Kaald Myhr land, og Tørre berg houger. ‒\par
De nærmeste gaarder paa dend Nordre Siide her fra ere de af \textit{Sneaassens} Sogn 3 Stiive miile i Nord ost. Landskabet herimellem er Een Myhret Skougdal, af Gran og Furru, fra Søer i Nord ost 2 1/2 Miil, hvor af \textit{Ongdals} bøjden tilhører een Stræckkning af 1: miil til en Dahl kaldes \textit{Rochdalen}, hvor fra \textit{Sneaasingernis district} begyndes: Dog er denne Skoug u-Nytteligt; ‒ thi Som dend Staar i een vaad kald Myr, Saa er der bare vandhveed i Træerne, og disse faar icke deris fulde voxter; vel have de \textit{Sneaasinger} fra \textit{Rochdalen} af i Nord, nogle høeSletter og Sæterboeliger j denne deris Skoug-Dahl, men \textit{ongdalinggerne} kand j deris andeel iche have nogen lejlighed til Saadant, for jordens ufrugtbarheds Skyld; breeden af denne Skoug Dahl fra væster i øster kand være 1/2 miil, viidere i Nord var hand icke bekiendt om Landskabet. ‒\par
4 Sp: Hvilke ere de nærmeste gaarden paa dend østlige Svenske Siide, nærmist til Grændse-fieldene, fra Sønden at Reigne? og af hvad beskaffenhed er Landet? og hvad Næring bønderne bruge?\par
\textit{Resp:} Østen for det lande Mærcke \textit{Straadals} Foss, Som hand meener, kiender hand \hypertarget{Schn1_28501}{}Schnitlers protokoller I. dend Søe \textit{Kald} Søen, dend Stræckker Sig fra Nord-væst i Sydost, 5 miile lang, og er 1 miil breed over; Deraf udkommer en Elfv ved Nafn \textit{Jerpa}-Elv, Som Riinder i Nord ost een liiden Miil i dend Søe \textit{Liit;} der, hvor dend falder i Søen, Staar paa den Nordre Siide dend Svenske \textit{Jerpe}-Skandse, af Jordvoller opbygget med en Mured høy Rund-deel inden for; hvilcken Skandze hand meener, holdes ved liige endnu, og at der ere Folck inde. Ved \textit{Kald} Søen erindrer hand Sig, at i Kriigens tiid har ved dends Nordvæstlige Ende, hvor dend Elv \textit{Kaldstrømmen}, fra \textit{Tørøyen}-Søe udkommendes, i \textit{Kald}Søen falder, Een Skandse af Tømmer bygget, Staaet jmod \textit{Sneaasens} grændser, omtrendt 6 nye Miile i Syd ost fra \textit{gaundals} Sæterboeliger i \textit{Sneaasens} gield, hvilcken Skandze blev kaldet \textit{Kaldstrøms} Skandze, og nu er forfalden. Paa dend Nordre Siide af denne \textit{Kald} Søe ligger \textit{Kalds}bøjden langs efter Søen, hvor fra et lidet Støcke og \textit{Kalds Annex}-kiercke Staar; Om disse Folckes deris Næring og brug viste viidnet ingen beskeed. 1 Miil i Nord væst fra \textit{Kald} Strømmen ligger En Enckelt gaard \textit{Kuldaasen}, hvis lejlighed hand og iche viste; Om det Store field i \textit{Jemteland} i Nordost fra \textit{Straadals} Fossen, kaldet \textit{Manshougen} udsiiger hand det Samme Som 3{die} Tiidne i \textit{Werdalen} ved 4{de} Sp:, med det tilleg, at dets østere Siide af \textit{Jemterne} kaldes \textit{Dørsvalen}; Dette \textit{Mandshougen} Stræcker Sig i Nord hend i mod \textit{Tørøyen} Søe; Denne \textit{Tørøyen} Stræckker Sig fra Nordvæst i Sydost 5 Miile lang, i Sin breede er dend bugtet, Sommestæds breed og SommeStædz Smal, og kand være 1 Miil over, hvor den er breedest; Fra denne Søe ved dends Sydostlige Ende udRiinder forberørte Elv \textit{Kaldstrømmen} imod 1 miil lang ind i \textit{Kald} Søen. ‒\par
5: Spørsmaal: hvilcket er det første field eller Stæd, Som hand kiender, at være ett lande Mærcke imellem Norge og Sverrig paa denne kandt, Sønden fra at reigne?\par
\textit{Resp:} Det første field, hand Kiender, er \textit{Høysætta}, over dets øverste Kold eller Top lande-Mærcket gaar; dette field er Rund og Spitz oven til, bart og Skallet uden græs og Maasse, er icke Større af Sig, end Som et godt bøsse Skud over; fra det nærmeste GrændseStæd i Søer Nembl: \textit{Straadals}-Fossen, veed hand icke, hvor langt det kand være; men dog har hand paa dend Søndere Siide af dette \textit{Høysætta}, indtet andet Seed end field; Paa dend væstere Siid af \textit{Høysætta} ligger \textit{Buur}-vandet, Rund af dannelse, et par bøsse Skud Stoert, hvor i \textit{Sneaasingerne} fiske øret; ‒ Væsten for dette \textit{Buur}vand ligger endeel af det før beskrevne \textit{Halbachs} field; Dend østere Siide af \textit{Høysætta} bestaar af Fielde, og i Nordost derfra ligger dend Søe \textit{Tørøyen}; Paa dend Nordre Siide følger Strax \textit{Gaundalsvuddu}, Som er en SkougMarck, 1/2 miil over fra Søer i Nord, dends længde fra væster i øster vidste hand icke; og dette er det 2{det} lande-Mærcke, viidnet kiender, hvor midt igiennem grændseStræckningen gaar; vel Staar ingen Mærcke tegn enten paa \textit{Høysætta}-Klimpen eller i denne \textit{goundals Vuddun:} mens dog er \textit{Høysætta} baade af de Norske og \textit{Jemterne} holden for ett Grændse-Skiel, og naar man Staar oven paa \textit{Høysætta}-Kollen, og Seer liige i Nord-ost over til \textit{Haaldes}-hatten, Som liigeleedes er ett beckiendt lande-Mærcke, Saa kan man See, at grændse-Skiellet gaar midt igiennem denne \textit{gaun}dahls \textit{vuddun}, først til \textit{Haaldes Holmen} og Siiden til \textit{Holdes-Hatten}; østen for \textit{gaundalsvuddu} ligger dend Søe \textit{Tørøyen}; dog førend man kommer til \textit{Tørøyen}, ligger ett liidet græsland med Skoug bevoxen, hvor \textit{Creaturer} kan have nogen beete i, kaldes øster-\textit{gaundalen}; Paa dend væstere Siide af denne \textit{gaundalsvuddu}\hypertarget{Schn1_28768}{}1 Vidne i Inderøen Fogderi. ligger een Sæterboelig Ved Nafn Væster-\textit{gaundal}, Som hører Præsten paa \textit{Sneeaasen} til, 1/4 Miil vejs fra grændse Skiellet. Paa dend Nordre Siide af \textit{gaundalsvudu}, hvor denne ophører, møeder ett bart Skallet field, omtrent til 1: Miil over til \textit{Holdes} Søen; Nafn paa dette field vidste viidnet iche, men grændse-Skillet gaar over dends østere Ende, i Nordost ad \textit{Haaldes} Søen; Længden af dette field viste hand icke, dog har hand Seed, at dend væstere Ende er høyere end dend østere, og gaar i Nord-Nord-væst ad \textit{Sneeaasens} bøjd; ‒ Nord ost for dette field følger \textit{Haaldes} Søen, Som Stræcker Sig fra Nordvæst i Sydost 1 Miil lang, og er 1/2 miil breed over, hvor i Fanges øret, Røe og Lache, Nembl: af \textit{Sneeaasingerne} paa dend Nord-væstlige Siide 1/2 miil, og af \textit{Jemterne} paa dend Sydøstlige Siide en, 1/2 Miil af Søen; Midt i denne \textit{Haaldes} Søe, ligger een \textit{Holm}, kaldes \textit{Haaldes-Holmen} eller \textit{MidtHolmen} fordj den har 1/2 miil paa dend Sydøstlige- og 1/2 miil paa den Nordvæstlige- Siide af Søen, Denne Holm er Run, og Flat oven paa, med Smaa Gran-Træer bevoxen, omtrendt ett Bøsse-Skud Stoer, og denne \textit{Haaldes Holmen} eller \textit{Midt Holmen} er det 3{die} LandeMærcke imellem \textit{Sneeaasens} Præstegield i Norge og \textit{Jemteland}, det hand og har hørt af \textit{Jemterne} tilstaaes. Paa denne Søe i Nord følger en Gran-Skoug 1/2 Miil lang i Nord-ost til dend berg klimp, Som hand kiender at være 4{de} Lande-Mærcke, af Nafn \textit{Haaldes-Hatten}, som er Rund, høy og Spitz oven til, bar Skallet uden Skoug græs og Maasse; Er icke Større, end at man kan Skyde med en \textit{Mu[s]qvet;} Mærcke findes der ickke, opSatt, mens dog er det ett beckiendt grændse-Skiel, som og af \textit{Jemterne} altiid har værit tilstaaed; Dend østlige Siide af forbeskreevne \textit{Haalden} Søe og \textit{Haaldes-Hatten}, er følgende; udaf \textit{Haalden} Søes østere Ende gaar en Striid Elv eller Foss, kaldes \textit{Kiildberg} Fossen et par bøsse Skud i Sydost ind i \textit{Tørøyen} Søe i \textit{Jemteland;} J Samme \textit{Haalden} Søes Nordostlige Siide, Riinder en Stor Elv fra \textit{Giævf}-Søen, ved Nafn \textit{giævfra}, Som fra bem{te}\textit{giævf}-Søe løeber fra Nordost i Syd-væst 1/2 miil lang ind i \textit{Haalden}-Søe; Midt imellem disse 2{de} Elver, Nemlig \textit{Kiildberg}-Fossen og \textit{Giævfra}, er bare Skallede fielde i øster; østen for \textit{Haaldes Hatten}, ligger for bemelte \textit{Giævfra}-Elv, og østen der for igien de Skallede fielde; Væsten for disse \textit{Haalden} Søe og \textit{Haaldes Hatten}, ligger en Dahl med adskillige fisk-vande i med lidet Gran Skoug imellem, Som tilhører \textit{Sneeaasingerne}, dog løber vandet her udaf i øster ind i \textit{Tørøyen}; Paa dend Syndere Siide ligger det før omtalte Skallede field; Paa dend Nordre Siide ligger og Dahl med vande i; Paa dend Nord-ostlige Siide af \textit{Haaldes Hatten} ligger eendeel berg- \textit{vohler} med Slet biercke- og gran-Skoug paa 1/2 Miil lang til \textit{giævf} Søen; Denne \textit{Giævf} Søe er Rund af Sig, 1: Miil lang og breed, hvor i fiskes, Røe, øret, Harr, og lache, af \textit{Sneaasingerne} paa dend Nordre og Væstere- og af \textit{Jemterne} paa dend østere- og Syndere- Siide; Midt igiennem denne Søe i Nord-ost gaar Grændse-gangen til \textit{giævf}-Søe \textit{Hatten}, en høy, Rund, og Spitz Sambt Skallet Fieldklimp, tædt ved \textit{Giævf}Søen liggendes i Nord-ost, Som hand meener Fra Foed til Foed at være 1 Miil over, og dette er det 5{te} Lande-Mærcke viidnet kiender, over hvis høyeste Spidz grændse-gangen gaar; Paa dend østere og viidere paa dend Nordre Siide er hand ubeckiendt; paa dend væstere Siide ere toe dale med Fiske vande i, hvorimellem en liiden berg\textit{vole} ligger. ‒\par
6{te} Om disse grændse Stæder ligger under visse gaarder, eller ere kongens Alminding? \hypertarget{Schn1_29005}{}Schnitlers protokoller I. Svarer. \textit{Høysætta} veed hand intet andet, end det hører Kongen til; de øfrige af Skoug og Fiskevande bruges af \textit{Sneaasingerne}. ‒\par
7: Spør: Om der har været nogen tvistighed imellem de Norske og Svenske undersaattere om disse grændse Stæder? \textit{Resp:} Nej, ickke det hand veed. ‒\par
8{te} Hvad Nytte, godhed og herlighed er der ved disse grændse-Stæder?\par
\textit{Resp:} j \textit{gaun}dalen er Sæterboelig, og i Søene Fiskerie. ‒\par
9: hvor langt ligge disse grændse Stæder fra bøjdene og lande vejene?\par
\textit{Resp: Høyesætta} og \textit{gaun}dals-\textit{vuddu} ligge vel liige i øster mod \textit{ogndalen}, men de Saavel Som de øfrige lande-Mærcker, have dog nærmere vej til \textit{Sneaase}-bøjden og landevejen, hvor langt egentl:? kand hand icke Siige. ‒\par
10: Hvor underholdning for Folck, og beete for hæstene ved disse grændse-Stæder bliver at faa?\par
\textit{Resp:} underholdning for Folck veed hand icke nærmere at faa, end paa \textit{Sneaasen}, om der Skulde noget; beete for hæstene vil der Søges i field dalene hits og her. ‒\par
11: Spørsmaal: Hvad Mænd hand veed at være beqvæmmest og kyndigst, til at veiviise grændse-Maalerne i Sommer? Svarer: her i bøiden er ingen, Siiden de icke naaer Saa langt i øster: mens \textit{Werdalingerne} og \textit{Sneaasingerne}, Som eje og indehave dend østre Stræckning ald til grændse-fieldene, maa bæst kunde viide det. ‒\par
12: Spør(s)maal: hvor mange \textit{Lap-Finner} i disse fielde tilholde? Svarer: nu omstunder er der kun En \textit{Familie}; hvor paa Viidnet blefv bortforlofvet; og Siiden det var Saa Siilde paa aftenen, Rætten opsatt til i Morgen. ‒\hspace{1em}\par
1742 dend 17: Maij begyndte rætten igien; med\par
2{det} Viidne i \textit{Sparboe} gield\textit{Jnderøens} Fogderie. ‒\par
heeder \textit{Erich Jensen giedsaasen}, Føed i \textit{Hammerdal} i \textit{Jemteland} af bønder-forældre, 72 aar gammel, gift, har 2 børn, er bonde paa \textit{giedsaasen} i \textit{Aangdals} bøjd\textit{Scheiis Annex}\textit{Sparboe}-gield\textit{Jnderøens} Fogderie. ‒\par
til 1{te} 2{det} 3{die} Spørsmaal Siiger det Samme Som Næst l{te} Viidne, tilleggendes ved 3: Sp:, at dend 1 1/2 Miils Stræckkning af Skoug-dahl til \textit{Sneaasen} er berg \textit{vohler} med Myhr Slot og Sæterboeliger, som \textit{Sneeaasingerne} der i bruge. Ved 4{de} Sp: Siiger det Samme Som næst forrige Viidne, forcklarendes derhos, at \textit{Kald} Søen er af breed og længde, som næste 1{te} Viidne har beskreeven, og at af denne Søes Søndere Ende udRiinder \textit{Jerpa} Elv, jtem at \textit{Kalds}bøjdens gaarder, ligge baade paa dend Nordre og østere Siide af \textit{Kald}-Søen, og at de ere med de andere Enckelte fraliggende Gaarder 35 i Tallet, hvor af dette \textit{Kalds Annex} bestaar. Om dend Søe \textit{Ahnin} og \textit{Ahnin}-Elv, Siiger hand Som 3{die} Viidne i \textit{Werdalen:} Dog at Elven, naar dend riinder af \textit{Kald} Søen, har det Nafn \textit{Jerpa}, og naar dend flyder af \textit{Liit}-Søen, gaar dend Synden for \textit{offerdals} Præstegield ind i \textit{Stoersiøn}. Fremdeelis udsiiger hand, som 3{die} Viidne i \textit{Werdalen} om \textit{Mandshougens}-field, og de østlige gaarder \textit{Sundet, graassiø}, og \textit{Kuldaasen}; tilleggendes, at Nord-ost fra \textit{Kuldaasen} 1: liiden field-Miil, ved Elven \textit{Kaldstrøms} Nordre Deel, Der imod, hvor dend falder af \textit{Tørøyens} Søe, ligger en liiden Gaard \textit{Egn}; hvis og de øfrige \textit{Kalds} gaarders Næring, hand beskriive, Som bemelte \hypertarget{Schn1_29294}{}Bilag A: Om Lapp-finnerne. 3{die} Viidne; Nord for denne \textit{Eng} gaard boer ingen bonde; Thi der er bare viilde fielde, hvor \textit{Lap-Fiinnene} tilholde Stræckkende Sig imod \textit{Sneaasens Findlier}. Ved 5{te} Sp: kommer hand over ett med næst 1{te} Viidne, forcklarendes derhos, at \textit{Buur}-vandet ligger noget i Syd-væst fra \textit{Høysætta}. \textit{Gaundals} Skougen fra væster, (at reigne fra fieldet, Som ligger ad \textit{Sneaasen} hen) i øster til \textit{Tørøyen} i \textit{Jemteland}, kand være 1 1/2 Nye Miile lang. Om \textit{Haalden}-Søe, og \textit{Midt Holmen} der i, jtem om \textit{Haaldes Hatten} har hand vel hørt det, Som 1{te} Viidne udsagt, men ej værit der Selv. Om østere og Væstere Sambt Nordre Landskab af \textit{Haalden}-Søe og \textit{Haaldes Hatten}, jtem om \textit{Giævf}Søens og \textit{Giævf}Søe-\textit{Hattens} beskaffenhed vidste hand indtet: dog hørt, at \textit{Giævf}Søe-\textit{Hatten} er Grændse-Skiel, og At \textit{Sneaasingerne} og \textit{Jemterne} have ligget, og Fisket Sammen i \textit{giæv}-Søen. Ved 9{de} Sp: udSiiger hand, at fra \textit{gaundalsvuddu} til \textit{Sneaase}-bøjd kand være 3: maalte Miile, og fra \textit{Høysætta} 1: Miil længere. J de øfrige Spørsmaale indtil Enden, kommer hand i alt over ett med nest første Viidne, og blev Saa bortladet. ‒\hspace{1em}\par
3{de} Viidne i \textit{Sparboe}-gield\textit{Jnderøens} Fogderie, Som efter Sogne-Præstens Foranstaltning var bleven tilsagt at møde, heeder Lars Larsen en Lapfinn er Fød paa \textit{Hammerdals} field i \textit{Jemteland}, af \textit{Finne}-forældre, noget over 40 aar gammel, gift, har 6 børn, har haft over 20 aar Sit Tilhold i \textit{Sneaasens}-fielde, og nu tilholder i \textit{Sparboes} fielde i Norge, Er Døbt i \textit{Hammerdals} Kierche, har gaaet Sidst til guds bord i \textit{Sneaasens} Kierche siistleeden høst, Siiger at viide, hvad Eed betyder, (og Syntes for Rætten at være en Skickchelig Fornuftig Mand, der og talte godt Norsk).\par
Af grændse Mærckerne Kiendte hand at være 1{st}\textit{Høysætta}, 2: \textit{Goundalsvuddu}, 3{de}\textit{Midtholmen} i \textit{Haalden} Søe og 4{de}\textit{Haaldes Hatten}, Paa hvilcke 4{re} Stæder hand Selv har værit; viidere i Nord hand ej var beckiendt og derpaa blev \textit{dimitteredt}. ‒\par
\textit{Majoren} betydede til Slutning, for \textit{ongdalingerne}, det Samme Som paa \textit{Suul} og i \textit{Helgaadalen} Skeede, og Sist i Forrætningerne er indført; hvor da og Rætten paa dette Stæd blefv Sluttet; Sambt af Laugrættet underskreeved og Forseiglet. ‒\hspace{1em}\par
\centerline{Peter Schnitler. (L. S.) Baard Tomesen Lustad (L. S.) Lars olsen Gulstad (L. S.)}\hspace{1em}\par
Efter 4{de} viidne Johan Schæilegreen, havde \textit{majoren} bud, hvilcken bonde boede her fra 1 1/2 miil her i \textit{Annexet:} men hand mødte icke, og berettedes af folckene, at hand ej var hiemme. ‒
\DivII[Bilag: Om Sparbu og Ogndal]{Bilag: Om Sparbu og Ogndal}\label{Schn1_29557}\par
\centerline{Følger Bilagene ved \textit{Sparboes} gield\textit{Jnderøens} Fogderie.}\par
\textit{Lit: A:} Tillæg til \textit{bilage} af \textit{Lit: C:} ved Selboe Fogderie ang: Lap Finnerne.\par
Efter Som ieg fra Tydalen af har omgaaes, og Reiset med Lap finnerne en 20: Miile field leedz til \textit{Sneaasen}, Saa maattet eendog ligget i deris field \textit{Køye} hos dem, Saa føjer \hypertarget{Schn1_29599}{}Schnitlers Protokoller I. nu til, hvad ieg derom viidere har i agt taget: Lapfinnerne finder man gerne at være hiulbeenede, at fødderne paa dennem udstaar; Aarsagen hertil har nu i Køyen erfahret, hvor ieg Saae ett af deris Spæde Finne-børn af liige Saa udckrommede Fødder, Nembl: (1) Naar et Fiinne barn er født, Liine de gamble det icke, Som hos os skeer med klæde, Skind, eller baand, men legge det paa et udhuulet fiæl, tilhugget efter ett barns Størelse, breed oven og Smalt needen til, paa hvis Siider huller eller Stropper ere, hvorved barnet paa fiælet tilSnøres, bare tildæckket med Reen-Kalfskind efter at de have puttet Reen-haar imellem dets Fødder, Saa at de gamble derved Selv ere Aaarsage til, at fødderne voxe udad, efterdj de ej biinde dennem Sammen (2) Saa Snart barnet bliver Saa Stoer, at det Sidder Selv hos de gamble paa Jorden, Fødderne overckors Slagne, Saa legges arme og hænderne gerne paa Knæerne, der at hviile, hvorved og Fødderne formeentlig vennes til at udviides (3) Jeg stiller derhen om det er een \textit{ratio Politica} hos Finnerne, at de med fliid vil have Føddern(e) Saa dannede; i det der Siiges at de hiulbeenede og Folck af Korte Fødder ere de \textit{adrai}teste til at Styre Skien, helst i hal- og Steile backer: J det minste holde de Saadan en kromhed ej for en Vanskabning. Man finder og Gemeenligen at Finnerne have Fladt bag hoved \textit{(Sinciput)} oven over Nackcken; Og fra hovedet udstaaende øre lipper; hvilcket første der af maa komme, at børnene ej med Kluder eller bløde huer, om deris hoved bebindes, men paa haarde fiæl, uden Pude under hovedet; bare med ett tyndt kalfskind paa, henlegges, det Sidste er vel der af, at huerne ej bindes børnene fremmen til over ørene, at Klemme dend til hovedet, men bagen derfor. ‒\par
Finnernes Modgang bestaar Sæhrdeelis derj (1) at ulven i disse aaringer, Saa Stærck har \textit{grasseret} med deris Dyr; og fortalte mig een Fiinn, Som Skydzede mig, at hand med Sit Sælskab Sin mæste tiid har opholdt Sig i \textit{Bedstads} Fieldene; men af ulvene Som i skokke-tall anckom, var bleven Saa berøvet, at hand om Aaret 20. 30. ja et Aar 50: Reendyr ved dennem havde mistet, og der for var bleven nødt til at flødte der fra, hiid længere i Søer; og dette u-dyer gjør, at mange ellers velholdene Finne-\textit{Familier} blive udarmede; om de endskiøndt Skyde een eller nogle der af bort, Saa ere der dog u-tallige igien. (2) Paa denne Aarsens tiid kalve Reenen gerne; naar nu u-veejr med kuld og Sneefog indtreffer, Saa Kalvene, Som nys ere faldene i de vilde fielde; Saa \textit{crevere} disse; Og denne Klagen af Fiinnerne hørte ieg Stærck, dend dag, ieg var i deris Køjer, og vii for ondt vejr ei kunde giive os ud i vejen, Saasom denne yngel, er deris eeniste Plou og Ager, hvor af de Skal leeve, og deris afgang er Fiinnens u-aar.\par
Dette Reendyrs eeget \textit{Naturel} erfoer ieg her og, da Finnen Som havde Skydset mig, vilde vende tilbage til Sit hiem; Thi der \textit{Finnen} Sloeg dyret, for at driive det til at trække Kieristen, hvorj hand Sad: vændte Dyret Sig imod Finnen, og Daskede finen med fremfoeden, Slag i Slag hasteligen, indtil Fiinnen fick dyret fadt i foeden og holdte det: dog var Samme Find dagen efter hos mig i \textit{ogndalen}, og havde ingen viidere Skade der af, end een øm krop, hvor hand var Daskedt; Saa \textit{Capricieus} et dyr er Reenen naar hand mod Sin villie eller over Evne bliver dreven. Dette til føiendis; At de Fiinner, jeg hidindtil har talt med, ønskede Sig gerne, at faae een eegen Øfrigheds Mand over Sig, som de kunde holde Sig til. ‒ og som man forestiller sig, at der dog vil Sættes grændse \textit{Jnspecteurer}, \hypertarget{Schn1_29647}{}Førefallet hindrer Reise til Finnliene. som \textit{Conserverer} de Sættende grændse-Mærcker, Saa kunde samme Mænd tillig have Jndseende med \textit{Finnernes politiske} Stand.\par
\centerline{Om \textit{Ongdalen}, en Marke bøjd ‒}\par
\textit{Ongdalen} har Sit Nafn af \textit{ongna} Elv, Som opckommer af \textit{Almaas} Siø, Saa kaldet, fordj ald meenige Mand, førend denne bøjd blev bygget, derj Fiskede; i denne Elv løeber \textit{Maga} Elfv af \textit{Maga} Siø, bærendes det Nafn deraf, at den er dannet Som en Fiske-mage; hvilcke alle i \textit{Protocollen} ere beskreevne. Denne Marckebøjd ligger nærmest til \textit{Jemtelands} grændse-Stæd, nembl: \textit{Høysætta}-field, 7: field miile der fra i væster, og fra nærmest Norske grændse bøjd \textit{Helgaadalen} 3 field Miile i Nord; og fra nærmist Norske \textit{Sneaasens} bøjd 3 Sttive miile i Syvæst. Dend har i øster \textit{Schicher}-fieldet med fleere \textit{vohler}, i Søer \textit{Halbachs}- fieldet, j Væster og Nord ond Myhret Skougdale om Sig; Den forsaavidt den fra \textit{Scheies Annex} ligger 1 1/2 Miil adskildt, bestaar af 8{te} Marcke-gaarder, lang fra østerste til væsterste gaard 1 Miil, og fra Søer i Nord 1/2 miil breed; Folckene har kun liidet at ernære Sig af, Nembl: af deris Kreatuer og liidet Fiskerie; Skougen er udhuggen til Saug brug; hvor fore de leeve kummerlig, og Siiden disse Frost-Aaringer er indfaldene, Saa er af de 8 gaarder, 5 øede, og de Som igien er bruge Furru brød, ja 2{de} der af, have i Mangel af Furru Meehl, maatte hacket halm, tørcke, og Malet det til brød. Dend \textit{Sorterer} under \textit{Sparboe} Præstegild\textit{Scheies Annex}, hvorj 16 Skiiløber Soldattere ere, foruden de \textit{Sparbøesche} Staaende \textit{Compagnier}. J dette \textit{Scheies Annex} er Mærckværdig \textit{Steenkier}, Som i gammel tiid, har været een betydelig kiøbstæd ved \textit{Steenkier}-Fiorden, Som er dend Samme med \textit{Trondhiems} Fiord, men nu \textit{Reduceret} til een bonde-gaard; Aarsagen hertil kan man forestille Sig, at ved denne \textit{Steenkier} grund er ingen havn, men vandet laugt, at ikke nu engang en Jægt kan lande diid (2) Vandet fra landet fryser om viintern 1/4: 1/2 ja 1. miil ud i Fiorden.\par
Vej imellem \textit{Ongdalen} og \textit{Jemteland} er jngen, og mindes ikke Folck, at nogen har kommet over til hinanden paa denne kandt; Saasom herfra bøjden til næste grændse Mærcke er 7 Miile. og Siiden fieldet-Land; hvorfore \textit{Jemterne} Naar de vil over til Norge, Fare enten igiennem \textit{Gaunsdals vuddu} til \textit{Sneaasen} eller alfare Vejen over \textit{Suul} fieldet forbj \textit{Kochsteen} og \textit{Suul}byggerne ind ad \textit{Werdalen}. ‒\hspace{1em}
\DivII[Mai 18.-19. Fra Ogndal til Snåsa]{Mai 18.-19. Fra Ogndal til Snåsa}\label{Schn1_29820}\par
\textbf{A{o} 1742{ve}} Dend \textbf{18{de} Maij}: Reist i \textit{Finnernes Kierester} fra \textit{Ongdal} til \textit{Aas} En af de Sønderste gaarder paa \textit{Sneaasen} 3 miile, hvor man, førend man kunde komme til dend bøjd, maatte faae hæste, at riide over \textit{Jmsa}-Elven, der løb baade dyb og striid; der ankom d: 19{de} og Sætte Reisen jgiennem ondt Myret Landskab fort til Tætteste bøjd i \textit{Sneaasen}, til \textit{Sverve} gaard ved kircken; Dend 10 nest efter \textit{Confereret} med Præsten hr. \textit{Peter Muus} og \textit{Missionairen} hr. \textit{Povel Muus} om Reisen til \textit{Findlierne}, Som ligge 12 field Miil fra \textit{Sneasen} i øster til \textit{Jemtelands} grændser; af hvilcke Saavelsom af bøjde-Lensmanden fick høre, at Elven \textit{Dalaaen}, hvor over man skulle, løb gandske aaben, desuden at Sneen nu om Dagen var Saa blød og Vandigt, at dend umueligt bar Reenen Saa langt et Støkke af 12 miile; Tilmed \hypertarget{Schn1_29903}{}Schnitlers Protokoller I. at Søerne mellem \textit{Sneaasen} og \textit{Finljerne} vare opgangne, eller nu Snarest kunde opgaae, at ingen kunde komme frem eller tilbage, forestillendes, at det var lettere for bøjden, ved et par Mand paa Skiie at giøre bud efter de ælste og kyndigste bønder i \textit{Finljerne}, at komme herneed til \textit{Sneaasen} paa Skiie, her for Rætten at aflegge deris viidne om Grændse Gangen, hvilcke paa Skiie kunde Rende omckring vandene til Fieldz, og giøre Sig Flotter over Elvene; Dette blev da \textit{Resolveret} og gienge \textit{Expressene} i Vejen d: 20 Maij om Aftenen.\par
Jmidler tiid igiennem gick man her paa \textit{Sverve} gaard de \textit{Communicerede Commissions documenter;} hvor iblandt af \textit{General Lieut: Wibes} grændse \textit{Commissions act} af A{o} 1690: \textit{pag:} 69: fandtes, at de Svenske grændse-\textit{Commissairer} af dend tiid ved Skriivelse til dend Kongel: Norske grændse \textit{Commission} af 9 \textit{augustj} Samme Aar have \textit{Reserveret} Sig deris Konges Rætt og \textit{prætension} til Søndere- og eendeel af Nordre-\textit{lje}-Sogner; hvilcke de Sagde at ligge paa dend Svenske Siide jnden Grændsen i \textit{Jemteland}, og at Dronning \textit{Cristinæ} Skriivelse her om til Kong \textit{Friderich III:} højloflig jhukommelse: af 28 maj 1649 skal være afgangen; fremdelis at de Svenske \textit{Commissairer}, Som Samme aar vare afferdigede til at holde grændseMøde paa Norske Siden med de Kongel: Danske \textit{Commissarier} angaaende \textit{Jdre} og \textit{Zerne} Sogner i \textit{Herjedalen}, have haft ordre om denne Sag at tale; Men efterdj de Danske \textit{Commissarier} ej Skal have mødt ved fore satte \textit{termin}, Saa er Sagen om Søndere- og Nordere- \textit{Lje} bleven beroendes \textit{in Suspenso}: Thj Skreev \textit{Major Snitler} til \textit{Sneaasens Pastor}, hr. \textit{Peter Muus} følgende: (S: T:)\hspace{1em}\par
Jeg har Sæhrdeelis aarsag til, paa dend Kongl: anordnede Grændse-\textit{Commissions} vegne, at begiere af Hr. \textit{Pastor} at efterSee, om ikke i Kiercke-Stoelen eller Kierckens \textit{Archiv} findes nogle gamle beviiser og \textit{documenter}: naar de 2{de}\textit{Annex} bøjder \textit{Søndere}- og \textit{Nordre- findlje} først ere bleven bebygde? om de med deris itzige Stræckning og Grændse-Maal fra første \textit{arrilds} tiid have ligget, Som de nu ligge, under \textit{Sneaasens} Præste-gield\textit{in Eclesiasticis}, og \textit{Jnderøens} Fogderie og Sorenskriiverj \textit{in Cameralibus} og \textit{Judicialibus}, og aldrig under \textit{Jemteland} enten noget dets \textit{pastorat} eller Tiinglaug have \textit{sorteret?} hvor af hvad beviiser haves, at mig \textit{videmeret} gienpart af Hr. \textit{Pastor} meddeeles, ved \textit{Commissionsacten} at bilegges; ert det, Som jeg tienstvenl: vil til bede mig. forblivendes. ‒ \textit{Titul}\hspace{1em}\par
\textit{Svevre} gaard paa \textit{Sneeaasen} d: 25 maj 1742. {\textit{P: Snitler.}}\hspace{1em}\par
Samme tiid blev og \textit{Expederet} breve til Sogne Præsten paa \textit{Sneaasen}, og Fogden af \textit{Jnderøen} at \textit{publicere} for almuen i grændse Stæderne fra prædicke Stoelen og paa Kierckebakkene, deris aller: pligt til grændse \textit{Commissionens} befordring.\hspace{1em}\par
\textbf{A{o} 1742{ve} d: 23{de} Maij,} anckom fra \textit{Finlje}-bøjdene om aftenen, Som Viidner Nembl: fra Søer-\textit{lje}\textit{Bendt Hansen øsnaar} og \textit{Joen Jonsen Skaale;} Fra Norder-\textit{lje}\textit{Lars Olsen} Skielbreeden og \textit{Hans Bendtsen Qvælien}, thj blev Rætten berammet til følgende Morgen for \textit{Sneaasens} hoved Sogn og \textit{Finlje annexer.}\hypertarget{Schn1_30226}{}Retten sættes i Snaasen.
\DivII[Mai 26.-30. Rettsmøte på Svarva i Snåsa (Herunder bilag: 104-105, 117-119.)]{Mai 26.-30. Rettsmøte på Svarva i Snåsa (Herunder bilag: 104-105, 117-119.)}\label{Schn1_30228}\par
Den 26 Maij blev Rætten Satt paa gaarden \textit{Sverve} i \textit{Sneaasens} Hoved Sogn, nærværende Lensmand \textit{Anders Erichsen Breede} og de 2{de} LaugRættes Mænd \textit{Lars} Friderichsen \textit{Midiaas}, 40 aar gammel og \textit{Ole Andfindsen Midiaas}, 38 aar gammel. Viidner af \textit{Sneaasens} hoved-Sogn vare tilstæde, \textit{Hans Olsen Kienstad}, og \textit{Gunder Pedersen Aasen, Ole Olsen Oune} og \textit{Ole Tollefsen Belboe} Sambt \textit{Andfind Jngbrigtsen Gran}.\par
\textit{Missionairen} Hr. \textit{Povel Muus} Efter bispens ordre mødte og; og gav tilckiende, at \textit{Lap: Fiinnerne}, Som Skulle kunde viidne om grændse-Strækningen Norden for \textit{Nordre-Finlje}, vare vel over 24: miile her fra, hvor indtet bud kunde paa denne aarsens tiid gaae hend, og jngen komme fra; Af Kongel: \textit{Civile} betiendtere indfandt ingen Sig, Som formoedentl: der af er, at Elvene og Søene ere opgaaet, og Lande-vejen u-\textit{passerl}: . J Samtl: deris overværelse blev dend Kongl: ordre til \textit{Major Schnitler Publiceret}, dernæst Eedens forcklaring af Lov bogen Viidnerne forelæst, Som derpaa aflagde deris \textit{Corporlig} Eed, til at Siige Sandhed, om hvis de viidste og havde hørt om Rætte grændsens-gang imellem Norge og Sverrig paa denne Kandt. Førend Viidnerne bleve tagne under \textit{Examination}, forcklarede \textit{Missionairen} hr. \textit{Povel Muus}, at \textit{Lap-Finnene} Norden for Nordre-\textit{Findlje} beqvæmmeligst kunde Møde j \textit{overHaldens Annexer Graang} og \textit{Haran} Norden her fra \textit{Sneaasen} beliggende, tilbydende Sig, at hand Samme Stædz gierne Selv ville møede, om hand dermed kand giøre dend Kongel: Grændze- \textit{Commission} nogen tieniste eller oplysning, og det Skulle forlanges; Dette tilføjendes, at \textit{Trondhiems} Ambtes \textit{district} Strækker Sig Norden for Nordre-\textit{Finlje} i Nord vel over 20 Miile, Som hand kan giætte, hend til Søndere Enden af det Stoere \textit{Børge}- Field\textit{inclusive} hvor hen hand Selv har gaaed, og haft 3 DagsReiser did; j dette \textit{Jntervall} imellem Nordre \textit{Lje} og \textit{Børge} fieldet er bare udørckener, bestaaende af Field, Myr, Skoug, og Vande, hvor ingen bonde boer, paa omtrent 6 miile i Væster, men bare \textit{Lap:Finner} tilholde. Rætten Svarede \textit{Missionairen} hr. \textit{Povel Muus}, paa hands giordte tilbud: det Skulle være dennem kiert, om hand vilde møde dem i \textit{Grong} eller \textit{Harangs Annexer}, at hielpe til at befordre dend allernaadigst ombefallede Grændse-\textit{Commissions} forrætning til deris Kongl: May{ts} Villies fuldbyrdelse. Hvortil hannem av \textit{Major Schnitler} tiiden skal blive \textit{Communiceret}, Til dend Ende, at hand kan lade \textit{Lap-Finnerne} jmellem Nordre-\textit{Findlje} og \textit{Børge}- fieldet, hvilcke fornemmes aleene om denne Grændze-Strækning at kunde give beskeeden,. tilsiige, at møede, hvor grændze-\textit{Commission} beqvæmmeligst kan finde at Sætte Rætten. ‒\par
Rætten overvejendes, at \textit{Finlje}-Viidner fra deris hiemstæder havde en 12 field Miile og paa denne Aarsens tiid Viinter-føeret letteligen kunde opSlaae, at de ej kunde komme paa Skiie tilbage uden med fare; \textit{Sneaasingerne} derimod havde her hiemme paa Stædet, Saa blev disse hiem-forlovede, og hine først tagne under forhøer; nembl: de fra Søndere \textit{Finlje} om Løverdagen d: 26. og de af \textit{Nordre Finlje} Mandagen nest efter d: 28. Maij: hvor efter \textit{Sneaasingerne} til \textit{Examen} blev foretagne d: 29 og 30 \textit{ejustem}: dog at den \textit{orden} i \textit{Protocollen} med Grændse gangens beSkrivelse fra Søer i Nord ej Skulle \textit{inverteres}, Saa bliver i \textit{Protocollen Sneaasingenes}, Som ligge Sønden for \textit{Finljerne}, og Følge efter \textit{touren} paa \textit{Ongdalens Annex} i \textit{Werdalens} gield, deris viidne først, og dernest \textit{Finljernes} der boe Norden for, deris \textit{Deposition} indført, hver Dag under Deris Rætte \textit{dato} og i overværelse af Samme LaugRættes Mænd, Som af det Nærmeste, nembl: \textit{Sneeaasens} bøjd vare tagne; \hypertarget{Schn1_30522}{}Schnitlers Protokoller I. j dets følge bliver A{o} 1742 d: 29. Maji ‒\par
4{de} Viidne af \textit{Sneaasens} gield og Sogn\textit{Jnderøens} Fogderie ‒ \textit{Hans Olsen Kinstad} ‒ Føed paa dend gaard \textit{Midiaas}, \textit{Sneaasens} Sogn af bønder forældre, 70 aar gammel, gift har 7: børn, boer og er bonde paa Gaarden kinstad i Snaasens Sogn. ‒\par
Til 1{te} Spørsmaal: Svarer: fra dend Synderste gaard paa \textit{Sneaasen, Hyllen}, i Sydost til nærmeste Grændse-Mærcke \textit{Gaundalsvuddu}, Som hand kiender, er 6 field Miile, hvilcke hand meente at være 5 Nye Miile; ‒\par
Dette \textit{Sneaasens} Sogn ligger j \textit{Jnderøens} fogderie; Det Stræcker Sig fra dend første Gaard \textit{Aanson} i væster til Sidste gaard \textit{Hammer} paa dend Nordre Siide af \textit{Sneaase}-vandet 2 1/2 Nye Miil, hvilcken Stærckeste bøjd ligger langs efter dend Elv \textit{Dahl-aaen} 1: Miil, og Siiden paa begge Siider af \textit{Sneaase}-Vandet til dend Sidste væsterste gaard paa dend Nordre Siide af Vandet, 1 1/2 nye miil; Dog ligge nogle Gaarder her af i Søer langs efter \textit{Emsa}- Elven; heele Sognet Skal bestaa af 130 bønder foruden husmænd; ‒\par
Heele Sognet efter Sin længde fra øster i væster nembl: fra \textit{Brænds} fieldet\textit{inclusive} til \textit{Hammer} Gaard er 5 1/2 Nye miile; efter Sin breedde fra Søer i Nord nembl: fra \textit{Rokdalen}, hvor ved de Skilles fra \textit{Ongdalen} J \textit{Sparboe} gield til \textit{Raug} Søen i en Skougdal, hvor ved de Skilles fra \textit{Overhaldens} Præstegield, er det 4 1/2 Nye Miile. ‒\par
Sognets Grændser i øster er til \textit{Finlje} bøjdene\textit{Brænds}fieldet, i Væster \textit{Sneaase}-vandet med fielde paa begge Siider, i Sønden en Skoug Dahl kaldes \textit{Rokdal} af Slet Knastret Gran- og Furru-Skoug, hvor i de have nogle høeslætter, og Sæterboeliger, J Nord er der en Kleen Skoug-Dahl, med bem{te}\textit{Raug}-Søe j. ‒\par
Landskabet i dette Sogn bestaar af deris jorde-brug, i Vande, berge, og Skouge: Fiske- Vande har de kun Eett Nembl: \textit{Sneaase}-vandet, hvorj Fanges øret og Røe; Det Stræcker Sig fra øster i Væster 3 1/2 nye Miil lang, og er breed 3/4 Nye Miil, mens paa begge Endene kun en liiden 1/4 Nye Miil. ‒\par
Fielde have de i Søer, Norden for \textit{Rokdalen}\par
1{te}\textit{Røbeen}-fieldet, ett bart Skallet field uden Græs og \textit{Maasse}, der Stræcker Sig fra Sydost i Nordvest, Nembl: fra dend Søndere Siide af \textit{Sneaase}vandet til hend imod \textit{Gaundals} fieldet 2 Nye Miile: Dog ligger imellem dette \textit{Røbeen}-fieldet og \textit{Gaundals} fieldet, ett Støcke af forbem{te}\textit{Rokdal} 1: Nye Miil breed; breed er dette \textit{Røbeen} fieldet paa Sin væstere Ende 1 Miil, men paa Sin østere Ende 1/2 miil kun omtrent, Ellers er det j Sin dannelse Rundt og grubbet oven paa. ‒\par
Det 2{det} field j øster fra \textit{Sneaase} bøjden er \textit{Gaundals} fieldet, ett bart field uden Skoug, Græs og Maase; ligeleedes Rundt og grubbet oven paa, Det Stræcker Sig fra Sydvæst nembl: fra \textit{Schicker}-fieldet, hvor med det er field fast, i Nord-ost til een Sæterboelig, Som hører Præste-boelet til, kaldet \textit{Jismonningen}, 5 Nye Miile lang, og er j Sin breede 4 Nye Miile over. ‒\par
Det 3{die} field i Nord-ost er \textit{Brænds}-fieldet, hvor hand ikke er beckiendt. ‒\par
Det 4{de} field i Nord-væst her fra bøjden væsten for bem{te}\textit{Raug} Søen, er \textit{Gied}-fieldet Som er bart uden Skoug, Græs og Maase og Grubbet oven paa. ‒ Det Strækker Sig fra øster nembl: fra forbemelte \textit{Raug} Søe i væster til en Skoug Dal, Som hører \textit{Stoed} til, kaldes \hypertarget{Schn1_30816}{}4 Vidne i Inderøen Fogderi.\textit{Baangna} 3 og er breedt 2 miil over, midt paa dette field skal være skielnet imellem dette \textit{Sneaasens} og \textit{Overhaldens} hoved Sogner; Denne tit omrørte \textit{Raug}-Søe er et liidet Vand, uden Fisk der j, Rund og ett bøsse Skud Stoert. Skoug er her ikke af nogen betydenhed til brug; Deris Næring bestaar af gaardsbrug. ‒\par
Elver j denne bøjd ere Særdeelis 2{de} nembl: \textit{Dalaaen} og \textit{Emsa}-Elven, hvorefter Gaardene ligge. \textit{Dahlaaen} veed hand ikke, hvor fra dend har Sit udSpring, ej heller hvor lang den er; mens det veed hand, at dend gaar fra øster i væster, dog ved Enden, vænder dend Sig i Nord-væst, førend hun falder i \textit{Snaase}-vandet. ‒\par
\textit{Emsa} Elven opckommer af en kiøn ved Sæterboeligen \textit{Jismenningen}, og løeber i væster 4{re} miil i \textit{Sneaase} vandet paa dend Søndere Siide omtrendt ved dends Mitte: Da der imod \textit{Dahlaaen} falder i \textit{Sneaase}vandets østere Ende paa dend Søndere Siide. ‒\par
2{de} Sp: Svarer: Fra dend Synderste gaard her i bøjden kommer man over \textit{Emsa}- Elven, Som her er omtrent 8 allen breed, Siiden igiennem en Skougdal, kaldet \textit{Emsdalen}, hvor kun Slet uuret Skoug er, 1/2 miil breed, der paa følger forbeskreevne \textit{Gaundals} fieldet 4 Nye Miile over, der efter dend Skoug dahlen \textit{Gaundalen} til Grændse-Mærcket \textit{Find}- bækken derj, en god 1/2 miil, til Sammen 5 miile; hvor lang og breed denne \textit{Gaundals vuddu} eller Skougdal er, viidste viidnet ikke. ‒\par
3 Sp: Svarer: Paa dend Syndere Siide ere ingen Gaarder nærmere, end \textit{ongdals} bøjden 3 Stiive miile her fra j Sydvæst, hvilcken Strækning art og beskaffenhed Samt bøndernis Næring Sammestedz, forcklarer hand Ligesom 1{te} Viidne j \textit{Jnderøen} ved 3{die} og 1{te} Spørsmaale. ‒\par
Deris Nærmeste Naboer i øst Nordøst ere \textit{Søndre}- og \textit{Nordre-Finljer} 12 field miile her fra liggende; Hvordanne landskabet er fra denne bøjd did hen til \textit{Brænds}-fieldet, vidste hand ikke, men de andere viidner, Som efterckomme, og boe paa vejen der hend, kunde forcklare det; Landskabet fra \textit{Brænds} fieldet til \textit{Finlje} har det 10{de} Viidne beskreevet.\par
Til 4: Spørsmaal Svarer: Det var ham ubeckiendt.\par
Til 5 Sp: Svarer: Det første grændse field, hand kiender, er \textit{Høysætta} Klimpen, hvilcket hand beskriiver paa Samme Maade Som 1{te} viidne i \textit{Jnderøen} ved 5{te} Spørsmaal, Dog Siiger derhos, at \textit{Buur}-vandet ligger i Sydvæst en god 1/2 miil fra \textit{Høysætta}, og har hand værit med ander \textit{Sneaasinger} for lang tiid Siiden diid, og Fisket der j, mens nu Siiden hørt, at de Svenske der j Fiske; Paa dend østere Siide er hand ikke beckiendt; Det øfrige Stadfæster hand med bem{te} 1{te} Viidne. ‒\par
Viidere i Nord er hand ikke beckiendt; vel har hand hørt, at \textit{Fin} bækken i \textit{Gaundals vuddu} er lande-Mærcke: men hand har ikke værit der. ‒ og kan der for ingen reede giøre.\par
Til det 6{te} S: at de maa være kongens alminding.\par
7: Sp: Svarer: Nej, men ellers \textit{Refererer} Sig til Sit nest forrige Svar ved 5{te} Sp: angaaende \textit{Buur} vandet ‒\par
Til det 8: Sp: Svarer: J Buurvandet er liidet Fisverie, j \textit{Gaundals vuddun} har Lieut \textit{Schouers} Encke Sæter-boelig for qvæget om Sommerne.\par
9: Sp: Svarer Som ved 1{te} Spørsmaal: 5 nye Miile.\hypertarget{Schn1_31062}{} Schnitlers Protokoller I.\par
Til 10: Sp: Svarer: til underholdning for Folck ved hand ingen Raad; men hæstene bliver der vel Raad for i field Dalene. ‒\par
det 11: Sp: Svarer, Det beste er, at Lænsmanden opNæfner nogen dertil. ‒\par
til det 12: Sp: Svarer: at \textit{Missionairen} best kan viide beskeed derom, hvor paa viidnet blev bortladt. ‒\hspace{1em}\par
5{te} Viidne i \textit{Sneaasens} Præstegield\textit{Jnderøens} Fogderie; er Gunder Pedersen \textit{aas}, Føed paa Samme gaard j \textit{Sneaasens} hoved Sogn, af bønder forældre, 46 aar gammel, gift, har 4 børn, boer og er bonde paa Aas,\par
Til det 1{te} 2{det} og 3{die} Spørsmaal Svarer ligesom nest forrige 4 viidne ‒\par
Ligesaa til det 4{de} Spørsmaal\par
Til det 5{te} Spørsmaal: Svarer, hand kiender det Lande Mærcke \textit{Høysætta} Som hand beskriiver ligesom 1{te} Viidne i \textit{Jnderøen}. ‒ J \textit{gaundalen} har hand værit, og hørt, at \textit{Fin}- bæcken derj Skal være ett Skille-Mærcke. ‒ men viidere beskeeden om Landskabet veed hand icke at giive, længere i Nord er hand ickcke beckiendt. ‒\par
6 Sp: Svarer liigesom nest forrige viidne. ‒\par
7 Sp: Svarer, det har hand ickcke hørt ‒\par
8 Sp: Svarer: j \textit{Gaundals vuddu} er høeSlott og beete for qvæget ‒\par
9: 10: 11: og 12: Sp: Svarer, Liige Som nest forrige Viidne, og blev \textit{dimitteredt}. ‒\hspace{1em}\par
6{te} Viidne i \textit{Snaasens} gield\textit{Jnderøens} Fogderie, er \textit{Andfind Jngbrigtsen Gran}, Føed i \textit{overhalden} paa Gaarden \textit{Rostad} af bønder-Forældre, 38 aar gammel, gift, har 5 børn, boer og er bonde paa gaarden Gran i \textit{Sneaasens} Sogn. ‒\par
Til 1{te} 2{det} 3{de} og 4{de} Spørsmaal: Svarer Det Samme Som 4{de} viidne i \textit{Jnderøen} ‒\par
5: Sp: Svarer: hand har hørt, at \textit{Straadals}-Fossen er ett Lande-Mærcke, men Selv icke værit der. Det andet Mærcke Siiger hand at være \textit{Høysætta}, hvilcket hand beskriiver, Som 4{de} Viidne i \textit{Jnderøen}. ‒\par
Og meener at det ligger fra \textit{StraadalsFossen} i Nord-ost 2 miile, Dog kand hand icke Siige det forviist. ‒\par
Om \textit{Buur vatnet} forcklarer hand det Samme Som 4{de} Viidne af \textit{Jnderøen} ‒\par
Om Landskabet paa dend østlige og Nord østlige Siide af \textit{Høysætta} Siiger hand det Samme, Som 1{te} Viidne ved 5 Spørsmaals N{o} 1; Paa dend væstlige Siide Siger hand at være Skoug og Siiden fielde; ‒\par
Paa dend Søndere Siide Field;\par
Den Nordre Siide af \textit{Høysætta} neml: \textit{Gaundals vuddu} beskriiver hand, liigesom 2{det} Viidne i \textit{Jnderøen} ved 5 Spørsmaal. J denne \textit{Gaundals vuddu} eller Skougdahl, Siiger hand, grændse Skillet at være en bæck ved Nafn \textit{Fin}-bæcken; hvad Som nu ligger paa dend østere Siide af Denne bæcken, er Svensk, og kaldes østere \textit{Gaundalen}, hvor vel j gammel tiid folck har boed, mens nu ingen boer, og er alt Sammen der tilgroed med Skoug; ‒\par
Hvad Som ligger paa dend væstere Siide af denne bæck, er Norsk og kaldes væstere \textit{gaundalen}, Dend er af forrige \textit{Sneaasens} Præst \textit{Mag{r}Nils Muus} for omtrent 40 aar siiden \hypertarget{Schn1_31337}{}6, 7 og 8 Vidne i Inderøen Fogderi. opRøddet, og bruges nu af hands Datter \textit{Lieutenant}\textit{Schouers} Encke til Sæter boelig og Fæe-hafvn;\par
Denne \textit{Fin}-bæcken opckommer i \textit{Gaundals vuddun}, og løeber fra Nord-ost i Syd-væst imod \textit{Høysætta}-Klimpen 1/4 vejs ind i \textit{Gauna} Elv, Som før er beskreeven; at løebe i øster ind i \textit{Tørøyen}-Søe; ‒\par
Viidere i Nord er hand icke beckiendt.\par
Til 6 Sp: Saavelsom til det 7{de} 8{de} 9{de} 10{de} 11{de} og 12{te} Svarer det Samme, Som 4{de} viidne ved disse Spørsmaal. ‒ hvor paa hand blev bortladet. ‒\hspace{1em}\par
Det 7{de} Viidne af \textit{Sneaasens} gield\textit{Jnderøens} Fogderie\textit{Jon Baardsen Seem}, var af Lænsmanden hiidskicket til at viidne, hand blev da af Rætten formanet at Siige Sin Sandhed (med Eedens forklaring for hannem) om grændsens gang jmellem begge Riigerne paa denne Kandt og der paa aflagde Sin \textit{Corporlig} Eed; Samme Viidne \textit{Joen Baardsen Seem}, er føed i \textit{Stods} gieldet\textit{Jnderøens} Fogderie paa dend gaard \textit{Seem}, af bønder forældre, 78 aar gammel, gift, 4{re} børn, er husmand paa gaarden \textit{Seem}, i \textit{Sneaasens} hoved Sogn.\par
Fra 1{te} til 4\textit{{de}} Sp: Siiger hand det Samme Som nest forrige 6{te} Viidne. ‒\par
Til det 5{te} Sp: Svarer hand: det 1{te} grændse Mærcke hand kiender i Søer, er \textit{Høysætta}, hvilcket \textit{hand} tillige med andet lande Mærcke \textit{Finbæcken}, Saa og \textit{Gaundals vuddu} med Sine omstændigheder, hand forcklarer ligesom nest forrige viidne, Nembl: det 6{te}. Længere i Nord var hand icke bekiendt.\par
Fra 6{te} til 12{te} Stadfæstede hand det Samme Som nest forrige Viidne. ‒\par
Her blev da jndleeveret ett Skriift fra \textit{Lieutenant}\textit{Schauerts} Encke angaaendes jndbem{te}Væstere-\textit{Gaundalen}, Som hun Skriiver af hendis Sl: Fader \textit{Mag{r}Nils Muus} nu for meere end 40 aar Siiden Skal være oprøddet; Det Samme og af de 2{de} Sidste Viidner er bleven Stadfæsted, Som noget der ligger paa Norsk Grund væsten for \textit{Fin}-bæcken, af hvilcket Skrift \textit{vidimeret Copie} j denne \textit{Commissions} Rætt blev ladet, men \textit{originalet} tilbage Edsket og leeveret hvor paa Viidnerne med laugRættet blev \textit{Dimitteret} til j Morgen tiilig; Da Rætten jgien bliver \textit{Continueret} med de øfrige Viidner af \textit{Sneaasens} gield at \textit{examinere}. Skriftet findes ved \textit{actens} Slutning Lit: A: ‒\hspace{1em}\par
A{o}1742{ve} dend 30 Maij blev Rætten igien foretaget med det 8{de} Viidne af \textit{Sneaasens} gield\textit{Jnderøens} FogderieOle Olsen \textit{oune}, Føed paa \textit{Maalon} i \textit{Sneaasens} Sogn, af bønder forældre, 70 aar gammel, er Encke mand, har 10 børn, er bonde paa bem{te} gaard \textit{oune} i \textit{Sneaasens} Sogn ‒\par
Efter de Spørsmaale, Som ere giorte til 1{te} Viidne af \textit{Jnderøens} Fogderie ‒\par
Til det 1{te} Spørsmaal Svarer: Liige som næst forrige 6{te} og 7{de} Viidne, Dette tilleggendes, at \textit{Brænds} fieldet er bart og Skallet uden Skoug og græs, med liidet Maass paa Sommestædz, har mange klimper og Grubber oven paa Sig, Det Stræcker Sig fra væster i øster 3 miil, og er breed fra Søer i Nord 2 miil, efter Nye bøjde-maaling. ‒\par
Angaaendes Elven\textit{-Dalaaens} udSpring, Siiger hand, at dend udkommer af \textit{Annold}- Søen, Som er rund og 1/2 miil over, uden Fisk; af denne Søe Rinder \textit{Dalaaen} fra øster \hypertarget{Schn1_31665}{}Schnitlers protokoller I. Liige i Væster 3 1/2 nye Miil j østere Ende af \textit{Sneaase}-vandet, Saaleedes Som 6{te} viidne ved 1{te} Sp: har udsagt.\par
Til 2 Sp: Svarer: Som 6{te} Viidne ‒ og 4{de}\textit{dito}.\par
Til 3 Sp: Svarer: hand veed vel, at \textit{ongdals}-bøjden ere de nærmeste gaarder 3 Miile her fra liggende j Sydvæst, men Selv icke værit der, og derfore jngen viidere beskeeden veed at giive. ‒\par
De nærmeste Norske gaarder paa dend anden Siide liige i øster her fra \textit{Sneaase} bøjden 12 field- eller 8 Nye- Miile liggende, er dend bøjd Søndere \textit{Findlje}, og i ost Nordost herfra, dend bøjd Nordre \textit{Findlje}; Landskabet her imellem er denne: Efter dend østere gaard her i \textit{Sneaasens} Sogn, følger et field kaldet \textit{Annolds}-fieldet, brat førend man kommer derpaa, men flat oven paa, og uden Skoug, dog er der Skoug paa Siiden, naar det bær neer ad Dalene, Det Strækker Sig fra væster i øster 2 Nye Miil, og er fra Søer i Nord 1/2 Miil over breed; Paa Fieldet følger j øster \textit{Annold} Søe, Som før er beskreeven; Efter Søen kommer en Skougdal 1 god miil over Siiden \textit{Brænds} fieldet 3 miil over, Det viidere landskab i øster har det 10 viidne beskreevet. ‒\par
Til 4{de} Sp: De nærmeste gaarder paa dend østlige Siide fra det grændse-Mærcke \textit{Gaundals vuddu} har hand hørt at være Kaldbøjden, et \textit{Annex} af \textit{Undersagers} Præstegield, 6 field el: 5 nye Miile, hvis brug og Næring er dend Samme Som de Norskes i \textit{Finljerne}, viidere gaarder Vidste hand icke at Siige; Landskabet der imellem veed hand icke, Saasom hand ej har værit der. ‒\par
Til det 5 Sp: Svarer: Det 1{te} Lande-Mærcke, hand veed af at Siige, er \textit{Højsætta}, hvor om hand giiver Samme \textit{deposition}, Som 5{te} viidne af \textit{Jnderøen}, liigesom hand og Stadfæster Samme Viidnes udsagn om \textit{Buur}-vandets leje.\par
Det 2{det} Lande-Mærcke, angiiver hand at være \textit{Gaundals vuddu}, hvorom hand Giiver Samme forcklaring, Som 6{te} og 7{de} Viidne. ‒ Dette tilleggendes, at hand har hørt, at dette \textit{Gaundals} Skoug-land fra væster i øster Nembl: fra \textit{gaundals} fieldet i væster til \textit{Findbæchen} er ea god 1/2 Miil, og her fra i øster til \textit{Tørøyen} i \textit{Jemteland} 1/2 miil;\par
Det 3: Lande mærcke, viidner hand at, vær \textit{Holdes Holmen} eller \textit{Mitt-Holmen}, hvorom hand giiver Samme forcklaring Som 1{te} viidne af \textit{Jnderøen} ved 5{te} Spørsmaal, tilføjendes at det bare Skallede field paa dend Nordre Siide af \textit{Gaundals vuddu}, hvor denne ophører, heeder \textit{gaundals} fieldet og er Enden der af Sammestedz. ‒\par
Det 4{de} Lande-Mærcke \textit{Holdes Hatten} og det 5{te} Landmærcke \textit{Giæfsøehatten} angiiver og beskriiver hand liigesom 1{te} viidne af \textit{Jnderøen} ved 5{te} Spørsmaal, tilleggendes, at paa denne Saa kaldede \textit{Giæfsøehatten} ere der 3 kiendelige Steenklimper, liggendes fra Søer i Nord efter hinanden, hvor af hand Siiger, dend mellemste at være det rætte og egentlige Skille-Mærcke. Viidere j Nord var Viidnet icke beckiendt.\par
Rætten tilSpurte viidnet, om hand icke vidste: hvor langt det var, fra \textit{Giæfsøehatten} til næste Mærcke Nembl: \textit{Jutahatten?} Svarede Nej, hand var der gandske fremmed. ‒\par
Til 6 Sp: Svarer:\textit{Gaundalsvuddun} er bleven brugt af \textit{Magister Nils Muus}, og bruges nu af hands Dotter, j en tiid over 40 aar. De opreignede Fiskevande for dend halve Norske Deel, ligge under gaardene her i \textit{Sneaasens} Sogn fra gammel tiid. ‒\hypertarget{Schn1_31915}{}Forslag fra Sogneprest Muus om Bebyggelse av Giæfsjøen.\par
Til det 7{de} Sp: Svarer: Nej. ‒\par
8 Sp: Svarer: Liigesom 1{te} Viidne af \textit{Jnderøen}.\par
9 Sp: Svarer: \textit{Høysætta, Gaundalsvuddu, Midtholmen} og \textit{Holdeshatten} kan ligge 5 Nye Miile her fra \textit{Sneaasen}; men \textit{Giæfsøehatten} 6 Nye Miile;\par
Til 10{de} Sp: Svarer: Det Samme Som 4{de} viidne ved Samme Spørs: har udsagt. ‒\par
Til 11: Sp: Svarer: Saa og til det 12{te}, Som bem{te} 4de Viidne af \textit{Jnderøen}, hvor paa viidnet ble udladt. ‒ og der efter\hspace{1em}\par
(9{de} Vid:) blev fremckaldet 9{de} Viidne af \textit{Sneaasens} Præstegield\textit{Jnderøens} Fogderie heeder \textit{Ole Tollefsen Belboe} føed paa \textit{Belboe} i \textit{Sneaasens} Sogn af bønder-forældre, 54 aar gammel, gift, 4 børn, boer og er bonde paa gaarden \textit{Belboe} i \textit{Sneaasens} hoved Sogn. Fra 1{te} til 12 Sp: beviidner hand det Samme Søm nest forrige 8 Viidne hvorpaa dennem blev forkyndt det Samme, Som for de andre viidner ved Rættens Slutning.\hspace{1em}\par
Sogne-Præsten til \textit{Sneaasens} Præstegield Hr. \textit{Peter Muus} var tilstæde og paa \textit{Major Snitlers} jndgivne Skriftlige begiering af 25 \textit{hujus} jnd leveerede en \textit{videmeret Extrakt} af Kierckestoelen angaaendes begge \textit{Finljerne} hvilcke Som \textit{Annexer} Under \textit{Sneaasens} Præstegield fra \textit{arrild} tiid, førend \textit{Jemteland} til Sverrigs Crone er bleven \textit{Cideret}, have \textit{Sorteret}, hvilcken \textit{Extract} af 30 maij \textit{huj: ad Acta} bielegges ved Slutningen under Lit: B: ‒\par
Velbem{te} Hr. \textit{Peter Muus} blev af Rætten anmoedet om hand viidste noget at forestille og tilckiende giive, til Deris Kongl: May{t} og Landsens beste, hvad enten det kunde angaae grændserne eller landets bebyggelse og forbedring?\par
\textit{Resp: Giæfsøen} er tilforn indført i \textit{acten}, Som vel med tiiden kunde formoeds at blive bebyggelig til deris Maj{ts} allerunderdanigste og undersaatlige ti(e)niste af Manskab og Skatt, naar deris Maj{t} allernaadigst ville behage at benaade de Sig først nedSettende jndbygere med 15 aars Skattefrihed, og Saa Siiden benaade mig og mine Arvinger med ett allernaadigste Skiøde, Paa landskabet der omckring, Saavidt Som det Norges Riige tilligger og til oprødning og bebyggelse kand blive tienlig, Som jeg demødiigst viidere vil jndstille til dend høj kongl: tilstundende Grændse-\textit{Commission;} Derhos forcklaredes, at om deris Maj{ts} allerhøjeste Naade vilde unde mig og mine arvinger Ved ett allern: Skiøde Ejendoms Rætten med bøxel og Landskyld, forpligter ieg mig at drage allerunderdanigste muelige omsorg for Samme \textit{Circumferences} opbyggelse og Dyrckelse. ‒\hspace{1em}\par
Fra \textit{Missionairen} Hr. \textit{Povel Muus} jndkom dend berætning om \textit{Finne-Familierne Sneaasen} tilhørende, at de vare_______ \centerline{23{ve} j Tallet.}\hypertarget{Schn1_32151}{}Schnitlers Protokoller I.\par
\centerline{\textbf{Bielage} over \textbf{Sneaasens} Hoved-Sogn\textbf{Jnderøens} Fogderie.}\par
\textbf{Sneaasen} har Sit Nafn enten af Een Platz her i bøjden kald \textit{Sneaasen} Som skal have værit dend ælste gaard her i \textit{Sneaasen}, eller af de omliggende Snee fulde berg-aaser; Det er omgiivet med Fielde uden hvor det Stoere \textit{Sneaase}vand i væster udgaar ad \textit{Stoeds} Præstegield; Dette Præstegield har een hovet- kaldes \textit{Winne} kircke af Præstegaarden Som har det Nafn, og 2{de} Smaa \textit{Annexer}Søndre- og Nordre-\textit{Finljer}, 12 field- eller 8 Nye Maalte Miile der fra j øster og Nordost liggende. ‒\par
Fiske Vande have bønderne af dette Sogn (1) \textit{Sneaase}vandet (2) Nogle paa Grændserne liggende Fiske-vande, hvor af eendeel af de Svenske dennem betvistes. ‒\par
Elvene ere Særdeelis 2 \textit{Em(s)a}- og \textit{Dalaaen} Elvene, langs hvorefter gaardene ligge. ‒\par
Skoug er icke meget igien til noget betydeligt brug. ‒\par
Bønderne ere Særdeelis goede Avelsmænd, at De have mest Deris Næring, af deris gaardz brug j det de føre til \textit{Thiem}, Korn, Meel Smør ost, og Flesk der af det kand kaldes \textit{Thiems} lille Spiise Cammer, dog Frøs kornet paa deels gaarder her af j Fior. ‒\par
Merckeligt er et bierg med jord bedecket; fra øster i væster omtrent 1 1/2 Miil lang og 1. bøsse Skud breed hvor oven paa endeel Gaarder ligge og bruges, kaldet \textit{Liem}- eller \textit{Kalke}-berge der har Sortagtig Steen, hvilcken ved \textit{Mineering} ud brudt j Muurovne legges og i 2 a 3 Samdyng brændes. Hvor efter naar dend med vand overøses Syer og løeser den Sig Selv til hviid kalck dend de ej alleene have til eegen fornødenhed men og til Naboerne kan afsette.\par
Vej fra \textit{Sneaasen} til \textit{Jemteland} er over \textit{gaundals} field igiennem \textit{Gaundals vuddu}, hvor Mærcket er, Til \textit{Finbæken} 5 og her fra til første \textit{Kalds}bøjd i \textit{Undersagers} Sogn 5 Nye Miile; Denne vej er nu icke med de beste, og der for icke tidt fares. ‒ Forleeden Krigs tiid have dog de Svenske nogle gange faret denne vej med \textit{partier} og bort ført Sogne Præsten \textit{Mag{r}Nils Muus}. Jmod denne bøjd Skal de Svenske j gammel Tiid have haft \textit{Kaldstrøms} Skandze af Træe nu forfalden; ‒\par
Her ere 130 bønder uden husmænd. ‒ 30 Skiiløbere, og 40 Staaende Soldattere. ‒ Men Viinter-vejen gaar over de tilfrosne Søer fra \textit{Tørøyen} j \textit{Jemteland} igiennem \textit{gaundalen} over \textit{Holden} og \textit{Jsmenning}-Søerne, Efter \textit{Emsa} Elven til \textit{Sneaase}-vandet. ‒\hspace{1em}\par
Hvad Sig Sogne Præstens hr. \textit{Peter Muusses} allerunderdanigste \textit{Proposition} om ett Støckke øede land paa dend Norske væstliige Siide af \textit{Giævsøe} vandet at faa ved allernaadigst Skiøde for Sig og Sine arvinger forundt; Saa maa jeg forud mælde at hands Fader, afdøede \textit{Mag{r}Nils Muus} har været i henseende til jordens Dyrckning en Nyttig Mand for \textit{publico} j dette Sit Præste gield: Selv har hand paa Sin ejendoms Gaard opRøddet af viilde Skoug en anden gaard Som her bygget og til ager og Eng giordt dueligt. ‒\par
(2) Paa en anden Gaard \textit{Hyllen} har hand af viild Marck giort Kostelig Enge-land ved flittig opRødning. ‒\hypertarget{Schn1_32376}{}10 Vidne i Inderøen Fogderi.\par
(3) \textit{Gaundals vuddu} hvor Grændse Mærcket er til \textit{Jemteland}, har hand opRøddet og ladet bebygge og hensatt Folck dertil at boe, endskiøndt hand 2{de} gange har maattet afbrende huussene, naar en Kriig med Sverrig er begyndt. ‒\par
(4) Paa hands lange Kiercke vej af 12 miile til \textit{Finlje Annexerne}, hvor ingen Folck boer, har hand ved \textit{Gust}-vandet opsatt en Stuevaaning for at hviile der, naar haardt vejr indfalt, og kaldt den Muuse Røe, udreedet og en husmand med Fæe og Mad, at need Sætte Sig der at boe; J vandet hvor ingen Fisk var, har hand med beckostning anden Stædz fra ladet bære og indSætte gedder, og abor, hvor af det vand nu er Riigt. ‒\par
Hand har lært bønderne at dyrcke deris Jorder at de nu der af til byen kan \textit{debitere} og Sælge. ‒\par
Som nu Sluttes hands Søn nu værende Sogne Præst til \textit{Sneaasens} gield maa være af Samme \textit{Genie} at ville See landet forbedret, naar hand allernaadigst dertil Maatte \textit{annimeres} og \textit{Soulageres} med Ejendom der paa for Sin beckostning og \textit{hazard} Dend hand maa Staa, j det hand maa betroe og \textit{Creditere} Fattige Folck Som jngen andenstædz hus have, baade Korn og Fæe, til Slig Nye Rødnings Platzer at medtage og dyrcke; Da Saa fremt Deris Kongl: Maj{t} denne Præstens allerunderdanigste foreslag allernaadigst ville \textit{aggreere}, Saa Siunes u-forgriibelig allerunderdanigst, at det kan lade Sig giøre naar de Kongl: grændse maalere have værit paa Stædet og ingen \textit{difference} over denne \textit{Circumference} jmellem begge Riigernes Grændse Maalere Skulde forefalde; Da en ordentl: Marckegang og besigtelse Samt afdeeling til visse gaarders qvæg af Kongl: May{ts} Foged og Soerenskriiver Skee, og efter allernaadigt befindende behag Præsten forundes kunde; Følgen \textit{effecten} her af ville blive, at ej alleene nogle Fattige \textit{Familier} vilde der faa livs ophold, men og deris Kongl: Maj{ts}\textit{Jnteresse} baade med ungt Mandskab og j Sin tiid med uddersattlige Skatter forbedres kunde. ‒\par
Følger nu \textit{Finljerne} af \textit{Sneaasens Annexer}, Som boe j Nord ost fra \textit{Sneaasens} hoved Sogn, og deris \textit{deposition}, hvilcken dog af Søndre \textit{Finlje}-Mænd d: 26{de} og af Nordre \textit{Finljerne} d: 28 Maij nest tilforn, for deris lange vejs skyld, var jndtaget; Af hvilcken aarsag bliver af \textit{Sneaasens} Præstegield\textit{Jnderøns} Fogderje\par
10 Viidne \textit{Bendt Hansen øsnaar}, føed paa Samme Gaard i Sønder-\textit{Finlje}\textit{Sneaasens} Præstegield af bønder-folck Samme Stædz, 70: aar gammel, gift, har 10 børn, er bonde paa bemelte \textit{øsnaar} ‒\par
1: Spørsmaal: \textit{Resp}: Dend bøjd \textit{Sønder-Findlje} ligger i Nord-væst fra Grændse- Mærckerne ad \textit{Jemteland, Røkviighyllan, BrunsiøflintenLægster Klompen} og dend \textit{Linie} østen for \textit{PenningKiesene} 4 1/2 gamle- Som hand meener at være 3 Maalte bøjde-Miile; i \textit{Sneaasens} Præste gield og \textit{Juderøens} Fogderie;\par
Landskabet af dette Søndere \textit{Findlje} bestaar af noget Eng-land der, hvor gaardene ligge, Fiskevande, Fielde, og Gran-Skoug i Dalene;\par
Korn faar de gandske Sielden, og leeve af Furru-brød, deris qvæg, Fiskerie og Skytterie; Der ere 8{te} gaarder og 15{ten} bønder; Denne bøjd gaard- imellem Stræcker Sig fra Sydost i Nord-væst 3 gamle- Som kan være 2 nye bøjde-Miile; Breedden af gaardenes giærder \hypertarget{Schn1_32591}{} fra væster i øster er kun Smal; Gaardene ligge længst Vandene Paa dend Nordøstlige Side, Dernest er paa begge Siider Skoug af Gran 1/8 miil breed, derpaa begynde fielde paa alle Siide, undtagen mod Nordre-\textit{Findlje} i Nord, hvormed de Sammenhænge ved een Skoug. ‒\par
Fisk-Vandene i bøjden ere Følgende: Dend 1{te} Søe i Nord væst, hvor Gaarden \textit{Holdesovne} ligger ved, paa østere Siide er \textit{Holden}, Strækkende Sig fra væster i øster 1/2 miil lang, 1/8 miil breed, Fra denne \textit{Holden}-Søe Riinder en Elv \textit{Holdes-aae} fra Nord-væst i Syd-ost 1 gammel- som kan være 3/4 Nye- Miil lang, et halft bøsse Skud breed, i det vand \textit{Langlingen}; Dette \textit{Langlingen} Strækker Sig fra Nord-væst i Syd-ost 1 1/2 gammel- Som kand være 1 Nye- Miil, hvor det er breedest, er det en god 1/4 Miil over, men paa begge Ender gaar det Smalt til, dog at dend østere Ende er liidet breedere, end dend væstere; Ved dette Vand paa Nord-ostlige Siide ligger de gaarder \textit{Ulland, Totland, Devig, Meebøjden}, (paa denne Siste gaard boe 4 bønder) Paa \textit{Devigs} gaards grund, Som hører \textit{Sneaasens} Præste-Boelet til, Staar en liiden Træe-Kiercke, kaldet \textit{Deviigs} Kiercke, Som er bygget, og holdes ved liige af bønderne Selv; Sønden for \textit{Meebøjden} og \textit{langlingen}-Søe 1 liiden Miil ligger dend gaard \textit{yngoldalen}. Viidere paa dette \textit{Langlingen} vands Sydvæstlige Siide ligger et andet liidet vand, kaldet \textit{Guus}-vandet, og en god 1/4 Miil der fra, Strækkende Sig fra væster i øster 3/4 Miil, breed fra Søer i Nord 1/8 Miil; Af dette \textit{Guus}-vand Rinder \textit{Guus-aaen} fra væster i Nordost 1/4 miil et halft bøsse Skud breed, J forbem{te}\textit{Langlingen} Søe. Af denne \textit{Langlingens} østere Ende udfalder en Elv, \textit{Langling}-Strømmen, og gaar 1/4 Miil i Sydost, ett bøsse-Skud breed, i dend Søe \textit{Uulen}; Denne \textit{Uulen}-Søe er mest rund, Strækker Sig fra Nord-væst i Syd-ost 1/4 Nye Miil, og næsten liige Saa breed over; Paa dend Nord-ostlige Siide her af ligge de 2{de} gaarder \textit{Uul} og \textit{Skaale}; Paa dend Sydvæstlige Siide der hend imod, hvor gaarden \textit{Uul} ligger paa dend anden Siide, indløber en Stoer Elv ved nafn \textit{ynggola}, ett godt bøss-Skud breed, Som kommer fra Syd-væst udaf Smaa field-vande, 3 miile der fra, og gaar i Nord-ost igiennem \textit{ynggol}-dalen i bem{te}\textit{Uulen}- Søe; Denne \textit{ynggol}-dahl, efter hands giisning, Strækker Sig fra Syd-væst i Nord-ost imellem \textit{Storblaa}-field og \textit{Kiellingsnasen} 1 miil hend til vandene, og har gran-Skoug; J denne dahl Norden for \textit{ynggola}-Elv 1/2 miil fra \textit{Uul} Søen, ligger dend gaard \textit{ynggoldalen}.\par
Fra \textit{Uulen} Søe udriinder en Elv i Syd-ost Ett bøsse-Skud lang ved Nafn \textit{Son-aae} j dend breede Elv \textit{Sonnen}, Som Strækker Sig i Sydost, 1/2 Nye Miil lang, og er 2 bøsse Skud breed; Denne \textit{Sonnen}-Elv med Sin Sydøstlige Ende Naaer liige i dend Nordvæstlige Ende af \textit{Rengen} Søe; Denne \textit{Rengen} Søe Strækker Sig fra Nord-væst i Syd-ost 1 Nye Miil lang, og 1/4 Nye Miil breed; Af denne \textit{Rengen}-Søe paa dend Syd-ostlige Ende udkommer en Elv, ved Nafn \textit{Reng}-Strømmen, Som løeber i Syd-ost omtrent 2 bøsse Skud lang i \textit{Qval}- Søen, Som er Rund 1/4 Nye Miil over; Af denne \textit{Qval}-Søe Riinder en Elv, Nafnl: \textit{TorskEjde} i Sydost 1/4 Nye Miil lang, ett godt bøsse Skud breed, i dend Søe \textit{Hotagen}, hvor i grændse Skielnet er; ‒ J denne \textit{Hotagen} indløeber en anden Elv fra væster først af \textit{Langvatnet}, en Søe liggendes i et bart field fra \textit{Hotagen} Søe 2 Nye Miile i væster; Denne \textit{Langvatten}-Søe Strækker Sig fra væster i øster 1/2 Nye Miil lang, 1 1/2 bøss Skud breed; Af dette \textit{Langvatnet} naar Elven udkommer, er dend kun Som en bæck, og kaldes \textit{Arvats}- bæcken, og løeber i øster 1/8 Nye Miil i dend Søe \textit{Arvevatnet}; hvilcket \textit{Arvevatn} Er Rund, \hypertarget{Schn1_32869}{} omtrent 1/8 Miil over; Udaf dette \textit{Arvevatn} Naar denne \textit{Arvatsaaen} udfalder, foreener dend Siig med en Elv, Nafnl: \textit{Juta}, Som kommer af \textit{Steen} fieldets Nordre Ende og dets østlige Siide, og gaar i øster 1 Nye Miil i denne \textit{Arvats-aaen}, og fra dette Stæd beholder Elven det Nafn \textit{Juta}, og løeber liige i øster 1/4 Nye Miil i \textit{Raavatnet}; Dette \textit{Raavatn} bestaar af 3 vande, og hver har Sit Særdeelis Nafn: Det første Norderste heeder \textit{Nord-vatnet}, det andet \textit{Bue-vatnet} og det 3{die}\textit{Døel-vatnet}: Dog hænge de 3 vande ved 2{de} trange Sund Sammen, at man kand Roe dem igiennem; Naar Elven falder ud af \textit{Raa Wandene}, kaldes dend \textit{Raavats}-Elven og løeber liige i øster omtrent l/4 Nye Miil i dett vand \textit{Lokringen}; ‒ hvilcket \textit{Lokringen} Strækker Sig fra væster i øster 1/8 Miil, og er et liidet Bøsse-Skud breed, der af naar Elven Riinder, kaldes dend \textit{Lokrings-aaen}, og løeber i øster Synden for i \textit{Røkviigen} i \textit{Hotagen}-Søe. Her erindrede Viidnet Sig at anmeldde hvor \textit{Arvats-aaen} og \textit{Juta} foreene Sig, der opstiiger ett field, kaldet \textit{Grubdals} fieldet, Strækkende Sig fra Væster i øster 1 Nye miil lang omtrent l/4 Miil breed imellem disse Elve, og der af ere opstiigende 2 høye klimper Kaldet \textit{Jut}-Hatten. ‒\par
Forbem{te}\textit{Hotagen}-Søe ligger fra Nordvæst i Sydost omtrent 3 1/2 Nye Miile lang, er uliige breed, hvor dend er breedest, er den l/4 nye Miil over. 1/4 Nye Miil Norden for \textit{Hotagen}-Søe ligger et vand, kaldes \textit{Hatling-vand}, Som Strækker Sig fra Nord j Søer l/8 Nye Miil lang, 2 bøsse Skud breed;\par
Dette \textit{Hatling-vand} bestaar af Toe vande, hvor imellem gaar en bæck, omtrent 3 Steenkast lang, og Saa breed, at man knap drager baaden derigiennem, fra dett eene vand i det andet; udaf dette \textit{Hatling} vand Riinder en Elv, Nafnl: \textit{Hatling-aaen} 1/4 Nye Miil i Søer i \textit{Hotagen}-Søe i dends vætslige Ende Paa dend Nordre Siide, 2 Steenckast Norden for det Stæd, hvor dend Elv \textit{TorsckEide} falder Synden for i \textit{Hotagen}-Søes væstere Ende. ‒ 1 Nye Miil i væsten for \textit{Hatlingvandet}, og 1 1/2 Nye Miil østen for dend gaard \textit{Meebøjden} ligger endnu ett vand, ved Nafn \textit{Gundervatn}, 1/4 Nye Miil Norden for det vand \textit{Rengen}; Hvilcket \textit{Gundervatn} bestaar af 3 vande: Det Norderste heeder \textit{Stue} vatn, er Rund og 1/4 Nye Miil Stoer, Dette \textit{Stue} vatn hænger ved en Elv, af 2 Stenkast lang, Sammen med det andet andet vand, \textit{Rausiøn} kaldet; Denne \textit{Rausiø} er mest Rund, 1/8 Nye Miil Stoer, og er ved ett Sund, et halft Steen kast lang, Sammenføjet med det 3{die} vand, Nafnl: \textit{Storvatn}, Som Strækker Sig fra Nord-væst i Syd-ost 1/2 Nye Miil lang, og er 1/8 miil breed; Af Disse \textit{Gunder}-vande udgaar \textit{Waten-aae}, og Riinder l/4 Nye Miil i Sydost i \textit{Qval}-Søen, før beskreeven.\par
Endnu 1 Nye Miil Norden for \textit{Deviigs} kiercken Paa dend Nordvæstlige Siide af \textit{Buurs}- klimpen (ett høytt field) ligger ett vand Nafnl: \textit{Liø-siøn}, Som er rund, 1/8 Nye Miil Stoer, og hører Søndere \textit{Findlje} til; Der af udkommer en Elv \textit{Lutra}, Som Rinder Nord-Nordvæst 1/4 Nye Miil lang i \textit{Sand}-Søen hvilcken Søe er Nordre \textit{Findljes}. ‒\par
J for opReignede Fiske-vande Fanges Rø og øret, undtagen \textit{Qval}-Søe, hvor Siig og harr er, og \textit{Hotagen}, hvor foruden anden fisk giedder faaes.\par
Fieldene ligge i Søndere \textit{Finlje}, og høre denne bøjd alleene til: først \textit{Søndere Gust}- field, for det andere Nordere \textit{Gust}field, for det tredie \textit{Kiællingsnasen}, for det Fierde \textit{Stor blaa} fieldet Hvilcke hænge med hinanden Sammen ved nogle field-dahle og ere liige Som \hypertarget{Schn1_33155}{} Eet field, bart og Skallet uden Skoug og græs, dog er der nogen Maasse paa; De Stræckke Sig fra Nord i Søer, efter hands Giisning 6 Nye Miile lang, og 2 miile breed, Somme Stedz meere og Somme Stedz mindere; ‒\par
Disse \textit{Gust}-fielde ere i Nord fieldfaste med \textit{Brænds}-fieldet, der er ett grændse field imellem \textit{Findljerne} og \textit{Sneaasen} Sogn i væster; ‒\par
\textit{Storblaa}-fieldet i Søer, har hand hørt Strækker Sig hen imod \textit{Gaundalsvuddu}. J øster har dette \textit{Sør-lie} det Stoere field \textit{Haarkiølen}, Som Strækker Sig fra væster i øster ind i \textit{Jemteland} 12 gamble- der kan være 8 Nye Miile, og fra Søer i Nord meenes at være 3 Nye Miile over;\par
Dette \textit{Haarkiølen} med Sin væstere Ende begynder 1/4 miil vejs fra \textit{Sand}-Søen, Som hører Nord-\textit{Findlje} til; ‒ J Søer har denne bøjd det field \textit{Middags}-field, Strækkende Sig fra væster i øster 1 Nye Miil lang, fra Nord i Søer 1/2 Miil breed, er et Snaut Bart Field uden Skoug og Græs, dog med Maasse;\par
Dette Field ligger fra \textit{Skaale} gaard 1/2 nye Miil i Søer, og fra \textit{Arve vatnet} 2: Bøsse skud i Nord; ‒\par
Paa dend Væstere Siide ligger en liiden Skoug Dahl ogsaa det field \textit{Kiellingsnasen}; Paa dend østere Siide er og en Større Skoug-dahl, Som Rækker i Syd-ost hend til \textit{Rokviigen}; og væsten for denne Skougdal ligger det vand \textit{Rengen}.\par
Til det 2 Sp: Svarer: jmellem dette Søndere-\textit{Findlje} og det l{te} Lande-Mærcke i Sydvæst Nembl: \textit{Juta}-hatten ligger det vand \textit{Uuld} Søen\textit{Middaqs} fieldet og \textit{Arvevatnet}; Fremdelis jmellem denne bøjd og det 2{det} Lande-Mærcke liige i øster fra \textit{Juta}-hatten nembl: Midtt over \textit{Raavatnet} fra væster i øster er bare Gran-Skoug; jmellem denne bøjd og det 3{die} Land-Mærcke \textit{Røkviighylla} liige i øster fra Mitten af \textit{Raavatnet} er gran- og nogen Furru-Skoug med de vande \textit{Rengen}- og \textit{Qval}-Søen; jmellem bøjden og 4{de} 5{te} og 6{te} LandeMærcker Nembl: \textit{Brunsiøflinten} ost-Nord-ost fra \textit{Røkviighyllen}, \textit{Hatling}vatnet Norden for \textit{Brunsiøflinten}, og \textit{Legster-Klompen} Norden for \textit{Hatlingvandet}, Et bare gran- med nogen Furru-Skoug; Endelig jmellem denne bøjd og \textit{Linien} østen for \textit{Penningkiesene} paa det field \textit{Haarkiølen} j Nord Nord væst fra \textit{Legster Klompen} er bare field. ‒\par
Til det 3{die} Sp: Svarer: Sønden for denne bøjd Søndre-\textit{Findlje} Kiender hand ingen Gaarder paa dend Nordske Siide nær veed Grændserne. De gaarder j dette Søndere \textit{Findlje}, hvor nær de ligge Grændserne? af hvad beskaffenhed Landet er? og hvad Næring bønderne bruge? Det har hand beSkreevet før ved l{te} og 2{det} Spørsmaal. ‒\par
Norden for denne bøjd Søndere \textit{Finlje} ligge de gaarder, Som høre Nordre-\textit{Finlje} til, Nembl: dend første nærmiste 2 gamble Som kan være 1 1/2 Nye Miile Norden for dend Sidste Norderste gaard i Syndere-\textit{Finlje}, \textit{Haaldesovne}, er \textit{Hoeland}, og jmellem disse 2 gaarder er bare gran- og biercke-Skoug, Sambt Myhrland; ‒ og dendne gaard \textit{Hoeland} ligger paa dend Nordre Siide af det vand \textit{Sand}-Søe, og denne Søe, 1/8 Miil breed over, ligger Norden for ovenmelte Gran- og bircke-Skoug jmellem begge benæfnte Gaarder, og er indbegreeben i de 1 1/2 Nye Miils \textit{distance}, Som disse 2 gaarder ligge fra hinanden. ‒\par
Dend 2{den} gaard 3/4 Nye Miilvejs i Nordvæst fra \textit{Hoeland} er \textit{Sandviigen}, liggendes ved \textit{Lax} Søen ved dends Nordere Siide, og er der imellem granskoug; Paa denne gaards \hypertarget{Schn1_33442}{} grund Staar Nordre \textit{Finljes} Kircke bygget af Træe, Som bønderne Selv have beckostet, og ved liige holde, kaldet der af \textit{Sandviigs} Kiercke. ‒\par
Dend 3{die} gaard i Sydvæst fra \textit{Sandviigen}, liggendens paa dend Søndere Siide af \textit{Lax} Søen 1 Ny Fierding Miil der over, er Næs. ‒\par
Dend 4{de} gaard ligger Norden for \textit{Lax} Søen ved dends Nordre Ende, en Nye 1/4 Miil til Landz- og en Nye 1/8 Miil til Vandz igiennem \textit{Lax} Søen fra \textit{Næs}-gaard, imellem hvilcke gaarder \textit{Lax} Søen ligger, og denne 4{de} gaard heeder \textit{Taasaasen}. ‒\par
Dend 5{te} gaard er \textit{Bratland} en Nye 1/4 Miil i Nord fra \textit{Taasaasen}, liggendes paa dend Nordre Siide af \textit{Bradtland-vandet}, hvilcket vand er imellem disse 2{de} gaarder.\par
Dend 6{te} gaard \textit{Schielbren}, en Nye 1/16 miil væsten for \textit{Bratland}, liggendes Norden for \textit{Schielbrends}vandet, der ved et Sund hænger Sammen med \textit{Bratlands}vandet;\par
Dend 7{de} gaard \textit{Leerbacken} ligger 3/4 Nye miil i Nordost fra \textit{Schielbreen} ved en Elv \textit{Qvælie-aaen} paa dends Nordre Siide ‒ jmellem disse 2: gaarder ligger en liiden Field-\textit{Ruve} med en liden Skoug paa dends begge Siider, og omsiider \textit{Qvæljeaaen}. ‒\par
Dend 8{de} gaard er \textit{Qvælien}, liggendes fra \textit{Leerbaken} en Nye 3/4 Miil i 0st-Syd-ost, Norden for Samme \textit{Qvæljeaae}, og er Gran-Skoug imellem disse 2: Sidste Gaarder; Disse ere alle gaardene i Norder-\textit{Findlje}, foruden 2{de} Nyebyggere, Som for nyeligen har needSatt Sig ved \textit{Frostviig}vandets Nordvæstlige Ende. ‒\par
Disse gaarder i Nordre-\textit{Findlie} de Sønderste der af kan have efter hands Giisning til nærmeste Lande-Mærcke, Nembl: østen for \textit{Penningkiesene} 6 gamle, Som kan være 4: Nye Miile; De Norderste der af formeenes at have til \textit{Svanesteen} grændze-mærcket omtrent 11{ve} gamle, Som kan Reignes for 8: Nye Miile meer og mindere efter Gaardenes leje. ‒\par
Landets Beskaffenhed og bøndernis brug kan være det Samme, Som deris i Sønder \textit{Finlje}.\par
Til det 4: Sp: Svarer: Sønden for det 1{te} Lande-Mærcke \textit{Juta Hatten} ligger et Stoert viidt langt Field, veed Nafn \textit{Steen}-fieldet, hvis leje og Egentl: viidde hand ikke veed; ‒ og Sønden for dette \textit{Steen} field er en Stoer Skoug, med Myhr land, Sønden derfor igien, og i Sydost for \textit{Juta} hatten, ligger een gaard \textit{Tongeraas}, hvor 2 a 3 bønder boer og Svarer under \textit{offerdals} hoved Kiercke, hvor til de have 3 Svenske Miile; Disse bønders Næring er Samme, Som andre field bønders; J forbm{te}\textit{Steen} field paa dens østere Siide i Søer Skal for nogle aar Siiden af de Svenske være optaget ett Nytt Sølv-Værck, Som de have hørt, i forleeden Sommer ej er bleven arbejdet paa: Aarsagen her til viidste hand ikke. ‒\par
Paa dette \textit{Steen} field i øster følger een Skoug, omtrent 1: miil lang og derpaa i øster ett field, kaldet \textit{Ansigt}-fieldet, meget højt og det høieste, Som er, med 3 høje klimper oven paa, staaendes fra væster i øster efter hinanden, lang, efter deris formeening, fra væster i øster 3/4 Miil, og næsten liige Saa breed; og dette \textit{Ansigt} field ligger nærmist i Søer fra det 2{det} Lande-Mærcke Raavatn Hvor i Mitten Skielnet gaar; og J Syd ost fra dette \textit{Ansigt} field ligge 2: gaarder kaldet \textit{Landøyen}, hvo(r)paa 4{re} bønder boe, og ligge i \textit{offerdals} hoved-Sogn: Dog at imellem disse gaarder og \textit{Ansigt} fieldet ligger en Skoug; ‒ Disse gaarder have Samme brug og Næringen Som andre tilfieldz. ‒\hypertarget{Schn1_33723}{}\par
J Søer og Sydost fra det 3{die} Lande-Mærrke \textit{Røkviighyllan} Ligger en Stoer Viid Furru-Skoug, hvilcken hand meener at være i Syd ost 5 el: 6: Nye Miile lang, hvor efter ligger ett \textit{Annex} i \textit{Liets} Præstegield, kaldet \textit{Føling}, hvor Gaardene alle ligge nær Sammen, ved ett liidet vand \textit{Heim} Søen; Disse gaarders tilstand er bedre end de forriges, jdet de faae korn Saa velsom anden Stædz i Dahl bøjdene.\par
Østen for det 4{de} Lande-mærcke \textit{Brunsiøflinten}, er atter en Stoer viid Furru-Skoug, vel 7: nye Miile lang i øster, hvor paa følger \textit{Hammerdals} hoved-Sogn, hvor Somme Stædz er goede Korn Jorder, Somme Stedz ikke. De Norderste gaarder der af ligge ved dend Søe \textit{Gaukssiøn}. ‒\par
Østen for det 5{te} Lande-Mærcke er Samme landskab Som ved næst forrige, hvilcket 5{te} lande-mærcke heeder \textit{Legster-Klompen}. ‒\par
Østen for det 6{te} Lande-Mærcke, Nembl: den østere Deel af \textit{Penningkiesene}, er det øfr: af \textit{Haardkiølen}, hvor Skatt-lagde \textit{Lap-Finner} tilholde. ‒\par
Til det 5{te} Sp: Svarer Det\par
1{te} Lande-mærcke Som hand kiender, er \textit{Juta-Hatten}, Som ved 1{te} Sp: før er beskreeven, hvor langt det er her fra i væster til \textit{giæv Søehatten?} viidste hand ikke, liigeSaa vidste hand ikke: hvor langt det er fra \textit{Juta} hatten til nærmiste \textit{Raavatn}: dog har hand hørt, at fra \textit{Juta Hatn} liige i øster gaar grændze gangen Midt over dette \textit{Raavatn} til \textit{Røkviighyllan}; og paa denne Maade vil hand have forstaaed det\par
2{det} Mærcke: \textit{Raavatn}, at det Nembl: Som hand har hørt, ligger i Lande-Mærcket, Saaleedis at fra 1{te} mærcke \textit{Jutehatten} Skal \textit{linien} gaae midt over dette \textit{Raavatn} til det mærcke \textit{Røkviighyllan}, dog, Som Sagdt, har hand bare hørdt det, men Selv har hand ikke værit hvercken paa \textit{Juta} eller ved \textit{Raavatn}. ‒\par
Rætten tilSpurde Viidnet: Siiden hand ved 1{te} Sp: har forcklaret, at \textit{Raavatnet} bestaar af 3: vande, nembl: \textit{Nordvatnet}, \textit{Bue-watnet}, og \textit{Døelvatnet}, ofver hvad for ett af disse vande hand har hørt, grændze gangen gaar? hand Svarede det maa være \textit{Nordvatnet}, Som hand har hørt. ‒\par
Landskabet paa dend Søndere Siide har hand før beskreeven. Paa dend nordre og østere Siide til \textit{Hotagen} Søe, Sambt paa dend Væstere Siide er Gran-Skoug; hvor langt er fra dette \textit{Raavatn} til neste \textit{Røkviighyllan?} veed hand ikke; og dette \textit{Røkviighyllen} angiiver hand for 3{die} Mærcke: og at være en Stoer Steenhylle, staaende ved dend Nordere Siide af \textit{Røkviigen}, hvilcken \textit{Røkviigen} er dend første begyndelse af \textit{Hotagen} Søe i Væster; ‒\par
Landskabet paa dend østere Siide er \textit{Hotagen} Søe; Paa dend Væstere og Nordre Siide er Skoug, og Norden for Skougen, er dend anden Nordere viig af \textit{Hotagen} Søe, kaldes \textit{groveln}, Paa dend Søndere Siide er \textit{Røkviigen}, Som er dend Søndre viig at \textit{Hotagen} Søe, og i Sydost fra denne \textit{Røkviigen} ligger dend før omtalte Furru Skoug. Denne \textit{Hotagen} Søe med dends viiger nærmere at forcklare, Saa har denne Søe 2{de} Viiger, dend eene Nordlige Viig Strækker Sig i Nord væst og kaldes \textit{Groveln}; Dend anden Søndere Viig Stræcker Sig fra Siøn liige i væster og heeder \textit{Røkviigen}; J denne Nordre viig eller \textit{Groveln} er det nu, Som ved 1{te} Sp: er beskreeven, at dend Elv \textit{TorskEide} ud af \textit{Qval} Søen falder, naar dend der beskriives at riinde i \textit{Hotagen}-Søe; Liige Saa er det denne \textit{Groveln}, hvor \textit{Hatling-aaen}\hypertarget{Schn1_34020}{} af \textit{Hatling}-Søen kommer i, naar dend Samme Stædz beskriives at falde i \textit{Hotagen} Søe, et Støcke i fra \textit{TorskEide} Elven: Der imod er det \textit{Hotagens} Søndere Viig, Som heeder \textit{Røkviigen}, j hvilcken \textit{Lokrings} Elven Riinder paa dend Søndere Siide; Midt imellem disse 2{de} Viiger gaar een Steen-Tang eller kløft ett Støkke ind i \textit{Hotagen} Søe, paa dend Søndere Siide af denne Tang, Tædt Norden ved \textit{Røkviigen} Staar dette grændse Mærke,\textit{Røkviighyllan}. ‒\par
Viidere her om forklares: Naar \textit{Linien} fra denne \textit{Røkviigshyllan} i Nord-ost træckkes til næste grændse-Mærcke \textit{Brunsiøflinten}, Saa vil Søndere \textit{Finlje} tilckomme, dend Nord væstlige Ende af \textit{groveln}, omtrent 1/2 Miil vejs til i \textit{Hotagen} Søe; og dette \textit{Brunsiøflinten} er det\par
4{de} Grændse-Mærke, hvor Viidnet Siiger Selv at have været veed og paa;\par
Dette er en berg klimp rundagtig og noget Spitz oven til, dog at en 5 el: 6 Mand kan Rømmes der oven paa, er af en Mands højde, og noget brat, dog kan man nogenstedz med Møje gaae did op.\par
Paa dend Nordre Væstere og Sydvæstlige Siide hen til \textit{groveln} er Furru Skoug; Paa den Sydlige Siide ligger \textit{Hotagen} Søe, Det østl: landskab er før beskreeven ved 4{de} Spørsmaal. Paa denne \textit{Brunsiøflinten} gaar grændsegangen over \textit{Hatlingvandet} 1 Nye Miil i Nordvæst til nærmeste grændse-mærcke \textit{Legsterklompen}, hvor Viidnet og har været Som bliver det\par
5{te} Lande-Mærcke: Denne \textit{Legsterklompen} beskriiver hand at være en berg Knol, Strækkende Sig fra væster i øster l/8 Nye Miil, og er breed ett bøsse Skud over, kan være en 3 Mands-høy, hvor det er højst, er bevoxet med gran og Furru-Træer, Som dog kun ere Smaa. Paa Nordre-, væstere-, og Syndere Siide er bierck og gran Skoug; dend østere Siide er før beskreeven ved 4 Spørsmaal. ‒\par
Fra denne \textit{Legsterklompen} gaar j Nord-væst grændse-gangen 1 god Nye Miil omtrendt til dend østere Ende af \textit{Penningkiesene}, Som er det\par
6{te} Lande-Mærcke, hand veed af at Siige: Disse \textit{Penningkiesene} forcklarer hand at være een berg-klimp, Som Stiiger op i vejret af det field \textit{Haarkiølen} paa dend Søndere Siide, omtrendt 3 Nye Miile, Som hand meener, fra dends Væstere Ende; Klimpen er i Sig Selv icke højere end en backe, Som man kan kiende Fra \textit{Haarkiølen} at opstiige, oven paa er den flad og bahr uden Skoug og Græs, j Længden Strækker dend Sig fra Søer i Nord 1/8 Nye Miil, breed kan dend være 3 bøsse skud over; Længere i Nord var Viidnet ikke beckiendt: Dog har hand hørt, at \textit{Svanesteen} i Nord er grændse-Mærcke. ‒\par
Til det 6{te} Sp: Svarer: De ere kongens alminding undtagn \textit{Raavatnet} ligger under dend gaard \textit{Skaale}, og \textit{Hatling} ligger under \textit{Meebøjds} gaarden. ‒\par
Til det 7{de} Sp: Svarer Bønder af \textit{offerdals} Sogn j Jemteland have tileignet Sig \textit{Raavatnet} og Skougen der omckring, og bruge den Endnu; De Svenske \textit{Lap-Finner} til-Ejgne Sig \textit{Grubdals}- og \textit{Middags}-fieldene, hvorj hand meener, at \textit{Finlje} Mænderne Skeer for nær\par
Til det 8{de} Sp: Svarer: j vandene kan der være fiskerie og \textit{Bæver} fangst i Skougene Skytterie paa fieldene, Maasse og Græsljerne ‒\par
Til det 9{de} Sp: Svarer hand beraaber Sig paa Sit forrige Svar ved 1{te} Spørsmaal udsagt ‒\hypertarget{Schn1_34253}{}\par
Til 10{de}: Sp: Svarer: til underholdning veed de ingen Raad, men beete for hæstene, meener hand, at Bliver Raad for i field dalene. ‒\par
Til det 11{te} Sp: Svarer; Naar hand kommer hiem i bøjden, vil hand Samtales med de andere bøjde-mænd, og See at en beqvæm Mand der til udfindes. ‒\par
Til det 12{te} Sp: Svarer at \textit{Missionairen} her paa Stædet bæst kand giive underrætning derom. ‒\par
hvor paa Viidnet blev \textit{dimitteret}. ‒ Efter at have tilforn forklaret, at fra Syddostligste gaard \textit{Skaale} til \textit{Raavatnet} er 3 gamle ‒og til \textit{Jutahatten} 3 1/2 gamle- Miile, hver gammel Mill efter giisning Reignet for 3/4 deel Nye Miil. hvor langt fra giævsøhatten til Jutahatten er, vidste icke.\hspace{1em}\par
Det: 11{te} Viidne j \textit{Sneaassens} Præstegield\textit{Innderøens} fogderie ‒\par
heeder \textit{Joen Jonsen Skaale}, føed paa \textit{Skaale} i \textit{Sønder findlje} af bønder forældre, 50 aar gammel, gift, har 10: børn er bonde paa \textit{Skaale}. ‒\par
Efter Rættens tilSpørgende beckræfter dette Viidne dett Samme, Som næst forrige fra Søndere \textit{Findlje} fra 1{te} til 12{te} Spørsmaal \textit{inclusive}; Rætten til Spurte ham i anleedning af 5{te} Sp: hvor langt det er fra det grændsemærcke \textit{Juta}-hatten til nærmeste \textit{Raavatnet}, hvor over grændse gangen gaar?\par
\textit{Resp:} En gammel 1/2 Miil, Som hand meener, kan være een Nye 1/4 miil; Fremdeelis tilspurt\par
hvordanne Landskabet er paa \textit{Raavatnets} Siider?\par
\textit{Resp:} Dend Søndere Siide beskriiver hand, liigeSom næst forrige viidne; Paa de øfrige Siider er dette \textit{Raavatn} omgiiven med granskoug. ‒ Fremdeelis tilspurt? hvor lang det er fra \textit{Raavatn} til \textit{Rokviighyllan?}\par
\textit{Resp:} 1/4 Nye Miil. hvor paa hannem og hands med-viidne fra Sønder \textit{Finlje} dend kongel: \textit{ordre} blev kundgiort, Som der Staar indført for \textit{ongdalingerne} i \textit{Sparboe} efter Rættens Slutning; og de derpaa hiemladt. ‒ Efter at have Svaret, at de ej viste hvor langt det er imellem \textit{giæf}Siøhatten og \textit{Juta} hatten. ‒\hspace{1em}\par
Som det var Saa Siilde paa Aftenen, blev Rætten opSatt til nest følgende\par
A{o}1742 dend 28{de} Maj: blev \textit{Examinations} Rætten foretaget med Viidnerne fra Nordre \textit{Finlje}, Nembl: Først Som er det\hspace{1em}\par
12{te} Viidne i \textit{Inderøens} Fogderie\textit{Sneaasens} Præstegield\textit{Lars Olsen Skielbreeden} føed paa Gaarden Skielbreden i Nordre \textit{Finlje}\textit{Sneaasens} Præstegield af bønder-Folck Samme Stædz, 50 aar gammel, gift, har 3 børn, boer og er bonde paa Samme Gaard. ‒\par
Til 1{te} Spørsmaal: Svarer: Dend østerste gaard j Nord-ost fra bøjden \textit{Nordre-Finlje}, ligger fra grændse-Mærcket \textit{Svane} Steenen ad \textit{Jemteland} 11{ve} gamle Miile, det hand har hørt, men Selv har hand icke værit der; Laugrættet meener, at en gammel field-Miil kan vel icke være meere, end 1/2 bøjde Miil; ellers ligger denne bøjd i \textit{Sneaasens} Præstegield\textit{Inderøens} Fogderie, hvor af det er ett \textit{Annex} Sogn, liigesom Søndere \textit{Finlje} og har en \hypertarget{Schn1_34527}{}\textit{Annex} Kiercke Der til hørende og ligger \textit{Norder Finlje} liigesom Søndere \textit{Finlje} 12 field Miile, Som hand meener kan Reignes for 8{te} Nye bøjde-Miile fra \textit{Snaasens} hoved Kiercke. ‒\par
Landskabet af dette Nordre \textit{Finlje} bestaar liigelediis af England, der hvor Gaardene ligge, Smaa Fiske vande, Fielde, og Granskoug i Dalene, Det liidet korn land, de have, er meget frostnemt Saasom de i 5 aar nu icke have faaed et korn til brød, og leeve de af Furru brød deris qvæg Fiskeri og Skytterie; der ere liigeleedes 8 gaarder, hvor 14 bønder boe paa foruden 2 Nye-byggere, Som have Satt Sig Norden for bøjden omtrent 1 1/2 Nye Miil ved dend Nord-væstlige Ende af \textit{Frostviig} vandet; Denne bøjd Gaardjmellem Strækker Sig fra Søer i Nord 1 Nye Miil foruden bem{te} Nye-byggere; breeden af bøjden fra gaarden Skielbren i væster og Gaarden \textit{Qvæljen} i øster kan være en Nye Miil; gaardene ligge paa Siidene af Vandene; hvorpaa følger Skoug og Fielde;\par
Hvor mange Gaarder der ere i Nordre \textit{Finlje}, hvad de heeder og hvorleedis Samt hvor langt fra Grændse-Mærckerne? jtem hvad brug og Næring bønderne her have? der om giiver hand Samme forcklaring, Som forrige 10{de} Viidne af Søndere-\textit{Finlje} ved 3: Spørsmaal. ‒\par
Fiske-Vandene her i Nordre \textit{Lje} ere følgende: Det Syderske Fiske-vand er \textit{Sand} Søen, Strækkend Sig fra Syd-ost i Nord-væst 1 gammel Miil Som hand meener at være 3/4 Nye Miil, breed kan dend være 1/8 Nye Miil meer og mindere; Ved hvis Nordre Siide dend gaard \textit{Hoeland} Ligger.\par
Dend 2: Søe i Nord væst fra \textit{Sand}-Søen er \textit{Lax} Søen Som Stræcker Sig fra Sydost i Nordvæst 1 Nye Miil lang og kan være en Nye l/4 breed; ‒\par
Fra \textit{Sand}-Søen Riinder en liiden Elv, ved Nafn \textit{SandSøeaaen}, 3 Bøsse Skud lang, og ett liidet bøsse Skud breed, denne Elv Naar hun gaar af \textit{Lax} Søen i Nord-væst i \textit{Bratlandsvandet} 1/4 Nye Miil omtrent, og et liidet Bøssskud breed; har dend det Nafn \textit{Lax} Søe-\textit{aaen}; \textit{Bratlands}-vandet Strækker Sig fra Syd-ost i Nord-væst en Nye 1/8 miil Laang, og 2 BøsseSkud breed, og hænge i Nord-væst Sammen med \textit{Schielbreen}-vandet ved et Sund, hvilcket Sund i Nordvæst gaar 3 bøssskud lang, og 1 bøsse-Skud breed; \textit{Schielbreen} vandet Strækker Sig fra øster i væster en Nye l/4 miil lang, 3 bøsse Skud breed;\par
Dend Elv Som af dette \textit{Schielbreens}-vand udfalder, heeder \textit{Sandøela}, og gaar i førstningen i Sydvæst, Siiden Kroget, Som hand meener, i væster jmellem Fieldene tæt ved \textit{Grongs Annex} bøjd i Over-\textit{haldens} Præstegield, paa dends Søndere Siide, efter hands giisning, i dend Store \textit{Namsen} Elv 11: Nye Miile. Paa dend væstre Siide ad \textit{Sneaasen} er ingen fleere Vande. ‒ Paa dend østere Siide ligger først \textit{Qvæ}-Søen, liggendes i Nord-ost fra forbem{te}\textit{Sand} Søen 3/4 Nye Miil, dend Strækker Sig fra Nordvæst i Sydost imod 1: Nye Miil, og er l/4 Nye Miil breed, meer og mindere; J denne \textit{Qvæ}-Søens Nord-væstlige Ende løeber een Elv \textit{Qvæljen}, Som opckSpringer udaf Gied-fieldet 2 Nye Miil, liggendes i væster fra \textit{Qvæ} Søen veed denne Elv ligger de 2{de} gaarder \textit{Leerbaken} og \textit{Qvæljen} paa dend Nordre Siide. ‒\par
Af denne \textit{Qvæ} Søes østere Ende udriinder en \textit{aae}, kaldes \textit{Qvæ-aaen}, i Nordost 2 bøsse Skud lang, og et 1/2 bøsse Skud breed ind i \textit{Muru} vandet; dette \textit{Muru} vand Strækker Sig \hypertarget{Schn1_34758}{} fra Nordvæst i Sydost omtrent 3/4 Nye miil, og er imod 1/8 Ny miil meer og mindere breed; Af dette \textit{Muru}-vand udkommer \textit{Muru} Elv; Som løber i øster en Nye l/4 miil lang, og et liidet bøsse Skud breed j \textit{Heidugeln} Søe; ved denne \textit{Heidugeln} Søe og længere i øster har Viidnet ikke værit, og der fore ikke kan giive nogen Viidere beskeeden. ‒ J for beskreevne vande, j de, Som Riinde i væster, fanges Røe og ørret men i de, Som løebe i øster, desuden harr og giedder; Foruden for opReignede vande ligger l/2 Miil Norden for QvæSøenFrostviig vandet, Som Strækker Sig fra Nordvæst i Syd-ost, Som hand har hørt af \textit{Finnerne}, 4{re} gamble miile, Som man kan Reigne for 3 Nye Miile, 1/4 miil omtrent breed: dog har hand Selv kun værit ved dend Nord væstlige Ende hos Nyebyggerne; J dette \textit{Frostviig}- vandet fanges kun Røe og øret; ‒ omckring og mellem disse Søer er Gran- og nogen biercke-Skoug.\par
Fielde i Nordre-\textit{Finlje} ere følgende:\par
Paa dend Sydvæstlige Siide ligger \textit{Bræns} fieldet, Skillendes \textit{Finljerne} fra \textit{Sneaasens} hovet-Sogn: Det Strækker Sig fra øster i væster ad \textit{Sneaasen} til omtrent 6 nye Mille, og fra Søer i Nord ad \textit{overhalden} til liigesaa langt; Dette field er bart og Skallet uden Skoug, Græs, dog med nogen Maase paa, er Rund oven paa og Flat, dog Stiiger hits og her nogle berg-klimper der af op; Paa dend nord-væstlige Siide er \textit{Gieting} fieldet, Stræckkende Sig fra Søer i Nord, efter hands formeening omtrent 6 Nye Miile lang, og fra øster i væster omtrent 4 Nye Miile breed; Dend Gaard \textit{Skielbreen} ligger ved dette fields Sydvæstlige Siide, og imellem dette og \textit{Bræns} fieldet Riinder dend før beskrevne Elv \textit{Sandøela}; Paa dette \textit{Gieting}fieldet er bare Maasse. ‒\par
J Nord ost 2 l/2 Miil fra \textit{Giedting}-fieldet ligger Qværnberget, Som er en Slet \textit{plain} med jord og Skoug begroed; Qværn berget der af kaldet, at der af Qværn Steene hugges; Dett ligger imellem Qvæ Søen og \textit{Frostviig}-vandet fra Søer i Nord 1/2 mil lang, og meener at være 2 bøsse Skud breed, Saa vidt Som de kan forstaa, under jorden; Væsten for dette Qværn berg ett par bøsse-Skud, Og 2 1/2 miil Nord ost fra \textit{Gieting}fieldet hvor det begynder i Sør og l/2 Nye Miil Nordost fra gaarden Qvæljen ligger \textit{Røeberg}, en berg-klomp, Rund af Sig og Spitz oven paa, et bøsse-Skud over af Størrelse; bart og Skallet er det oven paa, men har Skoug Runten om Sig. ‒\par
J øster fra bøjden ligger \textit{Haarkiølen}, Som forige 10{de} viidne af Søndre \textit{Finlje} har forcklaret ved 2{te} Spørsmaal; Paa dend Sydvæstlige Siide af denne \textit{Haarkiølen} ved Enden opstiiger en berg-klomp, kaldet \textit{Buurs Klompen}, liggendes fra \textit{Sand}-Søen en Nye 1/4 Miil i Syd ost, er Rund og Spitz oven paa, Stoer omtrent et bøsse Skud over bar og Skallet.\par
2 Spørsmaal: Svarer: Sønderst i øster er det field \textit{Haardkiølen}, Norden der for er Skoug Norden der for igien ere de vande \textit{Muru Heidugeln} og fleere til \textit{Svane} Steenen.\par
Til 3: Sp: Svarer: Sønden for dem er Søndere \textit{Finlj} bøjden, før beskreeven; Paa dend Sydvæstlige Siide ligger \textit{Sneaase}-bøjden, Som hand meener at være 8 Nye Miile derfra.\par
J Nord-væst 10 gamble Miile, Som hand meener at være 7 1/2 Nye Miile \textit{Harrang- Annex} bøjd af \textit{overhaldens} Præste gield, hvor fra de ere adskilte ved \textit{Gieting}fieldet; Paa dend Nordre Siide veed hand af ingen Naboer, uden \textit{Lap-Finner}. ‒\hypertarget{Schn1_34976}{}\par
Til det 4 Sp: Svarer: Dend Nærmeste gaard j Syd-ost paa dend Sydøstlige Svenske Siide er \textit{Hillsand}, 20 gamle Miile, Som man omtrent vil Reigne for 15: maalte Miile; ‒\par
Denne Gaard ligger j \textit{Strøms Annex}\textit{Hammerdals} Præstegield, Som hand har hørt at være det Norderste i \textit{Jemteland}; viidere gaarder vidste hand ikke af at Siige. Vej gaar der ikke nogen \textit{ordinair} over til \textit{Jemteland}; og haver hand fornummet at \textit{Jemterne} drage deris baader fra øster til Qværnberget ved Qvæ Søen om Sommeren igiennem Elvene og Vandene, og Naar de have ladet dem med Qværn-Steene, fare de med Strømmen tilbage. Om Viintern er ingen \textit{passage} uden paa Skiie, og kiøres aldriig med hæst og Slæde. Fra Søndere \textit{Findlje} kan fares om, viintern efter Vandene.\par
Til det 5{te} Sp: Svarer: hand veed ikke af andet Lande-Mærcke at Siige end \textit{Svane Steen} Som hand har hørt: Dog har hand ikke Selv værit der, og viidere beskeeden veed hand ikke der om at giive, end at hand har hørt, at det Skal være en stoer steen, ved Siiden af \textit{Svane vandet} Staaende. Viidere i Nord er hand ikke beckiendt: vel har hand hørt, at \textit{Ruud Furru} og \textit{Børge} fieldet Skulle være grændse-Mærcker; Men veed ikke, hvor de ligge.\par
6{te} Sp. Svarer: at det er vel Konge(n)s Alminding.\par
Til 7 Sp: Svarer: De Svenske \textit{Lap-Fiinner} have jndtaget og bruge heele \textit{Haarkiølen} til dends væstere Ende med \textit{Buurs} klimpen indbereignet: Dog haver de for kort tiid Siiden tilladet de Norske \textit{Finner} at Sidde jblandt dennem til fælleds brug, paa dend væstere Ende af \textit{Haarkiølen}. ‒\par
Norden der for have de Svenske bønder gaaet over Lande-Mærcket \textit{Svanesteen} j væster til Qværnberget ved Qvæ-Søen og hændtet Qværnsteene der fra, hvilcket brug de driive endnu: De have og gaaed j forrige tiider indtil \textit{Muru} vandet\textit{incl}: og Fisket: dog nu for kort tiid Siiden entholdet Sig fra bem{te}\textit{Muru} vandet; mens i \textit{Heidugeln}, Som er det nærmeste vand derved og j de øfrige vande østen for \textit{Heidugeln} og væsten for \textit{Svanesteenen} Fiske de endnu.\par
Efter Rættens til Spørgende forcklarede Viidnet, at hand har vel hørt, at bønder fra Søndre \textit{Finlje} for omtrent 10 aar siiden have værit inde paa \textit{Offerdals ordinaire} Lage-Ting for hærrets Rætten, og klaget over \textit{Jemte} bønderne og \textit{Lap-Finnerne}, at de giorde dennem jndtrængsel i Skougene og Fiske-vandene: men hvad Svar der paa er fuldt er Viidnet ikke beckiendt. /: Rætten her om kan giive dend oplysning, at \textit{Hr oberste Emahusen} her om til grændse-\textit{Commissionen} en omstændelig \textit{Relation} har givet :/.\par
Til det 8 Sp: Svarer: Det Samme Som l0{de} viidne af Søndere \textit{Finlje}, dette tilføjendes, at dennem Skeer forfang af \textit{Jemterne} i Qværnberget ved Qvæ Søen, Som tilhører Nordre \textit{Finlje}, og \textit{Jemterne} bruge uden deris tilladelse el: fortieniste.\par
Til 9{de} Sp: har hand Svaret ved 1t{e} Spørsmaal her -\par
Til 10{de} Sp: Svarer: Det Samme som det 10{de} viidne af Søndere \textit{Finlje}. ‒\par
Til 11: Sp: Svarer: J deris bøjd veed de ingen der er kiendt ved Grændserne: men de foreSlaae et par \textit{Lap-Finner}, Som holde til i Nord \textit{Lje} fieldene\textit{Winkel Zacharias}, og \textit{Joen Siursen}, Som di Skulle See at bringe til grændse Maalerne.\hypertarget{Schn1_35222}{}\par
Til 12 Sp: Svarer det Samme, Som Viidnerne i Søndere \textit{Finlje}. ‒ hvorpaa Viidnet blev \textit{dimitteret}\par
det 13{de} Viidne j Sneaasens Præstegield\textit{Jnderøens} fogderie ‒\par
heeder \textit{Hans Bendtsen qvæljen}, føed paa \textit{Øsnaar} i Søndere \textit{Findlje}\textit{Sneaasens} Præstegield, af bønder-forældre, 36 aar gammel, gift, uden børn, boer og er bonde paa \textit{Qvæljen} i \textit{Nordre Finlje} ‒\par
Til 1{te} og øfrige til 12{te} Spørsmaale \textit{inclusive}, Svarer hand det Samme, Som nest forrige viidne af \textit{Nordre Finlje} hvor paa hand blev bort forlovet. ‒\par
og blev dennem og Lændsmanden det Samme forckyndet, Som i \textit{ongdalen} i \textit{Sparboe} gield ved Rættens Slutning.\par
A{o}1742: d: 30 maij blev endnu fremstillet til at viidne een kyndig mand \textit{Jonas Jonsen Totland}, hvilcken, efter hørte Eedens Forcklaring, blev tagen i \textit{Corporlig} Eed og der efter udSagde; Nembl:\par
14{de} Viidne af \textit{Sneaasens} Præstegield\textit{Jnderøens} Fogderie, heeder \textit{Jonas Jonsen Totland}, Fød i Sydere \textit{Finlje} pan Skaalegaard, af bønder-forældre, 41: aar gammel gift, har 5 børn, har boed i Sønder \textit{Findlje} men nu for Fattigdom skyld fløttet fra gaarden, og Sidder til huuse paa \textit{Sverven} i \textit{Sneaasens} hoved Sogn. ‒\par
Til 1{te} 2{det} 3{die} 4{de} og 5{te} Spørsmaal, Svarer hand liigesom det 10 viidne af \textit{Jnderøen}, til leggendes denne Nøjere forcklaring om Grændsens gang, at, dend gaar\par
(1:) Fra \textit{GiæfSøeHatten} til \textit{Jutahatten} og Skal der imellem være 3 gamle- Som man paa een Giisning vil holde for 2 Nye- Mille og er Landskabet der af bare field ‒\par
(2:) Fra \textit{Jutahatten} til \textit{Raavatnet} er 1/2 gammel- Som man vil Reigne for en liiden Nye 1/2 Miil; Landskabet der jmellem er nogen liiden Skoug ved \textit{Juta}-Elven og \textit{Raavatnet} i Dalen, men naar det Stiiger op ad Fieldene, er Det bart; og Som tilforn er melt, at dette \textit{Raavandet} bestaar af 3{de} Nembl: \textit{Nord-Bue- Dølvandet}, Saa Siger hand, at grændse gangen gaar over det mellemste nembl: \textit{Bue-vandet} ‒ for det\par
(3) Fra \textit{Raavatnet} til \textit{Røkviighyllen} 2: gamble- Som man vil holde for 1 1/2 nye Miil.\par
(4) Fra \textit{Røkviighyllen} til \textit{Brunsiø-Flinten} 1: gammel- Som man vil holde for 3/4 Nye Miil ‒\par
(5:) Fra \textit{Brunsiøflinten} til \textit{Hatlingvatnet}, en 1/2 Nye Miil. ‒\par
(6{te}) Fra \textit{Hatlingvandet} til \textit{Legster-Klompen} en Nye 1/2 Miil ‒\par
(7) Fra \textit{Legsterklompen} til dend øster Ende af \textit{Penningkiesene}, 1: god Nye Miil; Om dette Sidste Mærcke giiver hand denne omstændelige Forcklaring at paa \textit{Penningkiesenes} Nord østlige Siide, er der en liiden Vatsdahl med en liiden Skoug j; Denne Dahl Skal Strække Sig i Nord ost omtrent 2 gamble miil eller omtrent 1 1/2 Nye Miil, hend imod \textit{Avinds}-bækken og giøre Skillenet imellem Norge paa dette Stæd og \textit{Jemteland}. ‒\par
(8) Fra denne Vatsdahl Nord ost fra \textit{Penningkiesene} til nærmeste Mærcke \textit{Avinds \hypertarget{Schn1_35517}{} bækken} Reigner hand 3 gamble- Som man vil holde for 2 1/2 Nye Miil; Landskabet her jmellem er fra \textit{Penningkiesene}, 1 miil bare field, det øfrige er Skougdalen. ‒\par
Denne \textit{Avindsbeken} Siiger hand at opkomme fra Smaa Kiønne paa \textit{Haarkiølen} og af forbem{te}Vatsdal, og løeber først j Nord-ost noget imod \textit{Svanesteenen}, Siiden liige i Nord ind j dend Søe \textit{Hejdugeln} 3 gamble- ell: 2 l/2 Nye mill lang. ‒\par
(9:) Fra \textit{Avinds} bæcken er til \textit{Svanesteen} 3 gamble, el: 2 1/2 Nye- Miil; Landskabet der jmellem er bare Skoug. Denne \textit{Svanesteen} beskriiver hand at være Som ett Skiær, liggendes i \textit{Svane}-vandet, bart og Skallet uden Træer paa Sig, rund Langagtig, ikke højere eller større end en koe. Dett \textit{Svanevand} Strækker Sig fra væster i øster En Nye 1/4: miil lang, og 1/8 miil breed, og Fanges derj harr og Siig; Det hænger Sammen paa dend væstlige Siide ved \textit{Sunde} og Elve, med de Søer \textit{Fugel}-Søen, \textit{Heidugeln Muru}-vandet og \textit{Qvæ}-Søen; Paa dend østere Siide hænger dette \textit{Svane} vand Sammen ved Sund med dend Søe \textit{Dragen}, Som hand har hørt at være 5 miile lang og en l/2 miil meer og mindere breed. Længere i Nord er hand ikke beckiendt.\par
Til 6{te} Sp: Svarer; De ere Kongens alminding Dog bruges Fiskevandene af \textit{Finlje}- mændene ‒\par
fra 7{de} til 12 Spørsmaaler hand liigesom 10 viidne svarer.\par
Paa Rættens tilspørsel om hand ikke vil være Vej viiser for grændse Maalerne? Svarede hand, hand vilde gierne, mens hand er een Fattig Mand, Der har gaaet fra Sin Gaard af Armoed og har indtet at Føede Sig, koene og 5 børn med, uden med Sine hænders arbejde hos fremmede; Dersom hand kunde faa underholdning vilde hand gierne gaae, Vejviiser; hvorpaa hadd blev \textit{dimitteret}. og Rætten Sluttet.\par
\centerline{Peter Schnitler. (L. S.) Lars Friderichsen Midiaas. (L. S.) ole andfindsen Midiaas. (L. S.)}\par
\centerline{\textbf{Bielage} øm de 2{de} Smaa \textbf{Annexer: Søndere-} og \textbf{Nordre- Finlje} af \textbf{Sneaasens} gield\textbf{Jnderøens}\textit{Fogderie} ‒}\par
De ligge begge fra deris hoved- nembl: \textit{Sneaasens}-Kiercke 12 field Miile, Sønder \textit{Finlje} i øster, og Norder \textit{Finlje} i Nordost der fra, og fra nærmeste \textit{Annex} Kiercke j \textit{Liits} Præstegield j \textit{Jemteland} 12 miile; der fore er det, at hver af disse bøjder har Sin liiden \textit{Annex} kiercke, af Træe, hvilcken bønderne Selv have beckostet, og Selv, ved liige holde, dend i Sønder \textit{Finlje} Staar paa \textit{Deviigs} gaard, og dend i Norder-\textit{Finlje} paa \textit{Sandviigs} gaard. ‒\par
De 2{de} bøjder Stræcker Sig fra væster i øster 7 Nye Miile, nembl: fra \textit{Brænds}fieldet hvorved de fra \textit{Sneaasen} Skilles til deris \textit{Deviigs} Kiercke 3 og herfra til grændse Skillet ad \textit{Jemteland LegsterKlompen} 4{re} Mille; breede ere de fra Søer i Nord 5 Nye Miile, nembl: \hypertarget{Schn1_35744}{} fra Sønden for \textit{Middags Fieldet} til \textit{Deviigs} Kiercke 2- herfra til \textit{Sandviigs} kiercke 1: og Siiden her fra til Norden om \textit{Frostviig}-vandet 2 Miile.\par
\textit{Deevigs} Kiercke j Sønder \textit{Lje} har A{o}1613 af Fienden værit afbrændt og A{o}1616 igien \textit{Repareret} og i Stand sat. Søndere \textit{Finlje} har værit før og længere bebygged, end Nordre \textit{Finlje}; Thi A{o}1626 først have nogle Fattige Mænd begiivet Sig til Nordre \textit{Finlje}, og igien bebygget den, Som de og A{o}1636 have Søgt og erholdet, da værend Lehns-herre \textit{Oluf Parsbergs} Tilladelse tll at opSætte Sig der en kiercke: Dog skal et par hundrede aar tilforn dette Nord-\textit{Finlje} af jndbyggerne have værit beboet; Thi af \textit{Sneaasens} Kierckes gamble \textit{documenter} Sees, at \textit{Lap Finnerne} Samme tiid skal have ihiel Slaget og udrøddet dem: Som dog af Søndere \textit{Finljes} Mænd skal være bleven hævnet, i det de have Giæstbøden \textit{Finnerne}, og da disse Sadde bag bordet ved Veggen, have bønderne med bordet Klemt \textit{Finnerne} op til veggen, og Saa med deris Langskaftede øxer Slaget dem ihiel. ‒\par
Bønder ere i Sønder-\textit{Finlie} 15 og i Nordre \textit{Finlje} 14 Mænd Som leeve af deris England, hvilcket der skal være fortræffeligt, Fiskerie og Skytterie; Korn have de nu ikke i 5 aar faaet noget at høstet; Soldattere ere her ickke udskreeven, ej heller lader det Sig vel giøre. Thi de ligger for langt bort baade fra kiercke- og \textit{Generals} Munstrings-Platzene. ‒\par
Dersom de Svenske Grændse Maalere Skulle fornye deris \textit{Pretention} paa Søndere- og eendeel af Nordre-\textit{Finlje} hvilcket Skeed er A{o}1649 og 1690, ved de da holdende Grændse \textit{Commissioner}, Saa er et godt \textit{document} af \textit{Sneaassens} Kierckes \textit{Archiv} hvor af dog kun et Tinglyst \textit{vidimeret Copie} i Kiercke Stoelen haves, faaet der beviiser at de 2{de} bøjder ikke under \textit{Jemteland} mens \textit{Sneaasens} Præstegield\textit{Jnderøens} Fogderie haver \textit{Sorteret}; førend \textit{Jemteland} til Sverrig er \textit{Cederet}; hvor om og Nærmere til Hr. Stiftbefalningsmand og Biskopen i \textit{Trondhiem} Skal giøre mine \textit{Reqvisitioner}, at erholde meere beviis. ‒\par
\centerline{Om Dend Svenske \textit{Province}\textbf{Jemteland}.}\par
\textit{Jemteland} ligger østen for følgende Norske Præstegield og bøjdelauger i \textit{Trondhiems} Stift nembl: fra \textit{Helagstøten} i Søer at reigne, østen for (1) Tydalens \textit{Annex} i Selboe Fogderie, (2) Mærrager \textit{Annnex} i \textit{Stiørdalens} Præstegield (3) \textit{Suul} bøjden (4) \textit{Helgaaedalen} i \textit{Werdalens} gield (5) \textit{Ongdalens} bøjd i \textit{Sparbøe} gield 6. \textit{Finljernes Annexer} i \textit{Sneaasens} Præste gield. ‒\par
Det er omgiiven med viide øede fielde paa Dend Søndere- østere- og Nordre Siider, og paa dend væstere Siide adskilt ved en Stor Skoug fra \textit{Medelpadia} og \textit{Helsingland}; Nor man Staar paa det Norske field \textit{Fongen} jmellen \textit{Tydal} og \textit{Mærrager} kan man oversee heele \textit{Jemteland} og finder det Saaleedes \textit{Situeret}, og midt i Landet liigesom en Rund Dahl, hvor en Stor Søe Som af Norske Bønder berettes at være 14 Miile lang og 6 miile breed, kaldet \textit{Storsiø}, omckring hvilcken bønder gaardene Som tættest Sammen ligge; Midt i denne \textit{Storsiø} er en \textit{Øe, Frosøy} kaldet, hvor paa en gammel Skandze, og \textit{obersten} af det \textit{Jemtelandsche} Regiment \textit{Residerer} ‒\par
Og der af er det vel, at i dend \textit{Bromsebroisske} freedz \textit{Tractat} af A{o}1646. i Særde\hypertarget{Schn1_36085}{} lished om \textit{Herdalen} Staar benæfnt, at hvad der af befindes at ligge paa dend Svenske Siide ved fieldene, det afstaaes, hvilcken om-Stændighed ej utryckeligen er indført om \textit{Jemteland}; Thi ellers Skulle Sverrig bare haft Dalen i \textit{Jemteland}, og foruden dend væstlige- paa dend Søndere- Siide fra \textit{Herjedalen}, og paa dend Nordre Siide fra \textit{Angermannia} være udelagt og afskaaren. ‒\par
Præstegield i \textit{Jemteland} er følgend, Som ieg har kundet fornemme, hvor af de 4 første ligge nærmist Norge, fra Nord ad Søer at reigne (1) \textit{Hammerdal} (2) \textit{Liit} (3) \textit{Offerdal}, (4) \textit{undersaager} (5) \textit{Rødøen}, (6) \textit{Sundet} (7) \textit{Brøndfløe} (8) \textit{Refsund} (9) \textit{Raven} (10) \textit{Hals} (11) \textit{Berg} (12) \textit{oviigen}, (13) \textit{Næs} ‒\par
Bønder-gaarder, Som her berettes, Skal der være 1800 foruden Cronens Godz, hvor af bønderne Selv Skal være Selv Ejere; Jeg har fornommet at adskillige \textit{Jemter} Særdeelis qvindfolck i Tydal og \textit{Merrage}, komme over fra \textit{Jemteland}, og Needsette Sig i Norge; hvilcket ikke af de Norske \textit{vice versa} Skeer i \textit{Jemteland}; ‒\par
Skandzer ere i \textit{Jemteland} imod \textit{verdalen}, hvor den alfare vej fra \textit{Due} Skandze over \textit{Suul} fieldet gaa hen, \textit{Due} og \textit{Jerpe} Skandse, hin Skal være gandske, og denne meget forfalden; Paa \textit{Frosøyen} Skal og have værit en liiden Skandze og være nu forfalden; \textit{Kaldstrøms} Skandze imod \textit{Sneaasen} viide Folck icke meere af at Siige. ‒\par
Eet Regiment til Fodz haves i \textit{Jemteland}, hvormed Saaleedis holdes, bonden maa give Soldaten aarlig 6 Kobberdaler, hver reignes at gielde imod en Dansk Sletdaler, og holde Solde Soldatten en Torp eller husmandtz Platz, hvorpaa Soldatten kan føde en koe og nogle Smaa Fæe, med fornøden boeskab; Naar nu Soldatten afgaar, maa bonden Skaffe en anden, eller Selv gaa; Nu y denne Viinter da Regimentet ligger ud j Stockholm, har en \textit{Oberst} fra Regimentet værit paa \textit{Recreutering} at udtage 260 Mand, Som i Stockholm vare bortdøede, og berettes her, at nu alt bønder 50 til 60 aar gamle, Som ej formaaet at Skaffe karl; har maattet Selv gaae ud Soldattere. ‒\par
Det Nye optagne \textit{Liusendalsche} Kaabber-værck j \textit{Herjedalen}, om det ej forbedrer Sig bliver vel needlagt; Det \textit{Handolsche} Nye Kaabber-værck j \textit{Jemteland} har hidindtil kun Været af ringe betydenhed. For omtrent 4 aar Siiden har man østen for Sneaasen paa den østere Siide af \textit{Steen} fieldet i \textit{Jemteland} j \textit{Offerdals} gield optaget et Nytt Sølv værck dog er Siiden 1740 om Sommeren, ikke bleven derpaa arbejdet; aarsagen vil man Siige at være, at \textit{malmen} i Stockholm Prøvet, er flundet for Slet. =\par
Bønderne i \textit{Jemteland} foruden deris Gaarder bruge meget at væve Læret og Drejel, Som de føre til Norge at Selge, med Jern-Smide fang, Som de her forhandle. ‒
\DivII[Mai 30.-juni 2. Fortsatt arbeide i Snåsa]{Mai 30.-juni 2. Fortsatt arbeide i Snåsa}\label{Schn1_36319}\par
Efter dend 30 maij 1742: Da \textit{Examinations} Retten paa \textit{Sneaasen} var Sluttet har ieg til 2 Junj udfærdiget \textit{Bielagene} og følgende \textit{Registre} og udtog af \textit{Protocollen} over alle beskreevne Stæders Nafne, til tieniste for Grændse \textit{Commissionen} og Maalerne Sambt Skreevet \textit{Circulaire} breve til grændse Præsterne af \textit{Sneaasen, Werdalen, Stiørdalen, Selboe,} og \textit{Røraas} at forckynde fra Prædicke Stoelene for Almuene Deris Kongl: Maj{ts} befalning, om grændse Maalingens befordring; Liige Saa Skreevet til Fogderne af \textit{JnderøenVerdal}- og \hypertarget{Schn1_36380}{}\textit{Stør}dalen, Selboe, og Guldalen med \textit{Communication} af Viidnerne og Vej-viiserne Samt grændse Stæderne for at \textit{Commandere} did de vedkommende med Lensmand og Arbeids Folk.
\DivII[Juni 4.-10. Schnitlers tilbakereise til Trondheim]{Juni 4.-10. Schnitlers tilbakereise til Trondheim}\label{Schn1_36400}\label{Schn1_36401} \par 
\begin{longtable}{P{0.7909266409266409\textwidth}P{0.059073359073359075\textwidth}}
 \hline\endfoot\hline\endlastfoot A{o} 1742. d 4 Junij reist fra \textit{Sneaasen} til Kløvgaard i \textit{Stods} gield\tabcellsep 2 Mile\\
den 5{te}\textit{Dito} herfra til \textit{Krogs} gaarden over Land og Vand\tabcellsep 3 Mile\\
den 6{te} herfra til \textit{Strømmen}, hvor maattet bie efter Skydz\tabcellsep 1 1/4 Mil\\
d. 7. til Yter\textit{øen} og \textit{Frosten}\tabcellsep 5 Mile\\
d. 8. til \textit{Stiørdalen}\tabcellsep 1 1/2 Mil\\
d. 9{de}\textit{expederet Capitain des armes} med \textit{Protocollen} til \textit{Røraas}, at levere til \textit{Directeuren}.\\
d, 10{de} givet mig paa Reisen til \textit{Tronheim}.\end{longtable} \par
 \par
{Peter Schnitler.}\hypertarget{Schn1_36531}{}Ekstrakt av Protokollen.
\DivI[Ekstrakt av protokollen:]{Ekstrakt av protokollen:}\label{Schn1_36533}
\DivII[Designasjon over de østligste fjell mot Herjedalen og gamle landemerker mellem Norge og Jemteland]{Designasjon over de østligste fjell mot Herjedalen og gamle landemerker mellem Norge og Jemteland}\label{Schn1_36535}\par
\centerline{\textbf{Designation} over (1) de østligste \textit{Fielde} ad \textit{Herjedalen}, og (2) over de gamle Lande-Merker imellem \textit{Norge} og \textit{Jemteland} Hvoraf Hine med \textit{Norsk}- og disse med \textit{Romersk} Tall oventil ere beteignede fra Søer i Nord, nemlig fra høyere- til venstre Haand at reigne, Samt Forteignelse og Beskrivelse Af hvert Grentze-Merke og derved beliggende Stæder, Fielde, Søer, eller Elve, i mine Anmerkninger derneden under anført, Af Vidnernes \textit{Examinations-Protocoll extrahered} Med Forklaring af hvert Fogderies Vidne, ved hvad Spørsmaal Svaret er at finde, til en Lettelse og Oplysning først for de Kongelige Grendse-Maalere, siden for den \textit{Kongelige}\textit{Norske Grendse-Commission}\textit{Continuered} fra d 28 April 1742, til hvilken tid saadan min \textit{speciell Extract} før er nedgaaet, til d 2 Junij samme 1742{de} aar paa \textit{Sverve} Gaard i \textit{Sneaasens} Sogn.}\hypertarget{Schn1_36598}{}Schnitlers Protokoller I.\par
←⊖ \textbf{Gamle Landemærker mellem Norge og Jemteland.}\label{Schn1_36611} \par 
\begin{longtable}{P{0.3929245283018868\textwidth}P{0.10424528301886793\textwidth}P{0.05613207547169811\textwidth}P{0.052122641509433965\textwidth}P{0.05613207547169811\textwidth}P{0.05613207547169811\textwidth}P{0.04610849056603774\textwidth}P{0.08620283018867925\textwidth}}
 \hline\endfoot\hline\endlastfoot XXVIII.\tabcellsep XXVII.\tabcellsep XXVI.\tabcellsep XXV.\tabcellsep XXIV.\tabcellsep XXIII.\tabcellsep XXII.\tabcellsep XXI.\\
\textit{Svanesteen}. som sidste Merke imod Trhiems Amtes bøygdelauger; hvorpaa \textit{Lap- Finnernes} Tilhold følger i Nord til \textit{Børjefield}, \textit{Trondhiems} Amt endnu tilhørig.\tabcellsep \textit{Penningkiesene}, deraf den østere Ende\tabcellsep \textit{Legsterklompen}\tabcellsep \textit{HatlingVandet}\tabcellsep \textit{BronsiøFlinten}\tabcellsep \textit{RøkviigHyllen}\tabcellsep \textit{Raavatnet}\tabcellsep \textit{Juta Hatten} paa Grubdals Field\end{longtable} \par
 \par
\centerline{pag: 1.}\label{Schn1_36695} \par 
\begin{longtable}{P{0.7220224719101124\textwidth}P{0.024831460674157303\textwidth}P{0.10314606741573033\textwidth}}
 \hline\endfoot\hline\endlastfoot \multicolumn{2}{l}{(5) \textit{Raufield} 3 Vidne i Stiørdalen til Spørsm. 4.}\tabcellsep \centerline{De Østerste Fielde ad \textit{Herjedalen}}\\
\multicolumn{2}{l}{(6) \textit{WaattaField} 3 V. \textit{ibidem} Sp: 4}\\
(7) Field-\textit{BolagenSiø} 3 V. \textit{ibid}. Sp: 4\tabcellsep  NB (1)\\
(8) \textit{Rutten}, et Field 3 V. \textit{ibid}. Sp. 4.\\
\multicolumn{2}{l}{(9) \textit{Gleefield} 3 V. \textit{ibid}. Sp: 4.}\\
(10) \textit{Hydkrogen}, en dal, 3 V. \textit{ibid}. Sp: 4.\tabcellsep NB (2)\\
\multicolumn{2}{l}{(11) \textit{Liusenvola} 3 V. \textit{ibid}. Sp. 4.}\end{longtable} \par
 \par
NB 1) Ved Øyesiun kan vel best erfares, hvor \textit{linien} fra Foden af \textit{Waatta}field til Foden af \textit{Rut}-Fieldet over denne Field-\textit{bolagen} Siø vil tages ‒\par
NB 2) om denne \textit{Hydkrogen} finder at erindre, at som den støder i Øster til de Svenskes eller \textit{Herdalingernes} brugende \textit{Grøndal}, da ialfald \textit{Linien} fra \textit{Gleefield} i Søer til \textit{Liusenvola} i Nord, at forstaae begge fieldenes Østre Foed, skulle gaae over et Støkke af \textit{Grøndal}, saa ramme og udfinde de H{rer} Grendse-Maalere ved deres Nærværelse og Øyesiun vel selv, hvor \textit{linien} liigest og naturligst fra Field til Field kan trekkes paa dette Sted.\hspace{1em}\par
I. (12) \textit{Biscops-Aae} med \textit{Nea} Elv 3 V. \textit{ibid}. Sp: 4. N. 1. og 4. 1{te} Lande-Merke ad Fieldene mod \textit{Jemteland}\hypertarget{Schn1_36865}{}Ekstrakt av Protokollen.\par
←⊖ \textbf{Gamle Landemærker mellem Norge og Jemteland.}\label{Schn1_36878} \par 
\begin{longtable}{P{0.16723826714801443\textwidth}P{0.12427797833935017\textwidth}P{0.07518050541516245\textwidth}P{0.07824909747292419\textwidth}P{0.10586642599277978\textwidth}P{0.04296028880866425\textwidth}P{0.1365523465703971\textwidth}P{0.11967509025270757\textwidth}}
 \hline\endfoot\hline\endlastfoot XX.\tabcellsep XVIIII.\tabcellsep XVIII.\tabcellsep XVII.\tabcellsep XVI.\tabcellsep XV.\tabcellsep XIIII.\tabcellsep XIII.\\
\textit{Giævsiø hatten} en Field klimp hvortil mitigiennem Giæv Siøn Skielnet gaar over dens høyeste Spids\tabcellsep \textit{HoldesHatten}, en berg Klimp, rund og Spids, hvorover Merket gaar.\tabcellsep \textit{Holdesholm} hvorover skielnet gaar.\tabcellsep \textit{Gaundals Wudu} derj Finnebækken Merke.\tabcellsep \textit{Høysetta} en Fieldklimp, dens øverste top er Lande Merke.\tabcellsep \textit{Straadalsfossen}.\tabcellsep \textit{Skiærvandshoug}. en Jord houg med Skoug paa, over dens Spids Grendsen gaar.\tabcellsep \textit{Skiærvandet} en Siø, over dens Mitte \textit{linien} gaar i Nord.\end{longtable} \par
 \label{Schn1_36955} \par 
\begin{longtable}{P{0.4418650793650794\textwidth}P{0.40813492063492063\textwidth}}
 \hline\endfoot\hline\endlastfoot II. \textit{Helags Stødten}, Grendse-skiel imellem \textit{Herdal} og \textit{Jemteland}. 3 V: i Stiørdalen Sp. 3. og 4. N. 1 og 2.\tabcellsep \textit{Helags} Stødten er vel af 2 Vidner hørt at være Skiellemerke mellem Norge og Jemteland: men af de andre benegtes.\end{longtable} \par
 \par
III. \textit{Storsola}, det Høyeste deraf et Lande-Merke mod \textit{Jemteland}. 3 V. \textit{ibid}. Sp. 4. N. 2. og 4., 5 V. i Selbo Sp: 11. / 6 V. ib. Sp: 4 (see min forrige tabel).\par
IV. \textit{Blaahammer} Klimpen, et lande-merke 1 V. i Stiørdal Sp. 5. / 2 V. \textit{ibidem} / 3 V. \textit{ibid}. Sp: 4. N: 3. og 4.\par
\textit{Eina}-Elv 1. V. i Stiørdalen Sp. 5. og \textit{ibidem} No. 3 og N. 6. / 2 V. \textit{ibidem}\par
\textit{Snasa Hougen}, et Field 1 V. i Stiørdal Sp. 5. / 2 V. \textit{ibid}.\par
V. \textit{Einbogen}, et Lande-Merke, 1 V. i Stiørdal Sp. 5 og \textit{ibidem} N. 2. og 3. / 2 V. \textit{ibid}. / 3 V. \textit{ibid}. Sp. 4. N. 4.\par
\textit{Remmen}, et Field, dets østre Foed, 1 V. i Stiørdal Sp. 5 N. 2 / 2 V. \textit{ibid}.\par
\textit{Klutkien}, en Klimp paa Nordre Foed af \textit{Remmens} Field 1 V. i Stiørdalen Sp. 5 N. 2. / 2. V. \textit{ibid}.\par
\textit{Merager}-bøygd og \textit{Annex} 1 V. i Stiørdalen Sp. 1. 2 og 5. N. 2. og 3. \textit{item} N. 7. / Idem Sp. 7. og 9. / 2 V. \textit{ibid}. / Endnu 1. V. i \textit{Werdalen} Sp. 3. / 2. V. \textit{ibid}.\par
\textit{Stor-Teveldalsskoug} 1 V. i Stiørdal Sp. 2. og 3 N. 3. Idem Sp. 5. N. 5. / 2 V. \textit{ibidem}.\par
\textit{Lil-Teveldalsskoug} 1 V i Stiørdal Sp. 4. og 5. N. 2 og 3. / 2. V. \textit{ibidem}\par
\textit{Stuedals}gaarder 1 V. i Stiørdal Sp. 2. og 3. / 2. V. \textit{ibidem}\par
VI. \textit{Langvola}, Lande-Merke, 1 V. i Stiørdal Sp. 2 og 4. \textit{item} 5. N. 3 og 4. \textit{Idem} Sp. 6. og 9. / 2 V. \textit{ibid}. / det høyeste deraf et Lande-Merke 3 V. \textit{ibid}. Sp. 4. N. 5.\hypertarget{Schn1_37180}{}Schnitlers Protokoller I.\par
←⊖ \textbf{Gamle Landemærker mellem Norge og Jemteland.}\label{Schn1_37193} \par 
\begin{longtable}{P{0.08570247933884297\textwidth}P{0.07867768595041322\textwidth}P{0.21074380165289255\textwidth}P{0.09413223140495867\textwidth}P{0.09413223140495867\textwidth}P{0.08570247933884297\textwidth}P{0.08429752066115702\textwidth}P{0.11661157024793388\textwidth}}
 \hline\endfoot\hline\endlastfoot XII.\tabcellsep XI.\tabcellsep X.\tabcellsep IX.\tabcellsep VIII.\tabcellsep VII.\tabcellsep VI.\tabcellsep V.\\
\textit{Finvola} et field mit der over gaar Lande Merket.\tabcellsep \textit{StorSiøSund}\textit{disputeres} af de Svenske\tabcellsep \textit{Kiølhougen}. et field, dets høyeste Mitte er Lande Merke, efter 1. V. i Stiørd. men det Østerste af dets Spitz er Merke efter 1 V. i Werdal.\tabcellsep \textit{HolsiøRuva} et Field dets mittelste Høyde er LandeMerke.\tabcellsep \textit{Kiørkgaals} Field, dens mindste Vestre deel er Norrigs\tabcellsep \textit{Storlie}. en Houg, dens Østre Foed er Landemerke.\tabcellsep \textit{Langvola} Field dets høyeste Mitte er LandeMerke.\tabcellsep \textit{Einbogen}, som er en bogt, eller Vending, Eina Elv giør paa sig ad Øster,\end{longtable} \par
 \par
\textit{Suul}-gaardene 1 V. i Stiørdal Sp. 3. / 2. V. \textit{ibid}. / 1 V. i \textit{Werdalen} Sp. 2. 3. og i Enden \textit{item} Sp. 5. N. 3. og Sp. 9. / 3. V. \textit{ibid}. Sp. 3. / 4. V. \textit{ibidem}.\par
\textit{Eesand-Siø}, og Esna-Elv, 1 V. i Stiørdal Sp. 3. / 2. V. \textit{ibid}.\par
\textit{Fund Siø}, og \textit{Funna}-Elv, 1 V. i Stiørdal Sp. 3. N. 1. \textit{item} Sp. 5 N. 6 / 2 V. \textit{ibid}.\par
\textit{Færen-Siø} med \textit{Forra}-Elv 1 V. i Stiørdal Sp. 3. N. 2. 3. / 2 V. \textit{ibid}. / 1 V. i Werdal Sp. 3. / 2 V. \textit{ibid}.\par
\textit{Fierjen-Siø} 1 V. i Stiørdal Sp. 3. \textit{item} Sp. 5. N. 6. / 2 V. \textit{ibid}.\par
Dens Elv \textit{Kaaberaae} 1 V. i Stiørdal Sp. 3. N. 3. \textit{Idem} Sp. 5. N. 5. og 6. / 2 V. \textit{ibidem}.\par
\textit{Faarbøygden} 1 V. i Stiørdal Sp. 3. N. 3. / 2. V. \textit{ibidem}.\par
\textit{Hegre-Annex} 1 V. i Stiørdal Sp. 3. N. 2. og 3. / 2 V. \textit{ibid}.\par
\textit{Skorra Siø}, med \textit{Skiørra}- eller \textit{Stiørra}-Elv, 1 V. i Stiørdal Sp. 3. N. 3. \textit{item} Sp. 5. N. 5. / 2. V. \textit{ibid}.\par
\textit{Strind}- eller \textit{Tronhiems-Fiord} 1 V. i Stiørdal Sp. 3. N. 3. / 2 V. \textit{ibid}. / 1 V. i \textit{Werdal} Sp. 3.\par
\textit{Holsiø} med \textit{Holsiø}-Aaen, 1 V. i Stiørdal Sp. 3. N. 3. Sp. 5. N. 6. / 2. V. \textit{ibidem}.\par
\textit{Handøls} Gaardene i \textit{Jemteland}, 1 V. i Stiørdalen Sp. 4. og 5. N. 3. / 2 V. \textit{ibidem}.\par
\textit{Wollan} Gaarder i \textit{Jemteland}, 1 V. i Stiørdal Sp. 4. / 2. V. \textit{ibid}. / 1 V. i \textit{Werdal} Sp. 4. og 5. N. 2. / 2. V. \textit{ibid}.\par
\textit{Skalstuen}\textit{ibidem}, 1 V. i Stiørdal Sp. 4. / 2 V. \textit{ibidem} / 1 V. i \textit{Werdal} Sp. 4. og 5. N. 2. / 2. V. \textit{ibid}.\hypertarget{Schn1_37540}{}Ekstrakt av Protokollen.\par
←⊖ \textbf{Østligste Fielde ad Herjedalen.}\label{Schn1_37550} \par 
\begin{longtable}{P{0.06071428571428571\textwidth}P{0.08532818532818533\textwidth}P{0.2888030888030888\textwidth}P{0.14276061776061774\textwidth}P{0.07384169884169885\textwidth}P{0.1361969111969112\textwidth}P{0.03445945945945946\textwidth}P{0.02789575289575289\textwidth}}
 \hline\endfoot\hline\endlastfoot IV.\tabcellsep III.\tabcellsep II.\tabcellsep I (12),\tabcellsep ll.\tabcellsep l0.\tabcellsep 9.\tabcellsep 8.\\
\textit{Blaahammer} en berg klimp.\tabcellsep \textit{Storsola}. det høyeste deraf Landemerke.\tabcellsep \textit{Helags Stødten} Dene Klimp paa \textit{Storfieldet} er Skielnet imellem \textit{Herjedal} og \textit{Jemteland}. See min Anmerkning herover ved dette N{o} herunder ‒\tabcellsep \textit{BiscopsAae} en bæk paa Storfieldet, Sydvest for \textit{Helags Stødten}.\tabcellsep \textit{Liusenvola} Øst for Haftorstødten.\tabcellsep \textit{Hydkroken}. en dal, dens østre Ende; See mit NB ved dette Navn herunder.\tabcellsep \textit{Gleefield}.\tabcellsep \textit{Ruten}.\end{longtable} \par
 \par
VII. \textit{Storlien}, en Houg, et Lande-Merke, 1 V. i Stiørdal Sp. 5. N. 4. og 5. / 2 V. \textit{ibidem}/ dens Østre deel Lande-merke, 3. V. i Stiørdal Sp. 4. N. 6.\par
VIII. \textit{Kiørkgaalsfield}, 1 V. i Stiørdal Sp. 5. N. 4. Et lande-Merke \textit{ibid}. N. 5. / 2 V \textit{ibidem} / 3 V. \textit{ibidem} Sp. 4. N. 7.\par
IX. \textit{Holsiøruva} 1 V. i Stiørdal Sp. 5. N. 5. Et Lande-Merke, \textit{ibid}. N. 6. det høyeste deraf 2 V. \textit{ibid}. / 3 V. \textit{ibid}. Sp. 4 N. 8. / 1 V. i Verdal Sp. 5. N. 1. \textit{Reen-Siø} med \textit{Reen}-Aae i \textit{Jemteland}, 1 V. i Stiørdal, Sp. 5. N. 6. / 2 V. \textit{ibid}.\par
X. \textit{Kiølhougen}, et Field, 1 V. i Stiørdal Sp. 5. N. 6. Lande-Merke; \textit{ibid}. N. 7. \textit{Idem} Sp. 7. og 9. / 2 V. \textit{ibid}. / 3 V. \textit{ibid}. Sp. 4. N. 9. / Østerst derpaa er LandeMerke, 1 V. i Werdal Sp. 5. N. 1. og Sp. 6. / 2 V. \textit{ibid}.\par
\textit{Skalsvandet} i Jemteland 1 V. i Stiørdal Sp. 5. N. 7. / 2 V. \textit{ibid}. / 1 V. i Werdal Sp. 4. og 5. N. 2 / 2. V. \textit{ibid}.\par
Kulkiendal, 1 V. i Stiørdal Sp. 5. N. 7. / 2 V. \textit{ibid}.\par
Stor Siøen, 1 V. i Werdal Sp. 2. og 5. N. 1. og 2.\par
StorSiø-Aaen 1 V. i Werdal Sp. 5. N. 2. Stor-Aaen\textit{Idem} Sp. 4 / 2 V. \textit{ibidem}.\par
XI. \textit{Storsiø-Sund}, et Lande-Merke 1 V. i Stiørdal Sp. 4. / 2 V. \textit{ibid}. / 1 V. i Werdal Sp. 1. 2. 4. og 5. N. 1. 2. \textit{item} Sp. 7. og 9. N. 4. / 2 V. \textit{ibid}. / 3 V. \textit{ibid}. Sp. 5. N. 1.\hypertarget{Schn1_37832}{}Schnitlers Protokoller I.\par
←⊖ \textbf{Østligste Fielde ad Herjedalen.}\label{Schn1_37842} \par 
\begin{longtable}{P{0.1521604938271605\textwidth}P{0.120679012345679\textwidth}P{0.11018518518518518\textwidth}P{0.1259259259259259\textwidth}P{0.11018518518518518\textwidth}P{0.11018518518518518\textwidth}P{0.120679012345679\textwidth}}
 \hline\endfoot\hline\endlastfoot 7.\tabcellsep 6.\tabcellsep 5.\tabcellsep 4.\tabcellsep 3.\tabcellsep 2.\tabcellsep 1.\\
\textit{Field-Bolagen Siø}.\tabcellsep \textit{Waatta Field}\tabcellsep \textit{Rau-Field}.\tabcellsep \textit{BratrieField}.\tabcellsep \textit{Rogen-Siø}.\tabcellsep \textit{Kraltvola}.\tabcellsep \textit{Vonsiøgusta}.\end{longtable} \par
 \par
\textit{Kogsteen}, 1 V. i Werdal Sp. 2. og 7. / 2 V. \textit{ibid}.\par
NB: Ved disse 2 Stæder vil ventelig forefalde \textit{difference}; thi de 8venske have i gl: tider villet haft \textit{Kogsteen} til Skielnet, formodentlig havende den \textit{raison}, at Vandet fra heele Storsiøen rinder i Øster, og \textit{Kogsteen} maa være Een af \textit{Suul}- fieldets Klimper: Men vore Vidner eenstemmig blive ved \textit{Storsiø-Sundet} at være Merket, givende den \textit{raison} at det Søndre Lande-Merke gaar i lige \textit{linie} over det Sund til det Nordre Merke; Mig siunes u-forgribelig, at man kan følge vore Vidnernes \textit{unamine} Vidnesbyrd med \textit{Linien} over StorSiø-Sundet.\par
\textit{Suulfield}, 1 V. i Verdal Sp. 2. / 2 V. \textit{ibid}.\par
\textit{Suul}-Elv, 1 V. \textit{ibid}. Sp. 3 og i Enden / 3 V. \textit{ibid}. / 4 V. \textit{ibid}.\par
\textit{Krog-Siø} og \textit{Kroga}-Elv 1 V. \textit{ibid}. Sp. 2. og 3.\par
\textit{Tromsdals} Elv 1 V. \textit{ibid}. Sp. 3 / 2 V. \textit{ibid}. / 3 V. \textit{ibid}, / 4 V. \textit{ibid}.\par
\textit{Steene}-Skanse. 1 V. \textit{ibid}. Sp. 3. / 2 V. \textit{ibid}. / 3 V. \textit{ibid}. 4 V. \textit{ib:}\par
Skognes-Skandse\textit{ibidem}.\par
\textit{Levangers} Plads og Market 1 V. \textit{ibid}. Sp. 3. og 10. / 2 V. \textit{ibid}.\par
\textit{Skognes}-Fiord 1 V. \textit{ibid}. Sp. 3. / 2 V. \textit{ibid}.\par
\textit{Nybyggere} i \textit{Helgaadalen}, 1 V. \textit{ibid}. Sp. 3. og i Enden / 2 V. \textit{ibid}. 3 V. \textit{ibid}. Sp. 1. 2. og 3. / 4 V. \textit{ibid}. / 1 V. i \textit{Inderøen} Sp. 3. / 2 V. \textit{ibidem}.\par
\textit{Jndsiøn}, 1 V. i Verdal Sp. 3. og i Enden, og Sp. 5. N. 3. / 2 V. \textit{ib:} / 3 V. \textit{ibid}. Sp. 3. / 4 V. \textit{ibid}.\par
\textit{Werens}-Aae, 1 V. i Werdal Sp. 3. / 2 V. \textit{ibid}. / 3 V. \textit{ib}. Sp. 2. og 3. / 4 V. \textit{ibid}.\par
\textit{Weren-Siø}, 1 V. \textit{ibid}. Sp. 3. og 5. N. 6. / 2 V. \textit{ibid}. / 3 V. Sp. 2. og 3. / 4 V. \textit{ibid}.\par
\textit{Wugku} Kirke 1 V. \textit{ibid}. Sp. 3. / 3 V. \textit{ibid}. / 4 V. Sp. 1.\par
\textit{Werdals} Øren 1 V. \textit{ibid}. Sp. 5. og 10. / 2 V. \textit{ibid}.\label{Schn1_38176} \par 
\begin{longtable}{P{0.5828571428571429\textwidth}P{0.2671428571428571\textwidth}}
 \hline\endfoot\hline\endlastfoot Mitstuen i Jemteland 1 V. ibid\tabcellsep Sp. 4. / 2 V. \textit{ibidem} beskreven Gaarder i \textit{Jemtel.}\\
Stalkierstuen\textit{ibidem, ibid}.\\
Sta, \textit{ibid}. ‒ ‒ ‒ \textit{ibid}.\\
Sundet\textit{ibid}. ‒ ‒ \textit{ibid}.\end{longtable} \par
 \par
Dueskandse\textit{ibid}. ‒ ‒ ‒ ‒ \textit{ibid}.\par
\textit{Jerpe}-Skandse\textit{ibid}. ‒ ‒ ‒ \textit{ibid}.\par
Hytten til Liusendals Verk. \textit{ibid}.\par
Reinberg og Tongbøl, Gaarder \textit{ibidem}\hypertarget{Schn1_38271}{}Ekstrakt av Protokollen.\label{Schn1_38273} \par 
\begin{longtable}{P{0.7703125\textwidth}P{0.0796875\textwidth}}
 \hline\endfoot\hline\endlastfoot Graasiø\textit{ibid}. 3 V. i \textit{Werdal} Sp. 4. / 2 V. i \textit{Jnderøen} Sp. 4.\tabcellsep G. i Jemteland.\\
Egn, en Gaard \textit{ibid}. 2 V. i \textit{Jnderøen} Sp. 4.\end{longtable} \par
 \par
XII. \textit{Finvola}, mit derover Lande-Merket 1 V. i Werdal Sp. 5. N. 3. / 2 V. \textit{ibid} 3 V. \textit{ibid}\par
XIII. \textit{Skiærvandet}, mit derover Landemerket 1 V. i \textit{Werdal} Sp. 5. N. 4. og 5. / 2 V. \textit{ibid}. / 3 V. \textit{ibid}. Sp. 1. 2. 5. og 9. / 4 V. ibid. Sp. 1. 2.\par
XIV. \textit{Skiærvandshougen}, Spidsen deraf Lande-Merke, l V. i Werdal Sp. 5. N. 5. og 6. / 2 V. \textit{ibid}. / 3 V. \textit{ibid}. Sp. 5. og 9.\par
XV. \textit{StraadalsFossen}, Landemerke, 1 V. i \textit{Werdal} Sp. 5. N. 6. / 2 V. \textit{ibid}.\par
Nordost herfra en Merke-Steen, 3 V. \textit{ibid}. Sp. 5. \textit{Idem} Sp. 3. 4. / 4 V. \textit{ibid}.\par
Snekkeraas 3 V. i Werdal Sp. 2. / 4. V. \textit{ibid}.\par
Fagerlie Fossen 3 V. i Werdal Sp. 2. / 4 V. \textit{ibid}.\par
Snekkermoen, Storlie, Billings-kiønnen, BillingsField, Billingsdal, og Tveraadalen 3 V. i Werdal Sp. 2. / 4 V. \textit{ibid}.\par
\textit{Ongdalen} i \textit{Skei-Annex}, \textit{Sparboe}-Gield\textit{Jnderøens} Fogderie, 3 V. i \textit{Werdal} Sp. 3. / 4 V. \textit{ibid}. / 1 V. i \textit{Jnderøen} Sp. 1. 2. 3. / 2. V. \textit{ibid}.\par
\textit{Halbaks} Field 3 V. i \textit{Werdal} Sp. 3. / 4 V. \textit{ibid}. / 1 V. i \textit{Jnderøe} Sp. 2. 5. / 2 V. \textit{ibid}.\par
\textit{Skiekkermoe}, en Gaard, 3 V. i \textit{Werdal} Sp. 3. / 4 V. \textit{ibid}.\par
\textit{Manshaugene}, Jemtelands Field, 3 V. i \textit{Werdal} Sp. 3. 4. / 4 V. \textit{ibid}. / 1 V. i \textit{Jnderøen} Sp. 4. / 2 V. \textit{ibid}.\par
\textit{Maak}, en Gaard i \textit{Ongdal}, 3 V. i \textit{Werdal} Sp. 3. / 4 V. \textit{ibid}. / 1 V. i \textit{Jnderøe} Sp. 2. / 2 V. \textit{ibid}.\par
\textit{Skiekeraadalen}, 3 V. i \textit{Werdal} Sp. 3. / 4 V. \textit{ibid}. / 1 V. i \textit{Jnderøe} Sp. 2. / 2 V. \textit{ibid}.\par
\textit{Skieker}-Field 1 V. i \textit{Jnderøe} Sp. 2. / 2 V. \textit{ibid}. / 4 V. \textit{ib:} Sp. 1. / 5 V. \textit{ib:} / 6 V. \textit{ib:}\par
\textit{Skieker-vand}, en Siø, og \textit{Skieker}-Elv, 3 V. i \textit{Werdal} Sp. 3. / 4 V. \textit{ibid}. / 1 V. i \textit{Jnderøen} Sp. 2. / 2 V. \textit{ibidem}.\par
\textit{Ahnin} Siø og \textit{Ahnin}-Elv i \textit{Jemteland}, 3 V. i \textit{Verdal} Sp. 4. / 2 V. i \textit{Jnderøe} Sp. 4.\par
\textit{Kald-Siø}, og \textit{Liit Siø} i \textit{Jemteland}. 3 V. i \textit{Werdal} Sp. 4. / 1 V. i \textit{Jnderøe} Sp. 4. / 2 V, \textit{ibid}.\par
\textit{Storsiø} i \textit{Jemteland}, 3 V. i \textit{Werdal} Sp. 4. / 2 V. i \textit{Jnderøe} Sp. 4.\par
\textit{Offerdals} Gield i \textit{Jemteland} 3 V. i \textit{Werdal} Sp. 4.\par
\textit{Kaldsbøygd} og \textit{Annex} i \textit{Jemteland}, 3 V. i \textit{Werdal} Sp. 4. / 1 V. i \textit{Jnderøe} Sp. 4. / 2 V. \textit{ibid}. / 8 V. \textit{ibid}. / 9 V. \textit{ibid}.\par
\textit{Kulaasen}, en Gaard i \textit{Jemteland}, \textit{ibidem} og\par
XVI. \textit{Høysæta} ‒, et Field- og Lande-merke, 3 V. i \textit{Werdal} Sp. 5. / 1 V. i \textit{Jnderøe} Sp. 1. 2. 5. og \textit{ibidem} N. 6., \textit{item} Sp. 9. / 2 V. \textit{ibid}. / 3 V. \textit{ibid}. / 4 V. \textit{ibid}. Sp. 5. / 5 V. \textit{ibid}. / 6 V. \textit{ibid}. / 8 V. \textit{ib:} og Sp. 9. / 9 V. \textit{ibid}.\hypertarget{Schn1_38781}{}Schnitlers Protokoller I.\par
\textit{Buur}-vandet 1 V. i \textit{Jnderøen} Sp. 5. / 2 V. \textit{ibidem} / 4 Vid. \textit{ibid}. / 6 V. \textit{ibid}. / 8 V. \textit{ibid}. / 9 V. \textit{ibid}.\par
\textit{Aas}-Vand og \textit{Aasvats}-Elven i Norge, 1 V. i \textit{Jnderøen} Sp. 2. / 2. E. \textit{ibid}.\par
\textit{Ongna} Elv i Norge\textit{ibidem}.\par
\textit{Almaas} Søe, \textit{Maga}-Siø og \textit{Maga}-Elv\textit{ibidem}.\par
\textit{Steenkier}, en Norsk Gaard, \textit{Steenkier}- eller \textit{Bedstad}-Fiord, \textit{ibidem}.\par
\textit{Rockdal} i en Norsk Skoug 1 V. i \textit{Jnderøe} Sp. 3. / 2 V. \textit{ibid}. / 4 V. \textit{ibid}. Sp. 1. / 5 V. \textit{ibid}. / 6 V. \textit{ibid}.\par
\textit{Sneaasens} bøygd og Sogn 1 V. i \textit{Jnderøen} Sp. 3. / 2 V. \textit{ibid}. / 4 V. \textit{ibid}. Sp. 1. / 5 V. \textit{ibid}.\par
XVII. \textit{Gaundalswuddu}, Lande-Merke, \textit{item}\textit{Gaundals} Sæter, \textit{Øster-gaundalen} (Svensk) \textit{WesterGaundalen} (Norsk) 3 V. i \textit{Werdal} Sp. 3. / 4 V. \textit{ibid}. / 1 V. i \textit{Jnderøen} Sp. 2. 3. 5. 8. 9. / 2 V. \textit{ibid}. / 3 V. \textit{ibid}. / 4 V. \textit{ibid}. Sp. 1. 2. / 5 V. \textit{ibid}. / 6 V. \textit{ibid}. og Sp. 5. / 7 V. \textit{ib:} Sp. 1. 2. / 8 V. \textit{ib:} od Sp. 5. 6. 9. / 9 V. i \textit{Jnder:}\textit{ibid}. / 10 V. \textit{ibid}. Sp. 1.\par
Gauna-Elv, 1 V. i \textit{Jnderøe} Sp. 2. / 2 V. \textit{ibid}. / 6 V. \textit{ibid}. Sp. 5.\par
Tørøyen, en Siø i \textit{Jemteland}, 1 V. i \textit{Jnderøe} Sp. 2. 4. 5. og \textit{ibid}. N. 4. / 2 V. \textit{ibid}. / 6 V. \textit{ibid}. Sp. 5. / 8 V. \textit{ibid}. Sp. 5. / 9 V. \textit{ibid}.\par
Kald Strømmen, en Elv i Jemtel: 1 V. i \textit{Jnderøe} Sp. 4. / 2 V. \textit{ibid}.\par
KaldStrøms Skandse\textit{ibidem}.\par
XVIII. \textit{Holdes Holmen}, eller \textit{Mitholmen}, et Landemerke, 1 V. i \textit{Jnderøen} Sp. 5. N. 2. og 3. / 2 V. \textit{ibid}. / 3 V. \textit{ibid}. / 8 V. \textit{ibid}. og Sp. 9. / 9 V. \textit{ibid}.\par
\textit{Holdes Siø}, Landemerke med forrige \textit{ibidem}.\par
XIX. \textit{Holdeshatten}, en Bergklimp og et landemerke 1 V. i \textit{Jnderøe} Sp. 5. N. 2. 4. / 2 V. \textit{ibid}. / 3 V. \textit{ibidem}. / 8 V. \textit{ibid}. og Sp. 9. / 9 V. \textit{ibid}.\par
Kildberg Foss, en Elv i \textit{Jemtel:} 1 V· i \textit{Jnderøen} Sp. 5. N. 4 / 2 V. \textit{ibid}.\par
XX. \textit{Giæv-Siø}, et Landemerke, \textit{Giævra}, en Elv og\par
\textit{GiævSiø-Hatten}, Landemerke, 1 V. i \textit{Jnderøen} Sp. 5. N. 4. og 5. / 2 V. \textit{ibid}. / 8 V. \textit{ibid}, / 9 V. \textit{ibid}. og i Enden Præstens \textit{proposition} / 14 V, \textit{ibid}. Sp. 5. N. 1.\par
\textit{Finlierne} i \textit{Sneaasens} Gield 2 V. i \textit{Jnderøen} Sp. 4. / 4 V. \textit{ibid}. Sp. 1. 3. / 5 V. \textit{ibid}. / 6 V. \textit{ibid}. / 7 V. \textit{ibidem} / 8 V. \textit{ibid}. 9 V. \textit{ibid}. og i Enden.\par
\textit{Hyllen}, den Sønderste Gaard paa \textit{Sneaasen}, 4 V. i \textit{Jnderøe} Sp. 1 / 5 V. \textit{ibid}. / 6 V. \textit{ibid}. / 7 V. \textit{ibid}. / 8 V. \textit{ibid}. / 9 V. \textit{ibid}.\par
\textit{Aanson}, den østerste Gaard paa Sneaasen 4 V. i \textit{Jnderøe} Sp. 1 / 5 V. \textit{ibid}. / 6 V. \textit{ibid}. / 7 V. \textit{ibid}. / 8 V. \textit{ibid}. / 9 V. \textit{ibid}.\par
\textit{Hammer}, den vesterste Gaard paa \textit{Sneaasen} 4 V. i \textit{Jnderøe} Sp. 1 / 5 V. \textit{ibid}. / 6 V. \textit{ibid}. / 7 V. \textit{ibid}. / 8 V. \textit{ib:} / 9 V. \textit{ibid}.\par
Dal-Aaen, Elv i \textit{Sneaasen}, 4 V. i \textit{Jnderøe} Sp. 1. / 5 V. \textit{ibid}. / 6 V. \textit{ibid}. / 7 V. \textit{ibid}. / 8 V. \textit{ibid}. / 9 V. \textit{ibidem}.\hypertarget{Schn1_39313}{}Ekstrakt av Protokollen.\par
\textit{Emsa}-Elven, \textit{ibidem}, 4 V. i \textit{Jnderøe} Sp. 1. 2. / 5 V. \textit{ibid}. / 6 V. \textit{ibid}. / 7 V. \textit{ibid}. / 8 V. \textit{ibid}. / 9 V. \textit{ibidem}.\par
\textit{Brændts} Field, 4 V. i \textit{Jnderøe} Sp. 1. / 5 V. \textit{ibid}. / 6 V. \textit{ibid}. / 7 V. \textit{ibid}. / 8 V. \textit{ibid}. og Sp. 3. / 9 V. \textit{ibid}. / 10 V. \textit{ibid}. Sp. 1. / 11 V. \textit{ibid}. / 12 V. \textit{ibidem}.\par
\textit{RaugSiø}, 4 V, i \textit{Jnderøe} Sp. 1. / 5 V. \textit{ibid}. / 6 V. \textit{ibid}. / 7 V. \textit{ibid}. / 8 V. \textit{ibid}. / 9 V. \textit{ibidem}.\par
\textit{Sneaase} Vand 4 V. i \textit{Jnderøc} Sp. 1 / 5 V. \textit{ibid}. / 6 V. \textit{ibid}. / 7 V. \textit{ibid}. / 8 V. / 9 V. \textit{ibid}.\par
\textit{Røbeen} Field i Sneaasen 4 V. i \textit{Jnderøe} Sp. 1 / 5 V. \textit{ibidem} / 6 V. \textit{ibid}. / 7. V. \textit{ibid}. / 8 V. \textit{ibid}. / 9 V. \textit{ibid}.\par
\textit{Gaundals}field\textit{ibid:} 4 V. i \textit{Jnderøen} Sp. 1. 2. / 5 V. \textit{ibid}. / 6 V. \textit{ibid}. / 7 V. \textit{ibid}. / 8 V. \textit{ibid}. og Sp. 5 / 9 V. \textit{ibid}.\par
\textit{Jsmænning} Sæter paa \textit{Sneaasen} 4 V. i \textit{Jnderøen} Sp. 1 / 5 V. \textit{ibid}. 6 V. \textit{ibid}. / 7 V. \textit{ibid}. / 8 V. \textit{ibid}. / 9 V. \textit{ibid}.\par
\textit{Giet}-field paa \textit{Sneaasen} 4 V. i \textit{Jnderøe} Sp. 1. / 5 V. \textit{ibid}. / 6 V. \textit{ibid}. / 7 V. \textit{ibid}. / 8 V. \textit{ibid}. / 9 V. \textit{ibid}.\par
\textit{Emsdalen} i \textit{Sneaasen} 4 V. i \textit{Jnderøe} Sp. 2. / 5 V. \textit{ibid}. / 6 V. \textit{ibid}. / 7 V. \textit{ibid}. / 8 V. \textit{ibid}. / 9 V. \textit{ibid}.\par
\textit{Finn-bæk}, Skielnet i Gaundals Wuddu, 4 V. i \textit{Jnd.} Sp. 2. 5. / 5 V. \textit{ibid}. / 6 V. \textit{ibid}. / 7 V. \textit{ibid}. / 8 V. \textit{ibid}. / 9 V. \textit{ibid}.\par
\textit{Øster-Gaundalen}, Svensk, 6 V. i \textit{Jnderøe} Sp. 5. / 7 V. \textit{ibid}. / 8 V. \textit{ibid}. / 9 V. \textit{ibid}.\par
\textit{Vester-Gaundalen}, Norsk, 6 V. i \textit{Jnderøe} Sp. 5. / 7 V. \textit{ibid}. og i Enden efter det 12 Sp. / 8 V. \textit{ibid}. / 9 V. \textit{ibid}.\par
\textit{Annold-Siø} i \textit{Sneaasen}, 8 V. i \textit{Jnderøe} Sp. 1. 3. / 9 V. \textit{ibidem}.\par
\textit{Søndre-Finlie}, bøygd under \textit{Sneaasen}, 8 V. i \textit{Jnderøe} Sp. 3. / 9 V. \textit{ibid}. / 10 V. ibid. Sp. 1. 2. 5. / 11 V. \textit{ibidem}.\par
\textit{Nordre Finlie} 12 V. i\textit{Jnderøe} Sp. 1.\par
\textit{Annolds} Field 8 V. i \textit{Jnderøe} Sp. 3. / 9 V. \textit{ibidem}.\par
XXI. \textit{Juta Hatten}, Lande-Merke paa Grubdals Field, 10 V. i \textit{Jnderøe} Sp. 1. 2. 4. 5. 7. 12. / 11 V. \textit{ibid}. / 14 V. \textit{ibid} Sp. 5. N. 2.\par
XXII. \textit{Raavatnet}, Lande-Merke, 10 V. i \textit{Jnderøe}, Sp. 1. 2. 5. 7. 12 / 11 V. \textit{ibid}. / 14 Vidn. \textit{ibid}. Sp. 5. N. 2. 3.\par
XXIII. \textit{RøkviigHyllen}, Lande-Merke, 10 V. i \textit{Jnderøen} Sp. 1. 2. 4. 5. / 11 V. \textit{ibid}. / 14 V. \textit{ibid}. Sp. 5. N. 3. 4.\par
XXIV. \textit{BronsiøFlinten}, Lande-merke, \textit{ibidem} og hos 14 V. Sp. 5. N. 4. 5.\par
XXV. \textit{Hatling} Vand, Lande-Merke, med \textit{Hatling} Aaen, \textit{ibid}. og N. 5. 6 hos 14 V. Sp. 5.\par
XXVI. Legster-Klompen, Lande-Merke, \textit{ibid}. og hos 14 V. Sp. 5. N. 6. 7.\hypertarget{Schn1_39792}{}Schnitlers Protokoller I.\par
XXVII. Østre Ende af \textit{Penningkiesene}, \textit{ibid}. og hos 14 V. Sp. 5. N. 7. 8.\par
XXVIII. \textit{Svanesteen}, Lande-Merke, 10 V. i \textit{Jnderøen} Sp. 3. 5. N. 6. / 11 V. \textit{ibid}. / 12 V. Sp. 1. 2. 5. / 13 V. \textit{ibid}. / 14 V. ib: Sp. 5. 9.\par
Holdesovne, en Gaard i Søndre-\textit{Finlie}, 10 V. i \textit{Jnderøe} Sp. 1. 3. / 11 V. \textit{ibid}.\par
\textit{Høland}, en Gaard, 10 V. i Jnderøe Sp. 3.\par
\textit{Holden}, en Siø i Søndre-\textit{Finlie}, 10 V. i \textit{Jnderøe} Sp. 1 / 11 V. \textit{ibidem}.\par
\textit{Holdes} Aae\textit{ibidem}.\par
\textit{Langlingen}, en Siø i Søndre \textit{Finlie}, 10 V. i \textit{Jnderøe} Sp. 1 / 11 V. \textit{ib}.\par
\textit{Langlingen}-Strøm\textit{ibidem}.\par
\textit{Uland, Totland, Devig, Meebøygden, Yngoldal, Skaale} ‒, Gaarder i \textit{Søndre Finlie}, 10 V. i \textit{Jnder.} Sp. 1. 6. / 11 V. \textit{ibid}.\par
\textit{Devigs} Kirke \textit{ibidem}.\par
\textit{Gus}vandet, \textit{ibidem}.\par
\textit{Uulen}, en Siø i Søndre-\textit{Finlie}, 10 V. i \textit{Jnder:} Sp. 1. 2. / 11 V.\par
\textit{Uul}, en Gaard \textit{ibidem}.\par
\textit{Yngola}-Elv. og \textit{Yngoldalen ibidem}.\par
\textit{Middags} Field, 10 D. i \textit{Jnderøe} Sp. 1. 2. 7. / 11 V. \textit{ibid}.\par
\textit{Storblaa}field, \textit{ibid}. Sp. 1.\par
\textit{Kielling Snasen ibidem}.\par
\textit{Sonnen}, en Elv, \textit{ibidem}.\par
\textit{Rengen-Siø, ibidem} Sp. 1. 2.\par
\textit{Reng-Strømmen ibidem}.\par
\textit{Qval-Siø ibid}. Sp. 1. 2. 5.\par
\textit{Torskeide}-Elv\textit{ibid}. Sp. 1. 5.\par
\textit{Hotagen}, en Siø, \textit{ibidem}.\par
\textit{Langvatnet}, en Siø \textit{ibid}. Sp. 1.\par
\textit{Arvats}bækken, \textit{ibid}.\par
\textit{Arve Watten, ibid}. Sp. 1. 2.\par
\textit{Arv-Wats} Aae\textit{ibid}. Sp. 1.\par
\textit{Juta}-Elv, \textit{ibidem}.\par
\textit{Raavatnets} Elv\textit{ibidem}.\par
\textit{Løkringen}, et Vand, \textit{ibid}. \textit{Løkrings} Elv\textit{ibid}. Sp. 5. / 11 V. \textit{ib}.\par
\textit{Røkviig}, en Viig af \textit{Hotagen} Siø, ibid. Sp. 1. 5.\par
\textit{Hotagen Siø, ibid}.\par
\textit{Gunderwatten}, og \textit{Watten}-Aae\textit{ibid}. Sp. 1\par
\textit{Buurs Klompen} 10 V. Sp. 1 / 11 V. ibid. / 12 V. ibid. og Sp. 7.\par
\textit{Liø-Siø} og \textit{Lutra}-Elv 10 og 11 Vidne Sp. 1.\par
\textit{Sand Siø ibid}. Sp. 1. og 3. / 12 V. Sp. 1. \textit{ibid}.\par
\textit{Søndre}- og \textit{Nordre Gusfield}, 10 og 11 V. Sp. 1 \textit{ibid}.\par
\textit{Haarkiølen}, 10 V. i \textit{Jnderøe} Sp. 1. 2. 5. N. 6. / 11 V. \textit{ibid}. / 12 V. Sp. 1.\hypertarget{Schn1_40231}{}Ekstrakt av Protokollen.\par
1) \textit{Høland}, 2) \textit{Sandviigen}, 3) \textit{Næs}. 4) \textit{Taas-aasen}. 5) \textit{Bratland}. 6) \textit{Skielbreen}. 7) \textit{Leerbakken}. 8) \textit{Qvælien}, Gaarder i Nordre \textit{Finlie}, 10 V. i \textit{Jnderøe} Sp. 3. / 11 V. \textit{ibid}. / 12 V. Sp. 1.\par
\textit{Sandviigskirke}, 10 V. i \textit{Jnderøe} Sp. 3. / 11 V. \textit{ibidem}.\par
\textit{Lax-Siø} i Nordre \textit{Finlie}, \textit{ibidem} og 12 V. Sp. 1.\par
\textit{Bratland}-Vandet, \textit{ibidem}, og 13 V. \textit{ibidem}.\par
\textit{Skielbreen}-Vandet\textit{ibidem}.\par
\textit{Qvælie}-Aaen\textit{ibidem}.\par
\textit{Frostwiig} Vandet\textit{ibidem}.\par
\textit{Steen}field i \textit{Jemteland}, 10 V. \textit{ibid}. Sp. 4. / 11 V. \textit{ibidem}.\par
\textit{Tongeraas} og \textit{Landøyen}, Gaarder i \textit{Jemteland ibidem}.\par
\textit{Ansigt}-Field\textit{ibidem}.\par
\textit{Føling Annex} i \textit{Liit} i \textit{Jemteland ibidem}.\par
\textit{Hammerdals} Hoved-Sogn\textit{ibidem}.\par
\textit{Groveln, ibidem} Sp. 5.\par
\textit{Sandøla}-Elv\textit{ibidem} 12 V. \textit{ibidem} Sp. 1. / 13 V. \textit{ibidem}.\par
\textit{Grongs Annex} i Overhalden\textit{ibidem}.\par
\textit{Namsen}-Elv\textit{ibidem}.\par
\textit{Qvæsiø}, i Nordre Finlie 12 V. \textit{ibid}. Sp. 1. 4. / 13 V. \textit{ibidem}.\par
\textit{Gieting}-Field\textit{ibidem} Sp. 1. 3.\par
\textit{Qvæ}-Aaen i Nordre-Finlie\textit{ibid}. Sp. 1.\par
\textit{Muru}-Vand\textit{ibidem} Sp. 1. 2. 7.\par
\textit{Muru}-Elv\textit{ibidem}.\par
\textit{Heidugeln-Siø ibidem}.\par
\textit{Qvernberg ibid}. Sp. 1. 4.\par
\textit{Røberg ibidem}.\par
\textit{Haran-Annex} i Overhalden\textit{ibid}. Sp. 3.\par
\textit{Hilsand}, en Gaard i \textit{Jemteland ibid}. Sp. 4.\par
\textit{Avinds}bækken 14 V. \textit{ibidem} Sp. 5. N. 7. 8. 9.\par
\textit{Svane}-Vandet\textit{ibidem}.\par
\textit{Dragen} Siø\textit{ibid}.\hypertarget{Schn1_40595}{}Schnitlers Protokoller I.
\DivII[Bilag til 1. volumen]{Bilag til 1. volumen}\label{Schn1_40597}\par
Lit: A.\par
\centerline{Høyædle og Welbaarne Hr. Stiftsbefalnings Mand \textbf{Christian Ulrich}\textit{von}\textbf{Nissen}.}\par
Een Stæd Nafnl: \textit{Gaundalen} her i fieldet 6 Miile fra bøjden og 1/2 Fierding fra grændse Skillet mellem her og \textit{Jemtela(n)d}, optoeg min Sl: Fader \textit{Mag{r}Nils Muus} for ongefehr 38 aar siiden til Sætter under Sin gaard \textit{Gran}, efter dend da værende Stiftsbcfalnings Mand Høj \textit{Respective Resolution} og tilladelse; og Som min Fader ved Døeden er afgaaet og bem{te} gaard \textit{Gran} er mig tilfalden, befrøgter ieg af nogen omrørte Sætter bølle at blive \textit{Molissteret}\textit{Suplicerer} derfore i dybeste under underdanighed til deris høj velbaarnhed, Det mig en gunstig \textit{Resolution} maatte meddeelis at Meerermelte Sæter Stæd \textit{Goundahlen} her efter Som i min Sl: Faders tiid maatte følge og tilhøre min Paaboende Gaard \textit{Gran}, Saa ieg desmeere kand være forSickret om af Naboerne ickke at blive \textit{Prejudiceret}, for Saadan deris Høy Welbaarnheds gunstige bevaagenhed, vil dend goede gud dem med ald Tiimelig- og Eviig- Velsignelse belønne hvor om af mig Skal Saa troelig blive bedet, Som jeg Leever i Dybeste \textit{Submission}{Høyædle og Velbaarne H{r} Stiftbefalnings Mands allerydmygeste tienerinde \textit{Benedicht Anna} Sl: \textit{Lieutenant Schouers}}\par
\textit{Snaasen} d. 4: Junj A{o} 1738.\hspace{1em}\par
Dette \textit{Remitteres} til Kongel: May{ts} Foged S{r}\textit{Jens Tausans} ærklæring. \textit{Trondhiem} d. 21{de} Junij I738. ‒ {\textit{C: U: v: Nissen}.}\par
\centerline{Ydmygest Ærklæring.}\par
Ved holdte \textit{Examination} over dette Sæter-stæd \textit{gaundalen} Fiinder ieg ikke andet, end Samme ligger udj hands Kongl: May{ts} alminding, gandske nær mod de Svenske grændser, og er for nogen tiid Siiden optaget af \textit{Mag{r}Nils Muus}, men at hand der til Skulde have, da værende Stiftbefalnings Mandens Tilladelse, Er mig gandske u-beckiendt, ej heller kan Samme findes blandt Fogderietz eller mine formændz Papiirer, i hvor meget ieg der efter har giort mig umage at Søege; Ellers finder jeg udj dend Nye \textit{
     \par \bgroup\itshape 5800010Matriculering \egroup\vskip6pt\par
   } der holtes 1724:, at under dend gaard \textit{Hyllen}, Som Skylder 1: øres Leje, er anført en Sætter j \textit{Gaun\hypertarget{Schn1_40769}{}Bilag A: Erklæring om Gaundalen. dalen}, tillige med et field-Enge-Slette Sammestedz, da gaarden Samme tiid blev forhøjet ‒ 12 mrkl. Som ikke er \textit{approberet}, men aldeelis indtet om, at bemelte \textit{gaundahlen} Skulle være lagt under gaarden \textit{gran}, Som \textit{Reqvirenten} melder, endskiøndt hand og Samme tiid paa liige Maade blev forhøjet. ‒ Dersom deris høj Welbaarnhed nu veed paateignet \textit{Resolution}, her efter vil henlegge denne Platz \textit{Gaundalen} til Sætter under \textit{Gran}, Som forlanges, j fald nogen derudj ej Skulle ville Rydde og boe, udbad ieg mig \textit{Resolution} til underdanigst efterRætning. ‒ \textit{Elnen} d. 16: Januarij Anno 1739. ‒\hspace{1em}\par
\centerline{J: Tausan.}\par
\centerline{\textbf{Resolution.}}\par
Med dend af Fogden holdte \textit{Examination} er det beviist, at dend jndbemelte af \textit{Mag{r}Nils Muus} optagne Sæter, er udj hand kongl: May{ts} alminding. Da Saaa fremt Samme ligger til det bøjdelav hvor under gaarden \textit{Gran Sorterer}, bør det derved Saavel efter \textit{Lovens} 3 bogs 12: Capit: 3 artic: Som dend allernaadigste Skatte Forordnings bydende, uden nogen viidere paa anche her efter Som hidindtil have Sit forblivende; Mens i fald det anderleedis maatte befindes, og dend af \textit{Reqvirent}jndens Fader udj berørte alminding opRøddede Sæter boelig, med tiiden kunde bliive Ett Skatteboel; Da haver Kongl: May{ts} Foged \textit{S{r}Jens Tausan} hands Kongl: May{ts} Rætt og \textit{Jnteresse} derudjnden følgelig dend \textit{alligerede} Lovens bogs og \textit{Capituls} 4{de} art: ved børligen at \textit{observere, Trondhiem} d: 10 martj 1739.\hspace{1em}\par
\centerline{\textit{C: U: v:} Nissen.}\hspace{1em}\par
Denne \textit{Resolution} er mig anviist og \textit{Copie} der af leeveret tilstaar \textit{Anna Sl: Schouers}.\hspace{1em}\par
\centerline{Riigtig \textit{Copie Tesst:}\textit{J: Tausan}.}\hspace{1em}\par
Forestaaende er Riigtig \textit{Copie} af \textit{orginalet}, Som jeg i grændse \textit{Commissions} Rætten har anviist. \textit{Tesst:} jeg, og kan forSickre \textit{Commissionen} at dersom min Sl: Fader \textit{Magist: Nils Muus} ej havde ladet opRødde dette øede stæd ved grændserne, Saa have de Svenske Visseligen vel tilEjgnet Sig det 1: Miil længere i væster til det Stoere \textit{gaundals} fieldet, og var, vel denne Rødning vel een af de aarsager, hvorfore min Sl: fader i Sidste Krigens tiid blev af de Svenske tagen, og ført fangen til Sverrig: \textit{Gran} Gaard paa \textit{Snaasen} d: 29 Maj. 1742.\par
{Benedichta Anna Muus Sl: Løjtnant Schauers.}\hypertarget{Schn1_40976}{}Schnitlers Protokoller I.\par
Lit: B:\par
Velædle og Velbyrdige hr. Major \centerline{Høistærede Velyndere!}\par
Mig er Deres meged \textit{Respective a dato}25. Maij næstleden indhendiged, af indhold; forlangende paa dend Kongl: Andordende \textit{Commissions} Vegne, at eftersee; om iche i Kirche Stoelen nogle gamle beviiser og \textit{Documenter} vare at finde, udinden; (1) naar de 2{de}\textit{Annex-} Bøjder, Søndre og Nordre Findlje først vare blefne bebyggede? (2) Om de med deres itzige Stræckning og Grendse-Maal fra første Arrilds tid hafve ligget, som de nu ligge under \textit{Sneaasens} Præstegield\textit{in Ecclesiasticis} og jnderØens Fogderje og Sorenskrifverje\textit{in Cameralibus} og \textit{judicialibus}, og aldrig under Jemteland, enten noged dets \textit{Pastorat} eller Tinglaug hafve \textit{Sorteret?}\par
Paa Saadan Deres Welædelheds forlangende har jeg \textit{Sneaasens} Kirche-Stoel igiennemseet; Og hvad dend første post angaaer, intet vist derom fundet. ‒ Mends hvad som til beviis udinden dend 2{den} post anbelanger, og kand tiene, har jeg forefundet efterskrefne som her under \textit{verbaliter} anføres.\par
\centerline{\textit{Lit: C.}}\par
{Gunstig, Velbyrdig, Fromme Hr. LandsHerre, og Strenge Ridder H{r}\textit{Oluf Pasberrig}. Gud Allermægtigste Eders Velbyrdigheds Strenghed fra ald skade og ulyche naadeligen Spare og bevare, og med timelig og Ævig Velfærd til Siæl og lif lenge og vel opholde.}\par
Foraarsages vi fattige Bønder her udi Nordfindlje, Eders Velbyrdigheds Strenghed med denne vores ydmygelige \textit{Supplication} at anddrage, at efterdie vi her langt borte udi fielden /: tolf miile fra Bøjden :/ er boesiddende, og for dend lange og viide vej skyld, iche kand komme til at Søge Kirchen og Guds tieniste, som vj gierne vilde; tilmed naar nogen af os ved døden afgaaer, er det umuligt for forberørte onde vey skyld, vores liig til vores rette Hoved-Kirche at kand henføres: Hvorfore vi paa det Underdanigste, af Eders Velb: Strenghed er begiærendes, at os maatte bevilges, at vi en liden træhus eller kirche, her hos os at maatte opbygge, som vi kunde høre Guds tieneste udi, naar Præsten os besøger; Saavelsom ogsaa en Kirchegaard, som vores døde legemer udi kunde begrafves, naar nogen ved døden afgaaer. ‒\par
Jligemaade, at efterdj vores fattige ringe jorder, som vi paaboer, nu nyligen er lagt for Landskyld, og anden Kongl: Rettighed at udgifve, og naar Gud Allernaadigste, vil tage sin velsignelse fra os med Kornens vext; saa maa vi lefve udi stor elendighed, og iche hafver nogen deel at hielpe os med, videre end som dend ringe fisk som vi kand faae her under vores lande ‒ hvorfore vi paa det ydmygeligste er begiærendes, at vi samme vores ringe fiske-vand, under og omkring vores Gaarder liggende, maae allene til brugs nyde og beholde, saa at iche jempter, Finner eller Sønderlids mænd os nogen indpas derpaa giøre; \hypertarget{Schn1_41076}{}Bilag C.: Utskrift av Snaasen Kirkestol for 1636. efterdi vi iche nogen anden synderlig næring eller brug hafver, vores fattige qvinder og Børn med at opholde, og de dog baade j Jempteland og Søndre Findlje hafve fiskevand noch omkring dem Self liggendes, at de iche saa mange miile Veys, os fattige folch til skade, skulle fiske her i de vand omkring Nordre Findlje liggendes er, efterdi vi som for er meldt nu skal udgifve paalagde aarlig Landskyld og Rettighed af de ringe jorder vi hafve. ‒ Vi afventer Eders Velb: Strengheds gode og milde betenchende herom. ‒ Vi formoder et gunstig og mildt Giensvar herom, for hvilched Gud allermægtigste Eders Velb: Strenghed Self rigeligen vil belønne ‒ Samme gode trofaste Gud vi Eders Velb: Strenghed vil hafve erønsket og befalet ‒ Af Nordre Findljed d. 14 Julij Anno 1636.\par
\centerline{Eders Velb: Strengheds underdanigste og pligtvilligste tienere altid.}\par
Ejnar Skieldbreed.Oluf Næss.Rolf Bratland.Mathis Sandvig.Oluf Tosaas.Erich Haaland.\par
\centerline{Giensvar.}\par
Jeg hafver mig med Bispen Raadført om denne første post, andlangende udi Nordliden udi \textit{Sneaasens} Gield, et Korshuus eller \textit{Capell} at lade bygge; Og da befinder vi paa Kongl: Maje(s)tets vores Allernaadigste Herres Vegne og naadigste behaug, Christeligt og tilbørligt at være, at det maa være almuen udi Sønderliden bevilget og tilladt, udi forskrefne Nordliden et Kaarehuus at lade bygge, samt en Kirche-Gaard derhos at lade andordne, paa hvilchen stæd Præsten kand giøre dennem dend tilbørlige tieniste, af en Christen medlidenhed, naar hand dennem samme steds besøgendes vorder, og iche af nogen pligt, naar Almuen det vil hafve, saa og deres afdøde der sammesteds at lade begrafvc, efter tilfaldende leilighed; dog slig \textit{Condition} og forord; ‒ Først at Almuen skal dog efter \textit{Reformatzen} være forpligtet, naar de det kand efterkomme for Guds Vejrlig eller anden forhindring, at søge deres rette Hofved- eller \textit{Annex}-Kirche, udi tilbørlig eller Forordnede tiider ‒ dernæst skal de Self være forpligtede til at bygge og vedligeholde foreskrefne Kaars kirche og kirchegaard, og iche nogen anden dermed at besværgg eller betynge, saa og halfve det udi Præstens minde, med hvis Præstelig tieneste hand dennem beviise skall, og naar Præstens leilighed saaledes er, at hand dennem udi Langsommelig tiid iche besøge kand, kand de lade en af deris middel, som læse kand, tage Postillen, og læse for dennem udi Samme Kaars-huus hver Søndag, og ellers naar fornødenhed det udkræfver. ‒\par
Belangendes deres fiske-vand, som de beklager at jempterne dennem fratrenger; da efterdi deres Foget \textit{Jens Friis} for mig haver berettet, at samme fiske-vand ligger iche ofver en fierding vejs fra deres gaarder udi bemelte Nordljen og de jempter som samme fiskerje bruger, ligger ofver 12 miile vejs derfra; derfore eragter jeg det billigt og ret at være, at de næstliggende fiske-vand blifver de Nordlides Mænd følgagtige, og iche de jempter, hvorom jeg ydermere vil lade min befalling udgaae til jempte-Fogden Erich Christensen at jempterne \hypertarget{Schn1_41170}{}Schnitlers Protokoller I. skall sig med foreskrefne Nordlide Mænds Vatten ej viidere befatte, eller derudi nogen forhindring dennem at giøre. ‒ til vitterlighed under min egen haand ‒ Datum \textit{Trondhiems} gaard d. 3. Augustj Anno 1636.\par
{Oluf Pasberrig. Egen haand.}\hspace{1em}\par
Anno 1637 d. 2. Febr: er denne bref læst paa \textit{Sneeaasen} for menige Almue, samt Findlje Mænd derhos værende.\par
{Lauge Hansen. Tingskrifver egen haand.}\hspace{1em}\par
At dette er rigtig \textit{Copie} af det, jeg til Deres Velædelheds forlangendes beviislighed har udi \textit{Sneaasens} Kirche-Stol kundet finde \textit{testerer}\centerline{\textit{Wenje} Præstegrd. d. 30 Maij 1742.}\centerline{Deres Velædelheds Tienstskyldigste tiener (sign.) P. J. Muus.}\hspace{1em}\par
Derhos sentis og En \textit{Copie Sub litera} A, angaaende Søndere \textit{Finlje} hvorlænge ded af \textit{Snaasens} Prægd. har været \textit{administreret} ‒ hvilchet af de i samme understrøgne \textit{Rader} fornemis ‒ og atter i sin rigtighed efter Kirchestolens \textit{testeres}\centerline{af Welædle Hr. Majors ga[n]dsche tiensthengifne tienere}\hspace{1em}\textit{Wenje} Præste gaard\textit{Anno et die ut supra}.  (sign.) P. J. Muus. \hspace{1em}\par
\centerline{\textit{A Monsieur Mons{r}Petter Schnitler Major d'jnfanterie au service du Roy} a \textit{Svarven}. (Titel utenpaa brevet).}\hypertarget{Schn1_41313}{}Anordning av 1588 om Kirker og Gudstjenester i Snaasen.\par
\textit{Copie. \centerline{Lit: A.}}\par
Vi efterskrevne \textit{Christian Fries} till \textit{Borsebye} Kongelig May{ts} Befalningsmand over \textit{Trundhiems} Lehn, \textit{Mag Hans Mogensøn Superintendent} udi \textit{Trundhiems} Stift\textit{Hans Sivertsøn}, og Hr. \textit{Jens Andersøn residerendes Canicher} udi \textit{Trundhiems} Dumkirke. Giøre vitterligt, at wi efter dend afg. kongelige May{ts} vor naadigste Herres Befalling hafver grandgivelig forfaret, hvorledis \textit{Sneaasens} Præstegield med Kirke tieniste her till hafver været forsørget, og saa nu middel og veie betænket, hvorledis det herefter Gud til ære og Menigheden til gavn, best betienis kunde og skall. Da hafver wi først befundet, at udi forgangne tid haver Menighederne været med stor forsømmelighed betient. Dernæst haver været her till udi samme Giæld 4 Kirker, \textit{Sneaasen, FindlieQvam} og \textit{Ool}, hvilke eftersom De ere vidt fra hverandre beliggende, og for uføre skyld ikke af \textit{Sneaasens} Præst tillhobe till gavns betienis kand, hafve wi derfore nu lagt de 2{de} Kirker \textit{Qvam} og \textit{Ool} till \textit{Stods }Præstegield, som er dernæst hosliggende, og af dend Præst beqvemmeligst og bæst betienis kand / \textit{saa blive nu i Sneaasen Gield alleniste 2{de} Kirker Sneaasen og Findlie, hvilke af een Præst betienis skall; udi Sneaasen skall skie tieniste hvær hælligDag, saa nær som 2 ganger om aaret, og da skal skie tieniste udi Findlie} / Till \textit{Sneaasen} ligger 56 gaarder, derpaa boee 60 Bønder, Till \textit{Findlie} ligger 5 gaarder, og derpaa boer 9 Bønder: Men efterdi de 2{de} Kirker er Præsten fralagt, og Hans rænte er deraf forkorted, hafver wi lagt Hannem till underholdning \textit{Sneaasen} og \textit{Findlies} Kirkes rænter og tiende, og at Hand deraf skall holde viin og brød til Kirkerne, og Almuen haver bevilget selv at ville holde deris Kirker ved magt efterdi de beholde Deris Præst hos sig altid, og faae oftere tieniste. Saa beløber Præstens aarlige Rænte ungefehr udi Tiende Korn 43 tdr. osttiende 2{de} voger udi Landskyld 14 Spand 2 Øre 1 ortug, Hertill offer Skiæpper, som beløber sig henved 12 tdr. Korn med anden Sædvanlig Præstelig rættighed og Rente udi forbemældte Sogner. Kand saais till Præstegaarden 11 tdr. Korn, kand holdis 12 Mælke Kiør, 4 unge fæe, og synis os eftersom forskrevet staaer Menigheden vell at være betiænt, og Præsten af hafve een \textit{Christelig} og god underholdning efter \textit{ordinantzen} og Kongl. May{ts} vilje. Om Fattige Almisse Folk hafver wi saa forordnet, at Præsten med 4 gudfrygtige og fromme Sogne Mænd skulde holde eet Klart Register paa de Almisse Folk derudi Giældet, og skulle de af Menigheden give dem deris Almisse, og ingen anden, thi det aldelis skall være forbøden, at fremmede fattige Folk sig med Betlerie Behielpe, og løbe fra det eene Land til det andet, men hvert Giæld skulle føde sine fattige, i saa maade blive og Guds arme efter Lejligheden forsørget. Till vidnisbyrd at saa er udi sandhed, som forskrevet, trykker wi vore \textit{Zigneter} herneden for\hspace{1em}\par
\textit{Sparboen} d. 25 Febr: 1588.\hspace{1em}\par
L.S.  L.S.  L.S.  L.S.\par
\centerline{Rigtig \textit{Copie test:}}{P: J: Muus.}\par
________
\DivI[2. volumen: 1742, på sommerføret.]{2. volumen: 1742, på sommerføret.}\label{Schn1_41503}\par
\centerline{\textbf{WIDNERS EXAMINATION OVER GRENDSERNE IMELLEM NORGE NORDENFIELDS OG SVERRIGE ANNO 1742 PAA SOMMERFØRET. 2. VOLUMEN.}}\begin{figure}[htbp]
\noindent\par
___________
\caption{\label{Schn1_41519}}\end{figure}
\centerline{2{DET} VOLUMEN AF GRÆNDSE COMMISSIONS PROTOCOLLEN Over de tagne Viidner Begyndt paa Nordre Finlie i Sneaasens Præstegield paa den gaard Sandviigen A{o}1742{ve} d: 25{de} Julij ‒ og Endet dend 19 Sept{r} paa Gaarden Soløe i Gilleschaals PræstegieldSaltens Fogderie i Nordland; Og der fra til Trondhiem dend 12 Novbr. tilbage kommet.}\begin{figure}[htbp]
\noindent\par
_________
\caption{\label{Schn1_41572}}\end{figure}

\DivI[Juni 18.-30. Fra Trondheim til Røros (konferanser) og tilbake til Trondheim]{Juni 18.-30. Fra Trondheim til Røros (konferanser) og tilbake til Trondheim}\label{Schn1_41574}\par
Efter Ankomsten fra Fieldene i \textit{Trondhiem} dend 10 Junij 1742{ve} forefandt \textit{Major}\textit{Schnitler} Hr. \textit{Oberste Rømmelings} Skriivelse af 30 \textit{Maij} næst tilforn, og der af fornam, at Velbemelte H{r}\textit{oberste} fandt forgodt, at \textit{Majoren} til dend 24: Junj, da de Kongl: Norske \textit{Jngenieurs} tænckte at være paa \textit{Røraas}, maatte jndfinde Sig der, at \textit{Conferere} med dennem, og \textit{Røraas} Værckets \textit{Directeur} om ett og andet angaaende Maalings foretagelse; J dets følge begav \textit{Majoren} Sig A{o}1742: d: 18 Junj paa Reisen fra \textit{Trondhiem} til \textit{Røraas}, Raadsloge der med bemelte \textit{officerer}, og kom der fra d: 30 \textit{ejusdem} til \textit{Trondhiem} tilbage.\hspace{1em}
\DivI[Juli 13.-23. Fra Trondheim over Snåsa til Nordli]{Juli 13.-23. Fra Trondheim over Snåsa til Nordli}\label{Schn1_41665}\par
\centerline{A{o}1742 d: 13{de} Julj:}\par
Begav \textit{Majoren} Sig paa Reisen til Vandz og over Land i Nordost til \textit{Sneaasen}, der hvor hand i Vaar havde Endet Sit Sidste Viidne-\textit{Examen}, medhavende forrige \textit{Missionaire Erich Helset}, Som Tolck, og \textit{Capitain des Armes} af Hr. \textit{oberste Emahusens Regiment Peter Jacob Royem} Som fra 16 \textit{Apriil} paa \textit{Røraas} til 2: Junij 1742 paa \textit{Sneaasen} havde ført grændse- \textit{Protocollen}, fremdelis at bruge Som \textit{Actuarium}, paa det at deris Kongl: May{ts} Tieniste for Sorenskriivernes udeblivelses skyld, ej Skulle liide ophold. D: 17 Julj andkom man til \textit{gran} Gaard i \textit{Sneaasens} Hoved Sogn, som fra \textit{Trondhiem} ligger 12: Miile. =\par
D: 18 \textit{ejusdem}\textit{confereret} med Sogne Præsten H{r}\textit{Peter Muus} og \textit{Missionairen} for \textit{Sneaasens} og \textit{Overhaldens Lapfinner} Hr. \textit{Povel Muus} og Tiiden og Stæden, hvor de \textit{Lapfinner} Norden for Nordre \textit{Finlje} indtil \textit{Børje}-field best og beqvæmmeligst Kunde Møede; og er for \hypertarget{Schn1_41797}{}Schnitler reiser til Nordre Finnli gott funden, i Stæden for \textit{Grong} eller \textit{Harran Annexer}, i \textit{overhaldens} Præstegield, Som ligger en 10 til 16 miile fra \textit{Limiten} at beramme Nordre \textit{Finlje} i \textit{Sneaasens} Præstegield til \textit{Examinations}- og Tingstæd, Saasom dette er dend \textit{Annex}-Kiercke, hvor hen Samme \textit{Lap Finner} Søge, og er for dennem Nærmest at gaa til; Endskiøndt det fra \textit{Sneaasens} HovedKiercke ligger i øst-Nordost en 12 field Miile, og didhen er en meget besværlig Vej. \textit{Missionairen Hr. Povel Muus} var og ganske villig Selv at følge \textit{Majoren} didhen, til dend anbefallede \textit{Commissions} biestand og befordring; Formedelst og at de Kongl: Betiendtere, Foged og Soerenskrivere, boe for langt fra Fieldene i Væster ved Fiordene, og desuden i denne tiid paa TingReiserne havde at beskille, Skickede \textit{Missionairen} Strax et bud til \textit{Lap Finnerne}, at moede for \textit{Commissions} Rætten til dend 23 \textit{hujus} og følgende Dage i Nordre \textit{Finlje}. Jmidlertid har \textit{Majoren} Samme 18{de} Julj\textit{Notificeret} Soerenskriver \textit{Rostedt} paa \textit{Stoed}, Tiidens og Stædets berammelse til dette \textit{Examen}: om hand der til kunde indfinde Sig.\hspace{1em}\par
D: 20: næst efter reiste \textit{Majoren} geleidet, foruden Tolcken og \textit{Actuarim}, af \textit{Missionairen} Hr. \textit{Povel Muus}, fra \textit{Sneaasens} Hoved-Kiercke over \textit{Annor}field til griisback-Kiøn{4 1/2 miil}\par
D: 21 over \textit{Stoer}- og \textit{Bugvands}-fieldet og Siiden \textit{Brændtz}-fieldet til Stoerbakken{5 1/2 m:}\par
D: 22: en Søndag naaede man til 1{te} gaard \textit{Tosaasen} i Nordre \textit{Finlje}, hvor vii i\par
Huuse kunde blive Natten over {1: miil}\par
D: 23: anlangede til \textit{Sandviig} ved kiercken {1/2 m: ____________ 11 1/2 Miile.}\par
Jmellem forbemelte Stoer- og Bugvands-fielde er Bugvandet Som enda laae med jis tilfrossen.\hspace{1em}\par
Dend første dag neml: d: 20{de} var da een jidelig Regn, der gick igiennem Skiind- og andre klæder, og blødede Myhr-Vejene Saare meget, og de 2{de} første Nætter maatte man J Skougen under Træerne udeligge. Samme 23 Julj om aftenen andkom til dette \textit{Sandviig}; 2{de} af de Nærmeste \textit{Lap Finner}Zacharias Olsen, Væsten for Giorms-Vandet 3 1/2 Miil i Nord ost herfra liggendes, og Joen Siursen, Nord-væst fra \textit{Liiming}-vandet 3 Miile i Nord her fra \textit{Sandviig} Siddendes; de andre vare langt bort gaaet hen til \textit{Børje}-fieldet; ‒\par
Dend 24 Julj Spurte man dem forud, om de vidste rætte grændze-gangen Norden for \textit{Svanesteenen}: Men Som de vare unge Midaldrende \textit{Finner} og ej kunde giive tilstræckkelig underRætning derom, maatte man udsætte deris \textit{Examination} jndtil Viidere, og giorde Saa atter bud til andre ældre \textit{Lap Finner} længere bort tilholdendes, at møede Rætten i \textit{Harrans Annex} i \textit{overhaldens} Præstegield 9: field Miile i Væst-Nord-væst her fra \textit{Sandviigen} liggendes, til d: 30 Julj eller Saa Snart mueligt der efter; hvor hen og \textit{Missionairen} Hr. \textit{Povel Muus} erbød Sig at følge \textit{Commissionen}.\par
Jmidlertid kaldede man de 2{de} Nyebyggere ved \textit{Frostviig}-vandet Nord-væst i Norre- \textit{Finlie} boendes, til sig, til at viidne, Samt \textit{Hans Qvæljen}, at forcklare ydermeere Sit forrige \hypertarget{Schn1_42110}{}Schnitlers Protokoller II. Viidne; hvor efter man de følgende Dage kunde tilbringe med at Reise i Nordvæst til \textit{Qvælje}- gaard, der fra ett berg at besee grændsens lje, og Siiden derfra over \textit{Gietings}-Fieldet viidere i Nord-væst at fare til \textit{Harran Annex}. ‒\hspace{1em}
\DivI[I Inderøy fogderi: ytterligere 4 vidner.]{I Inderøy fogderi: ytterligere 4 vidner.}\label{Schn1_42129}
\DivII[Juli 25. Rettsmøte på Sandvik i Nordli]{Juli 25. Rettsmøte på Sandvik i Nordli}\label{Schn1_42131}\par
\centerline{1742{ve} Dend 25{de} Julj:}\par
Begyndtes da \textit{Examinations} Rætten paa Gaarden \textit{Sandviig} i Nordre-\textit{Finlie} i \textit{Sneaasens} Præstegield j Overværelse af 2{de} LaugRættes-Mænd \textit{Ole Jonasen} 38 aar og \textit{Peder Arndtsen} 40 aar gl:; Tilstedeværende \textit{Missionairen} Hr. \textit{Povel Muus}, og Tolcken \textit{Erich Helset}; og Siiden Soerenskriiveren ej \textit{Comparerede}, Førte \textit{Capitain des Armes Royem} nu Som før, \textit{Protocollen}. ‒\hspace{1em}\par
Viidner, Som Skulle afhøres, vare (1) \textit{Hans Bendtsen Qvæljen} i Nordre \textit{Finlje}, hvilcken vel før paa \textit{Sverve}-gaard i \textit{Sneaasens} Sogn havde været for Rætten det 13{de} Viidne af \textit{Jnderøens} Fogderie, men nu ved denne lejlighed, da man var i hans bøjd, kunde giive nogen forlangende Forcklaring; (2) \textit{Povel olsen} og (3) \textit{Jon olsen} begge Nyebyggere ved \textit{Frostviig}-Vandet; For dennem blev Eedens forklaring af Lov: bogen oplæst og aflagde de 2{de} Sidst benæfnte deris \textit{Corporlig} Eed; dend 1{te} fra \textit{Qvælie} blef kun erindret, at Siige Sin Sandhed i Kraft af hands forrige aflagde Eed.\hspace{1em}\par
Der efter blev forbenæfnte \textit{Hans Bentsen Qvæljen}, Som i forrige \textit{Act} er det 13 Viidne af \textit{Jnderøens} Fogderie, til spurt:\par
(1) Siiden hand boer i Nord-ost herfra nærmist de Svænske grændzer: om hand icke veed grændze-Mærckerne viidere i Nord fra \textit{Svanesteenen} at reigne?\par
\textit{Resp:} Hand \textit{Refererer} Sig til Sit forrige Svar ved det 5{te} Spørsmaal derom giiven. ‒\par
(2) Siiden i \textit{acten} ved det 12 Viidne 1{te} og 8 Spørsmaale er en udsagn giort om \textit{Qværnberg}, at det ligger ved hands gaard jmellem dem og \textit{Frostviig}-vandet, jtem: at \textit{Jemterne} derj komme de Norske fornær, at de hugge uden deris tilladelse eller fortieniste Qværnsteene der af: Saa maatte Viidnet nu forcklare, om det er paa hands Gaards Grund, at dette qværnberg ligger, og at de \textit{Jemter} benytte Sig der af uden hands tilladelse?\par
\textit{Resp:} Det er paa hands gaards grund at qværnberget ligger, omtrent 1/4 Miil i ost fra hands gaard, hvilcket Qværnberg fra \textit{arrilds} tiid under hands gaard har ligget og endnu ligger. Det er vist, at \textit{Jemterne} komme aarlig aars om høsten med deris baader Til dette Qværnberg ved hands Gaard; 6 a 9 Mand j Tallet; A{o}1729:, da hand tiendte hos hands nu afdøede formand \textit{Lars Jversen}, Saae hand, at hver af Jemterne Gaf hands husbonde 4 §: i peng{r} eller dets værdie derfor, at de maatte bryde Qværnsteene paa hands Gaardz Grund; Dett næste Aar efter nemblig 1730{ve} da hand, (viidnet) efter hands Husbondes død havde faaet denne gaard \textit{Qvælien}, gaf \textit{Jemterne} alle tilsammen hannem liigeleedes Een liiden Qværnsteen i villighed, hvorfore hver af \textit{Jemterne} fyldte deris baader med Qværnsteene, og foer der med af. Men efter dend Tiid have \textit{Jemterne}, enskiønt Viidnet dem der om har Krævet, ej vidst ham nogen villighed derfor. ‒\hypertarget{Schn1_42370}{}15 Vidne i Inderøen Fogderi.\par
(3) Om, hvorvidt \textit{Jemterne} komme ham for nær eller bøjden i hands eller i almindings Fiske-Vande?\par
\textit{Re(s)p:} I det Fiske-Vand \textit{Qvæ} Søen, Som ligger nærmest til- og under hands gaard, have \textit{Jemterne} hidindtil ej Fisket: men i de øfrige Fiske-vande, Som høre bøjde-almindingen til, og ligge væsten for \textit{Svanesteenen}, (grændtze-Skiellet,) nemblig \textit{Murru- Heidugeln- Fugel}- Søens og \textit{Svane}-Vandene, der i gaa \textit{Jemterne} denne Norske bøjd Nordre-Finlje for nær, at de ej Alleene Selv bruge Fiskeriet deri, men og vil hindre de Norske bønder at benytte Sig der af. Fiske Som i de vande Falde, ere Øret, Røe, Giedder og Harre, Dette om Vandenes Fiskerie \textit{attesterer} og de nærværende Mænd ved Rætten.\par
Det Som Viidnet \textit{Hans Qvæljen} om forbenefnte Fiskevande har udsagt. Det Samme forcklarer hand og om nærmist i Nord fra \textit{Qvæ}-Søen 1/2 Miil liggende \textit{Frostviig}-Vand, hvilcket \textit{Jemte}-bønder og de Svenske \textit{Lap-Finner} bruge til deris Fiskerie, og vil icke tillade de Norske bønder at fiske der j, uden paa dend væstere Ende der af. Hvilcket Samme og Mændene ved Rætten beckræftede. Viidere Viidnet ej Vidste og der for blev \textit{dimitteret} ‒ efter at hand i \textit{Commissions} Rætten havde jndleveret en \textit{original} bøxel Sæddel af dend Kongl: Norske Foged Rasmus Sørensen \textit{Høgh} til En af hands formænd paa Gaarden \textit{de dato}19{de} Julji A{o} 1693. udgiiven, hvor af hannem \textit{vidimeret Copie} er tilstillet. ‒ (Sub \textit{Lit}. A \textit{N{o} 1} bielagt).\par
Følger nu det\par
15{de} Viidne af \textit{Jnderøens Fogderie}\textit{Sneaasens} PræstegieldNord \textit{Finljes Annex}\textit{Povel olsen} Føed paa dend Gaard \textit{Leerbaken} i Nordre \textit{Finlies Annex} af Bønder-Forældre, 34 aar gammel, gift har Ett barn, boer og er bonde paa een Nye opRøddes platz ved det vand \textit{Frostviigen}, liggende 1/2 Miil i Nord- og nord-væst fra \textit{Qvæ}Søe-Vandet; Og Som hand med Sin Grande Nyelig for 2 aar Siiden har opRøddet og bebygget denne deris paaboende Platz, Saa kaldes de der af Nyebyggere ved dend Nordvæstlige Ende af \textit{Frostviigen}. Blev der paa af Rætten tilspurt de Samme Spørsmaale, Som er skeed til de nest forrige Viidner af \textit{Jnderøens Fogderie}, og der paa Svarede i Særdelished\par
Til 1{te} Spørsmaal: \textit{Resp:} Deris Nyebyggerplatzer ligge, ligesom gaarden \textit{Qvælien} østerst i denne bøjd Nordre \textit{Finlje} 11{ve} gamle Miile fra Grændse-Mærcket \textit{Svanesteenen}; Hvor om Laugrettet meener at En af dem kan være Saa lang, Som En halv Maalt bøjde Miil; J det øfrige Svarer hand om Gaardenes leje og laudets beskaffenhed lige det Samme, Som det næst forrige 12{te} Viidne: Dog forcklarer hand \textit{Brændts}-fieldet, hvilcket Samme Stædz er beskreeven, at være 6 Nye Miile langt, Saaleedis være at forstaa, at til dette \textit{Brænds}-field maa Reignis, Og der under befattes de øfrige i væster vedliggende Fielde, Nembl: \textit{Annor}-field, \textit{Stoer}-field, \textit{Bugvands}-field og \textit{Brænds}fieldet i Sæhrdeelished Saa Kaldet; Thi dette \textit{Brænds}-field i Sæhrdeelished er kun 2 miile over fra væster i øster; hvilcket og LaugRættet med Viidnet \textit{attesteredt}. Fremdeelis i henseende til gaardene her i Nordre \textit{Finlje}, hvilcket det 12{te} Viidne med det 10 Viidne har Sagt at være 8{te} i Tallet, giiver hand dend forcklaring at foruden de opReignede 8{te}, er der endnu Een Gaard kaldet \textit{Eidet}, Som ligger imellem \textit{Sand}-Søen og \textit{Lax}-Søen paa dend Nordre Siide af \textit{Eides}-Elven, Som Rinder af bem{te}\textit{Sand}-Søe i \textit{Lax}-Søen.\par
Til 2{det} Spørsmaal: \textit{Resp}: Landskabet imellem deris Nye bygde Platzer i øster og \hypertarget{Schn1_42649}{}Schnitlers Protokoller II. Grændze-Mærcket \textit{Svanesteenen} bestaar i de forhen i \textit{Acten} opReignede Vande, nemblig \textit{Frostviig}-vandet, \textit{Heidugeln}, \textit{Fugel} Søen, og \textit{Svane-vandet}, Saa og i Gran- og Furru-Skoug paa begge Siider af disse Vande, og der uden for i Fielde. ‒\par
3{de} Spørsmaal: \textit{Resp}: Sønden for deris Nyebygde Platzer ligger der fra 1/2 Miil dend Gaard \textit{Qvæljen}, 1: miil dend Gaard \textit{Leerbaken}; Landskabet der jmellem er Gran- og Biercke-Skoug med nogle Field-\textit{Ruver}; J dend Stræckning jmellem \textit{Frostviig}-vandet og \textit{Qvæ}-Søen ligger 2{de}\textit{Qværn} berger under gaarden Qvæljen; Hvilcke Bønder, Ligesom de andre af Nordre \textit{Finlje} leeve af deris Eng-land, Fiskerie og Skytterie og Gemeenligen maa holde Sig af Furru-brød Som Sielden der Korn Voxer. ‒\par
Af Qværn berget Kan de jngen brug el: fortieniste giøre Sig, Siiden de boe Saa langt fra de andre Norske bøjder: med mindre \textit{Jemterne} kunde forbydes at bryde der Qværnsteene, da de Norske bønder Kunde dem til \textit{Jemterne} forhandle: Dog er herved at agte, at de qværner ej ere tienlige til hafre, men bare til Byg at Male paa. ‒\par
Landskabet væsten for Nyebyggerne her er Gran- og biercke-Skoug med Fielde, og har jngen bønder til Naboer, førend 8{te} Miile der fra i Væster \textit{Harrans}-bøjd i \textit{overhaldens} Præstegield: Dog holde her jmellem endeel \textit{Lap Finner} til.\par
Landskabet paa dend Nordre Siide er ligeleedes Gran- og biercke-Skoug med fielde i, og boer jngen bønder der fra i Nord, Men bare \textit{Lap Finner} hits og her adspreed.\par
Til 4{de} Spørsmaal: Svarer det Samme Som 12 Viidne.\par
Til 5{te} Spørsmaal: \textit{Resp}: Det ‒ (1) Lande-Mærcke, Som hand har hørt fra Sin barndoms alder af mange Folck, er \textit{Brunsiøflinten}; det (2){det}\textit{Hatlingvandet}, det (3) \textit{Lægsterklompen}, det (4) \textit{Penningkiesene}, det (5) \textit{Svanesteenen}, det (6) \textit{Rundfurru}; og det (7) Dend østere Ende af \textit{Børje}-Fieldet;\par
Hvilcket \textit{Børje}-Field hand af \textit{Finnerne} har hørt at Stræcke Sig fra øster i væster; hvilcke 7: benæfnte Stæder allesammen hand igientager, fra Barndom af at have hørt, at være Skille-Mœrcker jmellem Norrig og Sværrig paa denne Kandt. ‒\par
Til 6{te} Spørsmaal: \textit{Resp}: De opReignede grænse-Stæder ligge i Kongens Alminding: dog har hand hørt, at dend Gaard i \textit{Søndre FinlieMeebøjden}, bruger \textit{Hatlingvandet} til Fiskerie og \textit{Bæver}-fangst. ‒\par
Til 7{de} Spørsmaal: \textit{Resp: Jemte}-Bønderne gaae jnden for Grændze Skiellet \textit{Svanesteenen} jndtil Qværnberget jmellem \textit{Qvæ}-Søen og \textit{Frostviigen}, og bruge der deris qværnhugster; j Fiske-vandene gaae de indtil \textit{Murru}-vandet\textit{inclusive} og Fiske; de Svenske \textit{Lap Finner} Norden derfra tilEjgne Sig Skoug, og Fieldene til østen for \textit{Frostviigvandet}, og Fiskeriet Selv i dette Frostviigvandet, (hvor disse Nyebyggere boesidde), foregiivendes, at de derfor maa yde \textit{Taxa} til dend Svenske Crone; hvilcket alt Skeer de Norske bønder fornær. Liigesaa Veed Vidnet, at \textit{Jemterne} og de Svænske \textit{Lap-Finner} Fiske i \textit{Giorms\hypertarget{Schn1_42899}{}Om de første Rydningsplasser i nordre Finnli. vandet}, Som ligger kun 1/4 Miil liige i Nord fra \textit{Frostviig-vandet}: Endskiøndt hand Veed, at en Norsk \textit{Lap Fin Ole Nilsen} tilforn af dend Kongl: Norske Foged har bøxlet Samme \textit{Giorms}-vandet;\par
Dette tillegger Viidnet til oplysning angaaendes \textit{Svanesteenen:} jmellem de østligste gaarder i Nordre \textit{Finlje} og dend Første nærmeste Gaard i \textit{JemtelandStrøms AanexHilsand}, er, Som melt, 20 gamle Miile; j dette mellemRom ligger \textit{Svanesteenen} 2: Miile længere bort paa den østere kandt, nemblig Saa meget nærmere dend bemelte Svenske gaard \textit{Hilsand}. ‒\par
Til 8{de} Spørsmaal: \textit{Resp:} Det Samme som 10: og 12{te} Viidne, til dette Spørsmaal. ‒\par
Til 9{de} Spørsmaal: \textit{Resp:} Liigesom 12{te} Viidne til 1{te} Spørsmaal, nemblig at de østerste gaarder j Nordre \textit{Finlje} ligge 11: gamble Miile fra Grændze-Mærcket \textit{Svanesteenen}. ‒\par
Til 10{de} Spørsmaal: \textit{Resp:} Som 10 Viidne ved dette Spørsmaal. ‒\par
11{te} Spørsmaal: \textit{Resp:} Liigesom 12{te} Viidne ved dette 11{te} Spørsmaal, Tilleggendes dend 3{die}\textit{Fin Brede Thomes} Siddendes paa \textit{Nommel}-Fieldet, 1/4 miil Norden for \textit{Frostviig}- vandet hvilcken hand Meener at være dend Kyndigste. ‒\par
Til 12{te} Spørsmaal: Siger dett Samme som 10 viidne her ved Spørsmaalet og der paa blev hand \textit{dimitteret}. ‒\hspace{1em}\par
16{de} Viidne af \textit{Jnderøens Fogderie}\textit{Sneaasens} Præstegield, Nord-\textit{Finlies Annex}, \textit{Joen olsen} Føed paa dend Gaard \textit{Leerbaken} i Nordre \textit{Finljes Annex} af Bønder Forældre, 28 aar gammel, gift, har Ett barn, boer og er bonde paa en Nye opRøddet Platz Ved det Vand \textit{Frostviigen} liggende 1/2 Miil i Nord- og Nord-væst fra Qvæ-Søe-Vandet, med Viidere omstændigheder Som hands Naboe paa gaarden, det Næst forrige 15{de} Viidne har forcklaret. ‒\par
Fra 1{te} til 12{te} Spørsmaale: Svarer i alle \textit{Poster} liige det Samme, Som næst forrige 15{de} Viidne udsagt haver; og Saa blev denne \textit{Dimitteret}. Og Rætten paa dette Stæd Sluttet. ‒ Og af LaugRættet underSkreevet og forSeiglet.\hspace{1em}\par
\centerline{(sign.) Peter Schnitler. (L. S.)}Ole Jonason (L. S.) Peder arndtsen (L. S.)  \hspace{1em}
\DivII[Bilag om Finnliene]{Bilag om Finnliene}\label{Schn1_43122}\par
\centerline{Tillæg Til \textit{Bielage} Over \textit{Søndre-} og \textit{Nordre Finlier} j \textit{Sneaasens} Præstegield, oventilføjed efter det 14{de} Viidne af \textit{Jnderøens} Fogderie.}\par
Det er 4{re} Husfædres Alder (nemblig A{o}1626 Som ovenmeldt) Siden dend første Mand opRyddede og bygde dend 1{te} Gaard i Nordre \textit{Finlie}\textit{Bratland;} Og er det 3 husfædres Alder, Siiden den 2{den} Gaard \textit{Skielbreen} her Er bleven bygget. At nu dette Nordre \textit{Finlje} tilforn i ældgamle tiider af bønder-Folck maae have været beboet, men ved dend \textit{fameuse} Sorte død Sluttelig ødelagt: dertil haves Visse Kiendetegn. Den nu leevende bonde paa \textit{Tosaasens} gaard i dette Nordre-\textit{Finlje} hans Far-fader, har været dend, der først har bygget forbemelte \textit{Skielbreens} gaard; Hvilcken der hand needhug et gammelt 2-fafne tyck gran\hypertarget{Schn1_43205}{}Schnitlers protokoller II. Træe, for at Rødde en Tomt eller Toft til hans huusers bygning, under dets Roed har fundet grundvolden af en gammel Skorsteen, hvor af Steenene enda Saaes, ordentlig SammenMurede og Sortbrændte;\par
Paa dend Gaard \textit{Næss} Sønden ved \textit{Lags} Søen i Samme Nordre \textit{Finlje} et par hundrede Skrit østen for huus bygningerne Sees endnu \textit{rudera} af en gammel Rund-Muuret Kielder need i Jorden, Som en liiden Stue viid. ‒\par
Paa gaarden \textit{Tosaasen} Norden ved \textit{Lags} Søen i dends Sæterboelig ved bemelte Søes Nordre breede, findes endnu gammel Jern-Slag, og der ved liige som en huule i jorden, et tegn til en gammel Jern-Smelte-grube, hvor fordum jern af Myr-jord maa have været brændt. At nu dette Nordre-\textit{Finlje} maa have været Folck-riigt, beviises der af: Paa halvVejen jmellem \textit{Sneaasens} hoved Sogn og dette Nordrelie paa et field \textit{Stoer}-field icke langt fra Vejens gang Seer man en langagtig næsten Fiirckandtet graaberg-Steen, fladagtig oven paa, omtrent 3 allen lang 2 allen Tyck og 1 1/2 alen breed, af Mennisker opSatt paa 4 KoppelSteene hver Saa Stoer som et Menniske-hoevet, hvilcke ordentlig under Steen-Skivens 4 hiørner yderst paa Kandten ere lagde, hvor paa den Flade Steen Staar. ‒ Jndbyggerne viide ej; til hvad Ende den Steen did maa være opreist, dog viises den, af dennem til de Reisende som en Mærckværdighed af dend Siilde Alderdom, Sluttendes, at det enten har været et Alter-Stæd, hvor ofringer i hedningers tiid ere holden, eller et Tegn til et besynderligt FeldSlag, thi at det Skulle betyde halfvejen just jmellem \textit{Sneaasen} og Nordre-\textit{Finlje}, der til Synis arbejdet forstoert at være lagt paa denne Tunge Steens opReisning og Retning oven over de 4: Koppel-Steene.\par
Jndbyggerne her i \textit{Finljerne} er i kriigens tiid frii for Skatt, og i Freedz tiid betale kun halv Skatt, for dend Visse tieniste de giøre i Kriigs Tiider. ‒\par
Om Jndbyggernes Næring er i \textit{Acten} af \textit{Jnderøens} Viidner fra N{o} 10 til Enden talt; Af Skougen kan de ickke giøre Sig nogen Nytte, at forstaae til bielcke- eller Saug-brug. Thi foruden at disse bøjder ligge forlangt fra Søe-Siiden, og her ingen Elv er beqvæm, til at føere Tømmeret i Væster, er Skougen Knastret og Slet, Saa den ej tienner til andet end husbygning. ‒\par
Furru-Skoug have de ickke paa en 7 a 8 field-Miile nær, hvor fore de liide Stoer Møje, naar de i Haa-bollen, før Høeaandens tiid Saa langt fra maae hente deris Furrubarck, til at lave brød til der af. ‒\par
Jndbyggerne Staae Sig best, naar gud lader for dennem Kornet blive Moed, (hvilcket nu i 5 aar har feilet) og naar \textit{Lap Finnerne} ere i Velmagt, thi da bruge de Som boe \textit{Finnerne} Saa nær, med dennem en liiden handel, i det de bytte Vadmel, Toback og brændeviin med \textit{Finnerne} mod Reen- og biørn-Skiind, eller mod andre Vahrer, Som \textit{Finnerne Fabriqvere}, nemblig \textit{Fin}-Mudder, Støfler, Skoe, vanter, punger \textit{etc:} hvilcke bønderne til Kiøbstæderne bringe; Men denne \textit{Finnernes} vælmagt har for meere, end 20. aar Siiden Mærckelig aftaget; Og her til er aarsagen denne: Om viinteren 1719{ten} Skeede det \textit{Fatale} Neederlag eller Dødz-fald af den Svenske-Finske \textit{Armee} af Sult og Kuld i de \textit{Tydalsche} Fielde; Hvor over om Sommeren der efter ulvene Søgte Stærck til de døde liig, og da de den gang bleve vandte til Fieldene, hvor man tilforn Sielden havde fornommet dem, have de faaet \hypertarget{Schn1_43333}{}Retten sættes paa Gaarden Skielbreen. lugten og Smagen paa Reen-Dyerene, med hvilcke de Siiden dend tiid aarlig Aars Saa Stærck have \textit{Grasseret}, at en \textit{Fin} i ett Aar kan have mistet 10.20. à 50. Dyer, og \textit{Lap-Finnerne} derved ere udarmede, Saa at af en 50 \textit{familier} af \textit{Sneaasens} og \textit{Overhaldens Finner} de fleeste i denne tiid ere bleven Tiggere, og maae gaae paa bøiden, bønderne til besvær. ‒ Da som baade Kornet Slaaer feil, og \textit{Finnerne} have liidet at handle med, Saa er ved denne Tiid Jndbyggernis tilstand Slett; Ellers Siiden her i Nordre \textit{Finlie} Landskabet er viidt af begreeb, og beqvæm til rødning, Saa kan med tiiden Vændtes, at det meere Vil blive \textit{peupleret}. ‒ Sluttelig da dette Nordrelie er dend Sidste Norske bøjd der ligger Saa nær \textit{Jemteland}, tillegger ieg dette om deris \textit{militaire Etats} i \textit{Jemteland} at foruden det Ene \textit{Regiment} til fodz, hvorom i bielaget om \textit{Jemteland} efter 30 maij 1742: i 1{te}\textit{volumen} tilforn er melt, er der og Ett \textit{Compagnie} Ryttere, hvilcket nu i denne Vaar liigeleedes er \textit{udmarcheret} østad til \textit{Herrensand} ved dend \textit{Bottniske} Søe, der at holde \textit{Postering}, i fall \textit{Russerne} med deris \textit{galleier} Skulle, Som i forrige Kriig Skeede, komme op i dend \textit{Bottniske-Søe}, og foruroelige Kusterne i \textit{wester Bottnien}. ‒\hspace{1em}
\DivII[Juli 26. Fra Sandvik til Skjelbrei i Nordli]{Juli 26. Fra Sandvik til Skjelbrei i Nordli}\label{Schn1_43432}\par
A{o}1742{ve} dend 26 Julij: Reiste med \textit{Missionairen} H{r}\textit{Povel Muus} og Tolcken \textit{Erich Helset} til dend østligste gaard i Nordre \textit{Finlje qvælien} 1: miil i Nord ost fra \textit{Sandviig} liggendes, hvorfra vii Steege op paa ett derved værende bierg, og toege i øyesyen \textit{Haarkølen}, ved hvis Væstere Ende Saaes en Kløft eller Skare øst efter i Fieldet, Samt de der fra liggende \textit{Buursklimper}, liige leedes de Fiske-Vande \textit{Qvæ}-Søe, \textit{Murru-} Som med \textit{Heidugeln} og \textit{Fugel}-Søen løeber i øster i \textit{Svanevandet}, Samt \textit{ornes}-field Nord østlig fra \textit{Haarkølens} væstere Ende liggendes. ‒\par
Dend 27{de} nest efter Foer man i væster igiennem \textit{Lags}-Søen{1/4: Miil.} til \textit{Sandøls} Elvens udløb, der foere man over et Eid {1/4 Miil.} og Siiden igiennem \textit{Bratlandts}- vandet i væster {1/8: Miil.} Der efter gienge vii over et liidet Eid, og komme igiennem det \textit{Lille}-Vand i \textit{Skielbreens}-vandet, derved paa dend Nordre Siide \textit{Skielbreen}-gaarden ligger {3/8 Miil. _______} hvor vii gienge op. ‒ {= 1 Miil.}\hspace{1em}\par
Saasom man havde hørt, at der vare Mænd paa gaarden, hvilcke Skulle viide at giive nogen beskeeden om grændze-Mærckerne.\hspace{1em}
\DivII[Juli 27. Rettsmøte på Skjelbrei]{Juli 27. Rettsmøte på Skjelbrei}\label{Schn1_43564}\par
Samme 27{de} Julij Satte man da Rætten paa gaarden \textit{Skielbreen} i Nordre-\textit{Finlje}, overværendes de 2{de} LaugRættesMænd \textit{Lars olsen Skielbreen} 51 aar og \textit{Hans Carlsen Næss}. 34 aar gammel; \textit{Missionairen} H{r}\textit{Povel Muus}, og Tolcken \textit{Erich Helset} var ved Rætten overværendes, og \textit{Capitain des armes Røyem}, i Soerenskriiverens fraværelse, som \textit{Actuarius} førte \textit{Protocollen}. ‒\par
Viidnerne vare paa dette Stæd 1{t}\textit{Ole Joensen} og for det 2{det}\textit{Friderich Pedersen Skielbreen}; For hvilcke Eedens forcklaring af \textit{Loven} lydeligen blev læst, og der paa Eeden af dennem aflagdt\hspace{1em}\par
17{de} Viidne af \textit{Jnderøens} Fogderie\textit{Sneaasens} PræstegieldNordre \textit{Finlies Annex}, heeder \textit{Ole Joensen Skieldbreen}, er føed i \textit{Jemteland Liits} Præstegield, af bønder Forældre, har været \hypertarget{Schn1_43671}{}Schnitlers Protokoller II. her i Norge Siiden hand var 4: aar gammel, er nu 77 aar gammel, gift, har 4: børn, har været bonde her paa gaarden \textit{Skielbreen}, mens opladt dend for Sin Søn, og har nu Vilckaar der paa. Efter som denne Gaards og Landets beskaffenhed før er beskreeven; Saa Spørges Viidnet om\par
(1) hvilcke hand kiender eller har hørt at være Lande-Mærker jmellem Norge og Sverrig paa denne Kandt, Sønden i fra at reigne?\par
\textit{Resp:} Hand har hørt at \centerline{\textit{Brunsiøflinten, Penningkiesene,} ‒ \textit{Svanesteenen,} og \textit{Rundfurru} ‒,} at være Lande-Mærcker, og det har hand hørt af hans Fader. \textit{Brunsiøflinten} har hand værit Saa nær hos at hand har Seed den, men der paa har hand icke været; Ved \textit{Svanesteenen} har hand været, men ickke ved \textit{Penningkiesene} eller \textit{Rundfurru}, ej heller veed hand, hvordan denne \textit{Rundfurru} er Skabt, eller hvor dend ligger. Viidere i Nord var Viidnet ej beckiendt. Hvorpaa hand blev \textit{dimitteret}. ‒\hspace{1em}\par
18{de} Viidne af \textit{Jnderøens} Fogderie\textit{Sneaasens} PræstegieldNordre \textit{Finlies Annex} er \textit{Friderich Pedersen Skielbreen}, Føed paa gaarden \textit{Skielbreen} af bønder Forældre 42: aar gammel, gift, har 5: børn, beboer og bruger dend gaard \textit{Skielbreen}; Blev liigeleedis Som nest forrige viidne kun tilspurdt: om hvad grændtze-Mærcker hand vidste eller havde hørt, at være jmellem Norge og Sverrig paa denne Kandt? Hvortil han Svarede: Hand hørt at \centerline{\textit{Brunsiøflinten, PenningkieseneSvanesteenen,} og \textit{Rundfurru} ere Lande Mærcker.}\par
Selv har hand icke værit der, men hand har hørt det af andre; Og som hand icke længere i Nord var beckiendt, blev hand \textit{dimitteret}. og Rætten paa dette Stæd Sluttet.\hspace{1em}\par
Lars olsen Skielbreen. (L. S.) Hans karlsen næs. (L. S.) \centerline{Peter Schnitler (L. S.)}\hspace{1em}\par
Dend 28{de} Julj Skreevet til bonden i Nordre \textit{Finlie} paa \textit{Næss}\textit{Anders Jørgensen}, at hand i Lensmandens Stæd, Saasnart hand fornam de Kongel: Norske \textit{Jngenieurers} Komme paa grændserne, Som meenes Skeer mod høesten tilckommende Aar, Skulle hand Strax Samble Viidnerne af Nordre-\textit{Finlie}, om de af \textit{Jngenieurene} forlanges, Nemblig Hans Qvælien, Lars og Ole Joensen Samt Friiderich \textit{Skielbreen}, Saa og Vei-viisere breed-\textit{Thomas}, Vinckel-Zacharias, og Joen Siursen\textit{Lap-Finner}, Samt omtrent 12 arbejdere af bøjden, hvor til hand kan tage af \textit{Lap-Finner}, Saa mange faaes kan, hvilcke hand fører til \textit{Ingenieurene} paa Grændserne, ‒ uden ophold ‒\hypertarget{Schn1_43890}{}Schnitler fortsætter til Harran.
\DivII[Juli 28.-29. Fra Nordli til Harran]{Juli 28.-29. Fra Nordli til Harran}\label{Schn1_43892}\par
Samme 28{de}\textit{Julij} reiste man fra \textit{Skielbreen} dend væstligste Gaard i Nordre \textit{Finlie} i Væster, ladendes Skielbreens Vandet og \textit{otter}-Søen paa vændstre Haand igiennem Skoug og Myrland{1: Miil} over \textit{Gieting}-field, hvor man foer over endeehl Snee-Fonner, Som enda bar hæsten{6: miil} der i Skougen forblev man Natten over liggendes.\hspace{1em}\par
Dend 29{de}\textit{Julij:} Kom man igiennem Skoug og Myhrland, Siiden over \textit{Næss}- Elv, Som opckommer af \textit{Gieting}-fieldet og løeber 4{re} Miile Sydvæstlig i \textit{Namsen}- Elv Tæt under dend gaard \textit{Næss} Sønden derfor, til dend gaard \textit{Næss} i \textit{Harran Annex}\textit{Overhaldens} Præstegield{2: Miile _______ = 9: Miile.} Hvilcke 9: Miile, ere Field Miile af Bønderne angiivne: Dog Synis kun, at kunde reignes for 6 Maalte Miile, eller Liidet der over.\par
Og Som \textit{Finnernes} Skoele Mæster \textit{Friderich Botolphsen} af \textit{Missionairen} Hr. \textit{Povel Muus} og mig dend 26 Julj var Sendt fra Nordre \textit{Finlie} til Fieldz efter de Gamle \textit{Lap-Finner Ole Nilsen, Breed Thomes} og \textit{Jacob Olsen}, at føre dem til mig til \textit{Harran} til d: 31: Julj; Saa indfandt sig vel den første \textit{Ole Nilsen}, men i Stæden for Skolemæsteren kom den \textit{Finn Zacharias Olsen} med en anden ung \textit{Finn}, d: 31{de} Julj til mig i \textit{Harran} med berætning, at Breed-\textit{Thomes} havde fløttet fra Sit Sædvanlige Leje-Stæd, og at Jacob Olsen havde huller i Laaret og derfor icke kunde følge med ham hiid. Hvorfore ieg Samme 31: Julj affærdigede en \textit{express} til Jacob Olsen, at han maatte See at komme hiid til hæst, og beckostet en anden \textit{express} Nembl: SkoeleMæster-Jens til Fieldz en 14 Miile imod den østere Ende af \textit{Børje}-Fieldet efter Joen Andersen og efter forbemelte \textit{Breed Thomes}, om hand kunde faae opspurt hannem. ‒
\DivI[I Namdal fogderi: 6 vidner.]{I Namdal fogderi: 6 vidner.}\label{Schn1_44083}
\DivII[Aug. 6.-7. Rettsmøte på Solem i Harran]{Aug. 6.-7. Rettsmøte på Solem i Harran}\label{Schn1_44084}\hspace{1em}\par
\textit{A{o}1742{ve} dend 6{te} Augustj} blev \textit{Examinations} Rætten med de herhavende \textit{Lap Finner} paa gaarden \textit{Sollem} i \textit{Harrans Annex}\textit{overhaldens} Præstegield\textit{Nommedalens} Fogderie begyndt, i overværelse af en dertil beskickket \textit{vice} Lensmand \textit{Iohan Olsen Sollem} og de 2{de} Laugrættes Mænd \textit{Ioen Larsen Næs} af 67 aars Alder og \textit{Baard Isachsen Bya} 50 aar gammeI, Saa var og dend Antagne Tolck \textit{Erich Helset}, Som før har været \textit{Missionair} hos \textit{Lap-Finnerne} i disse fielde tilstæde; af Kongl: \textit{Civile} betiendtere mødte ingen; Fogden, Siiger de, at boe her fra i et andet Nemblig \textit{Nerøe} ‒ Ved Havet omtrent 14: miile her fra, og Soerenskriiveren icke langt fra hannem i Samme Præstegield; J Soerenskriiverens fraværelse førte da \textit{Capitaine des Armes Royem} Protocollen; Viidner, man havde, vare de \textit{Lap-Finner Ole Nilsen} og \textit{Zacharias Olsen}.\par
J Samtlige deris overværelse blev den \textit{Kongelig ordre} For \textit{Major Schnitler} til denne Rættes holdelse oplæst og for Viidnerne Samt for Tolcken \textit{Helset} Eedens Forcklaring af Lov- {b}ogen Kundgiordt og betydet, hvor paa de aflagde deres \textit{Corporlig Eed}, Viidnerne, at Siige {d}eris Sandhed om hvis dennem af Grændze-gangen var beckiendt, og Tolcken, at forcklare {d}eres udsagn Rætteligen.\hypertarget{Schn1_44204}{}Schnitlers Protokoller II.\par
Forud blev Lensmand og LaugRættet tilspurt\par
(1:) J hvad Præstegield og Fogderie denne bøid \textit{Harran} ligger? af hvor mange Gaarder den bestaar? hvordanne bøidens leje er? og hvad brug bønderne have at nære Sig af?\par
\textit{Resp:} Denne Bøjd \textit{Harran} er ett \textit{Annex} af \textit{overhaldens} Præstegield i \textit{Nommedalens} Fogderie; Dend har 14: gaarder og 31 bønder; Gaardene ligge fra øster i væster langs efter den Stoere \textit{Namsen}-Elv paa begge Siider af Samme Elv; Det er og den østerste bøjd af Præstegieldet, der ligger nærmest til de Svenske grænser paa denne Kandt; Bønderne leeve af deris Gaardz brug, og nogen Tømmer hugster til Saug brug. ‒\par
(2) Hvor langt ligger denne bøjd fra de nærmeste grændser til Sværrig?\par
\textit{Resp:} De kan icke viide det, Saasom deris bøjds Stræckning gaar icke Viidere i øster end en 1/2 Miil fra den østligste gaard \textit{Aasmulen}; Længere i øster have de indtet brug, ej heller komme der: Dog meene de at herfra bøjden til de Svenske Grændser Skal der være omtrent til 10 a 12 Miile. ‒\par
(3) Hvad er Landets beskaffenhed jmellem denne bøjd og Grændse-Mærcket?\par
\textit{Resp:} Landet bestaar af bare Skaug, Fielde, vande, og Elve, hvilcke af \textit{Lap Finnerne} indehaves og bruges liige til de Svenske Grændser; ved Elvene kan der være goed Gran- og Furru-Skoug; men til Fieldz er den liiden og Smal, Fielde som ligge imellem, ere (1) \textit{Tunsiø}- field, jmellem \textit{grøndals} Elv og \textit{Tunsiø}-Elven (2) \textit{Namsen}-field der Norden for imellem \textit{Namsen} Elv og \textit{Tunsiøn}; (3) der fra i Nord \textit{Rantze}-field imellem \textit{Saxern}-vand og \textit{Rantzeren}-vand; (4) J øster jmellem \textit{Tunsiøn} og \textit{Frostviigen Piil}-fieldet (5) Derfra i Nord imellem \textit{Tunsiøn} og \textit{Liimings}-vandet\textit{Ryg}-fieldet (6) i øster fra \textit{Liimings}-vandet, jmellem \textit{giorms}-vandet og \textit{Blaa}- Søen, Skal være \textit{Giorms}-field og der i øster (7) \textit{ornæs} field. \textit{Siøer}, Som falder i væster, er \textit{Tun}- Søe; De andre Søer østen derfor, hvor af Vandene løebe i øster, viide de liidet eller indtet at Siige af. Elve, ere (1) dend stoere \textit{Namsen} Elv, Som opckommer af \textit{Naames}-vand ved dend Sydvæstlige Siide af \textit{Børje}-fieldet, og Løeber i Sydvæst igiennem \textit{Overhaldens} Præstegield; J denne \textit{Namsen}-Elv fra havet op til Fossen i øster, Som er omtrent midt i \textit{Harrans} bøjden, Fiskes øret, og Lax. J denne \textit{Namsen}-Elv falde paa dend Søndere Siide (1) \textit{Sandøla-} (2) der fra i Nord \textit{Næss-} (3) Siiden i Nord \textit{grøndals-} (4) endnu Norden der for \textit{Tunsiø}-Elven, Som alle have deris løeb fra øster i Væster; Fra den Nordre Siide løebe Elve i \textit{Namsen}, (1) \textit{Biøra} Elv Strax østen for \textit{Ranums} hoved-Kiercke (2) Nord ost der fra \textit{Linset}-Elv Nord ost for dend gaard \textit{Aasmulen} (3) \textit{Flaatdals}-Elv (4) Nord ost derfra \textit{Biørnhus}-Elv; (5) Liigeleedes oven derfor \textit{Frørings}-Elv.\par
4: Hvilcke ere de nærmeste Gaarder paa dend Norske Siide fra Søer at reigne? Af hvad beskaffenhed er Landet? og hvad Næring bønderne der bruge? at forstaa fra denne bøjd?\par
\textit{Resp:} De nærmeste gaarder her fra denne bøjd i Syd ost er Nordre \textit{Finlie} 9 field-Miile herfra liggende; Landskabet herimellem er Skoug og Myhr-land og det Stoere \textit{Gietings}-field; Bønderne der i Nordre \textit{Finlje} leeve af deris gaardz brug og Fiskerie. Den Nærmeste bøjd i Nord og Nordvæst her fra er \textit{Høeland}, ett \textit{Annex} af \textit{Overhaldens} Præstegield, 1: god Miil her fra \textit{Harrans} bøjd liggendes, Landskabet her imellem er Skoug og Myhr; De bønder leeve af deris gaarder, og Fiskerie, j \textit{Hølands} vandene og Tømmer hugst til Saugbrug. ‒ J Nord fra \textit{Hølands} bøjden 2 l/2 Miil ligger den bøjd \textit{Folden}, et \textit{Annex} i \textit{Nærøe} Præstegield\textit{Nommedals}\hypertarget{Schn1_44522}{}Examinationsrætten sættes i Harran. Fogderie; Dends Kiercke kaldes \textit{Folder}Eid af den gaard hun Staar paa. Landskabet imellem \textit{Høland} og \textit{Folden} er Skoug og Myhr Saa og en Elv, Nafnlig Skaug-Elv, Som kommer af \textit{Hundsaae}-Vand ved \textit{Foldens} fielde og løeber fra Søer i Nord 2 1/2 i \textit{Foldens} Søes Botten eller østere Ende. Bønderne i denne \textit{Foldens} bøjd, boe omckring denne \textit{Foldens} Søes botten- og Nære Sig af deris Gaarder, Fiskerie i Søen og liidet Skougbrug til Saug. Viidere i Nord er de icke beckiendt.\par
5: Hvilcke ere de Nærmeste Gaarder Paa dend østere Svenske Siide Nærmist til Grændserne? fra Søer at reigne? af hvad beskaffenhed er landet? og hvad Næring bønderne der bruge?\par
\textit{Resp:} De Veed vel, det er Jemteland, men de veed icke noget om Gaardene at Siige. ‒\par
6 Om de Viide, hvilcke der ere de gamble Grændse-Mærcker fra Søer at reigne jmellem Norge og Sværrig paa denne kandt?\par
\textit{Resp:} Der om Viide indtet: men \textit{Lap Finnerne}, Som boe der ved, maa vel kunde giive underrætning der om. ‒\par
7{de} Om grændserne fra denne bøjd til Sverrig høre til nogen Gaardz Grund, eller ere Kongens Alminding?\par
\textit{Resp:} De hører ickke til nogen Gaardz-Grund, Saasom denne bøjds-Grændser Stræckke Sig ickke længer end en 1/2 Miil østen for dend yderste Gaard \textit{Aasmulen} i bøjden; Dend øfrige Viide Stræckning ad Sverrigs grænser er Kongens Alminding. ‒\par
8: Om de viide, at der har været nogen tvistighed jmellem de Norske og Svenske undersaatter om Grændserne i denne Eign?\par
\textit{Resp:} Det viide de icke af.\par
9: Om de viide hvad Nytte, godhed, og herlighed der er ved grændse Stæderne?\par
\textit{Resp:} Om der vare bønder Som havde Raad og Lejlighed der til, kunde der være goed Andleedning til Rødnings Land og bebyggelse ved \textit{Frostviig}- og \textit{Giorms}-Vandene.\par
10: Hvor langt de meene at grændsene ligge fra bøjdene og Lande-vejene?\par
\textit{Resp:} Som mældt kan det være fra denne bøjd en 12 Miile, men forcklarer derhos at fra Grændserne østen for \textit{Frostviig}- og \textit{Giorms}-Vande er der nærmere til Nordre \textit{Finlie} i \textit{Sneaasens} Præstegield end til denne \textit{Harrans} bøjd. ‒\par
11: Hvor underholdning for Folck, og græsning for hæste paa- eller ved- grændserne er at faa?\par
\textit{Resp:} Hvor underholdning der Folck vil faae viide de ickcke: men Græsning til hæste, fra Nordre \textit{Finlies} Vande liige til \textit{Børje}-fieldet bliver der goed Raad for, og icke vil fattes. ‒\par
12: Hvad Mænd kan være Vejviisere for de Efterckommende Grændse \textit{Committerede}, Som Skal op Maale Grændserne?\par
\textit{Resp:} Der kan jngen være Vej-viiser, uden \textit{Lap-Finnerne}, Som boe der ved Grændserne, hvor af de ingen kan Nafngiive, Saasom det er Saa langt her fra bøjden. ‒\par
13: Om Bønderne her forstaar det Finske-Sprog, at kunde fortolcke det paa Norsk?\par
\textit{Resp:} Neij, icke En forstaar det. ‒\par
Viidnerne bleve da fremckaldede og effter at dem af Tolcken var forcklaret, hvad en Eed havde at betyde og der paa, Som meldt, havde aflagt deris Eed, blev hver i Særdelished af Rætten tilspurdt, Som Tolcken dennem betydede, og igien til Rætten forcklarede deris Svar.\hypertarget{Schn1_44712}{}Schnitlers Protokoller II.\par
1{te} Viidne, heeder \textit{Ole Nilsen}\par
1 Spørsmaal hvor Føed? af hvem? om, hvor, og af hvem døbt? hvor opfød? og af hvem oplærdt i hands Christendom? hvor gammel? hvor hand har værit i hands opvæxet og ungdom, om paa et, eller fleere Stæder? hvor hand Sig nu opholder, og om er Stadig der eller Vancker fra ett Stæd til andet hend, og af hvad Aarsag? hvor længe hand har værit der paa Sidste Stæd? hvor af hand sig Nærer? om hand er gift og af hvem hand er Præste-viet? om hand har børn? om værit til gudsbord, Naar Sidst, hos hvad for Præst?\par
\textit{Resp:} Hand er Føed i Skougene ved \textit{Giorms}-vandet i \textit{overhaldens} Præstegield, der hvor hands Fader og Farfader har Siiddet. Hand er døbt i \textit{Harrans Annex} Kiercke\textit{Overhaldens} Præstegield af \textit{Capellanen} dend tiid, H{r}\textit{Hans Reisner}. J hands Ungdom har hand hørt kun liidet af Christendoms Lærdom, og det af \textit{Klochern} den tiid i \textit{Harran}, Førend Lector \textit{von Westen} kom, af ham blev hand meere oplærdt, ellers er hand opfød hos hands Forældre i \textit{Giorms}- Skougene. J hands Ungdom har hand tiendt hos en og anden af \textit{Lap-Finnerne} i disse \textit{Overhaldens} Fielde hvorfore hand fick i aarlig Løn, klæder og 1: \textit{Semuld}, det er en Hun-Reen med kalv. Der hand blev til Mand, Satt hand i hands Faders Rom i \textit{Giørms} Skoug, Paa hvilcket hand Erhvervede Sig Fæste-Sæddel af dend tiids Kongl: Norske \textit{Foged}, \textit{Jacob Jespersen}, hvilcken Fæste-Sæddel blev fremviist i Rætten og var \textit{dateret} dend 3 feb{r} 1699: og bielegges denne \textit{Act} og leveeres hannem \textit{vidimeret Copie} der af; Medens hand havde dette Rum, Sad hand i de Skouge om Sommers tiid,: men om Viintern, naar Sneen hart tilfrøs, at Dyerene ei kunde Sparcke den op, eller Slaae der igiennem med Fødderne, for at faa Mossen til deris Føede, maatte hand fløtte til det Field \textit{HaarKølen}, Som er ett Grændse-field imellem Norge og Sverrig, og med Sin væstere Ende ligger imod \textit{Finlierne} i Norge; Undertiiden Sad hand paa dend Norske undertiiden paa dend Svenske-Siide af dette Field, hvorfore hand til de Svenske \textit{Lap-Finner} i Villighed undertiiden maatte Svare 1/2 rd{r} i penger eller dets Værdie: Men for en 3 á 4 aar Siiden, da u-lyckke har Slaaet hannem til, at Ulvene og Sygdom har borttaget hands Dyer, er hand bleven udarmed, og en Tigger-\textit{Finn}, om Sommern hos de andre \textit{Lap-Finner} i Fieldene, om Viintern hos Bønderne i Bøjdelaugene. Er gammel imod 80 aar, som man kan reigne ud med ham, gift, og har 3: leevendes børn, og 12: børnebørn; Hand fick en Svensk \textit{Finne}-hustrue, og en Svensk Præst i \textit{Jemteland} Hr. \textit{Willum} Sammenviede dem i \textit{Strøms Annex} Kiercke i \textit{Hammerdals} Præstegield ‒ Hand har gaaet til Gudz-bord Sidst afvigte Viinter for Hr. \textit{Povel Muus} paa \textit{Sneaasen}.\par
2 Spørsmaas[!]: Hands Field-Sæde hvor vidt en \textit{Tract} og Stræckning dend haver, jtem hvad \textit{Situation} af Fielde, Skoug, og vande, og hvad Nafn hvert haver? under hvad Sogn og Fogderie det ligger? med hvad adkomst hand dertil haver?\par
\textit{Resp:} Hand besad \textit{Giørms}-vandet med dets Skouge og Fielde paa dend østere Siide til mod \textit{Ornes}-field; Paa dend Søndere Siide til \textit{Frostviig}-vandet; Paa dend Nordre Siide til \textit{Blaa}-Søen; Paa dend væstre Siide havde hand foruden Skougen \textit{Liimings}-Vandet og \textit{Utnes}- vandet Samt \textit{Vagteren}-Vand med dends Elv indtil mod \textit{Saxern}-vand, med det land, Som ligger derjmellem fra \textit{Saxen}-vand til dend Væstere Ende af \textit{Liimings}-Vand, hand besad og Norden for Utnes- og \textit{Vagtern}-Vande et Støckke Land omtrent 1: Miil breed.\hypertarget{Schn1_44944}{}1 Vidne i Namdalens fogderi.\par
A: \textit{Giørms} vand Stræckker Sig fra øster i væster 2 1/2 Miil Lang, og kan være 1/2 Miil breed; Det ligger Norden for \textit{Frostviig}-vandet 1/4 Miil vejs, og der imellem Gran- og biercke- Skoug; Fra \textit{Blaa}-Søen ligger det i Søer 1: Miil, og er der imellem Skoug og \textit{Giørms}-Field; Paa den østere Siide af \textit{Giørms} vand er Smaa Gran-Skoug 1/8 Miil breed til under \textit{Ornes}-Field; Paa dend væstere Siide af \textit{Giørms}-vand til \textit{Limings}-vandet er en god bøjde Miil af Skoug og Field bestaaende. ‒\par
B: \textit{Limings}-Vand Stræckker Sig fra Sød-ost liidet i Nord væst 3 maalte Miile lang og 1/2 Miil breed; J Nord ligger dette \textit{Liimings}-Vand fra \textit{Tun}-Søen 1: Kort Miil og der imellem er Skoug og \textit{Ryg}-Field; Fra \textit{Utnæs}-vandet ligger det i Søer og Sydost 1 1/2 miil, og der imellem er Field bierck- og Gran-Skoug. Paa dend Væstere Siide af \textit{Liimings}-Vandet til \textit{Namsens}- Field er 1/2 miil af Biercke Riis og Myrland. ‒\par
C: \textit{Utnes}-vand ligger fra øster i væster 1: mil lang, og 1/4 Miil breed; J Søer ligger det fra \textit{Rantserens}-vand 1 Miil; Der imellem er Smaa Field-\textit{vohler} og biercke Skoug; J væster ligger det 1 1/2 Miil fra \textit{Blaa}-Søen, Der imellem er Landskabet Biercke-Skoug og Smaa Berg-\textit{vohler}; Fra \textit{Utnes}-vand gaar Elven i mange Kroger 1/2 Miil veis i væster jnd i \textit{vagteren}-vand, og er imellem disse Toe vande og biercke Skoug. ‒\par
D: \textit{Wagtern}-vand Gaar fra øster i væster 1: Miil lang, og imod 1/2 Miil breed, J Nord ligger det fra \textit{Namsens}-field 1/2 Miil; J Søer er det fra \textit{Rantseren}-Elv 1: Miil; J Væster har det \textit{Saxen}-vand et bøsse Skud fra Sig liggendes; J forbenefnte 4{re} Vande Fanges Røe og ørret;\par
Denne hands beskreevne \textit{district} ligger i \textit{Overhaldens} Præstegield\textit{Nommedallens} Fogderie. Sin adkomst hertil el: Fæste-Sæddel har hand før indleveert i Rætten, sub L: A.\par
3: Sp: Om hands Fader og Forfædre have Siddet paa Samme Rom, Som hand havde, og har haft Samme Viide og Stræckning, som hand? \par
\textit{Resp:} Hands Fader Farfader og Olde-fader, have haft Samme Sæde, og brugt landet østen for Vandene til \textit{ornes}-fielddet, og i Sæhrdeelished til de Væstlige klimper der paa \textit{Saxolaavara} og \textit{GiepsKiach};\par
Dette \textit{Saxolaa-Vara} er et Field for Sig Selv og Rund, 1/4 Miil over, Liggendes fra dend østere Ende af \textit{Heidugel}-Søen omtrent 1: Miil i Nord, og fra \textit{Ornes}-Fieldet, at reigne fra dends øfre Deel 1/4 Miil i Søer; Jmellem dette \textit{Saxolaa-vara} og \textit{Heidugel}, er Skougland af Gran; Og jmellem det og \textit{Ornes}-field er først liidet bierck-Riis, Siden Stiiger Fieldet op til sin højde;\par
Paa dend væstere Siide af \textit{Saxolaa-Vara} er bierck- og paa dends østere Siide er Gran- og Furru-Skoug. ‒\par
\textit{Gieps-Kiach} er en bergcklimp paa \textit{Ornes}-field; Paa dend Søndere Siide og væstere Ende, Snau og bahr oven til, med biercke-Riis needen om Sig, er højere end \textit{Saxolaa-vara}, og 1/2 miil over; Langagtig, Stræckende Sig fra væster i øster ligger fra \textit{Saxolaa-vara} i Nord 1: miil og fra \textit{Rautekie-vara}, Som er en Rødagtig Flad og Snaug berg-klimp, 1/2 Miil i Søer; ‒ Paa alle dends Siider er \textit{Ornes}-field, og paa dend væstere Siide daler det neer til en Riisskoug.\par
4: Spørsmaal. Hvor mange Madlauger el: \textit{Familier} de ere Sammen, og hvor mange Folck de udgiøre i dette hands Field Sæde?\par
\textit{Resp:} Hand har Siddet der med Sine børn og Børnebørn; Og Siiden hand er bleven udarmed, Sidder der 3: hands børn gifte med deris \textit{Familier}, Som før er melt. ‒\hypertarget{Schn1_45204}{}Schnitlers Protokoller II.\par
5: Spørsmaal: Hvor langt dette hands Field-Sæde ligger fra Nærmeste grændser til Sverrig?\par
\textit{Resp:} Det Støder an til Sverrigs grændser. ‒\par
6: Sp: Hvordan er Landskabets beskaffenhed jmellem hands Sæde og Grændserne? og om Samme Landskab er Romt og tilstræckkelig nock til at føede ham og hands \textit{Familier}, foruden at hand Skal have nødig at fløtte?\par
\textit{Resp:} Siiden hands Sæde Støeder til grændserne, Saa er det beskreevet før; Ellers er hands \textit{District} dyblandet med flade Fielde i, hvor det regner, naar det Sneer i de højere Fielde; Naar nu Sneen ved paafølgende Frost haardt tilfryser at \textit{Reenen} ej kan Slaae derigjennem, maa hand fløtte andenstædz hen til høje Fielde, hvor det icke Saa meget Regner, men Sneen ligger Løs, for at Søge føde for hands dyer. ‒\par
7: Sp: Hvilcke ere de Nærmeste Bønder-gaarder paa dend Norske Siide, Nærmest ved grændserne fra Søer at Reigne?\par
\textit{Resp:} Det er dend bøjd Nordre \textit{Finlie}, hvor af 2: Nyeligen have bygget ved \textit{Frostviigens} Nord-Væstlige Ende, hvorhendtil hands \textit{district} har gaaed j Søer; Norden der fra, er ingen bonde gaard. ‒\par
8: Hvilcke og hvor mange ere de nærmeste \textit{Finner} ved Grændserne paa dend Norske Siide, fra Søer at Reigne? Af hvad beskaffenhed er deris land? hvor af de Sig Nære og til hvad Kiercke Sogn de henhøre?\par
\textit{Resp:} Sønden for Sig kiender hand ingen \textit{Finn} meer end \textit{Breed Thomes Tomesen}, Som Sidder paa dend Væstere Ende af \textit{HaarKølen}, og kan komme Selv for Rætten her tilstæde at forcklare det øfrige af Spørsmaalet.\par
J Nord fra ham er den \textit{Lap-Fin}\textit{Anders Torkelsen}, hvis Land Stræckker Sig til de Svenske grændser, og er hands Søn her paa gaarden Nærværendes, der om at giive Nærmere forcklaring, Fleere \textit{Finner} ickke er, Nær Greendserne paa dend Søndre Siide af \textit{Børje-fieldet}. ‒\par
9 Spørsmaal: Hvilcke er de Nærmeste bønder gaarder paa dend østere Svenske Siide nærmest til grændserne, fra Søer at Reigne? Af hvad beskaffenhed er Landet? hvor af de Sig Nære?\par
\textit{Resp:} 6 Miile udi-ost fra \textit{Avinds}bæcken at forstaa 6 Svenske Maalte Miile, ligger en Enckelt gaard, Nafnlig \textit{Hilsand}, og 1/2 Miil Norden for denne er en Anden Enckelt gaard ved Nafn \textit{Ringsøe}; Hvilcke have 4{re} Miil til deris kiercke \textit{Strøm} ett \textit{Annex} af \textit{Hammerdals} Præstegield i \textit{Jemteland}; fleere bønder gaarder i Nord hand icke kiender; Dend gaard \textit{Hilsand} ligger i Skougen i Nord ost fra \textit{Haarkølen}, paa dend Søndere Siide tædt ved \textit{Hoetagen}-Søe; Dend gaard \textit{Ringsøe} ligger liigeleedes i Skougen paa dend Nordre Siide tædt ved \textit{Hoetagen}-Søe; Hver af disse Gaarder beboes af en Enckelt bonde, som leeve af deris Gaards brug og Fiskerie Samt Vejderie, og faae de der godt korn, naar det icke frysser. ‒\par
10 Spørsmaal: Hvilcke og hvor mange ere de nærmeste \textit{Finner} paa dend østere Svenske Siide fra Søer at Reigne? af hvad beskaffendhed er Landet? hvor af de Nære sig?\par
\textit{Resp:} Paa \textit{Ornes}-fieldet østen for hannem Sidde Toe Svenske \textit{Lap-Finner} Nafnl: \textit{Joen Nilsen Finne}-Lænsmand og \textit{Anders Hendrichsen}; Deris \textit{district} Stræckcker sig paa dend Søndere Siide til Vatsdalen Nemblig \textit{Svanevandet} og \textit{Hoetagen}; hvor langt den Stræckcker Sig \hypertarget{Schn1_45400}{}1 vidne i Namdalens Fogderi. Norden for \textit{ornes}-fieldet, veed hand ickke, paa dend østere Siide gaar den til den gaard \textit{Ringsøes} grændser i \textit{Jemteland}. ‒\par
Disse 2: Svenske \textit{Lap-Finners} Landskab bestaar i \textit{ornes}-field til imod dets Væstere Ende, og i Skouge derom liggendes, paa Søndre- Nordre- og østere- Siide; \textit{Ornes}-field Stræckcker Sig fra Væster i øster en kort Dags-Reise, som efter hands betydning Rætten Slutter at maa være omtrendt 4{re} Maalte-Miile, og er dette Field Saaleedis Skabt, at midt der j gaar ligesom en field dahl fra væster i øster, og giør liigesom Toe adskilte fielde der af; Begge deris breedde fra Søer i Nord, Siiger hand at være en halv dags-Reise over, som Rætten Slutter at være 2 à 3 Miile. ‒\par
Skauglandet omckring dette Field, bestaar af Gran og Furru; Disse 2 Svenske \textit{Lap- Finner} forcklarer hand at være Riige og at have mangfoldige Reendyer; ellers Ernærer de Sig paa Samme Maade, Som de Norske \textit{Lap-Finner}; Hver af dem Svarer 2 rd{r} til den Svenske Crone, og 1 rd{r} til Præsten; Norden der for er den Svenske \textit{Finn}Siur pedersen, Som Siidder østen for Anders Torckildsen, hvis Søn her er tilstæde, og om hannem kan giive Forcklaring. ‒\par
11: Hvilcke ere Grændse-Mærckerne jmellem Norge og Sverrig fra Søer at reigne paa denne kandt?\par
\textit{Resp:} Hand Siiger Lande-Mærcker at være fra \textit{Penningkiesene} i Søer at Reigne\hspace{1em}\par
A:\textbf{Avindsbæch ‒}\par
B:\textbf{Svanesteenen}\par
C:\textbf{Saxolaa-vara ‒}\par
d:\textbf{Gieps-Kiach ‒}\par
e \textbf{Rautekie-Vara}\par
(F) \textbf{Warsands-vara} og Siiden viidere til\par
g: \textbf{Rast giuch} eller \textbf{Rast}-Elven i\par
(h:) \textbf{Børje}-Fieldet.\par
Viider i Nord er hand ickke bekiendt.\par
A \textit{Avindsbæchen} forcklarer hand Saaleediis, at dend Riinder liige i Nord i den østere End(e) af \textit{Heidugeln:} men dend bæck, som kommer væsten i fra, Pæger eller Siigter hend mod \textit{Svanesteenen}, dog løeber i \textit{Avindsbæchen} og med den paa bemelte Maade i \textit{Heidugeln};\par
B \textit{Svanesteenen} beskriiver hand liigsom 14{de} Viidne af \textit{Jnderøen}, dog med den forcklaring, at Svane-vandet Skal være 1: Miil lang og 1/2 Miil breed. ‒\par
C: \textit{Saxolaa-vara} og\par
D: \textit{GiepsKiach} ere før beskreven. ‒\par
E: \textit{Rautekie-vara}, er Som melt Rødagtig og Flad oven paa, Rund af Skabning, 1/2 miil over af Størelse, Snau og bahr af Sig, ligger liige i Nord fra \textit{GiepsKiach} 1/2 miil, og liige i Søer 1/2 Miil fra \textit{Warsands-vara}, Needen under og om Sig har det, det Snaue \textit{Ornes}-field, men paa dend væstere Siide dahler det need til Skougen. ‒\par
F: \textit{Warsands-vara} er Langagtig fra øster i væster omtrendt 1/2 Miil, og næsten liigesaa breedt fra Søer i Nord, er Snaut og noget højere end \textit{Rautekie-vara}, og dahler liigesom de forrige paa dend væstere Side need ad Skougen. ‒ Dette \textit{Warsands-vara} ligger paa dend \hypertarget{Schn1_45632}{}Schnitlers Protokoller II. Nordre og Væstere Ende af \textit{Ornes}-fieldet, der fra gaar Grændse gangen igiennem Skoug og Myhr-land liige i Nord først i dend væstre Ende af det væstere \textit{Raukie}-vand, omtrent en halv dags-Reise, siiden her fra langs eftter \textit{Rast-giuch} eller \textit{Rast}-Elven 1. liiden Miil i Nord til \textit{Børje}- fieldet. ‒\par
Dette Væstere \textit{Raukie}-Vand, ligger fra Væster i øster 1/2 Miil lang, og er 1/4 breed; Derj Fanges Røe og Øret; Af dette Væstere \textit{Raukie}-vand løeber en bæck liige i øster 2 Bøsse-Skud (at forstaa med Rifle) i et andet Vand kaldet østere \textit{Raukie} Vand, Som stræcker sig fra Væster i øster 1 1/2 Miil lang, og er 3/8:‒ miil breed hvor i Samme Slags Fisk Fanges. \textit{Rast}-Elven opckommer af Smaa Kiønne øvest i \textit{Børje} Fieldet og løeber i Søer hend imod 2 miil i dend væstere Ende af Væstere \textit{Raukie}-Vandet. ‒\par
\textbf{Børje}-Field beskriiver hand at Stræckcke Sig fra Øster i Væster, Dend østere Ende der af gaar need i \textit{Aaselle-Lap-Marck} i \textit{Sverrig}, Dend væstere Ende med Sin Søndere Deel Naaer need til \textit{Fiplings vats dal} i \textit{Bindallen}, \textit{Brønøens} Præstegield, \textit{Helgelands} Fogderie\textit{Nordlands Amt} i \textit{Norge}, og med Sin Væstere Deel need til \textit{Wæfsen} i \textit{Alstahoug} Præstegield i bem(e)lte Fogderie, en 6 maalte Miile i Nord fra berørte \textit{Bindalen}; Samme \textit{Børje}-field er 2 goede Dagers Reisse lang (en god Dags Reisse formeener Rætten at være omtrent 8 maalte Miile) breed er det en fuld Dags Reisse over om Sommeren; Fra dette \textit{Børje}-fields væstere Ende til \textit{Rast-giuken}, Som er et grændze-Skille-Mærcke paa Fieldet, Skal være goede halvanden Dags-Reisse, og det er den Deel som Viidnet Siiger, at \textit{Norge} tilckommer af \textit{Børje}-fieldet; Dend Deel deraf, som ligger østen for \textit{Rastgiuk}, Skal være en god halv Dags Reisse, og tilligge \textit{Sverrig}; Paa dend væstere Deel af \textit{Børje}-Fieldet væsten for Rast-giuk, Sidde Norske \textit{Finner}, og paa dend østere Deel der af østen for \textit{Rast-giuken} Svenske \textit{Finner;} Hvor mange af de Norske \textit{Finner} paa den Norske Deel af Fieldet Sidde, Veed hand ickke; Paa dend østlige Svenske Andeel meener hand at en 10 \textit{Familier} paa begge Siider tilholde; Ellers er dette Field Snaut og bahrt.\par
12 Om de grændser høre til nogen Gaardz Grund, eller ere Kongens Alminding?\par
\textit{Resp:} Det er Kongens Alminding. ‒\par
13 Sp: Om der har værit nogen tvistighed imellem de Norske og Svenske undersaattere om disse grændser?\par
\textit{Resp:} icke j hans tiid.\par
14 Sp: Hvad Nytte, godhed, og herlighed er der ved disse Grændse-Stæder?\par
\textit{Resp:} Der er Foeder og Beete for \textit{Reens} Dyerene, og Fiskerie i vandene Samt Vejderie i Skougene. ‒\par
15 Hvor langt de grændser ligge fra bøjden og Lande-vejen?\par
\textit{Resp:} Liigesom Laug-Rættet her paa Stædet ved 10 Spørs: sagt.\par
16 Hvor underholdning for Folck og beete for hæstene ved disse grændser bliver at faae?\par
\textit{Resp:} Liigesom Laug Rættet ved 11{te} Spørsmaal. ‒\par
17{de} Spørsmaal: Hvad Mænd kan være Vej-Viisere?\par
\textit{Resp:} Unge \textit{Finner} saasom \textit{Zacharias Olsen} og \textit{Joen Andersen} og Fleere andre. ‒\par
18{de} Sp: Hvilcket er det Nærmeste Stæd i Nord her fra, hvor \textit{Finnerne}, Norden for disse Viidner nærmest og beleiligst kan møede til næste Viidne-Forhøer?\hypertarget{Schn1_45880}{}1 og 2 Vidne i Namdalens Fogderi.\par
\textit{Resp:} J \textit{Bindalen} 3 dags Reisse her fra, Som kan være 11{ve} Stoere Miil.\par
Til Slutning tilspurtes Viidnet: om icke \textit{Demants}-field og \textit{Rundfurru} ere Lande Mærcker Norden for \textit{Svanesteenen}, og hvor de ligge?\par
\textit{Resp:} Hand kiender icke \textit{demans}-field og veed icke der af at Siige; \textit{Rundfurru} forcklarer hand at ligge langt østen for dend østere Ende af \textit{Børje}-fieldet mod bøiden \textit{aassele}, Som hand Siiger at ligge i \textit{Angermannia} i Sverrig hvorpaa hand blev \textit{dimitteret}.\par
Under \textit{Examen} af dette Viidne indfandt Sig de \textit{Lap-Finner} langt fra Fieldet, som \textit{Majoren} havde Skickket bud efter, hvilcke og, Siiden det var Saa Siilde paa aftenen, følgende Dag blev foretagen til \textit{Examination}.\hspace{1em}\par
1742{ve} dend 7{de}\textit{Augustij} ‒ Foretoeg man de \textit{Finner}, Som i gaar vare til komne Nemlig, Marrit \textit{Nilsdatter}\textit{Nils} Joensens hustrue, \textit{Tomes Tomessen, Joen Siursen} og \textit{Joen Andersen}, hvilcke som de liidet eller indtet af det Norske Sprog forstoed; Saa loed Rætten ved Tolcken Erich Helset for dennem Eedens forcklaring betyde, og de aflagde der paa deris \textit{Corporlig} Eed ‒\hspace{1em}\par
2 Viidne af \textit{Lap-Finnerne} af \textit{Overhaldens} Præstegield\textit{Nommedalens} Fogderie ‒\par
\textit{Marrit Nils Datter} er Føed ved denne Gaard \textit{Sollem} i \textit{Harrans} bøjd hvor hendis Fader tilholdte og var Tigger-\textit{Finn}, hun veed icke Rættere, end at være døbt her i Harrans-Kiercke af dend tids Præst i \textit{overhalden}; Hun er opfød her i \textit{Harrans}-bøjd, hvor, og paa \textit{Sneaasen} hun har tiendt; Af Sogne-Præsten paa \textit{Sneaasen} Hr. \textit{Peter Muus} er hun bleven oplærdt i Sin Christendom, er 40: aar Gammel, gift med Nils Joensen\textit{Lap-Finn} Som er ung og kun 25 aar gammel, har 2 børn, holder til med Sin Mand i \textit{Giørms}-Skougene i \textit{overhaldens} Fielde, har værit Sidst afvigt Viinter til guds bord hos Prousten i \textit{overhalden} Hr. \textit{Morten Lund} i denne \textit{Harrans Annex} Kiercke. ‒\par
Til 2{det} Sp: \textit{Resp:} Hendes Mand Sidder i hands Farfaders Ole Nilsen Field-Sæde, som efter bøxel Sæddeln er i \textit{Giørms} Skouge og før er beskrevet.\par
3 Spørs: er Forcklaret før ‒.\par
4 Spørs \textit{Resp:} De ere 3 Madlauger eller \textit{Familier}, børn af dend gamble \textit{Finn Ole Nilsen}, Som holde nu til i dette field-Sæde \textit{Giørms}-Skouge ‒.\par
Fra 5{te} til 9{de} Spørsmaal \textit{inclusive}, Svarer hun det Samme som første Viidne næst tilforn. ‒\par
til 10 Sp: \textit{Resp:} Der af veed hun indtet ‒.\par
11 Sp: Svarer: hun har hørt det af andre \textit{Finner}, Som have haft det for Sig paa tale, \textit{at Saxolaa-vara, GiepsKiach}, dend væstre Ende af Væstre \textit{Raukie}-vand og \textit{Rastgiuch} Skal være Lande-Mærcker paa denne kandt imellem Norge og Sverrig ‒\par
Til det 12 Sp: Svarer Som 1{te} Viidne. ‒\par
13{de} Liigesaa ‒ Som hun icke andet veed\par
Til 14{de} Svarer Som 1{te} Viidne. ‒\par
Til 15 Sp: \textit{Resp:} Hvor mange Miil grændserne ligge fra denne bøjd \textit{Harran}, kan hun ickke Siige: men hun veed dog, at det er nærmere til Nordre \textit{Finlie} ‒ end til denne ‒ bøjd ‒\par
16: 17{de} Sp: Svarer: det Samme Som 1{te} Viidne.\par
18: Sp: \textit{Resp:} hun veed indtet der til at Svare. ‒\hypertarget{Schn1_46231}{}Schnitlers Protokoller II.\par
3:{die} Viidne af \textit{Lap-Finnerne} af \textit{overhaldens} Præstegield\textit{Nommedalens} Fogderie, heeder \textit{Zacharias Olsen}, er Fød i \textit{Giørms}-Skougen, hvilcken hands Fader \textit{Ole Nilsen} havde Sig fra Fogden tilfæstet, døbt i \textit{Grongs Annex} Kiercke i \textit{overhaldens} Præstegield, af dend tiids Præst Hr. \textit{Hans Reisner}, er opfød hos hands Forældre, og lærdt noget liidet af hands Christendom Som hand har kundet faae høre af bønderne og Klockeren i Nordre \textit{Finlie}, 40 aar gammel, er Gift med en Svensk \textit{Finne} hustrue af \textit{Grubdals} fielde af \textit{offerdals} Præstegield, har 4: børn; Hand Sidder nu i \textit{Giørms}-Skouge, hands Faders forrige Field-Sæde og maa fløtte, naar Sneen er hart tilfrossen Liigesom hands Fader det 1{te} Viidne har forcklaret, da hand fahrer til hands hustrues Fader paa det \textit{Jemte-Grubdals} field, ellers forcklarer hand, at det er for 10 aar Siiden, hand fløttede til bemelte \textit{Grubdals}-field, og Sad der i 6: aar og for 4: aar er hand kommet tilbage til dette hands field Sæde i \textit{Giørms} Skouge; Berætter derhos at en anden \textit{Lap Finn} fra væstre \textit{Haarkølens} Ende \textit{Breed Thomes} var kommet til viidnet Sidst afvigte vaar, formedelst Kuldens langvarighed og med Viidnets forlov Satt Sig neer i hands Skouge, hvor det da har hændt Sig, at da denne \textit{Bree Thomes} hans husmand har lagt varme an i Skougen, er der geraadet jld i Skougen og dend heele Gran Skoug 2 1/2 miil lang afbrendt; Det er over 1: aar Siiden, hand har gaaet til Gudsbord, fordj hand treffede icke hr. \textit{Povel Muus} i Nordre \textit{Finlje}, dend gang denne, Nembl: H{r}\textit{Muus} var der; mens hand kom for Siilde, da hand var alt reist. Hand har vel nu Sidst forleeden Viinter Søgt Prosten H{r}\textit{Morten Lund} at gaa til gudsbord i denne \textit{Harrans Annex} Kiercke, mens fordj hand jngen Skrift Sæddel fra \textit{Sneaasens} Præst medhavde, blev hand icke her antagen.\par
Til det 2{det} 3{die} Og 4{de} Spørsmaal: Svarer hand det Samme Som 1{te} Viidne af \textit{Lap Finnerne}, dette tilleggendes ved det 4{de} Spørs: at hand vel i vaar har indtaget \textit{Bree Thomes}, men vil have ham derfra igien til næste Viinter, uden saa er, at de andre hands Søskinde vil tage ham til Sig ‒\par
Til 5{te} 6{te} 7{de} 8{de} 9{de} 10{de} Spørsmaale Svarer hand det Samme Som 1{te} Viidne af \textit{Lap Finnerne}.\par
Til 11{te} Spørsmaal: Svarer: De Lande-Mærcker hand har hørt og veed af at Siige ere fra \textit{Penningkiesen} at reigne ‒ \centerline{\textit{AvindsbæchenSaxolaa-Vara} ‒ \textit{GiepsKiach} ‒ \textit{Rautekie-vara} ‒ \textit{Warsands-vara} ‒}\hspace{1em}\par
Dend væstre Ende af Væstre \textit{Raukie}-vand, der hvor \textit{Rast-giuch} indfalder. ‒\par
\textit{Rast-giuch} eller \textit{Rast}-Elven, langs efter den, Som dend løeber Nord i \textit{Børje-fieldet}. ‒\par
Af \textit{Svanesteenen} veed hand icke at Siige, at det Skulle være Lande-Mærcke. Ellers beskriiver hand forbenefnte Lande Mærcker, Liigesom 1{te} Viidne af \textit{Lap-Finnerne}, undtagen \textit{Børje}-field om hvilcket hand ingen beskeeden kan giive. Dog Siiger hand at Væsten for \textit{Rastgiuch} er det Norske \textit{Finner}, og osten for \textit{Rast-giuch} er det Svenske \textit{Lap-Finner}, Som paa \textit{Børje}- Field tilholde.\hypertarget{Schn1_46516}{}3 og 4 Vidne i Namdalens Fogderi.\par
12{te} Sp: Svarer det Samme Som 1{te} Viidne af \textit{Lap Finnerne} ‒\par
Til 13{de} Spørsmaal: \textit{Resp:} De Svenske \textit{Finner} paa \textit{Ornes}-field ‒ Som ere hands Naboer i øster, vil gaa for Nær ind paa ham og tilEigne Sig dend Stræckkning i væster til mod \textit{Giørms}- vandet og \textit{Blaa}-Søen: Da hand dog Siiger, Sig at til komme det Støckke Land fra \textit{Giørms}- vandet og \textit{Blaa}-Søen i øster til \textit{Saxolaa-vara}, og \textit{Gieps-Kiach} paa dend væstere Ende af \textit{ornes}- fieldet; Saasom hands Fader og Forfædre have haft det Saaleedis.\par
14{de} 15{de} 16{de} 17{de} og 18{de} Svarer det samme Som 1{te} Viidne af \textit{Lap-Finnerne;} Derhos forcklarendes, at der er kuns faae Fattige \textit{Finner} i \textit{Bindalen}, som liiden eller ingen beskeeden Skal kunde giive om Grændserne Saa langt bort i øster, men i \textit{Wæfsen} ville ieg forefinde Mangfoldige formuendes \textit{Finner}, Som have deris drift i \textit{Børje}-Fieldet; og bæst Skal kunde forcklare dets og grændsernes \textit{Situation}. Hvor paa blev \textit{dimitteret}. Efter at have forcklaret, indtet at viide af \textit{demans}-field eller \textit{Rundfurru} ‒\hspace{1em}\par
4{de} Viidne \textit{Joen Siursen}, veed icke, om hand er Føed i \textit{Børje}-Field, eller \textit{Niarke} i \textit{Brønøens} Præstegield\textit{Helgelands} Fogderie, Paa hvilcke Fielde hands Fader har Vanchet og værit husmand hos andre; Er døbt i \textit{Harrans Annex} Kiercke\textit{overhaldens} Præstegield af dend tiids Præst Hr. \textit{Hans Reisner;} J hands ungdom har hand tiendt hos \textit{Finner} og er i \textit{Lector von Westens} tiid af \textit{Missionairen} dend gang, nu Tolck her, \textit{Erich Helset} bleven underviist i Sin Christendom, hand meener at være 50 aar gammel, er gift har 6 ‒ børn, Sidst afvigte Viinter i Juule-tiider har hand værit til guds bord i \textit{Sneaasens} hoved-Kiercke hos hr. \textit{Povel Muus} ‒\par
Til 2{det} Spørsmaal: Svarer: Hand holder til i \textit{Giørms}-Skouge, det Field-Sæde Som 1{te} Viidne \textit{Ole Nilsen} har besiddet, hvis Datter hand har til hustrue, og bruger hand det Land tilfælles med sin Hustrues Søskinde, til Sin og \textit{Families} underholdning; Dette Lands Stræckning og \textit{Circumference} af Vande, Fielde, og Skouge er af 1{te}\textit{Finn} viidne ved dette Spørsmaal før beskreeven.\par
3{de} Spørsmaal \textit{Cessat}. ‒\par
Til 4{de} 5{te} 6{te} 7{de} 8{de} 9{de} og 10{de} Spørsmaale Svarer det Samme Som 1{te} Viidne af \textit{Lap-Finnene}.\par
Til 11: Spørsmaal: Svarer det Samme Som 3{die} Viidne af \textit{Lap Finnerne} op Reignendes Nemblig fra \textit{Penningkiesene} til \centerline{\textit{Avindsbæchen} ‒ \textit{Saxolaa-Vara} ‒ \textit{Gieps-Kiach} ‒ \textit{Rautekie-vara} ‒ \textit{Warsands-vara} ‒}\hspace{1em}\par
Dend væstre Ende af væster \textit{Rautkie}-vand der hvor \textit{Rastgiuch} indfalder.\par
\textit{Rastgiuch} eller \textit{Rast}-Elven, langs efter den, Som den løber i Nord i \textit{Børje}-fieldet Hvilcke Mærcke-stæder hand beskriiver at være liigesom 1{te} Viidne af \textit{Lap Finnerne} giordt haver, ellers Siiger hand om \textit{Svanesteenen} liigssom 3{de} Viidne, icke at viide det at være grændse Skiell.\par
Til 12{te} 13{de} 14{de} 15{de} og 16{de} Spørsmaal Svarer som 1{te} Viidne af \textit{Lap Finnerne}. ‒\hypertarget{Schn1_46880}{}Schnitlers Protokoller II.\par
Til 17{de} Sp: \textit{Resp:} Hand, Nemblig Viidnet med fleere unge \textit{Finner}, kan være VejViisere.\par
18 Spørsm: \textit{Resp:} Som 3{de} Viidne af \textit{Lap} Finnerne.\par
Om \textit{Demans}-field og \textit{Rundfurru} har indtet hørt, veed heller ickke, hvor de ligge. hvorpaa blev \textit{dimitteret}.\hspace{1em}\par
5{te} Viidne af \textit{Lap-Finnerne} af \textit{overhaldens} Præstegield: \textit{Bree Thomes Thomesen}, Hand er Føed paa \textit{Ornes}-Field et \textit{Jemtelands}-Field, hvor hands Fader var løs eller en husmand, opfød hos hands Faders-Søster paa dend østere Ende af \textit{Ornes}-field og underviist i hands Christendom af andre Svenske \textit{Finner;} 50 aar gammel, gift med en Svensk \textit{Finne}-Koene af \textit{Haarkølens}-field, har 5: børn, i Viinter for Ett Aar har hand Sidst gaaet til Gudzbord i \textit{Strøms Annex}\textit{Hammerdals} Præstegield i \textit{Jemteland}. ‒\par
2 Spørsmaal \textit{Resp:} Hand har Egentlig Sit Sæde paa dend væstre Ende af \textit{Haarkølen}, hvor hand har tilholdet i 10 aar efter Hr. \textit{Povel Muuses} Giivne forlov: men i Vaar formedelst dens langvarighed og Ulvenes \textit{Grassering} paa \textit{Haar-Kølen} har hand maattet fløttet der fra og Søgt Rum i \textit{Giørms}-Skougene hos \textit{Zacharias Olsen}, hvor hand kuns Sidder for en kort tiid med forlov af bemelte Norske \textit{Lap-Fin}; Førend Viidnet kom til den væstere Ende af \textit{Haarkølen}, havde hand Sit tilhold hos hands Koenes Forældre paa østere Siide \textit{Haarkølen}. ‒\par
Efter Rættens tilspørgende Siiger hand? J de 10: aars tiid, hand har Siddet paa Væstre \textit{Haar-Kølens} Ende har hand Gemeenlig om viinteren Søgt \textit{Strøms}-Kiercke i \textit{Jemteland}, og om Sommern \textit{Sneaasens}-Kiercker i Norge; Hand har af dette Sit Rom paa \textit{Haarkølens} væstere Ende ickke Svaret Skatt hvercken til dend Norske eller Svenske øfrighed; Efter Rættens tilspørsel forcklarede hand Sit Sæde paa \textit{Haarkølen}, Fra \textit{Penningkiesene} i væster til \textit{Buursklimpen}, J Nord til \textit{vatsdalen}, i Søer til imod \textit{Skougen;} Som hand Vil meene, Skal \textit{Penningkiesene} ligge omtrent midt paa \textit{Haarkols}-fieldet, og er det breedt over Een Modelig Dags-Reise. ‒\par
3 Spørs: Svarer: Hands Fader har Siddet Som husmand paa \textit{ornes}-field hands \textit{Far}- fader paa \textit{Borje}-field dend Svenske Siide. ‒\par
4{de} Spørsmaal: \textit{Resp:} Gemeenligen har hand Siddet paa. Væstere Enden af \textit{Haarkølen} med Sin \textit{Familie} alleene: Dog undertiiden komme og Svenske \textit{Finner} og Sætte Sig der; Sae længe disse bliive Noget viidt borte fra ham Nær \textit{Penningkiesene}, Saa Spørge de ham ickke derom: men naar de Nærme Sig til hannem i dend Væsterste Ende, Saa maae de tage Forlov af ham. ‒\par
5{de} Spørs: \textit{Resp:} Hand meener og vil at hands Lands Stræckning Skal gaa til \textit{Penningkiesene} et Lande-Mærcke, om øfrigheden det Saa vil.\par
6{te} Spørs: \textit{Resp:} Hans Sæde paa \textit{Haarkølen} Som gaar til de Svenske grændser er Saa beskaffend, at der falder haardt vejr ind om Viintern, hvorfore hand maa Søge dend østere Ende af \textit{Haarkølen} Need ad Skougen i \textit{Jemteland}, for at faa luunet og Mildere luft for hands Dyers \textit{Conservations} Skyld. ‒ ‒.\par
7: Spørsmaal: \textit{Resp:} Gaardene i \textit{Søndre}- og \textit{Nordre-Finlier} i \textit{Sneaasens} Præstegield.\par
8 Sp: \textit{Resp:} J Søer i \textit{Kiellingsnasens}-fielde i \textit{Sneaasens} Præstegield Sidder dend \textit{Finn Nils Andersen}, j \textit{Gietings}-fieldet i Samme Præstegield dend \textit{Fin Joen Fridrichsen}, Der fra er \hypertarget{Schn1_47201}{}5 og 6 Vidne i Namdalens Fogderi. dend Nærmeste i \textit{Giørms} Skouge dend \textit{Finn Zacharias Olsen}, der fra i Nord Siddendes; Norden for denne \textit{Zacharias} tilholder \textit{Anders Torkelsen} i \textit{Blaa Søe}-Skougen ‒\par
9 Spørs: Som 1{te} Viidne af \textit{Lap} Finnerne. ‒\par
10 Spørs: \textit{Resp:} Østen for hands Sæde paa \textit{Haarkølen} paa dend Svenske Siide Sidde 2: \textit{Lap-Finner, Lars Nilsen Finne}-Lensmand, og Viidnets Vær-Fader, og Anders Jensen, viidere i Nord Paa \textit{ornes}-field Næfner hand de Samme Svenske \textit{Finner}, som 1{te} Viidne af \textit{Lap Finnerne} ved dette Spørs: ‒\par
11{te} Spørs: Svarer: Hand veed icke af fleere grændse-Mærcker at Siige end \textit{Penningkiesene} paa \textit{Haarkølen}, og \textit{Avindsbæchen;} Thi hand har værit Saa føje tiid Nemblig Siden i vaar i disse \textit{Giørms}-Skouge; at hand ickke veed grændserne der af. ‒\par
Til 12{te} Spørs: \textit{Resp:} Liigesom 1{te} Viidne af \textit{Lap-Finnerne} ved dette Spørsmaal har udsagt. ‒\par
Til 13{de} Spørs: \textit{Resp:} De Svenske \textit{Finner} vil til Eigne Sig heele \textit{Haa(r)kølen} til dends væstere Ende \textit{inclusive} Samt \textit{Avinds}-bæcken, Siigendes at det tilhører dennem; Viidnet at hand Sidder i Væstere Enden af \textit{Haarkølen}, der til har hand \textit{Missionairens} Hr. \textit{Povel Muuses} forlov, ej heller have de Svenske \textit{Finner} enten Ancket paa eller hindret hannem, at komme diid, eller at Sidde der. ‒\par
14{de} Spørsmaal: Svarer som 1{te} Viidne af \textit{Lap: Finnerne}. ‒\par
15{de} Spørs: \textit{Resp:} Fra \textit{Penningkiesene} til væstere Ende av \textit{Haarkølen} eller til \textit{Buursklimpen} kan der være en god halv dags Reise, og Fra \textit{Buursklimpen} til 1{te} gaard i Søndre \textit{Finlie} kan omtrent være 1: miil Veis. ‒\par
16{de} Liigsom 1{te} Viidne af \textit{Lap-Finnerne} ved dette Spørsmaal. ‒\par
17 Spørsmaal: Svarer: Viidnet Selv, Saa og dend \textit{Finn} i \textit{KiellingsnaseneNils Andersen} med Sine brødre ere de kyndigste der i ‒\par
18 Spørs: \textit{Resp:} Veed ickke, Saasom hand ickke Saa langt Nord er beckiendt. Hvorpaa hand blev \textit{dimitteret}. ‒\hspace{1em}\par
6{te} Viidne af \textit{Lap-Finnerne}\par
1{te} Spørs: Heeder \textit{Joen Andersen}, Føed i \textit{Blaa-Søe-} og \textit{onker}-vands-Skouge, hvor hands Forældre have altiid Siiddet, i \textit{overhaldens} Præstegield\textit{Nommedalens} Fogderie, døbt i \textit{Harrans Annex} Kiercke af dend tiids Præst Hr. \textit{Hans Reisner}, er opføed hos hands Forældre, af dend tiids \textit{Missionair Erich Helset} underviist i hands Christendom, 50 aar gammel, er gift med en Norsk \textit{Finne} koene fra \textit{Wæfsen}, har 6: børn; Med hands Alter-gang har det sig, liigesom med det 3{die} Viidne \textit{Zacharias Olsen}, ved dette 1{te} Spørsmaal har forcklaret; ‒\par
2 Sp: \textit{Resp:} Hands \textit{Finn}-Sæde Stræckker sig Sønden i fra at reigne efter dend østlige \textit{linie} fra \textit{Blaa}-Søens Søndere Ende med dends østere Ende langs efter den Elv, Som kommer af \textit{Loyre}-Vand i Nord-ost, Saa fra \textit{Løyre}-vand i Nord igiennem Skoug og Myhrland til dend Elv \textit{Rast-giuch}, og Siiden langs efter denne \textit{Rast-giuch} oppaa Ryggen af \textit{Børje}-Field; Derfra gaar dend i væster igiennem \textit{Naames}-Vand Største deels og indtager dends Søndre Siide, Siiden gaar den der fra i Søer over \textit{Ranserens}-Elv til imod \textit{Wagterens}-vand Saa der fra i øster efter dend Nordre Siide af \textit{Utnes}-vand indtil dend Søndere Siide af \textit{Blaa}-Søen. ‒\hypertarget{Schn1_47560}{}Schnitlers Protokoller II.\par
\textit{Blaa}-Søe Stræcker Sig fra Nordost i Sydvæst, Lang 2 1/2 miil, og 3/4 miil breed, ligger i Nord fra \textit{Giørms}-vandet, 1 miil, og i Søer fra \textit{Onker}-vand 1/2 miil. ‒ \textit{Onker}-vand Stræcker Sig fra øster i væster, er 1: Miil lang og 3/4 breed, ligger fra \textit{Blaa}-Søen i Nord 1: Miil og løeber af dends Sydostlige Ende en Elv i \textit{Blaa}-Søens østere Ende ‒\par
\textit{Loyre} vand Stræcker sig fra øster i væster 1/2 Miil lang, og 1/8 Miil breed, ligger i Nord ost fra \textit{Onker}-Vandet 1/4 miil; Fra dette \textit{Loyre}-vand udgaar en Elv i Syd-væst, og for Eener Sig med \textit{onker}-vandets Elv, hvor med dend løeber i dend Nordostlige Siide af \textit{Blaa}-Søen ‒\par
Væstere \textit{Raukie} vand Strækker Sig fra væster i øster 1/2 Miil lang og 1/8 Miil breed, ligger i Nord fra \textit{Loyre}-vand goede 3: Miil; Ved dette Væstere \textit{Raukie} vand giiver hand om \textit{Rastgiuchen} dend forcklaring, at dend Elv \textit{Rast-giuchen} opkommer udaf \textit{Børje}-fieldet i Nordvæst fra væstere \textit{Raukie}-vand, løeber fra Sit udsprang i Sør indtil dend kommer liige i væster imod væstere \textit{Rauch}-vand, da giør hun en bugt paa Sig ad øster, og løeber omtrent 2 bøsse Skud i øster ind i væstre \textit{Rauche}-vand. ‒\par
Østere \textit{Rauche}-vand Strækker Sig fra øster i væster 1: miil lang og er 1/2 miil breed, ligger østen for væstere \textit{Rauche}-vand 1: bøsse Skud. ‒\par
\textit{Naames}-vand Stræcker Sig fra øster i væster 2 miile lang og 1/2 miil breed ligger væsten for \textit{Rast} Elven en kort dags Reise, Som man giætter til at være 4{re} Miil. ‒\par
\textit{Rantzeren}-Søe Stræcker Sig fra Syd ost i Nordvæst 1/2 Miil lang, og en kort 1/4 miil breed, liigger Sønden for \textit{Naames} vand 1: Miil ‒ J \textit{Naames}-vand og \textit{Rantzerens}-Søe fanges alleene øret, i de øfrige vande øret og Røe. ‒\par
Et liidet Field \textit{Auchienvielle} ligger i Syd-væst 1 miil fra \textit{Rastgaiuchen}, hvor den giør Sin bugt, er langagtig fra Søer i Nord mod 1/2 miil over, Som Viidnets Fader bruger.\par
3 Sp: \textit{Resp:} Svarer Ja: og hands Faders bøxel Sæddel skal være hos Provsten Hr. \textit{Morten Lund} i \textit{overhalden} ‒\par
Til 4{de} Sp: \textit{Resp:} De ere 4{re} Madlauger el: \textit{Familier} Sammen, og udgiøre 23 Mennisker i alt ‒\par
5 og 6 Sp: Svarer Det Støder til Sværrigs grændser ‒ Om Sommeren holde de til i dette Field Rom: men om vintern formedelst at Sneen fryser for haardt Sammen, maa de fløtte enten til \textit{Haarkølen} eller til \textit{Børje}-fieldet at finde føde til deris Dyer ‒\par
Til 7{de} Sp: Svarer Som 1{te} Viidne af \textit{Lap-Finnerne} ved dette Spørsmaal ‒\par
Til 8{de} Sp: \textit{Resp:} Paa dend Søndere Siide af hands \textit{Territorium} Sidde \textit{Ole Nilsens} børn i \textit{Giørms} Skougen, Som før ere beskrevene; Paa dend Nordre Siide fra ham Norden for \textit{Børje}- fieldet er den \textit{Lap-Finn Thomes Andersen}, hvis lande Støder Sammen med hands. ‒\par
9 Sp: \textit{Resp:} Som 1{te} Viidne af \textit{Lap-Finnerne} ved dette Spørsmaal. ‒\par
10 Spørsmaal: Svarer: Midt østen for hannem er en Svensk \textit{Lap-Fin Pouel Bentsen;} Viidere Svenske \textit{Finner} i Nord kiender hand icke. ‒\par
Til 11{te} Sp: \textit{Resp:} De grændse-Mærcker hand veed ere følgende \centerline{\textit{PenningkieseneAvindsbæchenLoyre-vand}}{\textit{Rast-giuch} eller Elv, langs efter Som dend gaar i Nord ‒}\hypertarget{Schn1_47864}{}6 Vidne i Namdalens Fogderi.\par
12 Spør: \textit{Resp:} Som 1{te} Viidne af \textit{Lap-Finnerne} ved dette Spørsmaal. ‒\par
Til 13{de} Sp: Svarer: Jnden hands Faders Lands Stræckning 1: miil liggendes i Sydvæst fra dend Elv \textit{Rast-giuchen}, er et Liidet Field, af Nafn \textit{Auchienvielle}, hvilcket i Enden af 2{det} Spørs: er beskreevet; Dette Field vil de Svenske \textit{Lap-Finner} tilEigne sig af dend aarsag, at dette Field \textit{Auchienvielle} Skal ligge østen for dend \textit{linie} af \textit{Lifft}-vandet, hvor de vil have Grændse Skillet, der hos forcklarendes, at en Elv kommer fra \textit{Børje}-fieldet væsten for \textit{Rast-giuchen}, en god 1/2 Miil der fra, løbendes en halv dags Reise i et liidet Vand kaldes \textit{Lift}-vandet, Som er 1/2 miil lang og 1/4 miil breed, og ligger Ræt liige i væster 1/4 miil vejs fra det field \textit{Auchienvielle}, af dette \textit{Lift}-vand gaar en Elv viidere i Søer ind i dend østre Ende af \textit{onker}-vandet paa dend Nordre Siide; Hvilcken Vand-\textit{Linie} i alt de Svenske foregiive at være grændse Skiellet; Viidnet der imod Siiger, at hand, og hands Fader haver Rætt til dend grænse gang, Som hand har opgiivet, af dend Aarsag, at deris Forfædre have brugt og haft i besiddelse dend Samme fra u-mindelige aar; Jmidler tiid naar de Svenske vil benytte Sig af dette field \textit{Auchienvielle} beder endeel der af hands Fader om forlov hertil ‒\par
14{de} Sp: Svarer det Samme Som 1{te} Viidne af \textit{Lap-Finnerne} ved dette Spørsmaal. ‒\par
15{de} Sp: \textit{Resp:} Fra \textit{Rast-giuch} hiid til \textit{Harran} kan være 2 fulde Dagers Reiser, og til \textit{Nordre-Finlie} 1 1/2 dagers Reise; Naar Rætten nu \textit{examinerer} en leedig, løs, og ung \textit{Fins} dagsReise. Saa gaar hand Riigtig 8{te} maalte Miile om Dagen, følgelig Skal der blive fra \textit{Børje}-Field til \textit{Nordre Finlie} efter dend giisning 12: maalte miile. J almindelighed har Rætten mærcket af disse \textit{Finne}-Viidner, at Siiden faae af dem har vanchet meget i bøjdene, hvor Miilene findes ved Stolper afpælede, Saa kan man ickke forsickre, at det Miil-Maal, Som de have opgiivet, Skal være riigtig. ‒\par
Til 16 og 17{de} Svarer: Det Samme Som nest forrige Viidne.\par
Til 18{de} Svarede: at det er bæst i \textit{Wæfsen}. hvor hen kan være 18 Miil i Nord herfra. Dett Samme fandt og tilstæde værende \textit{vice}-Lensmand for bæst, Siigendes at de \textit{Finner} paa- og Norden-for \textit{Børje}-field, Søege Kiercke i \textit{Wæfsen}. hvor paa Rætten paa dette Stæd blev Sluttet. og Samtlige \textit{Finne}-Viidner indkaldet og dennem Deris Kongl: May{ts}\textit{order} i \textit{Laug}- Rættets overværelse betydet, at Saa Snart de i tilckommendes aar eller Siiden Skulle faa høre, at de Kongel: Norske Grændse-\textit{Committerede}, Som lade opMaale grændse Gangen, Skulle anckomme paa deris fieldz grændser ved \textit{Frostviigen} i Nordre \textit{Finlie}, Skal Sig u-opholdelig hos dennem indfinde (1) de afhørte Viidner (2) de Som kan tienne til vejviisere Nembl: \textit{Zacharias OlsenJoen Siursen} og \textit{Joen Andersen} tillige med en 16 à 20: Mand \textit{Finner} til Arbejdere med øxer at rødde Skoug i grændse-gangen, og at giøre Flotter eller Broer over Vande og Elve, eller at have baade ved haanden, hvor de faaes kan, at Sætte dem over paa; J Sæhrdelished betydes \textit{Breed Thomes Thomesen}\textit{Lap Finn}, Som har Sit \textit{ordinaire} Sæde paa Væstere Ende af \textit{Haarkølen}, noget tilforn betiids at ind finde Sig ved det Lande-Mærcke \textit{Legsterklompen} for at vej-Viise de Kongel: \textit{Committerede} over \textit{Haarkølen;} hvilcket Tolcken for dennem paa \textit{Finsch} forcklarede, hvor efter denne \textit{Act} af Laugrettet tilligemed blev underskrevet og forseiglet ‒\par
Baard Jsachsen Bya (L. S.) Joen Larsen Næs (L. S.) Peter Schnitler mpp (L. S.)\hypertarget{Schn1_48114}{}Schnitlers Protokoller II.
\DivII[Bilag om Overhalla prestegjeld]{Bilag om Overhalla prestegjeld}\label{Schn1_48116}\par
\textit{Bielage} til \textit{Overhaldens PræstegieldNommedals Fogderie}.\par
\textit{Harran} er dend østligste bøid af \textit{overhaldens} Præstegield i \textit{Nommedals}-Fogderie; Dend Kiercke kaldes \textit{Gartlands} af Gaarden, den staar paa, eller \textit{Glashougens-Annex} af dend Jordhoug, den er bygget paa.\par
Denne Bøids \textit{district} gaar ickke længere, end fra dend østligste gaard \textit{Aasmulen} 1/2 Miil i Skougen; Der fra er en 10: miile til grændse-Skillet imellem Norge og Sverrig, som allene af \textit{Lap-Finner} indehaves, og bestaar af Fielde, Skouge, Elve og vande jndtil dend væstre Ende af \textit{Ornes}-field; Gaardene af dette \textit{Harran} ligger langs efter \textit{Namsen} Elv paa begge Siider ‒ og \textit{Naames}-Vand ved \textit{Børje}-fieldets Søndere Siide omtrendt ved Mitten af fieldet, dog paa dets væstere Deel; Hvilcket \textit{Naames}-vand Stræckker Sig fra øster i væster 2 Miil lang, og 1/4 Miil breed, Fiskeriigt paa Øret; Det benyttes af \textit{Lap-Finnerne}, og giiver \textit{Namsen}-Elv og \textit{Nommedals} Fogderie deris Nafne; Thi af dette \textit{Naames}-vand udspringer \textit{Namsen}-Elv, løbendes først i Væster, Siiden i Sydvæst igiennem \textit{Overhaldens} Præstegieldz\textit{Harrans- Grongs- Ranums} og \textit{Skagøes} Kiercke-Sogner, omtrent 19 Miile i \textit{Biørums}-Fiord i Havet. ‒\par
J denne \textit{Namsen}-Elv løebe mangfoldige Elve; der af de Største fra øster paa dend Søndere Siide indgaar, Norden fra at reigne. (1) af \textit{Ranseren}-vand en Elv omtrent 1 Miil lang;\par
(2) Sønden der for af \textit{Udnes- Wagterens}- og \textit{Saxer}-vande. Elven omtrent 1: Miil. ‒\par
(3) Derfra i Sør af \textit{Tunsøe}\textit{Tunsøe}-Elv omtrent 3 Miile lang;\par
(4) Viidere i Søer \textit{Grøndals}-Elv, 2 Miile østen for den østligste Gaard i \textit{Harran, Aasmulen}, 2 Miile lang. ‒\par
(5) Saa og \textit{Næs}-Elv af \textit{Gietings}-field udkommendes omtrent 4 miile lang. ‒\par
(6) \textit{Sandøla} Elv først oven Paa \textit{Haarkiølens} field ickke langt fra \textit{Penningkiesen} udrindendes, har det Nafn \textit{Lurra}, Siiden efter at have \textit{passeret} i væster igiennem \textit{Sand}-Søen og \textit{Lags}-Søen i Nordre \textit{Finlie}, faa Sit Nafn, \textit{Sandøla}, og naar den de Smaa \textit{Bratlandt- Lille- Skielbreens}- og \textit{Otter Søe}-Vande har igiennemgaaet, iiler den, over 12 Miile lang, i \textit{Namsen} Elv. ‒\par
Førend denne \textit{Sandøla} Elv i \textit{Namsen} Sig udgyder falder fra Sydost i Nordvæst ved en Platz, kaldet \textit{Rohylbaken Lurru}-Elv, en 10 Miil lang derj; Og i \textit{Lurru} Elv, 1 god Miil førend dend foreener sig med \textit{Sandøla}, Løber \textit{Almen}-Elv fra \textit{Finvollen} (en Dahl imellem \textit{Bugvands}- field og \textit{Brænds}-field) mest fra øster i Væster, omtrent 7: Miile lang.\par
Paa den Nordre og Nordvæstlige Siide løebe Norden fra i \textit{Namsen}-Elv:\par
(1) af et vand paa \textit{Børje}-Field\textit{Guaude}-Elv,\par
(2) Sønden der fra af \textit{Frænnings}-vand en Elv\par
(3) Siiden \textit{Biørnhuus}-Elv ‒\par
(4) Endnu \textit{Flotdals}-Elv ‒\par
(5) \textit{Linset}-Elv ‒\par
og (6) af \textit{Høeland}-Vande\textit{Biøra}-Elv heel-kroget, Nord-østen fra \textit{Ranums} hoved-kiercke, 1/2 Miil lang.\par
Af disse \textit{Høelands} vande heeder det første Sydligste \textit{Yttere}-Vandet, 1/2 miil lang fra Sydvæst i Nord-ost; Det andet Mellemste, \textit{Grungstad}-vandet, derved liggendes, 1/2 miil lang liigeleedes i Nord-ost; Det 3{die} Vand, \textit{Øvre}- eller \textit{Øy}-Vand kaldet af en Gaard der ved, mest \hypertarget{Schn1_48507}{}Bilag til Overhaldens Prestegjeld. Nordlig liggendes, begynder først Som et Smalt Sund 1/4 miil lang, Siiden udviidendes sig, er 1 Miil lang; Østen for de først opReignede østlige Søer, hvor af Elvene løebe i væster i \textit{Namsen}-Elv, ligge Andre Vande, hvor af Elvene riinde fra væster i øster, saasom\par
(1) \textit{Frostviig}-vand, 1/4 miil østen for \textit{Tunsiø}, og 1/2 Miil Norden for \textit{Qvæsiø} i Nordre \textit{Finlie}, er 3 Miile lang, mest fra væster i øster, og imod 1/2 Miil breed; Af dends østlige Ende Elven Riinder gandske kort, i dend Nord væstlige Ende af \textit{Heidugeln}-vand. ‒\par
J dette \textit{Frostviigs}-vands Nordvæstlige Ende falder en Bæck af \textit{Liimings}-vandet, hvilcket er 3 Miile lang mest Sydostlig, og 1/2 Miil breed, og ligger Norden for \textit{Tunsiø}-vand 1: Miil. ‒\par
(2) Norden for \textit{Frostviig}-vand 1 Miil er \textit{Giørms}-vandet 2 1/2 Miil lang og 1/2 Miil breed; Der af gaar Elven af dends Sydostlige Ende i \textit{Frostviigens} Nordostlige Ende.\par
(3) \textit{Blaa}-Søe, 1 Miil Norden for \textit{Giørms}-vand 2 1/2 Miil lang og 1/2 Miil breed; der af ved dends Sydostlige Ende løeber bæckke i \textit{Giørms}-vandets Nordostlige Ende. ‒\par
(4) Liigesom \textit{Unker}-vand, 1 Miil Norden for \textit{Blaa}-Siøen, 1 Miil lang, og 1/4 Miil breed fiindes, hvor af Vandet i \textit{Blaa}-Søens Nordostlige Ende falder. ‒\par
(5) J Nord ost fra \textit{Unker}-vandet ligger et mindere vand, \textit{Løyre} kaldet, hvor af Elven i Sydvæst Rinder i \textit{Unker}-Vand-Elven. ‒\par
(6) Liige i Nord fra \textit{Unker}-vandet, væsten for dend field-klimp \textit{Aukienvielle} Skal være \textit{Lift}-Vandet, hvis Elv Rinder i Søer i østere Ende af \textit{Unker}-Vandet. ‒\par
(7) Norden for dette \textit{Lift}-vand udkommer en, kaldet \textit{Gaust}-Elv, af \textit{Børje}-field væsten for \textit{Rast}-Elven, gaar i Søer i \textit{Lift}-vandet. ‒\par
Disse 7: vande og Elve vil merckes, i fall de Kongl: Svenske Grændse-Maalere derpaa skulle giøre nogen \textit{Reflexion} ‒\par
\textit{Overhaldens} Præstegield, hvoraf dette \textit{Harran} er et \textit{annex}, har 5 Kiercker, nembl: \textit{Harrans-} Det østligste \textit{Grongs-} det næste \textit{Annex}, der efter \textit{Ranums} hoved kiercke, og Saa \textit{SkagøeAnnex}, hvilcke 4{re} Sogner alle ligge fra øster mæst i væster langs efter \textit{Namsen}-Elv, og Gaardene der af paa dens Nordre og Søndere Siide. 2 Miile i Nord fra \textit{Ranums} Hoved-Kiercke ligger den 5{te} som er en \textit{Annex}-kiercke, af Nafn \textit{Romsta}-Kiercke, ved \textit{Høelands}-vande. \par
Fra dend østerste Gaard i \textit{Harrans} bøid, \textit{Aasmulen}, er til dens, kaldet \textit{Gartlands}- Kiercke{2 miile} fra denne \textit{Harrans}- eller \textit{Gartlands}-Kiercke til \textit{Grongs}-Kiercke i Sydvæst {1 1/2 miil} fra \textit{Grongs}-Kiercke i væster til \textit{Ranums}-hoved-Kiercke{1 3/4 miil} herfra til \textit{Skagøe Annex} Kiercke i Sydvæst {1 miil} og fra \textit{Skagøe} kiercke til \textit{Biørums}-fiord i Sydvæst {1/2 miil _______ = 6 3/4 Miil}\label{Schn1_48817} \par 
\begin{longtable}{P{0.687542662116041\textwidth}P{0.031911262798634814\textwidth}P{0.03481228668941979\textwidth}P{0.06382252559726963\textwidth}P{0.031911262798634814\textwidth}}
 \hline\endfoot\hline\endlastfoot \textit{Harrans} bøjd Skal bestaae af\tabcellsep 14:\tabcellsep gaarder,\tabcellsep og 31:\tabcellsep bønder,\\
\textit{Grongs Annex}\tabcellsep 28\tabcellsep ‒\tabcellsep - 88\tabcellsep ‒\\
\textit{Ranums} Hoved-Sogn\tabcellsep 48\tabcellsep ‒\tabcellsep - 96\tabcellsep ‒\\
\textit{Skagøe Annex}\tabcellsep 33\tabcellsep ‒\tabcellsep - 54\tabcellsep ‒\\
\textit{Romstad} eller \textit{Hølands-Annex} Som liger væstlig fra \textit{Harrans}- Kiercke og Nordlig fra \textit{Ranums} Kiercke\tabcellsep 26\tabcellsep ‒\tabcellsep - 63\tabcellsep ‒\end{longtable} \par
 \hypertarget{Schn1_48906}{}Schnitlers Protokoller II.\par
Bønderne over Hovedet leeve af deris Korn-Jorder, som Skal være \textit{passables}, naar det ickke er Frost-aar, Som nu i nogle Aar har været, og af Fiskerie af Øreder og Lax i \textit{Namsen}- Elv og \textit{Høelands} vande; Homlen her, Som voxer, Skal Sæhrdeelis være goed. De af \textit{Harran, Skagøe} og \textit{Høland} have og Skoug at hugge Tømmer af til Saugbrug; Endeel bønder her i \textit{Harran}, liigesom i \textit{Finlierne}, bruge nogen Smaa Handel med \textit{Lap-Finnerne:} dog er \textit{Namsen} Elv ved Vaar-Flommen Saa Striid, at den gemenligen forvolder Elve-brud, og ved Sin Stærcke fart og oversvømmelse aarligen bortskyller noget af Elv-Mælene eller breedene. ‒\par
J dette Præstegield er at agte, at hvert Aar i \textit{September} Maaned er i \textit{Skagøe Annex} paa \textit{Halvor-Moe} Mælen ved Elven 1: liiden miil fra Fiorden, et Marcket af Feede Vahrer, Talg, Smør, Kiød, Buckke-skiind, Meel, Malt og Fisk.\par
Skiie-løber \textit{Soldatere} ere i \textit{Harrans- Grongs-} og \textit{Høelands- Annexer} 30: mand; de andre i væster ere Staaende \textit{Soldatere}. ‒\par
Veie gaar jngen herfra til \textit{Sverrig} i øster: men de Som didhen eller derfra hiid ville, gaae igiennem \textit{Søndre-Finlies} Vand-dal jgiennem \textit{Sneaasen} østen for \textit{Gied}-Field til \textit{Grong}, eller igiennem Nordre-\textit{Finlie} 9: miile over \textit{Gietings}-Field need til \textit{Harrans}-bøid. J afvigte Aar ere adskillige \textit{Jemte-Familier} fra \textit{Jemteland} hid over komne, hvilcke foregave, at hungeren har drevet dem af Landet; Der af fornemmer at 5 \textit{Familier} i \textit{Harran-}, 2 i \textit{Høllands-Annex} ere igien blevne, de fleere er gaaet til \textit{Foesnes}-gield, og Sæhrligen Søege de til \textit{Wæfsen} i \textit{Helgelands} Fogderie i Nordlandene. ‒\par
Om Landskabet østen for Grændserne i Sverrig har en født \textit{Jemt}, nu en Norsk bonde paa den liiden field-Gaard \textit{Rosendal} i den skoug jmellem \textit{Biørnlie} gaard ved \textit{Oy}-vand i \textit{Høland}, og \textit{Foldens} Fiord, Nafnlig Abraham, berettet mig, at norden for den Gaard \textit{Ringsiø} paa dend Nordre Siide af \textit{Hotagens}-vand i \textit{Jemteland}, 1 1/2 miile ligger en Gaard \textit{Seliaasen}, hvor en bonde boer paa, dend Sidste og Norderste Gaard i \textit{Strøms Annex}\textit{Hammerdals}-Gield i \textit{Jemteland}. ‒\par
Norden for dette \textit{Hammerdals} Præstegield er den Svænske \textit{Province Aangermanland}, stødendes til Norge i Væster: dog skal der imellem de Svenske og Norske bønder-Gaarder fra øst- i Væster en \textit{distance} af 20 Miile være; dette \textit{Aangermanland} er kun smalt fra Søer i Nord; bem{te} Norske bonde havde hørt 4 Præstegield derj næfnt, hvor af det væsterste er \textit{Ramselens} Gield, havendes 5 Kiercker (1) \textit{Ramselen} hoved Kiercke, (2) Field Siø nærmest til Norge, hvorj Gaarden \textit{Taasiø} ligger (3) \textit{Egslee} (4) \textit{Heriom} (5) N: \textit{Annexer}\par
(2) \textit{Aasela}-Gield Norden der for, hvor af Meenigheden mest ere \textit{Lap-Finner}; J dette \textit{Aangermanland} ere ickke Soldattere men botsmænds lægder. Skiellet imellem \textit{Jemteland} og \textit{Aangermanland} giør en houg fra øster i væster 1/2 Miil, Norden for den Gaard \textit{Ringsiø}; Paa den Nordre Siide af denne houg 1 Miil herfra og liige i Nord fra \textit{Ringsiø} er den nærmeste Gaard i \textit{Aangermanland}\textit{Ramselens} Præstegield\textit{Fieldsiø Annex}, Som og er den væsterste derj, nærmest de Norske grændser, hvor 1: bonde paaboer Nafnlig \textit{Taasiø}; disse 3 benæfnte Svænske Gaarder, nembl: \textit{Ringsiø, Seliaasen}, og \textit{Taasiø} have den \textit{Situation}, at derimellem ligge Skoug og Vande, intet Field; Bønderne der leeve af deris Gaarder, Som dog ere frostnæfnte, Fiskerie og Skytterie ‒ \textit{Difference}, om Grændse Skiellet vil mueligens møde os her; Thi forbem{te} nu Norske bonde Abraham har fortalt mig, at for 2 aar Siiden, da hand var i \textit{Jemteland}, og \textit{Almuen} om Grændse-Skiellet jmellem Norge og Sverrige paa Tinget er bleven forhørt, have bønderne \hypertarget{Schn1_49261}{}Bilag til Overhaldens Prestegjeld. der udsagt, at deris \textit{district} gick til \textit{Finlierne} i Norge Saa vidt i væster, Som Vandene ere, der falde i øster ‒ Denne \textit{Relation} kommer nu over Eett med de Kongl: Svenske-Grænse-\textit{Commissariers} i fordum tiid deris \textit{moverede} Fordring paa Søndre-\textit{Finlie} og endeel af \textit{Nordre Finlie}; Thi om \textit{Holden- Langlingen- Uull etc:} Vande, der alle riinde i øster, ligge Søndre \textit{Finlies} Gaarder, og ved \textit{Qvælie}-Elv og \textit{Frostviig}-vandet ligge Gaarder af Nordre-\textit{Finlie}, nemblig \textit{Leerbachen, Qvæsiø}, og \textit{Nybyggerne} ved \textit{Frostviig}. Paa dette \textit{Fundament} vil og de Svenske mueligens udstræckke deris \textit{pretention} til de vande og Elve, Som løebe i øster, i den Stræckning jmellem Nordre-\textit{Finlie} og \textit{Børje}-field; Thi de Norske \textit{Finner} have i Forhøeret udsagt: at Norges Grændser (NB om \textit{Svanesteenen} mælde de ickke eet Ord) gaae fra \textit{Avindsbækken} til den field-klimp \textit{Saxolaa-vara} Norden for \textit{Heidugeln}, over den væstere Ende af \textit{Ornes}-field nembl: de klimpe \textit{Giepskiach, Rautekie-vara, Varsans-vara}, Siiden i Nord langs efter \textit{Rastjuch}-Elven, indtil \textit{Børje}- field; Andre Norske \textit{Lap-Finner} af Samme \textit{district} Nemblig \textit{Jacob Olsen} ved \textit{Ramserens}-vand, \textit{Ole Zachariasen} og \textit{Joen Andersen} i \textit{Blaasiø}-Skougen ville, at grænse-Gangen skulle gaa fra \textit{Løyre}-vand Nordlig fra \textit{Varsans-vara}, til i \textit{Rastjouchen} og viidere; Derhos Sagde, at dennem af de Svenske \textit{Lap-Finner} Skeede \textit{disputer}, som ville gaae væsten for \textit{Rastjouch}-Elven og den field klimp \textit{Aukienvielle} til den Elv \textit{Gaustjouch}, Som falder Nord udaf \textit{Børje}-field og gaar i \textit{Lift} vandet i Søer, og Siiden i de østere Ender af \textit{Unker}-vandet, \textit{Blaa-Siøn}, \textit{Giørms}- og \textit{Frostviig}-vandene; Jeg kan og ickke lade u-anmældet, at Samme Sidste \textit{Finner} sluttede, at grændsene gick i de østlige Ender af \textit{Unker- Blaasiø- Giørm-} og \textit{Frostviig}-Vand fra \textit{Løyre}-vand: men Som denne Slutning Striidede med de andre første Viidners udsagn, førtes den ej ind i \textit{Acten}. ‒\par
Aarsagen til de Norske \textit{Finners} Viidne og \textit{Fundament} til deris paastand bestoed derj, at deris Forfædre, saaledes som de det udsagt, have haft dend Stræckkning i besiddelse og brug: ihvorvel de Sidst benæfnte Norske \textit{Finner} der hos forcklarede, at de Svænske \textit{Finner} benyttede Sig og liigesom de Norske, undertiiden af den field-klimp \textit{Aukienvielle}, til Foeder for deris Dyer, væsten for \textit{RastJouch}-Elv liggendes: Der imod de Svenske maa have til grund af deres \textit{prætention} Vandfaldet ud af Søene ad øster. Betydenheden af denne \textit{Qvæstionerende} grænse-Stræckning paa vor Siide bestaar derj, at Landet omckring \textit{Frostviig}, \textit{Giørms}-vand, \textit{Blaasiø}, og det væsten for liggende \textit{Liimings}-vand er beqvæm til Rødning og bebyggelse, hvormed tiiden \textit{Colonier} af bønder eller \textit{Lap-Finner} kan sættes, Som og for nyelig 2{de} Norske bønder af Nordre \textit{Finlie} have nyebygget ved \textit{Frostviig}-vandet; Saa det var tienlig, at den \textit{district} til væstere Ende af \textit{Ornes}-field for Norge; blev behauptet ‒\par
\textit{Lap-Finner} østen for \textit{Overhaldens} Præstegield ere de Samme, som \textit{Missionairen} af \textit{Sneaasen} har under opsiun til underviisning og Kierckelig betienning; Saasom de og ligge nærmere til Nordre \textit{Finlie} et \textit{Annex} af \textit{Sneaasen}, end til \textit{Overhalden}; Her hører ieg nu, at i forrige tiider disse \textit{Finner} Skal have svaret Skatt til deris kongl: May{t}, og Tiende til Kiercken: men i afdøede \textit{Lector von Westens} tiid derfra Siiden være friitagne; De tage og ingen Nye bøxel- eller fæste-Sæddel hos Fogderne paa deres Forældres hafde field-Sæder; Nu er det Saa, \textit{Finnerne} ved deres \textit{Fataliteter} af ulvens Stærcke \textit{Grassering} ere næstendeel udarmede, og ere all medliidenhed værd; Man finder og nu i Grænse-bøidene, Een og Anden af de unge \textit{Finn}-karle og børn (som har ellers været Sælsomt tilforn) lade Sig af bonden paa en Vis tiid til bondeArbeide leje, hvortil deres Slette Tilstand dennem driver: men ieg Stiller derhen og \textit{Submitterer}\hypertarget{Schn1_49637}{}Schnitlers Protokoller II. under høieres overvejelse: i fal den forehavende Grændse-Skielning ej Skulle have den forønskede fremgang eller vorde fastSatt: om det iccke var betænckeligt, at lade disse Folck Saa nær Grændsene blive Saa ganske entleediget fra ald undersaattlig \textit{Recognition} til den Norske Crone, og fra den Norske \textit{Civiile} øfrigheds opsyen; Thi der i blandt kan være, Som tillige Svare Skatt for det de Nyde paa Svensk \textit{territorio}, til Sverrigs Crone; Nu indholder det \textit{Stettinske} Freedz-Fordrag: Det Eene Riiges underSaatt skal beholde og nyde, det hand i det andet Riige har haft af Rettigheder; Hvor af Tiidens Længde kunde drages den Slutning, at \textit{Finner}, som fra \textit{arriltz} tiid til Sverrig og ickke til Norges Crone have Skattet, Skal beholde den friihed stedse, og forblive Sverrigs Undersaattere; Ei at tale om, naar en Fader af en \textit{Finne-Familie} døer, og Een af Sønnerne ej faar Nye bøxel- eller fæste-Sæddel af Fogden paa Faderens Sæde, Vickles børnene efter ham i Spliid og ueenighed; Og det kan hænde sig, at de nærmeste Svænske-\textit{Finner} kan forlockke eller fortryckke de Norske \textit{Finner} fra ett eller Andet Styckke af deris fæderne Sæde, naar de ingen Fæste-Sæddel have at forsvare sig med, eller Svare til.\par
U-forgriibelig Siunes: \textit{Finnerne} eller Een af deris Middel, af hvert Fogderie maatte hver gang til nærmeste Tingstæd frembære Samtlige deris Skatte-bøger, hvorj den \textit{ordinaire} Skatt til \textit{Debet} kunde indføres, og paa den Anden Siide den Kongl: Allernaadigste \textit{Moderation} eller aldeelis friigiivelse, enten i henseende til deres Fattigdom, eller til en opmuntring i deres Christendoms Lære afskriives; Hvorved man da \textit{Conserverede} Cronens \textit{Regale} til deris undersaatlige pligter, og tillige forskaanede dem for Skatters betaling.\par
Der er og inden \textit{Finnernes} Field- og Skoug-Sæder gode \textit{Rødnings}-Lande, Som i fremtiiden kan vendtes at vorde bebygget og \textit{peupleret}. ‒\par
Og det er det jeg til Slutning af \textit{Trondhiems} Ambt, da denne \textit{Session} i \textit{Harrans}-bøjd\textit{Overhaldens} Præstegield er den Sidste ieg i bemelte \textit{Trondhiems} Amt holder, har at tilføje, Nemblig Stæder, til Rødnings-Platzer og bebyggelse beqvæmme:\par
1: Ved \textit{Giæfsiøn} i \textit{Sneaasens} hoved-Sogn paa Grændserne, hvor om nøjere omstændigheder i Viidne-\textit{Protocollen} af Samme Sogn findes udførte. Her, er mig Sagt af Troværdige Mænd, skal findes goed Korn-land til Ager, og Saa Romt og viidt, at en anseenlig Kiercke bøjd der skulle kunde vendtes med tiiden at blive.\par
Om \textit{Gaundalen} i Samme \textit{Sneaasens} Sogn har jeg dette at \textit{Remarqvere}, at Salig \textit{Magister Nils Muus} fordum Sogne Præst til \textit{Sneaasen}, har opRøddet det nær ved Grændserne, og Satt husmænd der at boe; Hvilcken Rødnings-Platz hand og handz børn over 30 Aar har haft i besiddelse og brug, saa \textit{Conservere} og vel Samme hands børn dette \textit{Gaundalen} at vorde dyrcket og bebygget ‒.\par
2: ved \textit{Gingel}-vandet et par Miile Norden for \textit{Skielbreen}-gaard i Nordre-\textit{Finlie}, og 1/2 miil Sønden for \textit{Tunsiø}, i hvilcken og \textit{Gingel-Elven} i væster indløber ‒\par
3: Ved \textit{Tunsiøns} østere- og Sydvæstlige Ender, hvilcken \textit{Tunsiø} ligger imellem \textit{Gingel}- vandet og \textit{Liimings}-vandet. ‒\par
4: Ved \textit{Liimings}-Vandet 3 Miile lang, og 1/2 Miil breed Norden for \textit{Tunsiøen} 1: Miil, hvor af Elven løeber i øster i \textit{Frostviigen} ‒\par
5: Ved \textit{Frostviig}-vandet 3: Miile lang, hvor ved dend Nordvæstlige Siide for 2 aar Siiden 2{de} Norske bønder Sønner fra Nordre-\textit{Finlie} have needsatt Sig, og kaldes Nyebyg\hypertarget{Schn1_49863}{}Bilag til Overhaldens Prestegjeld. gere; Af dette \textit{Frostviig}-vand have nu de Svenske \textit{Lap-Finner} sig og benyttet, og Elven deraf løeber i øster i \textit{Heidugeln}, Som de Svenske mueligens os skal ville \textit{Qvæstionere:} dog have de Svenske disse 2{de} Norske Nye-byggere u-behindret og u-foruroliget hidentil ladet bygge og Sidde. ‒\par
6: Ved \textit{Giørms}-Vandet Norden for \textit{Frostviig}-vandet, 1 1/2 Miil lang og 1/2 Miil breed. ‒\par
Disse forbemelte Platzer fra N{o} 2 til 6 \textit{inclusive} ligge nu vel østen for \textit{Overhalds} Præstegield, og Skulde og kunde \textit{sortere} under \textit{Nommedals} Fogderie; Som ieg og har Seet, at forrige \textit{Nommedals} Fogder, have giivet \textit{Finnerne} Fæste-Sædler paa deris Field-sæder i denne Stræckning: dog ligge de beqvæmmere og nærmere til Nordre-\textit{Finlie}, et \textit{Annex} af \textit{Sneaasens} Gield, af hvis Præster og \textit{Missionaire} de \textit{Finner} paa denne tiid betienes; Hvorfore det Siunes, de Nye Rødnings-Platzer belejligst under \textit{Sneaasens} Præstegield kunde legges, helst den Præst kun har Een Kiercke, og de 2{de} Smaa Field-\textit{Capeller}, Nordre- og Søndre \textit{Finlier}: Der imod \textit{Overhaldens} Præst des uden har 5 Kiercker at forsiune.\par
7: Ved \textit{Namsen}-Elv i \textit{Overhaldens} Præstegield, 1/2 Miil østen for \textit{Harrans} østligste Gaard \textit{Aasmulen}, hvor en Stræckning af 3 à 4 Miile beqvæm til Rødnings-land Skal findes ‒\par
Som nu dette \textit{Trondhiems Amt}, Sæhrdeeles om gud Velsiigner Aarene med goed Kornvæxt med Tiiden Riimeligen vil blive Folcke-riigt; Saa understilles i de høieres Overvejende: hvorledes unge Folck, der have fornøden at nedsette Sig, kunde \textit{animeres} til at anlegge Nye- boliger eller \textit{Colonier} paa disse Stæder, nemblig, naar ved en lovlig Marcke Gang Stæderne til visse gaarder, og boeliger var afdeelte ‒\par
enten (1) ved Løfte, at det Skulle blive Nyebyggerens Odel og Eiendom, det hand jndtog at Rødde og bebygge, Samt friihed for Soldatter udskriivning og Skatter i en Viss lang tiid; Saaleedes har itzige Sogne-Præst paa \textit{Sneaasen}, Hr. \textit{Peter Muus} foreslaaet, og begiert, at faa den district ved \textit{Giæfsiøen} til odel allernaadigst giiven; Da hand tilbød sig, at ville lade den bebygge; Ellers kunde hertil og Merckelig tienne, om deres Kongl: May{t} ville ved Fogderne \textit{sublevere} Nye byggere med nogen forstræckning af Penge til Qvæg, og Korn, Som de med tiid og Stund igien maatte betale. ‒ Eller\par
(2) at lade Saadanne Nye-byggere frii bøxle Saadanne Platzer, dog deris Kongl: May{ts} Odel og Eiendom der af \textit{Reserveret:} mens \textit{assistere} og hielpe dem med Sæde-Korn og Cratuerer, Som dem maatte giives og Foræres, med forbem{te} Skatte- og udskriivnings-friiheder, Thj det er Gerne Fattige-Folck, som \textit{entreprenere} Saadanne Nye-bygninger, og ej have Raad til at kiøbe eller forskaffe sig nogen Gaard i Kiercke-bøiden, der for man maatte underholde dem det 1{te} Aar, da de Røddede og bygde, at de det Andet og følgende Aarer kunde vinde af Jorden Selv, hvor af de skulle leeve. Tiidens Længde \textit{Renderede} denne forskud og Gave fuldkommen, naar de \textit{Colonier} vare bragte i Stand. ‒\par
Naar og \textit{Lap-Finnerne} til Fieldz formeere sig og blive mangfoldige, Saa \textit{Submitteres} til betænckning, om de til Fieldene nærmest Rødnings-Stæder dennem til bebyggelse og dyrckelse kunde andviises; Hvortil vel forud ville \textit{præsupponeres}, at endeel deres børn hos bønder- Folck maatte gandske unge hensættes, af dennem opfødes, og til bonde-Arbejde og brug oplæres; Hvilcke Siiden paa Saadanne Rødnings Platzer nær hos deres \textit{Familier} og Folck kunde \textit{etableres;} Men her til ville \textit{Fonds} udfindes; hvor af disse unge \textit{Finne}-børen hos bønderne \hypertarget{Schn1_50071}{}Schnitlers Protokoller II. kunde opfødes, og Siiden naar de vare fremvoxne, under Armene Griibes, med ald den forstræckning, Som til en Gaards besættning kunde behøves; Tiidens Længde kunde opreise den Forskud, og \textit{Finnernes} Qviindfolck ere behændige og duelige til adskillig Hænders Gierning, Som til at Sye \textit{Mudder}, Handsker, Støfler, Skiinvaar, punge. \textit{Referenterne} af Landets \textit{Situation}, væsten og Østen for Grændse-Mærckene har til deels af Norske bønder, og \textit{Finner}, deels af forrige \textit{Missionaire} nu Tolck \textit{Erich Helset}. ‒\par
Dog staar derhen om disse \textit{Finn}-Folck er at bringe i orden til Saadan en Slags forsamling og Stadig boe-pæll; Thj af Geistlige saavelsom Værdslige betiendtere, jtem af bønder, hører ieg joe længere joe meere, at det er imod de Folckes \textit{Naturel} at lade sig Tæmme, og Sætte under en Slags ave. ‒\par
Herom nærmere skal udlade mig, naar ieg fremckommer længere i Nord. ‒\par
Om de angiivne \textit{Rundfurru}- og \textit{Demands} fielde maae Sluttelig mælde, at de ej i \textit{Limiten} er befunden: men skal ligge østen for, hin i \textit{Aaselle}-Lap-Marck, og denne under det Nafn \textit{Mars}-field østen for \textit{Farok}-vandet i \textit{Uma-Lap}-Marck ‒ Endelig er der imellem vore Viidner af \textit{Nommedals}-Fogderie dend \textit{discrepance} at de 1{te} 3{de} og 4{de} Viidner, Næfne dend væstere Ende af Væstere \textit{Raukie}-vand at være \textit{Limite:} men det 6 Viidne lader det væstere \textit{Raukie}- vand til de Svenske, og holder sig til \textit{Rastgiuch:} nu er det saa, at den er fra \textit{Rastgiuch}-Elven, der hvor dend vænder Sig i øster, til dette væstere \textit{Raukie}-vand, kun 2: bøsse-Skud, og følgelig indtet af betydenhed i denne \textit{discrepance:} dog holder for, at 6{te} Viidnes Forcklaring er vel dend Sickkerste, thj hand og hands Fader Sidde paa dette Stæd, og derfor bæst maa Viide \textit{Limiten:} Ved Øye-Siunet vil vel bæst kiendes, hvor \textit{Linien} liigest gaar fra Søndre Lande-Mærcke Nord efter. ‒\par
Angaaende 6{te} Viidne i \textit{Nommedalen}, hands anmældte Faders bøxel-Sæddel, som Skulle være hos Provsten i \textit{Overhalden}, har man ved anckomsten hos ham, d. 9 aug{ti}\textit{Reqvireret} og faat \textit{Copie} af bemelte bøxel-Sæddel, som hos legges Lit: B: ‒
\DivII[Aug. 8.-14. Fra Harran over Foldereid til Bindalsfjorden]{Aug. 8.-14. Fra Harran over Foldereid til Bindalsfjorden}\label{Schn1_50225}\par
Efter at \textit{Examinations} Rætten i \textit{Harran} i \textit{Overhaldens}-Gield\textit{Nommedals-Fogderie} Dend 7{de}\textit{Augustj}: var Endet, Reiste man derfra d: 8{de} næst efter til Landz til \textit{Næsgaard} 1/4 Miil Siiden til baadz paa \textit{Namsen}-Elv til imod Fiske-Fossene 1/4 Miil, der fra maatte man i land til Fiske-gaard; oven- og Needen for denne Fiske-gaard vare der 2{de} Fosser, og den Nederste deraf den Striideste, hvor ingen boord, Saug-Timmer langt mindre Baad Kunde \textit{Passere}, uden at Sønderslaaes; Eet Støckke væsten fra \textit{Fiske}gaard foer man til baadz igien paa \textit{Namsen}- Elv til \textit{Grong}, hvor \textit{Grongs}-Kiercke, et \textit{Annex} af \textit{Overhalden} Staar, 1 Miil her fra til den gaard \textit{Foss} i \textit{Ranums} hoved-Sogn{1 3/4 Miil}\par
D: 9 \textit{Augustj} skreevet til Hr. \textit{Povel Muus} paa \textit{Sneaasen, Missionairen} over \textit{Finnerne}, item til Hr. \textit{Oberste von Heinen}, Fogderne af \textit{Jnderøens} og \textit{Nommedals} Fogderier at \textit{assistere} de i næste Aar anckommende Kongl: Norske \textit{Jngenieur(e)ne} med at skaffe Til dennem frem de afhørte Viidner, Veiviisere og Arbeidz-Folck ‒\par
\textit{dito dato} Reist fra \textit{Ranum} over \textit{Namsen}-Elv i Nord ost siiden over \textit{Høelands-aae}, ellers kaldet \textit{Biøra}-Elv, Som af \textit{Høelands}-vandene i mange Bogter gaar kroget i \textit{Namsen}-Elv, saa ieg paa det korte Støckke vej 3 gange maatte lade mig Sætte over Samme \textit{Biøra}-Elv over til \textit{Glømmen}-gaard{1/2 Miil}\hypertarget{Schn1_50393}{}Schnitler fortsætter Reisen til Foldereid.\par
d: 10: herfra igiennem \textit{Høelands} første vand kaldet \textit{Yttre Vatten} i Nordost til Gaarden Eide 1/2 Miil Siiden til Landz over et Eid 2 bøsse Skud langt, og der efter over det Mellemste \textit{Høelands}-vand kaldet \textit{Grungsta Vatten}, af gaarden \textit{Grungsta} der ved liggendes, til det andet Eid i Nord-ost 1/2 Miil\par
herfra over \textit{Eidet} til Landz forbi \textit{Høelands}-Kiercke til Gaarden \textit{Flak} 3/4 Miil.\par
Fra \textit{Flak} til Vandz i 3{die}\textit{Høelandz}-vand kaldet \textit{Oy Vatten}, af en Gaard af Samme Nafn mest Nordlig liggendes først igiennem et langt Smalt Sund til dend Gaard \textit{Flott} hvor skiftet var 1/4 Miil.\par
Formedelst at saa tidt maatte omskiftes Snart med hæste Snart med Baad, Saa vandt man ei viidere den dag ‒\par
d: 11: \textit{Augustj} igiennem det øvrige af \textit{Oy-Vatten}, mest i Nord Streckkendes til \textit{Biørnlie}- gaard 1: Miil ‒ siger {1 Miil} fra denne \textit{Biørnlie Gaard} i Nord igiennem Skoug og Myhrland, hvor man og for Veiens u-Føere skyld maatte fare igiennem \textit{Skoug}-Elven nogle Gange til en Gaard i Skougen \textit{Rosendal} (hvor Abraham en \textit{Jemte} boede) {3/4 Miil} Her fra til \textit{Folden}-Fiord til \textit{Saur-} og Siiden til \textit{Möe}-gaarden{1/4 ‒ _______ 2 Miil.}\par
Hvilcket var en slem og besværlig vej.\par
1/4 Miil, førend man kommer fra \textit{Biørnlie}-gaard til \textit{Rosendal}, falder paa den høiere haand fra et høit Field en Foss need i Dalen, og giør \textit{Skoug}-Elven, hvilcken man Siiden Langs-efter følger i Nord til \textit{Folden}-Fiord.\par
Fra \textit{Møe} gaard ved \textit{Folden} er vel over Land liige i Nord 3 1/2 Miil til \textit{Granbosta-vand}, hvoraf Elven nedgaar i \textit{Bindals}-Fiorden 2 1/2 Miil: men den vej er ej med hæst og kløver fremckommendes; Hvorfore maatte fare efter \textit{Folden}-fiord, fra bemelte \textit{Möe} gaard i Nord væst til \textit{Falder-Eide} ‒ hvilcken \textit{Foldens}-fiord Siiden vender Sig ad Syd væst og er til Havet i alt 6 Miile lang, og 1/8 Miil meer og mindre breed over. Paa dette \textit{Falder Eide} Ligger dend Gaard \textit{FalderEide} og Kiercken \textit{Falder-eide}, et \textit{Annex} af \textit{Nærøens} Præstegield. ‒\par
Dette \textit{Nærøens} Præstegield har 5 Kiercker, (1) Hoved Kiercke \textit{Nærøe}, og 4 \textit{Annexer}, nemblig \textit{Folder-Eide}, hvorfra \textit{Nærøe} hoved-Kiercke i væst Syd væst ligger 4 1/2 Miil. ‒\par
\textit{KolverEide}, liggendes fra \textit{FalderEide} 3 Miile i væster til Søer,\par
\textit{Wegten}, paa den \textit{Øe-Wegten} fra Søer i Nord 2 Miile lang, 1/4 Miil breed. ‒\par
\textit{Lækøe}, paa Øen af samme Nafn, mest rund 1 Miil stoer. J forbemelte \textit{FalderEideAnnex} er 28 gaarder, og 49 bønder, som leeve af deris Gaarder, (hvor paa de nu i 3 Aar have haft Frost-Aarer) og Fiskerie; Saugbruget her tager med Skougen af. ‒\par
J dette \textit{Falder-Eide Annex} er den gaard \textit{Möe}, hvor ieg d: 13 \textit{hujus} kom fra; Denne med \textit{Lona} gaard ere de østerste i dette Sogn, og have deres Stræckning i øster ej længere end 1 1/2 Miil; Derfra i øster er en lang \textit{distance} til grændserne, Som bønderne her indtet viidste af at Siige, uden at den u-dyrcket bestaar af Fielde, og beboes alleene af \textit{Lap-Finner}. ‒\par
Vej fra dette østerste \textit{FalderEide-Annex} over til Sverrig er ingen: men de Som komme derfra hiid, tage Veyen enten igiennem \textit{Gaundalen}, eller \textit{Vats} Dalerne i Søndre- og Nordre \textit{Finlie} til \textit{Sneaasen}, og herfra igiennem \textit{Overhalden} hiid til \textit{Nærøes} Præstegield. ‒\hypertarget{Schn1_50728}{}Schnitlers Protokoller II.\par
d: 12{te}\textit{Augustj} Som Helligdag hviilede ‒\par
d: 13{de} ‒„‒ fra Gaarden \textit{Möe} over \textit{Folden}-fiord i Nord-væst til \textit{Folder}-Eide hvor Lænsmanden boer {1 1/2 Miile} Siiden over Eidet til Landz til \textit{Bindals}-Fiord{1/4 Miil} Hvilcket er det Sidste Skifte i \textit{Trondhiems-amt}\textit{Nommedals} Fogderie ‒\par
Her fra Eidet i Nord til første Skifte i \textit{Helgelands}-Fogderie\textit{Nordlands} Amt\textit{Brønøens} Præstegield\textit{Vasaas Annex}, til gaarden \textit{Øxningen} ved \textit{Bindals} Fiorden{2 Miil _______ = 3 3/4 Miil,}\hspace{1em}\par
\textit{Skielle-mærkerne} i mellem \textit{Trondhiems-} og \textit{Nordlands} Amter forcklares nærmere i \textit{Bielaget} N{o} 1:. ‒\par
d: 14: ‒ der fra i Nord igiennem \textit{Bindals}-Fiorden til \textit{BrækEidet}{2 Miil} Siiden til Landz over Eidet {1/4 _______ 2 1/4 Miil.}\hspace{1em}\par
[Her utelates ved trykningen et lengere avsnitt som finnes trykt i utgavens bind II, fra s. 2, l. 19 nedenfra, til s. 65, l. 25 ovenfra.]
\DivI[I Salten fogderi: 6 vidner.]{I Salten fogderi: 6 vidner.}\label{Schn1_50866}
\DivII[Okt. 14.-nov. 12. Fra Kolvereid til Trondheim]{Okt. 14.-nov. 12. Fra Kolvereid til Trondheim}\label{Schn1_50867}\par
d: 14 [oktober] Søndag ‒\par
d: 15 og 16 i Syden til væsten til Lundring{3 Miil}\par
d: 17 Liigesaa forbj \textit{Folden} fiord, hvor det aabne hav gaar paa, til \textit{Brakstad}{2 Miil}\par
d: 18 j Sydost til \textit{Surviigen}{2 M:}\par
19 i Søer til \textit{Halsoos}{1/2 M:}\par
d: 20 Liigesaa over \textit{Biørums} fiord, hvor \textit{Namsen} Elv falder udj, igiennem det Sund \textit{Lyngnen}{1 1/2 M:}\par
d: 21 Søndag ‒.\par
d: 22 i Søer til \textit{Aargaard} paa \textit{Nommedals} Eidet{1 1/2 Miil.}\par
d: 23 i Søer til Skiftet \textit{Elden} en Gaard omtrent midt paa \textit{Nommedals} Eidet i \textit{Bedstads}-Præstegield{1 1/4 Miil.}\hspace{1em}\par
Dette \textit{Bedstad} skal have faaet Sit Nafn af \textit{Beiter}, Kong \textit{Gors} Søn, som har boet der, og dend gaard \textit{Elden} være Saa kaldet af Samme \textit{Beiters} Jægt-baad \textit{Ellide;} Da imellem Kong \textit{Nor} og \textit{Gor} var aftalt at hin det faste Land og denne øerne, Som kund omSegles, sig Skulle underlegge, har Kong \textit{Gors} Søn brugt det paa fund, at hand om Viinteren har Sat Sig i Sin baad, hidset Seiglet op, og ladet Sig i baaden fra \textit{Bedstad} Sundet over dend \textit{Jsthmum}\textit{Nommedals}- Eidet ind i \textit{Lyngnen}, et Sund der gaar i \textit{Namsen} fiord ud i Havet, af Sine tiennere trække; Og derved tilEignet Sig dend halv øe nu bestaaendes av en 5: Præstegield nemblig \textit{Biørnøeraafiorden, øreland}, og \textit{Stads}-bøjden samt \textit{Læxviigen}; Midt paa \textit{Nommedals}-Eidet, hvor denne \textit{Beiter} med Sin baad \textit{Ellide} har hviilet, Skal nu den Gaard \textit{Elden} der af have sit Nafn ‒ \textit{Referente Jona Ramo, in Tractatu: Nori Regnum Cap: 1} ‒\hypertarget{Schn1_51102}{}Schnitlers Reise til Trondhjem.\par
d: 24 til \textit{Oosen}, en Platz under Gaarder Gaarden \textit{Hiellen} ved \textit{Beestad} Sundet .{ 1 Miil.}\par
d: 25 over Sundet til Præstegaarden \textit{Solberg} i \textit{Beestads} gield, hvor ieg med Præsten som havde været \textit{Missionaire} i øst-\textit{Finmarcken}, H{r}\textit{Elias Heltberg}, havde at \textit{Conferere}{1/2 Miil.}\par
d: 26 der fra i Søer til \textit{Skeviig} over Land til \textit{Beestads}-fiorden{3/4 Miil.}\par
d: 27 haft Modviind ‒\par
d: 28 Søndag ‒\par
d: 29 fra \textit{Skeviig} over Fiorden til \textit{Krogsvogen}{3/4} over Land til \textit{Borgen}, en Viig af ytterøe-fiorden, Som laag med jis belagt, og derfor ej kunde fare derover {1/4‒1 Miil.}\par
Paa denne vej foer man forbi Gaarden \textit{Egge} i \textit{Stods}-gield hvor i gammel tiid dend Høfding \textit{Kalv arnesen} haver boet, Som Sloeg Kongen \textit{Sanct Ole} paa \textit{Stikkelstad} ihiel, og Siiden Rømte ‒\par
Jmellem denne Gaard \textit{Eggen} og \textit{Steenkier} gaard i \textit{Sparboe}-gield staar 2 Rader fiirkant- Rundagtige Spitze Kampe-Steene, saa høje, at de naaer op under en karls Arm, omtrent en 16. i tallet, hvor af halv deelen paa hver Siide Staar ordentlig den eene Efter den anden, og Een desforuden Midt j, der om Siiges, at Laug-Rætt i fordum tiid er holden. ‒\par
d: 30. for dend tilfrosne Fiord maattet tage Lande-Veien til \textit{Saxhoug} i \textit{Jnderøens} Præstegield ad \textit{Ytterøe}-Fiorden{1 Miil.}\par
d: 31 her ved Vejr bleven opholdet, i midlertiid \textit{expederede Bielager} til \textit{Examinationerne} Samt et ohngefehrlig \textit{Carte}, med Maalings Routen efter de beviidnede grændser, og et \textit{Register} over de i \textit{Acten} forcklarede \textit{Nafne}, Som ieg ved Enden af Forrætningen har tilføiet; Der efter Reiset efter Veirets føielighed til vands fra Skifte til Skifte til \textit{Trondhiem}{6: Miile} hvor ieg\par
d: 12{te} Nov{br} Vel ankom ‒.\hspace{1em}\par
\textit{Trondhiem} dend 12{te} Nov{br} Anno ‒ 1742. \hspace{1em}Peter Schnitler mppr\par
\centerline{Følge}\par
1. \textit{Alphabetisk Register} over de i \textit{Examinations-Acten} beskrevne Stæders Navne . . . . . . . . [Utelatt ved trykningen. Det henvises til den trykte utgaves eget stedsnavn-register.]\par
2. \textit{Tabell} over de angivne og bevidnede Grendse-Merker imellem \textit{Norriges} og \textit{Sverrigs Provincier}, med en efterføyed kort Forklaring, af \textit{Acten extrahered}.\par
3. En ohngefærlig \textit{Delineation} af Grendserne, og derved paa begge Sider liggende Landskabers \textit{Situation}. ‒\hypertarget{Schn1_51389}{}Schnitlers Protokoller II.
\DivII[Forklaring over grensemerker mellem Norge og Sverige]{Forklaring over grensemerker mellem Norge og Sverige}\label{Schn1_51391}\label{Schn1_51392} \par 
\begin{longtable}{P{0.13955223880597015\textwidth}P{0.1458955223880597\textwidth}P{0.1332089552238806\textwidth}P{0.15223880597014924\textwidth}P{0.1458955223880597\textwidth}P{0.1332089552238806\textwidth}}
 \hline\endfoot\hline\endlastfoot I\tabcellsep II\tabcellsep III\tabcellsep IV\tabcellsep V.\tabcellsep VI\\
\textit{Svanesteenen}\tabcellsep \textit{Saxolaa-vara}\tabcellsep \textit{Giepskiak}\tabcellsep \textit{Rautekie-vara}\tabcellsep \textit{Varsans-vara}\tabcellsep \textit{Løyre-vand}\end{longtable} \par
 \par
\centerline{Kort Forklaring over ovenstaaende \textit{Grendse-Merker}, at reigne fra No. I. til XX fra Sønder i Nord. ‒}\par
I. \textit{Svanesteenen} er et Skiær i \textit{Svane-Vandet} imellem \textit{Nordre-Finlie} af \textit{Sneaasens} Præstegield, og \textit{Jemteland}, bart og skaldet, uden Trær paa, rund-lang-agtig, ikke større, end en Koe. Dette \textit{Svanesteen} at være Land-Merke, vill \textit{unanimiter} alle Vidner i \textit{Nordre-Finlie} have: Men de \textit{Lappe-Finner} Norden for \textit{Finlie} vill ei vide af dette \textit{Svanesteen}, men \textit{referere}, den Østre Ende af \textit{Heidugeln}-Vand, der hvor \textit{Avindsbækken} indløber, at være Grendse-skiel; Det 15de Vidne af \textit{Jnderøe} forklarer og, at imellem de Østligste Gaarder i \textit{Nordre-Finlie}, og første Bonde-Gaard i \textit{Jemteland} skulle være en 20. Field eller korte gamle Miile, og at dette \textit{Svanesteen} ligger fra \textit{Nordre-Finlie} 2. Miile længere ind ad \textit{Jemteland}: dog finder jeg u-forgribeligen, at ved Maalingen \textit{Svanesteenen} kan følges, i Kraft af Vidnerne i \textit{Nordre Finlie} deres eenstemmige Udsagn. ‒\par
II. \textit{Saxolaa-Vara:} Et Field for sig selv, og rund, 1/4. Miil over, fra den Østre Ende af \textit{Heidugeln}-Vand 1. Miil omtrent i Nord. ‒\par
III. \textit{Gieps-kiak} er en Berg-Klimp paa \textit{Ornesfields} Søndre Side og vestre Ende, snaug og bar oventil, med Bierke-Riis neden om sig; Er høyere end \textit{Saxolaa-Vara}, og 1/2. Miil over, langagtig, strekkende sig fra Vester i Øster, ligger fra \textit{Saxolaa-Vara} i Nord 1. Miil, og fra næste Grendse-skiel \textit{Rautekie-Vara} i Søer 1/2. Miil.\par
IV. \textit{Rautekie-vara} er en rødagtig, flad og snaug Bergklimp, rund af Skabning, 1/2. Mil over af Størelse, ligger lige i Nord fra \textit{Giepskiak} 1/2. Miil, og lige i Søer fra \textit{Varsandsvara} 1/2. Miil.\par
V. \textit{Varsands-vara} er langagtig fra Øster i Vester, omtrent 1/2. Miil og næsten lige saa breed fra Søer i Nord, snaugt, og noget høyere, end \textit{Rautekie-vara}. Dette \textit{Varsandsvara} ligger paa den Nordre Side og vestre Ende af \textit{Ornes-field}. Disse næst forbemeldte Grendse-Merker, hvorover Grendse-\textit{Linien} skal gaae, nemlig \textit{Giepskiak, Rautekie-vara} og \textit{Varsands-vara} ligge paa den Vestre Ende af \textit{Ornes}-field, og derfra daler det neer til Skougen, imod de i \textit{Acten} beskrevne Vande og Elve, i Vester.\par
VI. \textit{Løyre-vand} strekker sig fra Vester i Øster 1/2. Miil lang, og 1/8. Miil breed, liggendes i Nord-ost fra \textit{Onker}-vandet omtrent 1/4. Miil. Dette Vand, efter min Jdee maa ligge Nordlig fra \textit{Varsands-vara}: Men skulle det, ved \textit{Jngenieurenes} Nærværelse, befindes, at ligge vesterlig udaf \textit{Linien}, saa er det kun Eet Vidne, nemlig det 6te af \textit{Nommedalen}, som har navngivet dette \textit{Løyre}-Vand; de andre Vidner have \textit{determineret} Grendse-gangen fra \textit{Varsandsvara} igiennem Skoug og Myrland lige i Nord i den Vestre Ende af \textit{Vestre Raukie}-Vand ‒\hypertarget{Schn1_51666}{}Forklaring over Grensemerker.\label{Schn1_51668} \par 
\begin{longtable}{P{0.23181818181818178\textwidth}P{0.10818181818181817\textwidth}P{0.1339393939393939\textwidth}P{0.11848484848484848\textwidth}P{0.11333333333333333\textwidth}P{0.14424242424242426\textwidth}}
 \hline\endfoot\hline\endlastfoot VII.\tabcellsep VIII\tabcellsep IX.\tabcellsep X.\tabcellsep XI\tabcellsep XII\\
\textit{Vestre Ende af Vestre Raukievand}\tabcellsep \textit{Rastjouk}\tabcellsep \textit{Gavidsen-kiak.}\tabcellsep \textit{Fielle-field}\tabcellsep \textit{Garvefield.}\tabcellsep \textit{Østre-Brakfield.}\end{longtable} \par
 \par
VII. \textit{Vestre} Ende af \textit{Vestre Raukie}-vand ligger efter 1. Vidne af \textit{Nommedal}, fra \textit{Varsandsvara} lige i Nord igiennem Skoug og Myrland, omtrent 1/2. dags Reise, og efter 6{te}\textit{Nommedals} Vidne, fra \textit{Løyre}-Vand i Nord, 3. Miile; Dette Vestre-\textit{Raukie}-Vand ligger fra Vester i Øster 1/2. Miil lang, og 1/8. Miil breed, og deri løber den Elv \textit{Rast-jouk} fra \textit{Børje}field, saaledes at den rinder først fra Nord i Søer til imod Vestre-\textit{Raukie}- Vand, siden giør den en Bogt paa sig ad Øster, og rinder omtrent 2. Bøsse-skud i Øster ind i Vestre Enden af \textit{Vestre-Raukie}-Vand. Herved er at agte (1) at de 4. første Vidner af \textit{Nommedalen} have satt denne Vestre Ende af \textit{Vestre-Raukie}-Vand til Grendseskiel: Men det 6te Vidne \textit{ibidem} vil ei vide heraf, men setter \textit{Limiten} fra \textit{Løyre}-Vand lige i den Søndre Ende af \textit{Rastjouk}-Elven, som er omtrent 2. Bøsse-skud meere Vesterlig fra \textit{Vestre} Enden af \textit{Vestre-Raukie} Vand; Nu er det saa, dette 6te Vidne har sit Field-Sæde her til \textit{Rastjouk}, følgelig skulle best vide det: dog er dette kun en liden u-betydelig \textit{difference}, af et par Bøsseskud, og Øie-siunet kan best skiønne, hvor Maalings \textit{Linien} gaar ligest. (2) Dette 6te Vidne vilde \textit{inter referendum}, at Grendse-\textit{Linien} fra \textit{Løyre}-Vand skulle gaae i Søer efter de Østre Ender af \textit{Blaa}- Siø, \textit{Giorms}-Vand og \textit{Frostviig}-Vandet, vesten for \textit{Ornes}fieldz vestre Ende: mens de 4. første Vidner, som have \textit{determineret Limiten} efter de BergKlimper paa Vestre Ende af \textit{Ornes}-field, boe i denne Egn, og følgelig maa vide det bedre.\par
VIII. \textit{Rastjouk}, langs efter den, alle eenstemmig sige, \textit{limiten} gaar, er en Elv, opkommendes i Nord af \textit{Børje}-Field, hvorfra den gaar 1. Miil i Søer til imod \textit{Vestre-Raukie}- Vand.\par
IX. \textit{Gavidsen-kiak}, en høy, snaug og steil Field-Klimp oven paa \textit{Børje}-Field, rundagtig og spidz oven paa, liggendes noget paa bem.te \textit{Børjefields} Søndre Side, lige i Nord fra \textit{Rastjouk}; Mitt over denne \textit{Gavidsenkiok} gaar \textit{Limiten}.\par
X. \textit{Fielle-field}, det er fra \textit{Gavidsen-kiak} omtrent 1 1/2 Miil lige i Nord, ved en dal fra \textit{Børje}- field i Nord, fra næste Grendse-field, \textit{Garvefield}, i Søer, og fra det \textit{Svenske}\textit{Mars}-field i Sydvest liggendes adskilt; dette \textit{Fielle}-field strekker sig fra Syd-vest i Nord-ost 1 1/4. Miil lang, og er over ‒ breed 1/4. à 1/2. Miil. Over dets Mitte gaar Grendse-gangen, at den Vestre til \textit{Norge}, og den Østre Side til \textit{Sverrig} hører.\par
XI. \textit{Garve-Field} ligger 2 1/2. Miile i Nord fra \textit{Fielle-field}, og derfra rinde smaa bække need i Vester i det derunder liggende \textit{Farok-Vand}; dette \textit{Garve-field} er et snaugt fladagtigspidz Field fra Vester i Øster 1. Miil lang, og fra Søer i Nord 1/2. Miil breed. Grendsen gaar derover, hvorifra Vandet nedkommer til \textit{Farok}-Vandet.\par
XII. \textit{Østre-Brakfield} kan være 1. Miil langt og breedt; det ligger noget i Nord-vest fra \textit{Garve}-field 1 1/2. Miil-veigs, og er Furru- og Granskoug med Vande derimellem; Østen for dette Grendse-Field er intet Field, men slet jevnt Skoug-land. ‒ Grendse-Gangen\hypertarget{Schn1_51982}{}Schnitlers Protokoller II.\label{Schn1_51984} \par 
\begin{longtable}{P{0.19833333333333333\textwidth}P{0.4344444444444444\textwidth}P{0.2172222222222222\textwidth}}
 \hline\endfoot\hline\endlastfoot XIII\tabcellsep XIV.\tabcellsep XV.\\
\textit{Riefield}\tabcellsep \textit{Amber-} eller \textit{Baanesfield}\tabcellsep \textit{Stokkefield}\end{longtable} \par
 \par
her kunde eller vilde dette 6te Vidne ei fastsette, som \textit{Acten} nærmere viiser: dog siunes billig, at \textit{Limiten} trekkes Østen for dette \textit{Østre Brakfield} ved dets Foed; Thi (1) ligger dette Field i Nord-vest fra \textit{Garve-Field}, (2) Er det \textit{Norske} Skatte-\textit{Finner}, som indehave dette heele \textit{Østre Brakfield}. ‒\par
XIII. \textit{Riefield} ligger 1 1/2. Miil i Nord, dog vestlig fra \textit{Østre-Brakfield;} Østen for dette \textit{Riefield} intet andet Field, men luter Skoug er. Grendse-\textit{Limiten} paa dette Stæd vil gaae ved den Østre Foed af dette \textit{Riefield;} Thi af det ganske og heele \textit{Riefield} baade af dens Østre- og Vestre Side er svaret Skat til Norge alleene.\par
XIV. \textit{Amber-field} ligger i Nord-Nord-vest fra \textit{Riefield}, Grendsegangen vil gaae over Mitten af dette \textit{Amberfield;} Thi efter 7. 8. og 9{de} Vidner af \textit{Helgeland} svares Skat af dette \textit{Amberfields} Vestre Side, til \textit{Norrige}, og af dets østre Side til \textit{Sverrige}. ‒\par
J \textit{Raens} Præstegield have Vidnerne af ingen anden deres Grendse-field vidst, end af \textit{Riefield} i Søer, og \textit{Amber-field} i Nord: Men da man komm Nord i \textit{Bejern} i \textit{Gilleskaals} Præstegield, angav Vidnerne der et andet Grendse-Merke imellem \textit{Raens} Gield og \textit{Uma Lapmark}, nemlig \textit{Baanes-field}, hvilket de forklarede, at ligge fra \textit{Bejerns} første Grendse-Merke, nemlig \textit{Stokke-field} i Syd-ost 1. god dags Reise, eller 4. Miile, og at \textit{Amber-field} skal ligge et Støkke vesten for dette \textit{Baanesfield}; dette \textit{Baanes}-field, sagde de, at være et bart skallet Field uden Skoug; Græs og Maasse, slet ovenpaa, strekkende sig fra Vester i Øster, og at mit over dette \textit{Baanes}-field Grendse-\textit{Limiten} gik, givendes derfor den \textit{raison}, at af dets østre Ende rinder en Elv i Øster ad Sverrig, og af dets Vestre Ende en anden Elv i Vester ad \textit{Norrige} ind i \textit{Virvandet} i \textit{Raen}. Hvorledes denne \textit{Difference} imellem \textit{Raens} og \textit{Bejerens} Vidner, angaaendes \textit{Amber-} og \textit{Baanes}-fielde er at \textit{conciliere:} det kan vel ei vide; siden jeg ikke kunde have begge Stæders Vidner samlede til \textit{Confrontation;} kan følgelig ikke sige: Om \textit{Baanes}-field, og ikke \textit{Amber}-field i \textit{Raen} skal være Grendse-Merke? Hvilket er efter \textit{Bejerens} Vidner; Eller om \textit{Baanes}field er en Tang eller \textit{particul} af \textit{Amber}-field, og af \textit{Raens} Vidner indbefattes under det Navn \textit{Amberfield?} J hvilken sidste Fall begge Stæders Vidner kan \textit{accordere} og stande ved Magt; \textit{Jn dubio} siunes mig, at præsumtionen er for \textit{Raens} Vidner, at de maa vide best, og have sagt rettest om deres Bøygdz Grendser, nemlig at \textit{Amber-field} giør Grendse-Skielnet.\par
XV. \textit{Stokke-field} er bart uden Skoug, dog med Maasse paa, strekkendes sig fra Vester i Øster 1/2. Miil lang, og 1/8. Miil breed; det skal ligge fra \textit{Nase-field} i Nord-ost 1/4. Miil veigs (efter Vidnets Sigende) og fra \textit{Baanesfield} i Nord-vest 4. Miile. Over den Østlige Ende, af dette \textit{Stokke-field} gaar Grendse-\textit{Limiten;} Thi (1) er Fieldet der høyest, (2) \textit{Svangs-Elven} gaar udaf den østre Ende af det derhos liggende \textit{Svangsvand} ved den østre Ende af dette \textit{Stokke-field}.\hypertarget{Schn1_52306}{}Forklaring over Grensemerker.\label{Schn1_52308} \par 
\begin{longtable}{P{0.17708333333333334\textwidth}P{0.17708333333333334\textwidth}P{0.17\textwidth}P{0.16291666666666668\textwidth}P{0.16291666666666668\textwidth}}
 \hline\endfoot\hline\endlastfoot XVI.\tabcellsep XVII\tabcellsep XVIII\tabcellsep XIX\tabcellsep XX\\
\textit{Svangsfjeld}.\tabcellsep \textit{Streiteskiok}\tabcellsep \textit{Tebunvagge}\tabcellsep \textit{Viskes-kiok}\tabcellsep \textit{Gaterre-kiok}\end{longtable} \par
 \par
XVI. \textit{Svangsfield} ligger lige i Nord fra \textit{Stokkefield}, fra Søer i Nord 1 1/2. Miil lang, og fra Øster i Vester 1. Miil breed; J mellem dem begge, nemlig \textit{Svangs}- og \textit{Stokke-fielde} ligger \textit{Svangsvandet}, 1/2 Miil lang fra Søer i Nord. \textit{Svangsfield} er et bart slet Field, med Snee og nogen liden Maasse paa, har og nogle Berg-Knuder eller houger paa sig, lidet over Mitten til den østre Ende.\par
Næst paa den østre Ende af dette \textit{Svangsfield} gaar Grendsegangen; Thi der er det høyest, og derfra hælder det strax neer i Øster ad \textit{Sverrig}.\par
Norden for dette \textit{Svangsfield} er en haard steened Field-dal, 3. Miile lang fra Søer i Nord, og 1. Miil breed. Norden derfor er\par
XVII. \textit{Streiteskiok}, et bart slet field med noget Maasse paa, strekkende sig fra Vester i Øster 1. Miil lang, og fra Søer i Nord 1/2. Miil breed.\par
Mitt over \textit{Streiteskiok} gaar \textit{Limiten}, der hvor det er høyest. Norden herfor\par
XVIII. \textit{Tebunvagge}, en Field-dal, 1/2 Miil breed fra Søer i Nord, og 1. Miil lang fra Vester i Øster; Hvor igiennem de \textit{Norske Finners ordinaire} Fløtnings \textit{passage} gaar til- og fra \textit{Sverrige}.\par
\textit{Limiten} gaar her, over den østre Ende af \textit{Tebunvagge;} Thi der er den høyest.\par
Nu følge Grendserne imellem \textit{Saltdal} og \textit{Sverrigs}\hspace{1em}\centerline{\textit{Pita-Lapmark:}}\hspace{1em}\par
Norden paa \textit{Tebunvagge} følger\par
XIX. \textit{Viskeskiok}, et høyt Field uden Skoug og Græs, dog med nogen Maasse paa, fra Vester i Øster 1. Miil lang og fra Søer i Nord 1/2. Miil breed.\par
Over den østre Ende af \textit{Viskeskiok} gaar \textit{Limiten;} Thi der er den høyest. Jmellem denne \textit{Viskes-kiok}, og følgende \textit{Gaterre-kiok} ligger en field-dal, fra Vester i Øster 1. Miil lang, og fra Søer i Nord 1/2 Miil breed.\par
XX. \textit{Gaterrekiok} ligger fra \textit{Viskeskiok} i Nordost, og er fra Vester i Øster 1. Miil lang, og fra Søer i Nord 1/2 Miil breed; dette Field har en Rygg mit paa sig fra Vester i Øster. ‒\par
Over den Vestre Ende af \textit{Gaterrekiok} er Grendse-Gangen, hvor den er høyest.\hypertarget{Schn1_52509}{}Schnitlers Protokoller II.
\DivII[Bilag til 2. volumen (herunder en del person-lister over lapper, II 65-88 og 98-100)]{Bilag til 2. volumen (herunder en del person-lister over lapper, II 65-88 og 98-100)}\label{Schn1_52511}\par
N. 19. Tolf Skilling.\hspace{1em}\par
Kongl- May- Fouget of{r}Jnderøens FougderjRasmus Søfrensen høg, kiendis og hermed witterlig giør, at Jeg Paa hans Kongl- May{ts} Min allernaadigste arfve Herre og Kongis allernaadigste behaug og weigne haf{r} und og bewilget Nerwerrende Peder Pouelsen En pladtz Kaldis Qvelien udj Norlie field under Snaasens Præstegield beligende, huilchen Pladtz hand self haf{r} oprydett, og nu af Sorren skrifveren Med 6 Laugrettis Mænd er lagt udj Leye for otte Marchlaug aarlig at skylde till hans Kongl- May{ts} udj Skatt landschyld og andre \textit{Contributioner} effter hans Kongl- May{ts} allernaadigste Skatte forordninger, og skall for{ne}Peder Pouels: Niude forschr{ne} 8 mrkl: Leye hans lifs thid med alle de laatter og lunder, Samt hans andeel udj dett wand som kaldis qvæen, og hans Arfvinger effter hans død effter lowen at werre nermest at beside, imidler thid og ald den stund, hand suarer de der af gaaende Rettigheder i Rette thide, og sig ellers i andre maade som En lejlending schicher og for holder landtz lowen gemes, ditz til stadfestelse under min haand og Signette, Actum qvæeliend- 19 July A{o} 1693.\par
{Rasmus Høg mppria,}\hspace{1em}\par
A{o}1693 d:14 Octobr: lest paa et ordinarie thing paa Suarve paa Snaasen{test. N. O. Wind mppria.}\hspace{1em}\par
\textit{Produceret} i grændze Commissions Rætten i Dag d: 25. Julj 1742 her paa gaarden \textit{Sandviig} i Nordre \textit{Finlje Annex}\textit{Sneaasen} Præstegield ‒\hspace{1em}\par
{Lit A. N{o} 1:}\hspace{1em}\par
Nerverende Olle Nielsøn Er bevilgett at maa boe J Kiørmoe skougen, Tilhørende Ofuerhaldens Præstegield, og skal Hand Dend Tilbrugs Nyde, ald Dend stund Hand iche giør Nogen mand Forfang i Moed Norges lov. Saa hand ubehindrett for de Andre finner bemelte Kiørmoe schou maa beboe, Og maa hand der forblifue, ald dend stund hand betaller sin Sedvanlig Rettighed og Tiende som Hand her til Nummedahls Fougderie schal hvert Aar lefuere og betale og ingen anden Stedz,\hspace{1em}\par
Actum Ranum Dend- 3 Februarj 1699 Til witterlighed  Baldtzer Nielsen mppria. Jacob Jespersen mppria.  \par
{Lit‒A: N{o} 2}\hypertarget{Schn1_52692}{}Forklaring over Grensemerker.\par
{Lit: B.}\par
Ao 1703 d. 21. Januarij haver Anders Torckelsøn bøxlet Hans faders Torckel Olsens Skoug, og haver betalt som Sedvanligt bøxel Penge, Hvor for hand Samme Skoug Saa vidt hans fader nu i brug haver, maa Nyde at beboe og giør sig Saa nøttig som hand bæst ved og kand, dog Skal hand i alle Maade Forholde sig Skickelig imod sine Medkammerater, og i Rette tiid Aarlig betalle den Sedvanlige Skougleje.\hspace{1em}\par
Hildrum utsupra d. 21: Januarj 1703. {Paa min hosbonds vejgne Melcher Meier,} Conform Originalen, Test. M. N. Lund.\hspace{1em}\par
[Her utelates ved trykningen et lengere avsnitt som finnes trykt i utgavens bind II, fra s. 65, l. 31, til s. 101, l. 18.]\hspace{1em}\par
Bilag R.\par
\textit{Wii Friderich den Tredie, af Guds Naade, Danmarkes, Norges, Venders og Gothes Konning, Hertug udi Slesviig, Holsteen, Stormarn} og Ditmarsken, Greve udj Oldenborg og Delmenhorst. ‒\par
Giør alle vitterlig: at vj Naadigst for Et Hundrede Tusinde Enkende Rixdaler \textit{in Specie Contante} nøyagtige og fuldkommen betaling haver afstaaed, Saasom vj og nu med dette vort aabne brev, allernaadigst Skiøder og Afhænder, fra Os og vore Kongel: Arve \textit{Successorer} udj Regieringen, i for{te} vores arve-Riiger Danmark og Norge, til os Elskelige \textit{Joachim Jrgens} til \textit{Wæsterviig} og hans arvinger, disse efterskrevne vores Ambte og fogderier, Nord-landene, udj be{te} vores Riiger \textit{Norge}, Nembl. \textit{Helgeland, Salten, Senjen, Andenes, Tromsøen, Loefoden og Wæsteraalen}. Hvilke fornefnte Nordlandske Ampt og Fogderier, med dessen \textit{Residentzer} /: Saa vel som alt hvis af godtz \textit{ibm:} derfra til forne Soldt og Pandtsadt er, frit fore igien at jndløse :/ og alle detz nu til- og underliggende gaarde, gaarde-Parter, Bryggeværelser og Boe-hærlighed, Fiskerie af Salt-Søe og Ferske-vand, samt alle detz afgaaende Landskyld; Tiende af alle Slags græs og grund-Leje, Laxevarper, Saug-Leyer, Leeding, Søe- finde- Skatt, Landvære, Huusfrelse, Lappe-Skatt og Jægte-Reisninger, afgiften af \textit{Bodøen} og \textit{Lødings} Præstegiæld, med \textit{Indyer} gaard og godz, samt første bøxel, tredie aars Tage, Sigt og Sagefald, arbeid og alt andet deslige udj for{te} Amter og af for{te} godz falde kand, være Sig af des til og underliggende Land og Vand, Fosser og Strømmer, Lotter, Moes og Lunder: Jtem, all anden jnkomst, Rættighed og Hærlighed; være Sig Ager, Eng, Skoug, Mark, Fiskevand og Fæe-gang, Veide eller Vejde-Stæder, field, bierge, dahle, fiære, øde og u-biugte, vaad og tørt, jnden og uden gaards og all anden tillæg, med dessen \textit{Circumferentz} og hærlighed, som der nu tilligger, og af alders tiid der tiligget haver, og med rette der tilligge bør /: saa vidt os haver været berettiget :/ aldeeles intet undertagen i nogen maader; for{te}\textit{Joachim Jrgens}, hans arvinger og hvem paa deres vegne fuldmagt haver, skal og maa besidde, annamme, have, nyde, følge, \hypertarget{Schn1_52861}{}Schnitlers Protokoller II. bruge og beholde, Sig saa nøttig at giøre (: Landsloven gemæs :) som de Selv best veed, vil og kand, til Evindelig Eyendom, at angaae \textit{Sanctæ Hans Babtistæ} eller Midsommers Dag den 24 Junij A{o} 1664; dog os og vores Kongel: Arve-\textit{Successorer} udi Regieringen \textit{Souverænitet}, Kongel: \textit{Regalier} og Høyheder af forne{te}Nordlandske fogderier og godz, det være Sig Tolde, \textit{Accise}, Vrag, \textit{Jura Patronatus} og Birkehøyheder, u-forkrenkt og os aldeles forbeholden, Saa velsom Skatten af for{te} Nordlandske undersaatter, og det \textit{proportionaliter}, som de tilforn mod andre vores undersaatter i vort Riige Norge\textit{Continueret} haver, Nafnl: Halv Skatt; og om Ros-tieneste af forskrevne vores til hannem nu afhændte godtz vorder paabuden, da ikke viidere Ros-tieneste eller odel-skatt der af at holde, end Sædvanligt af forb{te} Fogderie Pleyer at Skee, Jtem Naar vj eller Vores Kongel: Arve \textit{Successorer}, Naadigst kunde blive til Sinds, fornev{te} Nordlandske Ampter, Fogderier, gods og Eyendom os igien at tilforhandle, da det os igien at komme til løsen, fra hver Særdeles som det da Eyer eller haver udj \textit{possession}, og det \textit{pro qvota}, eftersom det nu af os er afhændet for ut Supra, Et Hundrede Tusinde Enkende Rixdaler \textit{in Specie} tillige med billig erstatning, for hvis beviises af for{te} forrige bortsoldte og forpandtede godtz, efter forberørte \textit{Condition} indløse og \textit{Contante} betahle vilde. ‒ Forbydendes alle og een hver for{te}\textit{Joachim Jrgens} og hans med skrevne her imod, eftersom forskrevet staaer, at hindre, eller i nogen maader forfang at giøre, under vor hyldest og Naade. ‒ Givet paa vort Kongel: \textit{Residentz} udj \textit{Kiøbenhafn} den 12{te}\textit{Januarj Anno} ‒ 1666. ‒\par
\centerline{Under vort Zignethe \textit{Friderich R:}}\hspace{1em}\par
Er saaledes rigtig \textit{Copie} af \textit{originalen}{\textit{Test: Preben von Ahnen}.}\hspace{1em}\par
Læst for Rætten, Botolphj Laugting, holden ved \textit{Steegen} d. 25de Junij A{o} 1666. \centerline{\textit{Tes{t}Christen JensenMunchegaard}. ‒}\hspace{1em}\par
Med den hos mig beroende gienpart eller \textit{Copie Conform}{\textit{testor Margrete Sl Sverdrops}.}\par
_________Tabell over de bevidnede Grendse-Field.
\DivI[3. volumen: 1743, om sommeren.]{3. volumen: 1743, om sommeren.}\label{Schn1_53029}\par
\centerline{[Av \textbf{3die Volumen} følger her det som er utelatt i bind II, mellem s. 226 og s. 227.]}\hspace{1em}\par
Herefter følger\par
(1) \textit{Alphabetisk Register} [ikke trykt, bare den umiddelbart derpå følgende: Tabell over de bevidnede Grendse-Field]\par
(2) Bilagerne\par
(3) en Ohngefærlig \textit{Delineation} over Grendsens Gang med derved angrendsende Lande. ‒\hspace{1em}
\DivI[Tabell over bevidnede grensefjell]{Tabell over bevidnede grensefjell}\label{Schn1_53063}\par
\textbf{Tabell} over de bevidnede Grendse-Field og Merker fra Sør i Nord imellem \textit{Norge} og \textit{Sverrig}, af dette 3die \textit{Volumen extrahered}, efter Grendsefieldet \textit{Gaterrekiok}, det sidste i \textit{Protocollens} 2{det}\textit{Volumen}\label{Schn1_53092} \par 
\begin{longtable}{P{0.0913978494623656\textwidth}P{0.13709677419354838\textwidth}P{0.1553763440860215\textwidth}P{0.09596774193548387\textwidth}P{0.14623655913978495\textwidth}P{0.0913978494623656\textwidth}P{0.13252688172043012\textwidth}}
 \hline\endfoot\hline\endlastfoot \tabcellsep I\tabcellsep II\tabcellsep III\tabcellsep IV\tabcellsep V\tabcellsep VI\\
\textit{Gaterrekiok}\tabcellsep \textit{Skademevagge}, en dal\tabcellsep \textit{Skademekiok} over Mitten\tabcellsep \textit{Junkerdal}\tabcellsep \textit{Joxekiok}, over Mitten\tabcellsep Ujævnt Steenet Land\tabcellsep \textit{Dorrisvagge} en dal\end{longtable} \par
 \par
\textit{Gaterrekiok} var i afvigte Høstes \textit{Examinations Protocolls} 2{det}\textit{Volumen} det sidste og i Ordenen det XX{de}; over dens Vestre Ende \textit{limiten} gaar.\par
I. \textit{Skademevagge}, en Jorddal, følger i Nord paa \textit{Gaterrekiok}, lang fra Sør i Nord 1. Miil, og fra Vester i Øster ligesaa breed, bevoxen med Bierke-Skoug og Græss. Fra \textit{Gaterrekioks} Raamerke, som er dens Vestre Ende, hvor den er høyest, gaaer Grendsen midt over denne Dal i Nord, til næste Field \textit{Skademekiok}.\par
II. \textit{Skademekiok}, et Grendse-Field, fra Sør i Nord 1/16. Miil stort, men fra Vester i Øster omtrent 1/2. Miil langt, paa Sidene med Græss og Maasse begroet, mens ovenpaa med Jis og Snee stedse bedekket: dog ligger dens Strekning fra Sydvest i Nordost. Grendse-Gangen er nu fra \textit{Gaterrekiok} midt over den Jord-dal \textit{Skademevagge}, og midt over \textit{Skademekiok}, hvor denne er høyest. Norden for \textit{Skademekiok} ligger\par
III. \textit{Junkerdal}, riig paa Græss, Bierk og Older, 1/16. Miil breed fra Sør i Nord, og strekker sig fra Sydvest i Nordost 3/4. Miil lang. Norden for denne Dal er det field\par
IV. \textit{Joxe-kiok}, rundt, snaut og bart uden Skoug og Græss ovenpaa, 1/16. Miil stort; Grendsen gaaer fra \textit{Skademekiok} lige i Nord over \textit{Junkerdal} midt over \textit{Joxekiok}, hvor den er høyest. Norden for \textit{Joxekiok} ligger\par
V. Et ujævnt steenet Land, 1/4. Miil vidt fra Sør i Nord; derpaa følger i Nord en Field-dal med noget Græss i, uden Skoug, navnlig\par
VI. \textit{Dorrisvagge}, 1/4. Miil over fra Sør i Nord, men fra Vester i Øster 1/2. Miil lang; dette \textit{Dorrisvagge} ligger Østen for den Søndre Ende af \textit{Balvatten;} Norden for Dorrisvagge ligger et uslet Field-Land, 1/16. Miil vidt fra Sør i Nord til \textit{Skarjahagorre}.\hypertarget{Schn1_53268}{}Schnitlers Protokoller III.\label{Schn1_53270} \par 
\begin{longtable}{P{0.36223628691983123\textwidth}P{0.08607594936708861\textwidth}P{0.07890295358649789\textwidth}P{0.057383966244725734\textwidth}P{0.13270042194092826\textwidth}P{0.13270042194092826\textwidth}}
 \hline\endfoot\hline\endlastfoot VII\tabcellsep VIII\tabcellsep IX\tabcellsep X\tabcellsep XI.\tabcellsep XII.\\
\textit{Skarjahagorre,} et Fieldskare\tabcellsep \textit{Vuetsekiok}.\tabcellsep \textit{Vuetsegorre}\tabcellsep \textit{Sluppo}\tabcellsep \textit{Dorrokiok} over dens Mitte\tabcellsep \textit{Saulo-kiok}. Midt derover\\
\multicolumn{4}{l}{(Grendsen gaaer over disse 4. Steder efter Bækkenes Vandfald)}\end{longtable} \par
 \par
VII. \textit{Skarjaha-gorre}, et FieldSkare 1 Bøsseskud Østen for \textit{Balvattenet}, fra Sør i Nord omtrent 10. Skritt bredt, fra Vester i Øster 2 à 3. Riffel-skud langt, med noget Græss i; \textit{Limiten} gaaer nu fra \textit{Joxekiok} i Nord over den Vestre Deel af \textit{Dorrisvagge} til det høyeste af \textit{Skarjahagorre;} At den kommer over \textit{Dorrisvagges} Vestre Deel, dertil er Aarsagen, (1) at fra Høyeste af \textit{Joxekiok} skal \textit{linien} gaae lige i Nord saaledes til det Høyeste af \textit{Skarjahagorre} over \textit{Dorrisvagges} Vestre Deel, (2) Vandene skal falde fra den mindre, nemlig Vestre Deel af \textit{Dorrisvagge} i Vester til \textit{Norge}, og fra den større, nemlig Østre Deel i Øster til \textit{Sverrig}. ‒\par
Derpaa i Nord følger\par
VIII. \textit{Vuetse-kiok}, liggendes tæt ved \textit{Skarjaha-gorre} i Nord, 4. Riffelskud Østen for Balvatten; det er rundagtigt, en 4. Riffelskud over, paa Sidene Græss-groet, men ovenpaa bart.\par
IX. \textit{Vuetse-gorre} tager ved strax efter \textit{Vuetsekiok} i Nord, er et Field-Skare med Græss begroet, fra Sør i Nord 1. Riffel-Skud over bredt, og fra Vester i Øster 3. à 4. Riffel-Skud langt, ligger 1. Riffelskud Østen for \textit{Balvatten}.\par
X. \textit{Sluppo} er et Field-Skare tæt Norden for \textit{Vuetsegorre}, derfra bare ved en Field-Houg adskilt, ligger fra Balvatten i Øster 1. RiffelSkud, er fra Sør i Nord 1 Riffel-Skud over bredt, og fra Vester i Øster, nemlig fra den Norske Skoug i Vester til \textit{Sverrigs} Skoug i \textit{Pita Lapmark} 1. Miil langt.\par
Disse sidst benævnte Grendse-Steder Østen for \textit{Balvattenet} bestaae af et sammenhengigt Field; Grendsen over samme 4{re} sidste Steder have Vidnerne ei anderledes kundet \textit{specificere} eller med eegentlige Navne beskrive, end at deri ere mange smaa Bække, hvoraf en Deel i Øster ad \textit{Sverrig}, en Deel i Vester ad \textit{Norge} falde; Hvorfra nu Bækkene falde, der er det høyest paa Fieldet, og der bliver Raagangen; hvilke Vanders Fald Vidnerne vel kan anviise, men ej give Navn paa.\par
XI. \textit{Dorrokiok} er et Field, som i Nord ligger tæt ved \textit{Sluppo}, og dermed sammenhenger, fra Sør i Nord 2. Riffel-Skud over, med Græss begroet, og fra Vester i Øster 1. Miil lang; dette \textit{Dorrokiok} giør med de 4. forhen omtalte \textit{Skarjaha-garre, Vuetse-kiok, Vuetse-gorre} og \textit{Sluppo} Eet Field; Midt over \textit{Dorrokiok}, hvor den er høyest, gaaer Grendse-Skiellet lige i Nord fra de andre sidste Steders Raagang.\par
XII. \textit{Saulo-kiok}, en meget høy og spidz Berg- Klimp, hvorfra kan sees en 10. Miile ud i Havet til Lofoden, Er fra Sør i Nord 1/2 Miil lang, og fra Vester i Øster ligesaa stor, paa Vestre og Søndre Side græss-groed; Midt over \textit{Saulo-kioks} høyeste Klimp gaaer Grendse-Skiellet. Paa \textit{Saulo-kiok} i Nord følger en Field-dal med noget Græss i, fra Sør i Nord 1/4. Miil viid; derpaa i Nord følger det GrendseField\hypertarget{Schn1_53506}{}Tabell over de bevidnede Grendse-Field.\label{Schn1_53508} \par 
\begin{longtable}{P{0.06839080459770115\textwidth}P{0.08597701149425287\textwidth}P{0.08206896551724138\textwidth}P{0.10160919540229885\textwidth}P{0.24425287356321837\textwidth}P{0.10160919540229885\textwidth}P{0.1660919540229885\textwidth}}
 \hline\endfoot\hline\endlastfoot XIII.\tabcellsep XIV.\tabcellsep XV.\tabcellsep XVI.\tabcellsep XVII\tabcellsep XVIII.\tabcellsep XIX\\
\textit{Rutuekiok} Over Mitten\tabcellsep \textit{Vassiaskielma} over dets Høyeste\tabcellsep \textit{Vassiaskiok} Over dets Høyeste.\tabcellsep \textit{Østre Ende} af \textit{Østre- LaamiVand}\tabcellsep \textit{Sulietelma}, derover, hvor \textit{linien} fra Østre \textit{Laami} Vand til Vesten for \textit{Pita-jaure} gaaer\tabcellsep \textit{Gaise-giek} Over dets Syd-ostlige Ende\tabcellsep \textit{Steenfield} Der hvor den Østre Ende af \textit{Polli-jaure} Norden for er.\end{longtable} \par
 \par
XIII. \textit{Rutue-kiok}, 1/4. Miil fra \textit{Saulo-kiok}, rundagtigt, snaut og bart uden Græss, fra Sør i Nord 1/4. Miil bredt, fra Vester i Øster 1. Miil langt; Midt over denne \textit{Rutuekiok}, hvor den er høyest, gaaer Grendse-Gangen. Nord herfor er en Steen-dal 1/2. Miil over fra Sør i Nord, hvorj stedse Snee ligger. Norden for \textit{Rutuekiok} bem{te} 1/2 Miil er det Field\par
XIV. \textit{Vassias-kielma}, 1/4. Miil over fra Sør i Nord, rundt og lavt, steenet uden Græss paa; Over det Høyeste af dette \textit{Vassiaskielma}, lige i Nord for de forrige GrendseMerker gaaer Grendse-\textit{Routen}. ‒ Herpaa i Nord følger en Field-dal, steened, og uden Græss, fra Sør i Nord henger den sammen, fra \textit{Vassiaskelma} med næstfølgende Grendse-Field\par
XV. \textit{Vassias-kiok}, spidz oven til, og høyere, end \textit{Vassiaskielma}, bart uden Græss, fra Sør i Nord 1/2 Miil stort, fra Vester i Øster 1 Miil langt; Grendseskiellet gaaer over det Høyeste af denne \textit{Vassias-kiok}. ‒ J Nord herpaa følger\par
XVI. Den \textit{Østre Ende} af \textit{Østre-Laami-Vand}, hvilket Vand strekker sig fra Øster i Vester 1/4. Miil langt, deraf Aaen rinder i Vester ad \textit{Norge} ‒ J Østre Enden af dette \textit{Østre Laami}- Vand gaaer \textit{Limiten}, saa det heele \textit{Østre Laami}-Vand tilkommer \textit{Norge}. ‒ J Nord herpaa er\par
XVII. \textit{Sulietelma}, et langt Field, fra Vester i Øster strekkende sig 2. Dagers Fløtnings- Reise (som holdes for 6. Miile) og er af u-lige Bredde; Nu have de \textit{Norske} Vidner vel ikke vidst af, at sammenføye de Grendse-Merker, \textit{Østre Enden} af \textit{Østre Laami}-Vand i \textit{Salten}, og \textit{Sulietelma} i \textit{Folden}, (see II 155.) dog af det, som Vidnerns udsagt, følger riimeligen, at der og paa det Steed over \textit{Sulietelma} Grendse-Gangen vil gaae, hvor \textit{linien} fra \textit{Østre-Laami}Vands Østre Ende gaaer til tæt Vesten for det Vand \textit{Pita-jaure}, hvoraf Elven rinder i Syd-ost ad \textit{Sverrigs}\textit{Pita-Lapmark}, et Støkke Østen for det Vand \textit{Skaf-jaure}, hvoraf Aaen falder i Nord-Nord-Vest i den Syd-ostlige Ende af Verri-vatten. 1/2 Miil lige i Nord fra dette \textit{Sulietelma} er det Field\par
XVIII.\textit{Gaise-giek}, som ligger fra det \textit{Norske}\textit{Verri-vatten} i Syd-ost, og strekker sig fra dette Vand fra Nord-vest i Syd-ost omtrent 1 Miil langt, nogle Riffel-Skud over bredt; Siden nu \textit{Pita-jaure} med sin Nord-vestlige Ende ligger nær til dette \textit{Gaisegiekes} Syd-ostlige Ende, saa gaaer \textit{limiten} fra \textit{Sulietelma} tæt Vesten forbi \textit{Pita-jaures} Vestre Ende i Nord over \textit{Gaisegiekes} Syd-ostlige Ende. ‒ Herfra gaaer \textit{Limiten} videre i Nord over den Dal \textit{Vee-vagg}, i lige \textit{linie} til\par
XIX. \textit{Steenfield}, som ligger et Støkke Østen for det \textit{Norske} Vand \textit{Verrivatten}, med Maasse og Græss begroet, strekkende sig tillige med det bekiendte \textit{Sølva-vara}, som de Svenske fordum have brugt til en Sølv-Grube, hvormed dette \textit{Steenfield} i sin Østre Ende sammenhenger, 1. Miil langt, et Par Riffel-Skud over bredt; Norden for dette \textit{Steenfield} er en liden Dal, og i den Dal 2{de} Vande fra Øster i Vester efter hinanden liggende, hvoraf Aaen rinder i Vester\hypertarget{Schn1_53816}{}Schnitlers Protokoller III.\label{Schn1_53818} \par 
\begin{longtable}{P{0.2510362694300518\textwidth}P{0.1805699481865285\textwidth}P{0.22020725388601034\textwidth}P{0.1981865284974093\textwidth}}
 \hline\endfoot\hline\endlastfoot XX\tabcellsep XXI.\tabcellsep XXII.\tabcellsep XXIII.\\
Tæt \textit{Østen} forbi det Vand \textit{Polli-jaure}.\tabcellsep \textit{Dørdrik} over dets Østre Ende\tabcellsep Tæt \textit{østen for Laudok Aaes} Oprindelse\tabcellsep \textit{Kiesuris} over dets Vestre Ende\end{longtable} \par
 \par
fra eet i det andet Vand, og omsider i forberørte \textit{Verri-vatten;} Det Østere Vand heraf heder \textit{Polli-jaurj}, langagtigt af et Par Riffel-Skud vidt. Nu bliver Grendse-Gangen over \textit{Steenfield} i Nord der og paa det Steed, hvor paa dets Nordre Side den Østre Ende af det Vand \textit{Polli-jauri} ligger Fra \textit{Steenfield} gaaer \textit{Linien} videre Nord\par
XX. Tæt \textit{Østen} forbi det Vand \textit{Polli-jauri}, hvoraf Aaen i Vester rinder i \textit{Kittauri} Vand, og siden i Verri-vatten. ‒ Herfra gaaer \textit{Linien} lige i Nord over det Field\par
XXI. \textit{Dørdrik}, og især over dets Østre Part; dette \textit{Dørdrik} skal være fra Vester i Øster 1. Miil langt, og ligesaa bredt, med Græss og Maasse paa. De \textit{Norske} BefaringsMænd, som jeg har udskikket, at besigte og udmerke Grendse-Stederne, have fundet og beskrevet den Østre Ende desuden at være høyest, opstigendes ligesom en Huve-Poll, ganske blaa og snau; Vesten derfor ligger Snee og Jis, og denne Vestre Deel er lidet lavere, end den Østre Pynt, og paa denne Vestre Ende skal det være ganske bratt ned ad. See II 213 videre.\par
Over denne Østre Ende, hvor den er høyest gaaer nu Landz\textit{kiølen}. Fra \textit{Dørdrik} i Nord følger en Dal, \textit{Laudok}, 2. Miile lang, fra Sør i Nord, af lave Berg-Houger sammenhengendes, med noget lidet Græss derimellem, til næste Grendse field \textit{kiesuris}. De \textit{Norske} BefaringsMænd have nu forklaret denne Dal \textit{Laudok} saaledes, at derudinden imellem \textit{Dørdrik} og \textit{kiesuris} ligge lige i Nord fra \textit{Dørdriks} Østre Ende 2. andre Fielde navnlig \textit{Allavara}, rundt, omtrent 1/2. Miil stort, og \textit{Matoive} 3/4. Miil langt fra Sør i Nord, og halv saa bredt; Disse 2. Fielde have de seet med deres høyeste Topper at ligge Snor-lige i Nord fra \textit{Dørdriks} Østre Ende til \textit{kiesuris} Vestre Ende; hvilke om de ved Opmaalingen saa skulle findes, ere da at holde for de lave BergHauger, Vidnerne have talt om; Østerst i denne \textit{Laudok} gaaer Landskabet op i Vejret til snaue opstigende Fielde, og af disse Fielde fra adskillige Steder samles en Aae, navnlig \textit{Laudok-jok}, omtrent midt i Dalen imellem \textit{Dørdrik} og \textit{kiesuris}, hvilken Aae rinder i Vester i \textit{Vastein-vatten;} Der nu hvor \textit{Laudok} Aaen falder ned fra de høyere Fielde, der\par
XXII. Tæt Østen for denne \textit{Laudok} Aaes Oprindelse lige i Nord over det Øvrige af denne \textit{Laudok-dal} gaaer \textit{linien} til det\par
XXIII. \textit{Grendse-Field kiesuris}, og i sær over dets Vestre Ende; dette \textit{Kiesuris} er den Vestre Ende af et langt Field, som strekker sig fra Vester i Øster over 2. Dagers Reise alt ned til \textit{Sverrigs} Granskoug, navnlig \textit{Niak;} Fra hvilket \textit{NiakKiesuris} ved en Field-dal skal være adskilt; \textit{Niak}-Field i sig selv paa sin Vestre Ende er nogle Børsse-Skud bredt, men siden viider sig ud. \textit{Kiesuris} for sig selv paa sin Vestre Ende er høyt og spidz, hvorover \textit{Raa-gangen} gaaer; Ellers er det fra Sør i Nord nogle Børse-skud bredt, men fra Vester i Øster 1. Miil langt. De \textit{Norske} BefaringsMænd have taget til Merke paa \textit{Kiesuris} Vestre Ende en Hvass Tind, smal og spidz øverst i Enden, som en Knivs-odde, hvorover \textit{Kiølen} skulle gaae. See \textit{Protocollen}II 213‒214. Fra \textit{Kiesuris} i Nord skal ligge det Field\hypertarget{Schn1_54079}{}Tabell over de bevidnede Grendse-Field\label{Schn1_54081} \par 
\begin{longtable}{P{0.12734082397003746\textwidth}P{0.11779026217228465\textwidth}P{0.12415730337078651\textwidth}P{0.13370786516853933\textwidth}P{0.13052434456928838\textwidth}P{0.11142322097378275\textwidth}P{0.1050561797752809\textwidth}}
 \hline\endfoot\hline\endlastfoot XXIV.\tabcellsep XXV.\tabcellsep XXVI.\tabcellsep XXVII.\tabcellsep XXVIII.\tabcellsep XXIX.\tabcellsep XXX.\\
\textit{Rarto}. Dens Vesterste Tind\tabcellsep \textit{Jokeboll} Dens Østre Ende\tabcellsep \textit{Gallavara} Dens Østre Ende\tabcellsep \textit{Jovakiorvis} Dens Østre Ende\tabcellsep \textit{Gaudelisoive} Midt derover\tabcellsep \textit{Baasoive} Midt derover\tabcellsep \textit{Allagaik} Midt derpaa\end{longtable} \par
 \par
XXIV. \textit{Rarto}, fra Vester i Øster 1. kort Miil langt, og 1/4. Miil bredt; dette \textit{Rarto} har 7. Tinder paa sig fra Vester i Øster efter hinanden, blott og snaut ovenpaa, men paa Sidene græsset med Bierkeskoug paa; Over den Vesterste Tind, som høyest, gaaer \textit{Limiten;} See herom nærmere Forklaring af BefaringsMænd II 214. Herfra \textit{Rartos} Vestre Ende gaaes over den Elv \textit{Luli-edne}, tæt Vesten forbi \textit{Lule Vattens} Vestre Ende, til det Grendse-Field\par
XXV. \textit{Jokeboll}, fra Vester i Øster 1. Miil langt, 1/4. Miil over bredt, fladt og rundrygget ovenpaa, maasset og græsset; \textit{Linien} gaaer lige i Nord fra \textit{Rartos} Vestre Ende til \textit{Jokebolls} Østre Ende, og denne Østre Ende er og høyest, derfor gaaer Grendse-skiellet derover; de \textit{Norske} Befarings Mænd have herpaa merket en stor langagtig rund Graaberg-Steen fra Sør i Nord 3. Favner lang. See II 214. videre herom.\par
XXVI. \textit{Galla-vara}, derfra i Nord 2. Miile, et lavt Field med nogle Knuder paa, fra Vester i Øster 2. Miile langt, fra Sør i Nord 1/2 Miil bredt, med et Skare i; den Østre Ende ligger lige Norden for \textit{Jokebols} Østre Ende, og er høyest; hvorfor Kiølen gaaer herover. De \textit{Norske} BefaringsMænd have herpaa afmerket den Østerste Berg-Knoll. see II 214. Herfra over \textit{Sidas vatten}, (see BefaringsMændz Relation II 214‒215.) gaaes til\par
XXVII. \textit{Jovakiorvis}, lavt med nogle Knoller paa, fra Vester i Øster 1 Miil langt, fra Sør i Nord 3/4. Miil bredt; Paa den Østre Ende, hvor høyest, er \textit{Limiten:} BefaringsMænd have merket her en Snee-løs BergKnoll. See II 215. J Nord-Nord-ost herfra ligger det field\par
XXVIII. \textit{Gaudelisoive}, rundt og snaut, 1/4. Miil stort, ikke meget høyt; Midt over dette \textit{Gaudelisoive}, hvor høyest, er Raagangen. BefaringsMænd have merket 3. runde løse Steene, fra Vester i Øster liggende, hver saa stor, som et Høelass, dog den Østerste størst, og at imellem den Østerste og Middelste \textit{Kiølen} gaaer. See deres Forklaring II 215. J Nord til Osten herfra er\par
XXIX. \textit{Baasoive}, fra Vester i Øster 1/2 Miil langt, 1/8. Miil bredt, paa Vestre Ende snaut, høyt og bratt, paa Østre Ende fladt; Over dets Mitten gaaer \textit{Kiølen;} Fordi imellem \textit{Gaudelisoive} og \textit{Baasoive} er et Vand, \textit{Gaudelisvatten}, mest rundt, 1/8. Miil stort, hvoraf Aaen rinder i Nord-vest i den \textit{Norske}\textit{Skiommen}fiord; BefaringsMænd have udmerket 3 Tinder fra Sør i Nord efter hinanden liggendes, hvorover \textit{Kiølen} skulle gaae. see II 215. Paa \textit{Baasoive} følger i Nord en Græss-dal, \textit{Snaraflak}, 1/4. Miil fra Sør i Nord; J Nord-ost fra \textit{Baasoive} over denne \textit{Snaraflak} ligger\par
XXX. \textit{Allagaik}, mest rundt med nogle smaa Field-Knoller paa, 1/2 Miil over stort, allestedz græss-groet; Paa dette \textit{Allagaik} ere smaa Vand-Kiønner, hvoraf Bækkene nedfalde paa den nordre Side, omtrent midt paa; disse Bække skifte sig nu i Dalen, \textit{Gammavuøbme}, i 2., den Eene rindendes i Nordost ad \textit{Torne-Vand}, den 2{den} i Sydvest i den \textit{Norske Gaudelis-jok;} Midt paa \textit{Allagaik}, hvor Vandene skifte sig, bliver \textit{Kiølen}. Norden for \textit{Allagaik} er Dalen \textit{Gamma-vuøbme}, med Græss og Sillie i, fra Sør i Nord 1 Mil lang.\hypertarget{Schn1_54338}{}Schnitlers Protokoller III.\label{Schn1_54340} \par 
\begin{longtable}{P{0.15879120879120878\textwidth}P{0.15879120879120878\textwidth}P{0.12142857142857141\textwidth}P{0.1447802197802198\textwidth}P{0.2662087912087912\textwidth}}
 \hline\endfoot\hline\endlastfoot XXXI.\tabcellsep XXXII\tabcellsep XXXIII.\tabcellsep XXXIV.\tabcellsep {}[XXXV]\\
\textit{Skangala} over Mitten\tabcellsep \textit{Jerbelkiok} Midt over\tabcellsep \textit{Høygom} dal\tabcellsep \textit{Nullas} Midt over\tabcellsep \textit{Duøderek} eller Tonderen i Senjens Fogderie\end{longtable} \par
 \par
XXXI. \textit{Skangala}, fra \textit{Allagaik} i Nordost 1. Miil, er fra Sydvest i Nordost 1. Miil langt, og omtrent saa bredt, græss-riigt, noget høyt og fladt ovenpaa, paa den Vestre Ende høyest; \textit{Kiølen} gaaer over Mitten, fordi Vandene skille sig ad Midt Sønden derfor paa Nordre Side af \textit{Allagaik} i \textit{Gammavuøbme} dal. Over en Græss-dal Norden derfor 1/4. Miil viid, følger i Nordost fra \textit{Skangala}\par
XXXII. \textit{Jerbelkiok}, rundagtigt, 1. Miil vidt, fladt, rundrygget, med Snee sommestedz paa, paa Sidene græsset; \textit{Kiølen} gaaer midt herover, hvor den er høyest, i Nordost; Tilmed er midt Norden derfor \textit{Høygomdal}, hvoraf Aaen rinder i Vester. BefaringsMænd have paa \textit{Allagaiks} Mitte udseet en Knoll, som det høyeste; Paa \textit{Skangala} smaa Huuler, hvor tilforn KaaberMalm er udsprengt af, og et gammelt forfaldet Huus; og paa \textit{Jerbelkiok} opreist Steene til Merke. See Bilaget \textit{Lit}. J. [II 248.]\par
XXXIII. \textit{Høygom}-dal, fra Sør i Nord 1/4. Miil viid, fra Vester i Øster 2. Miile lang; J denne Dal rinder af Kiønner en Aae i Øster ad \textit{Tornevand}, en Anden i Vester, med Navn \textit{Kattere-jok}, i den \textit{Norske}\textit{Rombaks}Fiord; \textit{Kiølen} gaaer nu Østen for de Kiønne; hvoraf Vandet rinder i Vester ad \textit{Norge}.\par
XXXIV. \textit{Nullas}-Field, i Nord-ost fra \textit{Jerbelkiok}, er fra Vester i Øster 1. Miil, fra Sør i Nord 1/2 Miil stort, ovenpaa slet, imod \textit{Tornevatten} er det med sin Nordre Side steilt nedstubendes, sommestedz snaut, sommestedz Græsset; Midt derover, hvor høyest, gaaer \textit{Kiølen}. Herfra gaaer Grendsen i Nordost over \textit{Tornevatten}, fra dets Vestre Ende 1. Miil til det Field\par
XXXV. \textit{Duøderek}, eller \textit{Tonderen}, langs efter Mitten gaaer Raagangen.\hspace{1em}
\DivI[Jan. 23. Brev fra Rømeling til Schnitler]{Jan. 23. Brev fra Rømeling til Schnitler}\label{Schn1_54510}\par
N. 1.\par
Copie af Hr. Obriste Rømelings Skrivelses \textit{ordre} til \textit{Major Schnitler} datered \textit{Friderichstad} d. 23{de}\textit{Januarij} 1743.\par
Wann ich seit verwichenen Weihnachten, wegen einer mir zugestossenen heftigen BrustKrankheit, die meiste Zeit das Bette, und stetz die Kammer habe hütten müssen, so, obwohl es sich itzo einigermassen zur Besserung anlæst, habe vergangenen Sonnabendt an S{ne}\textit{Excell:} den H. GeheimenRath \textit{von Holstein} geschrieben, dass in Ermangelung von Kräften, ich mir diesen Winter die Trundhiemsche Reise zu thuen, nicht getrauete; Da aber wegen Regulierung der künftigen Grentz-Arbeit mich mit Ew. Wohlgeb. über die gehaltene \textit{Examens} nothwendig zu besprechen hätte, so würde mit heutiger Post Mhg. \textit{Major} beordren, fordersahmst nacher \textit{Christiania} zu kommen; Welches denn hiermit bewerkstelliget haben will, und meine ich, gegen die Zeit Jhrer Ankunft, auch im Stande zu seyn, mich daselbst einzufinden.\par
{\textit{Römeling} rigtig \textit{Copie} at være \textit{testerer}Peter Schnitler mppra}Summarisk Forklaring over Nordlands Amt.
\DivI[4. volumen: 1743, om høsten.]{4. volumen: 1743, om høsten.}\label{Schn1_54584}\par
\centerline{[Av \textbf{4de Volumen} følger her det som er utelatt i bind II s. 414, foran Bilag Litr. A.]}
\DivI[Summarisk forklaring over Nordlands amt]{Summarisk forklaring over Nordlands amt}\label{Schn1_54596}\par
\textbf{Summarisk Forklaring Over Nordlands Amt} dets 4re \textit{Østlige Fogderier, grendsende til Sverrig:}\par
Dette \textit{Nordlands} Amt Grendser i Sør til \textit{Tronhiems} Amt og i sær til \textit{Nerøe} Præstegield ved Søe-Siden, og \textit{Overhaldens} Præstegield op i Landet; i Nord til \textit{Finmarken}, og i sær til \textit{Loppens} Præstegield ved Søe Siden, og \textit{Altens} Præstegield op i Landet; J Øster til \textit{Sverrigs Lapmarker;} i Vester til det store \textit{Wester}-Hav, i hvilket de 2{de} Fogderier \textit{Lofoden} og \textit{Vesteraalen} ligge, og ved \textit{Vest-Fiorden} fra \textit{Saltens} Fogderie adskillet.\par
Grendserne i Sør til \textit{Tronhiems} Amt ere Vesten fra at regne:\par
\textit{Svartbar}, nu kaldet \textit{Svartoxen}, det yderste Skiær i Vesterhav.\par
\textit{Gimsen} ‒ en liden Øe, i Sydost derfra\par
\textit{Helgelandsflæss}, et Skiær, derfra i Syd-ost.\par
\textit{Gimlingen}, en liden Øe, Syd-ost derfra\par
\textit{Rynaas}Fiord, nu kaldet \textit{Ryingen}fiord, Østen derfor.\par
Den Søndre Side af \textit{Vass-aas} Kirke og Gaard, hvor Ageren endes, Østen derfor ‒\par
\textit{Flaa-Field}, som nu kaldes \textit{Flaafoden;} Østen derfor, og derefter i Nord-ost Fieldene til de \textit{Svenske Grendser;} See herom meere, \textit{Examinations Protocollen} 2det \textit{Volumen}, Bilage N. 1.\par
De Fielde Østen for \textit{Flaafoden} er nu det store\par
\textit{Børjefield}, strekkende sig fra Vester i Øster en 15 Miile langt; See \textit{Protocoll:} 2 \textit{Volum:} 1 Vidne i Helgeland Sp. 2. ‒\par
\textit{Nordlands} Grendser i Nord til \textit{Finmarken} ere\par
1) \textit{Brynnel}, en Holm i Vesterhavet\par
2) \textit{Sokkammer}, en BergHoug paa \textit{Alt-Eidet}, Østen derfor\par
3) Oppe til Fieldz Mitt-Vejs imellem de 2{de} Vande \textit{Reise-jaure}, og \textit{Rikas-jaure}. Sønden herfor\par
4) Paa Grendse-\textit{kiølen} imellem \textit{Norge} og \textit{Sverrig} er Et lidet Eid imellem 2{de} Vande, navnlig \textit{Teno-motkie}, og næst Østen for Eidet det Field-\textit{Korsevara}, Jmellem dette \textit{Tenomotkie}, og \textit{Korse-vare} vil Skiellet blive imellem \textit{Nordland} og \textit{Finmarken;} See herom videre Forklaring i dette \textit{Volum:}\hypertarget{Schn1_54863}{}\footnote{\label{Schn1_54863}Denne er nok til bind 2}411 f.\par
\textit{Nordlands} Grendser til \textit{Sverrig} ere i næstforrige 3die, og dette 4de \textit{Volum:} paa sine Steder beskrevne.\par
\textit{Nordlands} Amt er inddeelt, Søndenfra at regne, i\par
(1) \textit{Helgelands-}\par
(2) \textit{Saltens-}\label{Schn1_54899} \par 
\begin{longtable}{P{0.5383333333333333\textwidth}P{0.31166666666666665\textwidth}}
 \hline\endfoot\hline\endlastfoot (3) \textit{Senniens-} og\tabcellsep  hvilke af 1 Fogd forestaaes\\
(4) \textit{Tromsøens}\\
(5) \textit{Lofodens} og\tabcellsep  hvilke af 1 Fogd forvaltes\\
(6) \textit{Vesteraalen}\end{longtable} \par
 \hypertarget{Schn1_54935}{}Schnitlers Protokoller IV.\par
Fogderier\hspace{1em}\par
De 4{re} første Fogderier bestaae af fast Land, grendsende i Øster til \textit{Sverrigs Lapmarker}, og nogle nær under Landet liggende Øer; de 2{de} Sidste ere langt ud i \textit{VesterHavet} fra det faste land af liggende, af luter Øer bestaaende. Om de 4re Første Fogderier er det, her i \textit{Examinations Protocollerne} er handlet;\hspace{1em}\par
\textit{Nordlands Amtes Længde} fra Sør i Nord\hspace{1em}\par
1) \textit{Helgelands Fogderie} efter Søe-Kanten af det faste Land, fra Sør i Nord-Nord-ost er lang fra \textit{Helgelandsflæss} til \textit{Fugeløe}{20 Miile} see \textit{Protocoll:} 2 \textit{Volum:} i Bilagene til \textit{Helgeland} N. (3)\par
‒ til Fieldz, efter Grendse-Gangen til \textit{Sverrig}, at reigne fra det 1{te} GrendseMerke \textit{Gavidsen-kiok}, en BergKlimp paa det store \textit{BørjeField}, hen til \textit{Stokke}-GrendseField, hvor \textit{Bejerns}-Field-\textit{Finner} af \textit{Saltens} Fogderie sidde paa, fra Sør i Nord er lang 12. Miile.\hspace{1em}\par
2)det \textit{Saltens Fogderie}\hspace{1em}\par
Efter Søe Kanten fra Sør i Nord-Nord-ost, nemlig fra forberørte \textit{Fugeløe} til \textit{Lavangens} Gaard, den 1{te} af næste \textit{Senniens}-Fogderie; see \textit{Protoc:} 4de \textit{Vol:}II 250. er lang, imod {16 ‒} See \textit{Protocoll}. 3 \textit{Volum:}II 224 f.\par
‒ til Fieldz efter Grendse-Gangen til \textit{Sverrig}, at reigne fra bemeldte \textit{Stokke}- Field til \textit{Tornevattens} Søndre Side, er lang fra Sør i Nord imod {20 ‒}\hspace{1em}\par
3die \textit{Senniens Fogderie}\hspace{1em}\par
efter SøeKanten fra berørte \textit{Lavangens} Gaard til \textit{Tenskier}-Gaard, paa Vestre Side af \textit{Malangens}Fiord i Nord-ost, er lang {9 ‒} See dette \textit{Volum:}II 320 f. ‒\par
‒ Til Fieldz efter GrendseGangen ad \textit{SverrigsTorne Lapmark}, at reigne fra \textit{Tornevattens} Søndre Side til \textit{Stokkeborre} Grendse-Field i Nord-ost, imod {5 ‒}\hspace{1em}\par
4de \textit{Tromøens Fogderie}\hspace{1em}\par
Efter Søe-Kanten fra omtalte \textit{Tenskier}-Gaard til \textit{Segelvig}, den sidste og Nordreste Gaard i \textit{Tromsøens} Fogderie, i OstNordost er imod {9 ‒} See II 413.\par
‒ Til Fieldz efter Grendse-Gandsen (sic) ad \textit{SverrigsTorne Lapmark} fra bemeldte \textit{Stokke borre} Grendse-Field mest i Øster til forbi \textit{Teno-motkie}, som riimeligen vil skille \textit{Nordland} fra \textit{Finmarken}, er lang {9 ‒} See II 412.\hspace{1em}\par
Er saa \textit{Nordland} langt fra Søer i Nord efter Søekanten {54 Miil}\par
‒ ‒ ‒ til Fieldz efter Grendse Gangen {46 Miil}\hypertarget{Schn1_55245}{}Summarisk Forklaring over Nordlands Amt.\par
Bredden af \textit{Nordlands} Amt fra Øster i Vester regnes best i Ordenen af Fiordene, hvormed Landet iidelig er indskaaret: Saaledes fra Sør i Nord\hspace{1em}\par
\centerline{i \textit{Helgelands Fogderie}}\hspace{1em}\par
1) Fra 1{te} Grendse-Merke \textit{Gavidsenkiok}, paa det store \textit{Børge}Field, hvilket med sin Østre Ende omtrent skal skille \textit{SverrigsAangermanland} og \textit{Uma-Lapmark} ad, er Landet bredt i Vester til den \textit{Norske}\textit{Bindals}Fiordz Gab, omtrent {18 1/2 Mil}\par
2) Jmod 6. Miile i Nord fra \textit{Bindals}Fiordz Gab til Gabet af \textit{Vefsen} Fiord j Vester, er imod \textit{Uma Lapm.} fra \textit{Garve} Grendsefield til Gabet av \textit{Vefsens} Fiord i Vester ‒ omtrent {18 ‒}\par
2 Miile fra \textit{Vefsen}Fiord er i Nord til \textit{Raens}Fiord,\par
3) Fra \textit{Amber}-Grendse-Field ad \textit{SverrigsUma-Lapmark} er i Vester til den \textit{Norske}\textit{Raens} Fiordz Gab omtrent {15 1/2 ‒}\par
11 Miile er fra \textit{Raens}Fiord i Helgeland til første \textit{Beier}-fiord\hspace{1em}\par
\centerline{i \textit{Saltens Fogderie}.}\hspace{1em}\par
4) Fra \textit{Stokke}.GrendseField ad \textit{SverrigsUma-Lapmark} til Gabet af \textit{Bejer}Fiord i Nordvest {12 ‒} Men fra \textit{Junkerdal} lige i Vester til bem{te}\textit{Beier}Fiordz Gab er kun {11 Mil.}\par
2 1/2 Miil er fra \textit{Bejer}Fiord i Nord til \textit{Saltens}-Fiordz Gab ‒\par
5) Fra \textit{Junker-dal}, et Grendse-Sted ad \textit{SverrigsPita-Lapmark}, er efter den \textit{ordinaire Passage} i Vest-Nord-Vest til \textit{Saltens} Fiordz Gab {11 Miile} Men lige i Vester fra \textit{Østre Laami}Vandz Østre Ende til bemeldte Gab af \textit{Saltens}- Fiord er {9 Miil}\par
3 Miile er fra \textit{Saltens}Fiordz Gab i Nord-Nord-ost til Gabet af \textit{Foldens}Fiord\par
6) Fra \textit{Steen}Field, et Grendse-Field ad \textit{SverrigsLulaLapmark} i Vester til bem{te}\textit{Foldens} Fiordz Gab er {9 ‒}\par
2 1/2 Miile er fra berørte \textit{Folden}-Fiord i Nord-Nordost til Gabet af \textit{Rotangens}Fiord i \textit{Steegens} Gield.\par
7) Fra forommeldte \textit{Steen}Field i Vester til Norden er til \textit{Rotangens} Gab {9 ‒}\par
3 Miile er fra \textit{Rotangens}Fiord til \textit{Tys}Fiord i Nord-ost\label{Schn1_55565} \par 
\begin{longtable}{P{0.7907960199004975\textwidth}P{0.029601990049751243\textwidth}P{0.029601990049751243\textwidth}}
 \hline\endfoot\hline\endlastfoot 8) Fra \textit{Gallavara}, et Grendse-Field ad \textit{SverrigsLulaLapmark}, er langs med \textit{Tys}fiorden i Nordvest til Gabet\tabcellsep 9 Miile\\
Men lige over i Vester til \textit{Tysfiord}-Gabet er kun\tabcellsep \tabcellsep 6 1/2 ‒\end{longtable} \par
 \par
1/2 Miil Norden for \textit{Tysfiord} er \textit{Ofodens}Fiord\par
9) Fra \textit{Nullas} Grendse-Field, liggendes tæt Sønden for \textit{Torne}Vatten, ad \textit{Torne- Lapmark}, er til \textit{Ofodens}Fiordz Gab i Vester imod {8 1/2 ‒}\par
4 1/2 Miil er fra \textit{Ofodens}Fiord i Nordost til \textit{Gratangens} Fiord\hspace{1em}\par
\centerline{i \textit{Senniens Fogderie}.}\hspace{1em}\par
10) Fra \textit{Torne-vatten}, 1 Miil Østen for dets Vestre Ende i Vester til den \textit{Norske}\textit{Gratangens}Fiordz Gab er {5 1/2 Miil}\hypertarget{Schn1_55677}{}Schnitlers Protokoller IV.\par
4 Miile i Nord til Osten fra \textit{Gratangens} Fiordz Gab til \textit{Reisens} Fiordz Gab\par
11) Fra \textit{Toibal}, et Grendse-Field ad \textit{SverrigsTorne Lapmark}, er i VestNordVest til \textit{Reisens} Gab {9 Miile}\par
2 1/2 Miil fra \textit{Reisens} Gab i Nord er \textit{Malangens}Fiord\label{Schn1_55723} \par 
\begin{longtable}{P{0.7928571428571428\textwidth}P{0.04285714285714286\textwidth}P{0.014285714285714285\textwidth}}
 \hline\endfoot\hline\endlastfoot 12) Fra \textit{Haugo-jaure}, et Grendse-Vand ad \textit{SverrigsTorne-Lapmark} er i Vester til \textit{Malangens}Fiordz Botten\tabcellsep 8 ‒\\
Fra Botten i Nord til Fiord-Gabet\tabcellsep 2 ‒\\
\tabcellsep _______\tabcellsep 10 ‒\\
Men fra \textit{Haugo-jaure} lige over i Vester til Søekanten, nemlig \textit{Solberg}-Fiord er\tabcellsep 9 Miile ‒\\
5/8 Miil fra \textit{Malangens}Fiord i Øster til Norden er til Gabet af Balsfiord\\
13) Fra \textit{Rosto}-Sund imod \textit{TorneLapmark} er i Vester til Norden til Gabet af \textit{Bals}Fiord imod\tabcellsep \tabcellsep 9 ‒\\
3 Miile i OstNordost fra \textit{Bals}Fiord er til \textit{Ulfs}Fiord\\
14) Fra et Eid imellem \textit{Kolte}- og \textit{Kulkiem}-Vande, et Grendse-Merke ad \textit{Torne Lapmark}, i Vester til den \textit{Norske}LøngensFiordz Botten\tabcellsep 3 1/2 ‒\\
Derfra til \textit{Sørstrøms} Botten af \textit{Ulfs}Fiord\tabcellsep 1 ‒\\
Siden over Land i Vester til Søekanten\tabcellsep 2 ‒\\
\tabcellsep _______\tabcellsep 6 1/2 ‒\end{longtable} \par
 \par
Fra \textit{Ulfs}Fiord fares strax om \textit{Ulfs-tinden} til \textit{Løngens}Fiordz Gab\par
15) Fra \textit{Kalko-gaabb}, en Grendse-Houg til \textit{Torne-Lapmark} i NordNordVest til Gabet af LøngensFiord er gode {7 Miile}\par
2 Miile i Nordost fra \textit{Løngens}Fiord til \textit{Reisens}Fiord er ‒\par
16) Fra Halde, et Grendse-Field til \textit{TorneLapmark} er i Nord-Nordvest til \textit{Reisens} Gab {7 ‒}\par
1/4 Miil i NordOst til \textit{Qvænangen}, den sidste og Nordreste Fiord i \textit{Nordland}\par
17) Fra bem{te}\textit{Halde}-Field i NordNordvest til Gabet af \textit{Qvænangens} Fiord er {7 ‒}\begin{figure}[htbp]
\noindent\par
_______
\caption{\label{Schn1_55960}}\end{figure}
\hypertarget{Schn1_55963}{}Summarisk Tabell over Nordlands Amt.
\DivI[Summarisk tabell over Nordlands amt]{Summarisk tabell over Nordlands amt}\label{Schn1_55965}\par
\centerline{\textbf{Summarisk Tabell} over \textit{Nordlands} Amt i Almindelighed}\label{Schn1_55978} \par 
\begin{longtable}{P{0.16483228511530398\textwidth}P{0.04722222222222222\textwidth}P{0.0017819706498951782\textwidth}P{0.029402515723270437\textwidth}P{0.011582809224318659\textwidth}P{0.52479035639413\textwidth}P{0.008909853249475892\textwidth}P{0.008909853249475892\textwidth}P{0.01960167714884696\textwidth}P{0.022274633123689727\textwidth}P{0.01069182389937107\textwidth}}
 \hline\endfoot\hline\endlastfoot \Panel{Fogderier}{label}{1}{l}\tabcellsep \Panel{Tinglaug eller, Fierdinger}{label}{3}{l}\tabcellsep \Panel{Præstegield}{label}{2}{l}\tabcellsep \Panel{Kirker}{label}{3}{l}\tabcellsep \Panel{Gaardbrugende Bønder}{label}{1}{l}\tabcellsep \Panel{Summa Bønder}{label}{1}{l}\\
1 \textit{Helgelands ‒}\tabcellsep \textit{Nordsems} ‒\tabcellsep 1\tabcellsep \tabcellsep \tabcellsep \tabcellsep \tabcellsep \tabcellsep \tabcellsep 92\tabcellsep \\
\tabcellsep \tabcellsep \tabcellsep \textit{Brønøe}\tabcellsep 1\tabcellsep \textit{Brønøe} Hovedkirke af Steen i \textit{Nordsems} Fierding\tabcellsep \tabcellsep 1\tabcellsep \tabcellsep \\
\textit{Bindalens} ‒\tabcellsep 1\tabcellsep \tabcellsep \tabcellsep \tabcellsep \tabcellsep \tabcellsep \tabcellsep 74\tabcellsep \\
\tabcellsep \tabcellsep \tabcellsep \tabcellsep \tabcellsep \textit{Vassaas Annex}Kirke af Træ, i \textit{Bindals} Fierding\tabcellsep \tabcellsep 1\tabcellsep \tabcellsep \\
\tabcellsep \tabcellsep \tabcellsep \tabcellsep \tabcellsep \textit{Solstad Capell} af Træ \textit{ibid}\tabcellsep \tabcellsep 1\tabcellsep \tabcellsep \\
\textit{Sørsems} ‒\tabcellsep 1\tabcellsep \tabcellsep \tabcellsep \tabcellsep \tabcellsep \tabcellsep \tabcellsep 149\tabcellsep \\
\tabcellsep \tabcellsep \tabcellsep \tabcellsep \tabcellsep \textit{Viig Annex} Kirke af Træ, i \textit{Sørsems} Fierding\tabcellsep \tabcellsep 1\tabcellsep \tabcellsep \\
\textit{Velfiordens}\tabcellsep 1\tabcellsep \tabcellsep \tabcellsep \tabcellsep \tabcellsep \tabcellsep \tabcellsep 124\tabcellsep \\
\tabcellsep \tabcellsep \tabcellsep \tabcellsep \tabcellsep \textit{Nøstviig Annex}Kirke af Træ, i \textit{Velfiordens} Fierding\tabcellsep \tabcellsep 1\tabcellsep \tabcellsep \\
\textit{Vægens} ‒\tabcellsep 1\tabcellsep \tabcellsep \tabcellsep \tabcellsep \tabcellsep \tabcellsep \tabcellsep 100\tabcellsep \\
\tabcellsep \tabcellsep \tabcellsep \tabcellsep \tabcellsep \textit{Vægens ResiderendeCapellans} Kirke af Træ i \textit{Vægens} Fierd.\tabcellsep \tabcellsep 1\tabcellsep \tabcellsep \\
\tabcellsep \tabcellsep \tabcellsep \tabcellsep \tabcellsep \tabcellsep \tabcellsep ___\\
\tabcellsep \tabcellsep 5\tabcellsep \tabcellsep \tabcellsep \tabcellsep \tabcellsep 6\tabcellsep \tabcellsep 539\\
\textit{Alstahoug} ‒\tabcellsep 1\tabcellsep \tabcellsep \tabcellsep \tabcellsep \tabcellsep \tabcellsep \tabcellsep 108\tabcellsep \\
\tabcellsep \tabcellsep \tabcellsep \textit{Alstahoug}\tabcellsep 1\tabcellsep \textit{Alstahoug} Hovedkirke af Steen i \textit{Alstahougs} Fierding\tabcellsep 1\tabcellsep \tabcellsep \tabcellsep \\
\textit{Tiøtøe} ‒ ‒\tabcellsep 1\tabcellsep \tabcellsep \tabcellsep \tabcellsep \tabcellsep \tabcellsep \tabcellsep 153\tabcellsep \\
\tabcellsep \tabcellsep \tabcellsep \tabcellsep \tabcellsep \textit{Tiøtøe Annex}Kirke af Steen i \textit{Tiøtøe} Fierding\tabcellsep 1\tabcellsep \tabcellsep \tabcellsep \\
\textit{Herrøe} ‒ ‒\tabcellsep 1\tabcellsep \tabcellsep \tabcellsep \tabcellsep \tabcellsep \tabcellsep \tabcellsep 92\tabcellsep \\
\tabcellsep \tabcellsep \tabcellsep \tabcellsep \tabcellsep \textit{Herrøe Annex}Kirke af Steen i \textit{Herrøe} Fierding\tabcellsep 1\tabcellsep \tabcellsep \tabcellsep \\
\textit{Vefsens} ‒ ‒\tabcellsep 1\tabcellsep \tabcellsep \tabcellsep \tabcellsep \tabcellsep \tabcellsep \tabcellsep 391\tabcellsep \\
\tabcellsep \tabcellsep \tabcellsep \tabcellsep \tabcellsep \textit{Sannæs Capell}\tabcellsep 1\tabcellsep \tabcellsep \tabcellsep \end{longtable} \par
 \hypertarget{Schn1_56406}{}Schnitlers Protokoller IV.\label{Schn1_56408} \par 
\begin{longtable}{P{0.19898042414355627\textwidth}P{0.03535889070146819\textwidth}P{0.0034665579119086457\textwidth}P{0.04021207177814029\textwidth}P{0.031199021207177813\textwidth}P{0.3646818923327895\textwidth}P{0.11300978792822185\textwidth}P{0.004853181076672104\textwidth}P{0.016639477977161497\textwidth}P{0.030505709624796082\textwidth}P{0.011092985318107667\textwidth}}
 \hline\endfoot\hline\endlastfoot \Panel{Fogderier}{label}{1}{l}\tabcellsep \Panel{Tinglaug eller, Fierdinger}{label}{3}{l}\tabcellsep \Panel{Præstegield}{label}{2}{l}\tabcellsep \Panel{Kirker}{label}{3}{l}\tabcellsep \Panel{Gaardbrugende Bønder}{label}{1}{l}\tabcellsep \Panel{Summa Bønder}{label}{1}{l}\\
{See \textit{Protocollens} 2 \textit{Volumen} Bilage til \textit{Helgelands} Fogderie N. 3. II 36‒53.}\tabcellsep \tabcellsep \tabcellsep \tabcellsep \tabcellsep \tabcellsep \textit{Vefsens Residerende Capellans}, \textit{Dolstad} Kirke af Træ i \textit{Vefsens} Fierding\tabcellsep 1\tabcellsep \tabcellsep \tabcellsep \\
\tabcellsep \tabcellsep 4\tabcellsep \tabcellsep \tabcellsep \tabcellsep \tabcellsep 5\tabcellsep \tabcellsep 744\\
\textit{Næsne}\tabcellsep \tabcellsep 1\tabcellsep \tabcellsep \tabcellsep \tabcellsep \tabcellsep \tabcellsep \tabcellsep 183\\
\tabcellsep \tabcellsep \tabcellsep \textit{Næsne}\tabcellsep 1\tabcellsep \textit{Næsne} Hovedkirke af Træ i \textit{Næsne} Fierding\tabcellsep 1\tabcellsep \tabcellsep \tabcellsep \\
\tabcellsep \tabcellsep \tabcellsep \tabcellsep \tabcellsep \textit{Dønnes Annex}Kirke af Steen \textit{ibidem}\tabcellsep 1\tabcellsep 2\tabcellsep \tabcellsep \\
\textit{Raëns}\tabcellsep \tabcellsep 1\tabcellsep \tabcellsep \tabcellsep \tabcellsep \tabcellsep \tabcellsep \tabcellsep 417\\
\tabcellsep \tabcellsep \tabcellsep \textit{Raens}\tabcellsep 1\tabcellsep \textit{Hemnæss} Kirke af Træ i \textit{Raens} Fierding\tabcellsep 1\tabcellsep \tabcellsep \tabcellsep \\
\tabcellsep \tabcellsep \tabcellsep \tabcellsep \tabcellsep \textit{Moë Annex}Kirke af træ \textit{ibid.}\tabcellsep 1\tabcellsep 2\tabcellsep \tabcellsep \\
\textit{Rødøe}\tabcellsep 1\tabcellsep \tabcellsep \tabcellsep \tabcellsep \tabcellsep \tabcellsep \tabcellsep 111\tabcellsep \\
\tabcellsep \tabcellsep \tabcellsep \textit{Rødøe}\tabcellsep 1\tabcellsep \textit{Rødøe} Hovedkirke af Træ i \textit{Rødøe} Fierding\tabcellsep 1\tabcellsep \tabcellsep \tabcellsep \\
\textit{Lurøe}\tabcellsep 1\tabcellsep \tabcellsep \tabcellsep \tabcellsep \tabcellsep \tabcellsep \tabcellsep 112\tabcellsep \\
\tabcellsep \tabcellsep \tabcellsep \tabcellsep \tabcellsep \textit{Lurøe Annex}kirke af Træ i \textit{Lurøe} Fierding\tabcellsep 1\tabcellsep \tabcellsep \tabcellsep \\
\tabcellsep \tabcellsep \tabcellsep \tabcellsep \tabcellsep \textit{Trænen Capell} af Træ \textit{ibid.}\tabcellsep 1\tabcellsep \tabcellsep \tabcellsep \\
\textit{Mæløe}\tabcellsep 1\tabcellsep \tabcellsep \tabcellsep \tabcellsep \tabcellsep \tabcellsep \tabcellsep 163\tabcellsep \\
\tabcellsep \tabcellsep 3\tabcellsep \tabcellsep \tabcellsep \textit{Mæløe Annex}Kirke af Træ\tabcellsep 1\tabcellsep 4\tabcellsep \tabcellsep 386\\
\multicolumn{3}{l}{\textit{Summa Helgelands Tinglaug}}\tabcellsep 14\tabcellsep Præstegield\tabcellsep 5\tabcellsep \multicolumn{2}{l}{Kirker}\tabcellsep 19\tabcellsep Bønder\tabcellsep 2269\\
2det \textit{Saltens}\tabcellsep \textit{Gilleskaals}\tabcellsep 1\tabcellsep \tabcellsep \textit{Gilleskaal}\tabcellsep 1\tabcellsep \textit{Gilleskaals} HovedKirke af Steen\tabcellsep 1\tabcellsep \tabcellsep 162\tabcellsep \\
\tabcellsep \tabcellsep \tabcellsep \tabcellsep \tabcellsep \textit{Moljords Annex}Kirke af træ\tabcellsep 1\tabcellsep 2\tabcellsep 72\tabcellsep 234\\
\tabcellsep \tabcellsep \tabcellsep \tabcellsep \tabcellsep \tabcellsep ___\tabcellsep \tabcellsep _____\\
\textit{Bodøe}\tabcellsep 1\tabcellsep \tabcellsep \textit{Bodøe}\tabcellsep 1\tabcellsep \textit{Bodøe} Hovedkirke af Steen\tabcellsep 1\tabcellsep \tabcellsep 218\tabcellsep \\
\textit{Skierstad} og \textit{Dalens}\tabcellsep 1\tabcellsep \tabcellsep \tabcellsep \tabcellsep \textit{Skierstads Residerende Capellans} Kirke af Træ\tabcellsep 1\tabcellsep \tabcellsep 261\tabcellsep \end{longtable} \par
 \hypertarget{Schn1_56891}{}Summarisk Tabell over Nordlands Amt.\label{Schn1_56893} \par 
\begin{longtable}{P{0.24181605611847232\textwidth}P{0.036438035853468435\textwidth}P{0.0006625097427903351\textwidth}P{0.035775526110678094\textwidth}P{0.02981293842556508\textwidth}P{0.37498051441932967\textwidth}P{0.04836321122369447\textwidth}P{0.005962587685113016\textwidth}P{0.037763055339049105\textwidth}P{0.027162899454403738\textwidth}P{0.011262665627435697\textwidth}}
 \hline\endfoot\hline\endlastfoot \Panel{Fogderier}{label}{1}{l}\tabcellsep \Panel{Tinglaug eller, Fierdinger}{label}{3}{l}\tabcellsep \Panel{Præstegield}{label}{2}{l}\tabcellsep \Panel{Kirker}{label}{3}{l}\tabcellsep \Panel{Gaardbrugende Bønder}{label}{1}{l}\tabcellsep \Panel{Summa Bønder}{label}{1}{l}\\
{See \textit{Protocollens} 3 \textit{Volum}. II 224. ‒}\\
\tabcellsep \tabcellsep \tabcellsep \tabcellsep \tabcellsep \textit{Saltdals}\textit{Residerende Capellans} Kirke af træ\tabcellsep 1\tabcellsep \tabcellsep 81\tabcellsep \\
\textit{Foldens}\tabcellsep 1\tabcellsep \tabcellsep \tabcellsep \tabcellsep \textit{Rørstads}\textit{Residerende Capellans} Kirke af Træ\tabcellsep 1\tabcellsep \tabcellsep 132\tabcellsep \\
\tabcellsep \tabcellsep \tabcellsep \tabcellsep \tabcellsep \textit{Kierringøe Annex} under Rørstad af Træ\tabcellsep 1\\
\tabcellsep \tabcellsep \tabcellsep \tabcellsep \tabcellsep \tabcellsep ___\tabcellsep 5\tabcellsep _______\tabcellsep 692\\
\textit{Engel-} og\tabcellsep \tabcellsep \tabcellsep \textit{Steegens}\tabcellsep 1\tabcellsep \textit{Steegens} Hovedkirke af Steen\tabcellsep 1\tabcellsep \tabcellsep 80\tabcellsep \\
\textit{Leedingen}\tabcellsep 1\tabcellsep \tabcellsep \tabcellsep \tabcellsep \textit{Leedingens Annex}kirke af Træ\tabcellsep 1\tabcellsep \tabcellsep 60\tabcellsep \\
\textit{Hammerøe}\tabcellsep 1\tabcellsep \tabcellsep \tabcellsep \tabcellsep \textit{Hammerøe}\textit{Residerende Capellans} Kirke af Træ\tabcellsep 1\tabcellsep 3\tabcellsep 150\tabcellsep 290\\
\tabcellsep \tabcellsep \tabcellsep \tabcellsep \tabcellsep \tabcellsep ___\tabcellsep \tabcellsep _______\tabcellsep \\
\textit{Hægstads}\tabcellsep 1\tabcellsep \tabcellsep \textit{Lødingens}\tabcellsep 1\tabcellsep \textit{Lødingens} Hovedkirke af træ\tabcellsep 1\tabcellsep \tabcellsep \tabcellsep \\
\textit{Tiellesund}\tabcellsep 1\tabcellsep \tabcellsep \tabcellsep \tabcellsep \textit{Stokke Annex}kirke af Træ\tabcellsep 1\tabcellsep 2\tabcellsep \multicolumn{2}{l}{tilsammen}\tabcellsep \\
\tabcellsep \tabcellsep \tabcellsep \tabcellsep \tabcellsep \tabcellsep ___\\
\textit{Ofodens}\tabcellsep 1\tabcellsep \tabcellsep \textit{Ofodens}\tabcellsep 1\tabcellsep \textit{Evenæss} Hovedkirke af træ\tabcellsep 1\tabcellsep \tabcellsep \tabcellsep \\
\tabcellsep \tabcellsep \tabcellsep \tabcellsep \tabcellsep \textit{Ankenæss Annex}K. af træ\tabcellsep 1\tabcellsep 2\tabcellsep \multicolumn{2}{l}{tilsamm}\tabcellsep \\
\tabcellsep \tabcellsep \tabcellsep \tabcellsep \tabcellsep \tabcellsep \tabcellsep ___\\
\multicolumn{3}{l}{\textit{Summa Saltens Tinglaug}}\tabcellsep 9\tabcellsep Præstegield\tabcellsep 5\tabcellsep \multicolumn{2}{l}{Kirker}\tabcellsep 14\tabcellsep Bønder\tabcellsep 15\\
\tabcellsep \tabcellsep \tabcellsep \tabcellsep \tabcellsep \tabcellsep \tabcellsep \tabcellsep \tabcellsep \tabcellsep 1\\
\tabcellsep \tabcellsep \tabcellsep \tabcellsep \tabcellsep \tabcellsep \tabcellsep \tabcellsep \tabcellsep \tabcellsep 2\\
3die \textit{Senniens}\tabcellsep \textit{Astafiords}\tabcellsep 1\tabcellsep \tabcellsep \textit{Astafiords}\tabcellsep 1\tabcellsep \textit{Jbestads} Hovedkirke af Steen\tabcellsep 1\tabcellsep \tabcellsep 175\tabcellsep 9\\
1\\
3\\
\textit{Dyrøe}\tabcellsep 1\tabcellsep \tabcellsep \tabcellsep \tabcellsep \textit{HavnensAnnex}k. af Træ\tabcellsep 1\tabcellsep \tabcellsep 36\tabcellsep e\\
\textit{Baltestad} og \textit{Giisund}\tabcellsep 1\tabcellsep \tabcellsep \tabcellsep \tabcellsep \textit{Tranøe Annex}kirke af træ\tabcellsep 1\tabcellsep \tabcellsep 53\tabcellsep 1\\
\textit{Lendvigens Residerende Capellans} Kirke af træ\tabcellsep 1\tabcellsep \tabcellsep 57\tabcellsep n\\
7\\
\textit{Fuskevogs}\tabcellsep 1\tabcellsep \tabcellsep \tabcellsep \tabcellsep Til \textit{Hillesøe} Kirke i \textit{Tromsøe} søge af dette Gield\tabcellsep \tabcellsep 4\tabcellsep 12\tabcellsep 7\\
\tabcellsep \tabcellsep \tabcellsep \tabcellsep \tabcellsep \tabcellsep ___\tabcellsep \tabcellsep _______\tabcellsep 233\end{longtable} \par
 \hypertarget{Schn1_57418}{}Schnitlers Protokoller IV.\label{Schn1_57420} \par 
\begin{longtable}{P{0.1725563909774436\textwidth}P{0.12355889724310776\textwidth}P{0.07456140350877193\textwidth}P{0.16403508771929823\textwidth}P{0.09373433583959899\textwidth}P{0.05964912280701754\textwidth}P{0.11006683375104427\textwidth}P{0.0028404344193817877\textwidth}P{0.0056808688387635755\textwidth}P{0.030534670008354216\textwidth}P{0.012781954887218044\textwidth}}
 \hline\endfoot\hline\endlastfoot \Panel{Fogderier}{label}{1}{l}\tabcellsep \Panel{Tinglaug, eller Fierdinger}{label}{3}{l}\tabcellsep \Panel{Præstegield}{label}{2}{l}\tabcellsep \Panel{Kirker}{label}{3}{l}\tabcellsep \Panel{Gaardbrugende Bønder}{label}{1}{l}\tabcellsep \Panel{Summa Bønder}{label}{1}{l}\\
{See \textit{Protocollens} 4 \textit{Volum:}II 301 f.}\tabcellsep \textit{Torskenss, Gryllefiords, Bergs og Mæfiords}\tabcellsep 1\tabcellsep \tabcellsep \textit{Trondenæss} eller Tronæss\tabcellsep \tabcellsep \textit{Tronæss} Hovedkirke af Steen\tabcellsep 1\tabcellsep \tabcellsep 150\tabcellsep \\
1\tabcellsep \textit{Torskens}\textit{Residerende Capellans} Kirke af Træ\tabcellsep 1\tabcellsep \tabcellsep 30\tabcellsep \\
\tabcellsep \tabcellsep \tabcellsep \textit{Gryllefiords Annex}-Kirke derunder af træ\tabcellsep 1\\
\tabcellsep \tabcellsep \textit{Mæfiords Residerendes} Kirke af træ\tabcellsep 1\tabcellsep \tabcellsep 24\tabcellsep \\
\tabcellsep \tabcellsep \textit{Berg Annex}Kirke derunder af Træ\tabcellsep 1\tabcellsep 5\tabcellsep 204\\
\tabcellsep \tabcellsep \tabcellsep ___\tabcellsep \tabcellsep _______\\
\textit{Qvæfiords}\tabcellsep 1\tabcellsep \textit{Qvæfiords}\tabcellsep 1\tabcellsep \textit{Qvæfiord} kirke af træ navnlig Raa\tabcellsep \tabcellsep 1\tabcellsep \tabcellsep 100\\
\textit{Sands}\tabcellsep 1\tabcellsep \tabcellsep \textit{Sands}\tabcellsep 1\tabcellsep \textit{Sands} kirke af Træ\tabcellsep \tabcellsep 1\tabcellsep \tabcellsep 70\\
\multicolumn{3}{l}{\textit{Summa Senniens Tinglaug}}\tabcellsep 7\tabcellsep Præstegield\tabcellsep 4\tabcellsep Kirker\tabcellsep \tabcellsep 11\tabcellsep Bønder\tabcellsep 707\\
4de \textit{Tromsøens}\tabcellsep \textit{Hillesøe}\tabcellsep 1\tabcellsep \tabcellsep \textit{Tromsøes}\tabcellsep 1\tabcellsep \textit{Tromsøes} Hovedkirke af Træ\tabcellsep 1\tabcellsep \tabcellsep 158\tabcellsep \\
\textit{Helgøe}\tabcellsep 1\tabcellsep \tabcellsep \tabcellsep \tabcellsep \textit{Helgøe Annex}kirke derunder af Træ\tabcellsep 1\tabcellsep \tabcellsep 35\tabcellsep \\
{See \textit{Protocoll:} 4 \textit{Volumen}II 412 ff}\tabcellsep \textit{Skiervøe}\tabcellsep 1\tabcellsep \tabcellsep \tabcellsep \tabcellsep \textit{Carlsøe}\textit{Residerende Capellans} Kirke af Træ\tabcellsep 1\tabcellsep \tabcellsep 110\tabcellsep \\
\tabcellsep \tabcellsep \tabcellsep \textit{Skiervøe Resider}endes af træ\tabcellsep 1\tabcellsep \tabcellsep 158\tabcellsep \\
\tabcellsep \tabcellsep \tabcellsep \textit{Hillesøe Annex}kirke ligger i \textit{Tromsøe} Fogderie, betienes af \textit{Senniens} Præst til \textit{Lendvigen}\tabcellsep 1\tabcellsep \tabcellsep 30\tabcellsep \\
\multicolumn{3}{l}{ Summa Tromsøens Tinglaug}\tabcellsep 3\tabcellsep Præstegield\tabcellsep 1\tabcellsep \multicolumn{2}{l}{Kirker}\tabcellsep 5\tabcellsep Bønder\tabcellsep 491\end{longtable} \par
 \hypertarget{Schn1_57795}{}Summarisk Tabell over Nordlands Amt.\label{Schn1_57797} \par 
\begin{longtable}{P{0.5316176470588235\textwidth}P{0.060294117647058824\textwidth}P{0.1051470588235294\textwidth}P{0.14044117647058824\textwidth}P{0.0022058823529411764\textwidth}P{0.0014705882352941176\textwidth}P{0.008823529411764706\textwidth}}
 \hline\endfoot\hline\endlastfoot \Panel{Fogderier}{label}{1}{l}\tabcellsep \Panel{Præstegield}{label}{2}{l}\tabcellsep \Panel{Kirker}{label}{2}{l}\tabcellsep \Panel{}{label}{1}{l}\tabcellsep \Panel{Bønders Tall}{label}{1}{l}\\
5te \textit{Lofodens} Fogderie\tabcellsep Røst\tabcellsep 1\tabcellsep \textit{Røst} Hovedkirke af Træ paa den Søndreste Øe \textit{Røst}\tabcellsep 1\\
\textit{Verøe-Annex}Kirke af Træ, paa \textit{Verøe}\tabcellsep 1\tabcellsep 2\\
\tabcellsep ___\\
\textit{Mosknæss}\tabcellsep 1\tabcellsep \textit{Flakstad} Kirke af Træ\tabcellsep 1\\
\textit{Mosknæss}-Kirke af Træ paa \textit{Moskøe}\tabcellsep 1\tabcellsep 2\\
\tabcellsep ___\\
\textit{Boxnæss}\tabcellsep 1\tabcellsep \textit{Boxnæss} HovedKirke af Træ paa \textit{Lofodens}-øe\tabcellsep 1\\
\textit{Hoël Annex}Kirke af træ \textit{ibidem}\tabcellsep 1\tabcellsep 2\\
\tabcellsep ___\\
\textit{Borgens}\tabcellsep 1\tabcellsep \textit{Borgens} Kirke af træ \textit{ibidem}\tabcellsep 1\\
\textit{Valbergs Annex}Kirke af træ \textit{ibidem}\tabcellsep 1\tabcellsep 2\\
\tabcellsep ___\\
Dette \textit{Vaagens} Gield\textit{sorterer} under \textit{Lofodens} Verdzlige og \textit{Saltens} Geistlig \textit{Jurisdiction}\tabcellsep \textit{Vaagens}\tabcellsep 1\tabcellsep \textit{Vaagens} HovedKirke af Træ paa \textit{Vaagen}-øe\tabcellsep 1\\
\textit{Gimsøe AnnexKirke} af Træ paa samme Navns øe\tabcellsep 1\tabcellsep 2\\
\textit{Summa Lofodens}\tabcellsep Præstegield\tabcellsep 5\tabcellsep \multicolumn{2}{l}{Kirker}\tabcellsep 10\\
6te \textit{Vesteraalens Fogderie}\tabcellsep \textit{Hassels}\tabcellsep 1\tabcellsep \textit{Hassels} HovedKirke af Træ paa \textit{Ulvøe}\tabcellsep 1\\
\textit{Sanne Annex}Kirke af Træ\tabcellsep 1\\
\textit{Sortlands Annex}Kirke af Træ paa \textit{Langøe}\tabcellsep 1\\
\textit{Boë Residerendes} Kirke af Træ paa dito \textit{Langøe}\tabcellsep 1\\
\textit{Malnæss Annex}Kirke af Træ sammestedz\tabcellsep 1\end{longtable} \par
 \hypertarget{Schn1_58102}{}Schnitlers Protokoller IV.\label{Schn1_58104} \par 
\begin{longtable}{P{0.30403022670025187\textwidth}P{0.08992443324937027\textwidth}P{0.006423173803526448\textwidth}P{0.3425692695214106\textwidth}P{0.012846347607052897\textwidth}P{0.006423173803526448\textwidth}P{0.08778337531486145\textwidth}}
 \hline\endfoot\hline\endlastfoot \Panel{Fogderier}{label}{1}{l}\tabcellsep \Panel{Præstegield}{label}{2}{l}\tabcellsep \Panel{Kirker}{label}{2}{l}\tabcellsep \Panel{}{label}{1}{l}\tabcellsep \Panel{Bønders Tall}{label}{1}{l}\\
\tabcellsep \tabcellsep \tabcellsep \textit{Øxnæss VicePastors} Kirke af Træ paa Øen af samme Navn\tabcellsep 1\\
\tabcellsep \tabcellsep \tabcellsep \textit{Langnæss Annex}Kirke af Træ\tabcellsep 1\tabcellsep 7\\
\tabcellsep \tabcellsep \tabcellsep \tabcellsep ___\\
\tabcellsep \textit{Dverbergs}\tabcellsep 1\tabcellsep \textit{Dvergbergs} HovedKirke af Træ paa Øen Anden\tabcellsep 1\\
\textit{Biørnskinds Annex}Kirke af Træ \textit{ibidem}\tabcellsep 1\\
\textit{Andnæss Residerendes} Kirke af Træ ibidem\tabcellsep 1\tabcellsep 3\\
\textit{Summa af Vesteraalens}\tabcellsep Præstegield\tabcellsep 2\tabcellsep \multicolumn{2}{l}{Kirker}\tabcellsep 10\tabcellsep Bønder i begge Fogderier 1060\end{longtable} \par
 \par
\textit{Summene} hid \textit{transporterede} blive\label{Schn1_58220} \par 
\begin{longtable}{P{0.068\textwidth}P{0.36719999999999997\textwidth}P{0.15639999999999998\textwidth}P{0.0272\textwidth}P{0.057800000000000004\textwidth}P{0.044199999999999996\textwidth}P{0.0544\textwidth}P{0.07479999999999999\textwidth}}
 \hline\endfoot\hline\endlastfoot 1te\tabcellsep \textit{Helgelands}\tabcellsep Fogderie har Præstegield\tabcellsep 5\tabcellsep Kirker\tabcellsep 19\tabcellsep Bønder\tabcellsep 2269\\
2det\tabcellsep \textit{Saltens}\tabcellsep ‒\tabcellsep 5\tabcellsep ‒\tabcellsep 14\tabcellsep ‒\tabcellsep 1517\\
3die\tabcellsep \textit{Senniens}\tabcellsep ‒\tabcellsep 4\tabcellsep ‒\tabcellsep 11\tabcellsep ‒\tabcellsep 707\\
4de\tabcellsep \textit{Tromsøens}\tabcellsep ‒\tabcellsep 1\tabcellsep ‒\tabcellsep 5\tabcellsep ‒\tabcellsep 491\\
5te\tabcellsep \textit{Lofodens}\tabcellsep ‒\tabcellsep 5\tabcellsep ‒\tabcellsep 10\tabcellsep ‒\tabcellsep 1060\\
6te\tabcellsep \textit{Vesteraalens}\tabcellsep ‒\tabcellsep 2\tabcellsep ‒\tabcellsep 10\\
\tabcellsep \tabcellsep I alt Præstegield\tabcellsep 22\tabcellsep Kirker\tabcellsep 69\tabcellsep Bønder\tabcellsep 6044\end{longtable} \par
 \hspace{1em}\par
Af en Kongelig Betient er mig berettet, at Gaardbrugende Bønder i \textit{Nordland} ere\label{Schn1_58362} \par 
\begin{longtable}{P{0.0047752808988764045\textwidth}P{0.4918539325842696\textwidth}P{0.05252808988764045\textwidth}P{0.17668539325842697\textwidth}P{0.12415730337078651\textwidth}P{0\textwidth}}
 \hline\endfoot\hline\endlastfoot i\tabcellsep \textit{Helgelands}\tabcellsep Fogderie\tabcellsep 2242\tabcellsep Familier\\
\tabcellsep \textit{Saltens}\tabcellsep ‒\tabcellsep 1452\tabcellsep ‒\\
\tabcellsep \textit{Senniens}\tabcellsep ‒\tabcellsep 762\tabcellsep ‒\\
\tabcellsep \textit{Tromsøens}\tabcellsep ‒\tabcellsep 490\tabcellsep ‒\\
\tabcellsep \multicolumn{2}{l}{Lofodens og Vesteraalens Fogderie}\tabcellsep 1060\tabcellsep ‒\\
\tabcellsep \tabcellsep \tabcellsep \multicolumn{2}{l}{_______________}\\
\tabcellsep \tabcellsep \tabcellsep 6006\tabcellsep \multicolumn{2}{l}{BønderFamilier}\end{longtable} \par
 \hypertarget{Schn1_58443}{}Tabell over de bevidnede Grendse-Field.
\DivI[Tabell over bevidnede grensefjell]{Tabell over bevidnede grensefjell}\label{Schn1_58445}\par
\centerline{\textbf{Tabell}}\par
over de bevidnede \textit{Grendse-Field} og \textit{Merker} Jmellem \textit{Norge} og \textit{Sverrig} og især Jmellem de \textit{Norske}\textit{Senniens} og \textit{Tromsøens}-Fogderier, og \textit{SverrigsTorne Lapmark}, fra Sør i Nord at reigne, \textit{Extrahered} af \textit{Examinations Protocollens} 4de \textit{Volumen}, efter Grendse-Fieldet \textit{Duøderek}, eller \textit{Tondern}, Det Sidste i \textit{Protocollens} 3die Volumen\label{Schn1_58507} \par 
\begin{longtable}{P{0.653157894736842\textwidth}P{0.1968421052631579\textwidth}}
 \hline\endfoot\hline\endlastfoot 1.\tabcellsep 2.\\
\centerline{\textit{Duøderek}, eller \textit{Tondern}, fra en Berg-Knoll, \textit{Hierta} derpaa,  langs efter dets Mitte til den Ostlige Ende.}\tabcellsep \textit{Kiolmoive}, over dets Ostlige Ende\end{longtable} \par
 \par
Efter næstforrige 3die \textit{Protocolls} Udviis, gaaer Grendse-\textit{Linien}, Sønden-fra at reigne, mitt over det Grendse-Field \textit{Nullas}, som ligger tæt Sønden for \textit{Tornevattens} Vestre Ende, siden over \textit{Tornevattenet}, hvor det er omtrent 1 Miil over bredt, og fra dets Vestre Ende 1 Miil; Hvilket \textit{Torne-vatten} beskrives, at strekke sig fra Vester i Øster 6. Miile langt; derfra kommer man til det Grendse-Field \textit{Duøderek}, paa Nordre Side af \textit{Tornevatten} liggendes, i Nordost.\par
1. Dette \textit{Duøderek}, eller som det her kaldes, \textit{Tondern}, stikker fra Sydvest omtrent i Nordost en 2 Dagers Reise langt, dog ikke saavidt i Øster, som \textit{Tornevatten}; J den Vestre Ende skal det være saa bredt, som Bierke-Skougen der Vesten for er, nemlig omtrent 1 1/2 Miil, og denne Bredde meer og mindre beholder det indtil sin Nord-ostlige Ende; Fra \textit{Tornevatten} ligger det i Nord 1/4 Miil, i hvilket Mellem-Rom er Bierk og Græss nok, som og er paa \textit{Duødereks} Nordre Side; Ovenpaa er det sommestedz snaut; sommestedz græsset, sommestedz Jis-tækket, og knollet, sommestedz fladt; Paa den Vestre Ende rund-rygget med nogle runde Tinder paa, ikke saa meget høye, dog Een deraf saa steil, at der ei kan kommes op til; Paa Østre Ende høyt, og sammestedz paa alle Sider bratt; Paa den Nord-ostlige Side sommestedz flaut og steilt, sommestedz hælder det ned med Græss-groede Lier. ‒\par
Efter Mitten af dette \textit{Duøderek} gaaer Landz\textit{kiølen} lige til den Ostlige Ende; Thi der er det høyest, og derfra har man hørt, at Bække rinde til begge Sider.\par
De \textit{Norske} Befarings Mænd med \textit{Finne}-Skolemesteren, som med Vidnerne opsendt bleve, at besee og udmerke Grendsen, have \textit{observeret} en Berg-Knoll paa \textit{Duødereks} Vestre Deel, navnlig \textit{Hierta}, 2 Miile i NordNordost fra \textit{Nullas} Grendse-Field liggendes; og dette \textit{Hierta} bevidnede 3de Mænd, at have hørt, at være et Skielle-Merke.\par
Da dette \textit{Duøderek} ligger imellem \textit{Torne}- og \textit{Altes}-Vande, som stikke j Øster, vil riimelig \textit{Duøderek} og gaae i Øster. ‒ Fra dette \textit{Duødereks} Ostlige Ende gaaer Limiten i Nordost til det Grendse-Field \textit{Kiolmoive}.\par
2. \textit{Kiolm-oive} er fra Vester i Øster 1/2 Miil langt, halv saa bredt, ikke ret høyt, fladt ovenpaa. Over dette \textit{Kiolm-oives} Østre Ende gaaer \textit{Kiølen;} Thi Norden for dets Østre Ende ligger det Vand \textit{Kamas-jaure}, hvoraf Aaen rinder i Vester i det \textit{Norske} Vand \textit{Altesjaure;} hvorom meere næstefter følger. Fra \textit{Kiolmoives} Østre Ende kommer man i Nord til\hypertarget{Schn1_58690}{}Schnitlers Protokoller IV.\label{Schn1_58692} \par 
\begin{longtable}{P{0.16558441558441558\textwidth}P{0.10762987012987012\textwidth}P{0.17662337662337663\textwidth}P{0.22629870129870128\textwidth}P{0.17386363636363636\textwidth}}
 \hline\endfoot\hline\endlastfoot 3.\tabcellsep 4.\tabcellsep 5.\tabcellsep 6.\tabcellsep 7.\\
Mitt over et \textit{Eid} imellem 2de \textit{Kamas-Vande}\tabcellsep \textit{Tarfel-vare}, Mitt efter den.\tabcellsep \textit{Toibal}-Field der hvor følgende Eid støder paa i Nord.\tabcellsep \textit{Eid} imellem \textit{Haugojaure} og \textit{Korve-jaure}, mitt derefter.\tabcellsep \textit{Stokkeborre} der, hvor følgende Eid støder an i Nord.\end{longtable} \par
 \par
3. Et \textit{Eid} (\textit{isthmus}) bestaaendes af slet Maasse-Land imellem 2de Vande, det Norske \textit{Kamas-jaure}, og det \textit{Svenske Kamas-jaure}, 1 Riffel-Skud bredt; det \textit{Norske Kamasjaure} ligger fra Sydvest i Nordost 1/4 Miil langt, 3 RiffelSkud over bredt, og heraf rinder Aaen, med det Navn \textit{Kamasjok}, 1/2 Miil lang i Vester i det \textit{Norske}\textit{Altesjaures} Vand i dets Nordostlige Ende, som er en Poll eller Bugt, kalded \textit{Caitse lugte}. Det \textit{Svenske} Vand \textit{Kamasjaure} ligger 1 RiffelSkud Østen for det \textit{Norske Kamasjaure}, strekkendes sig ligeledes i Nordost, dog noget længere og bredere; Af dette \textit{Østere Kamasjaure} løber Aaen i Øster ad Sverrig. Mitt over dette Eid gaaer Landskiftet fra Sør i Nord. 1/4 Miil i Nord-ost fra dette Eid ligger\par
4. \textit{Tarfel-vare;} en smal Field-Houg med smaa Bierk paa Sidene, imellem 2de Vande, \textit{Tarfel-jaure} i Vester, og \textit{Kiepan-jaure} i Øster; Hint \textit{Tarfel vare} er 1/4 Miil fra Sør i Nord langt, og 1 Riffelskud bredt; det Vand \textit{Tarfeljaure} er rundt, 3 Riffelskud vidt, deraf Aaen \textit{Tarfel-jok} gaaer i Syd-vest 1/2 Miil lang i den \textit{Norske Kamasjok}. 1 Riffelskud Østen for dette \textit{Norske Tarfel-jaure} er det \textit{Svenske} Vand \textit{Kiepan-jaure}, fra Vester i Øster 1 god Miil langt, og fra Sør i Nord 1/8 Miil bredt, hvoraf Aaen stævner i Nordost ad \textit{Sverrig}. Grendse-Skiellet gaaer mitt efter denne Field-Houg \textit{Tarfel-vare} i Nord. ‒\par
5. \textit{Toibal}, et GrendseField, fra Vester i Øster 1. Miil langt, imod 1/4 Miil bredt, oventil rundrygget, med Maasse paa, og noget lidet Græss paa Sidene; dette \textit{Toibal} ligger Norden for den FieldHoug \textit{Tarfel-vare} med sin Vestre Ende, og Sønden for et Eid, der er imellem 2de Vande, \textit{Haugo-jaure} i Vester, og \textit{Korve-jaure} i Øster; Der nu, hvor Mitten af dette Eid støder til \textit{Toibals} Nordre Side, derhen gaaer Grendse-\textit{linien} over Fieldet \textit{Toibal}, som Vidnet siger her, at blive den Vestre Ende af bem.te \textit{Toibal}-Field. J Nord paa dette \textit{Toibal} følger bemeldte\par
6. \textit{Eid} (\textit{isthmus}) imellem \textit{Haugo-jaure} og \textit{Korve-jaure;} Hint \textit{Haugo-jaure} er rundt, 2. Riffel-Skud vidt, hvoraf \textit{Maals} Elven rinder i Vester i den \textit{Norske}\textit{Malangens}Fiord; dette \textit{Korve-jaure} er fra Vester i Øster 1 knap Miil langt, 1/8 Miil bredt, hvoraf Aaen flyder i Øster ad \textit{Sverrig;} Eidet imellem disse 2de Vande er et Par Riffel-skud over bredt fra Vester i Øster, af slet Land med Maasse og Steen-Urer paa; Mitt over og efter dette Eid gaaer GrendseSkiellet i Nord.\par
\centerline{\textit{Østen} for \textit{Tromsøens Fogderie}}\par
7. \textit{Stokkeborre}, et Field, fra Vester i Øster 1 1/2 Miil langt, 3/4 Miil over bredt, oventil fladt med Maasse paa; Vidnet i \textit{Balsfiord} Norden-for har udsagt, at dette \textit{Stokkeborre} i sig selv kun er 1/4 Miil stort, dog kan det have smaa Field-Voler og Tange fra Vester i Øster med sig sammenhengende, at, naar man vil reigne dem dertil, kan det blive længere, end 1/4 Miil.\par
Norden for dette \textit{Stokkeborre} ligger et Eid (\textit{isthmus}) imellem 2de Vande, nemlig \textit{Vestre- Raudekiølma-jaure}, og \textit{Østre Raudekiølmajaure;} Hvor nu dette Eid med dets Mitte støder an\hypertarget{Schn1_58985}{}Tabell over de bevidnede Grendse-Field.\label{Schn1_58987} \par 
\begin{longtable}{P{0.33716666666666667\textwidth}P{0.2946666666666667\textwidth}P{0.21816666666666665\textwidth}}
 \hline\endfoot\hline\endlastfoot 8.\tabcellsep 9.\tabcellsep 10.\\
Et \textit{Eid} imellem de Vande, \textit{Vestre Raudekiølma-jaure}, og \textit{Østre Raudekiølmajaure}, mitt derover\tabcellsep Et \textit{Sund}, imellem det \textit{Store Rosto-jaure}, og \textit{Mellem Rosto-jaure}, Mitt derover\tabcellsep \textit{Kobmo-jaure} Derover, eller tæt Vesten derforbj, efter Befindende.\end{longtable} \par
 \par
paa \textit{Stokkeborres} Nordre Side, derhen gaaer Landz\textit{kiølen} over \textit{Stokkeborre;} Og heri komme alle Vidnerne overEens: mens \textit{differere} deri, at de af \textit{Reisens}Fiord meene, at det bliver paa den Vestre Deel af dette \textit{Stokkeborre}, og de af \textit{Balsfiord} slutte, at det er Mitt paa \textit{Stokkeborres} Nordre Side, hvor bemeldte Eid støder til; Hvilket ved Befaringen kan eftersees og udfindes. Herpaa følger i Nord\par
8. Et \textit{Eid (jsthmus)} imellem de 2de Vande, \textit{Vestre Raudekiølma jaure}, og \textit{Østre-Raude- kiølma jaure}, af slet Land med Maasse paa, et Par Riffelskud bredt imellem Vandene fra Vester i Øster; \textit{Vestre Raudekiølma-jaure} skal være 1/8 Miil langt, og 2 Bøsseskud bredt, hvoraf Aaen rinder i Syd-vest i den \textit{Norske}\textit{Maals} Elv ad \textit{Malangens} Fiord; det \textit{Østre Raudekiølma Jaure} er fra Vester i Øster 1 Miil langt, og 1/8 Miil bredt, deraf Aaen stikker i Øster ad \textit{Sverrig} til \textit{Korvejok;} Mitt over dette Eid gaaer nu Landz\textit{kiølen}. Derfra i Nordost, som meenes, 1 Miil er\par
9. Et \textit{Sund} imellem det \textit{Svenske Store Rosto-jaure}, og det \textit{Norske Mellem-Rosto-jaure;} Hint det \textit{Store Rosto-jaure} strekker sig til Sør 1 Miil langt, 1/4 Miil bredt; af dets Søndre Ende rinder Elven \textit{Laïmi-jok} i Sør til Østen ad \textit{SverrigsTorne-Elv}. Dette, det \textit{Mellem-Rostojaure}, skal bestaae af 2de Vande, som ved en kort Aae af 1 Bøsse-Skudz Længde henge sammen fra Øster i Vester; det Østre Vand heraf er over 1/4 Miil langt fra Øster i Vester, og en 5. BøsseSkud bredt, og det Vestre Vand er mindre, end 1/4 Miil langt, 3 Bøsse-Skud bredt, saa at begge disse Vande af det \textit{Norske MellemRostojaure} giøre 1/2 Miil, og Aaen derimellem fra det \textit{Østre}- i det \textit{Vestre} Vand rinder fra Øster i Vester; Elven, som rinder af dette \textit{Mellem-Rostojaure} ad \textit{Norge}, heder \textit{Dolgas-jok}, og løber i Vester til Sør 1 1/2 Miil lang i det \textit{Norske} Vand \textit{Nedre-Rostojaure}. Mitt over dette Grendse-Sund imellem det \textit{Svenske Store Rosto-jaure}, og det \textit{Norske Mellem-Rostojaure} gaaer Kiølen. ‒ 1/2 Miil Norden for dette \textit{Rosto-Sund} er et Vand\par
10. \textit{Kobmo-jaure}, mest rundt, 1/4 Miil stort, hvoraf \textit{Kobmo-jok} rinder i Øster ad \textit{Sverrig;} Om dette \textit{Kobmo}-Vand ere ulige \textit{Depositioner:}\par
4 og 6te Vidner i \textit{Tromsøe} vide ei af, at derfra skal rinde en Aae i Vester; Men 5te Vidne \textit{ibidem} siger, at en Bæk derfra udgaaer i Vester 2 Miile lang i det \textit{Norske Vestre Rostojaure;} Alle disse 3 Vidner komme overEens derj, at dette \textit{Kobmo-jaure} giør \textit{Kiølen:} Mens 4de Vidne kan ei sige, hvorpaalaug det eegentlig er, at \textit{Kiølen} gaaer, om mitt derover, eller paa nogen af Sidene? 5te Vidne slutter, at LandeSkifte vil være mitt derover, for Vandenes Affald til begge Sider; 6te Vidne har hørt, at \textit{Kiølen} skal gaae tæt Vesten derforbj. 7de 8de og 10de Vidner i \textit{Løngens}Fiord vidne, at dette \textit{Kobmo}-Vand ligger Østen for \textit{Kiølen}, hvorlangt? kunde de ei vide, sigende derhos, at samme Vand tilkommer \textit{Sverrig} alleene, og at ingen Bæk derfra rinder i Vester til \textit{Norge;} 21de Vidne forklarer, at det ligger Østen for \textit{Kiølen} 1/2 Miil.\par
Det siunes u-forgribelig: Ved Befaringen ville sees, om dette \textit{Kobmo}-Vand ligger fra\hypertarget{Schn1_59278}{}Schnitlers Protokoller IV.\label{Schn1_59280} \par 
\begin{longtable}{P{0.24831460674157302\textwidth}P{0.601685393258427\textwidth}}
 \hline\endfoot\hline\endlastfoot 11.\tabcellsep 12.\\
\textit{Kapo-vara}, de fleeste sige: Mitt derover\tabcellsep \textit{Teupel} der over, hvor følgende Eid imellem \textit{Kolte} og \textit{Kutkiem}-Vande støder an paa dets Nordre Side\end{longtable} \par
 \par
næste Søndre Grendse-Merke nemlig Rosto-Sund lige i \textit{Linien} til nærmeste Grendse-Merke i Nord, \textit{Kapo-vara}, eller noget meget Østen for \textit{Linien?} Jtem om fra \textit{Kobmo}-Vandet rinder en Bæk til \textit{Norge}, som til \textit{Sverrig?} Skulle dette være, kunde \textit{linien} trækkes mitt over \textit{Kobmo}- Vandet: Hvis ikke, da tæt Vesten derforbj.\par
11. \textit{Kapo-vara}, et Field, fra \textit{Rosto-Sund} 1 1/2 Miil, og fra \textit{Kobmo-jaure} 1 Miil i Nord omtrent liggendes; Landet derimellem er u-jevnt, steenet, med smaa Sillie-Riis, noget Maasse og Græss i. Dette \textit{Kapo-vara} er rundt, imod 1/4 Miil stort, oventil rundagtigt, med noget Græss og Maasse begroet; Strax under dette \textit{Kapovara} i Vester (sige en Deel) ligger \textit{Kapo-jaure}, en Kiøn, 3 Bøsse-Skud lang fra Sør i Nord, og 1 Bøsse-skud breed, hvoraf Aaen rinder mest i Nord i den \textit{Norske}\textit{Omaïs} Elv; Fra dette \textit{Kapovaras} Nordre Side skal falde en Foss, \textit{Kapo-gors}, løbendes i Nord i bemeldte \textit{Omaïs} Elv.\par
8de Vidne forklarer, at en Kiøn \textit{Masel-jaure} (adskilt fra \textit{Kapojaure}) skal være under dette \textit{Kapo-vara}, hvoraf en Bæk rinder i Vester ad den \textit{Norske}\textit{Omaïs} Elv, og en anden Bæk i Øster ad \textit{SverrigsKobmo-jok;} Men om denne \textit{Masel}-Kiøn ligger paa Vestre, Østre, eller Søndre Side af \textit{Kapovara?} vidste han ei til visse.\par
Herved \textit{Kapo-vara} skal nu \textit{Kiølen} gaae: Men hvor paalaug? deri ere Vidnerne u-eenige: 1te og 2det Vidner i \textit{Balsfiord} have sagt, at \textit{Kiølen} gaaer over det Vand \textit{Kapo-jaure}, fordi derfra skal en Aae rinde i Vester, og en anden, \textit{Kobmo-jok} i Øster. 4de Vidne i \textit{Ulfsfiord} har hørt, at Fieldet \textit{Kapo-vara} giør Grendse-Skiellet, men veed ei, hvor og paa hvad Sted dette Skiell gaaer? 7de Vidne i \textit{Løngen} siger det samme. 11te og 21de Vidner forklare, at \textit{Kiølen} gaaer midt over det Field \textit{Kapovara}, Østen forbi \textit{Masel}- og \textit{Kapo-jaure}, som heraf ingen Bæk til \textit{Sverrig}, men af begge disse Vande Aaer rinde til \textit{Norge} i \textit{Omais}-Elv. 5te og 6te Vidner udsige, at Lande-skifte er Østen forbi \textit{Kapo-vara}, imellem dette \textit{Kapo}, og en Bæk paa dets Østre Side, som skal stikke i Øster i den \textit{Svenske Kobmo-jok}.\par
Uforgribelig meenes, at her vil eftersees, om de Kiønne \textit{Masel}- og \textit{Kapo-jaure} giøre Bække alleene til \textit{Norge}? og da vil de med Grendse-\textit{linien} paa Østre Side gaaes forbj, helst 21de Vidne forklarer, at disse 2de Kiønne ligge Sønden for \textit{Kapo-vara}. Dersom Østen for \textit{Kapo-vara} en Bæk findes at stævne til \textit{Sverrig;} Saa vil vel \textit{Linien} trekkes Østen for \textit{Kapo-vara}, imellem det, og den Østre Bæk: Men skulle den Østre Bæk ej være til, vil vel fra Sør gaaes mitt over \textit{Kapo-vara}. ‒\par
Paa Kapovara i Nord følger en Field-dal med noget Græss og Maasse i, 1/4 Miil viid fra Sør i Nord; og derpaa i Nord det Field\par
12. \textit{Deupel}, eller \textit{Teupel}, fra Vester i Øster 2 Miile langt, 1/2 Miil over bredt, til deels slet, til deels knubbet, med Græss oventil og paa Sidene. Over dette \textit{Teupel} gaaer nu \textit{Kiølen} fra næste Søndre Grendse-Merke \textit{Kapo-vara} eller fra dets Østere Side, hvor det bliver, paa det\hypertarget{Schn1_59581}{}Tabell over de bevidnede Grendse-Field.\label{Schn1_59583} \par 
\begin{longtable}{P{0.3042801556420233\textwidth}P{0.2546692607003891\textwidth}P{0.1885214007782101\textwidth}P{0.10252918287937743\textwidth}}
 \hline\endfoot\hline\endlastfoot 13.\tabcellsep 14.\tabcellsep 15.\tabcellsep 16.\\
Et \textit{Mutkie} eller \textit{Eid} Jmellem \textit{Kolte-} og \textit{Kutkiem-Jaure} ‒\tabcellsep \textit{Mall}, der over, hvor til næste Grendse-Merke \textit{Gaabb} gaaes\tabcellsep \textit{Gaabb} eller \textit{Kalkogaabb}, Mitt derover\tabcellsep \textit{Jekkas} Mitt derover\end{longtable} \par
 \par
Sted, hvor det \textit{Mutkie} eller Eid imellem \textit{Kolte}- og \textit{Kutkiem}-Vande støder an paa dets Nordre Side; Og heri komme alle Vidnerne overEens: Men 4de 8de 11te og 21de Vidner i \textit{Tromsøe} meene, at det vil blive imod den Vestre Ende af Teupel, 5te Vidne sluttet, at det bliver mest over Mitten af dette \textit{Teupel}, og 6te Vidne har hørt, at det er over \textit{Teupels} Østere Deel, at \textit{Kiølen} vil komme til at gaae, i Nordost. Ved Befaringen kan dette eftersees og udfindes. 2 a 3 Bøsse-Skud Norden for \textit{Teupel}, i hvilket Mellem-Romm smaa Bierk er, møder\par
13. Det \textit{Mutkie}, eller \textit{Eid} (\textit{isthmus}) imellem de Vande \textit{Koltejaure} og \textit{Kutkiem-Jaure;} hvilket Eid er 1 Bøsse-skud bredt imellem de 2de Vande, med Græss og Bierke-Skoug paa. \textit{Kolte-Jaure} er fra øster i Vester 1/2 Miil, Andre sige, 1/4 Miil langt, 2 Bøsse-skud over bredt, hvoraf den \textit{Norske}\textit{Omaïs}-Elv rinder i LøngensFiordz Botten. ‒ \textit{Kutkiem-Jaure} er fra Vester i Øster et Par Bøsse-skud langt, halv saa bredt, hvoraf Aaen flyder 2. a 3 Bøsse-Skud i Øster i det \textit{Svenske} Vand \textit{Kilpis-jaure}, som er 2 korte Miile langt mest fra Vester i Øster, og 1/4 à 1/2 Miil bredt. Mitt over dette \textit{Mutkie} gaaer \textit{Kiølen} i Nord.\par
Norden for dette \textit{Mutkie} er Græss- og Maasse-Land, 3 à 4. Bøsse-Skud langt, og derpaa følger\par
14. det Field \textit{Mall}, fra Vester i Øster 1/2 Miil langt, omtrent ligesaa bredt; Endeel deraf er ovenpaa fladt, endeel rygged med noget lidet Græss der, og paa Sidene, hvor det er sommestedz bratt, sommestedz fladtvoren. ‒ \textit{Kiølen} gaaer fra forbemeldte \textit{Mutkie} over dette \textit{Mall}, der hvor den Houg \textit{Gaabb} eller \textit{Kalko-Gaabb} ligger ved Nordre Side af dette \textit{Mall}, i Nordost. ‒\par
Nu meene 4de 5te og 6te Vidner, at det vil blive den Deel af Mall imod dens Østere Ende; 8de Vidne slutter, at det bliver over den Vestre Ende af \textit{Mall}; 21de Vidne siger, at der er 2de \textit{Maller}, hver 1/4 Miil stort, fra Vester i Øster 1/2 Miil fra hinanden liggende; det Vestre hører ganske \textit{Norge} alleene til, det Østere gaaer \textit{Kiølen} over, som meldt. ‒ 22de 23de og 24de Vidner vill, at \textit{Kiølen} gaaer mitt over dette Mall; Ved Befaringen kan dette udfindes.\par
Jmellem \textit{Mall} og \textit{Kalko-Gaabb} er et Lægd, eller en Field-Grubbe med nogen smaa Bierk, lidet Græss og Maasse i fra Sør i Nord; derpaa følger\par
15. \textit{Gaabb, Gaam} eller \textit{Kalko-gaabb}, en Græss-groed Bakke, rundvoren, ikke meget høy, 1 Bøsse-Skud over stor, liggende i Nord fra \textit{Malls} Østere Ende 1/4 Miil, eller som andre sige 1 Miil; Paa Vestre Side deraf ligger det Vand \textit{Kalko-jaure}, fra Vester i Øster 1/2 Miil langt, 2 Bøsseskud over bredt, hvoraf Aaen rinder i den \textit{Norske}\textit{Skiebots}Elv; Paa Østere Side er den Græssdal \textit{Silles-vuøbme}, hvoraf en Bæk stikker i Sydost i det \textit{Svenske}\textit{Kilpis}-Vand. \textit{Kiølen} gaaer mitt over dette Kalko-gaabb.‒ Tæt Østen for \textit{Gaabb}, andre sige, i Nordost fra Gaabb ligger.\par
16. \textit{Jekkas}, et høyt Field, fra Vester i Øster meenes at være 1/2 Miil, andre sige, imod 1/4 Miil langt, og nogle sige 1/4 Miil, andre 2 a 3 Bøsseskud bredt, fladt og steen-uret med runde Knoller paa, Maasse-groet, paa Vestre Side meget bratt, med noget Græss og smaa Bierk paa;\hypertarget{Schn1_59859}{}Schnitlers Protokoller IV.\label{Schn1_59861} \par 
\begin{longtable}{P{0.2401129943502825\textwidth}P{0.2497175141242938\textwidth}P{0.1824858757062147\textwidth}P{0.17768361581920902\textwidth}}
 \hline\endfoot\hline\endlastfoot 17.\tabcellsep 18.\tabcellsep 19.\tabcellsep 20.\\
\textit{Masel}, Mitt derover, og langs derefter\tabcellsep \textit{Salmoive}, Mitt derover og langs derefter\tabcellsep \textit{Jjoollam-oive} Mitt derover\tabcellsep \textit{Tulliehopek}, mitt derover\end{longtable} \par
 \par
Paa Nord-ostlige Side en 4 Bøsse-skud omtrent fra \textit{Jekkas}, ligger det Vand \textit{Galla-jaure} fra Øster i Vester 1/8 Miil langt, og 2. Bøsse-Skud bredt, hvoraf Aaen rinder i Vester i det \textit{Norske Kalko-jaure;} Mitt fra Fieldet \textit{Jekkas} falder og en liden Bæk ned, og løber i dette \textit{Galla-jaure;} Og fra dets Østere Ende paa den Søndre Side udvælder en Bæk i Øster ad \textit{Sverrig}, 1/2 Miil lang rindendes i det Vand \textit{Tiermesjaure}. ‒\par
Mitt over dette \textit{Jekkas} fra Vester i Øster gaaer \textit{Kiølen}. ‒ 1/2 Miil i Ost-Nordost fra \textit{Jekkas} er det Field.\par
17. \textit{Masel;} det Mellem-Rom imellem \textit{Jekkas} og \textit{Masel} bestaaer af steenet Landskab, med Myr og noget Maasse i, uden Skoug; dette \textit{Masel} er rundt, andre sige langagtig-rundt, ikke høyt, mindre, end 1 Bøsse-skud over stort, omtrent rundagtigt ovenpaa, som en HøeStakke, med noget Maasse paa. Fra Masels Søndre Side rinder en Bæk i Sydost i det \textit{Svenske}\textit{Tiermesjaure}, som ligger fra dette \textit{Masel} en god 1/4 Miil; Fra \textit{Masels} Nordre Side falde vel smaa Bække ned ad \textit{Norge}, men, naar de ere nedkomne, tørkes bort.\par
Mitt over dette \textit{Masel}, andre \textit{exprimere} det saa: Mitt over og langs med dette \textit{Masel} gaaer \textit{Kiølen}. 5 à 6 Bøsseskud i Ost-Nord-ost (som sluttes) fra \textit{Masel} ligger det Field\par
18. \textit{Salmoive}, fra Syd-vest i Nord-ost 1/2 Miil langt, og, som meenes, halv saa bredt, høyt, paa Søndre Side bratt, med Steen-Urer paa, de andre Sider gaae fladtvoren ned, sommestedz maassede, sommestedz Steen-urede. Til \textit{Sverrig} rinde smaa Bække fra den Søndre Side i det \textit{Svenske} Vand \textit{Tiermes-jaure;} Fra den Nordvestlige Side af \textit{Salmoive} falder en Aae, som igiennem adskillige Vande løbendes, omsider giør den \textit{Norske}\textit{Skibots} Elv. \textit{Kiølen} gaaer mitt over ‒ og efter \textit{Salmoive} fra Vest Syd-vest i Ost-Nord-ost.\par
Med dette \textit{Salmoive} er i Ost-Nord-ost fieldfast, uden Dal eller Skare at være imellem, (dog siger 21de s. Vidner, at et Skare derimellem findes)\par
19. \textit{Jjoollam-oive}, rundt, lidet høyere, end \textit{Salmoive}, fladagtigt ovenpaa, 3 a 4 Bøsseskud over stort, fladt-nedstubendis, sommestedz steen-uret, sommestedz maasset; Paa den Nordvestlige Side rinder en Bæk ad \textit{Norge} i Vester i det \textit{Norske}\textit{Skioppi-Jaure;} Paa den Sydostlige Side falde smaa Bække ned ad \textit{Sverrig}, som bortkomme. ‒ Mitt over dette \textit{Jjoollamoive} gaaer Grendse-\textit{linien} i OstNordost. 1/4 Miil Norden for \textit{Jjoollam-oive} er det Field\par
20. \textit{Tullie-hopek}, rundt, noget høyt, ovenpaa rundrygget med smaa rundagtige Knoller paa, 1. Bøsse-Skud over stort, paa Nordre Side bratt og steenet, paa de andre Sider nedhældendes, sommestedz maasset, sommestedz steenet. ‒ Fra den Nordre Side falder en Bæk, og flyder i det \textit{Norske Skioppi-Jaure;} Fra den Søndre Side nedrinder en Aae, \textit{Seijem-jok}, ad \textit{Sverrig}, løbendes 1/4 Miil lang i Sør, først i den Maasse-dal \textit{Sejem-lako}, og siden videre i den \textit{Svenske}\textit{Letas}Elv. Mitt over dette \textit{Tulliehopek} gaaer \textit{Limiten} fra Sør i Nord. 1 Miil, eller 3/4 \textit{Miil} Norden for \textit{Tulliehopek} er\hypertarget{Schn1_60135}{}Tabell over de bevidnede Grendse-Field.\label{Schn1_60137} \par 
\begin{longtable}{P{0.34319248826291077\textwidth}P{0.22746478873239434\textwidth}P{0.08380281690140844\textwidth}P{0.19553990610328637\textwidth}}
 \hline\endfoot\hline\endlastfoot 21.\tabcellsep 22.\tabcellsep 23.\tabcellsep 24.\\
\textit{Kiedoive}, over dets Østere Ende, eller Mitt derover, som Vandene falde til\tabcellsep \textit{Tidno-lako}, Mitt derover Jmellem Vandfaldene.\tabcellsep \textit{Urtasoive}\tabcellsep \textit{Halde}, mitt derover, imellem 2de Aaer\end{longtable} \par
 \par
21. det Field \textit{Kiedoive}; Landet imellem \textit{Tulliehopek} og \textit{Kiedoive} er slet, maasset og steenet med smaa Urer paa. \textit{Kiedoive} er rundagtigt, ikke meget høyt, 4 Bøsse-Skud over stort, rundfladtvoren med Maasse ovenpaa, og med Steen-Urer paa Sidene. ‒ Ved dets Østre Side opkommer en Bæk, \textit{Tidno-jok}, i \textit{Tidno}-Dalen, og rinder i Syd-vest i den \textit{Norske Skibots}Elv; En deel sige, at til \textit{Sverrig} ingen Bæk fra \textit{Kiedoive} gaaer: Men 21de Vidne forklarer, at en liden Bæk derfra falder ned ad \textit{Sverrig}, løbendes til Letas Elv. ‒ \textit{Kiølen} gaaer (sige en deel) over dets Østre Ende, Østen for Bækkens Oprindelse; 21de 22de 23de og 24de Vidner vill at \textit{Kiølen} gaaer mitt over \textit{Kiedoive}. Paa dette \textit{Kiedoive} følger i Øster\par
22. \textit{Tidno-lako}, en Maasse-Dal, imod 1/4 Miil breed og lang, uden Skoug; Af denne Dal er det, at bemeldte \textit{Tidno-jok} opkommer, rindendes til \textit{Norge}; Og af samme Dal opspringer en anden Aae, \textit{Urtesjok}, som stævner ad \textit{Sverrig} i Sør til Østen i den \textit{Svenske} Elv, \textit{Valte-jok}. Bemeldte \textit{Norske Tidno-jok} i denne Dal har sin Oprindelse af en Kilde; den Svenske \textit{Urtesjok}; udløber af et Vand, \textit{Laasi-jaure}, rundagtigt, 2 Bøsse-Skud vidt; Rommet imellem \textit{Tidnojoks} Kilde, og \textit{Laasi-jaure} kan være 2 Bøsse-Skud vidt. Mitt over dette Mellem-Rom imellem besagde \textit{Tidno-joks} Kilde og \textit{Laasi-jaure} gaaer Lande-skiftet, at lige meget af Dalen, nemlig 1 Bøsse-skud til \textit{Norge}, og 1 Bøsse-skudz Vidde til Sverrig kommer. Paa denne \textit{Tidno-lako} i Øster til Nord følger\par
23. \textit{Urtasoive}, et Field, 1/2 Miil langt, og ligesaa bredt, maasset og steenet, ovenpaa fladt, med smaa ronde Knoller paa, fladtvoren paa Sidene; Fra den Vestre Side af dette \textit{Urtasoive} rinder en Aae \textit{Smolko-jok}, 1/4 Miil lang, i den \textit{Norske}\textit{Kaafiord}-Elv; Fra den Søndre Side af dette \textit{Urtasoive} nedfalder en Bæk til \textit{Sverrig}, navnlig \textit{Pidsus-jok}, 1/4 Miil i Sør i det Vand \textit{Pidsus-jaure}. ‒ Mitt over dette \textit{Urtasoive} gaaer Grendse-Skiellet; 22de 23de og 24de Vidner sige det samme; dog at det vil blive noget hen paa dets Vestre Deel.\par
Herpaa følger en Dal, 1/4 Miil viid, med noget Græss, Maasse og Steen i, og derpaa i Øster\par
24. \textit{Halde}, nemlig det Søndre, et rundt høyt Field, 3/4 Miil stort, ovenpaa deels fladt, deels knollet, bart og snaut; Paa Østre Side bratt, paa de andre Sider noget fladtvoren, steenet over alt; Fra dets Nordre Side rinder en Bæk \textit{Kiaaleme-jok} i \textit{Reisens} Elv ad \textit{Norge}; Paa Søndre Side af dette \textit{Halde} udkommer af Jorden en Aae, \textit{Hornaks-jok}, rindendes i Sør 1/4 Miil lang i det \textit{Svenske} Vand \textit{Sommasjaure}. Mitt over dette \textit{Halde} gaaer Grendse-Skiellet i Øster; 22de 23de og 24de Vidner forklare, at det gaaer mitt imellem \textit{Kiaalemejok}, og \textit{Hornaksjok} over dette \textit{Halde}; ‒ dog vil dette \textit{Søndre Halde}-Field med det \textit{Nordre Halde}, som ligger 1 1/2 Miil fra hint i Nordost ved det \textit{Norske Kutkiejaure}, eller Vand, ei \textit{confunderes}. ‒ Paa dette \textit{Søndre Halde} Grendse-Field følger i Øster til Norden\hypertarget{Schn1_60420}{}Schnitlers Protokoller IV.\label{Schn1_60422} \par 
\begin{longtable}{P{0.17210743801652892\textwidth}P{0.11942148760330579\textwidth}P{0.11942148760330579\textwidth}P{0.12644628099173555\textwidth}P{0.18615702479338841\textwidth}P{0.12644628099173555\textwidth}}
 \hline\endfoot\hline\endlastfoot 25.\tabcellsep 26.\tabcellsep 27.\tabcellsep 28.\tabcellsep 29.\tabcellsep {}[30.]\\
\textit{Nappetiøve}, mitt over en Houg i Dalen\tabcellsep \textit{Samasoive} Mitt derover\tabcellsep \textit{Vardoive}, Mitt derover\tabcellsep \textit{Siertevare}, Mitt derover\tabcellsep \textit{TenoMotkie}, Mitt derover imellem 2. Vande\tabcellsep \textit{Korsevare} Mitt derover\end{longtable} \par
 \par
25. \textit{Nappetiøve}, en Dal med en liden Houg i, liggendes fra Vester i Øster mitt i Dalen; Dalen er græssed og steened, 3 a 4 Bøsse-Skud viid; J denne Dal paa den Nordre Side af Hougen er det, at \textit{Kiaaleme-jok} opkommer, og rinder tæt under \textit{Halde} hen ad den \textit{Norske}\textit{Reisens} Elv; Paa dens Søndre Side udspringer en Bæk, som flyder i det \textit{Svenske} Vand \textit{Sommasjaure}. Forommeldte Houg eller Bakke, 1 Bøsse-Skud lang og breed, med Maasse paa, heder eegentlig \textit{Nappetiøve}, og er det over denne Houg, at Raa-Gangen er i Øster. ‒\par
Et maasset og noget slet Land følger paa denne \textit{Napetiøve}, 1/4 Miil langt; derpaa i Syd-Ost fra \textit{Nappetiøve} følger det Field\par
26. \textit{Samasoive}, rundt, 1/4 Miil over stort, ikke meget høyt, slet og noget rundvoren ovenpaa, med Maasse og Steen paa, ligesom og paa Sidene fladagtigt, maasset og steenet. ‒ Paa Nordre Side under \textit{Samasoive} udkommer en Aae \textit{Harveskiolme}, rindendes i den \textit{Norske}\textit{Reisens} Elv, hvorlang? vidstes ei. Paa Søndre Side udspringer af Jorden under \textit{Samasoive} en Aae, som flyder i Sydvest i det \textit{Svenske Sommasjaure}. ‒ Mitt over dette Samasoive gaaer \textit{Kiølen}. ‒\par
Landskabet imellem dette \textit{Samasoive} og følgende Grendse-Merke \textit{Vardoive} er slet, maasset, og steenet, uden Skoug, omtrent 1/2 Miil vidt i Syd-ost; der møder da\par
27. \textit{Vardoive}, et Field, fra Nordvest i Syd-ost 1/4 Miil langt, 1 Bøsse-Skud over bredt, noget slet ovenpaa, med Maasse og Skoug begroet, paa Sidene fladtvoren, maasset og steenet. Fra Nordre Side af \textit{Vardoive} neden derunder rinder en Aae, 1 Miil lang i den \textit{Norske}\textit{Reisens} Elv; Fra Søndre Side af \textit{Vardoive} stikker en Aae \textit{Sidos-jok} i Syd-vest til den \textit{Svenske} Elv \textit{Valtejok. Kiølen} gaaer mitt over \textit{Vardoive}.\par
Fra dette \textit{Vardoive} til næstfølgende Grendse-Merke \textit{Siertevare} er 1 Miil i Sør til Øster; Landskabet herimellem er dalet, maasset og myret uden Skoug;\par
28. \textit{Siertevare} er et rundt Field, 2 Bøsse-Skud over stort, høyt og spidz opad, som en Høe Stakke, steenet og maasset, paa Sidene fladtvoren; Paa dets Nordre Side kommer en Aae (Navnet ej vidstes) neden under Fieldet, og rinder i \textit{Reisens} Elv, som han slutter, 1 Miil lang; Fra Søndre Side under Fieldet stævner en Aae ad \textit{Sverrig}, 1/4 Miil lang i \textit{Letas-Ene}. ‒ \textit{Kiølen} gaaer mitt over \textit{Siertevare}. ‒ En god 1/4 Miil i Sør fra \textit{Siertevare} er\par
29. \textit{Teno-Motkie} en Maasse-Dal med smaa Bierk i, 1/4 Miil lang i Sør, 1 Bøsse-Skud breed; J denne Dal opkommer en Aae \textit{Nallauksjok}, som rinder 1 Miil lang i den \textit{Norske}\textit{Reisens} Elv: Derimod gaaer en anden Aae (uden Navn) af samme Dal ad \textit{Sverrig}, 2 Bøsse-Skud lang i den \textit{Svenske LetasEne}. Det Rom imellem disse 2de Aaers Oprindelse er 1 Bøsse-Skud vidt, og Mitt over dette Rom imellem begge Aaerne gaaer Grendse-Skiellet. 3/4 Miil fra dette \textit{Teno-Motkie} i Syd-ost er\par
30. \textit{Korse-vare}; Landet af den 3/4 Miils Vidde derimellem er dalet, slet og maasset, med smaa Bierk i. \textit{Korsevare} er et rundt Field, 1/4 Miil over stort, ikke ret høyt, ovenpaa fladt,\hypertarget{Schn1_60688}{}Tabell over de bevidnede Grendse-Field.\label{Schn1_60690} \par 
\begin{longtable}{P{0.18140243902439024\textwidth}P{0.19695121951219513\textwidth}P{0.18140243902439024\textwidth}P{0.18140243902439024\textwidth}P{0.10884146341463415\textwidth}}
 \hline\endfoot\hline\endlastfoot 31.\tabcellsep 32.\tabcellsep 33.\tabcellsep 34.\tabcellsep 35.\\
\textit{Aakie-vara} Mitt derover\tabcellsep \textit{Kaakas-motkie} Mitt derover\tabcellsep \textit{Sapasmaras} Mitt derover\tabcellsep \textit{Nerrevarda} Mitt derover\tabcellsep \textit{Supsavara}\end{longtable} \par
 \par
med nogen Bierk paa, paa Sidene fladtvoren, steenet og maasset. Fra den Nordre Side ved Fieldet af Jorden udspringer en Aae \textit{Muretz-Jok}, som løber mest i Øster i den \textit{Finmarkiske}\textit{Altens}-Elv, som her kaldes \textit{Rikasjok;} Fra den Søndre Side af Jorden rinder en Bæk, 1 BøsseSkud lang, i Syd-Vest ad \textit{Sverrig} i \textit{Letas-Ene}. ‒ Mitt over dette \textit{Korse-vare} er Lande-Skifte. 1/4 Miil i Sør til Østen fra \textit{Korsevare}, hvilket Støkke er Myrland, med nogen Bierk og Maasse i, ligger\par
31. \textit{Aakie-vare}, et fladt Field med Bierk paa, ikke meget høyt, 1/4 Miil langt i Sør til Øster, 1/8 Miil bredt, paa Sidene fladtvoren, med Bierk og Maasse paa; 10de og fleere Vidner vidste ikke, at der faldt noget Vand fra nogen af Sidene: Men 22de Vidne sagde, at fra dette \textit{Aakie} rinde i Nord smaa Bække i den \textit{Norske Muretz-jok}, og fra samme Field smaa Bække i Sør i den \textit{Svenske LetasEne}. ‒ Mitt over dette \textit{AakieVare} gaaer Grendse-\textit{Linien}.\par
1 Miil fra Aakie i Sør til Øster er \textit{Kaakas-Motkie}, hvilket Landskab bestaaer af Maasse med Bierk i.\par
32. \textit{Kaakas-Motkie} er en Maasse-Dal med Bakker i paa begge Sider; Dalen i sig selv er omtrent 2 Bøsse-Skud lang i Sør til Øster, og ligesaa breed, med Maasse og Bierk i; dens Bakker paa begge Sider ligge ligeledes i Sør til Øster, 2 Bøsse-Skud lange og brede, med Maasse og Bierk begroed; Østen for den Østere Bakke ligger et Vand \textit{Kaaska-jaure}, 1/4 Mill i Syd-ost langt, 2 Bøsse-Skud bredt, hvoraf Aaen \textit{Hauga-jok} rinder ad \textit{Norge} i \textit{Finmarkens}\textit{Altens} Elv; Paa Vestre Side af Vestre Bakke løber den \textit{Svenske}\textit{Letas}Elv tæt der forbi, henad \textit{Enotekies Lappe}- Kirke. Mitt over dette \textit{Kaakas Motkie}, eller Eid gaaer Grendse-Skiellet. ‒ 1/8 Mil i Sør fra \textit{Kaakas Motkie} er\par
33. \textit{Sapasmaras}, en slet Mark imellem 2de Myrer, 3 a 4. Bøsse-Skud lang, og breed, med Maasse og Bierk i; Af den Østre Myr omtrent 2 Bøsse-skud fra \textit{Sapasmaras} opkommer en Bæk, og rinder i Nord, omsider i \textit{Finmarkens}\textit{Altens} Elv; Fra den Vestre Myr, omtrent 2 BøsseSkud Vesten fra \textit{Sapasmaras}, gaaer en Bæk ad \textit{Sverrig} i \textit{Letas-ene}, 2 Bøsse-Skud lang. ‒ Lands-\textit{Kiølen} gaaer Mitt over dette \textit{Sapasmaras}. 1/2 Miil i Øster til Syden fra \textit{Sapasmaras} er\par
34. \textit{Nerrevarda}, et Field, ikke høyt, langt fra Sør i Nord 1/2 Miil, og 1/8 Miil bredt, ovenpaa fladt og maasset, paa Sidene fladtvoren, med Maasse og nogen Bierk paa. Fra den Østre Side af dette \textit{Nerrevarda} af Jorden kommer en Aae \textit{Balkisjok}, og rinder ad Norge i \textit{Finmarkens}\textit{Altens} Elv, som her kaldes \textit{Rikas-jok;} Fra \textit{Nerrevardas} Søndre Side af Jorden opstiger en Aae og vender sig i Sør, forbi \textit{Enotekies} Kirke i \textit{Enotekies} Elv, Sønden for Kirken. ‒ Mitt over dette \textit{Nerrevarda} gaaer Landeskifte. 1/4 Miil i Sydost fra \textit{Nerrevarda} er\par
35. \textit{Supsavara}, eller \textit{Sudsavara}, en liden laug Bakke, rund, med Bierk og Maasse paa, 2 Bøsse-skud over stor; Paa den Nordre Side af Jorden rinder en Bæk uden Navn i \textit{Finmarkens}\textit{Altens} Elv, som her heder \textit{Rikas-jok;} Fra den Søndre Side af Jorden gaaer en anden Aae \hypertarget{Schn1_60974}{}\footnote{\label{Schn1_60974}Dette blir dumt pga. hvordan vi takler tabellene}\textit{Busek-}\hypertarget{Schn1_60980}{}Schnitlers Protokoller IV.\label{Schn1_60982} \par 
\begin{longtable}{P{0.1171474358974359\textwidth}P{0.11442307692307692\textwidth}P{0.20705128205128204\textwidth}P{0.0953525641025641\textwidth}P{0.09807692307692308\textwidth}P{0.21794871794871792\textwidth}}
 \hline\endfoot\hline\endlastfoot 36.\tabcellsep 37.\tabcellsep 38.\tabcellsep 39.\tabcellsep 40.\tabcellsep 41.\\
\textit{Posa-vara} over dets Nordre Ende\tabcellsep \textit{Urdevara} over dets Søndre Ende\tabcellsep En \textit{Bakke}, imellem \textit{Maseljaure}, og en \textit{Svensk} Bæk\tabcellsep \textit{Skiervoive} Mitt derover\tabcellsep \textit{Pitsekiolme} Mitt derover\tabcellsep \centerline{\textit{Kieldevadda} Mitt efter det, og over \textit{Kirrajaure}}\end{longtable} \par
 \par
\hypertarget{Schn1_61041}{}\footnote{\label{Schn1_61041}Se nederst på forrige side}\textit{jok} i Sør i den \textit{Svenske}\textit{Enotekies} Elv Sønden for Kirken \textit{Enotekies}. ‒ \textit{Kiølen} gaaer mitt derover. ‒ 1/4 Miil i Sør fra \textit{Supsavara} er\par
36. \textit{Posa-vara}, et Field, langt fra Sør i Nord 1/4 Miil, halv saa bredt, ovenpaa sletvoren, maasset og steenet, paa Sidene fladtvoren med Maasse og Bierk paa. Vande gaae ei herfra nogenstedz hen. ‒ \textit{Kiølen} gaaer her over \textit{Posavaras} Nordre Ende. 1/4 Miil i Øster fra \textit{Posavara} er\par
37. \textit{Urdevara}, et Field, noget høyt, fra Syd-vest i Nordost 1/2 Miil langt, halv saa bredt, ovenpaa slet og maasset, paa Sidene fladtvoren med Maasse paa. \textit{Kiølen} gaaer over Søndre Ende af dette \textit{Urdevara} (1) fordi der er det høyest, og (2) af den Aarsag, som følger:\par
1/4 Miil i Øster til Syden fra \textit{Urdevara} ligger det Field \textit{Skiervoive}, og imellem disse 2. Fielde er et lidet Vand, \textit{Masel-jaure}, 1/2 Bøsse-Skud langt fra Sør i Nord, og 1 Steenkast over bredt; Af dette \textit{Maseljaure} løber en Bæk ad \textit{Norge} i \textit{Altens-jok}, eller her kalded \textit{Rikas-jok;} 1 BøsseSkud Sønden for dette \textit{Masel-jaure} kommer af Jorden en Bæk, og rinder ad \textit{Sverrig} i \textit{Enotekies}- Elv, Sønden for Kirken; Jmellem dette \textit{Maseljaure} og den \textit{Svenske} Bækkes Kilde er en liden Bakke, 1 Bøsseskud over stor, maassed med Bierk paa; Mitt paa\par
38. Denne \textit{Bakke} imellem \textit{Masel-jaure}, og den \textit{Svenske} Bækkes Kilde er det, at LandeSkifte er. ‒ 2 Bøsse-Skud fra denne Grendse-Bakke i Øster til Syden er\par
39. \textit{Skiervoive}, et rundt Field, ikke meget høyt, 1 Bøsse-Skud over stort, oventil og paa Sidene fladtvoren, og maasset. \textit{Kiølen} gaaer mitt derover i Øster til Syden. ‒ 1/8 Miil fra \textit{Skiervoive} i Syd-ost er det Field\par
40. \textit{Pitse-kiolme}, ikke høyt, fra Nord i Sør 1/8 Miil langt, 2 a 3 Bøsseskud bredt, ovenpaa rundvoren, maasset, paa Sidene fladagtigt og maasset. ‒ Paa Nordre Side af dette \textit{Pitsekiolme} af Jorden opkommer en Aae \textit{Pitse-jok}, og tager sit Løb i Nord (som Eet Vidne sagde) igiennem \textit{Vutze-jaure}, et Vand, i den \textit{Norske Finmarkens} Elv, \textit{Altens-jok;} Paa Søndre Side af Jorden opstiger en Aae, \textit{Rubbesolle-jok}, og gaaer Sønden for Enotekies Kirke J \textit{Enotekies}-Elv; Landz- \textit{kiølen} gaaer mitt over dette \textit{Pitsekiolme}.\par
Her er at agte, at den \textit{Svenske} Field-Øvrighed (saaledes de \textit{Svenske} Betientere kaldes, hvilke fare om Vinteren langs efter Fieldene fra en \textit{Lappe}-Bye, til den anden, at oppebære Skatten, og pleje Retten) køre over dette \textit{Pitsekiolme} fra \textit{Enotekies} til \textit{Kotokeino-Lappe}-Kirke, og er dette \textit{Pitsekiolme} omtrent halvvejs derimellem.\par
1/2 Miil Østen for \textit{Pitsekiolme} er\par
41. \textit{Kieldevadda}, et slet Field, 1 Miil langt fra Vester i Øster, og halv saa bredt, med lidet Græss, men mest med Maasse paa; Ovenpaa dette \textit{Kielde-vadda} ligger et Vand \textit{Kirrajaure}, 6. Riffelskud langt fra Vester i Øster, og 1 Riffelskud bredt, hvori Rør-Fisk er. Deraf rinder Aaen \textit{Kirra-jok} i den \textit{Norske Finmarkens}\textit{Altens} Elv, 2 1/2 Miil, andre sige 1 Miil lang;\hypertarget{Schn1_61291}{}Tabell over de bevidnede Grendse-Field.\label{Schn1_61293} \par 
\begin{longtable}{P{0.14007490636704117\textwidth}P{0.15599250936329587\textwidth}P{0.10823970037453183\textwidth}P{0.11142322097378275\textwidth}P{0.2196629213483146\textwidth}P{0.1146067415730337\textwidth}}
 \hline\endfoot\hline\endlastfoot 42.\tabcellsep 43.\tabcellsep 44.\tabcellsep 45.\tabcellsep  46.\tabcellsep 47.\\
\textit{Salvasvadda} Mitt imellem 2 Vande\tabcellsep \textit{Kierresvara}, derover imellem 2 Vande.\tabcellsep \textit{Seurisvara} Mitt derpaa\tabcellsep \textit{Tirmesvara} Mitt derover\tabcellsep \textit{Bevresmutkie}, (et Eid) Mitt derover imellem de 2de Vande\tabcellsep \textit{Kalko-vadda} Mitt derover\end{longtable} \par
 \par
Af samme \textit{Kirra-jaure} falder en Aae med samme Navn til \textit{Sverrig}. Mitt efter \textit{Kieldevadda} og mitt over \textit{Kirra-jaure} gaaer Grendse-Skiellet.\par
42. \textit{Salvasvadda} begynder strax Østen for \textit{Kieldevadda}, et slet maasset Land, 1 Miil langt fra Vester i Øster, 1/2 Miil bredt. J den Vestre Ende av dette \textit{Salvasvadda} er en Myr \textit{Sierga-egge}, deraf opspringer en Bæk \textit{Kurrisjok}, og rinder i \textit{Altens} Elv, som meenes, 3 Miile lang; En anden Aae, hvis Navn ei vides, oprinder ikke langt fra den \textit{Norske}\textit{Kurrisjok} i samme Myr, ad \textit{Sverrig}. ‒ Grendse-Skiellet bliver i dette \textit{Salvasvadda} mitt imellem de 2de Vande, der stikke til \textit{Norge} og \textit{Sverrig}.\par
43. \textit{Kierresvara} ligger strax Østen for \textit{Salvasvadda}, et Field, ikke høyt, mest rundt, 3/16 Miil stort, fladt oven til med Maasse og nogen smaa Bierk paa; Paa Nordre Side af en Myr opkommer en Aae \textit{Tulle-jok}, løbendes 2 Miile lang i den \textit{Norske Finmarkens}\textit{Altens} Elv; Fra Østere Side af \textit{Kierresvara} udgaaer af en Myr en Aae, uden Navn, ad \textit{Sverrig}. ‒ Dette \textit{Kierresvara} har nu vel ikke Navn af \textit{Kiøl} eller \textit{Tondern}, men ligger dog mitt imellem næstforrige \textit{Salvas-vadda} og næstfølgende \textit{Seurisvara}, hvilke giøre \textit{Kiølen}, og derfor vil \textit{Linien} gaae over dette \textit{Kierresvara}, mitt imellem de 2de Vande, som vende sig til \textit{Norge} og \textit{Sverrig}.\par
44. \textit{Seurisvara} ligger 1/2 Miil Østen for \textit{Kierresvara;} dette \textit{Seurisvara} er ikke høyt, 1 knap Miil langt fra Sydvest i Nordost, halv saa bredt, ovenpaa fladt og maasset, men neden omkring Bierke-groet; Paa Nordre Side deels af Fieldet, deels af Myr derunder rinde smaa Bække ad \textit{Norge} i \textit{Altens}Elv; Paa Søndre Side komme deels af Fieldet, deels af Myren smaa Bække, som stævne til \textit{Sverrig}. Over dette \textit{Seurisvara}, mitt derpaa, hvorfra Bække nedfalde til begge Sider, gaaer Grendse-\textit{Linien}. 1/2 Miil Østen for \textit{Seurisvara} er\par
45. \textit{Tirmesvara}, et rundt Field, noget høyt, 2 Bøsseskud over stort, ovenpaa rundfladt, med Maasse bevoxen, men neden under er nogen smaa Bierk. Paa Nordre Side nær Fieldet komme smaa Bække af Jorden, løbende ad \textit{Norge} i \textit{Altens} Elv; Om derfra noget Vand til \textit{Sverrig} gaaer? vidstes ei. \textit{Kiølen} gaaer mitt over \textit{Tirmesvara}. 1/2 Miil Østen for \textit{Tirmesvara} møder\par
46. \textit{Bevresmutkie}, et slet Eid, med Maasse og smaa Bierk bevoxen, 3 a 4. Bøsse-Skud bredt, imellem 2 Vande; det Nordre heder \textit{Bajasjaure}, rundt 3/4 Miil over vidt, havendes Sig- Fisk; Heraf rinder Aaen \textit{Setza-jok}, 4 Miile lang i Nord-vest i den \textit{Norske}\textit{Altens} Elv; det Søndre Vand kaldes \textit{Bevresjaure}, fra Vester i Øster 1 Miil langt, halv saa bredt, har samme Slags Fisk; deraf udgaaer \textit{Bevresjok} til \textit{Sverrig}. Mitt over dette \textit{Mutkie}, eller Eid gaaer Landzskiellet. 1 sterk Miil i Nordost fra \textit{Bevresmutkie} er\par
47. \textit{Kalkovadda}, et slet maasset Land, 1/4 Miil langt og bredt; Mitt over dette \textit{Kalkovadda} giøres \textit{Kiølen} i Øster, saa at halvdeelen til \textit{Norge}, og halvdeelen til \textit{Sverrig} hører. J Nord-ost fra \textit{Kalkovadda} forekommer det Field\hypertarget{Schn1_61620}{}Schnitlers Protokoller IV.\label{Schn1_61622} \par 
\begin{longtable}{P{0.85\textwidth}}
 \hline\endfoot\hline\endlastfoot 48.\\
\textit{Paresoive}, mitt derover, tæt Sønden for Landskift-Vattenet\end{longtable} \par
 \par
48. \textit{Paresoive}, noget høyt, rundt, en 4. Bøsseskud stort, fladt ovenpaa, og maasset, paa Søndre Side med Bierk neden under. Om fra \textit{Paresoives} Søndre Side noget Vand gaaer ad \textit{Sverrig?} vidste man ikke; Paa dets Vestre Side er et lidet Vand, som er rundt, kaldet \textit{Landskift-vatten}, 1 Bøsse-Skud vidt; deraf kommer \textit{Karasjok}, rindendes i Syd-ost til den (nu \textit{Svenske})\hspace{1em}\par
\textit{Lappe}bye \textit{Afjevara}{8 Miile}\par
Derfra i Sydost til den (nu \textit{Svenske}[)] \textit{Lappe}Bye \textit{Karasjoki}, {8 ‒}\par
Siden herfra i Sydost til den (nu \textit{Svenske}) Lappebye \textit{Teno}{9 ‒}\par
Ved denne \textit{Teno-Lappe}bye foreener denne Elv sig med en Anden, fra \textit{Enarr}, eller \textit{Enarra}bye kommendes, og antager det Navn \textit{Teno-} paa \textit{Svensk} og \textit{Lappisk}, men paa \textit{Norsk}, \textit{Tana}-Elv; Fra \textit{Teno} fortgaaer denne \textit{Tana}-Elv til den (nu \textit{Svenske}) \textit{Lappe}-Kirke \textit{Otjojockj}, paa \textit{Norsk, Arisbye}, i Sydost {7 ‒ _______} Herfra den løber ud i den \textit{Norske Tanafiord} i \textit{Finmarken}{32 Miile}\hspace{1em}\par
\textit{Kiølen} gaaer tæt Sønden for \textit{Landskift-Vatten} mitt over \textit{Paresoive}.\hspace{1em}\par
\textbf{\centerline{Alphabetisk Register}}\par
over Navne af Bøyder, Gaarder, Fiorder, Elve, Vande, Fielde og Grendse-Steder, \textit{item} Øer og Holmer, J dette 4{de}\textit{Volumen} anførte, hvorved Tallet bemerker \textit{Paginam} [utelates].
\DivI[5. volumen: 1744-1745 (I 207-415) og 6. volumen: 1744-1745 (I 416-435).]{5. volumen: 1744-1745 (I 207-415) og 6. volumen: 1744-1745 (I 416-435).}\label{Schn1_61816}\par
\centerline{\textbf{[WIDNERS EXAMINATION OVER GRENDSERNE IMELLEM NORGE NORDENFIELDS OG SVERRIGE A{o}1744‒1745. 5{te} VOLUMEN. TROMSØENS FOGDERIE UDI NORDLAND, OG FINMARKEN.]}}\par
Efter at \textit{Majoren}\textbf{Ao}\textit{1744}. \textbf{den \textit{2} Martj}, nordest i \textit{Nordland} ved \textit{Qvænangens}fiord havde \textit{expederet} 4de \textit{Volumen} af hans \textit{Examinations Protocoll} over \textit{Senniens} og \textit{Tromsøens} fogderier, og\par
d. 6 \textit{Martij} næstefter forfattet en \textit{Deduction} over de \textit{Svenske} Field Bøyder \textit{Jrne} og \textit{Zerne}, \textit{in Originali} med Posten at sendes til Hr Obriste \textit{Mangelsen}, som den i Hr Obriste \textit{Rømelings} Sted allernaadigst beskikkede Grendse-\textit{Commissaire} Begav han sig
\DivI[Mars 7.-11. Fra Straumfjord til Talvik]{Mars 7.-11. Fra Straumfjord til Talvik}\label{Schn1_61928}\par
d. 7. \textit{Martij} paa Reisen fra \textit{Strøms}-Jndfiord i Nord til Osten til \textit{Meiland}{1 Miil}\par
Siden i Øster igiennem \textit{Qvænangens}fiord mellem de Øer \textit{Spildern} og \textit{Skorpøe} til det \textit{Nordland}ske \textit{Alt}-Eidet{2 ‒}\par
d. 8de \textit{Martij} var hellig\label{Schn1_61982} \par 
\begin{longtable}{P{0.8334415584415584\textwidth}P{0.008279220779220779\textwidth}P{0.008279220779220779\textwidth}}
 \hline\endfoot\hline\endlastfoot d. 9de næst efter gaaet over besagde \textit{Alt}-Eid i Øster til Norden, til den runde Berg-Houg \textit{Sokammer}, som skiller Nordlands Amt fra \textit{Finmarken},\tabcellsep 1/2\tabcellsep ‒\\
Fra hvilken \textit{Sokammer}-Houg paa Vestre Side af Myr udrinder en Bæk, navnlig \textit{Sokammer}-Elv, i Vester over \textit{Alt} Eidet i den \textit{Nordlandske}\textit{Alte}fiord; og paa Østre Side af \textit{Sokammer} nedkommer en Bæk, navnlig Ryss-Elv fra Fielde, og Løber mest i Øster i den \textit{Finmarke}ske \textit{Langfiord}, ved dennes Botten; Fra denne \textit{Sokammers} Østre Side til bemeldte \textit{Langfiord} er imod\tabcellsep 1/4\tabcellsep 3/4 ‒\end{longtable} \par
 \par
Og som intet Huus var paa Nordre Side af \textit{Alt-Eidet}, fôer man strax i Baad igiennem \textit{Langfiorden} i Ost-Nord-ost til \textit{Langnæss} det Søndre Næss af \textit{Langfiorden}{2 ‒}\par
\textit{Mart}: 10de Ligget for Uveir stille\par
d. 11te Fra \textit{Langnæsset} i Sønden til \textit{Talviig} i \textit{Altens} Fiord i \textit{Finmarken}{3/4 ‒} hvor med Provsten \textit{Falk} havde at \textit{conferere} om \textit{Altens Finner}, og at begiere af Amtmand \textit{Kieldsen} over \textit{Finmarken} de behørige \textit{Ordres} til LaugRettens Samling og Field\textit{finnernes} Nedkaldelse paa et beqvem Sted i \textit{Altens}fiordz Botten.
\DivI[I Finnmark fogderi: 54 vidner.]{I Finnmark fogderi: 54 vidner.}\label{Schn1_62148}
\DivII[Mars 12.-19. Innsamling av oplysninger]{Mars 12.-19. Innsamling av oplysninger}\label{Schn1_62149}\par
d. 12te og følgende Dage imidlertid indtaget af kyndige Mænd den behøvende Kundskab om Landets Leje Vesten-fra;\par
Saasnart man kommer fra \textit{Segelviig}, den Nordreste og Vestreste Gaard i \textit{Nordland} i \textit{Tromsøens} Fogderie, møder \textit{Andsnæss}, det Sydligste Næss af \textit{Finmarkeen} dets faste Land;\par
\textit{Andsnæss}-Field og Gaard, som er den 1te i \textit{Finmarken}, er beskreven i næstforrige 4de \textit{Volumen}II 411.\hypertarget{Schn1_62198}{}Schnitlers Protokoller V.\par
Næsset af dette \textit{Andsnæss} i sig selv er lavt, og Græssgroet, stikkendes i Nord 1 got BøsseSkud langt, 1 à 2 Bøsse-Skud over bredt.\par
Fra \textit{Andsnæss} til Vestre Næss af \textit{Frakfiord} er 1/4 Miil i Nord-ost; Dette Mellem-Rom er berget, ovenpaa græsset, men paa Nordre Side steilt og bart;\par
\textit{Frakfiord} er i Gabet 1/4 Miil viid fra Vester i Øster; fra Østre \textit{Frakfiord}-Næss til Vestre \textit{Bers}fiordz Næss er 1/4 Miil i Nord-ost; Landet derimellem er berget og græsset;\par
\textit{Bersfiord} er udi Gabet 1 Miil viid udj Øster. Foromtalte Fielde og Berge fra \textit{Andsnæss} til denne \textit{Bergs}fiord hænge i Sør sammen med øde Sneedekkede u-benævnlige Fielde og Berge hen til de Nordlandske Fielde ad \textit{Qvænangens}fiord.\par
Det Østre Næss af forbemeldte \textit{Bersfiord}, \textit{Løverdags}-Næss, er ikke ret høyt, fra Vester i Øster 1/16 Miil langt, og ligesaa bredt, saa det er rundvoren; J Sør hænger sammen med vilde u-bekiendte Fielde hen ad den \textit{Finmark}ske \textit{Lang}fiord; Dette \textit{Løverdagsnæss} er ovenpaa rundrygget, og bart; j Vester til \textit{Bersfiord} brat og snaut; Paa Østre Side oventil brat, men nedentil fladtvoren, med Bierk og Græss paa; Paa Nordre Kanten ad \textit{Altens}fiord er den Vestre Deel steil, men den Østre fladvoren;\par
Dette \textit{Løverdags}Næss giør det Vestre Næss af næste \textit{Ullfiord}.\par
\textit{Ullfiord} er i Gabet nogle Riffel-Skud viid. Det Østre Næss af denne \textit{Ullfiord} heder \textit{Nuenæss}, en rund field Klub, ikke høy, 1/8 Miil over stor, ovenpaa bar, paa Sidene Bierke- og Græssgroed, i Sør hænger sammen med vilde Sneede Fielde, hen ad den \textit{Finmark}iske \textit{Lang}- fiord; Dette \textit{Nuenæs} giør det Vestre Næss af \textit{Nuss}-Fiorden.\par
\textit{Nussfiord} er i Gabet 1/8 Miil. Det Østre Næss af denne \textit{Nuss}fiord heder \textit{Klubbenæss}.\par
Dette \textit{Klubbenæss} er noget høyt, med 2 a 3. snaue Tinder paa fra Nord i Sør, langvoren efter fiorden fra Nord i Sør imod 1/2 Miil, fra Vester i Øster en god 1/4 Miil, hen til \textit{Oxfiord}; Oventil bart med Skarer j; paa begge, nemlig Vestre og Østre Sider brat og snaut, paa Nordre Side til \textit{Altens}fiord hælder ned, men noget op ad brat. Dette \textit{Klubbenæss} giør det Vestre Næss af \textit{Oxfiord}\par
\textit{Oxfiord} er i Munden 1/4 Miil breed.\par
Det Østre Næss af \textit{Oxfiord} er et Lidet fladt Næss med smaa Bierk paa, navnlig \textit{Ystnæss}, strekkende sig fra Vester i Øster 1/2 Miil langt hen til \textit{Laakkerfiord}, og ligesaa bredt i Sør, hvor det med andre vilde Sneetækkede Fielde hænger sammen hen til \textit{Oxfiordens} Søndre Deel; Ovenpaa fladtvoren og bart, paa Vestre- Østre- og Nordre Sider brat, med smaa Bierk neden under sommestedz;\par
Dette samme Næss giør det Vestre Næss af Vestre \textit{Laakker}fiord. ‒\par
Og som 2de \textit{Laakker}-Fiorder ere til, liggende efter hinanden fra Vester i Øster, og fra hinanden omtrent 4 Bøsse-Skud, saa heder det Field derimellem\par
\textit{Laakkerfiord-Tinden}, 4 Bøsse-Skud fra Vester i Øster lang, fra Nord i Sør et par BøsseSkud, som Fiordene ere dybe, breed: dog med et Skare i; Paa hver Deel af Næsset staaer en Tinde op, efter hinanden fra Nord i Sør.\par
Det Østre Næss af den Østre \textit{Laakker}fiord heder \textit{Øster-Laakkernæss}, fra Vester i Øster hen til \textit{Altens-Klubnæss} 1 Miil lang, fra Nord i Sør hænger sammen med ubekiendte Snee-lagde Fielde hen til \textit{Lang}fiord; dette Østre \textit{Laakkernæss} bestaaer af høye Fielde, oventil og paa Sidene bart uden Skoug, med smaa rundagtige Tinder paa.\hypertarget{Schn1_62443}{}J Altens Fiord i VestFinmarken.\par
\textit{Klubnæss}, det Nordre Næss af \textit{Langfiord} hænger i Vester sammen med forbemeldte \textit{Østre Laakkernæss} Field, og strekker sig efter \textit{Langfiord} i Vest-Syd-vest 3/4 Miil lang, hen til \textit{Kaaven}Elv, noget høyt, ovenpaa rundagtigt og bart, paa Søndre Side ad \textit{Langfiorden}, og paa Nordre Side ad \textit{Stiernsund} steilt og Snaut;\par
Dette \textit{Langfiords} nordre Næss \textit{Klubnæss} er ei at \textit{confundere} med før omtalte \textit{Klubbenæss}, det Vestre Næss af \textit{Oxfiord}, see \textit{pag.} 208.\par
Fra bemeldte \textit{Klubnæss}, naar man farer i Vest-Syd-Vest over \textit{Kaaven}-Elv, forekommer\par
\textit{Kaav-Tinden}, strekkende sig fra \textit{Kaaven}-Elv i Vest-Syd-vest 1/2 Miil lang hen til \textit{Hvitluft}- Elven, langs ved \textit{Langfiorden}; Tinden er høy og spidz opad, Fieldet er hvass-rygget og bart, paa Syd-ostlige Side ad \textit{Langfiorden} steilt, og mesten bart.\par
J Vest-Syd-vest herfra er\par
\textit{Revarfield}, som strekker sig fra \textit{Hvidluft}-Elv langs med \textit{Langfiordens} Nordre Side i Vest-Syd-Vest hen til dens Botten 1 god Miil lang, er noget høyt, mest fladt, og bart ovenpaa; paa Søndre Side hælder ned til \textit{Langfiord} med nogen Bierk og Græss paa; Paa nordre Side hænger sammen med u-bekiendte Fielde hen imod \textit{Oxfiord} i Nord, og hen imod \textit{Bers}fiord i Vester.\par
Jnde paa \textit{Langfiord}-Botten er en Slette, kaldes \textit{Storeng}, hvor Græss slaaes, og \textit{Bogn}, en Lax-Elv i \textit{Langfiord} udfalder.\par
Østen for \textit{Bogn}-Elv, paa Søndre Side af \textit{Langfiord}, er \textit{Vaddekeip}, et Field, strekkende sig efter \textit{Langfiorden} i Ost-Nord-Ost 1/2 Miil lang; i Sør hænger det, 1/2 Miil over vidt, med \textit{Akevagg}- Field sammen, noget høyt, bart ovenpaa, paa Nordre Side oventil brat, men nedenunder hældendes ad \textit{Langfiorden}, med smaa Bierk paa. \textit{Vaddekeip} naaer i Øst-Nord-ost hen til \textit{Akevaggs} Elv; Østen for denne \textit{Akevaggs} Elv er\par
\textit{Ulvsvagg}-Field, liggendes langs med \textit{Langfiordens} Søndre Bredde i Ost-Nord-Ost 3/4 Miil lang hen til \textit{Ulvsvagg}-Elv, i Sydost henger med øde Snee-tækkede Fielde sammen hen ad \textit{Tallvigens} Fielde, hvorover er gaaendes Vej 1 Miil; Oven paa fladtvoren og bart, paa Nordre Side ad \textit{Langfiorden}, ligesom \textit{Vaddekeip}, skabt.\par
Østen for \textit{Ulvaggs} Field er \textit{Ulvaggs}-Elv, og Østen for denne \textit{Allegass}, et Field, strekkende sig efter \textit{Langfiord} i Ost-Nord-ost 1. god Miil langt, hen til det Næss, \textit{Langnæss}; J Sør er dette \textit{Allegass} 1/2 Miil vidt over til\par
\textit{Tallvig}-Field, rundt ovenpaa med en Klimp paa, og bart, paa Nordre Side ad \textit{Langfiord} brat, med noget Bierk og Græss neden under, sommestedz.\par
Hermed i Øster hænger sammen \textit{Langnæss}, som giør det Søndre Næss af \textit{Langfiorden} paa det faste Land; Dette \textit{Langnæss}-Field vender sig fra Gabet af \textit{Langfiorden} i Sør efter \textit{Altens} Fiordz Vestre Søe-Bredde 1/2 Miil lang hen til \textit{Tall}viig, en bekiendt Viig af \textit{Altens} Hoved-Fiord, ovenpaa fladt og bart, neden under ad \textit{Altens}Fiord græsset og Bierke-groet; Fra dette \textit{Langnæss}-Field stikker i Nord hen til Gabet af \textit{Langfiord} bem.te Næss \textit{Langnæss}, som i sig selv er smalt, 1 a 2 Bøsse-Skud over, med Græss og Bierk begroet.\par
Fra bemeldte \textit{Langnæss} til \textit{Tallvigens} første eller Nordre Næss, \textit{Jansnæss}, i Sør er, som meldt, 1/2 Miil; Landskabet derimellem nu nærmere at forklare, saa er det strax fra \textit{Langnæsset} først en 4 Bøsse-Skud slet med Eng og Bierk, derefter Berget, h[v]oraf den nordreste Deel\hypertarget{Schn1_62776}{}Schnitlers Protokoller V.\par
er høyest, kaldet \textit{Storberget}, bart og ganske brat ned ad \textit{Altensfiord}, den øvrige Deel ned ved Fiorden er med Bierk og sommestedz med Græss begroed.\par
Det Søndre Næss af \textit{Tallvigen} heder \textit{Kielsberg}, som fra det Nordre Næss \textit{Jansnæss} ligger i Sør, og er Gabet af \textit{Tallvigen} derimellem en 3 à 4 Bøsse-Skud vidt;\par
Fra bem.te Søndre Næss \textit{Kielsberg} til et andet Næss i Øster af samme Sides faste Land, \textit{Kraagnæss}, er imod 1/4 Miil, og imellem disse 2de nemlig \textit{Kielsberg} og \textit{Kraagnæss} er den Bugt \textit{Melsviig} 1/8 Miil dyb i Sør.‒ Og fra \textit{Jansnæss}; til dette \textit{Kraagnæss} er 1/4 Miil i Syd-ost. Bagen for eller Vesten for \textit{Langnæss}-Field, og Sønden for \textit{Allegass}-Field ligger\par
\textit{Tallvig-Field}, strekkende sig fra Nord, nemlig fra \textit{Allegass}-Field i Sør til forbi \textit{Storvandet} 1 Miil langt, omtrent 3/8 Miil bredt; med en rund Klimp paa Nordre Ende, ellers rundrygget og fladt, sommestedz med Lyng, sommestedz skallet ovenpaa, nedhældendes baade paa Østre og Vestre Sider, oventil sammestedz bart, nedentil deels Græss- og Bierke-groet, deels berget.\par
Fra \textit{Kraagenæss} er til \textit{Kaafiord} i Syd-ost en 3/8. Miil; J dette Mellem-Rom er en Viig, kalded \textit{Storviig}; Landet fra \textit{Kraagnæss} til \textit{Storvigen} er fieldet; og det Field heder\par
\textit{Kraagnæss}-Field, strekkende sig i Sør til Osten imod 1/4 Miil, fra Øster i Vester hen til \textit{Stor-vattenets} Botten ved 3/4 Miil stort, med mange bratte Skarer j, ovenpaa bart, paa Østre Side deels steilt og snaut, deels liet med Græss og Bierk paa.\par
Naar \textit{Kraagnæss}-Field tager Ende i Sør, møder Eng-Slette og fladt Land, et Par BøsseSkud vidt i \textit{Storvigen};\par
Herpaa følger i Syd-ost \textit{Storviig-Næsset}, et høyt Kalk-Berg, en 3. Bøsse-Skud langt i Sør, ganske smalt over i Vester, paa Vestre Side er nedslutendes, at man kan gaae did op, paa de andre 3de Sider steil-brat, ovenpaa med noget Bierke-Kratt begroet, paa Sidene bart;\par
Fra \textit{Kraagenæss}, er meldt, at være til \textit{Kaafiord} 3/8. Miil; Denne \textit{Kaafiords} Nordvestlige Næss heder \textit{Øsekar}-Næsset, og det Syd-ostlige, \textit{Simon-Næsset}. Paa \textit{Storvig-Næsset} følger i Sør\par
\textit{Øse-karr-Næss}, saa kaldet, at det seer ud, som et omvendt Øse-Karr, fra Nord i Sør 2 Bøsse-Skud langt, 1 Bøsse-Skud over, og meere bredt;\par
Det Syd-ostlige Næss av \textit{Kaafiord} er \textit{Simon}-Næsset, langagtigt i Syd-ost, ved 3 BøsseSkud, hen til \textit{Qvænviig}, bredt i Syd-vest et Par Bøsse-Skud.\par
J mellem disse 2de, nemlig \textit{Øsekarr-Næss}, og \textit{Simon-Næss}, er\par
\textit{Kaafiord}, som i mellem de Næss kaldes \textit{Kaafiord}-Sundet, og i Gabet er en 2 à 3 BøsseSkud viid. Paa Nordre Side af \textit{Kaafiord} ligger\par
\textit{Kaafiord-Field}, 1/2 Miil efter Fiorden i Vest-Syd-vest langt, i Nord sammenhængendes med andre u-bekiendte Fielde hen til \textit{Tall}v\textit{igen}; ovenpaa slet og bart, paa Søndre Side ad \textit{Kaafiorden} nedhældendes, med Furre og Bierk paa.\par
Paa Søndre Side af \textit{Kaafiord} er det Field \textit{Sakkeband}, strekkende sig fra \textit{Simon}-Næsset efter Fiorden i Vest-Syd-vest, imod 1/2 Miil langt, hen til Fiordbotten, et Par Bøsse-Skud over bredt, mest rundrygget og steenet ovenpaa, paa Nordre Side ned ad \textit{Kaafiord} steilt og bart.\par
\textit{Simon-Næsset} giør det vestlige Næss af \textit{Qvæn-viig}; denne \textit{Qvænviig} er i Gabet et Par Bøsse-Skud viid i Syd-ost, 2 à 3. Bøsse-Skud dyb; Jnde i Botten er slet Græss-Land med Furre og Bierke-Skoug, indtil\hypertarget{Schn1_63045}{}Ved Altens Fiord i VestFinmarken.\par
\textit{Skaane-vara}, som giør det østlige Næss af \textit{Qvænviig}; Dette \textit{Skaane-vara} er meget høyt, og rundt, 2 à 3. Bøsseskud over stort i Syd-ost hen til \textit{Gimmeluftviig}, ovenpaa rundtoppet, og bart, i Nord til \textit{Altens} Fiord, og i Vester til \textit{Qvænviig} brat og snaut, paa de andre Sider langsludtendes, med Furre og Bierk begroet.\par
\textit{Gimmeluft}-viig er 1 Bøsseskud viid i Gabet i Syd-ost, og ligesaa dyb.\par
Landskabet om denne \textit{Gimmelluft}viig er slet med Græss og smaa Skoug paa.\par
Strax Østen for \textit{Gimmelluftviig} er en Slette, navnl. \textit{Eggskall}-Bye, hvor Qvæn-Borgerne fra \textit{Tornestad} have søgt til at handle, som dennem er forbuden.\par
4 Bøsse-Skud i Syd-ost fra \textit{Gimmelluft}viig er \textit{Baasekop-viig}, som er kun en liden Bugt; deraf merkelig, at \textit{Norske Qvæner og Finner}, som boe i Skougen og paa Fieldene, der have deres Fiske-Skiaaer og Boder.\par
Østen for \textit{Baasekop-viig} gaaer et Næss udj \textit{Altens}Fiord, navnlig \textit{Baasekop-Næsset}.\par
3 Bøsse-Skud i Syd-ost fra \textit{Baasekop-Næsset} ligger \textit{Torleviig}, en liden Bugt.\par
Ved \textit{Torleviig} i Syd-ost er \textit{Kongshavn-Field}, meget høyt, og rundt, nogle Bøsse-Skud over stort, ovenpaa skaldet, paa den Nordre Side herfra udstikker i \textit{Altens}-Fiord et Næss, navnlig \textit{Kongshavn-Næsset}, rund-aflangt, med Lyng og Furre-Krat paa; de andre Sider af \textit{Kongshavn}field have nogen smaa Skoug paa sig. ‒\par
Østen for \textit{Kongshavn-Næsset} er en Bugt, \textit{Kongshavn-Bugten}, i Gabet i Syd-ost 1/8 Miil viid, 1/16. Miil dyb; Landet derved er slet, med Furre-Krat begroet.\par
Det Østre Næss af \textit{Kongshavnbugten} heder \textit{Gusøren}, som strekker sig til Munden af \textit{Altens}-Elv; Sønden for denne \textit{Gusøren}, 1/16 Miil fra Fiorden ligger Amtmandens \textit{assignerede}\textit{Altens} Gaard, paa en stor Slette ved \textit{Altens} Elv, som falder ind i \textit{Altens} Fiordz Botten.\par
Denne forklarede Landstrekning fra \textit{Andsnæss} (see \textit{pag.} 207) til \textit{Altens}Fiordz Botten er den nordreste Deel i Vester af Vest-\textit{Finmarkens} faste Land, hvoruden for i Nord ligge de Øer:\par
\textit{Loppen, Silden, Sørøe, Stiernøe} og \textit{Seiland}. Det Vestreste Stykke af denne Landstrekning fra \textit{Andsnæss} til \textit{Silden} reignes for Hav-Siden, og kan derimellem være i Øster til Norden 1 Miil\par
Fra \textit{Silden} af begynder \textit{Altens}Fiord, som strekker sig først i en Ost-Syd-ost, siden i Sydost til dens Botten {en 6 Miil _______ i alt 7 Miile.}\par
J den 1ne Miils Strekning fra \textit{Andsnæss} til \textit{Silden} indgaae 2de Fiorder i det faste Land omtrent i Sør, nemlig Vesten-fra at reigne, \textit{Frakfiord} og \textit{Bersfiord}:\par
(1) \textit{Frakfiord}, dens Vestre Næss heder \textit{Frakfiordnæss}, Dens Østre Næss \textit{Skallenæss}, ligger i Nordost 1/4. Miil med sit Vestre Næss fra \textit{Andsnæss}; Denne \textit{Frakfiord} er i Gabet 1/4 Miil viid fra Vester i Øster, og 3/8 Miil dyb i Sønden; fiskeriig, beboes af en 6 Mænd. (see \textit{pag.} 208)\par
2) \textit{Bersfiord}, dens Vestre Næss er Nordre-\textit{Skalle-næss}, dens Østre \textit{Lørdags-næss}; denne \textit{Bersfiord} er i Gabet 1 Miil breed, at forstaae lige over \textit{Silden}-Øe, (som med sin Søndre Ende ligger i dette Gab) og stikker i Sønden omtrent 1 1/4 Miil dyb, hen imod \textit{Jokkel}-Fiord, en Jndfiord af den \textit{Nordland}ske \textit{Qvænangens} Fiord, dens Nordostlige Ende, see \textit{pag.} 208 her.\par
Det øvrige Støkke af \textit{Finmarkens} faste Land, nemlig de 6. Miiles Strekning fra \textit{Silden-øe} til \textit{Altens}fiordz Botten udgiør den Søndre, og Syd-vestlige Land-Side af \textit{Finmarkens} første, \hypertarget{Schn1_63394}{}Schnitlers Protokoller V. nemlig \textit{Altens} Hoved-Fiord, fra hvilken \textit{Altens}Fiord adskillige Jndfiorder, eller Tvær-Fiorder indskiære dette faste Land fra Nord i Sør, eller i Syd-vest, nemlig Vesten-fra at reigne:\par
Saasnart man farer fra \textit{Silden-øe} forbi Gabet af den forklarede \textit{Bersfiord}, og dennes Østre Næss, \textit{Løverdags-Næsset}, see \textit{pag.} 208 hertilforn, saa møder\par
1) \textit{Ullfiord}, hvoraf det Vestre Næss, er samme \textit{Løverdags-Næss}, og det Østre heder \textit{Nuenæs}, see \textit{pag.} 208 her. Denne \textit{Ullfiord} ligger Østen for \textit{Bersfiord} en god 1/16. Miil, og Vesten for \textit{Nuss}-Fiord 1/8 Miil, er i Gabet nogle Riffel-Skud viid, og i Syd-vest 2/4 Miil dyb, havendes 2. boesiddendes \textit{Finner}.\par
2) \textit{Nuss-Fiord} ligger 1/8 Miil Østen for \textit{Ullfiord}, og en god 1/4 Miil Vesten for \textit{Oxfiord}; denne \textit{Nussfiords} Vestre Næss giør næst-omtalte \textit{Nue-Næs}, og Østre Næss \textit{Klubbenæss}, see \textit{pag.} 208 her. \textit{Nuss-fiord} er i Gabet 1/8 Miil viid, og 1/4 Miil i Syd-Ost dyb, havendes mitt paa sig et smalt Sund af et Par Riffel-Skudz Bredde. J denne \textit{Nuss}-Fiord paa den Østre Side indfalder Østen-fra en Bæk, 3 a 4 Riffel-Skud lang, udaf Vandet \textit{Nussfiord-Vatten}, 3 à 4 RiffelSkud langt, halv saa bredt, havendes Øreter og anden smaa, kalded Al-fisk. J denne \textit{Nuss}- Fiord boer Jngen.\par
3) \textit{Oxfiord} ligger en god 1/4 Miil Østen for \textit{Nusfiord}, dens Vestre Næss giøres af samme \textit{Klubbenæss}, dens Østre af \textit{Ystnæss}, see \textit{pag.} 208 her, i Gabet er \textit{Oxfiord} 1/4 Miil breed, og stikker fra Nord i Sønden til Vesten 1 Miil, siden vender sig i Øster 3/4 Miil lang, hen imod \textit{Langfiord}, den \textit{Finmarki}ske Jndfiord af \textit{Altens} Hoved-Fiord, saa at der er fra \textit{Langfiordens} til denne \textit{Oxfiords} Botten kun 1/4. Miil over til Landz;\par
1/2 Miil Østen for \textit{Oxfiord} ligge 2de Bugter, navnlig\par
4) \textit{Laakkerfiordene}, fra hinanden nogle Bøsse-Skud liggende i Øster; Det Vestre Næss af Vestre \textit{Laakkerfiord} er forbemeldte \textit{Yst-næss}; det Næss imellem de 2de \textit{Laakker}Fiorder heder \textit{Laakker-Tinden}, det Østre Næss af \textit{Østre Laakker-Fiord} er \textit{Østre-Laakker-Næss}, see \textit{pag.} 208 her tilforn.\par
5) \textit{Langfiord} ligger fra \textit{Østre-Laakker}fiord 1. Miil, og fra \textit{Oxfiorden} 1 1/2 Miil i Ost-Nord-ost; \textit{Langfiordens} Nordre Næss heder \textit{Klubnæss}, see p. 209 her, og dens Syd-ostlige Næss \textit{Langnæss}, see \textit{pag.} 209 tilforn. \textit{Langfiorden} er i Gabet imellem disse 2de Næss 1/2 Miil breed, og stikker fra Gabet ind i Landet ved 2 Miile dyb i Vest-Syd-vest, dog noget ind ad bliver den trang til 1/4 Miils Bredde; Denne \textit{Langfiords} Botten ligger i Ost-Nord-ost imod 3/4 Miil veis fra den \textit{Nordland}ske \textit{Altfiord}-Botten, imellem hvilke et Eid er, hvorover gaaes til Landz fra en Fiords- til andens Botten, for at undgaae den lange Om-Vej om \textit{Andsnæss} og \textit{Loppen}, see \textit{pag.} 207 her.\par
Paa Søndre Side af denne \textit{Langfiord} boe 5 Normænd tæt ved hinanden, og 8. \textit{Finne- Familier} i 6. \textit{Finne}-Gammer, eller Hytter; Paa Nordre Side 3de \textit{Finne-Familier} i 2de Gammer. Paa begge Sider af \textit{Langfiord} er Bierke-Skoug, men Fieldene oventil ere skallede; Jnde i Botten er og kun Bierk i Sør, og Østen, ingen Furre.\par
Elve og betydelige Bække rinde i \textit{Langfiorden:}\par
1) \textit{Røsse}-Elv, opkommendes paa \textit{Alt-Eidet} Norden og Østen for \textit{Sokammer}, samler sig af adskillige Bække, som falde ned fra Fieldene, og løber i Ost-Nord-ost ved 1/4 Miil lang i \textit{Langfiords} Botten paa dens Nordre Ende.\hypertarget{Schn1_63721}{}Ved Altens Fiord i VestFinmarken.\par
2) \textit{Bogn}-Elv kommer fra Syd-ost af Fieldene, og rinder, saavidt sees kan, vel 1 1/2 Miil i Nord til Vesten i den Søndre Ende af \textit{Langfiord}botten. 1/2 Miil Østenfor \textit{Bogn}-Elv er\par
3) \textit{Akevagg}-Elv, som oprinder paa høyeste Field, \textit{Akevara}, og gaaer 1/2 Miil lang i Nord omtrent, igiennem den Dal \textit{Akevagg}, i \textit{Langfiorden} paa dens Søndre Side. 3/4 Miil Østen for \textit{Akevagg}-Elv er\par
4) \textit{Ulvsvagg}-Elv, opkommendes af høyeste Field, paa \textit{Ulvs-vara}; den rinder igiennem den Dal \textit{Ulvs-vagga}, i Nord til Vesten 1/2 Miil lang i \textit{Langfiorden}, paa dens Søndre Side\par
Paa Nordre Side falde i \textit{Langfiorden} Bække\par
5) \textit{Hvidluft}-Elven, 1 1/4 Miil Østen for \textit{Langfiordbotten}, eller \textit{Røsse}-Elv; denne \textit{Hvidluft} Elv har sin Oprindelse af det Field \textit{Hvidluft}, og løber 1/8 Miil lang i Sønden i \textit{Langfiorden} paa den Nordre Side. 1/2 Miil Østen, eller i Nord-ost fra \textit{Hvidluft}-Elven er\par
6) \textit{Kaaven}-Elv, som Norden-fra udløber af et Vand, \textit{Kaaven-vattenet}, i Sør 1/8 Miil lang i \textit{Langfiord} paa dens Nordre Side; \textit{Kaav-vattenet} er fra Vester i Øster 1/2 Miil langt, og 1/8 Miil bredt: dog mitt paa smalt, giørendes et Sund af 2. eller 3. Bøsse-Skudz Vidde;\par
Fisk i \textit{Langfiorden} gives Torsk og Sej: dog i en Par Aars Tid har Fiskeriet formindsket sig;\par
J \textit{Bogn}-Elv og \textit{Kaaven-Vatten} ere Øreter,\par
J Denne \textit{Langfiord} avles ei Korn; Thi her bliver ej bart for Snee, førend 8. Dage efter Sanct Hans Tid.\par
Øer ere ikke inde i \textit{Langfiorden}, uden en øde steened Holm, nær ved Fiordens Nordre Land-Side, hvor \textit{Kaaven}-Elv udfalder, omtrent 3/4 Miil fra Langfiordens Gabs Nordre Næss, \textit{Klubnæss}.\par
Videre at fortfare med Jndfiordenes Beskrivelse af \textit{Altens} Hoved-Fiord paa den Søndre og Sydvestlige Side af \textit{Finmarkens} faste Land, saa følger paa N: 6. \textit{Langfiorden}, i Syd-Syd-ost den Viig\par
7) \textit{Tall-viig}; hvis Nordre Næss heder \textit{Jansnæss}, og Søndre Næss \textit{Kielsberg}; Jmellem hvilke Næsse Gabet af \textit{Tallvigen} er 3 à 4. Riffel-Skud bredt, (see \textit{pag.} 209 her tilforn) Fra Gabet indtil Botten kan \textit{Tallvigen} være 4 a 5. Bøsse-Skud dyb i Vest-Syd-vest, paa dens Nordre Land-Side staae de \textit{Kiøbenhavn}ske Handels Husse, og i Botten \textit{Altens} Hoved-Kirke af Træ; Jnden for Gabet viider denne Viig sig meere ud.\par
J denne Tallviig falde smaa Aaer:\par
a) Tallvigs-Elv, kommendes Vesten-fra af \textit{Tallvigs}-Fieldet, stikker i Øster 1/4 Miil lang i \textit{Tallvigens} Botten paa dens Søndre Side. (see \textit{pag.} 210 her tilforn)\par
Fra det Søndre Næss \textit{Kielsberg} er til det andet Næss i Øster, \textit{Kraagnæss} imod 1/4 Miil; Jmellem disse 2de Næsse er den Bugt \textit{Melsviig} (see \textit{pag.} 210) Jmellem \textit{Tallvigen} i sær, og denne \textit{Melsviig} falder den Bæk\par
b) \textit{Hals-elv} ud, paa Søndre Side af det faste Land, kommendes af \textit{Storvandet}, fra Vester i Øster, ved 6. Riffel-Skud langt; dette \textit{Storvand} ligger fra Syd-vest i Nord-ost 1/8 Miil langt, 3 a 4. Bøsseskud bredt, havendes Lax og Øreter.\par
Fra \textit{Tallvigens} Østre- eller \textit{Yttere} Næss \textit{Kraagnæss} i Syd-ost 3/8 Miil aabner sig\par
8) \textit{Kaa-fiord}, hvis Nordvestlige, og Sydostlige Næsse ere forklarede før pag. 210 her; Dens Gab imellem de 2de Næsse ere 2 a 3 Bøsse-Skud breed, og kaldes \textit{Kaafiord-Sundet}; Lang \hypertarget{Schn1_64042}{}Schnitlers Protokoller V. er denne \textit{Kaafiord} i Vest-Sydvest 1/2 Miil; og har indentil 2de nemlig \textit{Yttre}- og \textit{Jndre Strømme}; den \textit{Yttere Strøm} er fra Sundet, eller Gabet 3/8. Miil, og et Par Steen-Kast lang; den JndreStrøm, 1 Steen-Kast lang og breed, er imod 1/8 Miil inden for \textit{Yttere-Strømmen}, og fra FiordBotten 1 Bøsse-Skud;\par
J \textit{Kaafiord}-Botten paa dens Søndre Side falder \textit{Kaafiord}-Elv, i Nord-ost 1/16 Miil lang af \textit{Kaafiord}-Vand, langagtigt, ligesom Fiorden i Nord-ost, en god 1/4 Miil, bredt 1/16. Miil. ‒\par
Fra Gabet af denne \textit{Kaafiord} strekker Landet sig i Ost Sydost 1/2 Miil til Botten af \textit{Altens}- fiord;\par
Dette er saaledes den forklarede Landstrekning af \textit{Finmarkens} faste Jndland med dets Jndfiorder fra \textit{Andsnæss}, det Sydligste og Vestlige Næss af \textit{Finmarken}, indtil \textit{Altens} Botten, og 7. Miles Længde.\par
Udenfor dette faste Land i Nord, nemlig fra \textit{Andsnæss} i Nord- til Osten 1/2 Miil, fra \textit{Frakfiordens} Østre Næss, \textit{Skallenæss}, 1/4 Miil i Nord-vest, og fra \textit{Silden-øe} 1/2 Miil i Vester ligger den Øe\par
\textit{Loppen}, omkring 3/4 Miil stor, fra Nord-vest i Syd-ost 1/4 Miil lang, i den Nordvestlige Ende smal, men paa den Sydostlige Side, hvor den er videst, 1/8 Miil breed; paa Nordostlige Side slet og jævn, men paa Sydvestlige Side høyberged; Mitt paa den Nordostlige Side er en liden Bugt, kaldes \textit{Me-vær}, eller \textit{Mitt-vær}, 2 Bøsse-Skud viid, og halv saa dyb, og herfra 1/16 Miil i Nord en anden Bugt, \textit{Ytter-vær}, 4 Bøsse-Skud viid, og 1/4 Bøsse-Skud dyb;\par
Denne \textit{Loppen-øe} har en Kirke af Træ, paa den Sydostlige Side, samt Præstegaarden, Klokkeren, Lensmanden og 3 Bønder derpaa boendes: \textit{item} paa den Nordvestlige Side et Fugle-Berg, hvor Lund og Alk tilholde, temmelig høyt og steilt, oven paa slet, sommestedz græsset; sommestedz Lyng-groet med Multer paa, 1/8 Miil fra Syd-ost i Nord-vest langt, og halv saa bredt;\par
Lundfuglen graver sig her ind i Fieldet, hvor Lyng- og Jord-groede Sprekker ere, 1 Haand- eller Arm- eller Favn-dyb, saavidt den komme kan; derfra den af Bønderne ved en lang Stang med en lige Angell paa uddrages.\par
Men denne \textit{Loppen}-Øe haver ingen Skoug. ‒\par
Fra \textit{Loppen} i Øster 1/2 Miil ligger den Øe \textit{Silden}, inde i Gabet af \textit{Bersfiord}, fra Syd-ost i Nordvest 1/2 Miil lang, mitt paa, hvor den er videst, 1/4 Miil over breed, men i Endene smal udgaaendes, dog er den Søndre Ende inde i Fiord-Gabet noget bredere, end den yttere Nordre Ende; Saa \textit{Silden} er omkring 1 Miil stor, paa Vestre Side berged, paa Østre Side noget fladtvoren og slet, u-beboed, med lidet Bierke-Skoug paa, forlanded, ond at komme til, sommestedz græssed, som bruges af de Omliggende Nærboende. ‒\par
1 Miil j NordOst for \textit{Silden}, ligger den Øe \textit{Sørøe}, og i særdeleshed dens Søndre Næss \textit{Haaen}; Jmellem \textit{Sildenøe} og dette \textit{Sørøens} Søndre \textit{Haaen}-Næss begynder den 1te \textit{Finmarkens} Hoved-Fiord\textit{Alten}, saa at dens Gab imellem disse 2de Øer sig aabner;\par
Denne \textit{Altensfiord} gaaer ind fra Havet imellem \textit{Finmarkens} faste Land paa Søndre- og de Ud-øer \textit{Stiernøe} og \textit{Seiland} paa Nordre Side i Ost-Syd-ost, siden videre imellem \textit{Finmarkens} faste Land i Syd-ost til dens Botten, i alt ‒ 6 Miile lang; Hvilken \textit{passage} saaledes nærmere forklares: \textit{Altens}fiord fra Gabet imellem \textit{Silden} og \textit{Haaen} gaaer først i Ost-Syd-ost til \textit{Stiern\hypertarget{Schn1_64272}{}Ved Altens Fiord i VestFinmarken. sunds} Begyndelse imellem \textit{Ystnæss}, paa faste Landz Søndre Side, og Vestre \textit{Stiernøers} Odde paa Nordre Side, gode {1 1/2 Miil} derfra igiennem det Sund \textit{Stiernsund}, i Ost-Syd-ost til dette \textit{Stiernsunds} Østre Ende imellem \textit{Langfiords Klubnæss} paa faste Lands Søndre Side, og \textit{Østre Stiern}-odden paa Nordre Side, imod {2 ‒} Siden i Syd-Syd-ost til imellem \textit{Langnæss} paa faste Landz Søndre Side og \textit{Kierringfiord}-Næss paa Øen \textit{Seiland}{1/2 ‒} Videre i Syd-Syd-ost til \textit{Kraagnæss} paa det faste Landz Sydvestlige Side, og \textit{Altnæss} paa det faste Landz Nordostlige Side {1 ‒} saa Fiorden haver denne \textit{Cours} imellem begge Sider af \textit{Finmarkens} faste Land Derefter fremdeles imellem det faste Land i Syd-ost til Botten {1 Miil ______________} Som giør i alt dens Længde {6 Miile. ______________}\par
Bredden af \textit{Altensfiord} er saaledes: J Gabet imellem \textit{Silden}-øe og \textit{Haaen}-Næss er {1 Miil} Jmellem \textit{Ystnæss}, og \textit{Vestre Stiern}-odden{3/8 ‒} Jmellem \textit{Langfiords Klubnæss}, og \textit{Østre Stiern-Odden}{1/2 ‒} Jmellem \textit{Langnæss} paa faste Landz Søndre Side, og \textit{Kierringfiord-Næss} paa \textit{Seiland}, god {1 ‒} Jmellem \textit{Kraagnæss}, og \textit{Altnæss}, begge af \textit{Finmarkens} faste Land, og derfra fremdeles indtil Botten imod {1/2 ‒ _______________}\hspace{1em}\par
Denne \textit{Altensfiord} er fiskeriig paa Torsk og Sej, naar Silden gaaer til, som nu i 2de Aar er udebleven; og i \textit{Altens} Elv fanges Mængde af Lax;\par
At \textit{continuere} nu med Landstrekningens Beskrivelse af \textit{Altens}fiord fra sist forklarede \textit{Kongshavn}-Field og Bugt (see \textit{pag.} 211 her) saa settes forud, at den Deel af \textit{Altens}fiordz Botten fra \textit{Altens}Elvs Mund i Sør, til \textit{Røslogt} et Næss Østen for \textit{Pors}-elven 1/16 Miil, derimod i Nord, kaldes i almindelighed \textit{Rafsbotten};\par
Hvor \textit{Kongshavn}-Field i Syd-ost ophører, der begynder en Jord-Bakke, med Furre og noget Bierk begroed, navnlig \textit{Sandfaldhoug}, og strekker sig paa den Syd-vestlige Side, 1/16 Miil fra Fiorden, hen til Munden af \textit{Altens} Elv, i Øster 1/8 Miil lang, 3de Bøsse-Skud over breed; Paa begge, saavel Nordre som Søndre Sider er Furre Skoug, og imellem denne \textit{SandfaldHoug} ved \textit{Altens}Fiord, er viid angenæm Slette, hvorpaa Amtmandens Gaard, \textit{Altens} Gaard, staaer bygged; der hvor denne \textit{SandfaldHoug} i Øster hælder ned ad \textit{Altens} Elv, kaldes den Elvebakken, og der boe Folk ved Elven.\par
\textit{Altens} Elv er i Munden 1 Bøsseskud over viid. Østen for \textit{Altens}Elv møder det Field \textit{Reipas}, fra Elv-Munden 1/4 Miil i Sør liggendes, paa hvilket Mellem-Rom Furre-Skoug er; \textit{Reipas} strekker sig herfra fra Nord i Sør imod 1/2 Miil langt, halv saa bredt over i Øster, ovenpaa slet og bart, paa begge Sider ad Elvene noget brat, med Furre Skoug bevoxen.\hypertarget{Schn1_64538}{}Schnitlers Protokoller V.\par
Østen for \textit{Reipas} er en Aae, kalded Tvær-Elv, som løber sammen med \textit{Altens} Elv i Munden, og dermed tillige udgaaer i \textit{Altens} Fiordz Botten, kalded \textit{Rafsbotten}, paa dennes Søndre Side, see pag. 218 meere.\par
Østen for Tvær-Elven 1/8 Miil udrinder i \textit{Rafsbotten}\textit{Trunhiemsfar}-Elv; Jmellem disse 2de nemlig Tvær- og \textit{Trunhiemsfar}-Elve ligger det Field \textit{Borha}, fra \textit{Rafsbotten} 1/8 Miil i Sønden, hvorimellem Furre-Skoug er; dette \textit{Borha} strekker sig fra Nord-vest i Syd-ost 1 Miil langt, 1/2 Miil imellem Elvene meer og mindre bredt, oventil rundvoren og bart, med Mosse paa, paa Sidene til begge Elve fladtvoren med Furreskoug begroet. Paa den Nordostlige Side af denne \textit{Trunhiems}far-Elv ligger det Field \textit{Rapsniarg}, 1/4 Miil fra \textit{Rafsbotten}, hvorimellem Bierk og Furre er, i sin Skabning rundvoren efter Fiord Bottens Dannelse, 1/4 Miil over stort, ovenpaa fladtvoren og mosset, paa Sidene sommestedz bratvoren, sommestedz fladt med Furre og Bierk paa. Fra dette \textit{Rapsniarg} omkring \textit{Rafsbotten} i Nordvest til \textit{Altnæss} hede Fieldene med et almindeligt Navn \textit{Altnæss}-Fielde, hvoraf dog et hvert har sit særdeles Navn:\par
Tæt Norden for \textit{Rapsniarg} er en liden Aae \textit{Rafsbott-elv} som rinder ud i \textit{Rafsbotten} En god 1/4 Miil fra \textit{Rafsbott}-Elven i Nordvest er \textit{Pors}-Elv, og herimellem det Field \textit{Norjevara}, saa bredt i Nordvest, som Rommet imellem disse Elve er, begynder 1/8 Miil Østen for Fiordbotten, hvorimellem Bierk og Furre er, og er 1/16 Miil over stort i Øster, oventil bart med Mosse paa, fladtvoren paa Sidene med noget Furre og Bierk paa.\par
Med dette \textit{Nørje-vara} henger i Øster sammen et Field, hvoraf Tinden heder \textit{Gistu}; Fieldet er rundt og fladtvoren, bart med Noget Mosse paa, neden under med Bierk og noget Furre begroet.\par
Naar man kommer over \textit{Pors}Elven, forekommer \textit{Altnæss}Field, hvoraf \textit{Altnæss} stikker ud i Nordvest, 1/2 Miil fra \textit{Pors}-Elv Munden;\par
Norden for \textit{Altnæss} er en Bugt, navnlig \textit{Leerbotten}, hvori \textit{Leerbotts} Elv, eller paa \textit{Finnsk}, \textit{Rairo-jok} indfalder.\par
Det Field nu imellem \textit{Pors}-Elv og \textit{Rairo-jok} er \textit{Altnæss}Field i Særdeleshed, eller paa \textit{Finnsk}, Æli, som imellem begge Elve er 1/2 Miil langt, (at forstaae Søe-Miil) og ligesaa bredt, ovenpaa sommestedz slett, sommestedz knubret, og bart, paa Sidene deels brat, deels fladtvoren, med Bierk og lidet Furre paa.\par
Strax Norden for \textit{Altnæss}Field er \textit{Rairo-jok}, og 1/4 Miil norden for \textit{Rairo-jok}\textit{Skill}-Fiord; Herimellem ligger det Field \textit{Valginiarg}, 1/4 Miil efter Fiorden, fra Sør i Nord langt, og fra Vester i Øster 1/2 Miil vidt, ovenpaa fladtvoren og mosset, paa Sidene brat med smaa Bierk paa.\par
\textit{Skillfiord} er i Munden 1/4 Søe-Miil viid, 1/8 Miil dyb; Jmellem denne \textit{Skillfiord} og følgende \textit{Korsfiord} er \textit{Korsnæss}-Field fra Sønden i Norden 1/4 Miil vidt, og ligesaa bredt, med smaa Bierk paa Sidene. ‒\par
Fra \textit{Skillefiord} til \textit{Korsfiord} er i Nord 1/4 Miil; Denne \textit{Korsfiords} Søndre Næss heder \textit{Korsnæss}, dens Nordre Næss \textit{Snuda-Niarge}, bredt derimellem en knap 1/4 Miil; Fiorden stikker ind i Øster 1 Miil lang.\par
J \textit{Talviig} blev man paa Reisen opholdet til
\DivII[Mars 19. Fra Talvik til Altafjordbotn]{Mars 19. Fra Talvik til Altafjordbotn}\label{Schn1_64812}\par
19de \textit{Martij}, formedelst haardt Vejerligt, da man derfra foer i Syd-ost til Altensfiordz Botten 1 Sterk Søe-Miil; Og er herved at merke, at 1 \textit{Finmarkens} Søe-Miil er længere, end \hypertarget{Schn1_64828}{}Ved Altens Fiord i VestFinmarken. Een i \textit{Nordlandene}, og at Een \textit{Finmarkens} Field-Miil er halv saa lang, som Een Søe-Miil her, paa hvilken Maade \textit{Referenternes} og Vidnernes Udsagn om Miilenes Maal her i \textit{Protocollen} er at forstaae.
\DivII[Mars 20.-april 4. Innsamling av oplysninger]{Mars 20.-april 4. Innsamling av oplysninger}\label{Schn1_64845}\par
20de Kaldet \textit{Missions} Skolemesteren her paa Stædet, \textit{Niels Pedersen}, til mig, og afferdiget hannem med min skriftlige \textit{Ordre} til Fieldz, at tilsige ei alleene de \textit{privative Norske Lapfinner}, som \textit{sortere} under det \textit{Norske Masi-Capell}, 8 Miile Sønden for \textit{Altens} Botten, men og de fælles Undersaattere af \textit{Finner}, Norden for \textit{Landskiølen} hørende til den \textit{Svenske}\textit{Kautokeino Lappe} Kirke nemlig de af \textit{Kautokeino- Afjevara- Karasjocki}- og \textit{Juxbye}-Lappebyer, 16. til 18. Miile fra denne \textit{Altens} Botten liggende, at komme ned til mig, og forklare Grændsens Gang imellem Rigerne paa dette Sted.\par
De \textit{Lapfinner} fra \textit{Arisby}, eller \textit{Otjojockj-Lappe}-Bye Boede endda længere bort, og fornam man, at de sidst \textit{in Junio} og siden \textit{in Julio} om Sommeren opholde sig neere ved den \textit{Norske}\textit{Tana}-Fiord, da man beqvæmmeligst kunde foretage dennem under Vidne-Forhør. ‒\par
\textit{Missions} Skolemesteren var strax villig til denne Færd, og saasnart som han sig med fornøden \textit{provision} til denne Reise havde forsiunet, begav han sig i Vejen.\par
Amtmand \textit{Kieldsøn assisterede} mig med Raad og Daad i alt, hvis jeg til Deres Kongel. \textit{Majestets} Tienneste af hannem var begiærendes.\par
d. 24. \textit{Martj} indfandt sig efter \textit{ordre} de \textit{privative Norske Lapfinner} af det \textit{Norske Masi Capells} Meenighed, hvilke vel ingen Beskeeden om Grændse-Gangen, Sønden for \textit{Kautokeino}, vidste, men dog gave mig følgende Underretning om Landets Leje og Beskaffenhed fra \textit{Altens}- fiordz Botten op i Sønden imod \textit{Kautokeino}, og i Syd-ost imod \textit{Afjevara}, og \textit{Karasjok}, saavidt dennem bekiendt var:\par
Jsær 1) om Elvene, der Sønden og Østen-fra løbe i \textit{Altens}-Botten og Fiord, ved hvilke det indre Op-land \textit{distingveres}, og derefter dis bedre kan forklares\par
1. \textit{Altens}-Elv, hvor den først har sit Udspring, vidste disse Folk ikke: men det var dennem bekiendt at \textit{Altens} Elv rinder tæt Vesten forbi den \textit{Svenske}\textit{Kautokeino Lappe}-Kirke, og den \textit{Svenske Lappe}-Præstes Gaard sammestedz, hen til det \textit{Norske}\textit{Masi-Lappe-Capell}, i Nord til Vesten {8. Field Miile} siden i samme \textit{Curs} til \textit{Altensfiords}botten, hvor den i den saakaldede \textit{Rafsbotten} paa Søndre Side indfalder {8 ‒ _______ = 16. Field Miile}\par
J denne \textit{Altens} Elv løbe andre Tvær-Elve paa Vestre Side: 1. Miil Norden for \textit{Kautokeino} Kirke løber\par
(a) \textit{Skiabardas-jok} ind i \textit{Altens} Elv, 1 Miil lang i Nord-ost, af det Vand \textit{Skiabardasjaure}, hvilket Vand fra Nord-vest i Syd-ost er 1 Miil langt, og 1/4 Miil, hvor det er videst, bredt; J dette Vand ligger en liden Holm; Ved Vandet og Elven er Bierk. 7 Field-Miile fra \textit{Skiabardas} i Nord er\par
(b) \textit{Masi}-Elv, som kommer af det Vand \textit{Sodno-jaure}, og løber først i Sør 1 Miil, siden i Øster 1/2 = tilsammen 1 1/2 FieldMiile lang i \textit{Altens} Elv, et Par Bøsse-Skud Sønden for det \textit{Norske Masi-Capell}, som staaer paa en Slette 1/8 Miil viid, tæt ved \textit{Altens} Elvs Østre Side; \hypertarget{Schn1_65109}{}Schnitlers Protokoller V.\textit{Sodno-jaure} er 1/4 Miil langt fra Sør i Nord, 1 Bøsse-Skud over bredt; Ved \textit{Masi}-Elv er smaa Bierk; Ved \textit{Sodno-jaure} lidet Græss, og ved begge Steder til deels slett. Omtrent 6. Miile Norden for \textit{Masi-Capell} løber\par
(c) \textit{Eiby}-Elv ind i \textit{Altens} Elv, ligeledes paa Vestre Side fra Sør i Nord til Osten; den kommer først af det store \textit{Sollo-jaure}, og gaaer 1 Bøsse-Skud lang i det \textit{Lille Sollojaure}, derfra rinder den 2 Bøsse-Skud lang i \textit{Sielles-jaure}, herfra til \textit{Aukie-jaure} 1/4 Miil, siden ind i \textit{Altens Elv}, 2 Miile lang; \textit{Store Sollo-jaure} er fra Sør i Nord 1/2 Miil langt, 1/8 Miil bredt; Det Mindre \textit{Sollo-jaure} fra Sør i Nord 1/4 Miil langt, et Par Bøsse-Skud bredt; \textit{Sielles-jaure} fra Sør i Nord 1/2 Miil langt, 2 Bøsse-Skud bredt; \textit{Aukie-jaure} fra Sør i Nord 1 Miil langt, 1/4 Miil mitt paa, hvor det er videst, bredt, saa at \textit{Eibe}-Elv med sine Vande bliver gode 4 1/2 Miile lang.\par
Paa Østre Side løbe ind i \textit{Altens} Elv\par
(d) \textit{Lappo-jok}, kommendes Østen-fra, 1/2 Miil lang af det Vand \textit{Lappojaure}, som er i Nord-ost 1 FieldMiil langt, 1/8 Miil bredt; J Munden af \textit{Lappo-jaure} kommer\par
(e) \textit{Nappul-jok}, Sønden-fra af Fielde, 1 Field-Miil lang.\par
(f) \textit{Kalba-jok} rinder Østen-fra 1 Miil lang af \textit{Kalba-jaure} i \textit{Lappo-jaure} mitt paa Søndre Side; \textit{KalbaVande} er 2: fra Sør i Nord, 1 Bøsseskud fra hinanden, langagtige, det Søndre 1 Miil, det Nordre 1/4 Miil lang.\par
\textit{Fosser} i \textit{Altens} Elv ere:\par
a) \textit{Neidefoss}, 2 Miile Sønden for \textit{Masi-Capell}, brat nedfaldendes over et Berg. 1 Steenkast dyb, hvor ingen Lax kan komme op.\par
b) \textit{Børvel-foss}, 2 Miile Norden for \textit{Masi-Capell}, ganske trang imellem begge Siders Fielde, bratnedstyrtendes, at Laxen ej videre, end hertil komme kan.\par
Paa næstforrige, og denne Side opreignede Field-Vande have Siig, Røe og Gædder.\par
J \textit{Altens} Elv alt op til \textit{Børvelfoss} er ypperlig Lax-Fiskerie; ligesom og i Tvær-Elven, \textit{Eibe}-Elv, som bruges saaledes: Tvært over \textit{Altens} Elv giøres 2de Stængsler, den 1ne Sønden for den 2den Norden for \textit{Eibe}-Elv, med Teener i, og Garn paa begge Sider, hvorj Laxen løber; Andre mindre Stængsler bygges ligeledes med Teener i, og Garn paa Sidene, til ohngefær Mitt-Elvs; de 2de store Stængsler bruges af 32 Mænd til fælles Nytte; de Mindre haves af 16 Mænd paa hver Side, som skifte Sidene hver andet Aar; Neden derfor have 6 andre deres Stængsler. ‒\par
2 Tvær-Elv opkommer paa \textit{Tseunik}-Field i Syd-ost, og løber derfra Østen-for \textit{Altens} Elv 2 Miile lang, omsider i Munden af \textit{Altens} Elv, og med denne samme ud i \textit{Rafsbotten} af \textit{Altens}fiord. see \textit{pag.} 216 her.\par
3. \textit{Trunhiemsfar} Elv, (pag. 216 her see) oprinder i \textit{Ost-Sydost} af det Vand \textit{Siop-jaure} under \textit{Tseunik}-Field, og gaaer 1/8 Miil veis i Nordvest i det Vand \textit{Aukemak}, derfra siden i Vest- nord-vest 2 Miile lang i \textit{Rafsbotten}, at forstaae 1/8 Miil i Nord-ost fra \textit{Tvær-elvens} Mund.\par
\textit{Siop-jaure} er langt i Syd-ost 1/2 Miil, 1/16 Miil bredt, har ingen Skoug ved sig.\par
\textit{Aukemak} er langt i Syd-ost 1/8 Miil, 1 a 2. Bøsseskud bredt; har lidet smaa Bierk ved sig.\par
4. \textit{Rafsbott-elv} kommer af det Vand \textit{Lathekjaure}, og rinder i Vester 1/2 Miil lang i \textit{Rafsbotten}, fra \textit{Trunhiems} far-Elvs Mund 1/4 Miil. \textit{Lathek-jaure} er rundt, 1 1/2 Bøsseskud over bredt, havendes smaa Bierk om sig.\hypertarget{Schn1_65403}{}Ved Altens Fiord i VestFinmarken.\par
5. \textit{Pors}-Elv udspringer af \textit{Pors-jaure}ne, 3de Vande, fra Øster i Vester efter hinanden liggende: Det Østerste er langagtigt fra Øster i Vester, 1/4 Miil, halv saa bredt, deraf rinder \textit{Pors}-Elven i Vester 1/2 Bøsse-Skud lang i det mellemste \textit{Pors-jaure}, ligesaa stort og skabt som det 1te Vand, deraf fortgaaer \textit{Pors}-Elven i Vester et Par Bøsse-Skud i det Vestreste Vand, som er ligestor[t] med de forrige, derfra Elven ved Vandfald fra Eet Berg over det andet, som en iidelig Foss, nedstyrtes, 1/8 Miil lang, i \textit{Altens}Fiord, at forstaae en god 1/4 Miil i Nordvest fra \textit{Rafsbott}Elvens Mund; see \textit{pag.} 216 Ved disse \textit{Pors}-Vande er lidet smaa Bierk.\par
6. \textit{Rairo-jok}, eller \textit{Leerbotts} Elv kommer af et Vand \textit{Bojas-jaure}, og løber fra Øster i Vester 1/2 Miil lang i \textit{Altens}fiord, nemlig en god 1/2 Søe-Miil Norden for \textit{Pors}-Elvens Mund. ‒ see \textit{pag.} 216 her ‒ \textit{BojasJaure} er rundvoren, 1/8 Miil over bredt, hvorved smaa Bierk er. En 1/4 Søe-Miil Norden for \textit{Rairo-joks} Mund udløber\par
7. \textit{Skirve-jok} i \textit{Skillfiord botten} af \textit{Altens} Fiord; den rinder af et Vand, \textit{Laigis-jaure}, 1/2 Miil lang, fra Øster i Vester i bem.te \textit{Skillfiord}; \textit{Laigis-jaure} er fra Øster i Vester 1/4 Miil langt, 2 Bøsseskud bredt.\par
Fra \textit{Skillfiord} ligger i Nord 1/4 Miil \textit{Korsfiord}; derj gaaer\par
8) \textit{Korsfiord}-Elv af det Vand \textit{Fielvon-jaure} (som er paa Vestre Side af \textit{Zenusfield}) 1/4 Miil i Vester lang, i bem.te \textit{Korsfiords} Botten.\par
\textit{Fielvon-jaure} er fra Øster i Vester 1/4 Miil langt, et Par Bøsse-Skud bredt.\par
J disse \textit{pag.} 218 og 219 opreignede ferske Vande fiskes Røer.\par
Om Fielde i sidst forklarede Landstrekning Sønden for \textit{Altens}fiord og Vesten for \textit{Altens}-Elv:\par
\textit{Pag.} 215 er talt om den Bakke \textit{Sandfaldhoug}, at den fra \textit{Kongshavn}-Field strekker sig i Øster imod \textit{Alten} Elvs-Mund; 1 Miil Sønden for denne \textit{Sandfaldhoug} bestaaer Landskabet af Furre-Skoug og derefter imod tager\par
a) \textit{Skoddo-vara}, 1/2 Miil i Sør langt, og 1/8 Miil over bredt i Øster, med hvilken Østre Ende det fra \textit{Altens} Elv paa Vestre Side 1/4 Miil ligger; ovenpaa rundrygget og bart, paa Sidene noget brat, med Furre bevoxen. 1/8 Miil fra \textit{Skoddo-vara} i Sønden, og 1/2 Miil fra \textit{Altens} Elv i Vester, hvorimellem Furre-skoug er, møder\par
b) \textit{Gormo-vara}, 1/4 Miil langt i Sønden, og ligesaa bredt over; oventil fladt med Maasse og lidet Bierk paa, paa Sidene brat, og Bierke-groet, med lidet Furre i blant. 1/2 Miil i Sydvest fra \textit{Gormovara}, Vesten for Eiby-Elv, hvorimellem Bierke-Skoug er, ligger\par
c) \textit{Noppe}, strekkende sig i Vester et Par Miile, hen imod \textit{Makker-vara} og andre vilde Fielde oven for \textit{Qvænangens} Botten i \textit{Nordland}.‒ Sønden for \textit{Noppe}, og 1/2 Miil Vesten for \textit{Sielles-jaure}, findes\par
d) \textit{Nassa}, fra Nord i Sør 2 Miile langt, 1 Miil over bredt, oventil fladt, bart uden Skoug, med Maasse og lidet Græss paa, brat paa Østre og Vestre Side. Østen for \textit{Noppe}-Field, under Lit. c imellem Eyby-Elv og \textit{Altens} Elv, er\par
e) \textit{Beskads}-Field, det begynder med sin Nordre Ende der, hvor Eyby-Elv i \textit{Altens} Elv indgaaer, og strekker sig derfra i Sør langs med \textit{Altens} Elv, hvorved det ligger, 2. Miile langt, fra Vester i Øster 1 Miil bredt imellem Elvene, oventil fladt og bart uden Skoug, paa Vestre Side ad Eyby-Elv fladtvoren med Græss og Mosse paa, paa Østre Side ad \textit{Altens} Elv brat og berget. 1/4 Miil Sønden for \textit{Beskas}, hvor Bierkeskoug er, ligger\hypertarget{Schn1_65718}{}Schnitlers Protokoller V.\par
f) Volaas, imellem \textit{Aukie-jaure} i Vester og \textit{Altens} Elv i Øster, dog nær ved hint, og 1/2 Miil fra denne, fra Nord i Sør 1 Miil langt, 1 knap Miil over bredt, oventil fladt og mosset, paa Sidene nedhaldendes med Mosse og lidet Græss paa. Jmellem dette \textit{Volaas} og følgende \textit{Keuris}-Field er Bierk og Myrland med Mosse paa, af 2 Miles Strekning i Sønden:\par
g) \textit{Keuris} ligger 2 Miile Sønden for \textit{Volaas}, langs efter \textit{Masi}-Elv, fra Øster til Vester til Norden 1 1/2 Miil langt, 1/4 Miil over bredt, slet og skallet med Mosse paa, paa Søndre Side brat, paa Nordre Side fladtvoren, beggestedz med Bierk og Mosse bevoxen.\par
Fielde Østenfor \textit{Altens} Elv:\par
Om \textit{Reipas} Field er \textit{pag.} 215 før talt, at det ligger hen Jmod \textit{Altens} Elvs Munds Østre Side; Sønden for \textit{Reipas} 1/4 Miil er \textit{Kamasvara}, og derimellem Furre og Bierke-Skoug.\par
(h) \textit{Kamasvara}, 1/8 Miil Østen for \textit{Altens} Elv er fra Nord i Sør 1/8 Miil langt, 4 Bøsseskud over bredt, rundt og bart ovenpaa, brat paa Sidene med Mosse og lidet Bierk paa. 1/8 Miil lang er en Bierke skoug Sønden for \textit{Kamasvara}, og derpaa følger nær \textit{Altens} Elv\par
(i) \textit{Losemass}, vel høyt, rundvoren, nogle Bøsseskud over stort, bart med Mosse ovenpaa, fladt paa Sidene med noget Græss og Maasse paa, neden under ved \textit{Altens} Elv er noget Bierk. 1/4 Miil Sønden for \textit{Losemass} er\par
(k) \textit{Aujevara}, strekkende sig efter \textit{Altens} Elv en 6 Miile langt, i Sønden til Osten hen til \textit{Masi} Kirke, bredt fra Vester i Øster paa nordre Ende imellem \textit{Altens} Elv og \textit{Jetzjaure} 2 Miil, siden i Sør breder sig det videre ud hen til det Field \textit{Vorrieduder}, hvilket derfra ligger i Øster ved \textit{Jetz-jok} adskilt. Dette \textit{Aujevara} er ovenpaa slet og bart med Mosse paa, mens hvad derj findes dalet, er græsset, hvor Field-\textit{Finnerne} om Høsten tilholde, paa Vestre Side ned ad \textit{Altens} Elv brat, sommestedz Bierke-groet, sommestedz steenet. ‒\par
Pag: 215 f. er meldt, at nærmest omkring \textit{Altens}fiordz Botten ligge de Fielde \textit{Reipas, Borrha, Rapsniarg, Norjevara}, og \textit{Altnæss}-Field; Disse opreignede Fielde have nu Østen for sig det Field\par
1) \textit{Tseunik}, 2 Miile fra Nordvest i Sydost langt, halv saa bredt, mest slet og bart uden Skoug med Mosse paa, dog haver sommestedz Græss- og Bierk-groede Dale, hvor Field- \textit{finnerne} om Sommeren med deres Reen tilholde.\par
J disse \textit{Aujevaras} og \textit{Tseuniks, item} følgende \textit{Zenus}- og \textit{Vorrieduders} Fielde kan ingen BønderFolk boe; Thi baade er der for langt fra Fiskeriet i Søen, og for den korte Sommertid voxer der intet Korn.\par
Vider Nord er pag. 216 meldt, at nærmest \textit{Altens}-Fiord ligge \textit{Valginiarg} og \textit{Korssnæss}- field; Østen for disse Field findes nu det Field\par
m) \textit{Zenus}, fra Sør i Nord ved 2 Miile langt, og lige saa vidt i Øster; ved den Vestre Ende er det Field-fast med \textit{Valginiarg}, men imellem dette \textit{Zenus} dets Vestre Ende, hvor det er fladt, og \textit{Korsnæss} Field er en Bierke-Dal, et Par Bøsseskud viid; Oventil er dette \textit{Zenus} deels slet, deels knubret, bart, uden Skoug; og i Grubbene kan være noget Græss; Paa Østre Side er \textit{Zenus} fladvoren og Græss-groet; hvor vel \textit{Lapfinner} tilholde, men Bønder af foranførte Aarsager ej leve kan. See \textit{pag.} 261 meere.\par
Østen for dette \textit{Zenus} rinder en Elv \textit{Rautsk-jok} i \textit{Porsanger}-Fiord; og Østen for denne Elv møder\hypertarget{Schn1_66007}{}Ved Altens Fiord i VestFinmarken.\par
n) \textit{Vorrie-duder}, 1 Dags Reise, eller 3 Miile stort i Øster til imod \textit{Porsanger}-Fiord og Elv, med Berg Klimper paa, og Daler imellem; see meere herom \textit{p.} 225 Fieldet er bart uden Skoug, med lidet Mosse paa; Dalene ere deels græssede, deels steen-urede; Paa Østre Side ned ad \textit{Porsanger} heel brat.‒\hspace{1em}\par
d. 1 \textit{April}: loed og \textit{Finne} Lensmanden i \textit{Altens}botten komme med et par Kyndige Mænd, at forklare Landets Leje videre, saa meget som de kunde, af \textit{Finmarkens} faste Land i Øster og Nord:\par
\textit{Pag.} 216 her tilforn er \textit{Korsfiord} den sidste Fiord, som gaaer i det Østre faste Land af \textit{Altens} Hoved-Fiord; l/4 Miil i Nord-ost fra denne \textit{Korsfiord} ligger den Jndfiord \textit{Komagfiord}, deraf det Sydlige Næss \textit{Snuda Niarge} er det samme, som giør det Nordlige Næss af næstforrige Jndfiord \textit{Korsfiord}; og har denne \textit{Komagfiord} til Nordostlige Næss, \textit{Kraagenæss} paa \textit{Norsk}- eller \textit{Garnæss} paa \textit{Finnsk} Tungemaal, værendes derimellem i Munden 3 Bøsse-Skud viid, og ind i Landet i OstSydost imod 1/4 Miil dyb, i dens Botten en liden Bæk Østen-fra nedrinder; Med dens Mund ligger den SydØster fra \textit{Bekker-fiord} paa Øen \textit{Seiland} 1/2 Miil, hvorimellem \textit{Varge-Sund} er; nemlig imellem \textit{Seilands} Øen og det indre faste Land; \textit{Snuda-Niarge}, det NordOstlige Næss, er imellem \textit{Kors}- og \textit{Komag}fiordene i NordOst 1/4 Miil langt, halv saa bredt, slet og bart ovenpaa, paa Sidene fladtvoren, med Bierk begroet.\par
1/8 Miil fra \textit{Komagfiorden} i Nordost møder \textit{Læredsfiord}, dens Syd-vestlige Næss heder \textit{Somarset-Næss}, og henger i Sydvest med \textit{Kraagenæss} sammen; det Nord-ostlige Næss er \textit{Liniarg}; \textit{Læredsfiord} er i Munden 1/4 Miil viid, og i Øster 1/2 Miil dyb; Paa Nordre Side af denne \textit{Læredsfiord}, 3 à 4 Bøsseskud fra Gabet, gaaer en liden Jndfiord, \textit{Lill-Læreds}Fiord ind, 1/8 Miil dyb i Nordost, et Par Bøsse-skud i Gabet viid; J denne \textit{Læredsfiord} falder kun en Bæk Østenfra i Botten.\hspace{1em}\par
d. 4 \textit{April}: om Aftenen nedkom fra \textit{Kautokeino}-Fielde til \textit{Altens} Botten en Deel Field- \textit{Finner}, hvilke man havde giort Bud efter ved den \textit{Norske Missions} Skolemester; Hvorpaa man strax næste Mandag derefter, som var\hspace{1em}
\DivII[April 6.-11. Rettsmøte på Elvebakken i Alta]{April 6.-11. Rettsmøte på Elvebakken i Alta}\label{Schn1_66199}\par
\textbf{d. 6te April}: satte \textit{Examinations} Retten paa \textit{Elve-bakken} hos den \textit{Norske} Lensmand ved \textit{Altens} Elvs Mund, hvor denne udfalder i \textit{Altens}Fiordz Botten; Overværende Amtmanden over \textit{Finmarken} Hr \textit{Rasmus Kieldson}; den Kongelig \textit{Finmarkens} Foged \textit{Vedege} er herfra paa \textit{Vadsøen} over 30 Miile boendes, og Sorenskriveren var i hans Lovlig Ærende fraværendes; den \textit{Norske} Lensmand Ole Jversen mødte med de \textit{Norske} LaugRettes Mænd, saa og vare de \textit{Kautokeino}-Field-\textit{Finner}, som Vidner, tilstæde: Men \textit{Missions} Skolemesteren \textit{Niels Pedersen}, som efter de fleere Field-\textit{Finner} af \textit{Ajevara, Karasjokkj} og \textit{Juxbye} var opsendt, var ei endnu fra Fieldene tilbagekommet.\par
Efter brugte Maade tilspurdte man LaugRetten om Landets Leje og Leilighed paa den Kongelige \textit{Norske} Side forud, nemlig:\par
Sp. 1. J hvad Præstegield og Fogderie ligger denne Bøyd, og hvad Leje og Strekning den har?\hypertarget{Schn1_66294}{}Schnitlers Protokoller V.\par
\textit{Resp:} Dette Stæd, hvor Retten holdes, er et Vær, ved \textit{Alten}-Elvs Mund, paa dens Vestre Side, ligger i \textit{Altens} Præstegield, i \textit{Finmarkens} Fogderie\textit{Vaardøe}-Huus Amt; Strekningen af Præstegieldet er følgende.\par
Den yderste \textit{Finn}-Jord eller \textit{Finne}-Sæde, i Sønden paa Vestre Side er 1/2 Miil fra Botten i \textit{Langfiorden} paa dens Østre Side; den yderste Gaard er \textit{Laakkerfiorden} paa det faste Land ved \textit{Stierne}-Sund, et \textit{Nordmands}Sæde, fra forbem.te \textit{Finne}-Sæde i \textit{Langfiord} lige over Land, som meenes, 1. \textit{Finmarks} Søe-Miil i Nord beliggendes; den yderste \textit{Finne}-Sæde i Øster paa Søndre Kanten er 1 knap SøeMiil fra \textit{Alten}Elvs Mund i Sør, paa Vestre Side ved \textit{Alten}-Elv; det yderste \textit{Finne}-Sæde i Nord paa Østre Kanten er \textit{Læretz}-Fiord, hvilken siunes at ligge, efter de der boende \textit{Finners} Beretning, fra den Sønderste \textit{Finne}-Sæde ved \textit{Altens} Elv, i Nord, til Osten gode 3 Miile, men Laug-Retten meener at de ligge 3 1/2 Miile fra hinanden;\par
Længden af \textit{Altens} Præstegield fra Vester i Øster reignes saaledes: Fra den yderste \textit{Finne}-Jord i \textit{Langfiord} til det Sønderste \textit{Finne}Sæde ved \textit{Altens}-Elv i Syd-ost kan være 2 1/2 SøeMiile; Fra den Vesterste \textit{Laakkerfiord}-Gaard i Nord til den Østerste \textit{Læretz}fiord i Nord bliver 2 1/2 Miil.\par
Her er at agte, at \textit{Finne}-Jordene eller \textit{Finne}-Sæderne her i \textit{Finmarken} ei have hver sit særdeles Navn, men kaldes alleene af Fiorden, Elven eller Opsidderne. (2) F\textit{inne}-Sæderne kaldes og her i Landet af \textit{Finnerne}\textit{Finne}-bye, endskiønt det er kun 1. eller 2. Mændz Boelig.\par
J dette \textit{Altens} Præstegield er 1, navnlig \textit{Talviig}-Kirke af Træ, og et \textit{Finne-Capell} af Træ, ved \textit{Altens} Elv paa dens Østre Side, en 8. Field-Miile i Sønden fra Elv-Munden.\par
LaugRetten meener at \textit{Altens} Meenighed bestaaer hen imod 200 Mænd, hvoriblant reignes Selv-\textit{Fostrings}- eller ugifte Karle, og fattige: saa Field-\textit{Lappene} af \textit{Masi-Capell}, som Provsten \textit{Falk} har indgivet at være af \textit{Altens} Sogn ‒ ‒ 16 Mand og af \textit{Hammerfest} Sogn 2 Mænd; Lit. A.\par
Sp. 2. Hvad er Landets Beskaffenhed: Om bestaaer af Øer, eller fast Land?\par
\textit{Resp.} Præstegieldet bestaaer af Øer til deels, til deels af fast Land; det faste Land giør den Søndre og Østre, samt Vestre Land-Side; de Øer \textit{Stiernøe} og \textit{Seiland formere} den Nordre Land-Side af \textit{Altens}-Fiord; Af disse Øer hører til \textit{Altens}Gield, af \textit{Stiernøe} den Østre Side, som \textit{Rongsunds}-Vær paaligger; Thi den Søndre Side vendendes ad \textit{StierneSund} i \textit{Altens}fiord er formedelst dens bratte Forland u-beboed og u-bebyggelig; Resten af \textit{Stiernøe} hører til \textit{Hasviig-Annex} under \textit{Loppens} Præstegield; Af \textit{Seilands}Øe ligger under \textit{Altens} Gield den Søndre Ende, som bestaaer af \textit{Kufiord}, Stor- og Lille- \textit{Bekker}-fiordene; det øvrige af Øen \textit{Seiland} svarer til \textit{Hammerfæst} Sogn.\par
Desforuden er i dette \textit{Altens} Gield en Liden Øe \textit{Aarøe}, i A\textit{ltens}Fiord imellem \textit{Altenæss}, og \textit{Korsnæss}, tæt uden for den Bugt \textit{Leerbotten} beliggendes, fra Sydost i Nordvest 3/8 Miil lang, 1/8 Miil over breed, paa Søndre Side slet og Græss-landet, paa de andre Sider berged, ubeboed nu, før har den haft en Kirke og Opsiddere: men Kirken er forfløtted derfra til \textit{Tallviigen}, og Opsidderne have satt sig andenstedz, efterat Brænde-Hved paa Øen var \textit{consumered;} Nu bruges den til Slott af nærmest Boende paa faste Land.\par
\textit{Stiernøe} og \textit{Seiland} beskrives siden \textit{pag.} 247 og 251.\hypertarget{Schn1_66636}{}Ved Altens Fiord i VestFinmarken.\par
Medens dette forhandledes, ankom samme 6te \textit{April}: den \textit{Norske Missions} Skolemester, \textit{Niels} Pedersen, med Field-\textit{Lappene} fra \textit{Kautokeno-district}, af hvilke \textit{Lapper} man strax indtoeg Landskabets Beskaffenhed, i \textit{continuation} af \textit{Masi}-Capelles\textit{Norske Lappers} giordte Udsagn, Vesten for den \textit{Norske}\textit{Altens} Elv, op ad til Field-Kiølen, som herfra \textit{Altens} Fiordz Botten ligger i Sønden:\par
\textit{Pag.} 219 er før forklaret de Fielde Vesten for \textit{Altens} Elv, Norden-fra at reigne \textit{Skoddo-vara, Gormo-vara, Noppe, Nassa, etc.} og \textit{pag.} 217 er omtalt, at \textit{Masi}-Elv rinder fra Vest i Øster ved det \textit{Norske Masi-Capell} i \textit{Altens} Elv, 8. Field-Miile Sønden for \textit{Altens}fiordz Botten; Nu følger videre: Tæt Sønden for \textit{Masi}-Elv, nær Vesten for \textit{Altens} Elv er\par
\textit{Biggie-vara}, et Field fra Nord i Sør 1 1/4 Miil langt, fra Vester i Øster 1/2 Miil bredt, ovenpaa slet og mosset, neden under ved Sidene med Bierk og noget Græss bevoxen. 1/2 Miil Sønden for \textit{Biggie-vara} er\par
\textit{Volla-vara}, tæt Vesten for \textit{Altens} Elv, fra Nord i Sør 3/4 Miil langt, fra V. i Øster 1/4 Miil bredt, ligesaa skabt, som næst forrige \textit{Biggie-vara}. Efter en Bierkedal med noget Græss i, 3/4 Miil viid, følger i Sør\par
\textit{Rassekaldo}, tæt Vestenfor \textit{Altens} Elv, fra Nord i Sør 1/2 Miil langt, 1/4 Miil bredt, rundvoren oventil, bart, med Mosse paa, neden paa Sidene græsset og Bierke-groet. Med dette \textit{Rassekaldo} henger i Sør sammen\par
\textit{Lotse-vara}, fra Nord i Sør 1/2 Miil langt, halv saa bredt, oventil slet, bart med Mosse paa, paa Sidene noget Bierk, men intet Græss er.\par
Herpaa i Sør følger en Bierke-dal med Sinne-Græss i (hvilket er et langt grovt Græss, som \textit{Finnerne} bruge at lægge i Skoene) 2/4 Miil stor, med Myrland. Paa denne Dal følger i Sør det Field\par
\textit{Meronnonis}, kort 1 Miil fra Nord i Sør langt, 1/4 Miil bredt, oventil slet, bart og mosset, brat paa Sidene, derneden under Bierk, og noget Græss. Sønden for \textit{Meronnonis} er Bierkeskoug og Myrland, 1 Miil viid. Sønden for denne Bierke-skoug er en Aae \textit{Skiabardas-jok}, før \textit{pag.} 217 beskreven.\par
Fra \textit{Skiabardasjok} i Sør til Osten til den Svenske \textit{Kautokeino-Lappe} Kirke er 1 1/4 Miil, og Landet derimellem myret med Bierke-Kratt i.\par
At gaae nu Vesten for disse Fielde tilbage i Nord til \textit{Nassa}-Field, \textit{pag.} 219 forklaret; Saa ligger Sønden for \textit{Nassa} tæt derved, det Field\par
\textit{Jekus}, som strekker sig ved en Tang i Øster hen ad \textit{Keuris}-Field; dette \textit{Jekus} er fra Nord i Sør 1 Miil langt, og ligesaa bredt, og henger i Sør sammen med\par
\textit{Dasko-vara}, hvilket med andre vedhengende Fielde strekker sig i Sør en 5 à 6. Miile hen imod \textit{Meronnonis} Vestre Ende; begge, nemlig \textit{Jekus} og \textit{Dasko-vara} ere slette ovenpaa, bare uden skoug, med Mosse paa, neden under ved Sidene er noget Bierke-Kratt, men ingen Græss. Paa Sidene steilt og steenet, 1 Miil meer og mindre over bredt.\par
Fra \textit{Daskovaras} i Sør er en Field-dal, over 1 Miil langt hen til \textit{Skiabardas-jaure} i Sør, med noget Bierk og lidet Græss i. ‒\par
Sønden for denne Field-dal er det Vand \textit{Skiabardasjaure}, før \textit{pag.} 217 beskreven. Sønden for \textit{Skiabardasjaure}, imellem dette Vand, og \textit{Altens }Elv, der hvor den kommer Vesten-fra af sin Udspring ligger det Field\hypertarget{Schn1_66915}{}Schnitlers Protokoller V.\par
\textit{Ajek}, fra Vester i Øster 1 Miil langt, et Par Bøsseskud over bredt, paa den Vestre Ende, som ligger Norden for \textit{Altens }Elv, der hvor den Tvær-Aae \textit{Favros}-Elv indfalder, er dette \textit{Ajek} høyt- Tindet, bart oventil og mosset, paa Sidene brat, og mosset, neden under Bierke-groet, uden Græss.\par
At gaae nu fra \textit{Altens} Elv tilbage i Nord, hen til det forklarede Field \textit{Jekus}, saa ligger Vesten-for dette \textit{Jekus} 1 Field-Miil, og i Sydost fra \textit{Nassa}-Field 1 Bøsseskud, det Field\par
\textit{Jure}, fra Nord i Sør 2 1/2 Miil langt, 1/4 Miil over bredt, oventil slet, mosset, uden skoug og Græss, paa Sidene fladtvoren og mosset med lidet Græss paa Nordre Side; J den 1 Miils MellemRom imellem \textit{Jekus} og \textit{Jure} er Myr- og Mosse-Land uden Hved. Sønden for Jure 2 1/2 Miil er det Field\par
\textit{Siaravara}, rundt, 1/2 Miil stort, oventil sommestedz høyt, sommestedz fladt, bart med Mosse paa, paa Sidene fladtvoren, med Mosse og Græss paa; Jmellem \textit{Jure } og \textit{Siaravara} bestaaer Landskabet af Mosse Land, uden skoug og Græss. 1/2 Miil Sønden for \textit{Siaravara} er\par
\textit{Siovros}, i Sønden til Osten imod 2 Miile langt, 1/4 Miil bredt, med 2de Tinder paa, bart og mosset oven til, fladtvoren paa Sidene, og mosse-groet.\par
Fra \textit{Siovros}Field til \textit{Skiabradas}-Vand i Syd-ost bliver 1 Miil, og Landet derimellem er mosset, noget græsset, og Bierke-groet.\par
At henge nu disse \textit{Finmarkens} Fielde med de \textit{Nordlandske} tilsammen; Saa er \textit{pag.} 219 forklaret, at \textit{Noppe}-Field, det \textit{Finmarkske} strekker sig hen i Vester til \textit{Makkervara}, i \textit{Nordland} forklaret; Fra \textit{Siovros} til \textit{Rikasjaure} er i Sydost 1 Miil. Fra \textit{Jure}-Field ligger en 3 Miile i Sydvest \textit{Jorbel-jaure}, i \textit{Nordland} omtalt.\par
\textit{Fra}\textit{Kautokeino}-Kirke lige i Vester er \textit{Rikasjaure}, hvoraf \textit{Altens} Elv har sin Oprindelse, 3 Miile beliggendes; Fra \textit{Siara} er det i Sør 3 Miile. Disse nu ankomne \textit{Kautokeino}-Field-\textit{Finner} vidste nu at forklare \textit{Alten}-Elv, som de \textit{Norske Masi-Finner} fra \textit{Kautokeino} af have kun beskrevet: nemlig.\par
\textit{Altens}Elv kommer oprindeligen fra \textit{Rikas-jaure}; dette \textit{Rikas-jaure} er fra Nord i Sør 1 Miil langt, 1/4 Miil bredt; Af dets Søndre Ende rinder Aaen, som her kaldes \textit{Rikas-jok}, i Sønden 1. Miil, i det Vand, \textit{Votze-} eller, som det her udtales, \textit{Votzek-jaure}, som er rundvoren, 1/2 Miil stort, deraf Aaen naar udrinder, tager den det Navn \textit{Altens}-Elv, eller paa \textit{Finnsk Alatei-jok}, den kroger sig derfra ad Øster 2 Miile lang, der hvor den Aae \textit{Favros-jok} fra GrændseKiølen indgaaer, Herfra vender \textit{Altens} Elv sig i Nord til Osten, og løber saaledes 2 Miile lange hen til den \textit{Svenske Lappe}-kirke, \textit{Kautokeino}, Vesten forbi Kirken, og fortgaaer saa i Nord til Vesten, som \textit{pag.} 217 er forklaret; 16 Field-Miile i \textit{Altens}Fiordz Botten, saa den bliver i alt fra dens Oprindelse {21 1/2 Field Miile.}\par
\textit{Favros} oprinder 1/2 Miil Østen for \textit{Pitsekiulbme}, og løber 2 Mile mest i Nord i \textit{Altens} Elv.\par
At gaae nu til Beskrivelsen af Landskabet Østen for \textit{Altens} Elv, og \textit{continuere} den efter \textit{Aujevara}, saa vidt de \textit{Norske}\textit{Masi}FieldFinner \textit{pag.} 220 med deres Forklaring gienge; Saa følger paa \textit{Auje-vara} i Sør; næst ved \textit{Alten}Elvs Østre Side en Dal, 1/2 Miil breed i Sønden hen til det Field \textit{Vergnass}: Dette\par
\textit{Vergnass} ligger Østen for \textit{Altens} Elv 1/2 Miil, hvorimellem Bierk; 2 Miile langt i Sør, og en god 1/2 Miil over bredt, ovenpaa slet og mosset, uden Skoug og Græss, ligesaa paa Sidene. Paa \textit{Vergnass} følger en Dal 1/2 Miil viid, og derpaa i Sør\hypertarget{Schn1_67237}{}Ved Altens Fiord i VestFinmarken.\par
\textit{Lappo-jaure}, hvilket med sin Aae, og andre derj faldende Vande \textit{pag.} 218 før er beskreven. 1 Miil Sønden for \textit{Lappo-jaure} ligger det Field\par
\textit{Biennarove}, rundt, 1/4 Miil vidt, hvor af dets nordre Side den Aae \textit{Nappuljok} rinder 1 Miil lang i Nord i Munden af \textit{Lappo-jaure}, see \textit{pag.} 218 her tilforn. Jmellem \textit{Bienna-rove} og \textit{Altens} Elv er 1 Miil, og derimellem det Field\par
\textit{Sierradas}, fra Nordvest i Sydost 1 Miil langt, 1/4 Miil bredt, hvast og bart oventil, og mosset over alt. Nær ved \textit{Sierradas} i Syd-ost er\par
\textit{Altevara}, fladt og mosset oventil, saavel som paa Sidene, fra Nord-vest i Syd-ost 1/2 Miil langt, et Par Bøsseskud over bredt.\par
Tæt Sønden for \textit{Altevara} rinder den Aae \textit{Volgamas-jok} fra Syd-ost i Nord-vest 2 Miile lang i \textit{Altens} Elv paa dens Østre Side udaf det Vand, \textit{Lavo-jaure}, langagtigt 1/4 Miil i Syd-ost. Nær Sønden for \textit{Volgamas-jok} er\par
\textit{Bæljatz}Field, i Syd-ost efter \textit{Volgam}-Elv, 1 Miil langt, og 1/4 Miil bredt; sommestedz hvast, sommestedz slet ovenpaa, mosset allestedz, uden Skoug og Græss, men neden under til Elven er noget Bierk.\par
Fra dette \textit{Bæljatz Field} 1 1/2 Miil i Sønden hen til \textit{Kautokeino Lappe}Kirke er BierkeSkoug, Myr- og Mosse-Land. Østen for besagde \textit{Biennarove}, imellem dette \textit{Biennarove} og \textit{Kalbojaure} er det Field\par
\textit{Voskonunies}, fra Nord, nemlig fra \textit{Kalbojok}, i Sønden 2 Miile langt, 1/4 Miil bredt. 1/4 Miil i Sydost fra \textit{Voskonunies} ligger det Field\par
\textit{Daulo-garget}, imellem \textit{Kalbojaure} og \textit{Lavojaure}, 1/4 Miil langt i Sør, 1 Bøsseskud over bredt.\par
At gaae nu herfra i Nord tilbage til imod de \textit{Norske} Fiorder, saa er \textit{pag:} 220 og 221 bleven rørt om \textit{Zenus}- og \textit{Vorrieduder}-Fielde, \textit{item} om \textit{Rautsjok}; Herom gave nu de ankomne \textit{Norske}\textit{Porsangers} Field-\textit{Finner} følgende nærmere Forklaring:\par
Nær under \textit{Tseunik}-Field paa dets Østre Side ligge 2de, nemlig \textit{Rautes-jaure}-Vande, efter hinanden fra Syd-vest i Nord-ost, lige lange, begge 1 Miil udgiørendes, sammenhengendes ved en liden Aae af 2 Bøsseskudz Længde; Heraf rinder \textit{Rautes-jok} i Nordnordost først 3 Miile, siden i Øster 2 Miile lang i den \textit{Norske}\textit{Porsangers} Fiord, 1 Miil norden for Botten.\par
\textit{Vorrieduder}, forklarede samme \textit{Porsangers} Field-\textit{Finner}, at ligge strax Østen for \textit{Rautis}- Vande; dette \textit{Vorrieduder} er saa vidt, at det indtager den hele Strekning imellem \textit{Porsanger}- fiord, \textit{Rautes-jok} og \textit{Rautesjaure, Jetz-jaure}, og \textit{Jetz-jok}, \textit{Ajevare}-Markestæd, og i Øster, paa 1 1/2 Miil nær, \textit{Karasjok}-Elv, saa den i sin Nordre Ende, fra \textit{Rautesjok} i Vester til \textit{Porsanger}- fiordz Botten i Øster vel er en 2 a 3 Miile bredt: men i sin Længde fra Nord i Sør, nemlig fra \textit{Rautes-jok}, hvor den giør sit Løb i Øster ad \textit{Porsanger}-Fiord, til \textit{Ajevara} strekker det sig ved 12 Miile vidt, og fra Vester i Øster, nemlig fra \textit{Jetz-jaure} til dets Østlige Ende, paa 1 1/2 Miil nær \textit{Karas-jok}, er det 7. à 8. Miile bredt; bart og mest mosset, uden Skoug, u-bebyggeligt, see \textit{p.} 263.\par
Tæt Vesten under \textit{Vorrieduder}, 1 Miil Sønden for \textit{Rautes-jaure}, som og kan være 1. Miil Sønden for \textit{Tseunik}-Field ligger det Vand \textit{Jetzjaure}, fra \textit{Altens}Fiordz Botten i Syd-Syd-ost en 4. Miil; dette Vand er i Sydost 1. Miil langt, og næsten ligesaa bredt; deraf rinder \textit{Jetzjok} i Syd-Sydost, vesten om \textit{Vorrieduder}, 3 Miile til det Vand \textit{Sioss-jaure}, som er 3/4 Miil langt \hypertarget{Schn1_67571}{}Schnitlers Protokoller V. i Syd-ost, og halv saa bredt, herfra \textit{Jetzjok} gaaer i Sydost 3 Miile til \textit{Ajevare}, det \textit{Svenske} Tingstæd, Sønder derforbi, siden løber den derfra i Øster 3 1/2 Miile ind i \textit{Karas-jok}, i alt 8 1/2 Miil lang.\par
Tæt Sønden for næstomtalte \textit{Sioss-jaure} er\par
\textit{Siagge} Field, 2 Miile langt fra Nord i Sør til Østen, og 1 Miil bredt; ovenpaa fladt, og mosset allestedz, uden Skoug og Græss. Tæt Sønden for \textit{Siagge} er\par
\textit{Kalbo}-Field, fra Nord i Sør til Østen 2 Miile langt, hen til Mitten Østen for \textit{Kalbojaure} og 1/2 Miil over bredt, Nær Sønden for \textit{Kalbo}Field møder\par
\textit{Akkeness}, et temmelig høyt Field, rundt, 1/8 Miil stort, steenet og mosset; Fra den Østre Side af dette \textit{Akkenessfield} nedkommer den Aae \textit{Akkeness-jok}, og rinder 2 Miile i Øster i \textit{Karas-jok}. Paa \textit{Akkeness}-Field følger i Sør en Bierkedal 1/2 Miil lang; derefter i Sydost modtager\par
\textit{Neide-vara}, et rundt Field, 1/4 Miil stort, slet ovenpaa, og mosset, Strax paa \textit{Neidevara} i Syd-ost følger\par
\textit{Nullusvadda}, bart, og mosset, fra Nordvest i Sydost 1/4 Miil langt, og ligesaa bredt. Nær ved \textit{Nullusvadda} i Sydost er\par
\textit{Rovara}, rundt, 2 Bøsseskud over stort. J Sydost fra \textit{Rovara} er\par
\textit{Moddatasoive} i Sydost 1 Miil langt, halv saa bredt.\par
Østen for disse sidst opreignede Fielde fra \textit{Siagge} til \textit{Moddatasoive} er mosset og myret Land med lidet Græss, og noget Bierkeskoug samt smaa Field-Voler i, 1 1/2 Miil vidt i Øster hen til \textit{Karas-jok}, meer og mindre, ligesom Elven løber til.\hspace{1em}\par
Sp. 3. Hvad Slags Fisk og Søefugl i dette Præstegield?\hspace{1em}\par
\textit{Resp:} Haakiering (hvoraf Tran brændes nemlig af leveren) Torsk, Sej, Hveete, Hyse, Lange, Søe-Brassen, Flyndre, Auer, Steenbid, i Søen og Fiordene, \textit{item} Skate. J Elvene Øreter og Lax.\par
J de ferske Vande Røe, Siig, Øreter sommestedz.\par
Søefugl: Vilde Gæss, Ænder, Eed, Maaser af alle Slags, Skarv, Alke, Havelle, Lunder, Laugvie, Tænner, Kiel.\par
Angaaendes Silden, saa har den tilforn gaaet til, men paa et Par Aar har man ej fornommet til den; Meenige Mand har ikke Silde-Garn, men alleene Haav med en lang Stang ved, at fange den med.\hspace{1em}\par
Sp. 4. Hvad Havne?\hspace{1em}\par
\textit{Resp:} ud ad Hav-Siden \textit{FrakFiord} i \textit{Loppens} Sogn, \textit{Bersfiord ibidem}; paa begge Stæder have Skibe overvintret.\par
J \textit{Altens} Sogn, 2 Miile fra \textit{Altens}Fiordz Botten \textit{Korss}-Fiord, hvor Skibe om Vinteren have ligget. \textit{Talviig} i bem.te \textit{Altens}Fiord, god om Sommeren, men om Vinteren for Syd-vest- og Vesten-Vinde u-sikker.\par
J \textit{Hammerfæst} Sogn paa \textit{Hvaløe}\textit{Hammerfæst}-Havn, Een af de beste Vinter-Havne, dog ikke stor.\par
LaugRetten lægger til disse:\hypertarget{Schn1_67795}{}Ved Altens Fiord i Vest Finmarken.\par
\textit{Kongs}Havn i \textit{Altens}Fiord ved dens Botten, men sige derhos, at den om Vinteren er u-tryg for Driv-Jis.\hspace{1em}\par
Sp. 5. Hvad Fiorder?\hspace{1em}\par
\textit{Resp:} Fiordene er før beskrevne fra \textit{Finmarkens} Sydligste og vestlige Næss \textit{Andsnæss}, \textit{pag.} 208. 211 f. 213 f.\hspace{1em}\par
Sp. 6. Hvad Elve, Aaer, Fosse og Vande i dette Præstegield?\hspace{1em}\par
\textit{Resp:} De ere beskrevne før \textit{pag.} 212 f. 215 f. 217 f. 224-226.\hspace{1em}\par
Sp. 7. Hvad Myrer og Dale?\hspace{1em}\par
\textit{Resp:} De fornemste Dale i \textit{Altens} Gield ere \textit{Altens}Dal paa begge Sider af \textit{Altens} Elv, som strekker sig en 4 Miil i Sør efter \textit{Altens} Elv, 1/16 à 1/4 Miil breed imellem Fieldene paa begge Sider. J hvilken Dal Folk boe en 3/4 Miil vejs, og have Agere; Ovenfor at sætte sig, er vel langt fra Søen, hvor Jndvaanerne have deres meste Næring af;\par
\textit{Tver} Elv Dal, paa begge Sider af \textit{Tver}-elven, lang ved 1 l/2 Miile, breed imellem Fieldene paa begge Sider Elven 1/8‒1/4 à 1/2 Miil; Folk boe ikke her, men have deres Høe-Slotte.\par
Myr er i \textit{Altens} Skoug hist og her imellem Bergene, hvor i gode Aaringer Multebær voxer.\hspace{1em}\par
Sp. 8. Hvor Bønder-Gaarder, eller \textit{Finne}-Sæder ere? Hvad de frembringe? og hvad \textit{Creaturer} de holde?\hspace{1em}\par
\textit{Resp:} J Almindelighed boe Folk strøviis paa Fiord-Breddene af det faste Land og Øerne, \textit{item} sommestedz et Stykke op efter Elvene; J Særdeleshed i dette \textit{Altens} Sogn\par
J \textit{yttere}\textit{Laakker}Fiord 2 Normænd,\par
J \textit{Langfiorden} 5 Normænd, og 11 \textit{Finner}\par
J mellem \textit{Langnæss} og \textit{Talviig} 3 Normænd\par
J \textit{Talviig} 4 Normænd.\par
J \textit{Halsen} og \textit{Melsviig} 5 Normænd\par
J \textit{Kaafiord} 2 \textit{Finner}. J \textit{Eggeskall}2 \textit{Finner}\par
J \textit{Baasekop}1 \textit{finn}; J \textit{Torleviig} 1 \textit{finn}.\par
Ved \textit{Altens} Elv paa begge Sider boe 4 Normænd, og 16 \textit{Qvæner} og \textit{Finner}, hvoraf de sidste alle inden en 24 Aar, i den forrige Svenske Russiske Krig fra \textit{Torne Lapmark} ere nedkomne; Et Støkke fra Elven have nogle faae Norske \textit{Finner} deres Sæder; hvilke Folk selv have oprøddet deres Jorder ved \textit{Altens} Elv af \textit{Altens} Skoug.\par
J Rafsbotten sidder 1 \textit{Finn}.\par
J \textit{Leerbotten} 5 Normænd, og 3 \textit{Finner}.\par
J \textit{Skille}Fiord 4 \textit{finner}.\par
J \textit{KorsFiord}, \textit{Komag}Fiord, og \textit{Læretz}-Fiord af det faste Land sidde alleene \textit{Finner}.\par
Paa Øen \textit{Seiland} i \textit{Bekkerfiord} og \textit{Kufiord} boe 2 \textit{Normænd}, de andre \textit{Finner}, paa Fiordbreddene.\hypertarget{Schn1_68115}{}Schnitlers Protokoller V.\par
Paa Øen \textit{Stiernøe} ved \textit{Rongsund} 2 \textit{Normænd }og 4 \textit{Finner} de Øvrige af \textit{Seiland} høre til \textit{Hammerfæst}, og de øvrige af \textit{Stiernøe} til \textit{Loppens}-Gield.\par
J de 2de sidste Aarer har Smaa-Kopperne og Soot, en hidsig Sygdom, saa sterk \textit{grasseret}, at vel imod 200. Mennisker i dette Gield er bortdød.\par
Der saaes ikke her i Gieldet, uden i \textit{Altens} Elvs Dal, omtrent 3/4 Miil ind-ad, gemeenligen; Paa nogle andre Stæder er vel prøvet med Sæd, men i disse sidste Frost-Aarer er det efterladt.\par
Til \textit{Creaturer} haves Køer, Faar, nogle Stædz Jeder, Faa Søe-\textit{Finner} holde nogle tamme Reen, som de hos Field-\textit{Finnerne} have i Fieldene.\par
Hester haver Jngen, uden Amtmanden 2, Tvende Normænd hver 1. ved \textit{Altens}Botten; Provsten, Sorenskriver, og den Kiøbenhavnske Handelsmand hver 1. i \textit{Tallviigen}.\par
Qvænerne ved \textit{Altens}Elv bruge Reen og Oxe, at Køre med.\hspace{1em}\par
Sp. 9. Hvad for Skouge haves? Hvilke ere Kongens Alminding? og hvilke de \textit{Privates?} Om Saugbrug? SaltSyderj?\hspace{1em}\par
\textit{Resp.} Skoug paa Øerne er ringe bestaaende i Bierke-Kratt til Brænde-hved; Ved Søekanten er heller ikke meere. Furre-Skoug er inde i \textit{Kaafiorden} fra Botten af, 1/4 Miil lang, og ligesaa breed. Derefter er Furre fra \textit{Eggeskall} i Sydost hen til \textit{Altens} Elv 1/4 Miil viid, siden paa begge Sider af \textit{Altens} Elv i Sønden en 4 Miil lang, saa langt fra den paa Elven kan nedflodtes, 1/4 Miil breed paa begge Sider Elven, og mindre; Oven for de 4 Miile er nogen Skoug, men ikke saa duelig, som ej heller for Fossens skyld kan nedflodtes.\par
Ved \textit{Tver}-Elv er Furre, dog ikke synderlig tienlig, 1/2 Miil lang, 1/2 Miil breed.\par
Fra \textit{Tver}-elven til \textit{Trondhiems}far Elv 1/4 Miil veis er Furreskoug, og ligesaa dyb ind-ad, men det nærmeste deraf forbrugt.\par
J \textit{Raftsbotten} er Furre 3/8 Miil lang, 1/4 Miil breed. J \textit{Leerbotten} er Furre 1 Miil op til Fieldz, 1/8 Miil omtrent breed, men vanskelig at naae til, naar ingen Heste haves.\par
All denne Skoug er Alminding, hvilke Jndvaanerne betiene sig, og fornødentlig maa bruge til deres Hussers, Baaders, og Aarers Vedligeholdelse, Laxe-Bygninger i Elven; \textit{item} Jiærdseler eller Skie-gaarder, til at indhægne Agere og Engslætter med: Denne Alminding bruges af Jndvaanerne til fælledz, og har hver Gaard ej sin Skoug særdeles afdeelt.\par
Saugbrug er Een nyelig anlagt i \textit{Porss}-Elven1740. SaltSyderie haves her ej.\hspace{1em}\par
Sp. 10. Hvad Slags Vildt og \textit{Jnsecter?}\hspace{1em}\par
\textit{Resp:} Biørn, dog kun faa, Ulve, Ræver, Otere ved Søe-Kanten, Ryss-katter, Harer, Tiødere, Rüper til Fieldz; Ved SøeSiden vilde Gæss, Ænder, Eed og Maaser (Ei haves Elg, Hiorter) Vilde Reen langt op til Fieldz; \textit{item} Fiele-Frass, Maard, og Eeghorn kan dog findes noget lidet af.\par
Rotter, Slanger og Ormer ej haves: dog er en \textit{Jnsect}, graa paa Røggen, gul under Bugen, 1 Spand lang, tyk over Hovedet, som en liden Nægl, men gaaer smal ud til Stærten som en Syl, skyder sig hastig frem med Hovedet og Stærten, og har ellers 4 smaa lave Fødder, skiuler sig for Folk i gamle Træer, eller i Vand; kaldes her: Fiir-Beeninger.\par
Muus og Lemænninger gives, saa og Vaand.\hypertarget{Schn1_68306}{}Ved Altens Fiord i VestFinmarken.\par
Field-\textit{Finnerne} i \textit{Kautokeino} Fielde vide ej af Klæggene, som plage deres Reen: men om Høsten sætter sig en Slags Orm i Reenen imellem Kiødet og Huden, æder sig igiennem Huden, og seer hvid ud, randed runden om som en Orm, saa stor som en Nægl; naar den er fremkommet, falder den ned fra Reenen, paa Jorden, og døer; \textit{Finnerne} kalde den Orm Kuorm; og deraf er det, at de smaa Huller ere paa Reenskind.\hspace{1em}\par
11. Om Stæd eller Leilighed til Rødning i dette Gield?\hspace{1em}\par
\textit{Resp}: Dertil haves Leilighed i \textit{Altens}dal ved \textit{Altens}-Elv oven for der hvor nu er bygget; (2) i Tver-Elvdalen, dog er Jorden der frostnemt til Korn-Vext. Paa flerre Steder ved SøeKanten og Fiord-Bredden kunde og være Leilighed til Enge-Sletter: Men Søe-\textit{Finnerne}, hvilke som deres Forfædre lægge sig alleene og mest efter Fiskerie, bruge deres Jord til Dyrkelse ikke rettelig, i det en Deel af dem føre deres Giødning ud i Søen, eller lader den ligge uden Nytte.\hspace{1em}\par
12. Hvad Fielde de merkeligste? \hspace{1em}\par
\textit{Resp}: Fieldene ere Beskrevne før fra \textit{pag.} 207 f. 215 f. 219 f. 223 f. 225 f.\hspace{1em}\par
13. Om Told- Losse- og Lade Stæder, eller Herregaarder?\hspace{1em}\par
\textit{Resp}: Her er ingen Told- eller Kiøbstæd, ej heller Herregaard, men alleene de \textit{octroije}rede HandelsPladser hist og der i Landet.\par
Den tilstæde værende LaugRett med Lensmanden ønskede, at der maatte blive en Kiøbstæd nord i \textit{Nordlandene} enten ved \textit{Gibustad} i \textit{Senniens} Fogderie, eller i \textit{Tielsund} i \textit{Saltens} Fogderie (paa hvilket sidste Stæd det for Skibsfarten dog var beleiligst) da Een, eller fleere Mænd fra Vest\textit{Finmarken} kunde slaae sig sammen, og fare did med deres Varer efter deres Fornødenheder; dette kunde vel ikke saa beqvemmelig skee af Jndbyggerne i Ost\textit{Finmarken}, 1. fordj de have ingen Skoug til Baaders Bygning (2) derfra er en haard Søe-Vej forbi \textit{Nord-Capen} at fare med Baad: men disse Folk i Ost\textit{Finmarken} ville fra saadan ny \textit{Nordlandsk} Kiøbstæd med deres \textit{provision} paa Stædet forsiunes.\hspace{1em}\par
14. Hvorj Jndvaanernes Næring bestaaer?\hspace{1em}\par
\textit{Resp}: J Fiskerie og Enge-Slett til deres \textit{Creaturer}; Korn-Ager haves ikke i dette \textit{Altens} Gield, uden i \textit{Altens} Dal ved \textit{Altens} Elv; Med hvilken Sæd de fra \textit{Torne}-Land overkomne \textit{Qvæner} først have begyndt, hvilke Folk ere gode Jord-dyrkere.\hspace{1em}\par
15. Om \textit{Mineralia} eller \textit{Naturalia} her?\hspace{1em}\par
\textit{Resp}: J \textit{Kaafiord} paa den Nordre Fiord-Bredde, 1/4 Miil fra Gabet, have \textit{Participantere} fra Kiøbenhavn haft Skurfning efter Kaaber-malm, 1 aar, men siden forladt det. Paa \textit{Storviignæss} skal være Kalk-Berg, som Jndvaanerne ei vide at bruge.\hspace{1em}\par
16. Hvilke de nærmeste Gaarder til dette Præstegield?\hspace{1em}\par
\textit{Rs:} Af det \textit{Nordlandske}\textit{Skiervøe} Sogn er \textit{Alt}botten i Vester fra \textit{Langfiord} 3/4 Mil\par
af \textit{Finmarkens}\textit{Loppens} Sogn er \textit{Oxfiord} Vesten for \textit{Laakkerfiord} 1/2 Miil\par
af \textit{Hammerfæst} Sogn i Nord er \textit{Næverfiord} N.O. fra \textit{Læretzfiord} 1 Mil.\hypertarget{Schn1_68555}{}Schnitlers Protokoller V.\par
17. Hvad Vej herfra til \textit{Kiølen?}\hspace{1em}\par
\textit{Rs:} Jngen om Sommeren, see \textit{p}. 239 f. Sp. 33.\hspace{1em}\par
18. Hvad Field \textit{Finner} her ere?\hspace{1em}\par
\textit{Resp. Norske} Field \textit{Finner} ere 18. see Lit. A. Fælledz Field \textit{Finner} af \textit{Kautokeino}- Sogn 44. see nedenfor. Sp. 19.\par
\centerline{\textbf{1 Vidne af Finmarkens Fogderie} fra \textit{Kautokeino}-Fielde Over \textit{Altens}Fiordz Botten afhørt Field-\textit{Finn Joen Nielsen}}\par
født i \textit{Ajevara} under \textit{Kautokeino Lappe}Sogn af \textit{Norske Finn}-Forældre, som have boet i \textit{Norge} i \textit{Skierve}Sogn og derfra ere fløtted til \textit{Kautokeino}, døbt i \textit{Kautokeino Lappe}Kirke, i sin Christendom oplærdt af den \textit{Svenske} Præst og Klokker, 38 Aar gammel, gift, har 1 Barn, for 1 Maanedz Tid været til Gudz Bord i den \textit{Norske Masi Capell}, sidder ved \textit{Jokkel}Fiord, en Jndfiord af den \textit{Norske}\textit{Qvænangens} Fiord i \textit{Nordland} om Sommeren, og i \textit{Kautokeino}-Fielde om Vinteren;\hspace{1em}\par
19. Hvormange Field-\textit{Finner} høre til dette Stædz Field-\textit{district?} Som er fælledz skattende til \textit{Norge} og \textit{Sverrig?}\label{Schn1_68700} \par 
\begin{longtable}{P{0.07300613496932515\textwidth}P{0.6883435582822086\textwidth}P{0.036503067484662574\textwidth}P{0.05214723926380368\textwidth}}
 \hline\endfoot\hline\endlastfoot \textit{Resp:}\tabcellsep Til \textit{Kautokeino} Hoved-\textit{Lappe}kirke høre\tabcellsep 28\tabcellsep Mænd\\
\tabcellsep Endnu af den \textit{Lappe}bye \textit{Ajevara}\tabcellsep 10\tabcellsep ‒\\
\tabcellsep af \textit{Karasjock}\tabcellsep 6\tabcellsep ‒\\
\tabcellsep \multicolumn{3}{l}{_________}\\
\tabcellsep =\tabcellsep 44\tabcellsep Mænd\end{longtable} \par
 \hspace{1em}\par
Som skal være de Skattende; Fleere kan være, men ere fattige, og svare ei Skatt.\par
Under \textit{Kautokeino}-HovedKirke er den \textit{Svenske Arisbye-} eller, som den her udtales, \textit{Otzjockj} Kirke; hvorunder \textit{Otzjock}- og \textit{Jux}byer svare; et \textit{Annex}.\par
Den \textit{Svenske} HovedPræst til \textit{Kautokeino} skal nyde af \textit{Sverrigs} Crone et Tillæg af 30 tønder Byg, som Field-\textit{Finnerne} fra \textit{Torne} hannem maa tilføre; Den \textit{Svenske Capellan} til \textit{Otzjock} skal nyde af \textit{Sverrigs} Crone 50 rdl. Dansk-Mynt i Tillæg, og holder der en \textit{Lappe}-Skole;\par
Disse \textit{Kautokeino} Field\textit{Finner} kaldes Fælledz\textit{Finner}, fordi de svare Skatt til begge Riger.\hspace{1em}\par
20. Hvorledes disse Fælledz\textit{Finner} bruge \textit{Kautokeino}-Fielde? Om de have dem tilfælledz, eller hver sit FieldSæde afdeelt?\hspace{1em}\par
\textit{Resp}: De bruge Fieldene tilfælledz, saa og Fiske-Vandene.\hspace{1em}\par
21. Hvorvidt de Fælledz Field-\textit{Finners District} gaaer i Nord til \textit{Norges privative} ‒ eeget Land, og i Sønden til \textit{Sverrigs} eene Land?\hspace{1em}\par
\textit{Resp:} De kan ikke sætte eegentlig noget vist Markeskiell imellem det hidindtil holdte Fælledz Land, og \textit{Norges} eensidige Land; Thi de \textit{Norske Finner} fra Søe-Siden have gaaet op \hypertarget{Schn1_68889}{}1 Vidne i Finmarken. Altens Præstegield. til Fieldz til dennem i \textit{Kautokeino}-Fielde, og de fra \textit{Kautokeino} have gaaet ned til dem imod Fiordene, og saaledes er det holden fra Arildz Tid: Men det vide de, paa denne Kant, at de \textit{Finner,} som tilholde Norden for \textit{Karasjock}, \textit{Ajevara}-Lappebyer, \textit{item} Norden for \textit{Masi-Capell}, videre derfra i Sønden de Norden for \textit{Reisejaure} have ei mødt, ei heller møde for den fælledz Field-Øvrighed, til at skatte, eller der at dømmes: dog have de \textit{Kautokeino}-Field-\textit{Finner} fra Arildz Tid paa \textit{Reisens} fielde ved \textit{Reisejaure} og Elv om Sommeren haft deres Tilhold: Men de \textit{Finner} Sønden for \textit{Masi} og \textit{Ajevara} møde for Øvrigheden i \textit{Kautokeino} og \textit{Ajevara}, ligesom og de \textit{Finner} Sønden for \textit{Karasjok} møde sammestedz i \textit{Ajevara}; de \textit{Finner} af \textit{Jux}bye og \textit{Otzjok} møde i \textit{Otzjok}bye, eller \textit{Aris}bye; Paa de Stæder de betale Skatten til den \textit{Norske} og \textit{Svenske} Øvrighed, men de \textit{Svenske} øve Retten, og have deres Præster der alleene. ‒\par
Paa den anden Søndre Side ad \textit{Sverrigs} eene Land møder \textit{Landzkiølen}, hvorover de \textit{Svenske privative Enotekies Lapfinner} formeene disse \textit{Kautokeino}-fælledz\textit{Finner} at komme:\par
Dog ved det angivne Skiell imellem Fælledz Land og \textit{Norges} eeget Land vil ingenlunde forstaaes den gamle rette Landz\textit{kiøl} imellem \textit{Norge} og \textit{Sverrig}, hvilken Landz\textit{kiøl} gaaer Sønden forbi \textit{Kautokeino} ‒\hspace{1em}\par
22. Siden disse \textit{Kautokeino}- og \textit{Ajevara}-Field-Byer ligge Norden for Landz\textit{kiølen}; Fra hvad Tid denne Fælleskab derj har taget sin Begyndelse? naar de Kongelig \textit{Svenske} Betientere først have begyndt at tage Skatt? Naar de \textit{Svenske Lappe} Kirker i \textit{Kautokeino-} og \textit{Arisbye} eller \textit{Otzjock} først ere bygde?\hspace{1em}\par
\textit{Resp}. De vide ei, naar den Fælleskab har taget sin Begyndelse; men det har været saa fra Arildz Tid at den \textit{Norske} og \textit{Svenske} Øvrighed har oppebaaret Skatten. Det kan være for meere end 40. Aar siden, at \textit{Kautokeino} og \textit{Arisbye-}Kirker af de \textit{Svenske} ere bygde.\par
Den \textit{Norske Missions} Skolemester \textit{Niels Pedersen} fremtraadde, og forklarede, at i \textit{Kautokeino-}Kirke over en Dør staaer indskaaret det Aarstall 1701.\par
For den Tid har ingen Kirke der været, men vel et Forsamlings-Huus, ei heller har nogen stadig Præst der boet, men alleene en løs omreisendes Præst, som kan være det samme, som i Norge en \textit{Missionaire}, har været hos dennem. ‒\par
Ligesom Vidnerne i \textit{Tromsøen} i \textit{Protocollen 4 Volum.}II 385 have udsagt, saa fortælle disse Vidner, at have hørt, at \textit{Anud Curtelius} har været den 1te løse Præst fra \textit{Torne} som er kommet til dem; Denne efter nogle Aars Tienneste blev Præst i \textit{Øvre Torne}; og have disse Vidner ei hørt, at nogen Præst for denne \textit{Curtelius} har været i \textit{Kautokeino}.\par
Efter \textit{Curtelius} have de fornommet, har \textit{Johannes Tornberg} været der Præst i nogle Aar, som derefter er bleven forfremmed til \textit{Sierge-lugt}.\par
Efter ham er \textit{Anders Tornexsis} kommet som længe været Præst; J hvis Tid Kirkene i \textit{Kautokeino} og \textit{Arisbye} ere bygde; Han blev \textit{succedered} af \textit{Johannes Tornberg}, og denne av nu værende \textit{Johannes Junelius}.\hspace{1em}\par
23. Hvormeget de i \textit{Kautokeino} svare i Skatt til \textit{Sverrigs-Crone}, og hvormeget til \textit{Norge?} Hvad Love og Skikke undergiven?\hspace{1em}\par
\textit{Resp}. En fuld SkatteMand svarer aarlig til \textit{Sverrig} 5 \textit{Caroliner}, eller 1 Rdl. 4 Mark til \hypertarget{Schn1_69239}{}Schnitlers Protokoller V. den Svenske lavmand 28 s. til \textit{Profossen} 2 s. til Sv\textit{e}nske Præst 64 s. til Klokker 12 s.; Til den Kongel. \textit{Norske} Foged 1 Rdler. ‒\par
De dømmes af den Kongl. Svenske Øvrighed efter Svenske Lov; Lavmanden med 12 LavRettes-Mænd, som ere Field-\textit{Finner}, holder Retten; Skiftet efter en død Mand forretter Lensmanden, som er en \textit{Finn}, med 2 LaugRettesMænd.\par
J Geistlige Sager betienes de af den Svenske Præst i \textit{Kautokeino}; Naar de om Sommeren holde til ved de \textit{Norske} Fiorder, døbes deres Børn af den nærmeste \textit{Norske} Præst; Undertiden kan de komme i de \textit{Norske} Forsamlings Husse, men den \textit{Norske Missions} Betient besøger dem ej paa Fieldene i \textit{Norge}. ‒\par
Derhos sigende, de kan ei leve eller \textit{subsistere}, med mindre de tillades at komme til de \textit{Norske} Fiorder om Sommeren, hvor meere Græss er for Reenen, og mindre Mygg \textit{incommoderer} Reenen, og nogle af dem kan fiske i Fiordene, \textit{hvoraf} til den Norske Betient tiendes.\hspace{1em}\par
24. Hvilket er Grændse-Skiellet eller Landz\textit{kiølen} imellem \textit{Norge} og \textit{Sverrig}, som Han har hørt af sine Forældre, eller gamle Folk? Søndenfra?\hspace{1em}\par
\textit{Resp:} Han har hørt, Grændse-Kiølen at være\par
1. \textbf{Halde} mitt derover, som han har seet, men ikke været derpaa; han forklarer det ligesom 10de Vidne i \textit{Tromsøe} Fogderie 4 \textit{Vol:}II 352 f. Dog veed han ej at sige af de 2de Aaer, som derfra i Nord og Sønden skal rinde: dette \textit{Halde} har han seet\par
2. \textbf{Nappetiøve}, mitt der over.\par
Han har hørt, den at giøre \textit{Kiølen}, ligesaa at \textit{Kiaalemejok} rinder derfra til \textit{Norge} i \textit{Reisens} Elv: men selv ej været der.\par
3. \textbf{Samasoive}, eller \textit{Saamasoive} har han hørt det samme om, og seet det, men ej været derpaa; hvor paa Laug Grændse-skiellet herover skal gaae, har han ikke hørt.\par
4. \textbf{Vardoive} har han ligesaa hørt at giøre Landeskifte, men ej været derpaa.\par
5. \textbf{Bierte-vara} som maa være det samme, som i \textit{Tromsøen} er udtalt \textit{Siertevara}; har han hørt om, at ligge imellem de Grændse-Stæder \textit{Vardoive}, og \textit{Tenomutkie},\par
6. \textbf{Tenomutkie} har han hørt at giøre Landz\textit{kiølen}, men ikke været der.\par
7. \textbf{Korsevare} siger det samme som ved N. 6.\par
8de \textbf{Aakievare} siger det samme.\par
9de \textbf{Koskat-mutkie}, som maa være det samme, det i \textit{Tromsøe} er sagt at hede \textit{Kaakasmotkie}, har han hørt at giøre Landz\textit{kiølen}, men ei været der ‒\par
Om \textit{Sapasmaras}, i \textit{Tromsøen} angiven, vides her intet.\par
10 \textbf{Nerrevarda} har han hørt at være Grændseskiell, men ej seet det.\par
Om \textit{Supsavara} vides her intet.\par
11 \textbf{Posavara} hørt det samme om, men ej seet det.\par
12. \textbf{Urdevara}, over dets Søndre Ende har han hørt at være Grændseskiell og seet det, men ei været derpaa;\par
13. \textbf{Maseljaure} har han hørt at giøre Skiellet, men ei været der.\par
14 \textbf{Skier-oive}, som her udtales, siger det samme, som næsttilforn.\par
15. \textbf{Pitsekiulbme} (som det her \textit{pronuntieres}) har han hørt at giøre Kiølen, og at en Aae \hypertarget{Schn1_69520}{}2det Vidne i Finmarken. Altens Præstegield.\textit{Pitse-jok} rinder derfra i den \textit{Norske}\textit{Altens} Elv; Han har og faret over den; videre vidste han ikke derom, uden at det er der næsten Halv-vejs imellem \textit{Kautokeino}- og \textit{Enotekies Lappe}- kirker, dog nærmere til \textit{Kautokeino}, mod 5 Miile derfra. Dette \textit{Pitsekiulbme} seet. ‒\par
16. \textbf{Kieldevadda} fra \textit{Pitsekiulbme} 1 Miil i Øster. Har han ligeledes hørt at giøre Skiellet, og faret didover om Vinteren: men om Vandenes Fald veed ingen Beskeeden at give. Fra \textit{Kieldevadda} 1 Miil i Øster er ‒\par
17 \textbf{Salvasvadda,} som han ligeledes seet. Har han hørt at giøre Grændseskiell, men ej været der.\par
18. Om \textbf{Kierresvara} har dette Vidne intet hørt\par
19de \textbf{Seurisvara,} som her udtales \textit{Keurisvara} ‒ Har ligeledes hørt, at være \textit{kiølen}, videre veed han ikke. Dog seet det. Dette \textit{Keurisvara} er fra \textit{Salvasvadda} 1 Miil.\par
20de \textbf{Tirmesvara} fra \textit{Keurisvara} 1/2 Miil.\par
21de \textbf{Bevresmutkie,} som og her kaldes \textit{Bajas-mutkie}, har han hørt at giøre Landeskifte, og seet dem, og er fra \textit{Tirmesvara} 1/2 Miil i Øster.\par
22de \textbf{Rov-oive} ligger 1/2 Miil Østen for \textit{Bevres-mutkie}, rundagtigt og spidz oventil, ikke høyt, 1/4 Miil over stort, bart uden Græss og Skoug, mosset; Fra dette \textit{Rov-iove} paa den Nordvestlige Side rinder en Bæk, \textit{Akeness-jok} til \textit{Norge} i \textit{Karasjok}; Fra den anden Søndre Side af dette \textit{Rov-oive} gaaer en Bæk, uden Navn, i Sør ad \textit{Sverrig} i \textit{Børes-jaure}. Mitt over dette \textit{Rovoive}, meenes Grændse-\textit{linien} skulle gaae; Dette \textit{Rovoive} seet.\par
23. \textbf{Maderoive,} det han har seet. Ligger 1 Bøsseskud østen for \textit{Rov-oive}, rund-spidzagtig oventil, 1 Bøsseskud over vidt, ikke ret høyt, bart og mosset; Grændse\textit{linien} meenes at gaae mitt derover.\par
24. \textbf{Modtatas-oive,} det han seet har. Østenfor \textit{Mader-oive}, 1 Miil, langt fra Vester i Øster 1/2 Miil, ikke fuld 1/4 Miil bredt, hvor det er videst, slet ovenpaa, noget høyt, fladtvoren paa Sidene, og mosset; bart uden Græss og Skoug; Fra dette Field rinde smaa Bække til begge Sider baade til \textit{Norge} og \textit{Sverrig}, den til \textit{Norge} gaaer til \textit{Karasjok}. Mitt imellem Vandene over dette \textit{Modtatas-oive}, meenes, gaaer \textit{linien}. Landskabet imellem \textit{Modtatasoive} og \textit{Maderoive} bestaaer af Myr, BierkeKratt, Senni-Græss, og smaa Vande, hvoraf Bækkene rinde ad \textit{Norge} til \textit{Karasjok}. Derfra samles og en Bæk, og rinder i Sør ad \textit{Sverrig}; det Vand, hvoraf Bækken rinder til \textit{Norge}, heder \textit{Matek-jaure}, rundt, 1 Bøsseskud stort, hvoraf Aaen med det Navn \textit{Nullesjok} gaaer til \textit{Norge}; Det Vand hvoraf Aaen stikker i Sør til \textit{Sverrig}, heder \textit{Shomalopel;}\par
\centerline{\textbf{2. Vidne af Finmarkens Amt} fra \textit{Kautokeino} Østen for \textit{Altens-Fiords} Botten Field \textit{Finn Mikkel Aslaksen}}\par
født i \textit{Jokas-jarfs Lappe}-Sogn af \textit{Lappe}Forældre, døbt sammestedz, oplærdt i sin Christendom af hans Forældre, 60 Aar gammel, gift, har 3 Børn, sidste Paaske tid været i \textit{Kautokeino} til Gudz Bord\hspace{1em}\par
Fra 19. til 23 Sp. siger det samme, som næste 1te Vidne.\hypertarget{Schn1_69817}{}Schnitlers Protokoller V.\par
Til Sp. 24. veed Grændsegangen, og forklarer den at have hørt fra\label{Schn1_69822} \par 
\begin{longtable}{P{0.16950867052023122\textwidth}P{0.6804913294797688\textwidth}}
 \hline\endfoot\hline\endlastfoot {Alle af ham bevidnede Grændser har han seet fra N. 17. til Enden.}\\
N.\tabcellsep 10. \textbf{Nerrevarda}\\
\tabcellsep 11. \textbf{Posavara}\\
\tabcellsep 12. \textbf{Urdevara}\\
\tabcellsep 15. \textbf{Pitsekiulbme}\\
\tabcellsep 16. \textbf{Kieldevadda}\\
\tabcellsep 17. \textbf{Salvasvadda}\\
\tabcellsep 19. \textbf{Keurisvara}\\
\tabcellsep 20. \textbf{Tirmesvara}\\
\tabcellsep 21. \textbf{Bevresmutkie}\\
\tabcellsep 22. \textbf{Rovoive}\\
\tabcellsep 23. \textbf{Maderoive}\\
\tabcellsep 24. \textbf{Modtatas-oive}\end{longtable} \par
 \par
Siger det som næst forrige 1te Vidne.\par
25. \textbf{Kalkovadda.} En Slette, fra \textit{Modtatas-oive} i Øster til Sønden 1/4 Miil, Og 1/4 Miil viid, og breed, mossed, paa Søndre Side Bierk-groed; J den Østre Ende af denne Slette er et lidet Myrvand, 1 Bøsse-Skud langt fra Sør i Nord imellem 2de Berge, ganske smalt, deraf rinder en Bæk i Nord ad den \textit{Norske Karasjok;} Myrvandet i denne \textit{Kalkovadda} heder \textit{Auschesuppetok}, og af dette Vand har \textit{Karasjok} selv sin første Oprindelse, Grændse-\textit{linien} gaaer tæt Sønden forbi dette \textit{Auskesuppetok}.\par
26. \textbf{Paresoive,} eller som her udtales, \textit{Parse-oive}, ligger strax Østen for \textit{Auskesuppetok}, noget høyt, rundt, en 4 Bøsse-Skud over vidt, fladt med Mosse ovenpaa, paa Søndre Side med Bierk neden under; Vandet ved dette \textit{Pares-oive}, som Vidnerne i Tromsøe i 4 \textit{Vol:}II 384 have nævnt \textit{Landskift-Vatten}, det er det som Vidnet her kalder paa Finsk \textit{Auschesuppetok}. \textit{Limiten} gaaer mitt over det Høyeste af \textit{Paresoive}.\par
27. \textbf{Borvoive,} et Field er lige i Øster fra \textit{Parse-oive}, ved en slet mossed Dal, 1/6 Miil stor, adskilt; langt fra Vester i Øster 3/4 Miil, 1/4 Miil over bredt, ovenpaa, og paa Sidene fladtvoren, mosset og myret, uden skoug og Græss; Fra dets Østre Ende nedfalder en Bæk, som løber i Øster til Norden og \textit{formerer}\textit{Tana}-Elv; Fra den Søndre Side af dette Field rinder en Aae, navnlig \textit{Kiulam} i Sør ad \textit{Sverrig}, igiennem \textit{Beldo}-Vand i \textit{Kimi}-Elv; Landz\textit{kiølen} har han hørt, at gaae mitt over dette \textit{Borv-oive}.\par
28. \textbf{Gaiktem,} et Field Ligger fra \textit{Borv-oive} ved en kort Myr-dal adskilt, i Øster, rundvoren, nogle Bøsse-Skud stort, rundagtig opad, bart og Steen-uret over alt; Fra dets Nordre Side løber en Bæk i Nord i \textit{Gaune-jaure;} Fra den Søndre Side gaaer en Bæk ad Sønden i \textit{Kiulam}- Elv. Mitt over \textit{Gaiktem} er det, at \textit{Kiølen} gaaer.\par
29. \textbf{Raudo-oive} ligger i Øster fra Gaiktem 3/4 Miil, hvorimellem en myred og steened Dal uden Skoug er; Langt fra Sør i Nord 1 Miil, nogle Bøsse-Skud over bredt, bart og rundvoren ovenpaa, paa Sidene fladt og steenet; Paa Østre Side af dette Field ligger et Vand, \textit{Raudojaure}, 1/2 Miil, fra Sør i Nord langt, og nogle Bøsse-skud bredt, hvoraf Aaen rinder i \textit{Tana}- Elv, Østen for \textit{Østre Bosmed-jaure;} dette \textit{Raudo-jaure} er paa \textit{Raudo-oives} Østre Sides Nordre Deel; Paa samme Side dens Søndre Deel ligger et andet Vand \textit{Sperko-jaure}, noget langagtigt \hypertarget{Schn1_70118}{}3die Vidne i Finmarken. Altens Præstegield. et Par Bøsse-Skud, hvoraf Aaen rinder i Sør ad \textit{Sverrig} ind i \textit{Kimi}-Elv; Eidet imellem disse 2de Vande ligger Østen for \textit{Raudo-oive} Field ved dens Søndre Deel; Landz\textit{kiølen} gaaer over \textit{Raudo-oive}, dets Søndre Deel, hvor dette Eid støder an paa dets Østre Side; Ellers er dette Eid imellem Vandene 2 a 3 Bøsseskud bredt, med Bierke-kratt i, og steenet.\par
30. \textbf{Seidekierro,} et Field. Ligger 1/6 Miil i Øster med sin Søndre Ende fra næst bemeldte Eid, langt fra Sør i Nord 1 1/2 Miil, saa at dets nordre Ende naaer hen imod \textit{Tana}-Elv, 1 Miil bredt, ovenpaa slet og bart, paa Sidene fladt med smaa Kratt paa, og myret, hvorj \textit{Sene}Græss; Paa Søndre Side løber en liden Aae fra Fieldet; Sør ad \textit{Sverrig} i \textit{Kimi}-Elv, (uden Navn). Fra Nordre Side rinde vel smaa Bække i Nord ad \textit{Norge}, men saaledes, at de om Sommeren optørkes. \textit{Kiølen} skal gaae fra \textit{Eidet} over \textit{Seidekierros} Søndre Ende, hvor Eidet støder an paa\par
31. \textbf{Maselg-aukie,} en dyb Dal, Er tæt Østen for \textit{Seidekierro}, med Fielde paa begge, nemlig Vestre og Østre Sider, 1 BøsseSkud over fra Vester i Øster, fra Sønden i Nord 1 Miil langt, steen-ured; J denne Dal opkommer en Aae, \textit{Maselg-jok}, og stikker N:NordOst i \textit{Skiekkem-} Elv, og dermed i \textit{Tana}-Elv. 1 Steen-Kast derfra opstiger en anden Aae, og vender i Sønden ad \textit{Sverrigs}\textit{Kimi}-Elv. Det kan være 2. Miile, at \textit{Maselg-jok} løber til \textit{Skiekkem}-Elv, der hvor det er 2 Miile fra denns Jndløb, i \textit{Tana}-Elv; Eidet imellem disse 2 Aaer er Østen for \textit{Seidekierros} Søndre Deel; Hvorfor Landz\textit{kiølen} vill gaae fra \textit{Seidekierros} Søndre Deel over dette Eid, som er mosset.\par
32. \textbf{Maselg-oive,} et Field. Ligger tæt Østen for \textit{Maselg-aukie}, langt fra Vester i Øster 3/4 Miil, 1/6 Miil bredt, ovenpaa slet, paa Søndre Side fladt, paa Nordre steilt, mosset, uden Græss og Skoug derpaa, men neden under er Bierk paa begge Sider. Fra dets Østre Side rinder en Bæk af et Vand \textit{Storkal-jaure}, som fra Sør i Nord er 1/4 Miil langt, 1 Bøsseskud bredt; deraf Bækken 1/2 Miil rinder ad Nord i \textit{Maselg-jok}, og med den i \textit{Tana}-Elv; Fra Fieldets Søndre Side nedfalder [en] Bæk i Sør ad \textit{Sverrig}, (uden Navn) som med de andre endes i \textit{Kimi}-Elv, efterat have foreenet sig med \textit{Giekkiel}-Elv. Over dette \textit{Maselg-oive} til dets Søndre Side gaaer \textit{Kiølen} i Øster tæt Sønden for \textit{Storkal-jaure}.\par
33. \textbf{Laddegein-oive,} et Field. Ligger tæt Østen for \textit{Maselg-oive}, lavt, rundt, ovenpaa fladt, og paa Sidene fladtvoren, mosset uden Græss og Skoug; Dette \textit{Laddegein-oive} skal strekke sig meget vidt i Sør hen ad \textit{Kittil}, en \textit{Svensk} Bonde-Bye i \textit{Torne}-Land i NordOst hen ad \textit{Jndiager} i Nord-Nord-ost.\par
\centerline{\textbf{3die Vidne af Finmarkens Amt} fra \textit{Kautokeino} Søndenfor \textit{Altensfiordsbotten}\textit{Rasmuss Siversen}}\par
født i \textit{Ajevara}, af \textit{Norske} Søe-Finn-Forældre, døbt i \textit{Kautokeino-Lappe}Kirke, 48. Aar gammel, været i sidste PaaskeTid til Gudz Bord i \textit{Kautokeino}-Kirke, siger det samme, som forrige Vidner fra \textit{pag}. 230 til 232.\hspace{1em}\par
Grændse-gangen bevidner han ligesom næste 2det Vidne, fra N. 10. \textit{Nerrevarda}, til N. 33. \textit{Laddegein-oive inclusive}; Som han alle har seet. Ellers er denne \textit{Rasmuss Siversen Norsk} beskikked \textit{Finne} Lensmand i \textit{Kautokeino}, gift, har 7 Børn.\hypertarget{Schn1_70394}{}Schnitlers Protokoller V.\par
\centerline{\textbf{4de Vidne af Finmarkens Amt} fra \textit{Kautokeino} Sønden for \textit{Altens}Fiordz Botten\textit{Ole Olsen}}\par
født i \textit{Kautokeino} af \textit{Finne}Forældre, døbt sammestedz, 30. Aar gammel, gift, har 2 Børn, været ved sidste Helge-miss-Tid til Gudz Bord i \textit{Kautokeino}, udsiger ligesom forrige Vidner det fra \textit{pag}. 230 til 232.\hspace{1em}\par
Grændsen bevidner han imellem \textit{Norge} og \textit{Sverrig} at have hørt følgende at giøre: \textit{N}. 15. \textbf{Pitsekiulbme} 16. \textbf{Kieldevadda} 17. \textbf{Salvasvadda} 19. \textbf{Keurisvara.} ‒ det han har seet alleene. 20. \textbf{Tirmesvara} 21. \textbf{Bevresmutkie} 23. \textbf{Maderoive} 24. \textbf{Modtatas-oive} 30. \textbf{Seidekierro} 31. \textbf{Maselgaukie} 32. \textbf{Maselgoive}\par
\centerline{\textbf{5te Vidne af Finmarkens Amt} fra \textit{Norske Porsangers} Fielde Vesten for den \textit{Norske Porsangerfiord}, af den \textit{Norske}\textit{Kiestram Annex}-Kirke i \textit{Kielviig} Præstegield\textit{Norsk} Field-\textit{Finn Peder Joxsen}}\par
født i \textit{Porsangers Norske} Fielde paa Fiordens Østre Side, døbt sammestedz, 50. Aar gammel, gift, har 3. Børn; ved MikkelsmissTid været til Gudz Bord i \textit{Kiestram} Kirke, er nu \textit{Norsk Finne}Lensmand i \textit{Porsanger }Fielde. vidner følgende Grændsegang mellem \textit{Norge} og \textit{Sverrig} at have hørt:\hspace{1em}\par
\textit{N}. 19. \textbf{Keurisvara} 20. \textbf{Tirmesvara} 21. \textbf{Bevresmutkie} 24. \textbf{Modtatas-oive} 26. \textbf{Paresoive} 30. \textbf{Seidekierro} 31. \textbf{Maselgaukie} 32. \textbf{Maselg-oive} 33. \textbf{Laddegein oive}\par
Af disse Fielde har Vidnet alleene seet \textit{Laddegein-oive}, hvilket Field er slet, mosset, og saa vidt, baade i Sør og Øster, at man ej kan oversee det.\hypertarget{Schn1_70612}{}6te, 7de og 8de Vidne i Finmarken. Altens Præstegield.\par
\centerline{\textbf{6te Vidne af Finmarkens Amt} fra de \textit{Norske}\textit{Porsangers} Fielde Vesten for \textit{Porsangers} Fiord, hørende under den \textit{Norske Annex}Kirke i \textit{Kielviig} Præstegield\textit{Norsk} Field-\textit{Finn Hendrik Povlsen}}\par
født i \textit{Sverrigs}\textit{Enotekies Lappe} Sogn af \textit{Lappe}Forældre, døbt i \textit{Enotekies} Kirke, 48. Aar gammel, gift, har 5 Børn, været til Gudz Bord ved MikkelsmissTider, kommer i alt over Eet med næstforrige 5te Vidne, og har seet \textit{Modtatas-oive, Pares-oive, Seidekierro}, og de øfrige til Ende.\par
\centerline{\textbf{7de Vidne af Finmarkens Amt} sammestedz fra, som næstforrige, \textit{Norsk} Field\textit{Finn Ole Mortensen}}\par
født i \textit{Porsanger}Fielde, sammestedz døbt, 40 aar gammel, u-gift, været til \textit{Sacramentet} ved MikkelsmissTider sidstleden, udsiger det samme om Grændsegangen, som 5te og 6te Vidner, og desforuden hørt følgende Grændse-Stæder \textit{N.} 27. \textbf{Borv-oive} 28. \textbf{Gaiktem} 29. \textbf{Raudo-oive.} og seet de Grændse-Fielde fra \textit{N.} 24. \textit{en svite} til N. 33. \textit{inclucive}.\par
\centerline{\textbf{8de Vidne af Finmarkens Amt} sammestedz fra, som næstforrige \textit{Norsk} Field\textit{Finn Niels Nielsen}}\par
født paa det \textit{Norske}\textit{Zenus}-field Sønden for \textit{Altens} Fiordz Botten, døbt i \textit{Kautokeino} Kirke, 37 Aar gammel, ugift, sidst \textit{communiceret} ved \textit{Mikkelsmiss} i \textit{Kiestrand }Kirke siger om Grændsegangen det samme, som 5te Vidne, og seet de GrændseStæder fra \textit{N.} 24. til 33. \textit{incl:} ligesom 7 Vidne.\hspace{1em}\par
25. Hvordant er Landskabet paa den Kongelige \textit{Norske} Side?\hspace{1em}\par
\textit{Resp.} Jmellem \textit{Pitsekiulbme} og \textit{Kautokeino} er Bierkeskoug, Myrland, og mosset; ligeledes imellem \textit{Rikas-jaure} og \textit{Kautokeino;} Dog ligger det Field \textit{Korgevara} imellem \textit{Pitsekiulbme} og \textit{Kotokeino}, nær ved \textit{Pitsekiulbme}, ikke høyt, 1/2 Miil vidt; Jmellem \textit{Rikasjaure} og \textit{Kautokeino} ere Fieldene i \textit{Tromsøen} forklarede. Jmellem \textit{Kalkovadda} og \textit{Kautokeino} er fieldet Land med Bierke Kratt imellem, myret og mosset; J denne \textit{Kalkovadda} er det Myrvand \textit{Auskesuppetok}, hvoraf \textit{pag.} 234 er sagt, at \textit{Karasjok} haver sin Oprindelse; Denne \textit{Karasjok} gaaer nu i Nordost {1 Miil} i det Vand \textit{Giertos-jaure}, langt i Nordost {1/4 ‒} og bredt 1 Bøsseskud, derfra i Nordost {1‒} i det Vand \textit{Bolkiasjaure}, som er langt i Nordost {1/4 ‒} og 1 Bøsseskud bredt; derfra i Nordnordost {1 ‒} i det Vand \textit{Vuotasjaure}, som i Nordnordost {1/2 ‒} langt, og 1 Bøsseskud bredt, siden i OstNordost {2 ‒}\hypertarget{Schn1_70906}{}Schnitlers Protokoller V. til det Sted, hvor den Elv \textit{Bautajok} fra Sydost indfalder; derfra løber \textit{Karasjok}{4 Miil} hvor da \textit{Jetzjok} fra Nordvest indgaaer, ved \textit{Karas-jok; Finners} Sommer-Bye; Saa i Nordost til det Sted, hvor \textit{Karasjok-Finners} Vinter-Bye er, i Nordost {2 ‒} derefter i Øster til Norden dertil, hvor \textit{Tana}-Elv fra Syd-ost indkommer {2 ‒} Efter at \textit{Karasjok} med \textit{Tana}-Elv sig har foreenet, faar den det Navn af \textit{Tana}-Elv, og løber siden meget kroged Slangeviis i Nordost {5 ‒} hvor \textit{Juxbye} paa Østre Side ligger, og den Aae \textit{Nullijok} paa Søndre Side af \textit{Jux}bye indfalder; Hvorledes denne \textit{Tana}-Elv nu videre tager sit Løb ad den \textit{Norske}\textit{Tana}-Fiord, det vidste ei Vidnerne her.\par
Den Tvær-Elv \textit{Bautajok}, som er sagt, i \textit{Karasjok} indfalder 6 Miile fra dens Udspring af \textit{Auskesuppetok}, kommer fra SydVest af det Vand \textit{Kiaukie-jaure}, og løber en 4. Miile lang i \textit{Karasjok; Kaukie-jaure} er et lidet rundt Vand, et Par Bøsseskud stort, liggendes fra \textit{Borvoive} i Nordost 1 Miil.\par
\textit{Pag.} 234 er meldt, at fra \textit{Borv-oives} Nordostlige Ende rinder en Bæk i Nord {1/4 Miil} lang i det Vand \textit{Gaune}, som fra Vester i Øster er {1/2 ‒} og 1/4 Miil bredt; Fra \textit{Gaunes} Østre Ende gaaer Aaen i Nord-ost i \textit{Bosmed}, det Vestre, {1/2 ‒} hvilket Vestre \textit{Bosmed} er i Øster {1/2 ‒} og nogle Bøsseskud bredt, derfra gaaer Aaen 1 Bøsseskud i det Østre \textit{Bosmedjaure}, som er langt {1/4 ‒} et Par Bøsseskud bredt; derfra Aaen fremløber i Nord-ost {1 ‒} der hvor \textit{Seidejok} fra Sydost indfalder; Siden stikker den i Nord-nord-ost {2 ‒} hvor \textit{Skiekkem} Sønden-fra indgaaer, Fremdeles i Nordost {4 ‒} der \textit{Jndiagers Finner} tilholde; Herfra gaaer Elven først i Nordost, siden i Nord {4 ‒} ind i \textit{Karasjok}.\par
Saasnart denne Elv er gaaet ud af Østre \textit{Bosmed}-Vand, tager den det Navn \textit{Tana}-Elv, og naar den med \textit{Karasjok} har foreenet sig, ophører det Navn \textit{Karasjok}, og Elven kaldes derefter \textit{Tana}-Elv\par
Den \textit{Tana}-Elv løber Østen for \textit{Karasjok}, ved Østre \textit{Bosmed} 1 stærk Miil; ved \textit{Seidejok} 1 1/2 Miil, ved \textit{Skiekkem} 2 Miile, ved \textit{Jndiagers Finners} Sæde 3 Miile\par
Der hvor \textit{Jndiagers Finner} sidde ved \textit{Tana}-Elv, er 1 god Dags Reise i Syd-ost til \textit{Jndiager}- Bye. ‒ Den Tvær-Elv \textit{Seide-jok} kommer fra Syd-ost af det Grændse-Field \textit{Seidekierro}, og gaaer 1 Miil lang i Nordvest i \textit{Tana}-Elv.\par
Med den TværElv \textit{Skiekkem} har det den Beskaffenhed, at den, naar den fra GrændseDalen \textit{Maselg-aukie} udgaaer, har den det Navn \textit{Maselg-jok} 7 Miile lang, indtil den naaer det Field \textit{Skiekkem}, som ligger tæt Østen derved, derfra antager Aaen det Navn \textit{Skiekkem}-Elv, og rinder 2 Miile lang i \textit{Tana}-Elv.\hspace{1em}\par
26. Hvordant er Landskab paa den Kongelig Svenske Side Sønden for \textit{Kiølen?}\hspace{1em}\par
\textit{Resp:} Der er Bierkeskoug blanded med Furre, imellem Fieldene, dog ere Fieldene ikke saa mange, eller saa høye, som her paa den \textit{Norske} Side; Efter Fieldene er mægtig stor, Granskoug hvor de ingen Ende vide paa. ‒\hypertarget{Schn1_71227}{}8de Vidne i Finmarken. Altens Præstegield.\par
27. Hvad Vej herfra til Grændsen?\hspace{1em}\par
\textit{Resp:} Om Sommeren er Vejen fra \textit{Altens} Botten til \textit{Kautokeino} ei ridendes med Hest, for de store Elve og Vande; For samme Aarsag skyld, kommer og Jngen herimellem gaaendes: Om Vinteren fares allestedz over Field og Vande med Reen. ‒\hspace{1em}\par
28. Om været nogen Tvistighed om de GrændseStæder imellem \textit{Norske} og \textit{Svenske} Undersaattere?\hspace{1em}\par
\textit{Resp.} De fælles \textit{Kautokeino-Finner} anmeldede, at der er Tvistighed imellem dennem, og de \textit{privative Svenske Enotekies} Finner, som \textit{prætendere} at gaae nord over Landz\textit{kiølen}, nemlig over \textit{Pitsekiulbme} og \textit{Salvasvadda}, og bruge de FiskeVande \textit{Rikas-} og \textit{Votze-jaure}, \textit{Tulle-jaure}, norden for \textit{Salvasvadda, item Skilkas-jaure}, 1 Miil Vesten for \textit{Tulle-jaure}, og \textit{Rigna-}Vand, et Par Bøsseskud Norden for \textit{Skilkasjaure}. ‒\hspace{1em}\par
29. Hvad Nytte, Godhed og Herligheder er ved disse Grændse-Stæder?\hspace{1em}\par
\textit{Resp}: Det Landskab er for Jngen, uden \textit{Lapfinner} beboeligt.\hspace{1em}\par
30. Om de GrændseStæder høre til nogen \textit{Privates} Gaardz Grund, eller er Alminding?\hspace{1em}\par
\textit{Resp}: De kan ikke reignes til nogen \textit{Privates} Gaard.\hspace{1em}\par
31. Hvilke ere de nærmeste Field\textit{Finner} til dette Stæd?\hspace{1em}\par
\textit{Resp:} De nærmeste paa den \textit{Norske} Side i Øster, ere de fælledz Field\textit{Finner} fra \textit{Ajevara}, som om Sommeren efter \textit{Sanct Hans} Tid komme ned til den \textit{Norske} Fiord \textit{Repperfiord} i \textit{Hammerfæst} Præstegield. De nærmeste i Øster derefter ere fælledz Qvæner, som i \textit{Julij} Maaned pleje at komme ned i den \textit{Norske}\textit{Tana}fiord; de Fælledz \textit{Finner} derefter i Øster ere de i \textit{Jux}bye og \textit{Aris}bye, eller \textit{Otzjock}, som ligeledes pleje at komme ned i den \textit{Norske}\textit{Tana}fiord i \textit{Julio} ‒\par
De nærmeste Field\textit{Finner} paa den Kongelig \textit{Norske} Side i Vester eller Sydvest herfra er Jngen \textit{privative-Norsk} i \textit{Tromsøens} Fogderie\textit{Skiervøe} Sogn: men de fælledz Field\textit{Finner} fra \textit{Kautokeino}, og de \textit{privative}-Svenske \textit{EnotekiesFinner} pleje at fare om Vaaren ned til de \textit{Norske} Fiorder \textit{Qvænangen, Reisen} og \textit{Løngen}, og benytte sig af den \textit{Districtes} Fielde og Vande; Som og de tilstæde værende Vidner fra \textit{Kautokeino}, herfra ville begive sig i Vejen videre i Sydvest til den \textit{Norske}\textit{Reisens}Fiord.\par
De nærmeste Field\textit{Finner} paa den Kongelig Svenske Side ere De \textit{ØstLapper} af \textit{Enotekies} Sogn i \textit{Torne Lapmark}.\hspace{1em}\par
32. Hvor Underholdning og Fløtning for Folk er at faae, langs efter Grændse-Kiølen?\hspace{1em}\par
\textit{Resp:} De vide ingen Raad for Underholdning, thi da ere Field\textit{Finnerne} neere ved den \textit{Norske} Søe-Side; Med Heste er ikke fremkommendes, for Moradz, Vande, Elve, steile steenurede Fielde.\hspace{1em}\par
33. Hvorlangt herfra fiord-Botten til Grændsen?\hypertarget{Schn1_71536}{}Schnitlers Protokoller V.\par
\textit{Resp:} Herfra \textit{Altens}-Botten i Sør til \textit{Kautokeino}-Kirke er 16 og derfra til \textit{Pitsekiulbme}, 5 ‒ialt 21 Miile; Og fra \textit{Altens}-Botten i Syd-Sydost til \textit{Paresoive} er en 16 Miile, om Sommeren upassable Vej baade til Foedz og med Hest, for de mange Elves skyld.\par
For Vidnerne Field-\textit{Finnerne} kundgiorde man for Retten, at, saasnart den Kongelig Amtmand fandt det beqvemmeligst, og giørligt, og dertil udgav sin Befaling, skulle Vidnerne med 2de \textit{Norske} Besigtelses Mænd fare til Fieldz, og udviise de bevidnede Grændse-Stæder; hvilke den \textit{Norske Missions} Skolemester tager Merke paa, og opskriver samme, og giver det for de Kongelige \textit{Norske} Betientere tilkiende; Siden naar de Kongelige \textit{Norske} til GrændseMaalinger \textit{Committerede} om nogle Aarer efterkomme, at lade opmaale Grændsens Gang, have de med den \textit{Norske} Lensmand og \textit{Missions} Skolemester sig u-fortøved hos dennem at indfinde, og viise dennem Grændse-Merkene, og dem videre forklare, om behøves; De andre \textit{Kautokeino-} og \textit{Ajevara} Field\textit{Finner} havde nærværende \textit{Norsk Kautokeino-Finne}-Lensmand at tilsige, at de med deres Oxe-Reen da møde hos de Kongelige Grændse-Maalere, at hielpe dennem igiennem deres \textit{District}; hvortil en 16 Mand Arbeidere vil fornødiges.\par
Til Slutning er dette at tilføye, at man hørte af Field\textit{Finnerne}, at for en 4 Aars Tid har de \textit{Soenske} i \textit{Kautokeino} holdet et Forhør med \textit{Finnerne}, og spurdt efter, hvor de og deres forfædre fra Arildz Tid have siddet, og hvorvidt de ere gaaet ned i Nord og Vester? og \textit{Finnerne}, efter aflagde Eed, skal have udsagt, at de og deres Forfædre have brugt og nyttet Fieldene og Vandene hen til den \textit{Norske Søe}: men efter Field Kiølen eller den gamle Grændse-gang imellem \textit{Norge} og \textit{Sverrig} har den Svenske Øvrighed ei spurdt.\par
Til Slutning efter Tilspørsel forklarede Vidne \textit{Rasmuss Siversen}, at fra \textit{Kautokeino} til Landz\textit{kiølen} er \textit{Pitsekiolme} det nærmeste Grændse-Stæd, ligesom og \textit{Kieldevadda} det være kan;\par
Fra \textit{Kautokeino} Kirke slutter han at være ligesaa langt, som fra \textit{Altens} Botten til det \textit{Norske Masi Capell}, (hvorimellem er en 8 Miile;) til det Grændse-Merke \textit{Kalkovadda}, hvorfra \textit{Karasjok} har sin Oprindelse i Syd-Ost‒.\par
Hvormed Retten her ved sluttet, paa \textit{Elvebakken} Oven for \textit{Altens}Fiords Botten d. 11 April. 1744.\hspace{1em}\par
\centerline{Peter Schnitler}\hspace{1em}Niels AndersenElvebakken\hspace{1em} L.S. \hspace{1em} L.S \hspace{1em}Renholt LarssenElvebakkenOle IversenElvebakken\hspace{1em} L.S. Anders EriksenElvebakken L.S.  \hypertarget{Schn1_71749}{}Ekspedisjoner og reiser.
\DivII[April 18.-juni 14. Fra Alta til Trondheim]{April 18.-juni 14. Fra Alta til Trondheim}\label{Schn1_71751}\par
Til d. 15 \textit{April}: 1744. blev en Grændse-\textit{Tabell} med Forklaring over hvert GrændseMerke imellem det \textit{Finmarkiske}\textit{Altens} Præstegield i Nord, og \textit{SverrigsTorne Lapmark} i Søer udferdiget, og Hr Amtmand \textit{Kieldson} tilstilled, med Anmodning, samme, saa snart giørligt, \textit{provisionaliter} at lade besigte og afmerke.\par
De følgende Dage \textit{expederet} min \textit{Relation} over min Forretning i \textit{Vest-Finmarken} til vedkommende høye Steder. Og som jeg fornam, at fælles Field-\textit{Finner}, ei før end ved høyeste Sommers Tid til \textit{Tana-} og andre Østen for værende Fiorder nedkomme, og man førinden Jntet der kunde udrette; Reiste jeg fra \textit{Altens} Botten\par
d. 18 \textit{Apr}: tilbage til \textit{Nordland}, at \textit{conferere} med de Geist- og Verdzlige Betientere: hvorledes Kundskaben om den bevidnede Grændse-Gang best kunde \textit{conserveres} ved \textit{Missions} Betienternes OmReise hos Øst-\textit{Lapperne} i Fieldene, og \textit{ordinaire Lappe}-Tings, som i forrige Tider brugeligt har været, deres \textit{Retablering}; J samme Henseende fôer til Amtmanden\par
d. 18 \textit{Maj}: som dette Forslag om \textit{Lappe-Tingene} bifaldt.\par
d. 14 [Junj] Anlangede i \textit{Tronhiem}, at give Hr Obriste \textit{Mangelsen Relation} om det \textit{passerede}, og af hannem nærmere at underrettes, om hvis foretages skulle.
\DivII[Juni 23.-sept. 3. Fra Trondhjem til Hammerfest (med stans for lappeting eller eksaminasjoner i Saltenbotn, Folda, Tysfjord, Ofoten, Gratangsbotn, Malangen, Lyngsbotn, Kvenangen og Loppa)]{Juni 23.-sept. 3. Fra Trondhjem til Hammerfest (med stans for lappeting eller eksaminasjoner i Saltenbotn, Folda, Tysfjord, Ofoten, Gratangsbotn, Malangen, Lyngsbotn, Kvenangen og Loppa)}\label{Schn1_71871}\par
Med \textit{Jnstrux} forsiuned, begav jeg mig atter\par
d. 23. \textit{Junj} paa Reisen Nord-ad, og vandt til Østen for \textit{Saltens} Botten\par
d. 20 \textit{Julj}: Da jeg sammestedz bivaanet de anordnede \textit{Lappe}-Ting.\par
d. 24. ‒ forrettet det samme i \textit{Folden}\par
d. 31. ‒ i \textit{Tysfiorden}\par
d. 5 \textit{August}. i \textit{Ofoden} ‒\par
d. 11 ‒ Østen for \textit{Grøtangsbotten} i \textit{Senniens} Fogderie; desforuden holdte her et særdeles Vidne-\textit{Examen} over nogle Øst-\textit{Lapper}, fordi de forrige \textit{Lappe}-Vidner for de \textit{Norske} Besigtelses Mænd vare udeblevne; hvilket Tings-Vidne jeg afsendte\par
d. 15de fra \textit{Malangen} til de Kongel. \textit{Norske} Herre[r] GrændseMaalere.\par
‒ sammestedz biværet de anordnede \textit{Lappe}-Ting\par
d. 21de i \textit{Løngens} Botten i \textit{Tromsøens} Fogderie ligeledes.\par
d. 26. s. i \textit{Qvænangens} Fiord i ligemaade.\par
Allestedz, hvor den \textit{provisionelle} Grendse-Befaring ej skeet var, erindrede jeg de Vedkommende derom.\par
d. 31 ‒ ankom til \textit{Loppen} i \textit{Vest-Finmarken};\par
d. 3 \textit{Sept}: paa \textit{Hammerfest}.\par
{Peter Schnitler.}\hspace{1em}
\DivII[Sept. 8.-11. Rettsmøter på Hammerfest]{Sept. 8.-11. Rettsmøter på Hammerfest}\label{Schn1_72044}\par
\textbf{Ao 1744. d. 8 Septembr.} blev Retten foretagen paa \textit{Hammerfest} paa \textit{Hvaløe} i \textit{Hammerfest} Sogn i \textit{Vest-Finmarken}, overværendes Lensmanden \textit{Jsak Olsen} og de 2de LaugRettesMænd \textit{Aanod Andersen} og \textit{Siur Mathisen}, begge Søe-\textit{Finner} i den \textit{Norske}\textit{Reppe}-Fiord under \textit{Hammerfest} Præstegield; Den Kongel. \textit{Norske} Fogd var for langt borte i \textit{Øst-Finmarken}, og Sorenskriveren ligeledes langt fraværendes; Den \textit{Norske Missions} Skolemester \textit{Hendrik Gams} var tilskikked \textit{ordre}, at fremkalde \textit{FieldLapper}, og var derfor ikke tilstæde; \textit{Examinations}\hypertarget{Schn1_72120}{}Schnitlers Protokoller V. Retten blev da holden med den Field\textit{Finn Niels Andersen}, og efterat Eeden for hannem af Tolken Erik \textit{Hælset} var forklared, aflagde han sin \textit{Corporlig} Eed, som Vidne, at sige sin Sandhed om hvis hannem angaaendes Grændse-Gangen imellem \textit{Norge} og \textit{Sverrig} paa denne Kant var bekiendt:\par
\centerline{\textbf{9de Vidne i Finmarken}\textit{Niels Andersen, Fieldfinn}}\par
er født paa de \textit{Norske} Fielde strax oven for den \textit{Norske Næverfiord} ved \textit{Hvalsund} i \textit{Vest-Finmarken}; hans Fader har været en \textit{Norsk} Søe-Finn, men hans Moder en Field-Finn, døbt i den \textit{Norske}\textit{Hammerfest} Kirke, 29 Aar gammel, u-gift, han og hans Fader holder med deres Reen om Sommeren til i \textit{Refsbotten}, en Fiord ind i det faste Land, Østen for denne \textit{Hvaløe}; Om Vinteren holder han med sin Moder til ved \textit{Karasjok}, der havendes deres Reen, derimod hans Fader stadig bliver i \textit{Norge}, som \textit{Norsk} Søe-Finn; Vidnet og hans Moder betale ingen Skatt til \textit{Sverrig}, fordj de ligge ved \textit{Karasjok}, paa dens Søndre Side, hvilket maa komme deraf, at Faderen er en \textit{Norsk} Søe-Finn. Gaaer baade i \textit{Norge} og \textit{Sverrig} til Gudz Bord.\par
Dette Vidne veed ei fleere Grændse-Merker af at sige, end i forrige Orden\par
det 24de \textbf{Modtatas-oive} hvor han om Vinteren, men ikke om Sommeren har været; Han beskriver det at være slett ovenpaa, men paa den Søndre Ende høyere, end paa den øvrige Deel, fladtvoren paa Sidene; han meener det at være fra Søer i Nord 1/8 Miil langt, og omtrent 2 Bøsseskud bredt; Dette \textit{Modtatas-oive} har han hørt at giøre Grændse-skiellet imellem \textit{Norge} og \textit{Sverrig}: men paa hvad Sted eegentlig \textit{Linien} skal trækkes, det veed han ikke. Hvorlangt dette \textit{Modtatas-oive} skal ligge fra \textit{Karasjok} Markested, kan han ikke eegentlig sige: Dog gietter han til, at det skal være en 12 korte Field-Miile derimellem.\hspace{1em}\par
25de \textbf{Auschesuppetok} siger han at være et Vand, 1 Bøsseskud langt, hvoraf \textit{Karasjok}, har sit Udspring, og, som han meener, løber i Nord-ost; Her [har] han været om Vinteren, og hørt, at Grændse-Skiellet skal gaae her Rigerne-imellem, og at en anden Elv derimod er, som rinder i Søer ad \textit{Sverrig}, dog denne Elv har han ikke seet. Videre var dette Vidne ej bekiendt, og blev \textit{dimittered}, med den Befalning, at naar han fik Bud, skulle han være reedebond, at følge med til fieldz, og udviise de bevidnede Grændse-Steder; Og Retten med ham blev slutted.\hspace{1em}\par
\textit{Hammerfest ut supra}\hspace{1em}\par
{Peter Schnitler.}\hspace{1em}\par
L. S. Aanet andersen\textit{Repperfiord} L. S. Siur MathiasenRepperfiord\hypertarget{Schn1_72318}{}10de Vidne i Finmarken. Hammerfest Præstegield.\par
Som den \textit{Norske Missions} Skolemester \textit{Hendrik Gams} indfandt sig med nogle kyndige Vidner, hvilke i forrige Tider have været Field-\textit{Finner}, men nu ere stadige \textit{Norske} Søe-\textit{Finner}, berettendes, at Field-\textit{Finnerne} vare alt til Fieldz afdragne, og nu ej kunde saavel opfindes; Saa blev Retten\hspace{1em}\par
\textbf{Ao 1744. d. 8 Septembr.} paa \textit{Hammerfest} med samme Søe-\textit{Finner reassumeret}, i overværelse af Lensmanden \textit{Jsak Olsen}, og 2de LaugRettes Mænd Arent Olsen \textit{GyFiord}, og \textit{Jens Andersen SkreiFiord}, begge af Øen \textit{Seiland; Missions} Skolemesteren \textit{Gams} var og tilstæde; J deres Paahør forklarede Tolken \textit{Hælset} Eeden af Lovbogen, og derpaa aflagde de, som om Grændsen kyndige vare, deres \textit{Corporlig} Eed:\par
\centerline{\textbf{10de Vidne i Finmarken}}\par
\textit{Anders Olsen} nu \textit{Norsk} Søe-\textit{Finn} i den \textit{Norske Repper-Fiord} inde paa Faste Land, født i den \textit{Norske}\textit{Porsanger}Fiord af \textit{Norske} Field-\textit{Finner}, døbt af den \textit{Norske Kielvigs} Præst i en Forsamlings Kield i \textit{Porsanger}, 44 Aar gammel, gift med en \textit{Norsk Finne}-Qvinde, har 5 Børn, i sine unge Aar har han været en Field-\textit{Finn}, og holdet om Vinteren til i \textit{Avjevara}, men nu i 7 Aars Tid har han som Søe-\textit{Finn} stadig boet i den \textit{Norske Repperfiord}, nu i denne Sommer været sidst til Gudz Bord i den \textit{Norske}\textit{Hammerfest} Kirke.\hspace{1em}\par
Det 1te GrændseField, Vidnet veed af er i Ordnen\par
det 24de \textbf{Modtatasoive}, som han meener, at ligge fra \textit{Kautokeino Lappe} Kirke i Øster omtrent 5 à 6 Mile, og fra \textit{Avjevara} Markested i Sør en 7 à 8 Mile, eller 2 Dagers Reise; sigendes, det kan vel være, at det er fra Vester i Øster 1/2 Miil langt, og 1/4 Miil bredt, slet ovenpaa, noget høyt, fladtvoren paa Sidene, mosset, med Bierkeskoug neden under paa Sidene; At Vande fra dette Field rinde til \textit{Norge} og \textit{Sverrig}, saaledes som 1te Vidne i \textit{Finmarken}\textit{pag.} 233 det udsagt, har han vel hørt, men ei seet, fordj han har kun faret der om Vinteren. Siger og, at Grændse-Gangen vill gaae over dette \textit{Modtatasoive}, ligesom bem.te 1te Vidne.\par
Østen for dette \textit{Modtatas-oive} er en Dal, hvis Navn han ei veed; derj ligger 1/4 Miil Østen for \textit{Modtatasoive}\par
25. \textbf{Auskesuppetok} et Myrvand, som han om Vinteren har seet at være tilfrossen, og at være lidet og smalt, deraf har han hørt, at \textit{Karasjok} skall rinde i Nord i den \textit{Norske}\textit{Tana}- Elv; Om Grændse-Løbet her siger han det samme, som 1te [skal være: 2det] Vidne. \textit{pag.} 234.\par
26. \textbf{Paresoive}, eller Parse-oive beskriver han, og bevidner Grændse-Gangen derover, ligesom 1te [skal være: 2det; slik også i det følgende nr. 27-33.] Vidne \textit{pag.} 234.\par
27 \textbf{Borvoive} meener han og, som 1te Vidne, at ligge 1/6 Miil Østen for \textit{Parse-oive}, og at det kan være 3/4 Miil langt i Øster, og 1/4 Miil bredt; Beskriver og dens Skabning, som 1te Vidne, dog har om Vinteren ei kundet seet, om det myret, eller græsset; Ellers stadfæster det samme om \textit{Tana}-Elvs Oprindelse af dette \textit{Borvoives} Østre Ende, som rinder til \textit{Norge}, og at en anden Aae opkommer af dets Søndre Side, og løber igiennem \textit{Beldo}-Vand til \textit{Kimi}- Elv i \textit{Sverrig}, det har han hørt, seet og Aaen paa den Søndre Side af \textit{Borvoive}, men Navnet paa Aaen kiender han ikke. Grændse-Skiellet har han hørt at gaae over dette \textit{Borvoive}, men\hypertarget{Schn1_72601}{}Schnitlers Protokoller V.\par
hvor paa Laug det er, det kan han ikke sige: dog slutter, at det derover maa gaae Sønden for \textit{Tana}-Elvs Udsprang.\par
28. \textbf{Gaiktem-vara} beskriver han, som 1 Vidne, \textit{pag.} 234 og Grændse-gangen derover, ligeledes: Vandene siger han og, at falde til begge Sider, dog at den Bæk fra Søndre Side ad \textit{Sverrig} stevner i Søer ad \textit{Levlan-jaure}.\par
29. \textbf{Raudo-oive} ligger Østen for \textit{Gaiktem}, men hvor langt? kan ei mindes; Dalen derimellem er steened uden Skoug: men om er myred? veed han ej; Saasom han har kun været der om Vinteren; Smalt er og dette \textit{Raudo-oive} over, men hvor langt det er fra Søer i Nord, kan ej eegentlig sige; Alt det Øvrige, hvad 1te Vidne \textit{pag.} 234 f. om dette \textit{Raudo-oive}, og Vandene der Østenfor, \textit{item} om Grændse-gangen derover har udsagt, det samme bevidner han og.\par
30. \textbf{Seide-kierro} Vidner han det samme om, saa og om Grændse-Gangen derover, som 1te Vidne \textit{pag.} 235 Han har og hørt, at Bække fra dets Nordre Side rinde til \textit{Norge}: Men om nogen Bæk fra dets Søndre stikker i Søer til \textit{Sverrig}? det veed han ikke.\par
31. \textbf{Maselg-aukie} siger han det samme om, saa og om Grændse-Gangen over Eidet i denne Dal, samt Bækkenes Afløb til begge Sider baade til \textit{Norge} og \textit{Sverrig}, som 1te Vidne \textit{pag.} 235 saasom han har ligeledes været der om Vinteren, og seet det alt.\par
32. \textbf{Maselgoive} bekræfter han det samme om, saa og om Grændse-Gangen derover, som 1te Vidne, \textit{p.} 235. Dog, som han har kun været der om Vinteren, saa kan ei sige, om der er Græss paa, eller ej; Han har og vel hørt, at af det Vand \textit{Storkaljaure} en Bæk rinder ad \textit{Norge:} men hvorhen, eller hvorledes dette Vand er dannet, har han om Vinteren ei kundet kiende, ligesom han og intet veed af den Bæk, der fra skall rinde i Søer ad \textit{Sverrig}.\par
33. \textbf{Laddegeinoive} siger han det samme om, som 1te Vidne \textit{pag.} 235 dog veed ei, om det er græsset; Siden han har kun været der om Vinteren; han har hørt, at det i Søer skal strekke sig hen imod Grændserne af \textit{Kittil} i \textit{Sverrig}, og imod Grændserne af \textit{Jndiager} i Nordost: men hvorlangt didhen? veed ej. Over det Høyeste af dette \textit{Laddegein} meener han, at Landz-\textit{kiølen} gaaer. Hvorpaa han \textit{dimitteredes}.\par
\centerline{\textbf{11te Vidne i Finmarken}\textit{Joen Mathisen}, nu en \textit{Norsk} Søe-\textit{Finn},}\par
som agter at opholde sig i Vinter i \textit{Repper}-Fiord. født i \textit{Kautokeino}-Fielde, døbt i \textit{Kautokeino-Lappe}-Kirke, 50 Aar gl., gift, har 7 Børn, hidindtil har han været Field-Finn, og opholdet sig med sine Reen om Vinteren i \textit{Avjevara-}Fielde, og om Sommeren ved Søe-Siden i \textit{Norsk Finmarken}, men vill nu med alle blive her en boendes SøeFinn i \textit{Norge} ‒ været sidste Pintzehelgen til Gudz Bord i den \textit{Norske}\textit{Talvigs} Kirke.\par
Grændse-Stederne veed, og bevidner han, ligesom næst forrige 10de Vidne, nemlig \centerline{det 24de \textbf{Modtatas-oive}, 25de \textbf{Auskesuppetok} 26de \textbf{Pares- eller Parse-oive} 27de \textbf{Borvoive} 28de \textbf{Gaiktem-vara} 29de \textbf{Raudo-oive}\hypertarget{Schn1_72844}{}11 Vidne i Finmarken. Hammerfest Præstegield. 30te \textbf{Seidekierro},} hvorom han dette veed, at en Bæk fra dets Søndre Side stikker i Søer til \textit{Sverrig}. \centerline{31te \textbf{Maselgaukie} 32te \textbf{Maselg-oive} 33te \textbf{Laddegein-oive}} hvilke alle han forklarede, som ben.te 10de Vidne.\par
Landskabet til Fieldz Norden for Landz\textit{kiølen} forklarer begge Vidner saaledes:\par
\textit{Pag:} 226 er talt, at imellem \textit{Siagge-Kalbo-} etc. til \textit{Rovara}, og \textit{Karasjok} er Landet mosset, myret med Bierk, og nogle Field-Voler i; her tillægge nu disse Vidner, at Østen for \textit{Siagge} 1 Miil er det Field \textit{Doibelskaite}, langt i Nordost 2 Mile, og 1. Miil over bredt, lauvt, dog paa Nordre Ende bakket, ellers slet, myragtigt og mosset ovenpaa, paa Nordre Side ad \textit{Jetz-jok}, og paa Østre ad \textit{Karasjok} er Furre Skoug, temmelig stor.\par
Østenfor \textit{Neide}\textit{pag.} 226 omrørt, er et lidet rundt Field \textit{Kiaardallam}, nogle Bøsseskud stort, rundvoren oven til med Mosse paa, paa Sidene Bierke-groet. Jmellem dette \textit{Kiaardallam} og \textit{Karasjok} er Bierk og Myr, 1/4 Miil vidt, og skal dette \textit{Kiaardall[a]m} ligge Norden for Grændse-Fieldet \textit{Modtatas Øive} 2 Mile, hvorimellem Myr og Bierkeskoug.\par
Berette ellers, at imellem \textit{Altens}Elv og \textit{Kalbo-Jaure} i Øster er 2 Mile, og fra dette \textit{Kalbo-jaure} til nærmest af \textit{Karasjok} knap 2 Mile.\par
Om de Svenske anlagde Marke-Steder i disse \textit{Finmarkens} Fielde gave de følgende Kundskab:\par
Om \textit{Koutokeino} Hoved-Kirke og dets \textit{Annex Otzjok} er før talt \textit{pag.} 230 her; og at de Markesteder \textit{Koutokeino, Avjevara} og \textit{Karasjok} svare til \textit{Koutokeino} Hoved-Kirke, og de Markesteder \textit{Jux}bye, eller \textit{Tana}-bye samt \textit{Otzjok sortere} under \textit{Otzjok Annex}-Kirke; Disse Markesteder ligge nu saaledes efter hinanden:\par
Fra \textit{Koutokeino} Kirke, som ligger Østen for og ved \textit{Altens} Elv er til \textit{Avjevara} Markested, den disse Vidner sige at ligge ved den Aae \textit{Jetzjok} paa dens Søndre Side, 3 1/2 Miil Vesten for den Elv \textit{Karasjok} (førend \textit{Jetzjok} løber ind i \textit{Karasjok}) i Nordost{12 Mil} fra \textit{Avjevara} til \textit{Karasjok}-Markested, hvilken ligger ved Elven \textit{Karasjok} paa dens Søndre Side, er i Nordost omtrent, {6 ‒} fra \textit{Karasjok} i Nordost til \textit{Jux}bye, eller \textit{Tana}-Bye ved \textit{Tana}-Elv paa dens Søndre Side er {4 ‒} at merke, at imellem \textit{Karasjok}Bye, og \textit{Jux}Bye løber \textit{Karasjok}-Elv og \textit{Tana}-Elv sammen, og antager siden i Nordost det Navn af \textit{Tana}-Elv.\par
Fra \textit{Jux}-Bye i Ost-Nord-ost til \textit{Otzjok} Markested, hvor \textit{Annex}Kirken staaer, ved den Tvær-Elv \textit{Otzjok} paa dens Vestre Side, 1/2 Miil i Syd-ost, førend \textit{Otzjok} i \textit{Tana}-Elv indfalder, er{4 ‒ _______ i alt 26 Mile.}\par
Meere af Field\textit{tracten} vidste nærværende Vidner ikke at give Beskeeden om.\par
Man gik da fra Field\textit{tracten} i Nord til Søe-Kanten at beskrive, efter Bøndernes og LaugRettens Beretning, og begyndte fra \textit{Finmarkens} Begyndelse i Vester, der hvor man Nordest \hypertarget{Schn1_73170}{}Schnitlers Protokoller V. i \textit{Nordland} havde endet, nemlig fra \textit{Andsnæss}, det Sydligste og Vesterste Næss av \textit{Finmarken}, hvor \textit{Loppens} Præstegield anfanger; Hvilken Underretning man paa Reisen i \textit{Loppens}-Gield havde indtaget, og for Ordenen skyld her først indføres.\par
\centerline{\textbf{Loppen}}\par
det 1te og Vestligste Præstegield i \textit{Vest-Finmarken} har 2de Kirker af Træ, begge 48. \textit{Normænd}, 5 Søe-\textit{Finner}, 2 Field\textit{Finner}.\par
1) \textit{Loppens} Hovedkirke paa \textit{Loppen}-Øe, og\par
2) \textit{Hasvig Annex} Kirke paa \textit{Sørøe}, 1 1/2 Miil i Nordost fra \textit{Loppens} Kirke liggendes.\par
\textit{Loppens} HovedSogns Søndreste Gaard i Vester er \textit{Andsnæss} paa det faste Land; Naar nu derfra fares Søevejen i Nord-ost imellem det faste Land og \textit{Loppen}-Øerne til \textit{Mylingen}, den yderste af \textit{Silden}-Tindene,{1 Miil} siden i Øster til Sønden forbi \textit{Lørdagsnæss, Ulsfiord, Nysfiord} og \textit{Klubbenæss} til Gabet af \textit{Oxfiord}{1 1/2 ‒} Fra Gabet ind i \textit{Oxfiord} i Søer imod {1 ‒} derfra, hvor \textit{Oxfiord} vender sig i Øster {1/2 ‒ _______ i alt 4 Mile}\par
Der ligger, 1/8 Miil Vesten for \textit{Oxfiords} Botten den Søndreste Gaard i Øster, \textit{Nielsstrand:}\par
Men fra \textit{Andsnæss} over Land i Ost-Syd-ost kan det være en 3 Mile imellem disse Sønderste Gaarder.\par
Den Nordreste Gaard i Vester er Gaarden paa \textit{Loppen}-Øe, fra \textit{Andsnæss} i Nordnordost {1/2 Miil} Den Nordreste Gaard i Øster er Dal, paa Østre Side af \textit{Oxfiord} tæt ved Gabet, fra \textit{Loppen} i Øster {3 Mile} Men fra \textit{Nielsstrand} ved \textit{Oxfiord}Botten lige over Land i Nordnordvest {3/4 Mil}\hspace{1em}\par
\textit{Hasvigs Annex} Kirke paa \textit{Sørøe} dens Sogn har følgende yderste Gaarder:\par
Dens Sønderste Gaard paa \textit{Sørøe} i Vester er \textit{Hasvig}, hvor Kirken staaer; J Øster paa samme \textit{Sørøe} er \textit{Grundfiords} Gaardene, fra \textit{Hasvig i} Nord-ost {sterk 1 Miil}\par
De Gaarder i \textit{EidFiord} tæt Østen derfor høre \textit{Hammerfest} Præstegield til.\par
Af \textit{Stiernøe} hører og NordVestre Deel, nemlig de 2de \textit{Nordre Stiernvog} til dette \textit{Hasvig} Sogn; den øvrige større Deel af \textit{Stiernøe} ligger til \textit{Altens} Præstegield.\par
Den Nordreste Gaard af \textit{Hasvig} i Vester er i \textit{Dynnæss}Fiord, 1/8 Miil fra Botten paa dens Vestre Side; dertil fra \textit{Hasvig} at fare, Vesten om \textit{Sørøe}, bliver {3 1/4 Mile} men lige over Land i Nord-Nord-ost en {1 1/2 Mil.}\par
Hvilken \textit{Dynnæss}Fiordz Gaard er og den Østligste i Nord af \textit{Hasvig Annex}; Thi Østen for denne begynder \textit{Hammerfest} Gield, nemlig i \textit{Sandbotten}.\par
De Øer \textit{Loppen}, og \textit{Silden}, svarende under \textit{Loppens} Gield ere her \textit{pag:} 214 før beskrevne.\hypertarget{Schn1_73511}{}Om Loppens Præstegield.\par
\centerline{\textbf{Stiernøe}}\par
hvoraf den Nordre Side til \textit{Hasvig} Sogn\textit{Loppens} Præstegield, men det Øvrige til \textit{Altens} Præstegield hører ‒ Ligger Sønden for \textit{Sørøe} 1 Miil, og Østen for \textit{Silden}-Tinden 1 god Miil, strekker sig fra Vester i Øster til Sønden imod 2 Mile lang, og med sin Søndre Side vender sig til \textit{Stiern}- Sund, mitt paa bredest, over 1/2 Miil, men paa begge Ender smalere; Paa Søndre Side af en lige Leje, klipped, bratt, forlanded, og u-beboed; Paa Nordre Side rundvoren, havendes Jndfiorder og Bønder-Gaarder; Denne \textit{Stiernøes} Vestre Ende kaldes \textit{Vestre Stiern-odde}, og dens Østre Ende \textit{Østre Stiern-odde; Vestre Stiern-odden} ligger lige i Søer imod \textit{Hasvig}- og \textit{Klubnæss} paa \textit{Stiernøe}, hvor \textit{Rognsund} begynder, i Søer lige imod \textit{Ramnæss} paa \textit{Sørøe}.\par
Strax om \textit{Vestre Stiernodden} paa Øens Nordre Side aabner sig\par
1te Fiord \textit{Vestre Stiernvog}, stikkendes i Syd-ost 1/4-Miil; J denne \textit{Vestre Stiernvogs} Botten indløber en Bæk af 2de ferske Vande fra Sydost i Nordvest; nu u-beboed.\par
Strax Østen om denne \textit{Vestre StiernvogsNæss} er en liden, neml.\par
2den \textit{Østre Stiernvog}, en knap 1/8 Miil dyb i Søer, hvor 2de Bønder boe.\par
Fra \textit{Østre Stiernvog} en god 1/4 Miil i Øster møder\par
3die \textit{Smalfiord}, et Par Bøsseskud viid i Gabet, og 1/4 Miil dyb i Søer, u-beboed.\par
Østen for \textit{Smalfiord} et Par Bøsseskud forekommer\par
4de \textit{Lillkierringfiord}, 2 Bøsse-Skud i Gabet viid, og 1/8 Miil i Søer dyb; u-beboed.\par
Østen for \textit{Lill-kieringfiord} strax om Næsset er\par
5te \textit{Stor-kierringfiord}, 1/8 Miil viid i Gabet, 1/2 Miil lang ind ad Søer, og vender sig med sin Botten i Vester; u-beboed.\par
Det Østre Næss af denne \textit{StorkierringFiord} heder \textit{Klubnæss}, og herfra begynder \textit{Rognsund}, som imellem denne \textit{Stiernøe}, og \textit{Hvaløe}, 1/8 à 1/4 Miil viidt, stikker i Syd-ost til ind i \textit{Altens} Hoved-Fiord.\par
J Syd-Ost fra denne, \textit{Stor-kierring}Fiord en god 1/4 Miil sees 2de nemlig\par
6te \textit{Nordre}, og 7de \textit{Søndre Koll-Fiorder}, ved et Mellem-Næss fra hinanden adskildte, nogle Bøsse-Skud i Gabet vide, og omtrent imod 1/4 Miil dybe; begge u-beboede.\par
Fra \textit{Søndre Kollfiord} til \textit{Østre Stiernodden} er i Syd-ost 1/2 Miil; Landet ved dette \textit{Rognsund} er fladtvoren, med noget Bierk paa, og beboed; Ellers Landet ind paa Øen er berget og u-frugtbart.\par
\centerline{\textbf{Sørøe}}\par
hvoraf den Syd-vestlige mindre Deel til \textit{Hasvigs Annex}\textit{Loppens} Gield, og den Nord-ostlige større Deel til \textit{Hammerfest} Præstegield henhører, ligger fra \textit{Stiern-Øe} 1 Miil, fra \textit{Seiland}-Øe og \textit{Hvaløe} 3/4 Miil og mindre i Nordvest, og ligesom disse sidstbenævnte Øer giøre den Søndre- saa giør \textit{Sørøe} den Nordre Side af \textit{Sør-sund}, strekkende sig fra Sydvest i Nordost.\par
\textit{Haaën}, den sydligste Odde af \textit{Sørøe}, stikker fra Øen ud i Vest-Syd-vest 1/8 Miil imellem \textit{Hasvig}, og \textit{Hasvog}, merkelig deraf, at den med \textit{Silden}-Tinden paa den anden Side derimod, begynder \textit{Altens-Fiord}, og med den \textit{Vestre Stiernodde} af \textit{Stiernøe} derimod, begynder \textit{Sør-sund}.\hypertarget{Schn1_73807}{}Schnitlers Protokoller V.\par
Fra denne \textit{Haaën} nu at anfange Beskrivelsen af \textit{Sørøe}: Saa ligger den Søndre Side af \textit{Sørøe} imod \textit{Sørsund} omtrent lige, og allvejs i Nord-ost, nemlig fra \textit{Haaën} til \textit{Akkernæring}{4 3/8 Miil} Fra \textit{Akkernæring} vender \textit{Sørøe} sig i Nord, og gaaer saaledes til \textit{Tarhals}{1/2 ‒} Fra \textit{Tarals} snoër den sig ad Vester til \textit{Soppen}{4 3/8 ‒} Fra \textit{Soppen} gaaer den i Sydost til \textit{Haaen}{1 5/8 ‒ ___________ = 10 7/8 ‒}\hspace{1em}\par
Dette nu nøyere at forklare: saa fares fra \textit{Haaën} forbi \textit{Hasvig}, siden forbi \textit{HassFiord}, hvilke begge ved et Mellem-Næss adskilles, til \textit{Meltefiord-Næss} i Nord-ost {1/2 Miil}\par
J \textit{Hasvig} staaer \textit{Annex}-Kirken under \textit{Loppens} HovedKirke, og Handel-Steden for \textit{Loppens} og \textit{Hassvig} Almuer. \textit{HassFiord} er ubeboed.\par
Fra \textit{Meltefiords-Næss}, forbi \textit{Melte-Fiord} i Nordost til dens Ostlige Næss \textit{Skallnæss}{1/4 ‒}\textit{Meltefiord} derimellem 3/8 Miil dyb i Nord, ubeboed.\par
Fra \textit{Skallnæss} forbi \textit{Kipervig} i Nord-ost til \textit{Multernæss}{1/4 ‒}\textit{Kipervig} er i Nord 1/4 Miil dyb, ubeboed. Fra \textit{Multernæss} forbi \textit{Grundfiord} i Nordost til \textit{Klasnæss}{1/16 ‒}\textit{Grund}fiord stikker i Nord 1/8 Miil ind, hvor 2 \textit{Normænd} boe. Fra \textit{Klasnæss}, forbi \textit{Eidfiord} i Nordost til \textit{Ramnæss}{7/16 ‒}\textit{Eidfiord} gaaer ind i Nord 3/8 Miil dyb; Udaf Landet stikker et, navnlig \textit{Fuglberg}, Norden fra mitt udi Fiorden, og kløver denne i Toe, saa at Fiorden gaaer ind paa begge Sider af \textit{Fuglberget:} Denne \textit{Eidfiord} har riimelig sit Navn deraf, at derfra gaaes over et Eid i Nord, 3/4 Miil langt, paa Øen til \textit{Dynnæss}Fiord, Øens Nordre Side; her boer 1 \textit{Normand}. Fra \textit{Ramnæss} forbi \textit{Kobbefiord} i Nordost til \textit{Kobbesnæss}{1/4 ‒}\par
\textit{Kobbefiord} stevner ind i Nord 1/4 Miil dyb, ubeboed. Fra \textit{Kobbesnæss} i Nordost til \textit{Gaashobøyen}{1/4 ‒} nu øde. Fra \textit{Gaashob}øyen i N.O. forbi \textit{Husfiord} og \textit{Havnfiord} til \textit{Fellen} eller \textit{Feilen}, hvor 5 Mænd boe ved {1/2 ‒}\par
\textit{Husfiord} er i Gabet 1/4 Miil viid, og ligesaa dyb i Nord, u-beboed, saavel som \textit{HavnFiord}.\par
\textit{Havnfiord} er i Gabet nogle Bøsseskud viid, og l/4 Miil i Nord dyb; Jmellem \textit{Gaashob} og \textit{Hus}fiord boer 1 \textit{Normand} paa den Jord, \textit{Vatten}\par
Fra \textit{Fellen} forbi \textit{Sletnæss}-Botten i Nord-ost til dens Ostlige Næss \textit{Sletnæss}{3/8 ‒}\par
\textit{Sletnæss}Botten er 1/8 Miil dyb, og viid, hvor ved Østre Næss 2 \textit{Normænd} boe.\par
Fra \textit{Sletnæss} i Nordost til \textit{Værfiord-Næss}, som er det Vestre \textit{Næss} af \textit{Værfiord}{3/8 ‒}\par
Det Østre Næss af \textit{Værfiord} heder \textit{Værfield}, og \textit{Værfiord} derimellem er 1/4 Miil i Nord dyb, hvor 1 \textit{Normand} boer, og i Gabet viid {1/8 ‒}\hypertarget{Schn1_74186}{}Om Loppens Præstegield.\par
Paa \textit{Værfield} følger i Nord-ost \textit{Lønhavn}, nogle Bøsseskud viid, ubeboed; dens Nord-ostlige Næss heder \textit{Pigfield}, høyt med nogen Græsslie paa;\par
Derpaa imodtager \textit{Hellefiord}, nogle Bøsseskud i Gabet viid, 1/8 Miil i Nord dyb; dens Nord-Ostlige Næss heder \textit{Lill-Naava}, med nogen Græss-Lie paa; \textit{Hellefiord} er ubeboed.\par
Fra \textit{VærField} herhid kan reignes i N.Ost for {3/8 Miil}\par
Paa \textit{Lill-Naava} følger \textit{Nord Hellefiord}, eller \textit{Skibfiord} 1/4 Miil dyb i Nord, og i Kæften viid {1/8 ‒} Dens Nordostlige Næss heder \textit{Skibnæss}. Fra \textit{Skibnæss} i Nordost til \textit{Akkerfiord} er Landskabet berget, u-dyrkeligt, og ubeboet en {3/8 ‒}\textit{Akkerfiord} er i Gabet til dens Nordostlige Næss \textit{Akkernæring} viid (Fiorden ellers ubeboed) {1/8 ‒ _______ \textit{Summen} af \textit{Sørøens} N.O. Side 4 3/8 ‒}\par
Fra \textit{Akkernæring} gaaer \textit{Sørøen} i Nord til \textit{Mæfiord} som er ubeboed nu {1/8 ‒} hvilken \textit{Mæfiord} er i Gabet 3 à 4 Bøsseskud viid, og 1/8 Miil dyb i Nord; det Nordre Næss af \textit{Mæfiord} heder \textit{Moinæss:} Fra dette \textit{Moinæss} i Nord til Vesten til det Field \textit{Støvren}, {1/8 ‒} hvorimellem Landet er berget, ubeboet.\par
Strax derved forekommer den Bugt \textit{Kiødvig}, dyb i Nord 1/8 Miil, og viid i Gabet {1/8 ‒} ubeboed; det Nordre Næss af \textit{Kiødvig} heder \textit{Hvidtnæss}.\par
Fra dette \textit{Hvidtnæss} i Nord-Nord-vest til \textit{Tarhalsen}, den Nordre Odde i Øster af \textit{Sørøe}, hvorimellem Jngen kan boe {1/8 ‒ _______} Som giør den Østre \textit{Sørøens} Side fra Søer i N. til Vesten {1/2 ‒}\par
Fra \textit{Tarhalsen} gaaer strax \textit{Bastefiord} ind 3/4 Miil dyb, og er der fra \textit{Bastefiord} til dens Vestlige Næss \textit{Finnfiord-Næss} i Vester til Sønden i Gabet viid {3/4 ‒}\textit{Bastefiord} er u-beboed.\par
Samme \textit{Finnfiord-Næss} er det Østlige Næss af \textit{Finnfiord}, ubeboed, som er til Sønden dyb 1/2 Miil, og i Gabet, viid knap {1/4 ‒}\par
Det Land imellem \textit{Finnfiord} og følgende \textit{Sandbotten} er ubeboed, heder \textit{Skarvnæring}, langt {1/4 ‒}\par
Vesten for \textit{Skarvnæring} er den Fiord \textit{Sandbotten}, 3/4 Miil dyb i Syd-ost, og i Gabet viid i Vester til Sønden, hvor 1 \textit{Normand} boer {3/4 ‒}\par
Uden for denne \textit{Sandbotten} i Gabet ligger \textit{Sandøe}, rund, 1/4 Miil omkring. Det Syd-vestlige Næss af \textit{Sandbotten} heder \textit{Staalet}; og herfra begynder \textit{Hammerfest} Præstegield paa den \textit{Sørøens} Nordre Side. Samme \textit{Staalet} giør det Østre \textit{Næss} af \textit{Dynnæss}- fiord, og er et Par Bøsseskud stort, høyt, og bratt ad Søen, bart og u-dyrkeligt.\par
\textit{Dynnæss}Fiord er i Gabet viid {3/4 ‒} samme \textit{Dynnæss}Fiord stikker ind i Søer 1 Miil dyb, og har ved dens Botten 3 smaa Bugter eller Poller, navnlig \textit{Søerkios, Nordkios}, og derimellem \textit{Børekios}; Ved \textit{Søerkios}\hypertarget{Schn1_74503}{}Schnitlers Protokoller V. boe 2de \textit{Normænd}, og dermed begynder \textit{Hasvigs Annex} Sogn paa \textit{Sørøens} Nordre Side. Denne \textit{Dynnefiords} Syd-vestlige Næss heder \textit{Karhals}, nogle Bøsseskud stort; Fra \textit{Karhals} i Vester forbi \textit{Bøle-Fiord} til \textit{Blodskiden-Næss}{3/8 Miil}\par
\textit{Bølefiord} er 1/8 Miil dyb i Søer, ubeboed. Herfra kommer man til den Fiord, \textit{Ofodbotten}, dyb 1/4 Miil, og i Gabet viid {1/4 ‒}\par
\textit{Ofodbottens} Østre Næss heder \textit{Steensnæss} hvor 2 \textit{Normænd} boe, dens Vestre Næss er \textit{Ofodnæring}, derfra i Syd-vest til det Field \textit{Soppen} er Landet ubeboet, og vidt imod {1 ‒ _______}\par
\textit{Summa} den Nordre Side af \textit{Sørøe}{4 3/8 ‒}\hspace{1em}\par
Fra \textit{Soppen}-Field bøyer \textit{Sørøen} sig i Syd-ost nemlig til det Field \textit{Andod}{1/2 ‒} hvorimellem er \textit{Sørvær}, en u-dyb Bugt, hvor 4 \textit{Normænd} boe, efter Landet langagtig bem.te 1/2 Miil.\par
Fra \textit{Andod} fremdeles i Syd-ost til \textit{Ramnæss}, det Nordvestlige Næss af \textit{Brevig}, hvilket MellemRom er ubeboet, {1/8 ‒} Fra \textit{Ramnæss} forbi \textit{Brevig} i Sydost til \textit{Haaën}{1 ‒ _______}\par
\textit{Summen} af \textit{Sørøens} Sydvestl. Side {1 5/8 ‒}\par
\textit{Sørøens} Vidde runden omkring {= 10 7/8 Mil}\par
Denne \textit{Sørøe} er 2 Mile og mindre over breed, Landet inde paa Øen berget, u-frugtbart og øde; Folk, og det kun faa, boe strøviis ved Søe-Sidene; Hvor af \textit{Loppens} Præstegieldz\textit{Hasvig Annex}Sogn indehaver den Vestlige mindre Deel, nemlig til i \textit{Grundfiord} paa Søndre- og til i \textit{Dynnæss}-Fiord paa Nordre Side; Den øvrige Østlige og større Deel af \textit{Sørøen sorterer} under \textit{Hammerfest} Præstegield.\par
Om forbem.te \textit{Brevig} er dette i sær at tilføye, at den er et almindelig Fiske-Vær, 3/4 Miil vidt, og 1/2 Miil dyb, havendes 3 \textit{Normænd} til Jndbyggere; Dens Nordvestlige Næss heder \textit{Ramnæss}, dens Sydostlige Næss \textit{Knott}; hvilket \textit{Knott} skiller \textit{Brevig} fra \textit{Hasvog}, en Bugt; denne \textit{Hasvog} er et par Bøsseskud i Gabet viid, og 1 Steenkast dyb, beboed af 2. \textit{Normænd}, og sluttes i Sydost ved den Odde \textit{Haaën}, hvorfra man \textit{pag.} 247 har begyndt \textit{Sørøens} Beskrivelse.\par
\centerline{\textbf{ Om FinmarkensHammerfest Præstegield.}}\par
At følge nu Ordenen, som begyndt er \textit{pag.} 221 saa bliver til\par
1 Spørsm: svaret: Dette \textit{Hammerfest} Sted ligger paa \textit{Hvaløe} ved \textit{Hammerfest-Vaagen} i \textit{Hammerfest} Præstegield i \textit{Finmarkens} Amt, dets Vestre Deel. Strekningen af dette \textit{Ham}merfest-Præstegield er følgende:\par
Den Søndreste Gaard i Vester af \textit{Hammerfest}-Gield er \textit{Borfiord} paa \textit{Seiland}-Øe; Den Nordeste i Vester heder \textit{Sandøe}-Botten, Sønden for \textit{Staalet} paa \textit{Sørøe};\par
Den Sønderste Gaard i Øster paa det faste Land er \textit{Næverfiord}; Den Nordeste Gaard paa faste Land er \textit{Repperfiord-Klubben}.\hypertarget{Schn1_74842}{}Om Hammerfest Præstegield.\par
Jmellem \textit{Borfiord} og \textit{Sandøe-Botten} kan være 2 Mile i Nord;\par
Jmellem \textit{Næverf}[\textit{ord} og \textit{Repperfiord-Klubben} i Nord-ost lige over 1 Miil;\par
Jmellem \textit{Borfiord} og \textit{Næverfiord} lige over i Øster vill blive gode 2 1/2 Mile.\par
Jmellem \textit{Sandøe-Botten} og \textit{Repperfiord-Klubben} lige i Øster er gode 3 Mile.\hspace{1em}\par
2 Sp: Svar: \textit{Hammerfest} Gield bestaaer deels af Øer deels af fast Land:\par
Af Øer har det heele \textit{Hvaløe};\par
Af \textit{Sørøe} den Nord-ostlige Deel, nemlig til \textit{Gaashob} paa dens Søndre Side, og til \textit{Staalet} paa dens Nordre Side strax Østen for Dynnæss-Fiord, begge \textit{inclusive}; Thi \textit{GrundFiord} paa \textit{Sørøens} Søndre Side, og \textit{Dynnæss}-Fiord paa \textit{Sørøens} Nordre Side høre til \textit{Loppens} Præstegield\textit{Hasvigs Annex}. Af Øen \textit{Seiland} har det den Østre, og Nordre Side til \textit{Borfiord inclusive,} og den Søndre Side til Mitten af \textit{Vargesund}; Thi det Øvrige af Øen, \textit{Kufiord}, og \textit{Bekkerfiord} til mitten af \textit{Vargesund} hører til \textit{Altens} Præstegield.\par
Paa det faste Land har \textit{Hammerfest}-Gield fra \textit{Næverfiord} i Nordost til \textit{RepperfiordKlubben}, det Nordostlige Næss af \textit{Repperfiord}, \textit{inclusive}; Thi \textit{Altens} Præstegield har den nærmeste \textit{LæretzFiord}, derfra i Syd-vest; og \textit{Jngens} Præstegield har \textit{Refsnæss}, det Sydvestlige Næss af \textit{Refsbotten}, derfra i Nord-ost.\hspace{1em}\par
At forklare det nu hver for sig i Særdeleshed:\hspace{1em}\par
Først \textbf{Øen Seiland}, som den Syd-vestlige: \textit{Seiland} ligger i Nord-ost fra \textit{Stiernøe}, 1/8 à 1/4 Miil, ved \textit{Rognsund} adskilt; Jmod \textit{Ramnæss} paa \textit{Sørøe} i Søer til Osten, hvorimellem \textit{Sørsund}, 3/4 Miil vidt, gaaer; Fra det faste Jndland en knap 1/4 Miil i Nord, ved \textit{Varge}-Sund afdeelt.\par
Dette \textit{Seiland} er fra sit Vestlige Næss, \textit{Borfiord}-Næss til detz Østlige Næss \textit{Rastebye} 2 1/2 Miil langt, fra Søer i Nord 1 à 1 1/2 Miil bredt. Beskrives efter sin Leje saaledes:\par
\textit{Seiland} fra dets ostlige Næss \textit{Rastebye} til dets Sydvestlige Næss \textit{Hakstabben} giør den Nordre Side af \textit{Vargesund}, og stikker i Syd-vest 2 Mile lang.\par
\textit{Vargesund} er til det faste Land en knap 1/4 Miil over bredt.\par
Paa denne Øens Sydlige Side til \textit{Vargesund} er Landskabet, fra \textit{Rastebye} at reigne, berget, u-dyrket, og u-beboet 1 1/2 Miil vejs hen til \textit{Bekker-Fiord}, hvor Folk boer paa Østre Side af Fiorden, imod Sundet; Vesten for \textit{Bekkerfiord} hen til \textit{Hakstabben} boer og Jngen.\par
1. \textit{Bekkerfiord} er i Gabet 1/8 Miil viid, og stikker i Nord-vest imod 1/4 Miil dyb, hvor nogle faa Søe-\textit{Finner} boe, og lidet Bierkeskoug er.\par
Fra \textit{Hakstabben} gaaer \textit{Seiland} i Nord-vest 1 1/4 Miil til \textit{Borfiord}-Næsset; igiennem \textit{Rognsund}. J dette \textit{Rognsund} 1/2 Miil i Nord-vest fra \textit{Hakstab} er den\par
2. \textit{Store Kufiord}, 1/8 Miil i Gabet viid, 3/8 Miil i Nord-ost dyb, hvor \textit{Finner} boe. Strax om \textit{Kufiord} Næsset møder\par
3. \textit{Lill Kufiord}, et Par Bøsse-Skud viid i Gabet, 1/8 Miil dyb i Nordost, ubeboed.\par
Paa den Nord-Ostlige Side af \textit{Seiland} ere Fiorder: Saasnart man kommer om \textit{Seilands} Vestre Næss, \textit{Borfiord Næss}, aabner sig\par
4. \textit{Borfiord}, 1/8 Miil i Gabet viid, og ligesaa dyb; Mitt i Botten udgaaer et Næss fra Landet, som danner i Fiorden 2de Poller; Her boer 1 \textit{Normand}.\hypertarget{Schn1_75223}{}Schnitlers Protokoller V.\par
Fra \textit{Borfiord} i Nordost til \textit{SkreiFiord} er 3/4 Miil, og det Mellem-Rom kaldes \textit{Bor-weggen}, høyt og bratt ud til \textit{Sørsundet}, u-beboet.\par
5. \textit{Skreifiord} er i Gabet 1/4 Miil viid, og stikker i Syd-ost en god 1/4 Miil dyb; Ved dens Botten paa den Østre Side indgaaer en \textit{Tværfiord}, et par Bøsseskud viid, 1/8 Miil dyb i Øster; J denne \textit{Skreifiord} boe 2 Normænd, og haves Bierke-Skoug.\par
Det Ostlige Næss af \textit{Skreifiord} heder \textit{Kaarhavn}; Thi der gaaer en Bugt ind, for Skibe at ligge; Dette \textit{Kaarhavn} strekker sig i Nord-ost 1/4 Miil, og giør det Syd-vestlige Næss af\par
6. \textit{Gyfiord}, 1/4 Miil i Gabet viid, stevner i Søer 3/4 Miil ‒ siden ved en trang Strøm, i Vester en knap 1/8 Miil lang, ganske smal; Strømmen er 1 Bøsseskud breed, og en 4 Bøsseskud lang. Paa den Østre Side af denne Gyfiord indgaaer en Tvær-Fiord i Øster, nogle Bøsseskud i Gabet viid, og 1/8 Miil dyb.\par
Fra denne \textit{Gyfiords} Vestre Side gaar et kort Eid, 1 Bøsseskud over bredt, til \textit{Tværfiorden} af \textit{Skreifiorden}. Det Nordostlige Næss af \textit{Gyfiord} heder \textit{Væggen}, og strekker til \textit{Hønsebye-Fiord} i Nord-ost 1/8 Miil langt, steilt ned ad \textit{Sørsundet}.\par
7. \textit{Hønsebye}Fiord er i Kæften til \textit{Hønsebye} Næsset 1/4 Miil viid, og stikker i Syd-ost 1/4 Miil dyb; Mitt i Botten udgaaer fra Landet et lidet Næss som giør 2 Poller, den Eene deraf stevner i Søer, den Anden i Syd-ost: dog er at agte, at \textit{Hønsebye-Næss} ligger i Øster fra \textit{Væggen}, det Vestre Næss. \textit{Hønsebye-Næss} gaaer ud i Nord-vest, og deruden for ligge 2de Øer, nemlig \textit{StorVinne}, og \textit{Lill-Vinne}; Dette \textit{Hønsebye-Næss} med bem.te 2de \textit{Vinne}-Øer ligge efter hinanden i Nord-vest, og \textit{Formere} den Vestre Side af \textit{Vinnesunds}-Fiord: Da derimod \textit{Seilands} Land giør den Østre Side. Denne\par
8. \textit{Vinnesunds}Fiord er i Gabet 3 à 4. Bøsseskud viid, og gaaer en 3/4 Miil ind i Syd-ost; Det Østre Næss af denne \textit{Vinnesunds}Fiord heder \textit{Fagervigklub}, og ligger i Nord-ost fra \textit{Wæggen, Gyfiordens} Østre Næss.\par
Paa den Nord-ostlige Side af denne \textit{Vinnesunds} Fiord indgaaer\par
(a) \textit{Survig} Øster imod \textit{Hønsebye-Næss}.\par
(b) Norden derfor \textit{Fagervig}, 1/8 Miil indenfor \textit{Fagervig}-Næsset, hvoraf dette Næss har sit Navn; J begge disse Viger boe \textit{Normænd}.\par
Fra \textit{Fagervig}-Næsset gaaer Øen i Øster 1/4 Miil til \textit{Hundenæss}, hvor da Strømmen imellem \textit{Seiland}, og \textit{Hvaløe}, med det Navn \textit{Hammerfest}-Strømmen, begynder.\par
All Øens Nordlige Strekning fra dens Vestere Næss \textit{Borfiord-Næss} til \textit{Fagervig-Næsset} stikker langs med \textit{Sørsund} i Nord-ost gode 2 1/2 Miile lang. Det Jndre Øens Land er berget, Snee-dekket, og u-beboet, havendes vilde Reen.\par
\centerline{Følger nu \textbf{Hvaløe:}}\par
Den ligger fra \textit{Seiland}-Øe, ved \textit{Hammerfest-Strømmen}, (3 Bøsse-Skud fra Søer i Nord lang, og 3 Bøsseskud breed,) adskilt i Nord-ost; Fra \textit{Sørøens} Nord-ostlige Ende i Syd-ost næsten 3/4 Miil, som regnes fra \textit{Akkernæring} paa \textit{Sørøe} til \textit{Hval-Mylingen} paa \textit{Hvaløe} over \textit{Sørøe-Sund}; Ved sin Syd-ostlige Ende, som kaldes \textit{Stallagard}, er det fra det faste Jndland \hypertarget{Schn1_75514}{}Om Hammerfest Præstegield. afdeelt ved \textit{Hvalsund}, hvor der er kun 2 Bøsseskud over, saa \textit{Hvaløe} ligger paa den Kant Norden for det faste Land, og skilles derfra, som sagt, alleene ved \textit{Hvalsund};\par
Mellem dens Østre Ende og det faste Land er et Sund, kaldet \textit{Samlings}-Sund, 1/4 Miil over bredt, at forstaae fra \textit{Kragnæss} paa \textit{Hvaløe} over \textit{Samlings} Sund til \textit{Knappen} paa det faste Land.\par
Med sin Nord-ostlige Side ligger den fra det faste Land 1 god Miil, at forstaae fra \textit{Storvignæss}, over Fiorden \textit{Refsbotten}, til \textit{Saunæss} paa faste Land.\par
Denne \textit{Hvaløe} er langagtig rund, næsten 2 Mile lang fra Nord-vest i Syd-ost, næsten 1 Miil breed fra Syd-Vest i Nord-Ost.\par
Den Syd-vestlige Side begynder fra \textit{Strømnæss} i \textit{Strømmen}, og er i Syd-ost til \textit{Stallagard} 1 1/4 Miil lang, siger {1 1/4 Mil}\par
Den Syd-ostlige Side af \textit{Hvaløe} fra \textit{Stallagard} til \textit{Kragnæss} i Nord-ost en knap {1/2‒ }\par
Den Nordostlige Side fra \textit{Kragnæss} til \textit{Mylingen} i Nord-vest gode {1 1/2 ‒}\par
Den Vestre Side fra \textit{Mylingen} til \textit{Strømnæss} i Søer {1 ‒}\par
Paa den Syd-vestlige Side fra \textit{Strømnæss} i Syd-ost til \textit{Stallagard} er Landet fladtvoren ned til Søen med Bierk og Græss begroet, og beboes af 2 \textit{Normænd} og 3 \textit{Finner}.\par
Paa den Syd-ostlige Side mitt imellem \textit{Stallagard} og \textit{Kragnæss} er en, kalded \textit{Storbugt}, 3 à 4. Bøsseskud viid i Gabet, og 1 Bøsseskud dyb, hvor 2 \textit{Finner} boe, havendes Bierk og Græss-Land.\par
Den Nordostlige Side fra \textit{Kragnæss} i Nord-vest til \textit{Mylingen} har Fiorder:\par
1. Jnderst \textit{Torsk-Fiord} strax Norden om \textit{Kragnæss}, i Gabet knap 1/8 Miil viid, og ligesaa dyb. Jmellem denne \textit{Jnderst Torskfiord}, og følgende \textit{Mellem TorskFiord} i Nord-Nordvest er kun et Næss; derpaa imodtager\par
2. \textit{Mellem Torskfiord}, i Gabet 1/4 Miil viid, og ligesaa dyb\par
3. \textit{Ytterst TorskFiord} forekommer derefter i Nord-vest, ved et Næss adskilt fra \textit{Mellem-Torsk}Fiord; dette \textit{Ytterste Torsk}Fiord er i Gabet knap 1/4 Miil viid, og ligesaa dyb. Fra \textit{Ytterst Torsk}Fiord er 1/2 Miil i Nord-vest til\par
4. \textit{HvalFiord}, i Gabet 1/8 Miil viid, 1/4 Miil dyb i Syd-vest; J alle disse 4 Fiorder boer ingen Folk.\par
Landet imellem \textit{Hvalfiord} og \textit{Mylingen} er en god 1/2 Miil i VestNordVest, deels fladt og græss-groet, som beboes af 2 \textit{Normænd} paa Gaarden \textit{Forsell}, deels steenet og bratt ud ad Søen, ubeboet.\par
Den Vestre Side af \textit{Hvaløe} imellem \textit{Mylingen} og \textit{Strømmen} bestaaer af\par
5. den Vaag \textit{Hammerfest}, 1/2 Miil Sønden for \textit{Mylingen}, og en knap 1/2 Miil Norden for \textit{Strømmen}, i Gabet 2 BøsseSkud viid, 3 Bøsseskud dyb, en tryg Skibshavn, hvor og \textit{Hammerfest}-Træ-Kirke, Handelsteden, og 3 \textit{Normænd} boe. Sønden for \textit{Hammerfest} om Næsset \textit{Ryp-klub} aabner sig\par
6. \textit{Ryp-Fiord}, i Gabet knap 1/4 Miil viid, knap 1/8 Miil dyb, hvor 5 \textit{Normænd} boe.\par
Jnde paa Øens Land ere luter Fielde, flade og mossede.\par
Sundene ved denne \textit{Hvaløe}, foruden den før beskrevne Strøm, ere\hypertarget{Schn1_75823}{}Schnitlers Protokoller V.\par
1. \textit{Hvalsund} imellem \textit{Hvaløe} og det faste Land, regnes fra \textit{Stagnæss}, 1/4 Miil Vesten for \textit{Stallagard} til imod \textit{Tappen} paa det faste Land, 1/8 Miil Østen for \textit{Stallagard}, følgelig 3/8 Miil lang, 2 Bøsseskud bredt.\par
2. \textit{Samlings}Sund reignes fra \textit{Stallagard} paa Øen, og derimod fra \textit{Repperfiord-Klubben} paa det faste Land i Nord til Osten til \textit{Refsnæss} paa det faste Land 1 god Miil langt, 1/2 Miil og mindre bredt.\par
3. \textit{Sørsund} regnes fra \textit{Haaën} paa \textit{Sørøe} og derimod fra \textit{Vestre Stiernodden} i Nord-Ost til \textit{Akkernæring} paa \textit{Sørøe}, og derimod til \textit{Hval-Mylingen} paa \textit{Hvaløe} 4 1/2 Miil langt, 1. à 1/4 Miil bredt.\par
\centerline{\textbf{Sørøe}}\par
hvorj \textit{Hammerfest} Præstegield har sin Deel, er forhen forklared \textit{pag:} 247 f.\par
\textit{Af faste Land} har \textit{Hammerfest} Præstegield den Strekning i Nord-ost fra \textit{Læredzfiordz} Nordostlige Næss, der kommer \textit{Altens} Præstegield til, og \textit{pag:} 222 her tilforn er beskreven, indtil \textit{Repperfiords} Ostlige Næss, kaldet \textit{RepperFiordsklubben}, imod 3 Mile; hvilken Strekning saaledes nøyere forklares:\par
Fra \textit{Læredzfiords} Nordostlige Næss, \textit{Liniarg}, til \textit{Næverfiord} i Nord-ost er 2 Mile; Landskabet derimellem er berget, og u-beboet, deels bratt og bart, deels fladtvoren, neere ved SøeSiden med Bierk vel begroet, og sommestedz Græss-liet, som de Folk fra \textit{Læredsfiord} og \textit{Næverfiord} bruge at slaae.\par
1/4 Miil Sønden for \textit{Næverfiord} nedløber Søndenfra af Fieldene i \textit{VargeSund e}n Elv, see \textit{pag}. 261. Ved Munden af denne Elv ligger \textit{Varge}- eller \textit{Qvæne Klubben}, et Berg, rundvoren, ovenpaa fladt, deels steenet, deels Lyng-groet, 1/8 Miil over stort, havendes Græss og Bierkeskoug paa Syd-vestlige, og Nordostlige Side, hvor Field-\textit{Finnerne} om Sommeren tilholde; Hvilken \textit{Varge-Klub} ligger paa Nordostlige Side af Elv-Munden.\par
l/4 Miil fra denne Elv er \textit{Næverfiord}, hvis Syd-vestlige Næss heder \textit{Næverfiords}Næss, det Nord-ostlige Næss, \textit{Næverfiordz-Klubben};\par
\textit{Næverfiords}Næsset er langagtigt efter Sundet fladtvoren ovenpaa med Bierk begroet, græsset neer ved SøeKanten;\par
\textit{Næverfiords Klubben} bestaaer af 2de smaa runde Houger, spidse op i Vejret, skallede, liggendes efter hinanden i Nordost efter \textit{Hval}Sundet.\par
\textit{Næverfiord} er i Gabet 1 Bøsseskud breed, 6 Bøsseskud dyb i Søer, beboed af 8. Søe-\textit{finner}; og disse ere de første Folk af \textit{Hammerfestes} Meenighed paa det faste Land.\par
J denne \textit{Næverfiordz} Botten kommer Sønden-fra af Fieldene en Aae, \textit{Næverfiordz}-Elven, see \textit{pag}. 261.\par
Fra \textit{Næverfiord Klubben} i Nordost til en Bugt, \textit{Akkereid}, er 1/2 Miil, hvor 6. Søe\textit{finner} boe; Landskabet derimellem er fladt neer ved SøeKanten, med Græss og Bierk paa, dog ere smaa skallede Berge derimellem.\par
Det Sydvestlige Næss af \textit{Akkereid} heder \textit{Akkereid-Næsset}, noget høyt, med Lyng paa, men paa Sidene græsset, og Bierke-groet, rundt, 1 Bøsseskud over stort;\hypertarget{Schn1_76101}{}Om Hammerfest Præstegield.\par
Det Nordostlige Næss kaldes \textit{Tappen}, liggendes med sin Vestre Side til \textit{Akkereid}, og med sin Ostlige Side til \textit{Repperfiord}, imellem hvilke begge den stikker spidz ud ad \textit{Hvalsundet}, som en Tap; slet ovenpaa, fladt paa Sidene, mest overalt græsset;\par
Dette \textit{Akkereid} er et par Bøsseskud vidt, 1 Bøsseskud dybt; paa hvis Botten staaer et \textit{Finne Capell};\par
J Botten af dette \textit{Akkereid} nedkommer \textit{Akkereid}-Elv, hvorj Øreter og Lax fanges, fra Fieldene, see \textit{pag}. 261.\par
Strax Østen om \textit{Tappen} aabner sig\par
\textit{Repperfiord}; dens Nordostlige Næss er \textit{RepperFiord-Klubben}, noget høyt, fladt-rundagtigt ovenpaa, l/8 Miil stort ved \textit{Hvalsund}, graaberget med noget Lyng paa.\par
\textit{Repperfiord} er i Gabet 1/4 Miil breed, men indentil vider sig ud, siden gaaer smal sammen imod Botten, 1 Miil lang i Ost-Syd-ost.\par
J dens Botten nedkommer fra Syd-Syd-ost af Fieldene \textit{Repperfiord}-Elv, see \textit{pag}. 262.\par
Paa Syd-vestlige Brædde af \textit{Repperfiord} er Landet slet ved Søen, græss-groet, der ovenfor Bierke-Skoug, og ovenfor Skougen bare Fielde findes; der sidde 15 Søe\textit{finner}; Den Nordostlige Fiord-Brædde er meere berget og lynget, med mindre Græss og Bierk; derfor boe 7 \textit{Finner} her længere ind ad Botten; Saa i denne Fiord ere ialt en 22. \textit{Finner}. Fra \textit{Repperfiord} til mod \textit{Refsnæss} er knap 1 Mil.\par
Og saavidt gaaer \textit{Hammerfest} Præstegieldz \textit{District} paa det faste Indland.\par
Sluttelig er i dette \textit{Hammerfest} Præstegield 1 nemlig \textit{Hammerfest} Kirke af Træ paa Botten af \textit{Hammerfest}-Vaagen paa \textit{Hvaløe}; havendes 35 \textit{Normænd}, og \centerline{55. Søe\textit{Finner} ___ i alt 90. boesiddendes Mænd.}\hspace{1em}\par
til 3die Sp: Svares: Fisk og Søefugl, som i \textit{Altens} Gield\textit{pag}. 226 er \textit{specificered}: dog fanges her ikke Lax til nogen Udførsel, som i \textit{Altens} Elv.\hspace{1em}\par
Sp. 4. Sv: \textit{Havne} i \textit{Loppens} Gield inde i \textit{Bergsfiord} er en god viid Skibshavn.\par
J \textit{Hammerfest} Gield (1) \textit{Hammerfest-Vaagen}, tryg for en 20 Skibe. (2) paa Vestre Side af \textit{Hvaløe}, et par Bøsse-skud Syd ost fra \textit{Strømmen}. (3) Ved \textit{Jacob-EriksØy} god for en 15 Skibe. (4) \textit{Hvidsteen-Fiord} 1/4 Miil Sønden for \textit{Strømmen}. (5) \textit{Støvel-Havn}, strax Vesten for \textit{Strømmen}, (6) \textit{Kaarhavn}; de 3de sidste ere ved \textit{Seiland}; og derfra den første viid for en 20. men de tvende sidste kun for et Par Skibe.\hspace{1em}\par
Sp. 5. og 6. Svar: Fiorder og Elve ere ved Øerne og det faste Land med Fiordene beskrevne.\hspace{1em}\par
Sp. 7. Svar: Myrer og Dale findes paa Øen \textit{Seiland} 1. Sønden for \textit{Gyfiord} langs med \textit{GyfiordElven} med mangfoldig Bierkeskoug, 1/4 Miil dyb, 3 à 4. Bøsseskud breed. 2. Paa det faste Land fra \textit{Akkereid} Bugten op efter Elven er en Bierke-Dal 1/4 Miil meer og mindre breed, over 1 Miil lang. 3. Sønden for \textit{Refsbotten} langs med \textit{Ruus}-Elven 1 1/2 Miil lang, 3 à 4 Bøsseskud breed.\hypertarget{Schn1_76378}{}Schnitlers Protokoller V.\par
Sp. 8. Svar: Gaardene ligge vid adspredte paa Søe-Kantene, avle intet Korn, men alleene Græss; \textit{Creature} holde Jndbyggerne, som de i \textit{Alten}\textit{pag}. 228 ikke Heste, uden at en Betient kan have 1 Hest.\hspace{1em}\par
Sp. 9. Svar: Skoug er her ikke, uden Bierk og anden smaa Slags til Brændefangst, hvorpaa ikke Bræk skal være; de betydeligste Bierkeskouge i dette Gield ere (1) paa \textit{Seiland} Sønden for \textit{Gyfiord} langs efter Elven. (2) paa det faste Land ved den Elv, som 1/4 Miil Sønden for \textit{Næverfiord} i Hvalsund udgaaer. (3) Langs med \textit{Næverfiord}-Elven. (4) saa og \textit{RepperFiord}-Elven, som skal strekke sig i Søer 2. Dags Reise lang til \textit{Jetz-jaure}.\par
Grov Skoug af Gran og Furre haves ei i \textit{Loppen}- eller \textit{Hammerfest}-Gield; Hvorfor Søe- \textit{Finner} opreise deres Gammer eller Hytter af Bierke-Træer med Næver og Jord tækkede; og de fornemmere \textit{Normænd} maa hente Bygnings-Timmer af \textit{Altens} Skoug, naar de vill bygge sig Husse.\hspace{1em}\par
Sp. 10. Svar: Vild og \textit{Jnsecter}, som i \textit{Altens} Gield\textit{pag}. 228 f.\hspace{1em}\par
Sp. 11. Sv: Leilighed til Rødning meenes at være 1) Sønden for \textit{Gyfiord} paa \textit{Seiland} efter Elven 2) Sønden for \textit{Akkereid} paa faste Land efter Elven.\hspace{1em}\par
Sp. 12 Sv: Fieldene ere med Øernes og det faste Landz Leje og Beskaffenhed beskrevne.\hspace{1em}\par
Sp. 13. Sv: Jngen Kiøbstad, men for \textit{Loppen} og \textit{Hasvig} er paa \textit{Hasvig} paa \textit{Sørøe} Eet ‒ og for \textit{Hammerfest}-Almue i \textit{Hammerfest} Vaagen paa \textit{Hvaløe} et Handel-Sted eller Oplag med Kongelig \textit{Octroje} anlagt.\hspace{1em}\par
Sp. 14. Sv. Næringen bestaaer i Fiskerie og Engesletter. Finnerne bruge og paa Skøtterie noget.\hspace{1em}\par
Sp. 15. Sv. \textit{Mineralia} og \textit{Naturalia} ere ubekiendte.\hspace{1em}\par
Sp. 16. Sv: De nærmeste Gaarder til \textit{Hammerfest}Gield ere i Vester:\par
Paa \textit{Sørøes} Nordre Side \textit{Dynnæss}Fiord af \textit{Loppens} Gield, fra \textit{Sandbotten} af \textit{Hammerfest}- Gield over Land 1 Miil i Vester. Paa \textit{Sørøes} Søndre Side \textit{Grundfiord} af \textit{Loppens} Gield, fra \textit{Gaashob} af \textit{Hammerfestes} Gield 1 Miil i Vester.\par
Paa Øen \textit{Seilands} Vestre Side \textit{Kufiord}, fra \textit{Borfiord} af \textit{Hammerfest}-Gield 3/8 Miil i SydVest; hvilken \textit{Kufiord}-Gaard ligger under \textit{Altens} Præstegield; Paa \textit{Seilands} Søndre Side \textit{Bekkefiord} af \textit{Altens} Præstegield, fra \textit{Grundvog} ved \textit{Fagervig Klubben} paa Nordre Side Seiland 3 Miile.\par
Paa det faste Land \textit{Læredsfiord} af \textit{Altens} Gield fra \textit{Næverfiord} af \textit{Hammerfest}-Gield i Sydvest 2. Mile; paa samme faste Land \textit{Refsnæss} af \textit{Jngens} Præstegield, fra \textit{Repperfiord} i Nord-ost 1 Miil.\hspace{1em}\par
Sp. 17. Sv. Vei til \textit{Kiølen} ingen for Bønderfolk.\hspace{1em}\par
Sp. 18. Sv: Field\textit{lapper} see \textit{pag}. 230.\hypertarget{Schn1_76709}{}Om Hammerfest Præstegield.\par
Om Field\textit{kiølen} og Grændse-Gangen hvad man har kundet erfare, sees pag. 242 f.\par
Og hermed blev Retten her slutted med samme \textit{Jntimation}, som \textit{pag.} 240 indført.\hspace{1em}\par
\textit{Hammerfest} d. 11 \textit{Sept.} 1744. \hspace{1em}\centerline{Peter Schnitler.}\hspace{1em}\centerline{L. S. Arrent Olsen, Gyfiord}\centerline{L. S. Jens AndersenKaarhavn}\hspace{1em}
\DivII[Sept. 12.-14. Fra Hammerfest til "Fladtøe" (Latøy)]{Sept. 12.-14. Fra Hammerfest til "Fladtøe" (Latøy)}\label{Schn1_76775}\par
\textbf{Ao}\textbf{1744. d. 12 Sept:} reiset fra \textit{Hammerfest} i Nord til \textit{Hval-Mylingen} paa \textit{Hvaløe}{1/2 Miil} derfra i Øster til \textit{Forsell}, en Gamme paa samme Øe {1/2 ‒ _______ 1 Miil;}\par
Længere man for Uvejr over den Fiord \textit{Refsbotten}, hvor haardsøgt Søe gaaer, ei kunde komme ‒\hspace{1em}\par
d. 13 \textit{Sept:} var Søndag.\hspace{1em}\par
d. 14de fra \textit{Forsell} Foer man over Havet forbi \textit{Refsbottens}- og \textit{Snee}fiordene i Nord-ost til \textit{Fladtøe} 2 Miil Efterat man tilforn fra \textit{Reenøe} havde giort Bud efter nogle kyndige \textit{Finne}- Vidner i \textit{Refsbotten}, hvilke og indfandt sig paa \textit{Fladtøe}\hspace{1em}\par
d. 16. ‒\hspace{1em}
\DivII[Sept. 17.-19. Rettsmøte på Latøy]{Sept. 17.-19. Rettsmøte på Latøy}\label{Schn1_76872}\par
\textbf{Ao 1744. d. l7 Sept:} satte man Retten paa \textit{Fladtøe}, nærværendes Lensmand \textit{Simon Simonsen Maasøe}, med LaugRettesMænd, over de fra \textit{Refsbotten} ankomne \textit{Finne}-Vidner; For hvilke ved Tolken \textit{Hælset} Eeden af Lovbogen blev forklared; Som og derpaa aflagde deres \textit{Corporlig} Eed: \par
Efter vedtagen Orden erkyndigde man sig forud hos Lensmand og LaugRetten, samt hos Vidnerne, om Landets Leje og Strekning af dette Sted og Præstegield, fra SøeKanten op i Søer til Grændsen:\hspace{1em}\par
til 1 Sp: Svarede: \textit{Fladtøe}, hvor nu Retten holdes, ligger i \textit{Jngøens} Sogn og Præstegield i \textit{Vest-Finmarken} i \textit{Wardehuus} Amt; hvilket \textit{Jngens} Præstegield ligger i Nord-Ost nærmest til \textit{HammerFest}-Præstegield, og i Vester nærmest til \textit{Kielvigens} Præstegield; Dette \textit{Jngens} Gield har 3de Kirker, alle af Træ,\hspace{1em}\par
1. \textit{Jngens} Hoved Kirke paa \textit{Jngenøes} Østre Side fra nærmeste \textit{Hammerfestes} HovedKirke paa \textit{Hvaløe} i Nord-Nord-ost 3de Mile.\hspace{1em}\par
2. \textit{Jelmesøes Annex} Kirke\textit{Jelmesøes} Østre Sides Nordre Ende, 1 Miil Østen for \textit{Jngens} Hoved-Kirke.\hypertarget{Schn1_76991}{}Schnitlers Protokoller V.\par
3. \textit{Stappens Annex}Kirke paa en, nemlig \textit{Huss-Stappen}-Øe, 1 Miil Østen for \textit{Jelmes}øes \textit{Annex}Kirke, og ved 2 Mile Vesten til Norden fra \textit{Kielvigens} Hoved-Kirke.\par
Dette \textit{Jngens} PræsteKald \textit{vacerer} nu, og betienes dets Hoved-Kirke af nærmeste Søndre \textit{Hammerfest}-Præst, og de 2de \textit{Annexer} af nærmeste Østre \textit{Kielvigens} Præst.\par
\textit{Jngens} Hoved-Sogn har til Meenighed {21 \textit{Normænd} + 8 Søe\textit{finner} __________ 29. \textit{Mænd}} [\textit{Jelmes}øes \textit{Annex}{10 Normænd}\par
\textit{Stappens Annex}{ 5 \textit{dito} _________ = 44 \textit{Mænd}}\hspace{1em}\par
De Søndreste Gaarder af \textit{Jngens} Præstegield ere:\par
‒i Vester \textit{Refsnæss}, nu øde, fra \textit{Hammerfestes} Østerste Gaard \textit{Repperfiord} i Nord-ost 1 Miil,\par
‒i Øster \textit{Refsbotten}, fra \textit{Refsnæss} 1 Miil i Ost-Syd-ost.\par
Fleere Gaarder paa faste Land ei ere, \textit{Jngens} Gield tilhørende: Dog gaaer \textit{Jngens} Præstegieldz Strekning paa dette faste Land fra \textit{Refsnæss} til \textit{Myr-Nipen} 2 gode Mile i Nord-nordost;\par
De Nordreste Gaarder af Gieldet ere i Vester paa \textit{Jngen}-Øe, fra \textit{Refsnæss} i Nord 2 1/2 Mile \textit{(Finmarkiske)}.\par
J Øster er \textit{Tuënæss} paa \textit{Magerøe}, 2 1/4 \textit{Finmarkiske} Mile fra \textit{Jngen} i Øster til Norden, og 4 1/2 Mile i Nord-ost fra \textit{Refsbotten}.\hspace{1em}\par
Til 2det Spørsm: svaret: \textit{Jngens} Præstegield bestaaer deels af Øer, deels af fast Land:\par
Af Øer, er nærmest til \textit{Hvaløe} af \textit{Hammerfest}-Præstegield, og til det faste Land\par
\centerline{1. \textbf{Biørnøe}}\par
1 sterk Miil i Nordost fra \textit{Hvaløe}, 1/4 Miil fra \textit{Saunæss} paa faste Land i Nord-nord-vest, rundagtig, et par Bøsse-Skud over stor, høyklipped, uden Græss og Skoug, u-beboed.\par
J Nord-vest 1 sterk Miil fra denne \textit{Biørnøe} yderst i Havet sees 2de smaa u-beboede Holme, navnlig \textit{Refs-Holm}, rund 1/8 Miil omkring, og nær Norden for den \textit{SkibsHolm}, rund 1/4 Miil omkring stor.\par
\centerline{ 2den \textbf{Reenøe}}\par
fra \textit{Biørnøe} i Nordost 1/4 Miil, fra \textit{Bursta-vigen} paa faste Land knap 1/8 Miil i Nord-vest, lang i Nordost 3/8 Miil, et par Bøsseskud over breed, berged og steen-ured, med lidet Græss-Beete og Lyng paa, men ingen Skoug, uden smaa Bierke-Riis.\par
Et par Bøsseskud Norden for denne \textit{Reen}-Øe er \textit{Lill-Reenøe}, en liden Holm, berged, uden Græss og Skoug, u-beboed, hører til \textit{Stor-Reenøe}, hvis Jndbyggere samle Fugle-Egg derpaa; som ere 3 \textit{Normænd.} 1/2 Miil i Ost-Nord-ost fra \textit{Stor-Reenøe} er\hypertarget{Schn1_77269}{}Om Jngøens Præstegield.\par
\centerline{3. \textbf{Fladtøe,}}\par
1/8 Miil Vesten for det faste Land, lang i Vest-Nord-vest knap 1/8 Miil, 1 à 2 Bøsse Skud over breed, klipped og steen-ured, med lidet Græss paa, uden Skoug, beboed af 4 \textit{Normænd}.\par
\centerline{4. \textbf{Yttre-Fladtøe,}}\par
ligger et par Bøsseskud i Nord-vest fra forbem.te \textit{Jndre-Fladtøe}, berged med lidet Græss paa, som de Folk paa \textit{Jndre Fladtøe} bruge, rundagtig, et par Bøsseskud over stor, u-beboed.\par
\centerline{5. \textbf{Rolfsøe}}\par
forekommer 1/2 Miil i Nord-vest fra \textit{Biørnøe} og \textit{Reenøe}, fra Søer i Nord 1 Miil lang, og næsten ligesaa breed, hvor videst er, 2 Miile omkring stor, berged med Græss og Lyng paa, dog uden Skoug, beboed af 5 \textit{Normænd}.\par
Mitt paa dens Vestre Side indgaaer\par
1. \textit{VallFiord}, 1/4 Miil i Gabet, 3/8 Miil i Øster dyb, dens Søndre Næss er \textit{Stoppelnæring}, dens Nordre Næss \textit{Tuë-næring}; Paa Østre Side indgaaer\par
2. \textit{Langfiord}, imod \textit{VallFiord;} 1/4 Miil i Gabet viid, og 1/4 Miil dyb i Vester.\par
Denne \textit{Rolfsøe} har mange Viger: 1 Bøsseskud Sønden for \textit{Langfiord} er\par
1. \textit{Dyfiord} paa Østre Side af Øen, knap 1/8 Miil viid, i Gabet, 1/16 Miil dyb i Vester.\par
2. \textit{Rolfsfiord}, eller \textit{Rolfs-vær}, 1/8 Miil Sønden for \textit{Dyfiord}, saa viid og dyb, som samme \textit{Dyfiord}, er en SommerHavn for et par Skibe.\par
3. \textit{Søer-Havn}, 1/16 Miil i Syd-vest fra \textit{Rolfs-vær}, 1/16 Miil i Gabet viid, 1/16 Miil dyb i Nordvest\par
4. \textit{Gaasviig}, 1/4 Miil Vesten for \textit{Søer-Havn}, viid og dyb, som \textit{Søerhavn}.\par
5. \textit{Stoppelfiord}, 1/8 Miil i Nordvest fra \textit{Gaasviig}, viid, men noget dybere, som \textit{Gaasviig;} Herefter i Nord møder \textit{VallFiord}; deruden for er\par
6. TueFiord, 1/16 Miil Norden for \textit{Vallfiord}, 1 Bøsseskud over viid, 3 Bøsseskud dyb i Øster\par
7. \textit{Troll-Fiord}, fra \textit{Tuefiord} 1/4 Miil i Nord-ost, 2 Bøsseskud i Gabet viid, 6 a 7 Bøsseskud dyb i Søer.\par
8. \textit{Sandfiord}, 3 a 4 Bøsseskud i Sydost fra \textit{Troll-Fiord}, i Gabet 1/8 Miil viid, 1/16 Miil i Syd-vest dyb; Herpaa i Søer et par Bøsseskud følger den forbeskrevne Fiord \textit{Langfiord}\par
\centerline{6. \textbf{Jngen-øe}}\par
ligger 1/8 Miil og mindre Norden fra Rolfsøe; Er fra Vester i Øster 3/8 Miil lang, 1/4 Miil, hvor videst, breed, berged paa Vestre Side, paa de andre Sider noget slet, og græsset, uden Skoug, paa sin Nordre Side, Østen for \textit{Magfiord}, forsiuned med Hoved-Kirken, og 9. Beboere, \textit{Normænd;} Har\par
1.\textit{ Magfiord} paa bem.te Nordre Side, i Gabet 4 Bøsseskud viid, og i Søer ligesaa dyb.\par
2. \textit{Jngviig}, 1/16 Miil Østenfor \textit{Magfiord}, 1/4 Miil i Gabet viid, og i Øster et par Bøsseskud dyb;\par
Uden for disse \textit{Magfiord} og \textit{Jngviig} 1igge en hoben smaa Holmer.\hypertarget{Schn1_77517}{}Schnitlers Protokoller V.\par
3. \textit{Veinæssbugt}, 3 a 4 Bøsseskud i Syd-vest fra \textit{Jngviig}, 1/4 Miil viid, et par Bøsseskud dyb.\par
Denne \textit{Jngen øe} har den Herlighed, at 1/2 à 1 Miil uden derfor i Havet fanges den beste Hveete i \textit{Finmarken}; dog fordj Luften her Nord er meere taagagtig, end den der Søer i \textit{Senniens} Fogderie, saa faaer Hveeten ikke saa god Tørke, og derfor bliver dens Raf-Rækling ikke saa god, som den \textit{Sennienske}.\par
\centerline{7de \textbf{Jelmesøe}}\par
1 Miil Østen for \textit{Jngen-øe}, fra Nord-Nord-vest, nemlig fra \textit{Rehel-Næss} i Syd-Syd-ost til \textit{Strømsnæss}, en stor 1/2 Miil lang, 3/8 Miil over breed, klipped paa Vestre Side, men slet og græssed paa den Østre, hvor og Berg er med mangfoldig Alk og Maase-fugl i, hvor af Jndbyggerne nyde Eggene, Kiødet og Fiærene; 5 \textit{Normænd} boe her, dog have ingen Skoug; Paa Øens Nordre Side er\par
1. \textit{Akkerfiord}, 1/8 Miil i Gabet viid, 3/8 Miil dyb i Syd-ost, den Østre Fiord-Brædde er slet, Græss-groed, den Vestre Side klipped, med noget Græss paa Søe-Bræden;\par
Det Søndre Næss heder \textit{Akkernæss}, det Nordre \textit{Rehel-Næss;} Ved dettes Foed paa SøeBræden staaer en rundagtig spidzopgaaendes Steen-Støtte, ved Bonden en 20. Favner Men op i Toppen en 10 Favner kun tyk, ved 40 Favner høy, kalded \textit{Rehel-Fandsen}, i hvis Huller paa Sidene mangfoldige Alker værpe, og af Bønderne med lange Stænger fanges; Denne \textit{RehelFandse} tiener de Søe-farende til Efterretning: ved hvad Pladz de ere?\par
2. \textit{Sandfiord}, 1/4 Miil i Nord-ost fra \textit{Akkerfiord}, i Gabet 1/4 Miil viid, et Par Bøsseskud dyb.\par
3. \textit{Sortviig}, 1 Bøsseskud fra \textit{SandFiord} i Søer til Osten, 1/8 Miil i Gabet viid, et Par Bøsseskud dyb.\par
4. \textit{Kiibugt}, fra \textit{Sortviig} 3 a 4 Bøsseskud i Søer, 1/4 Miil viid, et par Bøsse-Skud dyb.\par
5. \textit{Knarviig}, paa Sydvestlige Side af \textit{Jelmesøe}, fra \textit{Kiibugt} i Syd-vest 1/16 Miil, i Gabet et par Bøsseskud viid, 1 Bøsseskud dyb.\par
Fra \textit{Knarviig} i Nord-vest til \textit{Juthavn-Næss} er 1/8 Miil, og Landet klippet;\par
Fra \textit{Juthavn-Næss} i Nord-Nord-ost til \textit{Akker-Næsset}1/4 Miil, er Landet berget; og saaledes \textit{Jelmes}-øe omkring 1 1/2 Miil stor, og har 8te \textit{Normænd} til Beboere.\par
Uden for \textit{Jelmesøe} fiskes Hveete til Raf-Rækling ligesom uden for \textit{Jngen}.\par
\centerline{8. \textbf{Haves-øe}}\par
Fra \textit{Jelmesøe} i Søer 1/4 Miil, fra det faste Land, nemlig fra \textit{Myr-nipen}, 1 Steenkast og meere over, ved \textit{Havesund} derfra adskilt; \textit{Havesøe} er langagtig fra Vester i Øster 3/8 Miil, 1/8 Miil over breed, hvor videst; beboed af 2de \textit{Normænd;} paa Vestre Side klipped, paa de andre fladagtig og græssed, uden Skoug.\par
\centerline{9. \textbf{Stappen-øer}}\par
fra \textit{Haves-øe} i Ost-Nord-ost 1 1/2 sterke Mile; fra \textit{Jelmesøe} i Øster 1 sterk Miil; fra \textit{Myr-Nipen} paa faste Land i Nord-ost 2 Mile; og fra nærmeste faste Land, \textit{Stikkel-Næring} i Nord 1Miil; Bestaaer af 4 Øer, fra Vester i Øster efter hinanden og 1 Bøsse-skud fra hinanden liggende:\hypertarget{Schn1_77746}{}Om Jngøens Præstegield.\par
1. Den Vesterste heder \textit{Stauren}, rund, l/16 Miil omkring stor, klipped og sommestedz græssed, u-beboed.\par
2. \textit{Stor-Stappen}, langagtig-rund, 1/4 Miil omkring, med overflødig Græss paa, noget berged; Denne Øe er et Fugl-Vær af Lund, Alk og Maaser, men af Folk ei beboes.\par
3. \textit{Huss-Stappen}, rund, 1/16 Miil omkring viid, klipagtig, uden Skoug, med Lyng og lidet Græss paa, har \textit{Annex}Kirken under \textit{Jngens} Præstegield paa Søndre Side, og 3 \textit{Normænds} Gammer, eller Jord-Hytter, som bruge de 2de næstforrige til Græss-Slotte.\par
4. \textit{Baagstappen}, den Østerste, rund, 1/32 Miil omkring stor, Lavberged, Lynged, uden Græss og Skoug, og Folk;\par
De 3de første \textit{Stappen}-Øer ere høyere af Klipper, og meere siunlige; kaldes gemeenlig derfor \textit{Moderen} med sine 2de Døttre, ved hvilken Moder forstaaes \textit{Stor-Stappen} mitt imellem \textit{Stauren}, og \textit{Huss-Stappen}.\par
Uden for \textit{Stappen}-Øerne 1/2 Miil fanges og Hveete til Raf-Rækling.‒\par
Siden paa dette Præstegieldz Øer er Bræk paa Skoug og Brændefangst, saa bruges Torv til at brænde, og Meenige Mand har Gammer af Jord, med Røg-Huller oven til igiennem Taget, at boe i.‒\par
Til det Faste Land af \textit{Jngens} Præste\textit{gield} førend man gaaer, maa man tilforn tilføye noget, og nærmere forklare det næstforegaaende Faste Land af \textit{Hammerfest}-Gield; Herom er nu \textit{pag:} 254 her før rørt, at den Field-Strekning (fra \textit{Vargesund} i Sydost) imellem \textit{LæredsFiords} Nord-ostlige Næss \textit{Liniarg} og \textit{Næverfiord}, er berged, og af \textit{Norske} Søe-\textit{Finner} u-beboed. Her tillægges nu den nøyere Beretning: at denne Field-Strekning i Sydost fra \textit{Vargesund} imellem benævnte 2de Fiorder hen imod \textit{Repperfiord} kaldes med et almindeligt Navn \textit{Zenus}, hvorom noget før \textit{pag.} 220 er talt; og bestaaer af flade, rundagtige; ikke meget høye Berge, deels med Lyng, deels med hvid Reen-Mosse paa, og med lidet Græss i Dalene imellem Fieldene, strekkende sig fra \textit{Vargesund} i Søer en 2de Dagers Reise hen til det Field \textit{Tseunik}, og det Vand \textit{Rautesjaure}, som vill holdes for en 5. à 6 Miile vejs; Hvilken Field\textit{tract} om Sommeren af \textit{Avjevaras} Field-\textit{Finner} indehaves, som og paa den Tid bruge at fiske i de \textit{Norske} Fiorder og Sunde.\par
Om dette \textit{Tseunik}-Field, hvormed \textit{Zenus} er sammenhengendes, er før rørt \textit{pag.} 220 og om \textit{Rautesjaure} og \textit{Rautesjok}, \textit{pag.} 220 og 225.\par
Den Elv som \textit{pag.} 254 er sagt at løbe 1/4 Miil Sønden for \textit{Næverfiord} i \textit{Vargesund}, heder \textit{Porse-jok}, kommendes Sønden-fra af det Vand \textit{Pors-jaure}, hvorfra den 1/16 Miil lang gaaer i Nord i \textit{Vargesund; Porsjaure} er rundt, en 6 Bøsseskud stort, med Øreter i.\par
Om \textit{Næverfiords} Elv er \textit{pag.} 254 talt, at den Sønden-fra indkommer i \textit{Næverfiordz} Bot ten; derom tillægges nu her, at denne \textit{Næverfiords} Elv rinder af \textit{Næverfiords} Skoug-Vatten 1/4 Miil lang i Nord i Fiord-Botten; \textit{Næverfiord Skoug-vatten} er 4 Bøsseskud langt, og halv saa bredt, havendes Røefisk.\par
\textit{Akkereid}-Elv, hvorom \textit{pag.} 255 er meldt, kommer Sønden-fra af det Vand \textit{Sole-jaure}, og løber 1/2 Miil lang i Nord i \textit{Akkereidbugten}.\par
\textit{Sole-jaure} er rundt, 3 Bøsseskud over vidt, med Røe-fisk i, liggendes under samme \textit{Zenus-}Field.\hypertarget{Schn1_78032}{}Schnitlers Protokoller V.\par
\textit{Repperfiord}-Elv, som er paa samme \textit{pag.} 255 omtalt, oprinder Sønden-fra af det Vand, \textit{Bastejaures} Vestre Ende, og løber i Nord 2 à 3. Mile lang i \textit{Repperfiord}Botten.\par
\textit{Baste-jaure} er fra Vester i Øster 3 Bøsseskud langt, og 2 Bøsseskud bredt, havendes RøeFisk.\par
Det Field ovenfor \textit{Repperfiord}-Botten paa Østre Side af Elven heder\par
\textit{Sio-vara}, 1/2 Miil langt fra Vester i Øster, og halv saa bredt, noget høyt, hvasst ovenpaa, deels steenet, deels Lynget, paa Østre Side ned ad \textit{Rüsse-}Field fladtvoren, paa de andre Sider steilt, med Lyng paa.\par
Sønden fra dette \textit{Sio-vara} imellem \textit{Zenus} Field, og de Østenfor værende Fielde er en SkougMark af Bierk, et Par Dagers Reise eller 8 Mile lang hen imod \textit{Jetzjaure}, 3 à 4 Bøsseskud breed.\par
Fra \textit{Repperfiords Klubben}, er \textit{pag.} 255 rørt til imod \textit{Refsnæss}, den 1te Gaard af \textit{Jngens} Gield paa faste Land, i Nord-Nord-ost er Knap {1 Miil;}\par
Landet derimellem bestaaer af et steenet, bratt Field, med det Navn \textit{Gierre-niarg}, med lidet Bierk derimellem, 3/4 Miil langt i Nord-nord-ost, 1/2 Miil og meere bredt, ovenpaa fladt, deels lynget, deels stenet, paa Vestre og Østre Sider ned ad Fiord-næssene fladtvoren med Bierk derimellem; J ost Sydost er det bratt ned ad til en Myr- og Bierke-Dal.\par
Det faste Land i \textit{Jngens} Præstegield begynder nu i Vester med det Vestre Næss af \textit{Refsbotten}-Fiord, \textit{Refsnæss}, som efter Fiorden er 2 à 3 Bøsseskud langt, 1 Bøsseskud bredt, ovenpaa tindet og snaut, bratt ned ad Fiorden og ad Søen, paa de andre Sider fladtvoren; Paa Sydostlige Side til \textit{Yttre-Vaagen} græsset, ellers bart.\par
Det Østre Næss af \textit{Refsbotten-}Fiord heder \textit{Saunæss}, 2 Bøsseskud langt, og bredt, fladt og bart ovenpaa, paa Sidene fladtvoren og lynget; Jmellem \textit{Refsnæss} og \textit{Saunæss} er FiordGabet 1 liden Miil over i Nord: Ellers reignes og Gabet af \textit{Refsbotten} længere, nemlig fra \textit{HvalMylingen} paa \textit{Hvaløe} til bem.te \textit{Saunæss} 1 1/2 Miil over vidt i Ost-Nord-ost; Og Fiorden i Sydost 1 Mil lang.\par
Tæt Østen for \textit{Refsnæss}, paa den Syd-vestlige Side af \textit{Refsbottens}Fiord, nær ved Gabet, ere 3 smaa Vaager, stikkendes i Søer ind i Landet, navnlig \textit{Yttere Refsnæss-Vaag, Jndre Refsnæss-Vaag}, og \textit{Lønæss-Bugten}; de 2de første ere 1 Bøsseskud i Gabet vide, og 4 à 5 Bøsseskud dybe; den 3die er imod 1/4 Miil viid i Gabet, og 1/16 Miil lang. Den Sydvestlige Side af \textit{Refsbotten-}Fiord er deels slet med Græss og Bierk paa, deels berget, med Lyng paa;\par
Den Nord-ostlige Side er berged; Jnde i Botten boe 5 Søe-\textit{Finner}, og om Sommeren derved ligge 3 Norske-Field-\textit{Finner}.‒\par
Mitt paa Botten af \textit{Refsbotten} ligger\par
\textit{Rüsse-Field} imellem \textit{Rüsse-}Elv og \textit{Gauk}-Elv (hvorimellem er et par Bøsseskud ved deres Udløb i Fiorden) i Sydost l/2 Miil langt, fladt ovenpaa og Lynget, med noget lidet Græss iblant, fladt paa Sidene baade ned ad Elvene, og i Syd-ost ned ad en Myr- og Bierke-Dal, som er 1/4 Miil viid med lidet Græss i.\par
\textit{Finnerne} kalde \textit{Rüsse}-Field og \textit{Motkeduøder}.\par
Østen for \textit{Rüsse}-Field og denne Myr-Dal ligger \textit{Garde-vara}, langt i Øster til Norden hen imod \textit{Reiper-vaag} i \textit{Porsanger}-Fiord 2 1/2 Miil, \hypertarget{Schn1_78278}{}Om Jngøens Præstegield. og 2 Mile over bredt; slet ovenpaa, deels lynget og græsset, deels mosset, deels steenet, paa alle Sider fladtvoren og lynget; Dette \textit{Garde-vara} er Sønden for det store \textit{Kaatemoras}-Field beliggende. ‒\par
Sønden for \textit{Garde-vara} 1 Miil findes\par
\textit{Skaide-Duøder}, og derimellem Bierkeskoug med Græss og Lyng i; Sønden fra denne Bierkeskoug, saa og Sønden fra \textit{Rüsse-vara} og \textit{Sio-vara} er \textit{Skaide-Duøder}, 2 Mile langt i Sønden hen til det Vand \textit{Baste-jaure}, og 1 Miil over bredt; fladt ovenpaa, og paa Sidene, med Lyng, Mosse og noget Græss bevoxen; Dette \textit{Skaide-Duøder} rekker med sin Søndre Ende, som sagt, hen til \textit{Bastejaure}, med sin Østre Side til \textit{Billefiord} i \textit{Porsanger}, med sin Vestre Side ned ad \textit{Baste-jok}, hvorimellem Bierke-Dalen er.\par
Sønden for \textit{Skaide-Duøder} 1/2 Miil begynder\par
\textit{Vorieduder}, hvorimellem er Bierk og qvisted Furre-Skoug, dette \textit{Vorieduder} naaer i Søer hen til \textit{Avjevara} ved \textit{Jetz-jok}, og i Sydost hen imod \textit{Karasjok}; see \textit{pag.} 225 f.\par
J foromtalte Fiord \textit{Refsbotten}, nær ved dens Botten kommer fra Syd-vest følgende Aae, navnlig\par
1. \textit{Master-Elv} af det Field \textit{Sivlijok-Rass}, eller paa \textit{Norsk: Masterelv-Field}, løbendes 1/4 Miil lang i Fiorden.\par
\textit{Masterelv-Field} ligger i Sydost langs med \textit{Refsbott}Fiorden hen imod Botten 1 Miil langt, og paa sin Nordre Ende fra \textit{Refsnæss} adskilt ved en Steenured Dal, 1/2 Miil viid; J Vester stikker det hen til \textit{Repperfiordz} Østre Næss, \textit{Klubben}, til dens Botten 1 Miil langt, og er en Deel af det store\par
\textit{Gierre-niarg}, \textit{pag.} 262 omtalt, ovenpaa, og paa Søndre Side deels Lyng-groet, deels steenet, paa Nordre Side til \textit{Klubben}, Havet og \textit{Refsnæss} steilt og steenet, med smaa Bierk her og der ned paa Søe-Bræden; oventil fladt, men paa Østre Side ad \textit{Refsbotten} med mange runde Knoller ophøyet.\par
En 4 a 5. Bøsse-Skud, fra \textit{Master}Elv i Søer, løber i samme \textit{Refsbotten}-Fiord den Aae\par
\textit{Rüss}-Elv fra Syd-ost af et Vand, \textit{Fatz-jaure}, som ligger paa det Field \textit{Garde-vara}; dette \textit{Fatz-jaure} er rundt, et par Bøsseskud vidt, havendes RøeFisk; Af dette Vand rinder \textit{Rüsse}- Elv ned fra \textit{Garde-vara}, igiennem \textit{Rüsse}-Bierke-Skoug, 1 Miil lang mitt i Botten af \textit{Refsbott}- Fiord; langs efter hvilken \textit{Rüsse}-Elv er mangfoldig Bierke-Skoug paa begge Elvens Sider, et par Bøsseskud breed.\par
J Ost-Syd-ost fra \textit{Gardevara} ere smaa lave Berg-Houger, rundvorne, deels med Lyng deels med Græss bevoxne; Jmellem hvilke Houger ere Vand-Kiønner, deraf Bække rinde i Øster i \textit{Porsanger}fiord, og disse Berg-Houger strekke sig 1/4 Miil vejs hen imod den Jnd-Fiord af \textit{Porsangers}Fiord, navnlig \textit{Smørfiord}. ‒\par
\textit{Gaukelv} kommer imellem \textit{Master}-Elv og \textit{Rüss}-Elv, 2 Bøsse-Skud Vesten for denne, i \textit{Refsfiord}-Botten, fra Syd-ost af det Field \textit{Siova}, løbendes 1 Miil lang mest i Nord i Fiorden.\par
J denne \textit{Refsbotten}-Fiord paa dens Østre Side, tvert oven for \textit{Liønæss}, er den Jnd-Fiord\par
\textit{Lill-Fiord}, i Gabet nogle Bøsse-Skud viid, i Syd-ost 1/8 Miil lang, i hvis Botten en liden Bæk indløber.\hypertarget{Schn1_78562}{}Schnitlers Protokoller V.\par
Norden for denne \textit{Lill-Fiord} 1/8 Miil er \textit{Slotten}, en Bugt, 3/8 Miil i Gabet viid, 1/8 Miil lang. Den Østre Side af \textit{Refsbott}-Fiorden er berged, steen-ured, med BierkeSkoug imellem sommestedz; Den Field-Strekning imellem \textit{Lill-Fiord} og \textit{Refsfiords}-Botten paa Østre Side er 3/8 Miil langt i Syd-ost, og kaldes\par
\textit{Raakie-vara}, 1/8 Miil over bredt, ovenpaa Lynget; Af dette \textit{Raakievara} stikker et Næss, navnl. \textit{Raakie-Niarg} i Nord-vest ud i Fiorden, tæt Sønden for \textit{Lill-Fiorden}, hvortil dette \textit{Raakie-Niarg} giør det Søndre Næss.\par
Østen for dette \textit{Raakie-vara} er en kort Myrdal med Bierk i, 1 Bøsseskud over viid; og Østen for Dalen\par
\textit{Kaattemore-Rass}, et Field, strekkende sig i Nord, og Nord-ost hen til Søe-Kusten, nemlig ad \textit{Sneefiord, Bagfiord, Myrfiord, Trollfiord}, \textit{Kull}fiord, \textit{Rygg-Fiord} og \textit{Kobbe}-Fiord, 1 Miil i Øster nær \textit{Reper-Vaag} i \textit{Porsanger}Fiord, 2 à 3 Mile vidt, slet ovenpaa, steen-uret, med nogle smaa runde Knoller paa; Paa dette \textit{Kaattemor-Rass} holde ikke engang Field-\textit{Finner} til om Sommeren, fordi der er intet Græss og Bierk paa, undtagen paa den Østre Side ned ad \textit{Porsanger}Fiord.\par
Landskabet imellem \textit{Slotten}-Bugt, og \textit{Saunæss} er berget med Steen-Urer og Lyng paa.\par
J denne \textit{Refsbotten}-Fiord boer Jngen paa Fiord-Brædene, uden 5 Søe-\textit{Finner} inde ved Botten, og om Sommeren 3 Field-\textit{Finner} ovenfor Botten med deres Reen.\par
Fra \textit{Refsbottens} Østre Næss, \textit{Saunæss} naar fares i Nord-Nord-ost 1/2 Miil, møder\par
\textit{Sneefiord}; Landet derimellem er berget og u-beboet, kaldes i Almindelighed ‒\par
\textit{Jakke-niarg}, fladt ovenpaa, deels steenet, deels Lyng-groet; Et Næss fra dette \textit{Jakkeniarg} udgaaer fra \textit{Saunæss} 1/4 Miil i Nord-Nord-ost, i Havet, navnlig \textit{Vestre Bursta-Næss}; 1/4 Miil herfra i Nord-ost er \textit{Østre Bursta-Næss}, som giør det Nord-vestlige Næss af \textit{Sneefiord}; Jmellem disse 2de \textit{Bursta-Næsser} er en Bugt,\par
\textit{Bursta-Bugten}, 1 Steenkast over i Gabet viid, et par Bøsse-Skud i Syd-ost lang. Strax om \textit{Østre Bursta-Næss} aabner sig \textit{Sneefiord}; dens Østre Næss heder \textit{Afløsning}, eller paa \textit{Finnsk: Gusket}, derimellem Gabet er knap 1/2 Miil vidt.\par
\textit{Sneefiord} er lang ind ad fra sit Vestre Næss, \textit{Bursta-Næss} 3/4 Miil, og fra sit Østre Næss, \textit{Afløsning}, knap 1/2 Miil i Søer ad Botten.\par
\textit{Bursta-Næss} er rundt, noget høyt, bart og steenet, et par Bøsse-skud stort.\par
\textit{Afløsnings-Næss} er og rundt, ikke saa høyt, som \textit{BurstaNæss}, steenuret, med noget Lyng paa.\par
Den Vestre Fiord-Bræde fra \textit{Bursta-Næss} til Botten er deels steened, deels græssed, hvor de Folk fra \textit{Reenøe} slaae Græsset. Den Østre Side er mest steened, dog nær inden for \textit{Afløsning} og nær ved Botten er noget græssed; denne \textit{Sneefiord} er ubeboed.\par
J denne \textit{Sneefiord} komme Sønden-fra 2de Bække:\par
Den 1te navnlig \textit{Sneefiord}-Elv rinder af \textit{Sneefiord-Vatten} under det Field \textit{KaattermorRass}, et par Bøsseskud lang i Nord i, \textit{Sneefiorden; Sneefiord-Vatten} er rundt, 2 Bøsseskud over stort, havendes Røe og Lax.\par
Den 2den Bæk heder paa \textit{Finnsk}\textit{Giedejok}, kommendes af Kiønner neden under \textit{Kaattemor-Rass}, og løber i Nordvest 1/4 Miil lang i \textit{Sneefiord}-Botten, et par Bøsseskud Østen for den næstforrige; hvorj samme Fisk fanges.\hypertarget{Schn1_78867}{}Om Jngøens Præstegield.\par
Jmellem \textit{Sneefiord}-Botten, og \textit{Slotten}-Bugt i \textit{Refsbotten} er et Eid, 1/8 Miil langt, overgaaendes. Strax Østen om \textit{Afløsnings}-Næsset er\par
\textit{Bagfiord}, hvis Venstre Næss er \textit{Bagfiord-Næss}, rund, noget høyt, steenet, 2 Bøsseskud over stort; det Nord-ostlige Næss heder \textit{Siarve-Niarg}, skabt som det Vestre; hvorimellem Gabet er imod 1/4 Miil vidt; \textit{Bagfiord} er i Syd-ost dyb mod 1/4 Mil, ubeboed. Fiord-Brædene bestaae af steenede Berge, uden Gress og Bierk.\par
Fra \textit{Bagfiords} Nordostl. Næss \textit{Siarve-Niarg} til \textit{Myrfiord} er 1/4 Miil; Mitt derimellem ligger \textit{Selviig}, et par Bøsseskud viid, og dyb i Øster, ubeboed; Herinde har de Mænd paa \textit{Fladtøe} 1/8 Miil Vesten derfor, deres Høe-Slotte. Fra \textit{Bagfiord} til \textit{Selviig} er Landet steenberget, ligesaa fra \textit{Selvig} i Nord til \textit{Myrfiord}.\par
\textit{Myrfiords} Søndre Næss er \textit{Baltze-Niarg}, det Nordre Næss \textit{Myr-Nipen} noget høyt, imod \textit{Havesund} 1/4 Miil langt, steenet og steilt ad Sundet. Til \textit{Myr-Nipen} fra \textit{Refsnæss}, gode 2 Miile i N.N.O. gaaer \textit{Jngens} Gieldz faste Landz \textit{district}. Dog boer Jngen der uden i \textit{Refsbotten}. ‒\hspace{1em}\par
til 3 Sp: Svar: Fisk og Fugl haves her som i \textit{Hammerfest} ‒ \textit{pag.} 255.\hspace{1em}\par
til 4de Sp: Svar: Havne er ikke i dette Gield, uden som ved \textit{Rolfsfiord}\textit{pag.} 259 er ommeldt.\hspace{1em}\par
til 5te ‒ 6. Sp. Svar: Fiorder og Elver i \textit{Jngens} Gield ere ved Øerne og det faste Land beskrevne.\hspace{1em}\par
til 7de Sp. Svar: Myrer og Dale beste er ved \textit{Rüsse}-Elven Sønden for \textit{Refsbotten}, strekkende sig i Ost-Syd-ost henimod \textit{Smørfiord} i \textit{Porsanger} imod 2 Mile ‒\hspace{1em}\par
til 8de Sp. Svar: Om Gaardenes Avling og \textit{Creaturer} som i \textit{Hammerfest}, pag. 256.\hspace{1em}\par
til 9de Sp. Svar: Skoug findes ikke paa Øerne; hvorfore Bønderne der brænde Torv' og boe i Jord Gammer, indentil med Bræder beklædt; \textit{Finnernes} Gammer er alleene af Bierke Stænger opreist, og med Steen og Jord udentil dekket.\par
Paa faste Land er alleene smaa Skoug af Bierk, Older, Silie og Rogn, hvilke sidste bruges til Skav for Creaturene; Af disse Bierke-Skouge er nu de fornemste (1) efter \textit{Gaukelven} i Søer 1/2 Miil lang, 9 à 10 Bøsseskud breed. Den 2den efter \textit{Rüss}-Elven, gaaer i 2de Deele, den Eene OstSydost ad \textit{Smørfiord} i \textit{Porsangen}, ved 2 Mile lang, 1/4 Miil og mindre breed; den Anden i Søer, Østen forbi \textit{Skaide-Duøder}, langs med \textit{Porsanger}-Fiord, siden Østen forbj \textit{Vorieduder}, alt hen til \textit{Karasjok}; Paa den anden Side af \textit{Karasjok} bliver den jo længer jo større, og vedvarer, efter at den Furre haver til sig taget, u-endelig-viid ind ad \textit{Sverrig} og \textit{Rusland}. Breed er denne Skoug imellem \textit{Gardevara} og \textit{Skaide-duøder} 1 Miil; Jmellem \textit{Skaide-duøder} og \textit{Porsanger}fiord 1/4 Miil; Østen for \textit{Vorieduder} meene nærværende Vidner at den udbreder sig over 1. Miil; Videre i Søer og Øster ei kiendte denne Skoug.\hspace{1em}\par
Sp. 10. Svar: Vild og \textit{Jnsecter}, som i \textit{Alten}\textit{pag.} 228 f. undtagen de Fiir-Beeninger, hvilke holde til i Furreskoug, dem vide de her ei af; Dog sige, at Reenen om Sommeren til Fieldz og i Dalene plages sterk af Klægg og Mygg\hypertarget{Schn1_79157}{}Schnitlers Protokoller V.\par
Sp. 11. Sv: Leilighed til Rødning her i Gieldet meenes at være ved \textit{Master}-Elv, og \textit{Gauk}- Elven; fremdeles inde paa Botten af \textit{Refsbottens}Fiord paa Østre Side ved den Viig, \textit{Selkop}\hspace{1em}\par
Sp. 12. Svar: Fieldene ere ved det faste Land og Øerne beskrevne.\hspace{1em}\par
Sp. 13. Sv: Jngen Kiøbstad her er: men paa \textit{Maasøe} er et \textit{octrojeret} Handelsted, som forsiuner \textit{Jngens} Præstegield, og \textit{Maasøe Annex} af \textit{Kielvigens} Gield\hspace{1em}\par
Sp. 14. Sv: Næring søges her som i \textit{Hammerfest}\textit{p.} 256.\hspace{1em}\par
Sp. 15. Sv. \textit{Mineralia} og \textit{Naturalia} vides ei af.\hspace{1em}\par
Sp. 16. Sv: De nærmeste Gaarder til \textit{Jngens} Præstegield i Vester paa faste Land er \textit{Repperfiord}, i Syd-vest 1 Miil fra \textit{Refsnæss}. Paa \textit{Hvaløe} den Jord, \textit{Forsell}, i Vest-Nord-vest 1 sterk \textit{Finmarksk} Miil fra \textit{Refsnæss}, men fra \textit{Jngens} Jorder i Syd-Syd-vest gode 2 1/2 Mile; Hvilken \textit{Repper}fiord og \textit{Forsell} høre til \textit{Hammerfestes} Gield;\par
De nærmeste Gaarder til \textit{Jngens} Gield i Øster af \textit{Kielvigens} Præstegield ere paa det faste Land i \textit{Kobbefiord}, til \textit{Jngens} Gieldz Gaarder i \textit{Refsbotten} i Ost-Nord-ost et Par Mile: men til \textit{Jngens} Gieldz\textit{Districtes} Ende, nemlig til \textit{Myr-nipen} i Ost-Sydost 2 Mile; På \textit{Magerøen} er \textit{Graakollen}-gaard nærmest til \textit{Tuënæss}, i Syd-Sy[d]vest 1 sterk Miil derfra ‒\hspace{1em}\par
Sp. 17. Svar: Vei fra \textit{Refsbotten} op til Grændsen fares ikke af Bønder-Folk: men alleene af Field-\textit{Finner}, nemlig fra \textit{Refsbotten} først til \textit{Avjevara} Markested, som de meene j SydSydost, Paa hvilken Vej \textit{Finnerne} gemeenligen tilbringe 3 Dage; hvorlangt fra \textit{Avjevara} i Søer er til nærmeste Grændse-Sted? vides ikke.\hspace{1em}\par
Sp. 18. og 19. Svar: \textit{Norske} Field\textit{finner} af \textit{Refsbotten}-Fielde, kiende de 3de hvilke om Vinteren holde til i \textit{Karasjok} ved Markesteden, og svare Skatt til den Kongl. Norske Foged, men intet noget til \textit{Sverrigs} Crone; Hvormange af de fælles Field\textit{lapper} i disse \textit{Jngens} Norske Fielde om Sommeren ligge, vide de ikke: men ved \textit{Refs}botten ere om Sommeren deraf 4. Mand.\hspace{1em}\par
Sp. 20. Sv: vides ej af.\hspace{1em}\par
Sp. 21. Sv: Vidnerne have altid hørt, at den FieldStrekning Norden for \textit{Jetzjok, Avjevara}, og \textit{Karasjok} ned til Søen, uden nogen Modsigelse have tilhørt de \textit{Norske} Field-\textit{Finner} alleene, uden at de have bekymret sig, eller skiøttet om den Svenske Field-Øvrighed: Men Sønden for \textit{Jetzjok}, og \textit{Karasjok} have de hørt, at fælles field\textit{finner} have tilholdet op til \textit{Kiølen}.\hspace{1em}\par
Sp. 22. Sv: Naar dette Fælleskab i Fieldene Norden for \textit{Kiølen} har taget sin Begyndelse? vide de ei: Men det huuse begge nærværende Vidner, som ere Brødere, at for omtrent 40 Aar siden har deres Fader, hos hvilken de da vare, arbeidet paa \textit{Kautokeino-Lappe}-Kirke, og Kørt Timmer dertil; Hvorfor han ligesom de andre Field-\textit{finner}, ingen Betaling nød; Førend denne \textit{Kautokeino}-Kirke blev bygd, var der kun en Forsamlings Gamme, hvorj den Svenske Præst forrettede Gudztienneste.\hspace{1em}\par
Sp. 23. Sv: vides ei.\hspace{1em}\par
Sp. 24. Om \textit{Grændsegangen} blev nu hver i sær adspurdt:\hypertarget{Schn1_79478}{}Om Jngøens Præstegield.\par
\centerline{\textbf{12te Vidne i Finmarken}\textit{Anders Ammondsen}, den gamle:}\par
født i Norge Østen for \textit{Porsanger} af \textit{Lappe}-Forældre; hans Fader har, som en Fieldfinn først siddet i \textit{Kautokeino} Fielde, siden fløttet ned til SøeKanten, og bleven en \textit{Norsk} SkatteSøefinn; døbt er Vidnet i den \textit{Norske}\textit{Kielvig}Kirke, 52 Aar nu gammel; Mens han var ung, foer han med sin Fader i \textit{Kautokeino}-fielde; Gift med en \textit{Norsk Finne-}Qvinde, har 8. Børn, er nu en \textit{Norsk} Søe Finn, boendes i \textit{Refsbotten}; været sidst til Gudz Bord i den \textit{Norske}\textit{Hammerfest-}Kirke sidst afvigte \textit{St Hans}Tid.\hspace{1em}\par
Om Grændsen imellem \textit{Norge} og \textit{Sverrig} forklarer han sig saaledes, at han ikke har vanket der selv, eller kan udviise dem, hvor de ligge, men af hans Fader, som havde været en Field-\textit{Lap}, og af gamle Folk har han hørt, at GrændseGangen og Landz\textit{kiølen} imellem \textit{Norge} og \textit{Sverrig} giøre følgende Fielde og Merker: \centerline{1. \textbf{Halde} 2. \textbf{Nappetiøve} 3. \textbf{Samas-oive} 4. \textbf{Vard-oive} 5. \textbf{Bierte-vara} 6. \textbf{Tenomutkie} 7. \textbf{Korsevare} 8. \textbf{Aakie-vare} 9. \textbf{Koskatmutkie} 10. \textbf{Nerrevarda} 11. \textbf{Posa-vara} 12. \textbf{Urdevara} 13. \textbf{Maseljaure} 14. \textbf{Skier-oive}}{15. \textbf{Pitsekiulbme}, her har han vel faret over for lang Tid} siden i sin Barndom: men han mindes det nu ikke, hvorledes det seer ud, eller hvor det er. \centerline{16. \textbf{Kieldevadda} 17. \textbf{Salvasvadda} 18. \textbf{Kierresvara} 19. \textbf{Keurisvara} 24. \textbf{Modtatas-oive}}\par
De Mellemværende fra N. 20. til N. 23. ei hørt om \centerline{29. \textbf{Raudo-ovie} 33. \textbf{Laddegein-oive}}\par
De Mellemværende Nr ei veed af, at have hørt. Denne opreignede \textit{Kiøl}-gang har han hørt, og veed af ingen Anden, der skal skille \textit{Norges} og \textit{Sverrigs} Grændser.\par
Sp. 25. og 26. Sv. han veed ej.\par
Sp. 27. Svaret før ved Sp. 17.\hypertarget{Schn1_79692}{}Schnitlers Protokoller V.\par
Sp. 28. Sv: Han har ikke hørt til nogen Tvistighed om \textit{Kiølen}, mens vel at de Fælles Field-\textit{Finner}, som tilholde imellem \textit{Kiølen} i Søer og \textit{Avjevara} samt \textit{Karasjok} Markesteder i Nord vill formeene de \textit{privative-Norske} Field\textit{finner} Norden for \textit{Avjevara} og \textit{Karasjok}, at skyde Vild Reen, og at fange Ryper, imellem besagde Markesteder og Grændse-\textit{Kiølen}.\hspace{1em}\par
Sp. 29. 30. Sv. det samme som 8de Vidne \textit{pag.} 239.\hspace{1em}\par
Sp. 31. Sv. De nærmeste Field-\textit{Lapper} i Øster herfra ere de Sønden for \textit{Porsangers}Botten.\hspace{1em}\par
Sp. 32. Sv. som 8de Vidne \textit{pag.} 239.\hspace{1em}\par
Sp. 33. Sv. vides ei.\hspace{1em}\par
Hvorpaa han blev \textit{dimittered}.\par
\centerline{\textbf{13de Vidne i Finmarken}\textit{Anders Ammondsen} den Yngre,}\par
Broder til næstforrige, født i \textit{Sverrigs}\textit{Jokasjarfs Lappe} Præstegield i \textit{Torne Lapmark}, af \textit{Lappe}-Forældre, som næstforrige, døbt i \textit{Jokasjarfs Lappe}Kirke, 49. Aar gammel, gift, har 5 Børn, har faret med sin Fader til Fieldz, da han var ung, nu Søe-Finn i den \textit{Norske} Fiord \textit{Refsbotten}, været sidst til Gudz Bord for 3 Uger siden i den \textit{Norske}\textit{Hammerfest}-Kirke i \textit{Vest- Finmarken}.\par
\centerline{Om \textbf{Grændse-Gangen}}\par
imellem \textit{Norge} og \textit{Sverrig} vidner han det samme, som hans Broder, det 12te Vidne, og til de følgende Spørsmaale svarer med ham det samme, indtil Enden.\par
Hvorpaa Vidnerne bleve \textit{dimitterede} med den Beskeeden: Naar de faae \textit{Ordre}, skulle de møde de Kongelige \textit{Norske} til Opmaalingen \textit{Committerede Officerer}, ved disse deres Ankomst paa Field-Ryggen, paa det Grændse-Field \textit{Halde}, som det 1te, hvorpaa de have aflagt deres Vidne; Og saa blev Retten her slutted, og af LaugRettesMænd underskreved og forseigled. \textit{Fladøe} d. 19 \textit{Sept}: 1744.\hspace{1em}\par
\centerline{Peter Schnitler}\centerline{L. S. Beles Haluersen}\centerline{L. S. søren darneksen (?)}\hspace{1em}\par
Siden disse Vidner alleene have hørt, men ei selv befaret den bevidnede Grændzegang, saa angav Vidnet Anders Ammondsen den ældre, følgende kyndige Field-\textit{Finner}, som kunde tiene til Vej-viisere for de Kongelige \textit{Norske Jngenieurer}, (1) \textit{Sara Ammondsen}, Broder til sidst-benævnte 2de Vidner, (2) \textit{Ammond Olsen}, begge \textit{Norske} Skatte-Field\textit{finner}, men nu for \hypertarget{Schn1_79931}{}Om Jngøens Præstegield. langt fraværende oppe ved \textit{Karasjok}, den \textit{Svenske} Markested, hvoraf den 1te om Sommeren oppholder sig i den Norske \textit{Snee}-Fiord i \textit{Jngens} Gield, den 2den i den Norske \textit{Kobbe}-Fiord i \textit{Kielvigens} Præstegield, (3) \textit{Ammond Nielsen}, som for begangne Blodskam med sin Kones Søster sidder i Kielvigen ‒ og (4) \textit{Morten Olsen}, som for samme Misgierning sidder paa \textit{Hammerfest} arrestered.\hspace{1em}\par
\centerline{Peter Schnitler}\centerline{L. S. Lars Monsenflaøen}\centerline{L. S. Simmon SimmonsenMasøe}\hspace{1em}
\DivII[Sept. 21.-22. Fra Latøy til Kjelvik]{Sept. 21.-22. Fra Latøy til Kjelvik}\label{Schn1_80007}\par
1744. d. 21 \textit{Sept:} Foer man fra \textit{Fladøe} i OstNordost til \textit{Havesund}{3/4 Miil} Fra \textit{Havesund} forbi \textit{Kulfiord} paa Høyere Haand i Øster til \textit{Maas-øe}{3/4 ‒ _______ 1 1/2 Miil} hvor man erkyndigede sig om Landets Leje.\par
d. 22 \textit{Sept.} fra \textit{Maasøe} i Øster til Norden, paa den høyere Haand ladendes \textit{Rygg-Fiord, Kobbe-Fiord}, igiennem \textit{Magerøe-Sund}, siden forbi \textit{Lafiord og Kaafiord}; derfra i Nordost til \textit{Kielvigen} paa \textit{Magerøe}{2 1/2 ‒ _______ 4 Miil} hvorfra man giorde Bud ind ad \textit{Porsanger}-Fiord, at faae Kyndige \textit{Finner} til Vidner. ‒\hspace{1em}
\DivII[Oplysninger om Kjelvik prestegjeld]{Oplysninger om Kjelvik prestegjeld}\label{Schn1_80092}\par
Her beskrives nu \textbf{Kielvigens Præstegield} for i Vejen:\par
til 1 Sp. Svar: \textit{Kielvigen}, det sidste Præstegield i \textit{Vest-Finmarken}, ligger nærmest Østen for \textit{Jng-øens} Præstegield, og Vesten for \textit{Kiøllefiords} ‒ det 1te Præstegield i \textit{Ost-Finmarken}, har norden for sig \textit{Nord-Søen}, og Sønden for sig \textit{Karasjok}, den nu værende \textit{Svenske Lappe}- Markested;\par
\textit{Kielvigens} Præstegield har\par
1. \textit{Kielvigens} HovedKirke paa \textit{Magerøe}, mitt paa dens Østre Side.\par
2. \textit{Maasøes Annex}Kirke paa \textit{Maasøens} Østre Sides Søndre Ende, 2 1/2 \textit{ordinaire} Mile i Vester til Sønden fra \textit{Kielvigens} HovedKirke.\par
3. \textit{Kistram Finne-Capell} inde i \textit{Porsanger}Fiord, paa dens \textit{Vestre} Side 3 1/2 Mile i Syd- Syd-vest fra \textit{Porsanger}Fiordz Vestre Næss, og 4 1/2 Mile ligeledes i Syd-Syd-vest fra \textit{Kielvigens} Hoved-Kirke; Alle 3. af Timmer bygde.\par
Forrige \textit{Sverholts Annex}Kirke er for Mangel af Tilhørere fløtted til \textit{Laxefiorden}.\label{Schn1_80211} \par 
\begin{longtable}{P{0.30546875\textwidth}P{0.0265625\textwidth}P{0.365234375\textwidth}P{0.059765624999999996\textwidth}P{0.09296875\textwidth}}
 \hline\endfoot\hline\endlastfoot \textit{Kielvigens} HovedSogn bestaaer af\tabcellsep \tabcellsep \textit{Normænd}\tabcellsep 32\tabcellsep \textit{Mænd}\\
\textit{Maasøe-Annex} af\tabcellsep 17\tabcellsep \textit{Normænd}\\
\tabcellsep 8\tabcellsep Søe\textit{finner}\\
\tabcellsep 2\tabcellsep Field\textit{finner},\tabcellsep 27\tabcellsep \textit{Mænd}\\
\textit{Kistrams Capell} af\tabcellsep 34\tabcellsep Søe\textit{finner}\\
\tabcellsep 13\tabcellsep Field\textit{finner}\tabcellsep 47\tabcellsep ‒\\
\tabcellsep \tabcellsep \tabcellsep \multicolumn{2}{l}{_________}\\
\tabcellsep \tabcellsep =\tabcellsep 106\tabcellsep ‒\end{longtable} \par
 \hypertarget{Schn1_80309}{}Schnitlers Protokoller V.\par
De Sønderste Gaarder av \textit{Kielvigens} Præstegield paa det faste Land ere:\par
i Vester \textit{Klipfisk-viig} i \textit{KobbeFiord}, og \textit{Lax-elven} i \textit{Porsangers} Botten,\par
i Øster er \textit{Haarviig} paa Østre LandSide af \textit{Porsanger}fiord, fra \textit{Kobbefiord} i Øster 2 Mile; Fra \textit{Porsangers} Botten i N.N.O. 6 1/2 Mil\par
De Nordreste Gaarder af \textit{Kielvigen} ere\par
i Vester paa \textit{Maasøe}, fra \textit{Kobbe}-Fiordz Botten i Nord-nordvest 1 1/4 Miil;\par
i Øster paa \textit{Magerøes} Østre Side \textit{Kamøen}, fra \textit{Maasøe} i Øster til Norden 2 Mile, fra \textit{Haarvig} i Nord-Nordvest 3 Mile.\par
til 2 Sp: Svar: \textit{Kielvigens} Gield bestaaer deels af Øer, deels af fast Land;\par
Øer ere:\par
\centerline{1. \textbf{Maasøe}}\par
Den ligger fra det faste Land, nemlig fra \textit{Fiskenærings}-Næss i Nord knap 1/8 Miil, fra \textit{Jelmesøe} i Øster til Sønden 1/8 Miil, fra \textit{Stappen}-Øer i Syd-vest 1 Miil; \textit{Maasøe} er fra Søer i Nord 3/8 Miil lang, smal i Søndre Ende, bredere til 1/4 Miil i Nordre Ende, mest klipagtig, med Lyng og lidet Græss paa, uden Skoug; har en \textit{Annex}Kirke af Træ, under \textit{Kielvigens} Præstegield, paa Søndre Endes Østre Side, et \textit{octrojeret} Handelsted, og desuden 12 Jndbyggere, \textit{Normænd}. Paa Nordre Side af denne Maasøe indgaaer ‒\par
\textit{Sandfiord}, 1/16 Miil i Gabet viid, 1/32 Miil dyb i Søer; det Vestre Næss heraf heder \textit{Troll}- Houg, meget høyt, langagtigt efter Fiorden indtil Botten, steen-uret halv- og græsset den anden halve Deel, steilt paa Vestre og Østre Sider, ellers fladtvoren.\par
Det Østre Næss heder \textit{Olmenæss}, ikke saa høyt, som \textit{Troll}Houg, rundagtigt, fladt ovenpaa, steenet neden til, men oventil græsset og lynget, 1/16 Miil stort.\par
\textit{Maasøe-Kalv} ligger 1/16 Miil fra \textit{Maasøe} i Nordvest, rund, 1/16 Miil omkring stor, paa SøeBrædene berged og steened, men ovenpaa flad, lynggroed, med lidet Græss imellem, ubeboed, bruges af \textit{Maasøe}-Boerne.\par
\centerline{2. \textbf{Magerøe}}\par
ligger fra \textit{Maasøe} i Øster en stor 1/2 Miil, fra det faste Land i Nord 1/8 Miil og mindre (hvorfra den ved \textit{Magerøe}-Sund adskilles), fra \textit{Stappen}-Øer deels i Øster, deels i Søer 1/4 Miil, hvorimellem mange u-beboede Holmer og Skiær ere, har paa Østre Side \textit{Porsanger} Fiord, og paa Nordre Side den vide \textit{Nord-Søe};\par
\textit{Magerøe} er lang fra Vester i Øster 2 Mile, og fra Søer i Nord ligesaa breed, mest klipped af høye flade skallede Fielde, med noget Græss-Land paa Søndre og Østre Sider, beboed af 25 \textit{Normænd}.\par
Har Fiorder paa Vestre Side Norden fra at reigne:\par
1. \textit{Lyng-poll}; dens Nordre Næss heder \textit{Maanæss}, dens Søndre Næss \textit{Kiønnæss}; \textit{Maanæss} er fra Nord-vest i Ost-Syd-ost indtil \textit{Lyngpolls} Botten 3/8 Miil langt, og fra Vester i Øster 1/16 Miil bredt, paa Østre Side ind ad Øen bratt og steenuret, men paa de andre Sider ad Havet og Fiorden fladtvoren, med Græss nedenpaa, men oventil bart og steenet, meget høyt, ovenpaa fladt og bart. ‒\hypertarget{Schn1_80557}{}Om Jngøens Præstegield.\par
\textit{Kiønnæss} er en smal udstikkendes Odde, 1 Bøsseskud over stort, ikke høyt, ovenpaa slet og græsset, paa Sidene fladtvoren med Græss begroet.\par
J mellem disse \textit{Maanæss} og \textit{Kiønnæss} er \textit{Lyng-poll} i Gabet 1/4 Miil viid, og 3/8 Miil i Sydost dyb; Den Sydvestlige Side af \textit{Lyng-poll} er høyklipped, steen-ured, bratt ned ad Fiorden; J hvilken Fiord ingen Skoug er, og ingen Folk boer. ‒ J Botten af denne \textit{Lyngpoll} kommer Østen-fra en Aae, navnlig \textit{Lyng-Poll}-Elv, 1/4 Miil lang, af det Vand, \textit{Lyngpollvatten}, som er rundagtigt, 4 Bøsseskud over stort. ‒ Fra \textit{Lyngpoll} i Søer til \textit{Vandfiord} er 3/8 Miil; Herimellem, nemlig 1/16 Miil nær \textit{Vandfiord} er en Viig,\par
2. \textit{Graakoll} kalded, 1/16 Miil i Munden viid, 1/8 Miil i Øster dyb. Landskabet imellem \textit{Graakoll} og \textit{Lyngpoll} er berget, paa Nordre Deel ad \textit{Lyngpoll} steilt, paa Søndre Deel ad \textit{Graakoll} fladtagtigt ned ad, men steilt oventil, og bestaaer af \textit{Kiønnæss} Field, som rekker i Søer hen til \textit{Graakoll}, og giør dertil det Nordre Næss; Det Søndre Næss af \textit{Graakoll} er \textit{Supavig-Næss}, som strekker sig, Østen forbi \textit{Supavig}, i Søer hen imod \textit{Vandfiords} Nordre Næss; Den Nordre Fiord-Bræde af \textit{Graakoll} er steenberged, u-slett; den Søndre Fiord-Bræde er fladagtig neden til, med lidet Græss paa, høyt og bratt oventil, og bart; Jndj ‒ og ved Botten er Landet deels berget, deels fladt, med lidet Græss paa, hvor 4 \textit{Normænd} boe. ‒\par
3. \textit{Vandfiord}, som følger i Søer paa \textit{Graakollen}, har til Nordre Næss \textit{Vandfiord-Næring}, rundagtigt, et par Bøsseskud over stort, meget høyt, fladt oventil, steenet overalt. ‒ Det Søndre Næss af \textit{Vandfiord} heder \textit{Strømsnæss}, spidz udstikkendes ad Søen, høyt, fladt ovenpaa bratt paa Sidene baade ud til Søen, og Fiorden, steenet; Jmellem disse Næss er 1/4 Miil, at \textit{Vandfiord} i Gabet er viid, og 1/2 Miil er den i Nord-ost, og omsider i Oster dyb, u-beboed; FiordBrædene ere fladagtige paa Nordre Side alt hen til Botten, og græss-groede, paa Søndre Side flad og græsset til Mitt paa Fiorden, siden ind ad steilt og berged.\par
Fra \textit{Vandfiord} fares i Syd-ost til den Bugt \textit{Svarthull} god 3/8 Miil; Herimellem, nemlig 1/8 Miil fra \textit{Vandfiord}, ligger den Viig,\par
4. \textit{Finn-viig}, dens Nordre Næss heder \textit{Sprængnæss}, Land Strekningen imellem \textit{Vandfiord} og dette \textit{Sprængnæss} udgiør forbemeldte \textit{Strømsnæss}, bart, og ubeboet; \textit{Sprængnæss} er rundagtigt, 1 Bøsseskud over stort, ikke høyt, ovenpaa fladt, og Lynget, paa Sidene steenet og noget bratt. Det Søndre Næss af \textit{Finnviig} er \textit{Ku-næss}, dannet omtrent, som \textit{Sprængnæss}; Jmellem disse 2de Næss er \textit{Finnviig} 2 Bøsse-Skud i Gabet vidt, og ligesaa dybt i Nord-ost. Landet af \textit{Finnvigens} Søe-Bræder er sletagtigt, med Græss paa, hvor Bøndernes Gammer staae, ellers lynget; Oven derfor er det berget og steenet; Her boe 3 \textit{Normænd}, uden Skoug. ‒ 1/4 Miil fra \textit{Finnviig} er til den Bugt\par
5. \textit{Svarthull}; dens Vestre Næss er \textit{Skat-øren}, dens Østre Næss \textit{Taggen}, hvorimellem Bugten er 1/4 Miil viid. ‒\par
\textit{Skat-øren} er kort-fladt ud til \textit{Magerøe}-Sund, siden gaaer steilt, og spidz op ad, meget høyt, graasteenet overalt, knap 1/8 Miil stort.‒\par
\textit{Taggen} er kun et par Steen-Kast over stort, noget høyt, steilt og steen-uret;\par
\textit{Svarthull} er 5 à 6 Bøsseskud dyb i Nord, og runden omgivet med Field, hvori nogle smaa Græss-Lier, u-beboet.\par
At merke, at\hypertarget{Schn1_80803}{}Schnitlers Protokoller V.\par
\textit{Magerøe-Sund} stikker i Ost-Syd-ost, og begynder i Vester fra \textit{Vandfiord-Næringen} til \textit{Skat-øren} paa \textit{Magerøe}-Side 1/2 Miil langt; og fra \textit{Stikkel-Næring} til \textit{Lafiord} paa det faste Land 3/8 Miil langt; Bredt er det imellem det faste Land og \textit{Magerøe} 1/8 Miil, og mindre. ‒\par
Fra \textit{Taggen} til den Fiord \textit{Ballhopen} er i Øster 1/8 Miil;\par
6. \textit{Ballhopens} Vestre Næss heder \textit{Veinæss}, med \textit{Taggen} sammenhengendes; Dens Øst{re} Næss er \textit{Sarnæss}; Hvorimellem \textit{Ballhopen} i Gabet en knap 1/4 Miil er breed, stikkendes 3/8 Miil ind i Nord til Osten til Botten. \textit{Veinæss} er rundt, kantet ud til Søen, 1 Bøsseskud over vidt, noget høyt, ovenpaa fladt, og Lyng-groet. \textit{Sarnæss} er langs med Søen 1 Bøsseskud over stort, lavt, fladt, og græsset; Landet inde i Fiorden paa Fiord-Brædene er slet og med Græss bevoxen, men oven derfor berget, uden Skoug; Herinde, under \textit{Sarnæss} boe 2 \textit{Normænd}.\par
Strax om \textit{Sarnæss} i Øster møder\par
7. \textit{Sæter-vaagen}, et par Bøsse-Skud over ad Øster i Gabet breed, til det Østre Næss \textit{Sætervaag-Næss}; Denne Vaag er 4 Bøsseskud dyb, krogendes sig i Vester, u-beboed; SøeBrædene ere flade, uden Græss og Skoug, med Lyng paa, men siden stige de op i Vejret til steenede Berge.\par
1/4 Miil Østen for \textit{Sætervaag-Næsset} ligger \textit{Steenvig-Næss}; Landet derimellem er berget og Græss-liet, og under Græss-Liene er det slet ned ad Søen; Dette \textit{Steenvig-Næss} er det Vestre Næss af\par
8. \textit{Søer-Honning-vaag}; dens Østre Næss heder \textit{Juledag}, deraf, at paa en Jule-Dag har en Baad, med Kirke-Folk i, stødt paa dette Næss, og Folkene ere druknede; Denne \textit{Søer- Honning-Vaag} er i Gabet knap 1/4 Miil viid, og 1/4 Miil dyb i Nord; Landskabet inde i denne Vaag er klippet, og mitt paa Østere Side indgaaer en Viig i Ost-Nord-ost, ved 4 Bøsseskud viid, og ligesaa dyb, hvor en Vinter-Havn er for en 2 Skibe.\par
Fra Bonden af denne Vaag gaaer et Eid over Land 1/8 Miil vidt til \textit{Skibfiord}. ‒ Næstforomm.te \textit{Juledag-Næss} er lavt, hvast, steenet, fra Søer i Nord 2 Bøsseskud langt, fra Vester i Øster 1 Bøsseskud over bredt.\par
Fra \textit{Juledag-Næss} i Øster til \textit{Nordre Honning-Vaag} er 1/16 Miil, hvorimellem er en kort Viig, omkring hvilken Landet er steenet og steilt.\par
9. \textit{Nordre-Honning-Vaag} er fra Vester i Øster til Nord 3 à 4 Bøsseskud viid, og 1/16 Miil lang i Nord til Botten; Dens Vestre Næss heder \textit{Axel}, 1 Bøsse-Skud over fra Vester i Øster bredt, og 1/16 Miil langt efter Vaagen i Nord, steilt og steenet, meget høyt; Dens Østre Næss er \textit{Skad-Tangen}, lavt nedentil, men oventil paa Sidene steilt, og ovenpaa rundagtigt, steenet, strekkende sig i Nord til Botten 1/16 Miil; Paa Østre Land-Side af denne Vaag, hvor det er lavt, boe 5 \textit{Normænd}.\par
Fra denne Nordre \textit{Honning-Vaag} i Nord-ost til \textit{Kielviig} er 1/4 Miil; Landet derimellem er graaberget, høyt, deels steilt, deels nedsludtendes, ovenpaa rundrygget, u-beboet.\par
10. \textit{Kielviig} er i Munden 1/8 Miil viid i Nord-Nord-ost, og 1/16 Miil i Vester dyb; Det Søndre Næss heder \textit{Kielvig-Odden}, som neere ved Søen er lav og knudred, 1 Bøsseskud over fra Søer i Nord; Det Nordre Næss er \textit{Skarv-Steenen}, noget høyt, bratvoren, graasteenet, 2 Bøsseskud over fra Søer i Nord; Paa begge \textit{Vigens} Sider er det steilt og steenet, i Botten lidet fladagtigt nedentil, men ovenfor bratt, berget og steen-gruset.\hypertarget{Schn1_81006}{}Om Jngøens Præstegield.\par
Mitt i Botten paa det flade Land staaer \textit{Kielvigens} Hovedkirke af Træe med Præstegaarden, \textit{Octrojens} Handelshusse, og 6 \textit{Normændz} Stue-Gammer som boe der. ‒ 1 Bøsseskud fra Botten i Øster ligger en Holm, \textit{Kielen};\par
Fra \textit{Kielviig} i Nord-Nord-ost til \textit{Hellenæss} er 1/2 Miil; Derimellem Landet er høyberget med Tinder paa, og skallet, u-beboet. \textit{Hellenæss} er fladt nedentil og steenet, et par Bøsseskud over stort;\par
Strax om \textit{Hellenæss} i Nord er \textit{Kamøe}-Fiord; Det Nord-Ostlige Næss af \textit{Kamøe}Fiord heder \textit{Skarsvaag-Næring}, derimellem i Nord-Nord-vest er en god 1/2 Miil.\par
11. \textit{Kamøe}Fiord imellem samme Næsse er i Syd-vest en god 1/2 Miil dyb hen til \textit{Kamøe}, ved Botten; Fra den Søndre Side af Botten stikker Fiorden, med det Navn:\par
\textit{Skibfiord} ind i Vester til Nord, ved 3/8 Miil dyb, 1/8 Miil breed. Fra den Nordre Side af Botten indgaaer \textit{RisFiord}, en liden Jndfiord, 1 Bøsse-Skud viid, 1/8 Miil dyb i Syd-vest.\par
Fra \textit{Risfiords} Gab i Nord-vest 1/4 Miil er en anden Jndfiord\par
\textit{Dugsfiord}, 1/4 Miil i Munden viid, og ligesaa dyb i Nord-vest.\par
Fiord-Brædene af \textit{Kamøe}fiord ere bergede, steile, og mest steenede, med lidet Lyng og Græss paa. J denne \textit{Kamøe}fiord paa Øen \textit{Kamøe} boe 3. \textit{Normænd}.\par
Strax Sønden for \textit{Skarsvaag-Næring} er en liden Bugt,\par
\textit{Opnan} kalded, et par Bøsse-Skud viid, og halv saa dyb i Nord-vest, u-beboed; hvor Folk fra \textit{Nordland} om Sommeren ligge, at fiske Qveete og Torsk: Dog strekker sig dette \textit{Skarsvaag-Næring} om Bugten hen til \textit{Hornvigen}, at det heraf giør det Syd-ostlige Næss; Thi det Nord-vestlige Næss af \textit{Hornvigen} kaldes \textit{Hornet}; Jmellem disse 2de Næsse er\par
12. \textit{Hornviig} ved 3/4 Miil i Munden viid, 3/8 Miil dyb i Sydvest; Fra Botten af denne \textit{Hornviig} indgaae 3 smaa Jndfiorder: den Østerste er\par
\textit{Kolfiord}, 1 Bøsseskud viid, 1/8 Miil dyb. Strax om Næsset af \textit{Kolfiord} er den anden ‒\par
\textit{Risfiord}, 2 Bøsseskud viid, 1/8 Miil dyb. Den 3die Jndfiord\par
\textit{Vest-fiord} er strax derved i Nord-vest, 1/8 Miil viid, og dyb. Fiord-Brædene af \textit{Hornviig} ere bergede og steenede, og ingen Folk boer her inde.\par
Saa snart man kommer om det Næss, \textit{Hornet}, i Nord-vest, møder\par
13. \textit{Skonvig}, dens Nord-vestlige Næss er \textit{Nord-Capen; Skonvig} er i Gabet 1/4 Miil viid, og ligesaa dyb, Fiord-Brædene bergede, og u-beboede. \textit{Nord Capen} stikker i Nord spidz ud, lav, flad, steened, knap 1 Bøsseskud over i Vester, og dette \textit{Nord Capen} er den nordreste Pynt af \textit{Finmarken}, og af heele \textit{Norge}.\par
Vest om \textit{NordCapen} kommes til\par
14. \textit{Sandfiord}, 1/4 Miil viid i Gabet, halv saa dyb i Søer, berged og steilt paa Fiord-Brædene; Dens Syd-vestlige Næss er \textit{Tuë-Næss}, fladt nedentil ved Søen, og høy-steilt op ad, steenedt. Paa dette \textit{Tuënæss} boe 2 \textit{Normænd}, men i \textit{Sand}Fiord Jngen.\par
Strax om \textit{Tuënæss} begynder\par
15. \textit{TuëFiord}, 3/8 Miil viid i Munden, 3/4 Miil dyb i Syd-ost, berged paa Sidene, men paa Botten græssed, u-beboed. Dens Syd-vestlige Næss er \textit{Gamfiord-Næss}; Thi en liden Bugt af det Navn der strax Sønden for indviger; Derpaa følger den Fiord:\par
16. \textit{Snærpe-poll}, 1/2 Miil Viid i Syd-vest, 1/4 Miil dyb, steened med lidet Græss paa Bræ\hypertarget{Schn1_81279}{}Schnitlers Protokoller V. dene, u-beboed; Dens Syd-vestlige Næss er \textit{Geis-næring}, sammenhengendes med \textit{Maanæss}, det Nordre Næss af \textit{Lyngpoll}, hvorfra man \textit{pag.} 270 begyndte Beskrivelsen af \textit{Magerøens} Søe-Kuster.\par
\centerline{3. \textbf{Kamøe}}\par
liggendes inde i en Fiord af \textit{Magerøe}, kalded \textit{Kamøe}fiord, ved dens Botten, rundagtig-kanted ved 3/8 Miil omkring stor, paa Syd-vestlige Søe-Bræde noget slet med Græss paa, ellers berged med Græss-Lier iblant; Paa den Sydvestlige Side, hvor det er slet, boe 3. \textit{Normænd}.\par
\centerline{4. \textbf{Store Kobbøe}, ubeboed}\par
ligger i \textit{Kobbe}-Fiorden, som gaaer ind i det faste Land, 1/16 Miil fra Fiordens Østre SøeBræde, og 1/4 Miil inden- eller Sønden for \textit{Kobbefiordens} Østre Næss, \textit{Stikkelnæring}, rundagtig, 1/4 Miil omkring, mitt paa noget klipped, men paa Brædene flad, og Græss-riig, hvor \textit{Normænd} af \textit{Maasøe} bruge at slaae Græsset.\par
\centerline{5. \textbf{Lill-Kobbøe}, u-beboed}\par
ligger i samme \textit{Kobbe}-Fiord, en god 1/2 Miil, inden- eller Sønden for \textit{Stor-Kobbøe}, nær ved Botten, ved den Bugt, kalded \textit{Øster-Botten}, fra Søer i Nord 1/16 Miil lang, næsten ligesaa breed, flad, steened, og deels lynged.\par
\centerline{6. \textbf{Tamsøe}, ubeboed}\par
ligger mitt i \textit{Porsanger}fiord, 1 god Miil fra Fiordens Vestre Næss, 1/2 Miil lang fra Syd-vest i Nord-ost, halv saa breed, flad og slet, paa Søe-Bræden med Græss, og indentil med ReenMose og smaa Bierke-Kratt begroed; Der er om Sommeren en stor Mængde af Fugle, som Maaser, vilde Gæss, Ee-Fugl, Tinner og Teister, hvilket foraarsager, at den er et fordeelagtigt Duun- og Egge-Vær.\par
Fleere Øer og Holmer mange findes, men ubeboede, og af ingen synderlig Betydenhed.\par
Efter Øerne beskrives nu paa følgende Maade\par
\textbf{Kielvigens faste Land} ved Søe-Siden:\par
\textit{Pag.} 265 her før er ommeldt, at det nærmeste \textit{Jng-øens} Gieldz\textit{district} strekker sig i Nord-Nord-ost til det Næss \textit{Myr-Nipen}, paa det faste Land; Derfra nu i Øster begynder \textit{Kielvigens} Præstegieldz\textit{District}, saa at dens faste Land ved Søe-Kanten fra bemeldte \textit{MyrNipen} gaaer i Øster til Norden 4 sterke \textit{Finmark}iske Miile til \textit{Sverholt-Klubben}, det Østere Næss af \textit{Porsangers}Fiord;\par
Hvilken Land-Strekning i Særdeleshed forklares saaledes:\par
Fra \textit{Myr-Nipen} i Øster til nærmeste \textit{Eiter-Fiord} er {3/8 Miil} Landet derimellem er steenet og berget, uden Græss og Skoug, u-beboet \textit{Eiter-Fiord} er imellem sine Næss i Munden viid {1/8 ‒} Strax om \textit{Eiterfiords} Østre Næss aabner sig \textit{Kulfiord}, viid i Gabet {1/2 ‒}\textit{Kulfiords} Østre Næss rekker til næste \textit{Ryggefiordz} Vestre Næss i Syd-Syd-ost {1/4 ‒}\hypertarget{Schn1_81498}{}Om Jngøens Præstegield.\textit{Ryggefiord} i Munden viid {1/8 Miil} Strax om \textit{Ryggefiords} Østlige Næss forekommer \textit{Kobbefiord}, viid i Gabet {3/4 ‒}\textit{Kobbe-Fiords} Østre Næss strekker sig langs med \textit{Magerøe}-Sundet i Ost-Syd-ost hen til \textit{Lafiords} Vestre Næss {3/8 ‒}\textit{Lafiord} i Gabet viid {1/8 ‒}\textit{Lafiordens} Østre Næss naaer i Øster til \textit{Kaafiordens} Vestre Næss; Landskabet heraf er høyt, steilt, steenet, neden under paa Søe-Bræden med Græss begroet, nu u-beboet, hvilket græss af \textit{Normænd} andenstedz fra afslaaes, langt i Øster {1/4 ‒}\textit{Kaafiord} er i Munden viid {1/4 ‒} Fra dens Østre Næss til \textit{Porsangers}Fiordz Vestre Næss er i Øster {1/4 ‒} Landet derimellem er smaaberget med Lyng paa, ubeboet. \textit{Porsanger}Fiord er til sit Østlige Næss i Gabet viid {1 3/4 ‒ _______ som udgiør \textit{ordinaire}‒ 5 1/8 Miile}\hspace{1em}\par
Det indre faste Land af \textit{Kielvigens} Præstegield op fra Søe-Kanten er et Stykke op ad før \textit{pag.} 262 f. beskreven, hen imod \textit{Porsanger}-Fiordz Vestre Side.\hspace{1em}\par
Sp. 3. Svar: Fisk og Fugl haves i \textit{Kielvig}-Gield, som i \textit{Alten}\textit{pag.} 226 som og i \textit{Porsanger}- Elv Lax fanges, men ikke i saadan Mængde, som i \textit{Altens} Elv. Lund-Fugl gives her ikke. Sild faaes her ikke, som i \textit{Nordland}; Om det deraf kommer, at Jndvaanerne have ingen Reedskab, eller Kundskab, til at fange den? stilles derhen: J det mindste Lægge de sig ikke efter den Slags Fangst.\hspace{1em}\par
Sp. 4. Svar: Havne paa \textit{Magerøe} ere\par
(1) i \textit{Balhopen} under \textit{Sarnæss} for 3 a 4 Skibe.\par
(2) i \textit{Søer-HonningsVaag} for et par Skibe\par
J \textit{Porsangerfiord} under det faste Land\par
(3) i \textit{Repervaag} paa \textit{Porsangers} Vestre Side for 4 a 5 Skibe\par
(4) i \textit{Smørfiord} sammestedz for 30 à 40 Skibe.\hspace{1em}\par
Sp. 5. Svar: \textit{Fiorder} i \textit{Kielvigens} Gield samt Viger paa \textit{Maasøe} og \textit{Magerøe} ere ved Øerne beskrevne p: 270 f.\par
Fiorder og Viger i det faste Jndland ere Sønden-fra at reigne: nemlig fra \textit{Myr-Nipen}, hvor \textit{Kielvigens} Gield begynder, 3/8 Miil i Øster\par
(1) \textit{Eiterfiord}, dens Vestre Næss heder \textit{Eiterfiord-Næss}, hvast udgaaendes, høyt, bratt og steenet, 1 Bøsse-Skud langt i Søer efter Fiorden; Det Østre Næss er \textit{TrollFiord Næss}, lavt og fladt, deels steenet, deels lynget med lidet Græss imellem, strekkende sig efter \textit{Kulfiord} i Søer til Osten 1/2 Miil langt.\par
Imellem de 2de Næss er \textit{Eiterfiord} i Gabet viid 1/8 Miil, og ligesaa dyb i Søer, hvor ingen Folk boer.\par
(2) \textit{Kulfiord} har, som meldt, \textit{Trollfiord}-Næss, til Vestre- og \textit{Fiske-Næsset} til Østre- Næss; dette \textit{Fiske-Næss} er lavt, spidz udstikkendes neere i Søen, men oventil rundvoren, og \hypertarget{Schn1_81767}{}Schnitlers Protokoller V. steen-uret, med Reen-Mose paa, strekkende sig i Søer 1/2 Miil ind til Botten; Jmellem disse 2de Næss er \textit{Kulfiord} i Munden en god 1/2 Miil viid, og ligesaa dyb i Søer til Bonden. Her boer ingen Folk inde;\par
Bem.te \textit{Fiske-Næss} rekker i Syd-Syd-ost til næste \textit{Ryggefiordz} Vestre Næss, \textit{Slotte} 1/4 Miil.\par
(3) \textit{Rygge-Fiord} har berørte Vestre Næss \textit{Slotte}, høyt, steenberget, og bratt ud ad Søen, fladt ovenpaa, rekkendes i Søer efter \textit{Rygge}Fiorden til dens Botten 3/8 Miil; Det Syd-ostlige Næss af \textit{Rygg}-Fiord heder \textit{Afløsning}, lavt, rundvoren, deels steenet, deels myret, 1 Bøsseskud over stort; Jmellem disse 2de Næss er \textit{Rygge}fiord i Munden 1/8 Miil viid og 3/8 Miil dyb i Søer; har Bierke-Skoug, men ingen Indbyggere.\par
Det samme \textit{Afløsning} giør og det Vestre Næss af\par
(4) \textit{Kobbefiord}, som i Gabet er viid 3/4 Miil, og fra \textit{Afløsnings} det Vestre Næss stikker i Søer 3/4 Miil ind til Botten, men fra sit Østlige Næss i Syd-Syd-vest didind er 1 Miil lang; J Enden af Botten gaaer igiennem et kort trangt Sund en Poll, kaldes \textit{Vester-Pollen}, 1/16 Miil i Søer til Vesten lang, 1 Bøsseskud breed; Paa den Østre Side af Fiorden nær ved Botten, indgaaer en Bugt, der kaldes \textit{Øster-Botten}; 1/16 Miil viid og dyb; J hver af disse Poller boe 2de Søe-\textit{Finner.}\par
Denne \textit{Kaafiords} Østre Næss heder \textit{Stikkelnæring}, høyt, bratt og steenet, strekkende sig langs med \textit{Magerøe}-Sund i Ost-Syd-ost 3/8 Miil hen til \textit{Lafiords} Vestre Næss, \textit{Naava}.\par
(5) \textit{Lafiord} ligger i Ost-Syd-ost 3/8 Miil fra \textit{Kaafiord}, er 1/8 Miil viid, og 3/8 Miil fra Syd-vest til Botten lang, hvor den giør 2de Poller, den Østre kaldes \textit{Østre-Pollen}, den Vestre \textit{Strømmen}; J hvilken \textit{Strømmen} 3 Søe-\textit{Finner} boe.\par
Denne \textit{Lafiord} har til Vestre Næss bemeldte \textit{Naava}, høyt, bratt, rundvoren, et par Bøsseskud stort, steenet neden under, og moset ovenpaa, med Lyng i blant;\par
Dette \textit{Naava}-Field strekker sig fra Gabet i Syd-Syd-vest 3/8 Miil lang indtil Botten af \textit{Lafiorden; Lafiordens} Østre Næss heder \textit{Gamnæss}, høyt, fladt-nedhældendes med Græss paa.\par
Fra \textit{Lafiord} til næste \textit{Kaafiord} i Øster er 1/4 Miil;\par
(6) Landet derimellem er høyt, steilt, steenet, dog neden under paa Søe-Bræden græsset, nu u-beboet.\par
\textit{Kaafiords} Vestre Næss heder \textit{Dybpoll-Næss}, høyt og rundt, deels steenet, deels lynget, 1 Bøsseskud over stort;‒\par
\textit{Kaafiords} Østre Næss, \textit{Svaberg} er ganske lidet, fladt, græsset neden til, og oventil mosset, 1 Bøsse-Skud over vidt.\par
Jmellem disse 2de Næss er \textit{Kaafiord} 1/4 Miil i Gabet viid, og ligesaa lang i Søer; Landskabet inde i \textit{Kaafiorden} er paa Brædene slet, græss-groet, med Bierke-Kratt paa, u-beboet.\par
Fra \textit{Kaafiord} til \textit{Porsanger}-Fiord i Øster er 1/4 Miil;\par
(7) \textit{Porsanger}-Fiord er i Gabet imellem dens Næss 1 3/4 Miil viid, siden indentil 1 Miil og meere over breed, indtil mod Botten, hvor den er 1 knap Miil over viid; Lang er den paa sin Syd-vestlige Side i Syd-Syd-vest fra det yderste \textit{Porsangers-Næss} til Botten {7 5/16 Miil.} Paa sin Nord-ostlige Side i Syd-Syd-vest fra det Næss, \textit{Sverholt-Klubben} til Botten ved {8 Miil.}\hypertarget{Schn1_82006}{}Om Jngøens Præstegield.\par
Som i særdeleshed saaledes forklares:\par
\textit{Porsanger}fiordz Vestlige Næss, som yderst er, kaldes \textit{Lill-Porsanger-Næss}, fladt, udstikkendes i Nord til Osten, et par Bøsseskud over stort, med Lyng paa, steilt oventil.\par
En god 1/8 Miil i Syd-ost fra \textit{Lill-Porsanger Næss} er \textit{Stor-Porsanger-Næss}, hvorimellem \textit{Porsangvig} 1/8 Miil i Søer til Vesten dyb.\par
\textit{Stor-Porsanger-Næss} stikker spidz og fladt ud i Nord-Nord-ost, et par Bøsseskud over stort, steenet, opad rundagtig-høyt\par
Fra \textit{Stor-Porsanger-Næss} i Søer til Vesten er en knap 1/2 Miil til den Jndfiord, \textit{Raskeil}; Landet derimellem berget, med smaa Græss-Pletter paa. Det Nordre Næss af \textit{Raskeil} heder \textit{Qvalnæss}, steilt ned ad Fiorden, rundagtig opgaaendes, bart og steenet, 4 Bøsseskud over fra Nord i Søer;\par
Det Sydlige Næss af \textit{Raskeil} kaldes \textit{Katta}, lavt neden ved Fiorden, og steenet, derfra fladt-opgaaendes, bart, en 2 Bøsse-Skud over stort i Søer; Jmellem disse 2de Næss er den Jndfiord, ‒\par
1) \textit{Raskeil} i Munden viid 1/16 Miil, et par Bøsseskud dyb i Nord-vest; Fiord-Brædene ere flade, steen-grusede, med lidet Græss imellem.\par
Fra \textit{Katta} til \textit{Repervaag} i Søer til Vesten er 1/2 Miil; Landet derimellem er deels berget og steenet, deels fladt med Lyng paa, og Græss derimellem; u-beboet.\par
Det Nordre Næss af \textit{Repervaag} heder \textit{Fugleberg}, steilt ud ad \textit{Porsanger}, men fladt ad Vaagen, 1 Bøsseskud over stort; Det Søndre Næss af \textit{Repervaag} er \textit{Repvaag-Næss}, fladtrundagtigt oventil med Lyng paa, 1/16 Miil i Søer efter \textit{Porsanger} stort; Jmellem disse 2de Næss er\par
2) \textit{Repervaag}, 1/8 Miil viid, og ligesaa dyb; J Botten udgaaer fra det Vestre Landz Side en Odde, som giør i Vaagen 2de Poller ‒\par
Fra \textit{Repvaag-Næss} i Søer til Vesten ad \textit{JndreBugten} er 1/16 Miil;\par
3) \textit{Jndre-Bugten} er 1/16 Miil viid og dyb, steen-grused og flad paa Brædene, u-beboed.\par
J Søer til Vesten 1/4 Miil derfra forekommer\par
4) \textit{Molvig}; Landet derimellem er berget, fladtvoren med Lyng paa; \textit{Molvig} er i Munden et par Bøsseskud over viid, og ligesaa dyb, omkringgiven med steen-grusede Berge, u-beboed. Fra \textit{Molvig} i Syd-Syd-vest til \textit{Svartvig} er 1/2 Miil; Landet derimellem berget og steilt, med Lyng nedentil, og op ad med Bierke-Kratt paa.\par
5) \textit{Svartvig} er i Munden 1/16 Miil viid, og ligesaa dyb, med steile Berge omgiven, hvorpaa smaa Græss-Pletter hidst og her sees, u-beboed; Fra \textit{Svartvig} i Syd-vest til \textit{Skarbergvig} er 1/8 Miil; Landet derimellem berget og u-beboet;\par
6) \textit{Skarbergvig} er i Munden et par Bøsseskud viid, og engang til saa dyb, steil paa Sidene, men flad paa Botten, med Lyng og Bierke-Skoug paa. Fra \textit{Skarbergvig} i Syd-Syd-vest til den Bugt \textit{Normands} Sæde er 1/4 Miil; Landet derimellem steen-berget og steilt.\par
7) \textit{Normands} Sæde er ganske kort over, indentil lavt med smaa Bierk begroet, nu u-beboet. Fra \textit{Normands}Sædet i Syd-vest til \textit{Smørbringen}, det Sydlige Næss af \textit{Smørfiord} er 1/2 Miil; dette \textit{Smørbringen} er fladt neere, oventil rundagtig-høyt, med nogen Bierk paa,  1/8 Miil stort.\hypertarget{Schn1_82219}{}Schnitlers Protokoller V.\par
Det Nordre Næss af \textit{Smørfiord} heder \textit{Rødsteen}, fladt og rundagtigt med Bierk paa, et par Bøsseskud over stort; Jmellem disse 2de Næss er\par
7) \textit{Smørfiord}, i Gabet 3/8 Miil viid, 1/2 Miil dyb i Vest-Syd-vest.\par
Strax om \textit{Smørbringen} møder\par
8) \textit{Olderfiord}, saa at samme \textit{Smørbringen} er det Nordre Næss af \textit{Olderfiord}; Denne \textit{Olderfiord} er i Gabet viid 1/8 Miil, og 1/2 Miil dyb ind ad.\par
1 Bøsseskud udenfor \textit{Olderfiordz} Søndre Næss i Øster ligger \textit{Langøe}, rundvoren, 1/4 Miil omkring, slet, lyng-groed, med noget Græss paa Nordre Side, u-beboed.\par
Fra \textit{Olderfiords} Søndre Næss, at fare omkring Fiord-Bræden, er til \textit{Kistrand-FinneCapell}, hvor \textit{Kistrandbugten} begynder {3/8 Miil.}\par
Landet derimellem slet og græsset;\par
9. \textit{Kistrand-bugten} er til dens Sydlige Næss \textit{Veinæss} 3/8 Miil viid, slet paa Brædene, og græssed, ubeboed.\par
Fra \textit{Veinæss} i Syd-Syd-vest til \textit{Billefiord} er {1/4 Miil}\par
Landet derimellem fladt og lynget.\par
10. Billefiord er i Syd-Syd-vest til dens Søndre \textit{Bille}Fiordz Næss i Gabet viid 3/4 Miil; og stikker ind i Syd-vest, giørendes ved Botten 2de Arme, hvoraf den 1te kaldes \textit{Yttre Billefiord}, og er fra \textit{Bille}fiords Nordre Næss dyb i Sydvest 3/4 Miil; Den 2den Arm Søndenfor heder \textit{Jndre Billefiord}, og kan reignes fra \textit{Billefiords} Sydlige Næss at være i Vester 3/8 Miil dyb; FiordBrædene ere slette med Græss og Bierk paa. Det Søndre \textit{Billefiords} Næss er i Søer imod 1/4 Miil lang, fladberget, knudret, med Bierk og noget Græss paa.\par
1/2 Bøsseskud udenfor \textit{Søndre Billefiords} Næss er \textit{Troll-holmen}, langagtig-rund, 1/4 Miil omkring, berged og knudred, u-beboed.\par
Det Søndre \textit{Billefiords} Næss giør og det Nordre Næss af \textit{Tabesbugten}; Dens Søndre Næss heder \textit{Tabesnæss}; Hvorimellem ligger \par
11. \textit{Tabesbugten}, 1 Miil i Gabet viid i Syd-Syd-vest, og 1/4 Miil og mindre dyb i Vester, u-beboed; Omkring denne Bugt staaer smaa Skoug, og paa en Slette i Botten Furre, 1/4 Miil breed.\par
\textit{Tabesnæss} see \textit{p}. 281 er fladt, lynget, 1/8 Miil over smalt paa Odden, mens breder sig imellem \textit{Tabes}-Elv og \textit{Rautes} Elv ud fra Søer i Nord til 1 Mils Vidde; Et Stykke fra Odden gaaer det fladagtig-høyt op i Veiret, med Reen-Moesse paa;\par
Uden for \textit{Tabesnæss} 1/16 Miil ligger en liden Holm, \textit{Melkerøe}, langagtig-rund, 1/16 Miil omkring, noget høyberged, med lidet Græss iblant, uden Skoug, og u-beboed.\par
Fra \textit{Tabesnæss} i Søer til \textit{Porsanger}-fiordz Vestre Botten er 1 Miil, og Landet fieldet, langsludtendes, med Moese og Bierk paa.\par
Jmellem \textit{Billefiords} Næss og \textit{Tabesnæss} kommer \textit{Tabes} Elv Vesten-fra ind i \textit{Tabesbugten}; og imellem \textit{Tabesnæss} og Fiordens Vestre Botten udrinder en liden Aae Vesten-fra i \textit{Porsanger}Fiord.\par
\textit{Porsanger}-Botten er i Øster over breed 1 Miil; Denne Botten deeles ved en Tang Søndenfra af det faste Land i 2de Armer, navnlig \textit{Vester-Botten}, og \textit{Øster Botten}; Hver af de Bottener er i Søer 1 Miil lang, 3/8 Miil og mindre breed; Tangen af det faste Land kaldes \textit{Niargo}, og \hypertarget{Schn1_82480}{}Om Jngøens Præstegield. stikker i Nord ud i Fiorden 1 Miil lang; den 1 fierding fra det faste Land er 1/4 Miil breed, den følgende 1/2 Miil siunes at være imod 1/2 Miil viid og rundagtig, den sidste og nordreste Fierding gaaer smal ud, og er mitt paa kun et par Steenkast over breed, siden vider den sig ud til 1/8 Mils Bredde til imod Enden, hvor den spidz udstikker. Denne \textit{Niargo}-Tang er berged, med Bierk, Lyng og Reen-Moesse paa, u-beboed.\par
At beskrive nu den Nord-ostlige Fiord-Bræde af \textit{Porsanger:} Saa er fra Fiordens \textit{Østerbotten} i Nord-ost 2 gode Mile til \textit{Børss}-Elv Landet u-beboet, berget.\par
3/8 Miil i Nord-nord-ost fra \textit{Børse}-Elv er \textit{Diubvig}, nogle Bøsseskud i Gabet viid, og 1/8 Miil i Øster dyb. En knap 1/4 Miil herfra i Nord-nord-ost er \textit{Leerpoll}, 1/8 Miil i Munden viid, et par Bøsseskud dyb.\par
1/2 Miil i Nord-nord-ost fra \textit{Leerpoll} til den Slette, \textit{Jndre Brenne}, hvor \textit{Finner} boe; Landskabet imellem \textit{Leerpoll} og \textit{Jndre Brenne} er paa Fiord-Bræden slet, med Græss og Bierk paa, 1/4 Miil fra Fiorden breed; Derfra i Øster ere lave langsludtende Berge alt over til \textit{LaxeFiords} Botten, hvilke Berge ere mesten lyngede og moesede, ellers bare; J Lægdene imellem Bergene er og Bierk.\par
Fra \textit{Jndre-Brenne} i Nordnordost er 1/2 Miil til \textit{Yttere Brenne} ‒\par
Fra \textit{Yttre Brenne} i Nordnordost er 1 Miil til \textit{Kæs-klubben}; Landskabet fra \textit{Jndre Brenne} hidtil \textit{Kæsklubben} er ligesom det imellem \textit{Leerpoll} og J\textit{ndre Brenne} næst tilforn er forklaret.\par
\textit{Haarviig} ligger fra \textit{Kæsklubben} i Nordnordost 1 Miil, i Gabet et par Bøsseskud viid, og ligesaa dyb, slet og græsset omkring Vigen med Bierke-Kratt paa; Landskabet imellem \textit{Kæss-Klubben} og \textit{Haarvig} er største deels berget, langsludtendes, sommestedz moeset, sommestedz bart, og sommestedz Bierke-groet, men nær ved \textit{Haarvig} er der Slette, med Græss bevoxen.\par
Fra \textit{Haarvig} i Nordnordost til \textit{Sinkelvig} er 1/4 Miil; Landet derimellem er berget, langsludtendes og skallet.\par
\textit{Sinkelvig} er 1/16 Miil viid, og 1 Bøsseskud dyb, Landet deromkring er langsludtendes med Bierk og lidet Græss paa, nu u-beboet.\par
Fra \textit{Sinkelvig} i Nordnordost til det Field \textit{Reen-bænken} er 1 1/4 Miil, hvorimellem 2de \textit{Diubvig-Bugter} ere.\par
Den \textit{SøndreDiubvig} ligger 1/4 Miil Norden for \textit{Sinkelvig} i Gabet 1/16 Miil viid, halv saa dyb; Den anden \textit{Nordre-Diubvig} er fra \textit{Søndre-Diubvig} 1/8 Miil, ligesaa viid og dyb, som den Søndre; Landskabet alvejs derimellem er berget, steilt og bart; men inde i \textit{Diubvigene} er det et Stykke slet med Bierke-Kratt paa, siden opstiger det berget i Veiret.\par
\textit{Reenbænken} er med de andre Fielde Sønden og Norden-for sammenhengendes, lige, som de, høyt, steilt og bart ud ad Fiorden; Har ellers sit Navn deraf, at oven i Fieldet er et Lægd, et par Bøsseskud langt, skabt som en Bænk, hvori Snee om Sommeren overligger, og OxeReenen i hedeste Varme lægger sig i Svale.\par
Fra \textit{Reenbænk} i NordNordost til \textit{Miaavig-Næss} er en god 1/4 Miil; Landet herimellem er steilt og snaut, fladt ovenpaa.\par
\textit{Miaavig-Næss} ligger nær Sønden for \textit{Miaavig} (som er en liden langsludtendes Viig) og gaaer fladagtig ned ad Fiorden, skallet, rundt oven-paa.\hypertarget{Schn1_82691}{}Schnitlers Protokoller V.\par
Fra \textit{Miaavig-Næss} i Nordnordost til \textit{Sverholt-Klubben}, det Østre Næss af \textit{Porsanger}, er en knap 1/2 Miil; Landet derimellem er steilt, berget og bart, fladt oppaa.\par
\textit{Sverholt-Klubben} er paa Nordre Side ad Havet steilt, men til begge, nemlig \textit{Porsanger}- og \textit{Lax}-Fiordene langsludtendes, bart allestedz, fladt ovenpaa.\par
2 Miile fra \textit{Porsangers} Botten i Søer, nemlig til \textit{Tabesbugtens} Søndre Næss paa Vestre og til \textit{Børss}-Elven paa Østre Side ligge mangfoldige u-beboede Holmer nær udenfor- og langs med FiordBrædene, som ere gode Fugle-Vær, hvoraf \textit{Finnerne} nyde Eeder-Dunn, og Egg.\par
J denne \textit{Porsanger}fiord boe\label{Schn1_82749} \par 
\begin{longtable}{P{0.15532128514056223\textwidth}P{0.5154618473895582\textwidth}P{0.0716867469879518\textwidth}P{0.1075301204819277\textwidth}}
 \hline\endfoot\hline\endlastfoot \Panel{}{label}{1}{l}\tabcellsep \Panel{}{label}{1}{l}\tabcellsep \Panel{\textit{Normænd},}{label}{1}{l}\tabcellsep \Panel{\textit{Finner}}{label}{1}{l}\\
Yderst i Fiorden,\tabcellsep som i \textit{Raskeil}\tabcellsep 6 ‒\tabcellsep ‒\\
\tabcellsep i og ved \textit{Repervaag}\tabcellsep 6 ‒\tabcellsep ‒\\
Jndad\tabcellsep i \textit{Smørfiord}\tabcellsep \tabcellsep 1 ‒\\
\tabcellsep i \textit{Olderfiord}\tabcellsep \tabcellsep 2 ‒\\
\tabcellsep paa \textit{Kistrand}\tabcellsep \tabcellsep 2 ‒\\
\tabcellsep i \textit{Jndre Billefiord}\tabcellsep \tabcellsep 12 ‒\\
\tabcellsep i \textit{VesterBotten} af \textit{Porsanger}fiord med \textit{Porsanger}-Elv\tabcellsep \tabcellsep 1 ‒\\
\multicolumn{2}{l}{Paa Østre Land-Side af \textit{Porsanger}fiord ved \textit{Børss}-Elv}\tabcellsep \tabcellsep 5 ‒\\
\tabcellsep i \textit{Leerpoll}\tabcellsep \tabcellsep 6 ‒\\
\tabcellsep paa \textit{Jndre Brennen}\tabcellsep \tabcellsep 2 ‒\\
\tabcellsep i \textit{Haarvig}\tabcellsep \tabcellsep 3 ‒\\
\tabcellsep \multicolumn{3}{l}{___________________}\\
\tabcellsep i alt\tabcellsep 12 \textit{Normænd}\tabcellsep 34 \textit{Finner}\end{longtable} \par
 \par
til 6 Sp: Svar: Elve, Aaer og Vande ere i \textit{Kielvigens} Gield:\hspace{1em}\par
først i \textit{Maasøe Annex:} J \textit{Eiter}- og \textit{Kul}-Fiorder rinde kun Bække.\par
1) \textit{Ryggefiord}-Elv kommer Sønden fra af \textit{Ryggefiord-Vatten}, som er fiirkantet, 3 Bøsseskud over vidt, og har Øret-Fisk, deraf Aaen rinder 1/16 Miil lang i Botten af \textit{Ryggefiord}.\par
2) J \textit{Kobbefiords} Vestre \textit{Poll} er det, at den største Aae rinder, 3 Bøsseskud lang, Sønden-fra af det Vand \textit{Vesterpollvand}, som er fra Vester i Øster 4 Bøsseskud langt, halv saa bredt, af dets Vestre Ende Aaen udløber; J Vandet haves Øreter.\par
3. J \textit{Lafiords} Østre \textit{Poll} rinder Sønden-fra den Aae \textit{Lafiord}-Elv, imellem de Fielde \textit{Kaattemorass} paa Vestre- og \textit{Gardevara} paa Østre Side, 1 1/2 Miil lang, af \textit{Lafiord}-Vattens Østre Ende, hvilket er fra Vester i Øster 3 Bøsseskud langt, og halv saa bredt; J denne Elv faaes Lax, dog kun til Huus-Behov.\par
J \textit{Kaafiord} gaaer kun en Bæk.\hspace{1em}\par
Dernæst i \textit{Kielvigens} HovedSogn ere Elve og Vande:\par
J \textit{Porsangers} Jndfiorder paa Vestre Side:\par
1. Mitt i \textit{Repervaag} Bugten kommer fra Syd-vest \textit{Finnebye}-Elven af \textit{Storvatten}, og rinder 1/16 Miil lang i Nordost i bem.te Bugt; \textit{Storvatten} er fra Søer i Nord 1/16 Miil langt, halv saa bredt; J Vattenet og Elven fanges Øreter, Fra det Field \textit{Gardevara} komme adskillige Bække, og løbe i dette \textit{Storvatten}.\hypertarget{Schn1_83056}{}Om Jngøens Præstegield.\par
2. J \textit{Smør-Fiord} rinder \textit{Smørfiord}-Elv, 1/2 Miil lang, fra Syd-vest af \textit{Motkie-jaure}, som fra Sydvest i Nordost er 1/4 Miil langt, halv saa bredt; J Elven fanges Øreter.\par
J \textit{Olderfiord} gaaer kun en Bæk.\par
3. J \textit{Yttre Billefiord} indgaaer \textit{Gorbe-vuønn-jok}, næsten 1 Miil lang fra Syd-Syd-vest af det Vand \textit{Siolme-jaure}, som er fra Syd-Syd-vest i Nord-Nord-ost 1/2 Miil langt, 1 Bøsseskud bredt; J Elven og Vandet fanges Øreter.\par
4. J \textit{Tabesbugten}, 1/4 Miil Norden for \textit{Tabes-Næss}, indkommer \textit{Tabes}-Elv, som \textit{Finnerne} kalde \textit{Rautes-jok}, hvorom før \textit{pag.} 225 er bleven rørt, udaf det Vand \textit{Rautes-jaure}, hvoraf den rinder omtrent halvveis i Nord, siden i Øster halvvejs, saaledes i alt en 5 Mile i denne \textit{Tabesbugt}; I denne \textit{Tabes}-Elv faaes Øreter; heraf vil \textit{p.} 278 forklares.\par
5. J \textit{Vestre Botten} af \textit{Porsanger}fiord indløber \textit{Porsanger}-Elv, i Munden næsten 1 Bøsseskud over breed; Den kommer Sønden-fra af det Field \textit{Vorieduder}, og løber et Stykke i \textit{N.} til i det Vand \textit{Laune-jaure}, siden derfra 1 1/2 Miil lang i \textit{N.} til O. i \textit{Porsangers} Botten. \textit{Launejaure} er 1/2 Miil langt i \textit{N.}, 1/8 Miil bredt. \textit{Porsanger}Elv er i \textit{Kielvigens} Gield næsten den eeneste, hvori Lax fanges, dog ikke i nogen Mængde.\par
2 Mile fra \textit{Porsangers} Østre Botten i Nordnordost løber\par
6. \textit{Børse}-Elv i \textit{Porsanger} paa dens Østre Side; Fra \textit{Gaisak}-Field i Søer kommendes, og løber i \textit{N.N.V.} ved 2 Mile lang Østen forbi \textit{VarejieZiek}, siden kroger sig i Vester, og gaaer saaledes 1 Miil lang i bem.te \textit{Porsanger}Fiord.\par
7. \textit{Kæs}-Elv, mod 3 Mile \textit{N.} for \textit{Børs}-Elv, løber af en Kiøn i en Bierkeskoug Østen fra 1/2 Miil lang i S.V. i \textit{Kæsbugten}.\hspace{1em}\par
Sp. 7. Dale er: i \textit{Repervaag} ved \textit{Finnebye}-Elv.\par
(2) Efter \textit{Smørfiord}Elv. (3) Efter \textit{Porsanger}-Elv, den beste 2 Mile lang, viid paa Ø. Side Elven.\hspace{1em}\par
Sp. 9. Svar: Skoug af Furre er ved \textit{Tabes}, eller \textit{Rautes}Elv, 1/8 Miil fra \textit{Tabes}bugten begyndendes, og 1/2 Miil lang efter \textit{Tabes}Elv, 1/4 Miil og mindre breed: dog ere Træerne kun Korte.\par
2. Efter \textit{Porsanger}-Elv er her i Gieldet den beste Furreskoug, dog begynder den sterkeste deraf først 1 à 1 1/2 Miil Sønden for \textit{Porsangers} Botten, og skal staae omkring det Vand \textit{Laune-jaure}, 1 Miil i \textit{Circumference}; og deraf er det, at Furren her i Gieldet er saa bekostelig.\par
3. Ved \textit{Børss}-Elv paa Østre \textit{Porsangers} Land-Side skal og være nogen, men liden og kort Furreskoug,\par
Bierke-Skoug i \textit{Maasøe Annex} er\par
4. i \textit{Kobbefiord}, hvor de \textit{Normænd} fra \textit{Maasøe} hente deres BrændeHved fra; Thi paa Øen haves ingen.\par
i \textit{Kielvigens} HovedSogn er ved \textit{Porsanger}:\par
5. Jnden for \textit{Repervaag} efter \textit{Finnebye}-Elven,\par
6. ved \textit{Smørfiord}-Elven er mangfoldig Bierk, see \textit{pag.} 265.\par
7. ved \textit{Olderfiord} ligesaa.\par
8. ved \textit{Billefiord},\hypertarget{Schn1_83404}{}Schnitlers Protokoller V.\par
9. J- og indenfor \textit{Tabes-bugten}, Fra hvilke Jndfiorder de Folk paa \textit{Magerøe} lader hente deres Brændehved; Saasom de intet deraf paa Øen have. ‒\hspace{1em}\par
Sp. 10. Svar: Vild og \textit{Jnsecter} her, som i \textit{Jngøens} Gield\textit{pag.} 265.\hspace{1em}\par
Sp. 11. Svar: Leilighed til Rødning eller Ny-Byggere er paa \textit{Magerøe} i \textit{Balhopen} ved \textit{Sarnæss}, og paa \textit{Tamsøe} i \textit{Porsanger}fiord; Paa det faste Land paa \textit{Porsanger}fiordz Vestre Side:\par
1) J \textit{Smørfiorden} ‒\par
2) paa \textit{Kistrand} ‒\par
3) J \textit{Billefiord}\par
4) J \textit{Sandvigen}, vesten imod \textit{Trollholmen}.\par
5) J \textit{Kolvigen}, 1/4 Miil derinden for. Paa \textit{Porsanger}Fiordz Østre LandSide\par
6) J \textit{Haarvig} ‒ den Jndre.\par
7) Ved \textit{Kæs-Elven}\par
8) J \textit{Leerpollen}.\hspace{1em}\par
Sp. 12. Svar: Fielde Østen for \textit{Porsanger-}Botten skal være det før omtalte vide \textit{Vorieduder}-Field, som paa den Vester Side af \textit{Porsanger}-Elv er noget høyt, men paa Østre Side af samme Elv lavt og dalet\hspace{1em}\par
Sp. 13. Svar: Den \textit{Octrojerede} Handel i \textit{Kielvig} paa \textit{Magerøe} forsiuner \textit{Kielvigens} HovedSogn, hvorhen og Almuen maa levere deres Fisk og andre \textit{producter}.\hspace{1em}\par
Sp. 14. Svar: \textit{Normænd}, som beboe Øerne, søge her som i Almindelighed i \textit{Finmarken}, deres Næring med Fiskerie paa Havet, og holde ganske faa Køer og smaa Fæ; Korn voxer her ikke, ei heller haves Skoug paa Øerne, saa de brænde deels Torv, deels maa hente deres Hved 3 a 4. Mile meere og mindre fra Jndlandet igiennem Fiordene med stor Besvær og Bekostning; Noget driver og Havet, hvorfra? vides ej, til Øerne og Søe-Kusterne af Rag-Hved, som af Vind-fældede Træer, eller andre Timmer-Stokker, hvilke Vandet har udført, eller Søen brudt; Søen er da disse \textit{Normænds} eeneste \textit{element}, og Landvæsenet agtes kun lidet af dennem; Det siunes, de kunde tilvejebringe sig meere Græss-Land til fleere \textit{Creatur}ers Underhold: Men man seer deres Møg-Dynger, hvorved Jorden skulle dyrkes, enten i Fiæren, at Søen den kan bortskylle, eller andenstedz henligge frugtes-løs, uden at bruges; Man finder ei Spinrok, Væv-stoel, Skoemagere, eller HandværksFolk hos dem; thi de ere vandte til, at kiøbe sig Lærret, Vadmel, Skoe, Støvler, Skind-Stakker, som ere Søe-Kioler af Skind etc:\par
Søe\textit{Finnerne} derimod begaae sig bedre: De sidde inde i Fiordene paa det faste JndLand nær ved eller i Skougen, hvor brænde hved dem intet koster, de have ei alleene Søen, som \textit{Normændene}, til deres Næring, men have og Leilighed til Skiøtterie i Fieldene, til at faae Fugl, vilde Reen og andre Dyr, samt bedre Beqvæmmelighed (naar de kun vill) at rødde, og forhverve sig meere Eng-Land; Saa disse \textit{Finners} Vilkaar paa det faste Land er bedre end hine \textit{Normænd} deres paa Øerne. ‒\hspace{1em}\par
Sp. 15. Sv. \textit{Mineralier} og \textit{Naturalier} vides her ei af.\hypertarget{Schn1_83621}{}Om Jngøens Præstegield.\par
Sp. 16. Sv. De nærmeste Gaarder til \textit{Kielvigens} Præstegield ere i Vester af \textit{Jngøens} Præstegield paa faste Land \textit{Refsbotten}, og paa \textit{Magerøe Tuenæss}; J Øster af \textit{Kiølle}-fiordz Præstegield paa det faste Land ere i Søer \textit{Laxfiord-Botten}, fra \textit{Porsangerbotten} i Nordost 6 Field- eller 4. Søe-Mile, i Nord er \textit{Sverholt}, fra \textit{Maasøe} i Øster ved 4 Mile, fra \textit{Laxfiordbotten} i Nord til Osten en 5 Mile.\hspace{1em}\par
Sp. 17. Sv: Vei imellem dette Præstegield og \textit{Sverrig} fares ej af \textit{Normænd}, men alleene af \textit{Finner} fra \textit{Porsanger}-Botten til \textit{Karasjok}, det Svenske Markested, om Sommeren til foedz, og om Vinteren i Kieredster ‒\par
Og som det vilde Hav imellem \textit{Nord-Capen} paa denne \textit{Magerøe}, og \textit{Kinner-odden} af \textit{Omgangs} Land, imod en 6 Miile vidt, staaer lige ind paa \textit{Porsanger}-Fiord indtil dens Botten, over 10 Mile dybt, som paa denne Aarsens Tid opreiser en stor Søe-Gang; Saa hindrede den, at, endskiønt man ventede efter Field-\textit{Finnerne} i 8 Dage, disse dog ei kunde komme ud af Fiorden til \textit{Majoren}, og denne ei ind til dennem for sterk Uvejers Skyld;\par
At man nu ved Vinterens og Vejrets Tiltagende ei skulle vorde \textit{coupered} fra \textit{Ost-Finmarken}, \textit{resolverede} man, da Vinden og Søen en Dag sagtedes,
\DivII[Sept. 30. Fra Kjelvik til Kjøllefjordsbotn]{Sept. 30. Fra Kjelvik til Kjøllefjordsbotn}\label{Schn1_83737}\label{Schn1_83739} \par 
\begin{longtable}{P{0.74375\textwidth}P{0.0946590909090909\textwidth}P{0.01159090909090909\textwidth}}
 \hline\endfoot\hline\endlastfoot d. 30 \textit{Sept:} at fare Øst-ad over \textit{Porsangers}-Gab forbi dets Ostlige Næss \textit{Sverholt-Klubben}, hvilket \textit{Sverholt} skiller \textit{Vest-} og \textit{Ost-Finmarken} ad\tabcellsep 1 1/4 sterk SøeMiil,\tabcellsep \\
Siden videre over Gabet af \textit{Laxefiord} til deels, J Øster til Norden, til \textit{Stor-Finnkirken}, et Næss af \textit{Omgangslandet}\tabcellsep 1 1/4 ‒\\
derfra ind til \textit{Kiøllefiordz} Botten i Sydost\tabcellsep 1/2\tabcellsep 3 Mile\\
\tabcellsep \multicolumn{2}{l}{___________________}\end{longtable} \par
 \par
d. 1 \textit{Octob.} giorde man herfra Bud ind ad \textit{Laxefiorden}, med \textit{ordre} til \textit{Missions} Skolemesteren og \textit{Finne}Lensmanden, at samle de kyndigste Søe- og Field-\textit{Finner}, der om Grændsens Gang kunde give nogen Beskeeden;\par
d. 2 og følgende Dage havde man bekiendte \textit{Normænd} og Søe-\textit{Finner} for sig, som om dette \textit{Kiøllefiordz} Præstegieldz Leje og Leilighed meddeelte følgende Underretning:
\DivII[Oplysninger om Kjøllefjord prestegjeld]{Oplysninger om Kjøllefjord prestegjeld}\label{Schn1_83848}\par
til 1 Sp. Svar: \centerline{\textbf{Kiøllefiord}}\hspace{1em}\par
er det første Præstegield i \textit{Ost-Finmarken}, nærmest Østen for \textit{Kielvigens} Præstegield, og Vesten for \textit{Vatsøe} Præstegield, har uden for sig i Nord, \textit{Nord-Søen}, og inden for sig i Søer til Fieldz, \textit{Arisbye}, eller \textit{Otzjock}, den \textit{Svenske Lappe}-Markested.\par
\textit{Kiøllefiords} Præstegield har\par
1) \textit{Kiøllefiords} Hoved-Kirke paa Botten af \textit{Kiølle}-Fiord paa \textit{Omgangs}Landet dets Vestre Side.\par
2) \textit{Omgangs Annex}Kirke, staaendes paa \textit{OmgangsOdden}, den Østerste af dette \textit{Omgangs-} Land, og det Vesterste Næss af \textit{Tana}fiord, 4 Mile Østen for \textit{Kiølle}-fiordz HovedKirke til Søes, og næsten ligesaa langt til Landz.\par
3) \textit{Tana-Annex}Kirke, staaendes ved \textit{Tana}-Elv, lige oven for \textit{Gulholmen}, en kort 1/4 Miil Sønden for \textit{Tana}-Elvs Mund, paa dens Vestre Elv-Bræde af det Jndre faste Land, fra \textit{Kiølle}- fiordz Hoved-Kirke i Syd-ost knappe 5 Mile, og Sønden for \textit{Omgangs Annex}Kirke 4 Mile.\hypertarget{Schn1_83975}{}Schnitlers Protokoller V.\par
4) \textit{Lebbesbye FinneCapell}, omtrent mitt i \textit{Laxe}-Fiord, paa dens Østre Fiord-Bræde nær ved et Finne-Sæde, kaldet \textit{Lebbesbye}, i Botten af en Bugt af samme Navn, 3 Mile Sønden for \textit{Kiølle-Fiordz} HovedKirke, 3 1/2 Miile Vesten for \textit{Tana-Annex}Kirke; Alle 4 Kirker er af Træ .‒\hspace{1em}\label{Schn1_84005} \par 
\begin{longtable}{P{0.4163265306122449\textwidth}P{0.15612244897959185\textwidth}P{0.07979591836734694\textwidth}P{0.12489795918367348\textwidth}P{0.07285714285714286\textwidth}}
 \hline\endfoot\hline\endlastfoot 1) \textit{KiølleFiords} Hoved Kirke har\tabcellsep \Panel{\textit{Normænd},}{label}{1}{l}\tabcellsep \Panel{\textit{Søefinner},}{label}{1}{l}\tabcellsep \Panel{\textit{Fieldf.}}{label}{1}{l}\tabcellsep \Panel{\textit{Summa}}{label}{1}{l}\\
‒\tabcellsep 21\tabcellsep \tabcellsep \tabcellsep 21\\
2) \textit{Omgangs Annex}\tabcellsep 7\tabcellsep \tabcellsep \tabcellsep 7\\
3) \textit{TanaAnnex}\tabcellsep \tabcellsep 19\tabcellsep 11\tabcellsep 30\\
4) \textit{Lebbesbye FinneCapell}\tabcellsep \tabcellsep 28\tabcellsep 4\tabcellsep 32\\
\tabcellsep \multicolumn{4}{l}{_________________________}\\
\tabcellsep \tabcellsep \tabcellsep \multicolumn{2}{l}{\textit{Summa 90}}\end{longtable} \par
 \hspace{1em}\par
\textit{Kiøllefiords} Præstegieldz yderste Gaarder i Vester ere i Søer \textit{Lax-Fiord-Botten}, fra \textit{Porsang}- botten i \textit{N.O}. 7 Field Mile, i Nord \textit{Sverholt}, fra \textit{Lax-Fiordbotten} i Nord 5 Mile; De yderste Gaarder i Øster ere i Søer ved \textit{Tana}-Elv\textit{Bolma}, fra \textit{Laxfiordbotten} i Øster til Sønden 5 FieldMile; J Nord \textit{Omgangs}-Gaarden paa Omgangs Landets Ostlige Odde, fra \textit{Bolma} i Nord over 6 Søe-Mile.\par
til 2 Sp. Svar: \textit{Kiøllefiords} Gield bestaaer af den Halv-Øe \textit{Omgang}, og det Øvrige faste Land, saa det ligger aabent for det vilde Hav i Nord-Søen; Jnde i Fiordene ligge vel Holmer, men u-beboede, hvilke ved Fiordene beleiligst beskrives. Den Halv-Øe \textit{Omgang} er mest rundagtig, ved 9 Søe-Mile omkring stor, mestedeelen omflødt, sammenhengendes ikkun ved et kort Eid, eller \textit{Jsthmus} af 4 Bøsseskudz Vidde, paa den Søndre Side, med det Jndre Faste Land, hvorpaa ved SøeKanten kun boe 19. \textit{Normænd}, det øvrige er øde og u-beboet; Landskabet inde paa Halvøen er berget med nogle Græss-Dale imellem, hvor Field\textit{finner} om Sommeren sidde, uden Skoug, undtagen lidet smaa Bierke-Riis, hist og her kan findes; Dette \textit{Omgangs}Landz Søe-Kuste beskrives siden ved Gieldets Strekning paa næstfølgende Side, fra \textit{Storfinn-kirken} af at reigne. ‒\par
Det faste Land af \textit{Kiøllefiords} Gield forklares nu saaledes:\par
J \textit{Kielvigens} Gield har man \textit{pag:} 274 ophørt ved \textit{Porsanger}-Fiordz Østlige Næss, som er \textit{Sverholt-Klubben}; Dette samme \textit{Sverholt} er det Vestlige Næss af \textit{Laxefiord} i \textit{Kiølle}Fiordz Gield, strax om hvilket Næss \textit{Laxefiord} aabner sig i Søer; Det Østlige Næss af denne \textit{Laxefiord} heder \textit{Stor-Finnkirken}, et Næss fra \textit{Omgangs}-Land udstikkendes i Nord-vest, temmelig høyt, brat berget og bart, spidz opad som et Taarn; Jmellem disse 2de Næss er \textit{Laxefiord} i Gabet viid 1 1/4 Mil. Strax Norden om \textit{Storfinnkirken} møder Fiorden\par
\textit{Kiøllefiord}, 1/4 Miil viid, hvoraf bem.te \textit{Storfinn-kirken} er det Søndre- og \textit{Kiølle-Fiord- Næss} det Nordre Næss; Dette sidste er høyt, brat og skallet, fladt ovenpaa, et par Bøsseskud over stort; Fiorden er 1/2 Miil i S.O. dyb. 1/4 Miil i Nord-ost fra \textit{Kiøllefiords} Næss er\par
\textit{Oxe-fiord}, dens Syd-vestlige Næss heder \textit{Kielnæss}, lige som \textit{Kiølle}fiordz Næss, skabt, Landskabet herimellem er steen-berget brat og høyt, forlandet, u-beboeligt.‒\par
Det Nordostlige Næss af \textit{Oxefiord} heder \textit{Strub-Kin}, lige-dannet, som \textit{Kielnæss}; Derimellem er \textit{Oxefiord} i Gabet 1/2 Miil viid, og stikker ligesaaviid i Søer 1/2 Miil lang; Fra Botten indgaaer en liden Poll, kalded \textit{Valen}, i Syd-ost 1/8 Miil dyb, nemlig fra den Østre Land-Sides \hypertarget{Schn1_84309}{}Om Kiøllefiords Præstegield. Botten; Thi fra den Vestre Land-Sides Botten strekker den fra sig ud i Søer til Vesten en Arm, 1/8 Miil og mindre smal, imod 1/2 Miil lang; saa at Fiorden med denne Arm bliver i alt ved 3/4 Miil dyb i Søer til Vesten.\par
Fra \textit{Strubkin} i Nord-ost til \textit{Kinner-odden}, det Nordreste Næss af \textit{Omgangs}-Land er 1/2 Miil; \textit{Kinner-Odden} er høyberget, steneed; Landet derimellem er ligedant, forlandet og ubeboeligt.\par
Østen for \textit{Kinner-odden} 1/4 Miil er \textit{Makeil}, det Vestre Næss af \textit{Normands}Sæd-\textit{Fiorden}, høyberget, som de andre Næss, og Landet derimellem steenet, steilt, forlandet, u-beboet; Det Østre Næss av \textit{Normands} SædFiord kaldes \textit{Kamøe}, og ligger fra \textit{Makeil} 1/2 Miil i Øster; Denne \textit{Kamøe} er omflødt, fra det faste Land 2 Bøsseskud i Nord, afliggendes, rundvoren, 1/16 Miil omkring stor, høyklipped, u-beboed. Jmellem disse 2 Næsse er\par
\textit{Normands}Sæd\textit{Fiord}, saa kalded, [ af] at \textit{Nordlands}farer have ligget der med deres Jægter; Denne \textit{Normands} Sæd-\textit{Fiord} er fra det Vestre Næss i Søer 1/4 ‒ og fra det Østre Næss 3/4 Miil dyb. 1/8 Miil Østen om \textit{Kamøe} forekommer\par
\textit{Skidenfiord}, dens Østre Næss heder \textit{Bisp}, liggendes fra \textit{Kamøe} i Øster 1/8 Miil; det Næss \textit{Bisp} er høyt, bart, og steilt; Af Siden paa dette Næss-Field opstaaer en spidz, hvass Tinde, vel 30 Favner høy, som eegentlig bær det Navn, \textit{Bisp}; Hvorimellem \textit{Skidenfiord}, 1/8 Miil viid, i Syd til Vesten er 1/4 Miil dyb. Strax Østen om \textit{Bisp} begynder\par
\textit{SandFiord}, 1/4 Miil i Gabet viid til dens Ostlige Næss \textit{Varnæss}, som og kaldes \textit{Sletnæss}; og fra Vestre Næss i Søer til Vester 1/4 ‒ men fra Østre Næss 3/4 Miil dyb; ‒\par
\textit{Varnæss} er rundvoren, 1/8 Miil stort, fladt og slet, med Græss og Lyng paa, u-beboet. ‒ Fra \textit{Varnæss} i Ost-Syd-ost er 1 Miil til\par
\textit{Omgangs} Odden, den Østerste Pynt af \textit{Omgangs}-Landet, og det Vesterste Næss af \textit{Tana}- Fiorden; Denne \textit{Omgangs Odde} er flad med Græss og Lyng paa, har en vakker Slette af 1/3 Miil, hvor \textit{Omgangs Annex}Kirke, og det aarlige \textit{Norske} Tingsted staaer, og 3 \textit{Normænd} boe. ‒ Den Fiord imellem \textit{Varnæss} og \textit{OmgangsOdden} er\par
\textit{Koifiord}, 1 Miil i Gabet viid, 1/2 Miil i Syd-Vest dyb, med 2de Poller i Botten; Ved dens Botten ligger den Øe, \textit{Koi-øe} berged, 1/4 Miil omkring stor, ubeboed. ‒ 1/2 Miil Sønden for \textit{OmgangsOdden} ligger den Bugt\par
\textit{Finnkonkeil}, et par Bøsseskud viid, og 1/16 Miil dyb. Landet derimellem er høyberget, bart, ubeboet. Sønden for \textit{Finnkonkeil} 1/4 Miil er\par
\textit{Mollvig}, et par Bøsse-Skud i Gabet viid, og ligesaa dyb; begge disse Bugter ere ubeboede. Sønden om \textit{Mollvig} møder\par
\textit{DiFiord}, 1/8 Miil i Gabet viid, og 1/4 Miil dyb i Vester, u-beboed. Sønden for \textit{Difiord} 1/8 Miil forekommer\par
\textit{Rafiord}, i Munden viid 1/8 Miil, og ligesaa dyb i Vester, u-beboed. Strax Sønden om dens lidet Næss møder\par
\textit{Skarfiord}, 5 à 6 Bøsseskud i Gabet viid, 1/8 Miil i Vester dyb, u-beboed. ‒ 3/4 Miil Sønden for \textit{Skarfiord} ligger\par
\textit{Kieskelvig}, et par Bøsseskud i Gabet viid, 1 Bøsse-skud dyb i Vester, u-beboed. Fra \textit{Omgangs Odden} til denne \textit{Kieskelvig} holdes for at være 2 gode Mile. Fra \textit{Kieskelvig} kroger \hypertarget{Schn1_84562}{}Schnitlers Protokoller V.\textit{Omgangs} Land sig i Vester, og giør \textit{HopsFiord}, som i Gabet er 1/2 Miil viid til dens Sydlige Næss \textit{Diger-mulen}.\par
\textit{Hop-Fiord} stikker i Vester til Norden 1 1/4 Miil lang, hen til Ejdet, hvilket Eid er langt en 4. Bøsse-Skud over, og giør at \textit{Omgangs Pen-insul} med det Jndre Faste Land henger sammen; Naar man over dette Eid i Vester er kommet, møder\par
\textit{Eids-vaag}, som gaaer ud i Vester til Norden god 3/4 Miil til det Næss paa \textit{Omgang}, kaldet\par
\textit{Eidsvaag-Næss}, høyt, og bart; i \textit{Hops}fiord boe 3 \textit{Normænd}, men paa \textit{Eidet} og i \textit{Eidsvaag} Jngen. ‒ Fra \textit{Eidsvaag-Næss} i Vest-Nord-Vest er 1 Miil til\par
\textit{Løke-vig-Næss}; Mitt imellem disse 2de Næss er\par
\textit{Kiefiord}, 1/16 Miil viid, knap 1/4 Miil i Nord-ost lang. ‒ J Ost 1/16 Miil fra \textit{Kiøf[i]ord} er den Havn \textit{Viig}, hvor Jmellem \textit{Eidsvaag-Næss} og \textit{Kiøfiord} boer 1 \textit{Normand}; og imellem \textit{Kiefiord} og \textit{Løkevig-Næss} 3 \textit{Normænd} paa Sletter: Ellers er Landskabet derimellem fieldet og steenet. Fra \textit{Løkevig-Næss} i Nord-Nord-vest er et Par skud til\par
\textit{Trøtvig-Næss}, derimellem er \textit{Løkevig}, u-beboed. Trøtvig-Næss er noget høyt, steenberget, fladagtigt ovenpaa. Fra \textit{TrøtvigNæss} i Nord til\par
\textit{Storfinnkirken} er 3/8 Miil, og Landet steenberget. Hvilket \textit{Storfinnkirken} er det 1te Næss, hvorfra \textit{pag.} 284\textit{Omgangs}Beskrivelsen begyndtes, og saaledes er dette \textit{Omgangs} Land runden om vel 9 Miile stort. ‒\par
Før er rørt om \textit{Hops}Fiorden, at dens Søndre Næss heder \textit{Digermulen}; Dette \textit{Digermulen} holdes og for at være det Vestre Næss af \textit{Vestre-Tana}-fiord; Ligesom \textit{OmgangsOdden} er det Vestre Næss af heele \textit{Tana}fiord. Jmod denne \textit{OmgangsOdden} i Syd-ost 1 sterk Miil ligger det Østre Næss af \textit{Tana}fiord,\par
\textit{Tana-Hornet}, høyt og spidzt op ad tver-bratt, og bart. 1/2 Miil i Øster til Syden fra \textit{TanaHornet} er\par
\textit{Perle-vaag}; Landet derimellem er slet med Græss og Lyng paa; Ligeledes det og er slet omkring \textit{Perlevaag; Perlevaag} er i Gabet et par Bøsse-skud viid, og ligesaa dyb. ‒\par
Fra \textit{Perlevaag} i Øster til Syden er knap 1/2 Miil til\par
\textit{Kiølnæss}; og Landet derimellem ligesom \textit{Kiølnæss} i sig selv er lavt, slet, græsset og lynget. Fra \textit{Kiølnæss} i Syd-ost er 1/4 Miil til\par
\textit{Løkvig}; Landet derimellem er ud ad Søen slet med Græss paa, siden gaaer det ind ad op i Veiret berget; det samme giør og Landet om \textit{Løkvig}.\par
Dette \textit{Løkvig} er det Sidste af \textit{Kiøllefiords} Præstegield, og boer Jngen imellem \textit{TanaHornet}, og \textit{Løkvig}; Er saa dette \textit{Kiøllefiords} Gield langt fra Vester i Øster, efter det faste Landz Søe-Kuste gode {7 1/4 Miile}\par
som saaledes \textit{specialiter} forklares:\par
Fra \textit{Sverholt}, det Vestre Næss af \textit{Laxefiord}, hvor \textit{Kielvigs} Gield sig ender, i Oster til Norden til \textit{Storfinn-kirken}, det Østre Næss af \textit{Laxe}fiord, er stor {1 1/4 Miil}\par
Fra \textit{Storfinnkirken} i Nordost til \textit{Kinner-odden}, det Nordeste Næss af \textit{Omgangs}- Landet,{1 3/4 ‒}\par
Fra \textit{Kinner-odden} i Øster til Sønden til \textit{OmgangsOdden}, det Vestreste Næss af \textit{Tanafiord}{2 ‒}\hypertarget{Schn1_84898}{}Om Kiøllefiords Præstegield.\par
Fra \textit{OmgangsOdden} i Ost-Syd-ost til \textit{Tana-Hornet}, det Østre Næss af \textit{Tana}- fiord, vel {1 Miil} fra \textit{TanaHornet} i Sydost til \textit{Løkvig}, hvor \textit{Kiølle}fiordz Gield endes {1 1/4 ‒ ___________ giør gode 7 1/4 Mil}\hspace{1em}\par
Oplandet oven for Søekusten maa siden af Field-\textit{Finnerne} forklares; som skeer \textit{pag.} 317.\hspace{1em}\par
til 3 Sp: Sv: Fisk og Fugl gives i dette \textit{Kiøllefiords}-Gield, som \textit{pag:} 226 i \textit{Altens} Gield; J Særdeleshed fanges her, og udføres Torsk, som virkes til Rundfisk, Raaskiæring og Stokfisk, og Lax i \textit{Tana}-Elv, som af det Slags er den Fisk-rigeste i heele \textit{Finmarken}; Jmellem \textit{NordCapen} paa \textit{Magerøe}, og \textit{Kinnerodden} paa \textit{Omgangs} Land søge mangfoldige Hvale ind, om Sommeren, ind ad \textit{Porsanger} og \textit{Laxe}-Fiordene; ligesom meldt er, at de giøre i \textit{Tromsøens} Fogderie i \textit{Nordlands} Amt; og fortælles her, at i gamle Dage \textit{Hollænderne} skal have ligget i \textit{Laxefiorden}, og i denne Egn fanget Hvaler, hvoraf man finder endnu denne Dag Kiendetegn i \textit{Laxe}fiorden paa \textit{Timmervigs}Landet, nemlig Hval-Been at ligge efter dem.\hspace{1em}\par
til 4 Sp. Svar: Havner haves i \textit{Laxefiord}\par
1) J den Jndfiord \textit{Lille Torske}Fiord paa \textit{Lax}fiordens Østre Side, for et par Skibe.\par
2) i den Jndfiord \textit{Bekkerfiord}, sammestedz strax Synden om \textit{Torske-Fiord} for et par Skibe.\par
3) Paa \textit{Omgangs} Halv-øe J \textit{Vigen}, paa dens Vestre Side for et par Skibe. Sammestedz paa den Søndre Side\par
4) J \textit{Hop}, en Vaag for 4 à 5 Skibe.\par
J \textit{Tana}-Fiord gives ingen Vinter-Havn; Skibene om Sommeren komme ej nærmere Handlen, paa \textit{Gulholmen}, end ved \textit{Stagnæss}\hspace{1em}\par
til 5 Sp. Svar: \textit{Fiorder} i \textit{Kiøllefiords} Præstegield ere:\par
Jnde i det faste Land, Østen om \textit{Porsanger}-Fiordz Østlige Næss \textit{Sverholt-Klubben} begynder strax\par
1. \textit{LaxeFiord}, hvis Vestre Næss er samme \textit{Sverholt}, og Østre Næss er \textit{StorFinnkirken} paa Omgangs Landet, hvorimellem \textit{Laxe}fiord i Gabet er viid 1 sterk Miil, og i Søer til Vester er dyb, ved {5 Mile} Som i Særdeleshed saaledes forklares:\par
Fra \textit{Sverholt} i Søer til det Søndre \textit{Timmervig-Næss} er {1 Miil}\par
Landskabet derimellem er høyberget, med lidet Græss paa Fiord-Bræden, men oven derfor steenet, et ubærgeligt Forland, og u-beboeligt.\par
\textit{Timmervig} er i Gabet 3 Bøsseskud viid, 6 Bøsseskud dyb; Paa Sidene heraf er Landet steilberget, men inde i Botten fladt med Græss, og Bierkeskoug, strekkendes i Vester imod 1/2 Miil, hvorpaa i Vester følger slette bare Berg-Hejer alt over til \textit{Porsanger}fiord; Hvorhentil herfra \textit{Timmervig} meenes at være 1 knap Miil over Land i Vester; Dog er i denne \textit{Timmervigs} Botten saa grundt, og lavt Vand, at man \hypertarget{Schn1_85179}{}Schnitlers Protokoller V. ikke uden ved Flod-Søe, og det kun paa eet Sted, med Baad kan komme i Land; Hvoraf det er, at nu Jngen her boer. ‒ Uden for ‒ eller Østen for dette \textit{Søndre Timmervigs-Næss} et par Bøsseskud ligger\par
\textit{Kartsøe}, en Holm fra \textit{Laxefiordens} Vestlige Næss \textit{Sverholt} 1 Miil liggendes i Søer, ved Fiordens Vestlige Bræde, rundagtig, 1/16 Miil omkring stor, fladberged, steened, med noget lidet Græss og Multer paa, u-beboed.\par
Fra \textit{Timmervig-Næss} i Søer til Vesten er til \textit{Lill-Porsanger}Fiordz Nordre Næss, \textit{Nubbe-Næss}{3/4 Mil} Landet herimellem er smaa-bakket med Bierke-Riis paa, lidet græss-groet, forlandet, og derfor u-beboeligt.\par
Mitt imellem \textit{TimmervigNæss} og \textit{Nubbenæss} ere \textit{Hendriks-Holmene}, et par Bøsseskud fra Vestre Fiord-Bræde, tæt efter hinanden liggende; den Nordre 1 Riffel-Skud lang, halv saa breed, slet og græssed; den Søndre rundvoren, 1 Steenkast over viid, steened og slet.\par
\textit{Lill-Porsangers} Søndre Næss heder \textit{Veinæss}; Jmellem dett og \textit{Nubbe-Næss} er \textit{LillePorsanger}fiord i Gabet viid {1/4 Mil} og stikker i Syd-vest en knap 1/2 Miil dyb. Paa denne Jndfiordz nordre Bræde er Landet allvejs indtil Botten slet, Eet Bøsseskud bredt, med Bierk bevoxen, siden stiger det op i Veiret til bare skallede Berg-Hejer, som rekke alt hen til \textit{Store Porsanger}-Fiord i Vester; Den Søndre Fiord-Bræde er smaa-bakked, bar neere ved Fiord-Bræden, men op ad Bierke-groed; Jnde i Botten af denne \textit{Lill-Porsanger}-Fiord er Landet slet i Syd-vest 1/4 Miil vejs med mangfoldig Bierk bevoxen. Fra denne Fiord-Botten lige over i Vester slutes at være 1/2 Miil, men i Syd-vest til \textit{Kiæs}-Elv, hvor Post-Veien gaaer, reignes det for 1 Miil over til \textit{Porsanger}-Fiord. I denne Jndfiord boer Jngen, Men Opsidderne paa Øen \textit{Veinæss} have deres Høe-Slotte herinde. Uden for- eller Østen for \textit{Lill-Porsanger}fiordz Sydlige Næss ligger Øen\par
\textit{Veinæss}, som ved Flod-Søe omflydes, men i Fiære-Tid fra det faste Land er tilgængelig, trekanted, 1 à 2 Bøsseskud i hver Kant breed, slet, græssed, beboes af \textit{Finner} om Sommeren. Fra \textit{Veinæss} i Søer til Vesten er til \textit{Masternæss}{1 Mil}\par
Landet derimellem er noget slet paa Fiord-Bræden, et par Bøsseskud bredt, u-beboet, med Bierk-Riis bevoxen, derovenfor er bare Berg-Hejer (som ere flade skallede Berge) i Vester alt over til \textit{Porsanger}Fiord 2 FieldMile. Fra \textit{Masternæss} i Søer til Vesten er til \textit{Korsnæss}{3/4 ‒} Landet ligedant, som næst tilforn omtalt, u-beboet: men paa \textit{Korsnæss} sidde Field- \textit{finner} om Sommeren.\par
Fra \textit{Korsnæss} i Søer til Vesten er til Botten, god {1 ‒} Landet derimellem, er ligesom næstforrige, u-beboet; men i Botten ere 2de \textit{Finne}- Byer, eller \textit{Finne}-Vaan-steder; med Bierk og 1/4 Mils Slette. Jmellem \textit{Korsnæss} og Botten, dog nærmere til hine, og til det Vestre Landz Side ligger \textit{Rein-øe}. ‒ {_________ i alt dyb 4 3/4 Mil}\hypertarget{Schn1_85357}{}Om Kiøllefiords Præstegield.\par
\textit{Reinøe}, fra Nord i Søer mod 1/4 Miil lang, 1/12 Miil omtrent breed, høy-klipped, med lidet Græss, Bierk, og Reen-moesse paa, beboes om Sommeren af Søe-\textit{Finner}.\par
Paa den \textit{Østre} LandSide af \textit{Laxe}-Fiorden ere Jndfiorder:\par
\textit{Adams}fiord fra Botten i Nord til Osten {3/8 Miil} Landet derimellem er høyberget, med smaa Bierke-Riis imellem, u-beboet. Denne \textit{Adams}Fiord er i Gabet nogle Bøsse-Skud viid, og 1/16 Miil dyb; Landet omkring denne \textit{Adams}Fiord er berget, steenet, med noget Bierk iblant; Jnde i Botten er en liden Slette, et par Bøsseskud stor, hvor \textit{Norske} Søe-\textit{Finner} om Sommeren have deres Bye, eller Tilhold.\par
Fra \textit{Adams}Fiord i Nord til Osten er til \textit{Runnes-Næss} en god {1/4 ‒} Landet derimellem høyberget, steenet, u-beboet. Paa \textit{Runnes-Næss} i Nord følger\par
\textit{Runnes-vigen}, i Gabet viid imod {1/2 ‒} og 1/8 Miil dyb. Landskabet om denne Vig er slet ved Fiord-Bræden, 1/8 Miil bredt, siden stiger det op i Veiret, og bliver berget; Omkring Vigen er smaa Bierk, og lidet græss, og boe der 3 \textit{Norske} Søe-\textit{Finner.}\par
Det Nordre Næss af \textit{Runnes-Vig} heder \textit{Bomnæss}, uden for hvilket \textit{Bomnæss} et par Bøsseskud i Vester er\par
\textit{Bonøe}, lang fra Vester i Øster 1/4 Miil, omtrent halv saa breed, slet og lynged, uden Græss og skoug, u-beboed. J Nord-nordost fra \textit{Bomnæss} til \textit{Hamnæss} er {1/4 ‒} Landet derimellem er bratagtigt, og have \textit{Norske} Søe-\textit{Finner} om Sommeren her deres Sommer-Sæde. Nær udenfor \textit{Hamnæss} ligger den Holm\par
\textit{Langøe}, fra Søer i Nord knap 1/8 Miil lang, halv saa breed, slet med Lyng og Græss paa, u-beboed. Strax om \textit{Hamnæss} er\par
\textit{Landersfiord}, det nordre Næss heraf heder \textit{Skiaa-Næss}, hvorimellem \textit{Landersfiord} i Gabet er viid {3/4 ‒} Landet om \textit{Landers}fiord er sommestedz steenberget, sommestedz slet, med Græss og god Bierke-Skoug i Botten, beboet af 4 Søe-\textit{Finner}. Strax Vesten for \textit{Skiaa-Næss} er\par
\textit{Skiaa-Holm}, u-beboed, lang i Nord-ost 1/6 Mil, halv saa breed, slet og græssed. Fra \textit{Skiaa-holm} i Nord til \textit{Løgholm-Næss} er en kort {1/4 ‒} derimellem en Bugt\par
\textit{Læbesbye-Bugt} ligger, hvor \textit{Læbesbye Finne-Capell} under \textit{Kiøllefiords} Præstegield, og 3de Søe\textit{finne}-Gammer ere. Uden for ‒eller Vesten for \textit{Løgholm-Næss} findes\par
\textit{Saraholm}, u-beboed, lang fra Søer i Nord 1/8 Miil, halv saa breed, slet med Lyng og Græss paa; Herinden- eller Østen-for ligge 2de\par
\textit{Lille}- og \textit{Store Bratholmer}, høyklippede, med lidet Græss paa; rundagtige, den \textit{Store} kan være 1/8 Miil omkring viid, og ligger 1/8 Miil Norden for den \textit{Lille}, som kan være engang saa liden.\par
Fra \textit{Løgholm-Næss} i Nord til Osten er til det Næss \textit{Kalak}{3/8 ‒} Landet derimellem berget og u-beboet. Dette \textit{Kalak} er det Sydlige Næss af \textit{Bekkerfiord}, dens Nordre Næss heder \textit{Bekkerfiord-Klub}; \textit{Bekkerfiord} herimellem er viid {1/4 ‒}\hypertarget{Schn1_85612}{}Schnitlers Protokoller V.\par
\textit{Bekkerfiord} stikker ind til Botten i Øster 1/2 Miil; Fiord-Brædene heraf ere slet bakkede med Græss og stor Bierk bevoxne, 1/8 Miil og mindre breede, beboede af 8 Søe-\textit{Finner}. Strax om \textit{Bekkerfiord-Klub} begynder\par
\textit{Lill-Torske}-Fiord, hvis Nordre Næss heder \textit{Lill-Torskefiord-Klub}; Jmellem de 2de Næss Fiorden er i Gabet viid {1/8 Mil} og 1/8 Miil dyb i Øster. Landet omkring Fiorden er Lav-berget, u-beboet. Strax om Næsset er\par
\textit{Store TorskeFiord}, hvis Nordre Næss kaldes \textit{Stor-TorskeFiord-Klub}, hvorimellem Fiorden er viid {1/8 ‒} og stikker i Øster 1/2 Miil dyb. Landskabet inde i Fiorden er bratagtigt, noget slet iblant, med Bierk runden om bevoxen, med lidet Græss iblant, u-beboet.\par
Fra \textit{Stor TorskeFiord-Klub} i Nord til Oster er {3/8 ‒} til \textit{Varnæss}; Landet derimellem høyberget, bart, forlandet, u-beboeligt. Dette \textit{Varnæss} er det Sydlige Næss af \textit{Maarøe}-Fiord, dens Nordre Næss i Ost-Nord-ost er \textit{Maarøe}-Fiord-\textit{Klub}, derimellem\par
\textit{Maarøe}-Fiord i Gabet er viid 3/8 Miil, og fra \textit{Varnæss} 1/2 Miil dyb ad Botten. Landet om Fiorden er høyfieldet, med lidet Græss paa Fiord-Bræden, med Bierk iblant, u-beboet. Udenfor \textit{Maarøe}fiord-\textit{Klub} i Vester mod 1/8 Miil er\par
\textit{Maarøe}, lang fra Søer i Nord 1/4 Miil, halv saa breed, høyklipped paa Søndre Ende, slet paa Nordre Ende, med Bierk, Græss og Lyng paa, beboes af 4 \textit{Normænd.}\par
Fra \textit{Maarøe}Fiord-\textit{Klub} i Nord til \textit{Eidsnæss} er 1/4 Miil; Landet derimellem er brat, fieldet, forlandet, u-beboet; Dette \textit{Eidsnæss} er det Søndre Næss af \textit{Eidsfiord}; dens Nordre Næss er \textit{Eidsvaag-Næss}, hvorimellem \textit{Eidsfiord} er 1/8 Miil breed i Gabet, og dyb i Oster til Syden knap 3/4 Miil.\par
\textit{Eidsfiord} kaldes den, fordi den stævner til det Eid, 4 Bøsseskud over vidt, hvorved \textit{Omgangs-Landet} med det Jndre Faste Land sammenføyes; Landet om denne \textit{Eids}fiord er brat, høyfieldet, undtagen imod og paa Eidet, der er Slette med Græss og smaa Bierk paa, dog u-beboed.\par
Fra \textit{Eidsvaag-Næss} i Nordvest til \textit{Kiefiords} Østre Næss er knap 1/2 Miil; Landet derimellem er høyfieldet, bart, med lidet Bierke-Riis paa, dog ligger derimellem en liden Viig, hvor 2 \textit{Normænd} boe, og den SkibsHavn er.\par
\textit{Kiefiord} er i Gabet viid 1/8 Miil, dyb i Nord 1/4 Miil; Landet derinde er fieldagtigt, med smaa Bierk-Riis og lidet Græss iblant, u-beboet.\par
Fra \textit{Kiefiord} i Nord-vest til \textit{Trøtvig-Næss} er 1/2 Miil; Landet derimellem er fieldet, uden Skoug; Strax uden om \textit{Kiefiord} boe 4 \textit{Normænd} i \textit{Brengammen}, hvor det er bakagtigt, græsset, med noget Bierke-Riis i; Ellers er den Strekning u-beboed.\par
Er saa fra \textit{Varnæss}, udentil at reigne, i Nord til dette \textit{Trøtvig-Næss} gode {1 1/4 ‒}\par
Herfra til \textit{Storfinnkirken}, som er det Søndre Næss af \textit{Kiølle}fiord, og det Østre Næss af \textit{Laxefiord}{3/8 ‒ _______}\par
Er saa \textit{Laxefiord} paa Østre Side lang {5 1/4 Mil}\hypertarget{Schn1_85838}{}Om Kiøllefiords Præstegield.\par
Den 2den Hoved-Fiord i \textit{Kiølle}-fiordz Præstegield er \textit{Tana}-Fiord, hvorom ved Ankomsten der skal skee Forklaring, see herefter \textit{pag:} 300 f.\hspace{1em}\par
til Sp. 6. Svar: Elve, Aaer, Fosser og Vande i \textit{Kiølle}fiordz Gield paa \textit{Omgangs} Land ere ingen, uden Bække, hvorj til deels smaa Øreter opgaae. J \textit{Laxefiord} dens Jndfiorder gaae og kun Bække; Mens i dens Botten er det den største, som indgaaer, \textit{Laxfiord}-Elv, hvilken kommer Sønden-fra af det Vand \textit{Ziølle-jaure}, som er 3 Bøsseskud i Nord langt, og 2 Bøsseskud bredt, fra \textit{Gaisak} 1 Miil i Nord; Deraf den rinder 1 Miil i \textit{N}. ‒ see \textit{Adamsfiord}-Elv\textit{pag.} 294.\hspace{1em}\par
Sp. 7. Svar: Dal er ved \textit{Laxefiorden} paa dens Vestlige Side fra \textit{Timmervig} i Vester langs med \textit{Timmervigs} Aaen, l Bøsseskud og mindre breed, 1/4 Miil lang.\hspace{1em}\par
Sp. 8. Sv: Gaardene avle intet Korn, men alleene lidet Høe; Holmene have lidet Eederduun og FugleEgg.\hspace{1em}\par
Sp. 9. Sv: Skoug er ingen paa \textit{Omgangs} Landet, men i \textit{Laxefiordens} Botten Og dens Jndfiorder er Bierk til Brændehved, dog ingen Furre; Hvorfore Jndbyggerne boe i Gammer, eller Hytter, bygde af Steen og Jord.\hspace{1em}\par
Sp. 10. Sv: Vild og \textit{Jnsecter} i \textit{Kiøllefiord}, som i \textit{Kielvigens} Gield\textit{pag.} 282 og 265.\hspace{1em}\par
Sp. 11. Sv. Leilighed til Rødning er i \textit{Læbesbye} i \textit{Laxe}fiord, ligesaa paa \textit{Hops}-Eid: dog herfra er noget langt til Fiskerie.\hspace{1em}\par
Sp. 12 Sv. Fielde ved SøeKanten og omkring Fiordene ere derved forklarede: men de op i Landet beskrives siden af Field-\textit{Finnerne}. \textit{p:} 293 f.\hspace{1em}\par
til 13 Sp. Svar: Toldsteder her ikke: den \textit{Octrojerede} Handel i \textit{Kiøllefiords} Botten forsiuner \textit{Kiøllefiords} Præstegieldz Almue.\hspace{1em}\par
til 14 Sp. Sv: Indbyggernes Næring bestaaer eene i Fiskerie, og fornemmelig af Torsk i Havet, og af Lax i \textit{Tana}-Elv.\hspace{1em}\par
til 15 Sp. Sv: \textit{Mineralia} kiendes ei her.\hspace{1em}\par
til 16 Sp. Sv: De nærmeste Gaarder i Vester af \textit{Kielvigens} Gield paa Søndre Side ere\par
\textit{Haarvig}, fra \textit{Sverholt} af \textit{Kiølle}fiordz Gield i Syd-Syd-vest 2 Mile; Paa Nordre Kant\par
\textit{Kamøe} paa \textit{Magerøe}, fra \textit{Sverholt} i Vester til Norden 1 3/4 Miil. De nærmeste Gaarder af \textit{Vads-øe}Gield i Øster fra \textit{Kiøllefiordz} Gield ere i Søer [åpent rom,] J Nord\par
\textit{Makour} paa faste Land, fra \textit{Omgangs}Gaardene i Øster 4 Mile. ‒\par
til 17 Sp. Sv: Vej herfra er ingen, uden for \textit{Finner}, som forklare den ved 27de Sp:\hspace{1em}
\DivII[Okt. 10.-13. Rettsmøte i Kjøllefjord]{Okt. 10.-13. Rettsmøte i Kjøllefjord}\label{Schn1_86095}\par
\textbf{A}o \textbf{1744. d. 10 Octob:} ankom fra \textit{Laxefiorden, Missions} Skolemesteren sammestedz, \textit{Jon Hendriksen}, med 2de \textit{Norske} Søe \textit{finner} og 1 Norsk Field\textit{finn;} hvorpaa man strax satte Retten i \textit{Kiøllefiord}, i Nærværelse af Svend Eriksen, paa Lensmandens vegne og 2de Laug- \hypertarget{Schn1_86130}{}Schnitlers Protokoller V. Rettes Mænd; Videre Kongelige Betientere ej vare tilstæde; J Hines Overværelse blev den Kongelig allernaadigste \textit{ordre} til dette Forhørs Holdelse oplæst, og efter at Eedens Forklaring af Lovbogen ved Tolken Erik \textit{Hælset} var oplæst og betydet, aflagde Vidnerne deres \textit{Corporlig} Eed til at sige deres Sandhed om Grændse-Gangen imellem \textit{Norge} og \textit{Sverrig}, saavidt dennem bevist var;\par
\centerline{\textbf{14de Vidne i Finmarken}\textit{Erik Clementsen, Norsk Finne} Lensmand}\par
i den \textit{Norske}\textit{Laxe}fiord i \textit{Kiøllefiords} Præstegield i \textit{Ost-Finmarken}, født i \textit{Laxefiord} af \textit{Norske} Søe-\textit{Finner}, døbt i \textit{Sverholts} forrige \textit{Annex}Kirke, 60 Aar gammel, gift, har 5 Børn, været for en Maanedz Tid til Gudz Bord i det \textit{Norske Læbesbye-Capell} i \textit{Laxefiord}; J sin Ungdom har han faret med Reen tilFieldz, som \textit{Norsk} Field\textit{Finn}, hen til \textit{Tana}-Elv og \textit{Arisbye}, men som han ej foer længer ovenfor i Søer, blev han af de Kongl. Svenske Betientere ingenlunde hindred, ei heller Skattlagd, eller kalded for den Svenske \textit{Lappe}-Rett: men siden han blev noget gammel, har han begivet sig Field-Reisen, og satt sig ned, som stadig Søe-\textit{Finn} i den \textit{Norske Laxefiord;} Jmidlertid han har været Field-\textit{Finn}, har han alleene skattet til Norge ‒\hspace{1em}\par
til Sp. 18. Svar: \textit{Norske} Field\textit{Finner} ere 4. og Søe-\textit{Finner} 28. i \textit{Laxefiord}, alleene \textit{privative Norske} Undersaattere af det \textit{Norske Læbesbye Finne Capell} i \textit{Laxefiord} under \textit{Kiøllefiords} Præstegield, det samme og nærværende \textit{Norske Missions} Skolemester \textit{Jon Hendriksen}, som har disse Folk under Underviisning, stadfæstede; de \textit{Norske} Field\textit{Finner} sidde om Sommeren med deres Reen paa det Næss eller Tang imellem \textit{Laxefiord} og \textit{Porsangerfiord}, men imod Vinteren fløtte de i Søer imod \textit{Tana}-Elv og til \textit{Arisbye}, hvorfor de intet til den \textit{Svenske Crone} svarer.\hspace{1em}\par
til 19de Sp. Hvormange fælledz Field-\textit{Finner} Sønden for \textit{Laxefiord} Botten i \textit{Arisbye Annex}Sogn ere, som svare Skatt til begge Riger; det veed han ikke; Ellers paa Tilspørsel forklarede, at disse \textit{Arisbye} fælles Field\textit{Finner} komme hid Nord om \textit{Korsmøss, (først in Majo)} og ligge paa Fieldene imellem \textit{Arisbye} eller \textit{Tana}-Elv og de \textit{Norske}\textit{Laxe}- og \textit{Tana}-Fiorder, indtil Om \textit{Ols-Uge}, (sidst \textit{in Julio}) da de Søer efter til \textit{Arisbye}, over \textit{Tana}-Elven tilbagefløtte, og der Vinteren over forblive; Jmidlertid de ere her saa nær Fiordene om Sommeren, søge de ikke Norske Kirke, eller Ting, eller de \textit{Norske} Forsamlings Husse, hvor \textit{Missions} Betienten læser; Hvilket og tilstædeværende \textit{Missions} Skolemesteren bekræftede, sigendes, at have alleene med de \textit{Norske} Søe- og Field-\textit{Finner} at bestille, og gaaer han ikke til Fieldz til nogen af de Field\textit{Finner}.\hspace{1em}\par
til 20 Sp: Sv: De fælledz Field\textit{finner} bruge Fieldene, Vandene og Skougen baade Norden- og Sønden for \textit{Arisbye} tilfælles, uden at noget Field-Land imellem dem er afdeelt, farendes, og liggendes, hvor de vill.\hspace{1em}\par
til Sp. 21. Svar: De vide ikke andet, end at det Land, Sønden for \textit{Karasjok, Juxbye} og \textit{Arisbye} hidindtil har været holdet for fælledz, saa at \textit{Arisbye} og \textit{Juxbye-Finner}(som her til\hypertarget{Schn1_86428}{}Om Kiøllefiords Præstegield. forn har været) have skattet til begge Croner, mens det Land Norden for \textit{Arisbye} og \textit{Juxbye} har været agtet alleene for Norsk Grund, saa at de Norske Field-\textit{Finner}, som sidde Norden for \textit{Arisbye} og \textit{Juxbye} have aldrig haft med den Kongel. Svenske Øvrighed at bestille; dog ere, som meldt, de \textit{Arisbye}-fælledz\textit{Finner} vandte til, at fløtte hid Nord over \textit{Tana}-Elven nær de Norske Fiorder, om Sommeren.\hspace{1em}\par
Sp. 22. Svar: Naar det Fælleskab i Skattens Oppebørsel af begge Croner har taget sin Begyndelse; vides ej: men det har været saaledes med Skatten, saavel som med de Kongelige \textit{Svenske} Markesteders Holdelse alt fra u-mindelig Tid: Det erindrer dog Vidnet, at nogle faa Aar for Amtmand \textit{Lorkes} Tid, nemlig imellem 40. og 50 Aar siden er \textit{Arisbye Annex}- Kirke ved den Aae \textit{Otzjok}, som løber Søndenfra i \textit{Tana} Elv, af de Kongelige Svenske Betientere bleven først bygd, og for den Tid har ingen Kirke, men alleene Svensk Markested og Ting ved samme \textit{Otzjok} været; Og har den Svenske \textit{Missionaire} før præket i en Timmer-stue.\hspace{1em}\par
Sp. 23. Sv: Hvad fælles Field\textit{finner} i \textit{Arisbye} svare i Skatt til \textit{Sverrigs} Crone, og hvormeget til \textit{Norge?} veed han ei. De dømmes ellers af den Kongelig Svenske Øvrighed efter den Svenske Lov; og \textit{Koutokein} er HovedKirken for denne \textit{Arisbye Annex}Kirke.\hspace{1em}\par
Sp. 24. Sv. Grændse-skiell veed Vidnet intet om at forklare, uden det at have hørt at giøre \textit{Kiølen}, nemlig \centerline{\textbf{Jauris-Duøder} og \textbf{Beldo-Duøder:}}\par
Men hvor, og hvorlangt de ligge fra de Markesteder \textit{Arisbye} og \textit{Juxbye}, det veed han ikke\par
Sp. 25. Svar: Landskabet paa den Kongelig \textit{Norske} Side beskrives saaledes fra SøeKanten og i Søer:\par
1. Det Jndre Landskab Sønden for Porsanger:\par
\textit{Pag:} 281 før er meldt, at \textit{Porsangers} Hoved-Elv kommer fra \textit{Vorieduder}, og løber igiennem \textit{Laune}-Vand i Nord til Øster i \textit{Porsangers} Vestre Botten, og at paa den Vestre Side af \textit{Porsanger}-Elv er samme \textit{Vorieduder}-Field, som strekker sig i Søer hen til \textit{Avjevara}, og i Sydost hen imod \textit{Karasjok}; Østen for \textit{Porsanger}-Elv, og Sønden for \textit{Porsanger}-Botten er det Field\par
\textit{Gaisak}, som rekker til \textit{Porsanger}-Elv, paa 1/2 Miil nær, med sin Vestre Ende; Med den Nordre Side til deels ligger den 1 Miil fra \textit{Porsangers} Vestre Botten i Søer, men strekker med samme Nordre Side sig længere Nord hen til \textit{Porsangers} Østre Botten, og gaaer Østenom \textit{Porsangers} Østre Fiord-Bræde over 1 Miil lang hen imod \textit{Børs}-Elv, derfra stævner det hen i Nord til Osten imod \textit{Laxe}fiord-Botten paa 2 Mile nær; J Øster har det det Field \textit{Guølpak}, og derefter seer paa 2 Mile nær, \textit{Arisbye}; J Søer seer det field \textit{Vorieduder} 1 Miil fra sig afliggendes, saa det er fra Vester, ved \textit{Porsanger}-Elv i Øster til \textit{Guølpak} og imod \textit{Tana}-Elv 2 Dagers Reise, eller 5 Mile langt, og fra Nord, ved \textit{Børs}-Elven, i Søer hen imod \textit{Vorieduder} omtrent ligesaa bredt; Dette \textit{Gaise} er et almindeligt Navn paa adskillige sammenhængende Fielde, somme høye, deels runde, deels tindede, somme flade, moesede med Steene imellem oventil, men neden under imellem Fieldene med Græss, Lyng og Moese begroede, alle uden Skoug.\hypertarget{Schn1_86690}{}Schnitlers Protokoller V.\par
Jmellem \textit{Gaisak} og \textit{Porsanger}-Elv er en Dal med Bierk og noget Furre i;\par
Jmellem \textit{Gaisak} og \textit{Porsangers} Vestre Botten er slet Myr- og Steen-Land med lidet Bierk;\par
Jmellem \textit{Gaisak} og \textit{Børss}-Elv er det Field \textit{VarejieZiek} liggendes ved \textit{Porsangers} Østre Fiord-Bræde, efter Børss-Elv, i Øster, 1 Miil langt, indtil Børss-Elv kroger sig i Søer;\par
Dette \textit{VarejieZiek} er 1 Miil langt fra Vester i Øster, og halv saa bredt, fladt ovenpaa og bart, med noget Lyng paa, nedenunder ved Sidene er Bierk og noget Furre med Græss iblant.\par
Jmellem samme \textit{VarejieZiek} og \textit{Gaisak} er en Dal 1/8 Miil breed.\par
Jmellem \textit{Gaisak} og \textit{Laxefiords} Botten er slet Land med nogle Bakker i, Bierke- Græss- Moese- og Lyngroet;\par
Jmellem \textit{Gaisak} og \textit{Guølpak} er en Steen-Dal, 1/8 Miil breed;\par
Jmellem \textit{Gaisak} og \textit{Tana}-Elv imod \textit{Arisbye} og \textit{Juxbye} daler Fieldet ned med Bierk og Furre nær ved \textit{Tana}-Elv;\par
Jmellem \textit{Gaisak} og \textit{Vorieduder} er 1 Miils Bierkeskoug, og i denne Skoug det Vand \textit{Jijaure}, langagtigt 1/8 Miil, hvoraf Aaen \textit{Vall-jok} i Øster til Søer rinder 2 Mile lang i \textit{Tana}-Elv, 1 [?] Miil Sønden for \textit{Juxbye}.\par
2. \textit{Landskabet} Sønden for \textit{Laxefiord}:\par
Før er meldt, at fra \textit{Laxe}fiord-Botten i Søer til \textit{Gaisak} er 2 Mile, og derimellem Landet slet, med nogle Bakker, samt Lyng, Moese og Græss i; og at \textit{Laxe}fiord-Elv kommer 1 Miil Sønden-fra udaf et lidet langagtigt Vand \textit{Ziølle-jaure}, løbendes i \textit{Laxe}fiord-Botten.\par
1/8 Miil Østen for \textit{Gaisak} ligger nu det Field \textit{Guølpak}, fra Vester i Øster 1 DagsReise, eller 3 Mile langt, henimod \textit{Lill-Fossen} i \textit{Tana}-Elv, og omtrent (som slutes) ligesaa bredt; Dette \textit{Guølpak} er ovenpaa fladt, moeset og steenet.\par
3/8 Miil Norden for \textit{Laxefiords} Botten paa dens Østre Side indgaaer \textit{Adamsfiord}, deri gaaer \textit{Adams}-Fiord-Elv, kommendes fra \textit{Guølpak}-Field i Søer, og rinder igiennem et lidet Vand, mitt paa, 1 Dags Reise, eller 3 Mile lang i Nord.\par
Sønden for \textit{Vorieduder, Gaisak} og \textit{Guølpak} møde de Elve \textit{Karasjok}, og \textit{Tana}-Elv;\par
Ved \textit{Karasjok} Elv paa dens Søndre Side ligger det Svenske Markested, kaldet \textit{Karasjok};\par
4 Mile fra dette \textit{Karasjok} Markested i Nord-ost findes ved \textit{Tana}-Elv paa dens Søndre Side det forrige Svenske Markested \textit{Juxbye}: men fordj Folket her er deels uddøet, deels bortflyttet, saa er det nu øde.\par
2 Mile i Nord-Nordost fra \textit{Juxbye} ligger ved den Aae \textit{Otzjok} paa dens Vestre Side, 1/2 Miil, førend den falder i \textit{Tana}-Elv, det Svenske Marke- og Tingsted samt \textit{Annex}Kirke, \textit{Arisbye}, som af Aaen ogsaa kaldes \textit{Otzjok}, som fra \textit{Laxefiord}Botten ligger i Søer 6 Mile, og fra den Norske Tanafiords Botten i Syd-Syd-vest 7 Mile.\par
Landskabet Synden for \textit{Karasjok} og \textit{Tana}Elv, oven for \textit{Juxbye} og \textit{Arisbye} er Vidnet ikke videre bekiendt, end at det bestaaer af vilde Fielde, Skouge og Vande, hvor han ej har været.\hspace{1em}\par
Sp. 26. \textit{Cessat} ‒\hspace{1em}\par
Sp. 27. Svar. Vej, som Field\textit{finner} tage til og fra \textit{Arisbye} er: om Vinteren: og Sommeren:\hypertarget{Schn1_87043}{}Om Kiøllefiords Præstegield.\par
Fra \textit{Adamsfiord} i \textit{Laxefiord} efter \textit{Adamsfiord}-Elven i Syd-Syd-ost til det Field \textit{Guølpak}{3 Mile} over \textit{Guølpak}{2 ‒ } fra \textit{Guølpak} ned til \textit{Tana}-Elv{1/2 ‒ } Saa efter \textit{Tana}-Elv og den Aae \textit{Otzjok} til \textit{Arisbye}{1/2 ‒ _______6 Mile}\par
Fra \textit{Laxefiord} til \textit{Juxbye} tages samme Vej, som bliver imod 8 Mile.\hspace{1em}\par
28 Sp. \textit{cessat}; og have de ikke hørt deraf.\hspace{1em}\par
Sp. 29. og 30. \textit{Cessant.} ‒\hspace{1em}\par
Sp. 31. Svar: De nærmeste \textit{Finner} er ved \textit{Tana}fiordbotten, fra denne \textit{Kiøllefiord} 5 Mile.\hspace{1em}\par
Sp. 32. og 33. \textit{cessant.}\par
\centerline{\textbf{15de Vidne}\textit{Peder Olsen, Norsk Field-Finn},}\par
født i den \textit{Norske}\textit{Tana}-Fiord af \textit{Norske} Søe-\textit{Finner}, døbt i den Norske \textit{Omgangs} Kirke, et \textit{Annex} under \textit{Kiølle}-fiordz Præstegield, imod 70 Aar gammel, gift, har 7 Børn, været for 1 Maanedz tid til Gudz Bord, i det Norske \textit{Læbesbye FinneCapell}, er en Norsk FieldFinn, om Sommeren sidder med sine Reen ved Søe Siden imellem \textit{Porsanger}- og \textit{Laxe}fiordene, om Vinteren, naar det er kaldest, ved \textit{Tana}-Elv og \textit{Otzjok} i blant \textit{Arisbye}-fælledz\textit{Finner}, uden nogen tid derfor at have svart Skatt til den Svenske Øvrighed;\par
Til alle Spørsmaale fra \textit{N}. 18. til Enden svarer han det samme, som næst forrige 14 Vidne.\par
\centerline{\textbf{16de Vidne}\textit{Josep Andersen Norsk} Søe-Finn i \textit{Laxefiord}}\par
født i \textit{Arisbye} af fælles Field-\textit{Finner}, døbt i \textit{Arisbye}-Kirke, et \textit{Annex} under \textit{Kautokeino Lappe}- Præstegield, imod 70 Aar gl., gift, har 3 Børn; har været hos sine Forældre indtil han giftede sig her i \textit{Norge} i \textit{Laxe}fiorden med en Søe-Finns Datter, siden den Tid han stadig har været her i \textit{Norge}, som en Norsk Skatte-Finn; for 1 Maanedz Tid været til Gudz Bord i den Norske \textit{Læbesbye-Finne}Kirke;\par
Til alle Spørsmaale svarer han det samme, som næst forrige 2de Vidner ‒\par
Som da disse Vidner ovenfor \textit{Arisbye} i Søer ei vare bekiendt, og intet videre om den rette gamle Grændse-Gang imellem \textit{Norge} og \textit{Sverrig} vidste at forklare; Saa blev LaugRetten saavel som Vidnerne med Lensmanden befalede, at, naar de Kongel. Norske \textit{Jngenieurer} FieldLeedz efterkomme, de da skulle giøre dennem all den Bistand og Undsætning, som de kunde, til deres Reises og Forretnings Fuldbyrdelse.\par
Mens \textit{Missions} Skolemester[e]n i \textit{Laxefiord}-Fielde\textit{Jon Hendriksen} har jo før jo heller at forføye sig til \textit{Arisbye}-Field\textit{finner}, som skatte til begge Riger, og ere sande og fælles Under-\hypertarget{Schn1_87327}{}Schnitlers Protokoller V.\par
saattere saavel af \textit{Norges} som \textit{Sverrigs} Crone, at tilsige dennem, at nogle af de ældste og kyndigste u-fortøved skal indfinde sig hos mig ved den \textit{Norske}\textit{Tana}Elv, omtrent ved \textit{Bolma}, 2 Mile nær \textit{Arisbye} i Nord-ost, inden den 24de i indeværende Maaned, hvor jeg har at tale med dennem angaaendes deres Kongl. \textit{Majestets} høye Tieneste, og fører \textit{Missions}Skolemester[e]n dennem selv til mig didhen;\par
Hvortil \textit{Majoren} gav Skolemester[e]n sin skriftlig \textit{ordre}; Og Retten paa dette Sted blev slutted.\hspace{1em}\par
\textit{Kiøllefiord} d.13 \textit{Octob}. 1744. \hspace{1em}\centerline{Peter Schnitler.}\centerline{L. S. Pale Jonsøn i Brenngamme}\centerline{L. S. Christen evertsøn i Skidenberg}\hspace{1em}
\DivII[Okt. 14. Fra Kjøllefjord til Viken i Kifjord]{Okt. 14. Fra Kjøllefjord til Viken i Kifjord}\label{Schn1_87415}\par
d. 14de Reiset fra \textit{Kiøllefiord}; og da Søe-Vejen fra \textit{Kiølle}fiordz Botten gaaer i Nordost, siden uden om \textit{Omgangs} Syd-vestlige Side i Syd-vest, derefter i Øster, og omsider i Nordost i alt 2 Mile lang til \textit{Kiefiords} Botten, saa loed man \textit{Bagagen} føre deromkring til Søes, og selv gik man fra \textit{Kiølle}fiordz Botten over Land i Syd-Syd-ost til \textit{Kiefiords} Botten {1/4 Miil.} Man loed sig da sætte over \textit{Kiefiord}, og gik til Landz i Syd-ost til den Havn \textit{Vigen}{1/4 ‒ ________ = 1/2 Miil}\hspace{1em}\par
Jmidlertid \textit{Bagagen} foer til Søes fra \textit{Kiefiord} omkring didhen 3/8 Miil. ‒
\DivII[Okt. 15. Eksaminasjoner i Viken]{Okt. 15. Eksaminasjoner i Viken}\label{Schn1_87476}\par
d. 15 \textit{Octob}: loed man til sig hente 2de nu værende \textit{Norske} Søe-\textit{Finner} af den \textit{Norske Langfiord}, en Jndfiord af \textit{Tana}-Fiord, som tilforn have været \textit{Svenske} saa kaldede \textit{Arisbye- Finner}, og toeg dem under \textit{Examen}\par
\centerline{\textbf{16de Vidne}\textit{Peder Andersen}, nu \textit{Norsk} Søefinn i \textit{Langfiord}}\par
som forklarede, saa vidt hannem bekiendt var, Field\textit{tracten} fra \textit{Arisbye} ‒ den Svenske \textit{Annex} Kirke, i Søer, paa følgende Maade:\label{Schn1_87538} \par 
\begin{longtable}{P{0.667162471395881\textwidth}P{0.06613272311212814\textwidth}P{0.11670480549199085\textwidth}}
 \hline\endfoot\hline\endlastfoot \textit{Otzjok}-Elv, hvorved \textit{Arisbye} Kirke staaer paa dens Vestre Side, kommer opprindeligen i Søer af det Field \textit{Suøkas} dets Østre Ende, og løber\tabcellsep 1/2\tabcellsep Field (det er: kort) Miil\\
i Nord i det Vand \textit{Merisjaure}, derfra i Nord i \textit{Madjaure} ved\tabcellsep 5\tabcellsep Field Mile\\
hvilket Vand er et par Bøsseskud bredt, og i Nord\tabcellsep 1/4\tabcellsep ‒ Miil\\
langt, omsider herudaf i \textit{Tana}Elv\tabcellsep 1/4\tabcellsep Miil\\
\tabcellsep \multicolumn{2}{l}{_______________________}\\
i alt saaledes\tabcellsep 6\tabcellsep Field Mile lang\end{longtable} \par
 \hypertarget{Schn1_87610}{}Om Kiøllefiords Præstegield.\par
\textit{Suøkas}-Field er fra Syd-vest i Nord-ost 1/2 Miil langt, 1/8 Miil bredt, fladt ovenpaa, med noget Bierke-Riis paa, fladtvoren paa Sidene med Bierk nedentil.\par
\textit{Merisjaure} er fra Søer i Nord 1 knap Miil langt, 1/8 Miil bredt.\par
Ved \textit{Madjaures} Vestre Side mitt paa staaer \textit{Arisbye} Kirke.\par
Paa Østre Side ved Søndre Ende af forbem.te \textit{Merisjaure} staaer de Kongelige Svenske Field- Betienteres Gamme, eller Torv-Hytte, hvorhen til fra \textit{Arisbye} Markested er 1 Dags Køer, eller som han det regner, omtrent 5 Mile, hvor de skifte Skydsen.\par
Fra denne Gamme fare de Kongel: Svenske Betiente i Sydost over Myr- og GræssLand 1 Miil til det Vand \textit{Betzeko-jaure}, fra Søer i Nord 1/8 Miil langt, halv saa bredt.\par
Landskabet Norden for \textit{Arisbye} Kirke imellem Kirken og \textit{Otzjoks} Mund, eller Udløb i \textit{Tana}-Elv bestaaer af Furre- og Bierkeskoug.\par
Landskabet Sønden for \textit{Arisbye} Kirke imellem \textit{Otzjok} og \textit{Tana}-Elv er ligeledes Furre og Bierk, ved 4 Bøsseskud meere og mindre viid.\par
Sønden for Skougen er det Field \textit{Giesk-adam}, imellem \textit{Otzjok} og \textit{Tana}-Elv fra Vester i Øster ved 2 Mile bredt, og fra Nord i Søer 1 Dags Reise, eller ved 2 1/2 Mile langt, ovenpaa fladt, og saavel der, som paa Sidene Moesegroet, neden til ad Elvene havendes Bierk og Furre.\par
Paa \textit{Giesk-adam} følger i Søer en Græss-Dal, breed fra \textit{Tana}-Elv til forberørte \textit{Suøkas};\par
Sønden for Græss-Dalen er \textit{Bais-duøder}, hvilket han ei veed at beskrive, som han ei [har] været der, see \textit{p.} 318.\par
Paa Vestre Side af Tana-Elv, i Nord-vest fra \textit{Otzjoks} Elvs Mund, ligger det Field \textit{Guølpak}, beskreven \textit{pag.} 295 Vesten for \textit{Guølpak} det Field \textit{Gaisak}, beskreven \textit{pag.} 293 og i Sydvest fra \textit{Gaisak} det Field \textit{Vorieduøder};\par
Østen for \textit{Otzjok}-Elv, hvor \textit{Arisbye} Kirke er, ligger \textit{Adnevara}, et Field, som naaer med sin Nord-vestlige Side hen imod \textit{Tana}-Elv, og med den Syd-vestlige Side til \textit{Otzjok}, i Syd til Osten 1 sterk Miil langt, 1/2 Miil over bredt, slet ovenpaa, med Moese og Bierke-Riis begroet, paa Sidene ad Elvene med Furre og Bierk forsiunet.\par
Paa Østre Side af dette \textit{Adne-vara} rinder den Aae \textit{Vækiag}, udaf \textit{Væke-jaure}, 1 Miil Østen for \textit{Otzjok}, 2 Miile lang i Vester til Norden, og udfalder i \textit{Tana}-Elv, 1 Miil Norden for \textit{Otzjoks} Mund, og 2 Mile Sønden for \textit{Bolma}-Elvs Mund i \textit{Tana}-Elv;\par
\textit{Væke-jaure} er langt fra Søer i Nord 1 Miil, halv saa bredt.\par
Ved 4 Bøsseskud Sønden for \textit{Adne-vara} er Fieldet \textit{Skallo-vara}, 1/4 Miil efter \textit{Otzjok} langt, halv saa bredt, med en Tinde mitt paa, ellers fladt, og Reen-moeset.\par
Fra \textit{Bais-duøder}, hvorom paa næst forrige Side er rørt, skal der være til \textit{Mudkie-duøder} 1 Dags Reise i Søer: men hvor langt siden derfra i Søer til Field-\textit{Kiølen?} veed han ikke, som han intet kan sige noget om Grændse-Skiellet imellem Rigerne, eller videre i Søer er bekiendt.\par
\centerline{\textbf{17de Vidne}\textit{Hans Jonsen,}}\par
kommet fra \textit{Arisbye}, nu værende \textit{Norsk} Søe-Finn i samme \textit{Langfiord}, veed ei meere at forklare, end næst forrige, og intet om Grændse-Gangen at vidne; Hvorfore man ei fandt nødig, \hypertarget{Schn1_87897}{}Schnitlers Protokoller V. at tage disse 2de sidste i Vidne-Eed, men nøyede sig med, at optegne deres giordte Forklaring om det Stykke af fælles Field-\textit{tract}, som dennem bekiendt var, i Overværelse av de 2de \textit{Norske} Søe-\textit{Finner}, \centerline{\textit{Lars Jonsen Vigen, Peder Pedersen Bekkerfiord.}}\hspace{1em}
\DivII[Okt. 16.-17. Fra Viken til Leirpollen i Tana]{Okt. 16.-17. Fra Viken til Leirpollen i Tana}\label{Schn1_87919}\par
d. 16de Reiset fra \textit{Vigen} paa \textit{Omgangs}-Landet til Søes i Søer 1/2 Miil, siden igiennem \textit{Eidsfiord} i Øster til \textit{Hops-Eidet} 1/2, ‒ i alt {1 Miil}\par
Dette \textit{HopsEid} er den \textit{Jsthmus}, som sammenholder \textit{Omgangs peninsul} med det Søndre faste Land, og kunde være ved 1/16 Miil over bredt, hvorover Baaden og \textit{bagagen} maatte drages; Paa dette \textit{Hops}Eid voxer mangfoldig \textit{Cochleare} vildt.\par
Fra dette \textit{Hops}Eid i Øster foer man igiennem \textit{Hops}Fiord, til \textit{Hops-Vigen}, paa Søndre Side af \textit{Omgangs}Landet{1 ‒ _______ = 2 Mile}\par
d. 17de faret fra \textit{Hopsvigen} igiennem \textit{Hops}fiorden i Øster til Syden forbi den Jndfiord \textit{Langfiord}, paa høyere Haand, til \textit{Digermulen}, det Nordre Næss af Vestre \textit{Tana}fiord, havendes derimod paa Venstre Haand \textit{Kieskel-Klubben}, {3/4 Miil}\par
Fra \textit{Digermulen} saae man paa Venstre Haand i Nord-ost \textit{Tana-Hornet}, det Østre Næss af \textit{Tana}-Fiord;\par
Og foer saa videre igiennem og over \textit{Tana}-Fiorden, i Syd-Syd-ost til \textit{Østre- Tana}, ellers kalded \textit{Leerpollen}, igiennem \textit{Tana}-Elvs Mund, forbi dens Vestre Næss, \textit{Grønnæss} paa høyere og dens Østre Næss \textit{Stangnæss} paa venstre Haand, til den første \textit{Finne}-Bye paa Nordre Land-Side, \textit{Yttre-Vigen}{1 1/2 Miil _______ = 2 1/4 Mil}\par
Paa hvilken Vej et haardt taaget Vejer os paaKomm.\par
‒Samme \textit{dato} giort Bud til den Norske \textit{Missions} Skolemester, og \textit{Finne}-Lensmand {i}\textit{Tana}Fiord, at lade de kyndigste Søe- og Field\textit{finner} til mig nedkomme.\hspace{1em}\par
d. 19de Ankom \textit{Missions} Skolemesteren og \textit{Finne}Lensmanden med nogle \textit{Finner}: Men som man under Samtale forud hørte, at disse noget om Field\textit{tracten}, men lidet om Landz\textit{kiølen} vidste, besluttede man vel, ved ordentlig \textit{Examen} at indtage af disse den Kundskab, de havde: men dog siden at fare op til det yderste Grændse-Sted af \textit{privative Norsk} Grund ved \textit{Bolma}-Elv, og at see at faae did ned til sig de saakaldede Svenske \textit{Arisbye} Field\textit{finner};\hspace{1em}\par
d. 20de var \textit{Tana}-Elv til imod Munden Jis-frossen, at man ei kunde tage Vejen forbi \textit{Guldholmen} op efter \textit{Tana}-Elv i Baad til \textit{Bolma}; nær ved \textit{Arisbye}, det Svenske Markested; Thi maatte man fare ind til \textit{Leerpolls} Botten, fra \textit{Stangnæss} i Sydost {3/4 Miil,} for derfra siden at begive sig i \textit{Finnernes} Kieredster til Fieldz ad \textit{Bolma}-Elv.\hspace{1em}\par
d. 21. \textit{Octob.}\textit{expederet Missions} Skolemesteren her fra \textit{Leerpolls}Botten til \textit{Arisbye}, det Svenske Markested, med min skriftlig \textit{ordre}, at tilsige de derved værende Field-\textit{Finner}, som skatte til begge Croner, følgelig holdes for fælles og sande Undersaattere af begge Riger, at \hypertarget{Schn1_88227}{}Om Kiøllefiords Præstegield. nogle af de kyndigste og Troværdige Mænd jo før jo heller havde at indfinde sig hos mig i Naboelauget ved \textit{Bolma}-Elv, for at tale med mig; Saasom det er begge Kongel. \textit{Majesteters} Foreening og naadigste Villie, at Grændsen imellem \textit{Norge} og \textit{Sverrig} af u-villige og sandrue Mænd skulle udsiges.\par
‒ Givet samme \textit{Missions} Skolemester mit Pass til fri Skydz med fornøden Reen til og fra \textit{Arisbye} tilbage.
\DivII[Okt. 21.-31. Rettsmøte i Leirpollbotn]{Okt. 21.-31. Rettsmøte i Leirpollbotn}\label{Schn1_88256}\par
‒ Samme \textit{dato} begyndte at sætte \textit{Examinations} Retten i en \textit{Finne}Bye, eller Gamme i \textit{Leerpoll}-Botten med de ankomne Norske Søe- og Field-\textit{Finner}; Saasom \textit{Missions} Skolemesteren meente, at han ei før, end efter en 12 à 14 Dage kunde fra \textit{Arisbye} til \textit{Bolma} tilbagekomme;\par
Overværendes ved Retten var den \textit{Norske Finne} Lensmand af \textit{Tana}Fiord\textit{Morten Clementsen}, med Laug Rettes Mænd, \textit{Norske} Søe\textit{Finner}; Til at \textit{assistere} Tolken Erik \textit{Hælset}, blev \textit{adhibered} en \textit{Norsk} Field\textit{finn Hans Mortensen}, siden \textit{Dialecten} af det \textit{Nordlandske Lappe}- Maal havde forandret sig noget fra den \textit{Finmark}iske\par
Vidner, som om \textit{Norges privative} eeget Land fra Søer op til det saa kaldede \textit{Sverrigs}- eller rettere at sige \textit{Fælles Finmarken} kunde give Forklaring, vare\par
\centerline{\textbf{18de Vidne}\textit{Peder Nie{l}sen, Norsk} Field-\textit{Finn}}\par
født i de \textit{Norske}\textit{Tana}-Fielde, gift, har 5 Børn, 73 Aar gammel, sidder med sine Reen om Sommeren ved den Norske \textit{Leerpoll}-Elv, om Vinteren ved de \textit{Norske}\textit{Bolma}-Vande.\par
\centerline{\textbf{19de Vidne}\textit{Ole Olsen, Norsk} Field-\textit{Finn}}\par
født i \textit{Tana}-Fielde, 50 Aar gammel, gift, har 5. Børn, sidder med sine Reen, hvor næst- forrige\par
\centerline{\textbf{20de Vidne}\textit{Peder Andersen, Norsk} Field\textit{finn}}\par
født i \textit{Varangers} Fielde, 52 Aar gammel, gift, har 7 Børn, sidder om Sommeren i \textit{Tana}- fiordz Østre Fielde, fra \textit{Tanahornet} til \textit{Molvigen}, om Vinteren ved \textit{Bolma}-Vande\par
\centerline{\textbf{21de Vidne}\textit{Ole Pedersen, Norsk} Field\textit{finn},}\par
født i \textit{Varanger}Fiord, sidder sammesteds, som næst forrige; 44 Aar gl., gift, har 1 Barn.\hspace{1em}\par
\textit{Vidner}, som til deels om \textit{Norges} eeget \textit{privative}\textit{Finmarken}, til deels om det af \textit{Sverrig} brugende tilfælles \textit{Finmarken}, og noget lidet om Grændse-Gangen imellem \textit{Norge} og \textit{Sverrig} vidste, vare:\hypertarget{Schn1_88509}{}Schnitlers Protokoller V.\par
\centerline{\textbf{22de Vidne}\textit{Clement Nielsen} i \textit{Kefiord, Norsk Søefinn}}\par
født i \textit{Arisbye}, den Svenske Markested, døbt sammestedz, 51 Aar gammel, gift, har 4 Børn, tilforn været en \textit{Arisbye} Field-\textit{Finn}, men nu siden 3 Aar sidder stadig Søe\textit{Finn} i den Norske \textit{Kefiord} i \textit{Vestre Tana}.\par
\centerline{\textbf{23de}\textit{Mathies Mathiesen, Norsk Fieldfinn,}}\par
født i \textit{Jndiager}, eller paa \textit{Finnsk}\textit{Anar}, 36 Aar gammel, gift, har 1 Barn, tilforn været en \textit{Arisbye}-Field-\textit{Finn}, nu siden 3 Aar, en Norsk Field\textit{Finn}, siddendes Sønden for denne \textit{Leerpoll}, og Østen for \textit{Tana}-Elv, om Sommeren, men om Vinteren ved \textit{Bolma}.\par
\centerline{\textbf{24de}\textit{Erik Bonjækas, Norsk SøeFinn},}\par
født i \textit{Sverrig}, imod 50. Aar gammel, siden nogle Aar siddet ved den Norske \textit{Tana}-Elv, som SøeFinn, gift, har 3 Børn;\hspace{1em}\par
Alle disse benævnte Vidner have i sidst afvigte Sommer gaaet til Gudz Bord i den Norske \textit{Tana Finne-Capell}.\par
For dennem blev Eedens Forklaring af Lov-Bogen ved Tolken betydet, og de derpaa aflagde deres Corporlig Eed, at vidne, hvad de vidste, sandferdeligen:\par
De 4. første fra \textit{N.} 18. til \textit{N.} 21. forklarede det Norske \textit{territorium} fra SøeKanten op i Søer, saaledes; med den tilstæde værende Almue:\hspace{1em}\par
Til 1 Sp: Svar: Denne \textit{Finne}-Bye, eller Torv-Hytte i \textit{Leerpolls} Botten ligger i det Norske \textit{Tana-Finne-Capells} Sogn i \textit{Kiøllefiords} Præstegield, om hvis Leje, Strekning for \textit{pag.} 283 til 287.\hspace{1em}\par
til Sp. 2. 3. 4. i Almindelighed er svaret.\hspace{1em}\par
til Sp. 5te Svar: Beskrivelsen af \textit{Tana}-Fiord er \textit{pag.} 291 hertil forspared:\par
\centerline{\textbf{Tana Fiord}}\par
ligger Østen for \textit{Laxe}Fiord, og ligesom \textit{Omgangs} Halv-Øe ved det smale \textit{Hops}-Eid, 1/16 Miil over bredt, fra det Søndre faste Land \textit{distingveres}, saa skilles og \textit{Tana}Fiord i Øster ved samme \textit{Hops}Eid fra førstbemeldte \textit{Laxefiord}; Thi \textit{Eids}-Fiorden, som stevner i Øster til \textit{Hops}Eidet, er en Jndfiord af \textit{Laxe}fiorden, og \textit{Hopsfiord}, som gaaer vestlig ind til \textit{Hops}Eidet, er en Jndfiord af \textit{Tana}fiord; Ellers er fra \textit{Storfinnkirken}, \textit{Laxe}fiordz Østre Næss omkring \textit{Omgangs}Landet først i Nord, siden i Øster til \textit{Omgangs Odden}, det Vestlige Næss af \textit{Tana}-Fiorden gode 3 1/2 Mile.\hypertarget{Schn1_88783}{}Tanafiord. Kiøllefiords Præstegield.\par
Denne \textit{Tana}Fiordz Vestre Næss er bemeldte \textit{Omgangs Odde}, og dens Østre Næss \textit{TanaHornet}, dog ligger dette fra hint i Syd-ost, og er \textit{Tana-Fiords} Gabet derimellem 1 sterk SøeMiil vidt.\par
\textit{Tana-Fiord} er paa sin Vestlige Side fra \textit{Omgangs Odden} til \textit{Østre Botten} i \textit{Vestre-Tana} i Syd-Syd-vest lang {5 Mile} ‒ Paa sin Østre Side fra \textit{Tana-Hornet} ligeledes i Syd-Sydvest til \textit{Stangnæss}, som er Begyndelsen af \textit{Tana}-Elv i \textit{Øster-Tana}, lang {3 Mile}\par
Dette nu i Særdeleshed at forklare:\par
\textit{Pag:} 285 f. er i Sær viiset, hvorledes fra \textit{Omgangs-Odden} i Søer til Vesten, til \textit{Kieskelvigklubben}, det Nordre Næss af \textit{Hops}Fiord, er {2 Mile}\par
Fra \textit{Kieskelvigklubben} over \textit{Hops}fiord til \textit{Digermulen}, det Søndre Næss af \textit{Hops}- fiord i Søer {1/2 ‒}\par
\textit{Digermulen} er et høyt bart Field, rundfladt ovenpaa, 1 Miil langt i Søer, efter \textit{Tana}-Fiorden, og 1/2 Miil efter den Jndfiord, \textit{Lang-Fiord}, brat paa Sidene baade til \textit{Tana}- og \textit{Lang}fiorden, siger langt i Søer {1 ‒}\par
Sønden for \textit{Digermulen} er den Vestre Land-Side ligeledes fieldet, dog flad-nedsludtendes med tyn og lidet Bierke-Riis paa Fiord-Bræden, ubeboed, indtil Botten {1 1/2 ‒} J hvilken Botten først boe Søe-\textit{Finner}. {_______ giør de 5 Mile}\hspace{1em}\par
Den Østre Land-Side af \textit{TanaFiord} begynder fra dens Østre Næss, \textit{Tana-Hornet}, som er høyt, brat paa begge Sider baade til Havet og til Fiorden, bart, fladt ovenpaa, et par Bøsseskud vidt i Søer, og 4 Bøsseskud i Øster langt;\par
Efter \textit{Tana-Hornet} i Søer er den Østre LandSide af \textit{Tana}Fiord fielded, fladvoren neden til Fiorden, men steil opad, i Søer til Vest hen til \textit{Yttre Mollvig}, u-beboed, {1 Miil}\par
Denne \textit{Mollvig} er imod 1/8 Miil dyb i Øster, og i Gabet viid i Søer {1/4 Mil}\par
Land Siden omkring \textit{Mollvigen} er noget slet ved Søe-Bræden med Lyng paa, men oven til stiger det op til bare Blaa-Field, ubeboed.\par
Fra \textit{Mollvig} i Søer til Vesten til \textit{Yttre Troldfiord Næss}{1/2 ‒} Landet derimellem er blaafieldet, steilt, ubeboet.\par
\textit{Troldfiord} er bare en Bugt, 1/4 Miil dyb i Øster, og i Gabet viid {1/4 ‒} omkring-fielded, u-beboed; det Søndre Næss heraf heder \textit{Jndre-Troldfiord-Næss}.\par
Fra denne \textit{Troldfiord} i Søer til \textit{Stangnæss} i \textit{Østre-Tana}, det Østre Næss af \textit{Tana} Elv, {god 1 ‒} Landet derimellem er brat-fieldet, bart, ubeboet. {_______ giør de 3 Mile}\par
\textit{Jnd}- eller \textit{Tvær}-Fiorder af \textit{Tana}-Fiord paa Vestre Side:\par
Paa \textit{Omgangs} Land ere beskrevne før \textit{pag.} 285 f.\hypertarget{Schn1_89058}{}Schnitlers Protokoller V.\par
\textit{Hops}fiord har til Nordre Næss, som før meldt, \textit{Kieskelklubben}, til Søndre Næss \textit{Digermulen}, derimellem er \textit{Hopsfiord} i Gabet 1/2 Miil viid, men ind ad bliver trangere; Fra \textit{Kieskelklubben} er den lang i Vester til Norden, til \textit{Hops}-Eidet 1 1/4 Miil, men fra \textit{Digermulen} 1 1/2 Miil;\par
Paa Nordre Side fra \textit{Kieskelklubben} 1/4 Miil indgaaer \textit{Hops-Vaagen}, ved 6 Bøsseskud viid, 1/8 Miil dyb i Nord; i \textit{Omgangs} Landet.\par
1/4 Miil Vesten for \textit{HopsVaagen} stikker ind i \textit{Omgangs} Landet i Nord \textit{Jver-fiord}, 1/4 Miil i Gabet viid, 1/16 Miil dyb.\par
Paa Søndre Side har \textit{Hopsfiord} 3de Tvær-Fiorder:\par
(1) \textit{Langfiord}, dens Østre Næss er det foromtalte \textit{Digermulen}, dens Vestre Næss heder \textit{Langfiord-Næss}, derimellem \textit{Langfiord} i Munden 1/4 Miil er breed, og i Syd-vest 1 1/2 Miil lang.\par
Den Østlige LandSide ad Søer af \textit{Langfiord} er blaa-fielded og brat, u-beboed, alt til Botten; Den Nord-Vestlige LandSide er vel fielded, men nedhældendes med Bierk paa FiordBræden; 1/8 Miil fra \textit{Langfiord-Næss} paa \textit{Langfiordens} Nordvestlige Side boe 2. Søe-\textit{Finner}; J Botten af \textit{Langfiord} er en Slette med Bierk paa, 1/4 Miil op fra Søen, siden stiger den op til Fielde, og her boe 4. Søe-\textit{Finner}.\par
Fra \textit{Langfiord-Næss} i Vester 1/4 Miil er \textit{Skougvigen}, og derimellem ere høye bratte og bare Fielde, u-beboede.\par
\textit{Skougvigen} er i Kæften 1/8 Miil viid, 1/8 Miil dyb; Landet omkring denne Viig er bratfieldet, u-beboet. Fra \textit{Skougvigen} til \textit{Skoug Fiord} i Vester er 3/8 Miil; Landet derimellem fieldet, steilt, u-beboet.\par
[(2)] \textit{Skougfiord} er i Gabet viid 1/8 Miil, og 1/4 Miil dyb i SydVest. Fiord-Brædene af denne \textit{Skougfiord} ere høyfieldede, bar-blaae og bratte; Jnde i Botten er 1/4 Miils Slette med Bierk og noget Græss, u-beboet. Fra \textit{Skougfiord} i Vester til Nord indtil \textit{Hops}Eidet er 3/8 Miil; Landet derimellem brat- og bar-fieldet, u-beboet.\par
Denne Jndfiord [(3)] \textit{Hopsfiord} er den fisk-rigeste af alle i \textit{Tana} HovedFiord; og den sidste paa Vestre Land Side.\par
Jndfiorder af \textit{Tana}Fiord, som ere paa Østre Side, ere før forklarede \textit{pag.} 301.\par
Botten af \textit{Tana}-Fiord viider sig ud til 1 1/2 Miil, og deeles i \textit{Vestre-Tana}, og \textit{Østre-Tana}; Til \textit{Vestre Tana} hører den større Deel af Botten til, som ligger Vesten for Tana-Elv, nemlig (1) \textit{Vester-Botten} (2) \textit{Øster-Botten}, de Jndfiorder (3) \textit{Vætfiord}, og (4) \textit{Ke-Fiord}; Til \textit{Østre Tana} regnes den mindre Deel Østen for \textit{Tana}-Elv, nemlig \textit{Leerpollen}.\par
(1) \textit{Vester-Botten} dens Nordre Næss kaldes paa \textit{Finnsk Saga-niarg}, dens Søndre Næss \textit{Alderbagte}; og er \textit{Vester-Botten} i Kæften 1/4 Miil viid, og 1/8 Miil i Vester dyb, med Bierk runden om begroed, beboes af 2 \textit{Søefinner}.\par
(2) \textit{Østerbotten} strax derved har til Vestre Næss \textit{Kiæsse-Niarg}, til Østre Næss \textit{Sommerset}, knap 1/8 Miil i Kæften breed, og stikker i Søer 1/2 Miil dyb, med Bierk runden om bevoxen, krogendes sig med sin Botten ad Øster, beboed af 2 Søe\textit{finner}. Knap 1/4 Miil herfra i Øster forekommer\par
(3) \textit{Vætfiord}, som har \textit{Sagga-Niarg} til Vestre og \textit{Nem-Niarg} til Østre Næss, i Kæften 1/8 Miil breed, 3/8 Miil i Søer dyb, med Bierk paa i Botten, u-beboed; Thi der er grundt ind til Botten. 1/2 Miil Østen for \textit{Vætfiord} er\hypertarget{Schn1_89358}{}Tanafiord. Kiøllefiords Præstegield.\par
(4) \textit{Kefiord}, derimellem er Landet bratfieldet, bart, u-beboet; \textit{Kefiords} Vestre Næss heder \textit{Kenæss}, og dens Østre Næss \textit{Kap-Næss}, derimellem er \textit{Kefiord} i Gabet ad Øster 1/8 Miil viid, og gaaer 3/4 Miil dyb ind, halv vejs i Søer til Vesten, og halv vejs i Søer til Osten; Paa begge Fiord-Bræder er det berget og steilt, ubeboet; men i Botten er Slette med mangfoldig Bierkeskoug, som strekker sig hen til \textit{Tana}-Elv; Et par Bøsseskud fra Botten paa Fiordens Vestre Side boer 1 Søe\textit{finn}.\par
Fra Kefiord i Øster 3/8 Miil er \textit{Tana}-Elv, og Landet derimellem brat, berget og u-beboet.\par
\textit{Tana}-Elvs Vestre Næss er \textit{Grønnæss}, et lavt rundvoren Field, et par Bøsse-skud over stort, paa Nordre Side ad Fiorden brat og bart, paa Østre Side ad \textit{Tana}-Elv græss-groet, havendes Sønden for sig Bierk nok langs efter \textit{Tana}-Elvs Vestre Elv-Bræde.\par
J Nordost fra \textit{Grønnæss} knap 1/4 Miil ligger \textit{Tana}-Elvs Østre Næss, \textit{Stangnæss}, lavt, bar-steenet, smalt udad, 1 Bøsseskud over stort;\par
Dette \textit{Stangnæss} er og det Nordre Næss af den Jndfiord \textit{Leerpollen}, hvis Søndre Næss er \textit{Leerpoll-Næss}, en Sandbanke, oven for hvilken i Søer mangfoldig Bierk er langs efter \textit{Tanas} Østre Elv-Bræde.\par
Denne \textit{Leerpoll}, som ligger paa Østre Side af \textit{Tana}-Elv, stævner først ind i Øster, siden i Sydost 3/4 Miil dyb; den halve Nordre Fiord-Bræde er noget slet neden til, men oventil stiger op i Vejret til Fielde, beboed af 1 Søe-Finn; Den anden halve Østre Bræde, saa og den heele Vestre Land-Side er høy-fielded og brat; Botten deraf har en angenem Slette med Græss og god Bierkeskoug, og 2de Søe-Finn-Byer, eller Torv-Hytter.\par
Denne \textit{Leerpoll} Østen for \textit{Tana}-Elv er det, som man kalder \textit{Østre-Tana}, og fryser om Vinteren til, undertiden ud til Kæften.\par
Elve og Aaer i disse Fiorder paa Vestre \textit{Tanafiords} Side:\par
\textit{Tana}-Elv tales siden om \textit{pag:} 304.\par
J \textit{Skougsvigen} og \textit{Skougfiord} er det Bække, som indløbe. J \textit{Langfiord} Botten gaaer \textit{Langfiord} Elv Sønden-fra 1 Miil lang af et Field, som ligger 1/4 Miil Norden for \textit{Langfiords}Botten, hvorfra den rinder i Nord til Vesten;\par
J \textit{Vestre Tanas} Vestre Botten løber \textit{Laud-jok} fra Syd-vest ud af det Vand \textit{Laud-jaure} 1/4 Miil lang i \textit{N.}ost;\par
\textit{Laud-jaure} er rundt, et par Bøsseskud over vidt, havendes Aurer-Fisk.\par
J \textit{Øster-Botten, Vætfiord} og \textit{Kefiord} er bare Bække, rindende af VatsKiønner i bem.te Jndfiorders Bottene.\par
Aaer paa \textit{Tanafiords} Østre Land-Side, Nordenfra:\par
Fra \textit{Tana-Hornet} 1 Miil i Søer er \textit{Yttre Mollvig}, derj \textit{Mollvig}-Aae rinder Østen-fra 1/4 Miil lang.\par
Fra bem.te \textit{Mollvig} imod 3/4 Miil i Søer er \textit{Trollfiord}, derj \textit{Trollfiord}-Elv af Field løber først 1/2 Miil i Nord, siden 1/2 Miil i Vester. Herfra 3/4 Miil i Søer ligger \textit{Jndre-Mollvig}, hvorj \textit{Mollvig}-Aae fra Øster i Vester 1/2 Mill lang rinder.\par
Derfra 1/4 Miil i Søer er \textit{Leerpoll}, i hvis Botten løber \textit{Jolle-jok}, eller paa Norsk \textit{Leerpoll-} Elv af det Vand \textit{Gorre-jaure} først i Nord-vest, siden i Nord 1 1/2 Miil lang; \textit{Gorre-jaure} er rundt, 1 Bøsseskud over stort; J \textit{Jolle-jok} fanges Øreter, i \textit{Gorre} Ourer.\hypertarget{Schn1_89651}{}Schnitlers Protokoller V.\par
Til Sp. 6. Svares om \textit{Tana} Elv:\par
Forud meldes (1) at man bruger den \textit{Scala} af en \textit{Nordlansk} Miil, som tilforn er vedtagen, ved Søe-Kanten, imod hvilken \textit{Nordlansk} Miil en Field-Miil reignes for 3/4 deel af dens Længde, og efter denne korte \textit{Scala} forstaaes de Mile, som \textit{Tana}-Elv reignes efter, i Anleedning af Field- \textit{Finnernes} Vidne, som ved deres Miil-Tall forstaae korte Field-Mile:\par
(2) \textit{Tana}-Elv beskrives efte[r] det 23 og 24de Vidners, nemlig \textit{Erik Bonjækas}, og \textit{Mathies Mathiesen} deres Beretning:\par
\textit{Tana}-Elv deeles i Toe, \textit{Øvre}- og \textit{NedreTana};\par
\textit{Øvre-Tana} kaldes og \textit{Jndiager}- eller \textit{Anar}-Elv, (fordi \textit{Jndiager-Finner} om Sommeren derved sidde, og fiske Lax.) og reignes for den Deel af Elven fra dens Udspring paa Grændse- \textit{Kiølen}, som her kaldes \textit{Jauris-duøder}, i Søer, til det Sted i Nord, hvor den Elv \textit{Karasjok} paa Vestre Side indfalder;\par
\textit{Nedre Tana}-Elv holdes for den Nordre Deel fra \textit{Karasjoks} Mund, i Nord til \textit{Tana}-Elvs Udløb i \textit{Tana}-Fiorden\par
\textit{Tana} Elv har sit Løb i Almindelighed fra Grændsen ud til Søen i Nord-Nord-ost som i Særdeleshed saaledes er forklaret:\par
\textit{Tana}-Elv udrinder af \textit{Gaune-jaure}, et Vand i Grændse-\textit{Kiølen}, eller \textit{Jaurisduøder}, paa dens Søndre Deel, og derfra 1 Miil vejs bærer det Navn af \textit{Keine-jok};\par
Fra samme \textit{Gaune-jaure} løber den omtrent i Nord-ost først, siden i Nord til \textit{Jndiagers} Sommersæde mod 3 Dagers Rejse lang, som man vill holde for korte {9 Field Mile} Fra \textit{Jndiagers} SommerSæde, i Nord omtrent, til \textit{Karasjoks} Mund ‒ ved {3 ‒ _______}\par
\textit{Øvre Tana}Elvs Længde fra S. i N. {12 ‒}\textit{Conferatur}\textit{pag.} 237 f. her ‒\par
Neden- eller Nordenfor \textit{Karasjoks}-Mund kaldes Elven \textit{Nedre Tana} Elv;\par
\textit{Nedre Tana}Elv fra \textit{Karasjoks} Mund, gaaer i Nord-Nordost til \textit{Valjoks} Mund, hvor denn fra \textit{Ji-jaure} Vand Vestenfra løber ind i \textit{Tana}Elv, lang {2 ‒} see \textit{pag.} 294. Fra \textit{Valljoks} Mund fremdeeles i Nord-NordOst, til \textit{Juxbye}, et \textit{Svensk} Markested tilforn, men nu øde, {1 ‒} Fra \textit{Juxbye} fremgaaer \textit{Tana}Elv i Nord Nord Ost til den \textit{Lillefoss}, som er den 2den fra \textit{Tanafiords} Botten, dog kroged{1 ‒} Fra \textit{Lillfossen} i Nordnordost, dog kromagtig til den Elv \textit{Otzjoks} Mund, ved hvilken \textit{Otzjok} den Svenske \textit{Arisbye}-Kirke staaer, {2 ‒} Fra \textit{Otzjoks} Mund i NordNordost til \textit{Storfossen}, som er den 1te fra \textit{Tana}- Fiordz Botten {2 ‒} Fra \textit{Storfossen} i NordNordost til \textit{Bolma} Elvs Mund, hvor den yderste \textit{Norske} Søefinn, som er den 3die fra \textit{Tanafiord}-Botten, ved \textit{Tana} Elv boer {1 ‒} Fra \textit{Bolma} i NordNordost til den Slette \textit{Moketveie}, hvor den 2den Norske Søe-\textit{Finn} fra \textit{Tana}fiordz Botten op efter Elven har bygget {1 1/4 ‒} Fra \textit{Moketveie} i NordNordVest til en Dal \textit{Sormaukie} den Øvere, eller Søndre, sterk {1/2 ‒}\hypertarget{Schn1_90004}{}Tanafiord. Kiøllefiords Præstegield. Fra \textit{Sormaukie} i Nord til den Søndre Ende af \textit{Storholmen} knap {1/2 Miil} Fra \textit{Storholmens} Søndre Ende i Nord til Osten til Slette \textit{Bonjækas}, hvor den 1te Norske \textit{Finn} fra \textit{Tana}fiordz Botten op ved Elven sidder {1/2 ‒} Fra \textit{Bonjækas} i Nordnordost til det \textit{Norske Tana Capell} ved \textit{Tana} Elv{1/2 ‒} Fra \textit{Capellet} i Nord til \textit{Grønnæss}, det Vestre Næss af \textit{Tana}-Elv{1/4 ‒} Fra \textit{Grønnæss} er i Nordost til \textit{Stangnæss}, det Ostlige Næss af \textit{Tana}-Elv{1/4 ‒ _______} Jmellem hvilke Næss \textit{Tana}-Elv sig i \textit{Tanafiord} udgiver.\par
Længden af \textit{Nedre}\textit{Tana}Elv{12 3/4 Miil _______}\par
Længden af hele \textit{Tana}-Elv, at forstaae korte Field-Mile, som Field-\textit{finnerne} regner paa {24 3/4 Miil}\par
Af Hvilke FieldMile kan holdes for at være Een for 3/4 \textit{Nordlandsk} Miil, og 1/2 \textit{Finmark}isk Søe-Miil.\par
Efter denne Beretning indehaver den Svenske Crone af \textit{OstFinmarken} fra Grændse- \textit{kiølen} til forbi \textit{Arisbye}-Kirke imod \textit{Storfossen}, hen imod {20 Field Mile} med den Geistlige og Verdzlige \textit{Jurisdiction}, og vi \textit{participere} med de Svenske kun for en ringe Deel i Skatten for samme Landstrekning:\par
Derimod \textit{Norge} af samme \textit{Ost-Finmarken privative} alleene igien have fra \textit{Tana}Fiordz Botten op efter Elven i Søer til lidt over \textit{StorFossen}, noget over {4 3/4 Field Mile} Breed er \textit{Tana}-Elv i Munden imellem \textit{Stagnæss} og \textit{Grønnæss} en knap {1/4 Miil} 1/2 Miil inden for \textit{Stangnæss} fra Vestre FiordBræde, hvor \textit{Tana Capell} staaer, og \textit{Guldholmen} i Elven ligger, til den Østre LandSide ved 4 Bøsseskud Der hvor \textit{StorHolmen} ligger, er den breed Jmellem \textit{Sormaukie} og det Vestre Land, Ligesom Flommen om Vaaren gaaer, eller har afgaaet, at forstaae fra \textit{Holmen} til Landet 2 à 1 Bøsseskud Jmellem \textit{Bolma} og det Vestre Land 2 à 1 dito J \textit{Storfossen} er den kun over breed 6 Trin, Ved \textit{Otzjoks} Mund er den i Vaarflommen 2 Bøsseskud breed, ellers saa grund, at man overvader den. Ved \textit{Juxbye} er den breed ‒1 Bøsseskud Siden jo længere Søer, jo smalere den bliver ‒\par
Til Efterretning meldes og, at de 3de Nybyggere ved \textit{Tana}Elv, hvorom \textit{pag.} 304 og her er rørt, ikke ere \textit{Normænd}, thi disse boe kun ud paa Øerne; Ei heller \textit{Norske} Søe\textit{finner} af Afkom, thi disse boe inde i Fiordene; Ei heller \textit{Norske} Field\textit{Finner}, thi de sidde og vanke omkring gemeenligen paa Fieldene med deres Reen; Men det er Qvæner, som her i Landet kaldes de \textit{Svenske} Bønder, hvilke fra \textit{Storfinland} eller \textit{Torne-Land} hid over komme; Og af disse Folk er en temmelig Deel baade hid til \textit{Finmarken}, og Nord i \textit{Nordland} overfløtted i forrige og nu sidste \textit{Svensk-Russiske} Krige; hvilke Qvæner er vel bevandte med at fiske Lax i Elvene, og duelige Jord-Dyrkere.\par
Denne \textit{Tana}-Elv er den Laxriigeste af alle i \textit{Finmarken}, saa der aarlig virkes og tilforhandles fra 100. til 300 Tønder Lax; Alle \textit{Norske} Søe- og Field-\textit{Finner} af det Norske \textit{TanaCapells} Sogn, og ingen \textit{Normænd} andenstedz fra, ere berættigede til at fiske i denne \textit{Tana}Elv op til \textit{Storfossen}; De \textit{Arisbye}-fælles Field-\textit{Finner} fiske gemeenligen Sønden- eller Oven for denne \textit{Storfoss}, dog skal nogle faa deraf bruge Elven Norden derfor 1 Miil ned til \textit{Bolma-}Elv;\hypertarget{Schn1_90339}{}Schnitlers Protokoller V.\par
Oven for dennem ved \textit{Karasjoks} Mund har et Støkke af \textit{Tana}Elv, af det Kongelig Svenske Rente-Cammer til \textit{Torne}-Borgere været forpagted, med Perle-fangsten i \textit{Karasjok}; 3 Mile Sønden for \textit{Karasjoks} Mund have \textit{Jndiager-Finner} deres Sommer-Sæde ved og Laxe-fangst i \textit{Tana}- Elv, alt op efter.\par
Her neere bygge \textit{Finnerne} med Bierk paa Sidene i Elven, og sætte Garn derimellem, hvorj Laxen i \textit{Maj} og \textit{Junio} fanges; Om St \textit{Hans} Tid kommer den \textit{Norske Octroijerede} Handelsmand fra \textit{Kiøllefiords} Handelsted til \textit{Guldholmen} yderst i \textit{Tana}-Elv, og tilforhandler sig all Laxen saavel af de Norske som \textit{Arisbye-Finner} hvilke \textit{Arisbye Finner} trænge til at faae de Norske Varer, Salt, Meel etc. for deres Lax.\par
\centerline{\textit{Holmer} i \textit{Tana}-Elv:}\par
\textit{GuldHolmen} ligger nær Sønden for Elv-Munden, 1/2 Miil oven for \textit{Stangnæss}, det Nordostlige Næss, (Jndtil dette Næss Handel-Skibet, som indtager Laxen, kun kan komme) 1/4 Miil oven for \textit{Grønnæss}, det Vestlige Elvens Næss, 1/2 Bøsseskud fra Vestre Elv-Bræde, der hvor det Norske \textit{Tana-Finne-Capell} staaer, og ved 4 Bøsseskud fra den Østre Elv-Bræde; \textit{GuldHolmen} i sig selv er 1 Bøsse-Skud langt i Søer, 1/3 deel saa breed, slet-steened, uden Skoug og Græss, u-beboed: dog staae Kiøbmandens Handelshusse og Pakboer derpaa.\par
\textit{StorHolmen}, i \textit{Tana}-Elv fra \textit{GuldHolmen} i Søer 3/4 Miil liggendes, 1 Bøsse-Skud nær den Vestre ElvBræde, og 2 Bøsseskud fra den Østre Elv-Bræde; er knap 1/4 Miil i Søer lang, og halv saa breed, slet med Bierk og Græss paa, u-beboed.\par
Bække, Aaer og Elve, som rinde i \textit{Tana}-Elv.\par
Bække ere mange, hvoraf en Deel ved Dalernes Beskrivelse siden \textit{pag.} 309 berøres.\par
Aaer paa Vestre Side indgaaendes\par
1. \textit{Valljok}, som meenes, ved 1 Miil, Sønden for det forrige \textit{Svenske} Markested \textit{Juxbye}, kommendes fra det Vand \textit{Ji-jaure} Vesten fra; hvorom meere er vidnet \textit{pag.} 294.\par
2. \textit{Karasjok} løber fra Syd-Syd-vest i \textit{Tana}-Elv, omtrent 2 Mile Sønden for \textit{Valljoks} Mund, og oprinder i Grændsen, omtrent 3/4 Miil Vesten for \textit{Tana}-Elvs Opkomst; Det Herom er meere forklaret \textit{pag.} 237. Det 23de Vidne veed ej det Sted, hvor \textit{Karasjok} opkommer.\par
Aaer paa Østre Side indgaaendes\par
3 3/4 Miil Sønden fra \textit{Tana}Elvs Mund, at forstaae fra \textit{Stangnæss}, løber i \textit{Tana} Elv fra Syd-ost\par
3. \textit{Bolma}-Aae, oprindendes udaf det \textit{Øvre Bolma-jaure}, hvorfra den 1 Miil lang i Nordvest gaaer i det \textit{Nedre Bolma-jaure}, og siden her udaf 1/2 Miil i \textit{Tana}-Elv;\par
\textit{Øvre Bolma-jaure} er 1/8 Miil fra Søer i Nord langt, halv saa bredt;\par
\textit{Nedre Bolmajaure} er fra Søer i Nord 3/4 Miil langt 1/3 deel saa bredt, fra \textit{Tana} Elv 1/2 Miil i Øster liggendes; J disse Vande fanges Ourer, Røe, Siig ‒og ere de merkelige deraf, at der er det yderste og Søndreste Vinter-Sæde for \textit{Laxe-} og \textit{Tana} Fiorders\textit{privative Norske} Field-Finner.\par
2 Mile Sønden for \textit{Bolma}Elvs Mund indkommer\par
4. \textit{Vækiæg-jok} fra Syd-ost af \textit{Vækejaure}, og rinder 2 Mile lang i Nordvest i \textit{Tana}-Elv;\hypertarget{Schn1_90626}{}Tanafiord. Kiøllefiords Præstegield.\par
\textit{Væke-jaure} er fra Søer i Nord 1 Miil langt, halv saa bredt; Denne Aae og Vandet ere deraf at merke, at \textit{Arisbyes} fælles Field\textit{finner} her have deres Vinter-Sæde. 1 Miil Sønden for Vækiæg‒Aaes Mund nedkommer\par
5. \textit{Otzjok}, mest Sønden-fra, efter 22de Vidnes Udsagn, af en Vatskiøn, tæt norden for \textit{Saxe-jaure}, hvorfra den rinder imod 1 Miil lang i Nord i det Vand \textit{Merisjaure}, siden heraf i Nord 2 1/2 Miil i \textit{Mad-jaure}, og omsider 1/4 Miil i Nord-vest i \textit{Tana}-Elv.\par
Disse Vande, og at Arisbye ‒ den Svenske \textit{Finne}-Kirke staaer mitt paa den Vestre Side af \textit{Mad-jaure}, er før \textit{pag.} 297 forklaret.\par
J denne \textit{Otzjok} paa dens Vestre Side, 1/4 Miil Sønden for \textit{Madjaure} indgaaer fra Syd-vest den Aae\par
6. \textit{Kiævo-jok}, opkommendes af smaa Kiønner paa Vestre Side af \textit{Niaug-oive}-Field, og løber 2de Dagers Reise lang, igiennem \textit{Lill}- og \textit{Stor-Kiævo}-Vande, i bem.te \textit{Otzjok};\par
\textit{Lill-Kiævo-jaure} er 2 Bøsse-Skud langt i Nord, halv saa bredt, og ligger i Sønden 1/4 Miil fra \textit{Stor-Kiævojaure}, som er 1/4 Miil langt, 1 Bøsseskud bredt, og ligger Sønden for \textit{Otzjok}, der hvor \textit{Kiævo-jok} indgaaer, 1/4 Miil. Om \textit{Garegass} Aae\textit{p:} 318.\par
7. \textit{Skiekkem}-Aae, som med det Navn \textit{Maselg-jok} oprinder i Grændsen, er før vidnet om \textit{pag.} 235. 238.\par
8. \textit{Raudo-jok}, fra Grændsen rindendes i \textit{Tana}-Elv er bevidned \textit{pag.} 234.\par
\centerline{\textit{Fosser} i \textit{Tana}-Elv:}\par
1. \textit{Storfossen} fra Elv-Munden i Søer 4 3/4 Miil, dannes deraf, at \textit{Tana}-Elv har der et lidet Vand-fald, og af begge Siders Fielde sammentrænges, at den der er kun 6 Trin over breed; fra Søer i Nord 1 Steen-Kast lang; J Fossen ligge store rundagtige Steene, hvorimellem Laxen vel gaaer op og ned, men med Baad kan ei derover roes, som over Land der forbi maa drages. 4 Mile Sønden for \textit{Storfossen} er\par
2. \textit{Lill-Fossen}, værendes 2 Mile Sønden for \textit{Otzjoks}-Mund, og 1 Miil Norden for \textit{Juxbye, formeres} ligeledes af Fielde paa begge Sider, hvilke giøre Elven der trang til 1/2 Bøsse-Skudz Bredde; Denne Lill-Foss er ved 2de Bøsse-Skud lang fra Søer i Nord, opgaaes af Laxen, og overroes med en liden Elv-Baad.\par
\centerline{Elv-Brædene af \textit{Tana}-Elv.}\par
Den Vestre Elv-Bræde fra \textit{Grønnæss} til \textit{Tana-Capell}, som for et par Aar er didhen bygget, 1/4 Miil vejs, er med Bierke-Skoug, et par Bøsse-Skud breed, forsiuned, dog u-beboed.\par
Den Østre Land-Side derimod fra \textit{Sandbanken} 1/4 Miil vejs i Søer, imod \textit{Tana-Capellet}, har stor Bierke-Skoug, et par BøsseSkud breed, siden stiger den op til Fielde, u-beboed.\par
Den Vestre Elv-Bræde fra \textit{Tana Capell} i Syd-Syd-vest til et Enge-Slette \textit{Bonjækas}{1/2 Miil} er slet, 1/8 Miil og mindre breed, med stor Bierk paa; her boer Jngen, førend paa det Enge Slette \textit{Bonjækas}, hvor en fra \textit{Sverrig} overkommen \textit{Qvæn} nylig har nedsatt sig, som er den 1te fra Elv-Munden.\hypertarget{Schn1_90886}{}Schnitlers Protokoller V.\par
Den Østre Elv-Bræde herimod er smalere, med Bierkeskoug paa, 1 Bøsseskud breed, og sommestedz skiæres den af ved steile Fielde ned ad Elven, u-beboed.\par
Den Vestre Elv-Bræde fra \textit{Bonjækas} i Søer {1/2 Miil} til \textit{StorHolmens} Søndre Ende er slet, 1/8 à 1/4 Miils breed, med stor Bierkeskoug bevoxen, siden den til nakene Fielde opstiger, u-beboed.\par
Den Østre Elv-Bræde herimod er kun halv saa breed, og Bierke-groed.\par
Den Vestre Elv-Bræde fra \textit{Storholmens} Søndre Ende til imod den Dal \textit{Sormaukie} paa Østre Side {1/2 Miil} er ligesom næst forrige halv-Miil paa Vestre Land-Side skikked.\par
Den Østre Elv-Bræde lige derimod til \textit{Sorm-aukie} er, som næstforrige Støkke paa Østre LandSide. Den Østre Elv-Bræde fra \textit{Sormaukie} i Syd-Syd-vest til \textit{Moketveje}{1/2 Miil} er slet med Bierk paa, men ikke saa breed;\par
\textit{Moketveje} er en Slette, 1/8 Miil viid, med Græss og Bierk, hvor den 2den Nybyggere nys har satt sig; og herfra er Postvejen til Landz over \textit{Saddevare} til \textit{Varangens} Vestre Fiord-Botten {1 Miil} Den Vestre Elv-Bræde herimod er slet, 1/8 Miil breed, med Bierk bevoxen, u-beboed.\par
Den Østre Elv-Bræde fra \textit{Moketveje} i Syd-Syd-vest til \textit{Bolma}{1 1/4 Miil} er deels fielded, deels slet, 1/16 Miil meer, og mindre breed, u-beboed: dog har den 3die \textit{Qvæn} for faa Aar siden nybygget ved \textit{Bolma}-Aae, paa en Slette, 1/8 Miil viid, imellem 2 høye bratte Fielde, som er den yderste og Sønderste \textit{Norske} Boemand op ved \textit{Tana}- Elv; nærmest det Svenske \textit{Arisbye} Marksted. Den Vestre Elv-Bræde herimod er som den Østere, og u-beboed.\par
Fra \textit{Bolma} i Syd-Syd-vest til \textit{Storfossen} er {1 Miil}\par
Fra \textit{Storfossen} i Syd-Sydvest til \textit{Otzjoks} Mund {2 Mile} Den Østre saavel som den Vestre Elv-Bræde af disse 3 Milers Strekning er fielded med smaa Sletter imellem, som ere Bierke-groede.\par
\textit{Storfossen} er at agte derfor, (1) at en 2 Bøsseskud Sønden for \textit{Fossen} ligger i Elven en Holm, kalded \textit{Foss-Holmen}, som skal være et gammelt Bøyde-Raamerke imellem \textit{Tana-} og \textit{Arisbye Finner}; see \textit{p.} 313.\par
(2) Paa Vestre Elv-Side imellem \textit{Storfossen} og \textit{Otzjoks} Mund hensættes gemeenligen udarmede Field\textit{Finner}. Hvilke leve af Lax-Fiskerie om Sommeren, og af at fange Ryper om Vinteren; Laxen føre de gerne ned til den \textit{Norske} Handel paa \textit{Guldholmen} om St \textit{Hans} Tider, og kiøbe sig derfor Meel, Salt \textit{etc.} Disse fattige \textit{Finner} søge til den Svenske \textit{Arisbye-} som nærmeste Kirke; Om, og til hvem de skatte, vidste Vidnerne her ikke.\par
Fra \textit{Otzjoks} Mund i Syd-Syd-Vest til \textit{Lill-Fossen} er {2 Mile} Fra \textit{Lillfossen} Ligeledes til \textit{Juxbye}{1 Miil} (\textit{Conferat}. pag: [ ]) paa Søndre Side af \textit{Juxbye} løber en liden Aae \textit{Nulli-jok} i \textit{Tana}-Elv. Fra \textit{Juxbye} i Syd-Syd-vest til \textit{Valljoks} Mund ved {1 ‒} Fra \textit{Valljok} ligeledes til \textit{Karasjoks} Mund mod {2 ‒}\par
Vestre og Østre Elv-Bræder have høye og lave Bakker og Sletter derimellem, \hypertarget{Schn1_91145}{}Tanafiord. Kiøllefiords Præstegield. 1/8 Miil meer og mindre brede, bevoxen med Lyng og Moese, saa og Bierk og noget Furreskoug; All den Strekning fra \textit{Otzjoks} Mund herhid er u-beboed: dog sidde \textit{Arisbye- Finner} heroven for paa Fieldene om Vinteren.\par
Fra \textit{Karasjoks} Mund omtrent i Søer til \textit{Jndiagers} Sommer Sæde {3 Miil} er Elv-Brædene med Bierk og Furre bevoxne. Fra \textit{Jndiagers} Sommer-Sæde i Syd- Syd-vest, dog kroget, angives at være 3 Dagers Reise, som man slutter at blive korte Field-Mile {9 ‒} Elv-Brædene skal her og have Furre og Bierk:\hspace{1em}\par
til Sp. 7. Svar: \textit{Dale i Tanafiord} paa Østre Landz Side:\hspace{1em}\par
Fra \textit{Tana-Hornet} 1 Miil i Søer er\par
1. \textit{Yttre Molvigdal}, 1/8 Miil i Gabet viid, mod 1/4 Miil dyb i Øster, hvorj smaa Skoug, og \textit{Molvig}-Aae er, u-beboed: Dog sidde \textit{Norske} Field\textit{finner} der om Sommeren. Fra \textit{Molvigdal} imod 1 Miil i Søer er\par
2. \textit{Trollfiord}-Dal, 1/4 Miil i Gabet viid, 3/4 Miil i Øster lang; Hvorj \textit{Trollfiord}-Elv fra et Field løber først 1/2 Miil i Nord, siden 1/2 Miil i Vester, med smaa Skoug og Græss bevoxen, u-beboed, bruges om Sommeren af Norske Field\textit{Finner}. 3/4 Miil her fra i Søer findes\par
3. \textit{Jndre Molvigdal}, 1/8 Miil i Munden viid, imod 1/2 Miil lang i Øster, med lidet Bierk, Eene og Græss i, hvorj \textit{Molvig}-Aae fra Øster i Vester 1/2 Miil lang rinder; bruges af Norske Field\textit{finner}.\par
J \textit{Østre Tana} er\par
4. \textit{Leerpoldal}, i Botten af \textit{Leerpoll}, 1/4 Miil i Gabet viid, 1 Miil lang i Øster med smaa Skoug og Græss i, hvor 2de Norske Søe\textit{finner} boe; Herj rinder Østen-fra af Fielde \textit{Leerpoll}- Elv, 1 1/2 Miil lang. ‒\par
Ved \textit{Tana}-Elv\par
5. \textit{Tana}-Dal langs efter Elven i Syd-Syd-Vest, 1/4 Miil meer og mindre breed, med GræssSletter, smaa Skoug, og Sønder-paa med Furre forsiuned.\par
Til denne Hoved-\textit{Tana}-Dal gaae andre Tværdale: Paa Vestre Land-Side 1 Miil Sønden for \textit{Guldholmen}\par
6. \textit{Golgok-dal}, hvorj \textit{Golgok}-Aaen rinder. 1/2 Miil Sønden herfor\par
7. \textit{Mask-dal}, hvorj \textit{Mask-Aae}. 2 Mile Sønden herfor, og 3/4 Miil Norden for \textit{Storfossen}\par
8. \textit{Skaaraaedal}, hvorj \textit{Skaar}aaen løber.\par
Paa Østre LandSide af \textit{Tana}-Elv\par
9. \textit{Kaskar-dal}, fra \textit{Guldholmen} 1/2 Miil i Søer ‒\par
10. \textit{Evarasdal}, mod 1/2 Miil Sønden for \textit{Kaskardal}.\par
11. \textit{Yttre Sørm-aukiesdal}, 1/4 Miil Sønden for \textit{Evaras} ‒\par
12. \textit{Jndre Sørm-aukiesdal}, fra den \textit{Yttre} 1/8 Miil i Søer. Herfra i Søer 1/4 Miil\par
13. \textit{Lodjokdal}, 1/4 Miil i Gabet viid, 1 Miil i Øster lang til \textit{Varangsbotten}, saavidt og Bierke-skougen rekker; Herj Aaen \textit{Lodjok} fra Fielde rinder 1/2 Miil i Syd-vest, siden 1 Miil i Vester. ‒\par
14. \textit{Rosa-dal}, 3/4 Miil Sønden derfor.\hypertarget{Schn1_91415}{}Schnitlers Protokoller V.\par
15. \textit{Bolma-dal} omtrent 3/4 Miil derfra i Søer, 1/8 Miil i Gabet breed, i Syd-ost efter Vatsdalen 2 1/4 Miil lang, hvorj \textit{Bolma}-Elv rinder; J de øvrige er det kun Bække, som gaae. ‒\hspace{1em}\par
til Sp. 8. Svar: Korn voxer ikke i \textit{Tana-Capells} Sogn; Vel har de fra \textit{Sverrig} overkomne \textit{Qvæner}, ved \textit{Tana}-Elv boendes 2 à 3 Mile Søer fra Elv-Munden, forsøgt at saae Korn, men det er ei bleven monedt; Da nu \textit{Qvænerne} ved \textit{Altens} Elvs Mund i \textit{Vest-Finmarken} avle Korn, og \textit{Tana} Elv i \textit{Ost-Finmarken}, der hvor \textit{Qvænerne} have ny-bygget, dens \textit{Clima} ganske lidet \textit{differerer} fra \textit{Altens} Elvs Mund, saa gives for den Forskiell i Avlingen den Aarsag, at i \textit{Alten} er Furre-Skoug, hvis Jord er tørr: derimod i \textit{Tana} er Bierk, hvoraf Landet er fugtigt og kaaldere.\hspace{1em}\par
til Sp. 9. Svar: J \textit{Tana}Fiorden paa begge dens Fiord-Bræder og LandSider er saa got' som ingen Skoug: uden i \textit{Trollfiord}Botten og ved dens Elv paa Vestre LandSide Bierk' ligesaa er i \textit{Vestre Tanas} Vestre Botten, og i \textit{Østre Tana}, nemlig \textit{Leerpollen}, omkring Botten noget Bierk; Op efter \textit{Tana}-Elv hen til \textit{Otzjok}-Kæften er hielpsom Bierkeskoug, paa ElvBrædene, ligeledes er noget smaa Skoug ved de mindre Tvær-Aaer, rindende i \textit{Tana}-Hoved-Elv; Ved 1 1/2 Miil i Syd-ost fra \textit{TanaElv} oven for det \textit{Nedre Bolma}-Vand er vel noget Furre, men \textit{Bolma}-Aae saa liden, at Timmeret derpaa ei kan nedflottes;\par
Ved \textit{Otzjok}-Elv, hvor den Svenske \textit{Arisbye} Kirke staaer, er Furre, hvoraf bem.te Kirke er bygget samt Præste-boeligerne, hvorved og Skougen temmelig er udarmed; dog hugge \textit{Arisbye} Fælles Field\textit{finner} endnu derj Boord og smaa Elv-Baader, og selge til de \textit{Norske} Søe- \textit{Finner} i \textit{Tana};\par
Efter \textit{Tana}-Elv længere i Søer fra \textit{Lill-fossen} hen op imod Grændsen er ligeledes Furre og Bierk, 1/4 Miil meer og mindre breed, den Søndreste deraf skal være u-nyttig, den Nordre bedre.\par
Efter \textit{Karasjok}, som gaaer i \textit{Tana} Elv paa dens Vestre Side, fra Munden saa høyt som Vidnet i 4 Dage har faret er skiøn Furreskoug, 1/2 Miil meere og mindre breed, hvor de Nybyggere \textit{Qvæner} ved \textit{Tana}Elv have taget deres Huuss-Timmer, og de 3de Svenske \textit{Qvæner}, boesiddendes paa den Syd-ostlige Side af \textit{Karasjok}, hugge Boord og smaa Elv-Baade, og selge til de \textit{Norske}\textit{Tana}-Søe\textit{Finner}.\hspace{1em}\par
Sp. 10. Svar: \textit{Vildt} og \textit{Jnsecter} ere her i \textit{Tanen}, som i \textit{Alten}\textit{pag:} 228. Om \textit{Kuorm}, eller Reen-Ormen, gives den Forklaring nærmere, at den avles sidst om Høsten imellem Huden og Kiødet paa Reenen og ved \textit{October} Maanedz Udgang begynder at stikke Hull paa Huden, ligesom nu Ormen voxer, giør den Hullet større, indtil dette efter \textit{Kaarsmiss} bliver saa romt, at Ormen falder af; Den Reen, som slagtes nu ved Mikkelsmiss, førend Ormen imellem Hud og Kiød fødes, dens Skind er reen for Hull: Men den, som slagtes efter \textit{October} om Vinteren for \textit{Kaarsmiss}, den har Orm, og Huller; Denne \textit{Kuorm} er saa mangfoldig undertiden, og stinger særdeles Aars-gamle Reen-Kalve saa sterk, at de deraf styrte.\hspace{1em}\par
Sp. 11. Svar: Leilighed til Rødning her i \textit{Tana} er:\par
a. J \textit{Hopsfiorden} paa \textit{Omgangs} Land paa Vestre Land-Side af \textit{Tanafiord}.\par
b. J \textit{Kefiord} i \textit{Vestre Tana}.\par
c. J \textit{Leerpoll}Botten i \textit{Østre Tana}, og paa dens Syd-Vestl. Side. see \textit{pag.} 321.\hypertarget{Schn1_91727}{}Tanafiord. Kiøllefiords Præstegield.\par
d. langs efter \textit{Tana}-Elv.\par
e. Ved \textit{Karasjok} paa Vestre Side.\hspace{1em}\par
Sp. 12. Svar: Fieldene beskrives ved Field\textit{tracten}\textit{pag.} 316.\par
Sp. 13. Sv: Den \textit{Octroyerede} Handel i \textit{Kiøllefiord} forsiuner \textit{Tana-Capells} Almue.\par
Sp. 14. Svar: Angaaendes Jndbyggernes Næring her i \textit{Tana} er før svaret \textit{pag:} 291 saavidt som det rører Søe-\textit{Finnerne}; Field\textit{finnerne} foruden deres Lax-Fiskerie i \textit{Tana}-Elv, have og deres Reen, Skøtterie og ferskVandz Fiskerie at leve af.\hspace{1em}\par
Sp. 15. og 16. Svaret see \textit{pag.} 291.\par
Sp. 17. Svar: Vei herfra \textit{Tana}fiord til den 1te \textit{Svenske Marke}sted \textit{Arisbye} fares i Baad paa \textit{Tana}-Elv, dog maa Baaden drages over Land 1 Steen-Kast forbi \textit{Storfossen}, nemlig om Sommeren: Om Vinteren i \textit{Finnernes} Kieredster paa Jisen langs \textit{Tana}-Elven, og siden paa \textit{Otzjok}-Aaen, som er fra \textit{Tana}Fiord Botten {7 Field Mile}\par
Fra \textit{Arisbye} eller \textit{Otzjoks}Munden gaaes om Sommeren langs med \textit{Tana}-Elv paa dens Elv-Bræde igiennem Skouge alt op til Grændse-\textit{Kiølen}; Saaledes er for et par Aar siden en \textit{Qvæn} fra \textit{Storfinland} kommet med Kone og Børn, drivendes med 4 Køer paa Vestre \textit{Tana}- Elv-Bræde, over \textit{Kiølen} imellem \textit{Parse-oive} og \textit{Borvoive}, der hvor \textit{Tana}-Elv oprinder, alt ned til \textit{Tana}fiord, og har undervejs fundet Græss nok paa Elv-Bræden for sine Køer; Samme \textit{Qvæn}, som er her det 24de Vidne, beretter og, at man med løs Hest kunde ride den Vej, og dertil finde Græss nok.\par
Sønden for \textit{Kiølen} rinder den store \textit{Kimi}-Elv ad \textit{Sverrig} i den Bottniske Søe, imod \textit{Tana}- Elv, som løber ad \textit{Norrig} i Nord-Søen.\par
Sp. 18. Svar: \textit{Norske} Field\textit{finner} i \textit{Tana}fielde ere {11. Famil.} hvortil nyelig er kommet fra \textit{Arisbye}{1. _______ = 12 ‒}\par
Disse \textit{privative Norske} Field\textit{finner} bruge alleene Norske Fielde heele Aaret igiennem nemlig om Sommeren sidde fra \textit{Tanahornet} i Søer til \textit{Stangnæss}, den Vestre LandSide af \textit{Tana}- Fiord,{6. Field\textit{finner};} Østen for \textit{Leerpoll}Botten{5 ‒} Østen for \textit{Tana}-Elvs\textit{Sørm-aukie} Dal{1 ‒ _______}\par
Naar nu Reen-Moesen her ved disse SøeKanter er fortæred, fløtte de alle imod Vinteren i \textit{November} Maaned i Søer til \textit{Bolma}, hvor Reen-Moesen er i Behold, Østen for \textit{Tana}-Elv, og sidde der Vinteren over omkring begge \textit{Bolma}-Vande og paa begge Sider af \textit{Bolma}-Aae, som altid har været uden Modsigelse holden for \textit{privative} Norske Grund, saa de Kongelige Svenske Betiente hverken Geistlige, ej heller Verdslige nogen Tid med den Grund, eller med de Field\textit{Finner} have befattet sig.\par
Did til dette \textit{Bolma} fløtte og om Vinteren, om Jule-Tider \textit{Laxefiords Norske} Field\textit{Finner}, (see \textit{pag.} 292) hvoraf gemeenlig de 2de \textit{Finner} selv personlig med deres Reen henkomme, men de 2de Øvrige, som ved Søen tilbage blive, sende deres Reen med dennem derhen.\hypertarget{Schn1_92060}{}Schnitlers Protokoller V.\par
Disse \textit{privative} Norske Field-\textit{Finner dependere} alleene af den Kongelig \textit{Norske} baade Geistlige og Verdzlige \textit{Jurisdiction}, besøges og af de Norske \textit{Missions}Betientere baade om Sommeren ved Søe-Siden, og om Vinteren ved \textit{Bolma} ‒\par
Om \textit{Porsangers Norske} Field\textit{finner} fik man her den Underretning, at de her have hørt, at samme om Sommeren skal sidde Vesten for \textit{Porsanger}-Fiord, og paa Fieldene oven for Botten: men om Vinteren ved \textit{Jetzjok}, og \textit{Karasjoks} Søndre og Nordre Side, uden noget derfor at svare til den Kongelig Svenske Øvrighed.\hspace{1em}\par
Sp. 19. Svar: Om \textit{Arisbye} Field\textit{Finner} sige de Vidner her det samme, som de i \textit{Kiøllefiord}\textit{pag.} 292, med den nøyere Forklaring, at de meene, der skal være nogle og 30 \textit{Familier} af de Skattende Field\textit{Finner}, hvilket Tall dog hastig Forandring formedelst Bortfløtning og Dødzfald er undergiven;\par
Disse \textit{Arisbye} Field\textit{Finner} sidde om Sommeren imellem de Norske \textit{Laxe-} og \textit{Tana}Fiorder, \textit{item} oven for \textit{Tana}Fiordz Vestre Botten langs med \textit{Tana} Elv paa dens Vestre Side fra FiordBotten i Søer til \textit{Storfossen}; de bruge alle La\textit{x}e-Fiskerie Sønden-fra at reigne i \textit{Otzjok}, og fra \textit{Otzjoks}Kæften i Nord til \textit{Storfossen}; Nogle faa af dennem gaae end videre over \textit{Storfossen} hen til \textit{Bolma}-Aae 1 Miil, og fiske iblant de \textit{Norske} Field\textit{finner}; Laxen selge de, som de Norske, til den Norske Handel paa \textit{Guldholmen;}\par
Om Vinteren sidde et par Mand af de \textit{Arisbye-Finner} ved \textit{Karasjok}; de Øvrige ved \textit{Otzjok}-Elv Norden og Sønden for \textit{Arisbye}-Kirke; og nogle, som af dennem fattige ere, 1 Miil Østen for \textit{Otzjok}, ved \textit{Væke-jaure} og \textit{jok}, 1 à 2 Mile Sønden for de \textit{Norske Bolma-Finner}.\par
Den Norske \textit{Tana}Fiordz\textit{Missions} Skolemester \textit{Peder Tudersen} sagde og, ikke at have med \textit{ArisbyeFinner}, medens de ere her i Norske Fielde ved Søe-Siden om Sommeren, at bestille.\hspace{1em}\par
Sp. 20. Svaret er før \textit{pag.} 292.\par
Sp. 21. Svar: Førend beskrives det \textit{privative-Norske} Land af \textit{Finmarken}, som holdes for at ligge under \textit{Norges} eene Geistlige og Verdzlige \textit{Jurisdiction} nu omstunder, vill forud meldes, at i \textit{Finmarken} hidindtil til \textit{Tanafiord}\textit{inclusive} ere 4 Slags Folk: (1) \textit{Normænd} af Afkom, (2) Søe\textit{Finner} (3) \textit{privative Norske} Field-\textit{Finner} og (4) fælles, eller gemeenlig kaldede Svenske Field\textit{Finner}.\par
(1) \textit{Normænd} indehave alle Øerne, som ligge Norden for \textit{Vest-Finmarken}, indtil \textit{Magerøe inclusive}, hvor \textit{Nord-Capen} er paa; Thi \textit{Øst-Finmarken} har ingen omfløded Øe uden for det faste Land, naar man undtager \textit{Vardøe;} Disse \textit{Normænd} nære sig alleene af Fiskerie i Havet, bruge ikke Agerdyrkning, eller Skiøtterie, fattes og Skoug til Brænde-hved.\par
(2) Søe\textit{Finner} sidde inde i Fiordene, og have deres Fiskerie inde hos dem, Bierk til Brænde-fangst, Skiøtterie i Fieldene ovenfor Fiordene, men ingen Agerdyrkning; Paa SøeKusten imellem Fiordene sidde faa, deels \textit{Normænd}, deels Søe-\textit{Finner}.\par
(3) \textit{Norske} Field-\textit{Finner} besidde de Nordeste Fielde og Dale nærmest Søen, nærende sig af deres Reen, Vildt-Skiøtterie, og ferske Vand-Fisk.\par
(4) \textit{Svenske} Field-\textit{Finner}, som de her i Landet gemeenlig kaldes, men jeg meener, bør nævnes Fælles Field-\textit{Finner}, efterdi de skatte baade til \textit{Norge}, og \textit{Sverrig}, indehave den Søndre største Deel af \textit{Finmarken}, og nære sig ligesom de \textit{Norske} Field\textit{Finner}.\hypertarget{Schn1_92393}{}Tanafiord. Kiøllefiords Præstegield.\par
Det 5te Slags Folk kan tillægges, kaldede \textit{Qvæner}, som ere \textit{Svenske} Bønder, komne fra \textit{Torne}-Land, eller \textit{Stor-Finland}, som have satt sig ned ved Elvene, og nære sig gemeenlig af Laxe-Fiskerie, og deres Qvæg; De \textit{Qvæner} ved \textit{Altens} Elv have desuden Agerbrug: dog disse, naar de sidde paa \textit{privative Norsk} Grund, kan henreignes til \textit{Norske} Bønder.\par
De 3de første Slags Folk ere nu alleene \textit{Norske} Undersaattere: Og har det med de 2de første, nemlig \textit{Normænd} og Søe\textit{Finner} ingen Vanskelighed; Eftersom de \textit{Svenske} fra SøeSiden ere udelukte:\par
Om det 3die Slags, nemlig de \textit{Norske} Field-\textit{Finner}, give her Vidnerne og Almuen den Forklaring, at de med deres Drift og Brug ere altid gaaet, Vesten-fra at reigne:\par
Til den Elv \textit{Jetzjok}, paa hvis Søndre Side det \textit{Svenske} Markested \textit{Avjevara} ligger, rindendes fra Vester omtrent i Øster i \textit{Karasjok}-Elv; Videre\par
til den Elv \textit{Karasjok}, hvor \textit{Porsangers} Field\textit{finner} om Vinteren sidde paa dens Nordre og et Støkke paa dens Søndre Side; Fremdeles\par
til det forrige \textit{Svenske} Markested \textit{Juxbye}, ved \textit{Tana}-Elv; Saa\par
til \textit{Foss-Holmen} i \textit{Tana}-Elv, et par Bøsseskud Sønden for \textit{Stor-Fossen}; hvilken \textit{FossHolm} fra gammel Tid har været et (om man saa skal kalde det) \textit{Bøyde-Raamerke} imellem \textit{Tana}- og \textit{Otzjok}-Field\textit{finner}: dog bruge nu omstunder \textit{Arisbye} Svenske \textit{Finner} om Sommeren denne \textit{Foss-Holm}, at stænge derfra over Elven for Laxen; Og nogle faa af dennem gaae over \textit{Storfossen} i Nord hen til \textit{Bolma}-Aaes Mund, iblant de \textit{Norske Finner}, at bruge deres Fiskerie.\par
\textit{Foss-Holmen} er 3 Bøsseskud lang i Søer, og 2 Bøsseskud breed, slet med Græss og Bierk paa. (\textit{Conferatur}\textit{pag.} 230 og 292 f.) Endelig oven for \textit{Bolma}-Vande imod \textit{Væke}-Vand.\par
Derimod de saa kaldede Svenske \textit{Finmarkens} Field-\textit{Finner} have siddet til \textit{Avjevara} ved \textit{Jetzjok}, at forstaae fra Grændse-Kiølen af; . . . videre\par
til \textit{Karasjok}-Elv; Saa\par
til \textit{Juxbye} ved \textit{Tana}-Elv\par
til over \textit{Foss-Holmen} i \textit{Tana}-Elv; Endelig til \textit{Væke-jaure inclus:}, 1 Miil Sønden for det \textit{Norske Bolma-jaure}, Fra hvilke Steder de indehave Landet i Søer alt til Grændseskiellet.\par
Om hver især af disse Steder gives følgende Beretning:\par
\textit{Avjevara} er et Svensk Marke- og Tingsted nu paa Søndre Side af \textit{Jetzjok}-Elv, men tilforn staaet paa Nordre Side;\par
\textit{Avjevara-Finner sortere} under den \textit{Svenske}\textit{Koutokeino}-Hoved-Kirke, sidde om Sommeren ved den \textit{Norske}\textit{Altens}Fiord, og det \textit{Norske Hvalsund}, en Deel deraf ligge ved \textit{Karasjok}, og fiske Lax; Om Vinteren ligge de i Fieldene Sønden for \textit{Jetzjok}, og have deres Vild-Bane efter Vilde Reen oven derfor, Vesten for \textit{Øvre-Tana}-Elv i Søer alt op til Grændsen, \textit{Jaurisduøder};\par
Dette \textit{Avjevara}-Markested skal ligge 3 Mile Østen for \textit{Sios-jaure}, hvorigiennem \textit{Jetzjok} fra \textit{Jetz-jaure} rinder; og ved 3 Mile Vesten for \textit{Jetzjoks} Mund, hvor denne i \textit{Karasjok} udløber; \textit{Jetzjoks} Mund skal være 1 1/2 Miil Sønden for \textit{Karasjoks} Mund, hvor denne i \textit{Tana} Elv udfalder.\par
\textit{Karasjokby} været før en \textit{Svensk Lappebye} paa Søndre Side af \textit{Karasjok}-Elv, og \textit{sorteret} under \textit{Koutokeino} Hoved Sogn: men nu forladt, benyttes af \textit{Arisbye} Svenske \textit{Finner}; Der sidde og paa dens Søndre Side 3de Svenske \textit{Qvæner}, svarende deres Rettigheder til den Sven-\hypertarget{Schn1_92768}{}Schnitlers Protokoller V.\par
ske Øvrighed, nærmer til \textit{Karas}-Kæften. \textit{Juxbye} ved \textit{Tana}-Elv før været et \textit{Svensk} Markested, nu øde.\par
Om \textit{Væke-jaure} see \textit{pag.} 307.\par
Nu om stunder have de \textit{Svenske} kun 3de Markesteder, \textit{Kautokeino}, hvor Hoved-\textit{Finne}- Kirken er, \textit{Avjevara}, svarendes under denne Hoved-Kirke, \textit{Arisbye}, hvor \textit{Annex}-Kirken staaer ved \textit{Otzjok}-Elv ‒\par
\textit{Koutokeino} ligger Vester i \textit{Vest Finmarken}\par
\textit{Avjevara} derfra i Nordnordost omtrent 10. Mile, eller 2 Dagers Reise, (\textit{conferat.}\textit{pag.} 245.)\par
\textit{Arisbye} fra \textit{Avjevara} i Øster til Norden 3 Dagers Reise; holdendes det for 14. FieldMile.\par
Da 24de Vidne for en 10 Aar var i \textit{Koutokeino}, bestoed dens og \textit{Avjevaras} Meenighed af en 60. \textit{Finnefamilier}. (\textit{conferatur}\textit{pag.} 230.)\hspace{1em}\par
til Sp. 22. og 23. er svaret før \textit{pag.} 293. som og svares her; Tillæggendes den Beretning: J forrige Tider har Præsten i \textit{Koutokeino} forestaaet begge baade \textit{Koutokeino} som HovedKirke, og \textit{Arisbye}, som en \textit{Annex}-Kirke: men for 3. Aar siden er en særdeles Svensk \textit{Capellan} satt ved denne \textit{Arisbye} Kirke, som tillige er \textit{Missionaire}, og holder Skole for en 7 à 5 \textit{Finne}- Børn i det \textit{Finnske} Tungemaal; Denne \textit{Capellan} da han Aaret efter hans Ankomst foer ned til den \textit{Norske} Handel paa den \textit{Norske Guldholm} ved \textit{Tana}-Elvs Mund, har i sin \textit{privat} Stue præket og deelt \textit{Sacramentet} til de Svenske \textit{Arisbye} Field-\textit{Finner}, da værende i \textit{Norge}, og paa sin TilbageReise fra \textit{Guldholmen} hos Een og Anden af de \textit{Norske Qvæner} ved \textit{Tana}-Elv, paa \textit{Norsk} Grund forrettet \textit{Ministerialia}; J Aar skal han paa sin Opreise fra Handelen ligeledes have Betient et sygt Menniske hos de \textit{Norske Qvæner}, dog sige det ikke for vist.\hspace{1em}\par
til Sp. 24. Svarer 23de Vidne \textit{Mathies Mathiesen}\par
om \textbf{Grændse-Skiellet:}\par
Han har staaet omtrent ved \textit{Karasjok}Kæften, og seet derfra \textit{Jaurisduøder}, som ham siuntes var en 2 Mile Sønden fra ham, at samme \textit{Jaurisduøder} begyndtes i Nord; Hvorlangt det strekker sig i Søer? veed han ikke; Men det har han hørt: hvor \textit{Karasjok} og \textit{Tana}-Elv paa dette \textit{Jaurisduøder} oprinde og stikke i Nord, der skal Lande-Skifte være imellem \textit{Norge} og \textit{Sverrig}; Ligeledes har han hørt, at fra \textit{Karasjok}-Kæften i Søer til dette Landeskifte skal være ved 4 Dagers Reise, nemlig fra \textit{Karasjok}-Kæften til de \textit{Jndiagers} Sommerbye 1 Dags- og derfra til \textit{Tana}-Elvs Udspring 3de Dagers Reise: Men selv har han ikke været der, eller længere Søer, end til \textit{Jndiagers} Sommerbye ved Øvre \textit{Tana}-Elv.\par
Efter Rettens Tilspørgende: Om Vidnet ikke skulle ville gaae med andre Norske Besigtelses Mænd, langs efter \textit{Tana}-Elv i Søer indtil denne Elvs Oprindelse i \textit{Jaurisduøder}, og derfra siden fortgaae til \textit{Karasjoks} Begyndelse i samme \textit{Jaurisduøder}, og anvise, samt udmerke dem, for billig Betaling? Svarede Vidnet: Naar den Tid kom, og han havde Helsen, maatte der blive Raad til.\par
Om \textit{Beldo-vadda} eller andre Grændse-Merker har ei hørt noget.\hspace{1em}\par
24de Vidne \textit{Erik Bonjækas} især har hørt, at \textit{Jaurisduøder} giør Grændse-Skiellet imellem \textit{Norge} og \textit{Sverrig}, og at den heele Field-Rygg, som skiller paa denne Kant \textit{Norge} og \textit{Sverrig} ad, i Almindelighed saa kaldes, fordi Vande derfra rinde baade til \textit{Norge} og \textit{Sverrig}.\hypertarget{Schn1_93126}{}24de Vidne i Finmarken. Kiøllefiords Præstegield.\par
Dette \textit{Jaurisduøder} hvor det først begynder i Vester? og hvor det omsider ender sig i Øster? veed han ikke; Dog kan han sige, at det strekker sig i Øster; Og dengang han gik over dette \textit{Jaurisduøder}, der hvor \textit{Tana}Elv opkommer og stævner til Norge, var dette \textit{Jaurisduøder} saa langt, at han ei kunde oversee det, hverken til dets Vestre, eller dets Østre Ende.\par
Samme Tid da han foer derover, hørte han sige, at paa det Sted ved \textit{Tana}Elvs Udspring, skulle \textit{Jaurisduøder} være saa breed fra Søer i Nord, at en løs Karl har Een heel Samdøng fornøden, at gaae derover, som han slutter at være 5 FieldMile; Saavidt han kan mindes, oprinder \textit{Tana}Elv paa den Søndre Side, eller Deel af \textit{Jaurisduøder}, at derfra er kortere Vej til \textit{Sverrig}, end til \textit{Norge}.\par
Dette \textit{Jaurisduøder} er slet ovenpaa, bart, med Reen Moese paa; Jmellem kan der være lidet dalet, hvor Elve og Bække rinde, og derved staaer Græss.\par
Over hvert \textit{Grændse-merke} i sær paa dette \textit{Jaurisduøder} følger nærmere Forklaring, saavidt Vidnet veed:\par
\centerline{(a) \textbf{Salvasvadda}}\par
en Moese-Slette, derj et Fiske-Vand \textit{Salvas-jaure}; hvor Vidnet har fisket i; 1/8 Miil langt fra Søer i Nord, et Par Bøsseskud bredt; deraf rinder Aaen \textit{Salvas-jok} nordlig, og kommer omsider i \textit{Altens} Elv; Om dette \textit{Salvasvadda} har han hørt, at Grændse-Skiellet der skal være, men hvor paa Laug eegentlig? kan ei sige.\par
\centerline{(b) \textbf{Bevresmutkie}}\par
Ved dets Søndre Vand har han været, og fisket derj, og veed, at derfra rinder Aaen, \textit{Bevresjok} til \textit{Sverrig} ad \textit{Kimi}-Elv, 3 Mile lang i Søer, ikke langt fra \textit{Kimi}-Elvs Opkomst; \textit{Bevres-jaure} er fra Syd-vest i Nord-ost 1 Miil langt, næsten ligesaa-bredt: dog er han ikke viss paa \textit{Compass}-Strægen. 1 Miil herfra, som han siunes, ligger\par
\centerline{(c) \textbf{Modtatas-oive}}\par
Der har han hørt, at Grændse-Skiellet gaaer: men hvor paa Laug? kan han [ei] eegentlig sige; Det veed han, at der ligger et lidet Vand Østen ved \textit{Modtatas-oive}, hvorfra \textit{Karasjok} har sit Udspring ad \textit{Norge}, og har han selv været ved det lidet Vand: Men hvad det heeder? veed han ikke. Eller[s] forklarer han sig: Disse forbenævnte Merkesteder har han hørt at giøre skiellet; men hvordanne og i hvad Strekning de ligge? veed ej eegentlig.\par
\centerline{(d) \textbf{Parse-oive,}}\par
hvorved \textit{Karasjok} opkommer, og rinder i Nord, har han været ved, og hørt, at det giør Grændse-\textit{Limiten}.\par
\textit{Jokken} skal ligge 1/2 Miil Vesten for \textit{Parse-oive}, og 1 1/2 Miil Vesten for \textit{Gaune-jaure}.\par
\centerline{\textbf{Borvoive}}\par
har han ikke hørt nævne, men der hvor \textit{Tana}-Elv opspringer og rinder i Nord, det veed han, at giøre Lande-Skifte; Og har været ved \textit{Gaune-jaure}, hvoraf den rinder i Nord.\hypertarget{Schn1_93333}{}Schnitlers Protokoller V.\par
Østen for \textit{Jaurisduøder} ligger\par
\centerline{\textit{2}. \textbf{Beldo-vadda}}\par
men hvor langt? kan han ei sige; Om Vinteren har han været der, og vidner, at have hørt, at dette \textit{Beldovadda} ligesom \textit{Jaurisduøder} giør Grændse-\textit{Kiølen}. Videre om Grændse-Gangen ei vidste.\par
Sp. 25. Svar: Landskabet paa den Kongelig \textit{Norske} Side av \textit{Finmarken} vill paa tvende Slags Maader ansees,\par
(1) Det faste Land, som hidindtil er holden for \textit{privative Norsk} Grund, nemlig fra SøeKanten op til \textit{Jetzjok}, til \textit{Karasjok}, hvor \textit{Porsangers Norske Finner} sidde, til \textit{Juxbye}, til \textit{FossHolmen} og oven for \textit{Bolma}-Vande til imod \textit{Væke-jaure}, see \textit{pag.} 313.\par
(2) Det Jndre Land her oven for i Søer alt op til Grændse-\textit{Kiølen}, som af de Svenske indehaves med Geistlig og verdzlig \textit{Jurisdiction}: dog at af disse Søndre Field\textit{Finner} tillige skattes til \textit{Norge}:\par
Angaaendes det 1te \textit{Yttre Norske} Land, saa vide Vidnerne her ingen Beskeeden, hvorledes det dermed er fra \textit{Porsangers} Fiordz Botten i Søer til \textit{Jetzjok}, hvor \textit{Avjevara} ligger, men det er forklaret før i \textit{Kiøllefiord}\textit{pag.} 293.\par
Paa det Næss imellem \textit{Porsanger}- og \textit{Laxe}-Fiordene angive Vidnerne her, strax Sønden for \textit{Sverholt, Spirt-Niarg}, strekkende sig i Søer imellem begge bem.te Fiorder; Det er fladt, med Lyng og Moese paa.\par
Sønden derfor er \textit{pag:} 294 forklaret, at ligge \textit{VarejieZek, Gaisak}, Den Bierkedal med \textit{Valljok}-Aae i, og omsider \textit{Karasjok}.\par
Fra \textit{Laxefiord}Botten i Søer er sammestedz før forklaret, at være en Moese- og Græss- Dal med Bakker i, 2 Mile lang, derefter den Østre Ende af \textit{Gaisak}-Field, som i Søer neddaler til \textit{Tana}-Elv, hvor \textit{Juxbye} har været.\par
Det Land imellem \textit{Laxefiord} og \textit{Tana}Fiord bestaaer af den HalveØe \textit{Omgang}, og det Søndre faste Land; \textit{Omgang} er beskreven før \textit{pag.} 284. Hvortil her af Vidnerne tillegges, at det indre Opland af \textit{Omgang}, som bestaaer af luter nakene Fielde, kaldes med et almindeligt Navn \textit{Kiorgess-Niarg}. Landet Sønden for \textit{Omgangs}Eidet har følgende Fielde:\par
(1) \textit{Laisduoder} begynder strax Sønden for den Jndfiord, \textit{Langfiord}, strekkende sig i Søer ved 2 Dagers Reise lang hen til det Field \textit{Guølpak}, imellem \textit{Laxe}- og \textit{Tana}-Fiordene, med smaa deels runde deels spidse Tinder paa, Lyng- og Moese-groet.\par
(2) \textit{Guølpak} naaer imod \textit{Otzjok}-Kæften til \textit{Tana}Elv, med Søndre og imod \textit{Lill-Fossen} i \textit{Tana}-Elv med den Østre Ende see meere \textit{pag.} 294. Ellers er dette Land imellem \textit{Laxe-Fiord} og \textit{Tanafiord} 3 Mile meer og mindre bredt.\par
Landskabet paa Østre Side af \textit{Tana-Fiord} imellem \textit{Tana-Hornet} og \textit{Stangnæss} i Øster hen ad \textit{Svartnæss} imod \textit{Vardøe}, og videre fra \textit{Svartnæss} i Øster til det yderste Næss \textit{Kiberg} er slet-fieldet, deels steenet, deels Moeset med noget Græss imellem.\par
Landskabet Sønden derfor fra \textit{Leerpollen} i Sydost til \textit{Varanger}-Botten er med Bierk bevoxen fra begge Fiord-Bottene op et Støkke, siden sletfieldet med Moese og noget Græss paa.\hypertarget{Schn1_93668}{}24de Vidne i Finmarken. Kiøllefiords Præstegield.\par
Landskabet fra \textit{Moketveje} paa \textit{Tana}Elvs Østre LandSide, fra ElvMunden i Søer 2 3/4 Miil liggendes, i Ost-Syd-ost over \textit{Sadde-vare} til \textit{Varangers Vestre Botten} er slet fieldet med noget Bierk paa begge Sider ad Fiordene,\par
Landskabet fra \textit{Bolma} Gaard, den Sønderste Norske Bonde-Gaard ved \textit{Tana}Elv, fra Elv-Munden 3 3/4 Miil liggendes, er i Øster til \textit{Varangers Vestre Botten} Bierkeskouget, med smaa Bakker imellem.\label{Schn1_93709} \par 
\begin{longtable}{P{0.6991467576791809\textwidth}P{0.09573378839590443\textwidth}P{0.05511945392491467\textwidth}}
 \hline\endfoot\hline\endlastfoot Fra \textit{Tanahornet} i Øster til \textit{Svartnæss} er\tabcellsep 7 FieldMile\tabcellsep \\
herfra til \textit{Kiberg}\tabcellsep  1 ‒\tabcellsep \\
\tabcellsep _________\tabcellsep  8 Mile\\
Fra \textit{Leerpollen} til \textit{Varanger}Botten er\tabcellsep \tabcellsep  2 ‒\\
Fra \textit{Moketveie} til \textit{Varangerbotten}, hvor Post-Vejen er,\tabcellsep \tabcellsep  1 ‒\\
Fra \textit{Bolma} sammestedz hen\tabcellsep \tabcellsep  1 ‒\\
\tabcellsep _________\tabcellsep \end{longtable} \par
 \hspace{1em}\par
2. Landskabet af \textit{Finmarken} paa den Kongelig \textit{Norske} Side Norden for Landz\textit{kiølen}, som de saakaldede Svenske Field\textit{finner} besidde, er nu følgende:\par
Andenstedz \textit{pag:} 230 og 292 f. er bevidnet, at det strekker sig Nordenfra, fra det \textit{Norske Masi-Capell}, i \textit{Altens} Gield, og derfra omtrent i Øster, fra \textit{Avjevara}, det \textit{Svenske Markested} Sønden ved Jetzjok-Elv, videre fra \textit{Karasjok}, forrige \textit{Finnebye}, fra \textit{Juxbye}, Svensk Markested tilforn ved \textit{Tana}-Elv, og fra \textit{Arisbye}, Svensk Markested og \textit{Finne}-Kirke alt hen til den rette gamle Landz\textit{kiøl} i Søer;\par
Her sige Vidnerne, at den Svenske indehavende \textit{District} begynder i Nord fra \textit{Juxbye, Foss-Holmen}, Norden for \textit{Arisbye}, og fra \textit{Væke-jaure inclusive}: dog at \textit{Foss-Holmen} altid har været holden kun for et Bøyde-Raamerke imellem \textit{Tana}- og \textit{Arisbye-Finner}.\par
Den Sønderste Deel af denne de Svenskes \textit{Finmark}iske \textit{district} nærmest norden for og langs efter \textit{Kiølen} bestaaer i slette bare Fielde med Reen-Moese paa, og smaa mellem-rindende Bække, hvorved Bierke-Riis er; som af Field-\textit{Finnerne} er \textit{reservered} til Vild-Bane for VildReenSkøtterie om Høsten og Vaaren; hvorfore de ei pleje at ligge der med deres Tamme Reen, at deres Hunde ej skal bortskræmme de Vilde;\par
Saaledes skal \textit{Koutokeino-Finner} bruge den Vild-Bane Vesten for \textit{Karasjok}; De \textit{AvjevaraFinner} den Østen for \textit{Karasjok} halvvejs hen til \textit{Tana}-Elvs øvre Deel. \textit{Arisbye Finner} der neden under Vesten og Østen for \textit{Øvre}\textit{Tana}Elv hen til imod \textit{Mudkieduoder; Jndiager-Finner} have Østen for \textit{Avjevara-Finner} den Strekning Vesten og Østen for \textit{Tana}-Elv, og Østen for \textit{Arisbye-Finner} det Field \textit{Mudkie-Duoder} og saa videre i Øster; Denne heele Field-Strekning langs Norden for og nærmest \textit{Kiølen} er ellers til Jntet nyttigt, uden til Vild-Reen-Bane.\par
Landet nu imellem de Svenske benævnte Markesteder og denne Vildbane er nu følgende:\par
Sønden for \textit{Siosjaure} og \textit{Jetzjok}, hvor \textit{Avjevara} ligger, er Field-Landet beskreven før \textit{pag.} 226 og 245.\par
Sønden for \textit{Karasjoks}Kæften, hvor den i \textit{Tana}Elv løber, er en Furre- og Bierke-skoug, 3 Mile lang i Søer hen imod \textit{Jndiagers Finners} Sommer-Sæde, og sønderst 1 1/2 Miil breed imel\hypertarget{Schn1_94011}{}Schnitlers Protokoller V. lem \textit{Karasjok} og \textit{Tana}-Elv; J denne Skougs Søndre Deel, dog nærmere til \textit{Karasjok} ligge 2de smaa runde Berge, navnlig \textit{Eskuras-Varer}, fra Nord i Søer efter hinanden, med Reen-Moese paa, noget høye, hvert en 4 Bøsseskud over store;\par
Sønden for denne Skoug begynde de slette Moese-Fielde, Vild-Banen tilhørige.\par
Sønden for \textit{Otzjok}-Kæften, hvor den i \textit{Tana}-Elv falder, er Landet slet med smaa Furre og Bierk, en 4 Bøsseskud meer og mindre breed i Vester;\par
Sønden for denne Skoug er det Field \textit{Gieskadam}, \textit{pag.} 297 beskreven;\par
Sønden for \textit{Gieskadam} er en lang Græss-Dal med Moese og smaa Skoug, imellem \textit{Tana}- Elv og \textit{Merisjaure} i Øster, 1 1/2 fieldMiil breed i Søer til \textit{Baisduoder}, igiennem hvilken Dal \textit{Kiævojok} rinder ad \textit{Otzjok}-Elven.\par
\textit{Baisduoder} siges her, at være 1 Miil fra Nord i Søer, halv saa bredt, slet oventil, og fladtvoren paa Sidene, med Reen-Moese paa.\par
3/4 Miil Syd-Vest for \textit{Baisduoder} er \textit{Niaug-oive}, hvorimellem Myr- Lyng- og Moese-Land med Bierk j; \textit{Niaug-oive} er fra Syd-vest i Nord-ost 1/2 Miil langt, 1/4 Miil bredt, slet ovenpaa, brat paa den Side ad \textit{Kiævo-jok}, ellers fladtvoren.\par
Sønden for \textit{Niaugoive} 1 Miil er \textit{Mudkie-duoder}; Landet derimellem er først nærmest \textit{Niaug-oive} Bierk, siden Furre og Bierk, med Myr og Moese.\par
Ved \textit{Niaugoives} Vestre Side opkommer \textit{Kiævo-jok}, der rinder i Nordost ad \textit{Otzjok};\par
Jmellem \textit{Niaug-} og \textit{Mudkie-Duoder}, dog nærmere til denne oprinder af Vatz-Kiønner den Aae \textit{Garegass}, i Nordvest 1 Miil lang i Øvre \textit{Tana}-Elv, 1/4 Miil Sønden for \textit{Karasjoks} Mund.\par
\textit{Mudkie-duoder} har 23de Vidne seet, at være et stort, ovenpaa slet og bart Field, neden under havendes Furre imod \textit{Tana}-Elv; hvilket Field \textit{Jndiager Finner} sig tileigne, og forbyde \textit{Arisbye Finner} at komme did op, \textit{reserverende} sig det til VildReen-Skøtterie.\par
Her paa laug ved Vestre Side af \textit{Tana} Elv, 3 Field Mile Sønden for \textit{Karasjoks} Kæft, er \textit{Jndiagers} Sommer-Sæde, hvorfra i Søer hen til \textit{Kiølen} Landet Vesten for \textit{Tana}-Elv bestaaer af slette bare Fielde med Reen-Moese paa, og rindende Bække imellem, hvorved Bierke-Riis staaer.\par
Landskabet Sønden for \textit{Vækejoks} Kæften, hvor den i \textit{Tana} Elv indgaaer, er imellem \textit{Otzjok} og \textit{Væke-jokAdnevara}, og der Sønden for \textit{Skallo}-Field, før \textit{pag.} 297 beskrevne, Østen for \textit{Otzjok}-Elv.\par
Sønden for \textit{Skallo-vara} er slet Skoug-Land af Furre og Bierk, alt hen imod \textit{Betzeko}, og Østen for \textit{Merisjaure}; hvilket \textit{Merisjaure} er beskreven pag. 297 og at den Svenske FieldØvrighedz Gamme derved staaer, hvorhen 22de Vidne siger, de have fra \textit{Arisbye} Kirke 1 Dags Køer, eller ved 4 Mile i Søer;\par
Fra denne Gamme forklarer han, de have i got Vejer 1 Dags ‒ men i tungt Føre 2 Dagers Køer i Øster til \textit{Jndiager}- eller \textit{Anar}-Bye, saaledes:\par
Fra Gammen over slet Bierke-skougland i Øster {1/2 Miil} til \textit{Betzeko}-Vand, som er rundt, 1/4 Miil stort. Fra \textit{Betzeko} løber Aaen i Søer i \textit{Seukejaure}, {1/2 ‒}\textit{Seuke}-Vand er 1/4 Miil bredt, langt fra Nord i Søer {1 ‒} Fra \textit{Seuke-jaure} stikker Aaen i Søer, {1/2 ‒}\hypertarget{Schn1_94326}{}24de Vidne i Finmarken. Kiøllefiords Præstegield. i \textit{Kiaukisjaure}, rundt, stort, {1/2 Miil} Fra dette Vand stævner Aaen i Søer i en større Elv, \textit{Gamas}; men Øvrigheden farer fra \textit{Kiaukisjaure} i Øster igiennem Furre- og Bierke-Land {1/2 ‒} over 3 smaa runde Vande, hvert 1 Bøsseskud stort, og 1 Bøsseskud fra hinanden liggende i Øster, deraf Bækken rinder i Søer i \textit{Gamas}-Elv; Fra de smaa runde Vande køre de i Øster {1/4 ‒} til \textit{Maaskesjaure}, 1/2 Miil bredt, og fra Vester i Ø. langt {1 ‒}\par
Tæt Sønden forbi dette \textit{Maaskesjaure} løber \textit{Gamas}-Elv, og af \textit{Maaskesjaures} Vestre Ende stikker Aaen i Søer i samme \textit{Gamas};\par
Fra \textit{Maaskesjaure} reise de Svenske Betientere igiennem en Furreskoug {1/4 ‒} til et Vand (Navnet ej vides) langt i Øster {1 ‒} og halv saa bredt; Jon Olsen kalder dette \textit{Voidas-jaure}. Over dette Vand fare de i Øster giennem Furreskoug {1/2 ‒} over et andet Vand (uden Navn) rundt, 1 Bøsse-skud stort; Jon Olsen kalder det \textit{Lotasjaure}. Fra dette mindre Vand i Øster igiennem Furreskoug {1/4 ‒} til det 3die u-benævnte Vand, rundt, 1 Bøsseskud stort (\textit{Lotas} ligeledes kaldet.) Her over igiennem Furreskoug i Øster {1/2 ‒} til 4de Vand, rundt, 1 Bøsseskud stort, hvorved \textit{Jndiagers Finner} Boe; Jon Olsen kalder det \textit{Bielba-jaure}. Siden til \textit{Jndiager}Vand, hvor-langt? veed ej videre. {_______ langt 7 1/4 Mil.}\par
Samme 22de Vidne har været paa Vild-Reen-skøtterie, \textit{Beldo-vadda} saa nær, at han har seet det, men ei været der; Fra \textit{Jndiagers Finner}, hvor de sidde ved 4de Vand, har han haft 3 Dagers Reise om Vinteren til \textit{Beldo-vadda} i Søer til Vesten, hvilket \textit{Beldo} han sluter at ligge fra \textit{Jaurisduøder} 1 Miil i Øster; Han har hørt, at \textit{Tana}-Elv skal opkomme imellem \textit{Jaurisduoder} og \textit{Beldo-vadda}, og at en anden Elv fra \textit{Beldo-vadda} skal rinde ad \textit{Sverrig}. Men om noget Grændse-Merke vidste intet at vidne.\par
J forbeskrevne Reise fra \textit{Otzjok} til \textit{Jndiager} er rørt om den \textit{Gamas}-Elv; Derom giver han den Beretning:\par
\textit{Otzjok}, siger han, at opkomme oprindeligen af en Vatz-Kiøn, tæt Norden for \textit{Saxejaure}; Dette \textit{Saxe-Vand} er 1/4 Miil stort, og rundt, liggendes fra \textit{Merisjaure} i Søer 1 Miil; Af samme \textit{Saxe}-Vand rinder \textit{Saxe-jok} i Søer 1/2 Miil lang i Gamasjok: Denne \textit{Gamas}-Elv oprinder af \textit{Røeter}-Field, 1 Miil Vesten der for, hvor \textit{Saxejok} indløber, og fortgaaer i Øster alt hen til \textit{Jndiager}-Vand, 2 sterke Dagers Reise langt, indtagendes Norden-fra adskillige Aaer og Bække.\hspace{1em}\par
Sp. 26. Svar: Om Landskabet paa den Kongelig Svenske Side Sønden for Landz\textit{kiølen} siger 24de Vidne følgende:\par
\textit{Kimi}-Elv oprinder af \textit{Moedke-jaure}, som er fra Sydvest i Nord-ost 2 Mile langt, imod 1/8 Miil bredt; Af dette \textit{Moedke-jaures} Nordostlige Ende udgaaer \textit{Kimi}-Elv ‒ først i Øster 2 Mile, hvor den tager \textit{Bevres-jok} ind til sig fra Grændse-\textit{Kiølen}, siden vender \textit{Kimi}-Elv sig ad Søer, og løber derefter 14. korte Field-Mile i Søer, til det \textit{Svenske} Markested \textit{Kittila}, siden derfra i Søer til \textit{Kimi} Markested, hvor mange Mile? veed ej.\hypertarget{Schn1_94603}{}Schnitlers Protokoller V.\par
J denne Kimi-Elv rinder fra Grændsen af \textit{BevresjaureBevres-jok} 3 Mile lang i Søer, der, hvor \textit{Kimi}-Elv snoer sig ad Sønden.\par
Fra \textit{Moedke-jaure} lige hen til \textit{Kittila} paa begge Sider af \textit{Kimi}-Elv boe Svenske Bønder, sommestedz Enkelt-viis, sommestedz 3. 4. à 6 Mand tilhobe, 1/4‒1/2‒1. ja 3‒4. à 5 Mile fra hinanden, og bruge Agerdyrkning, holdendes Heste og Qvæg, dog er Jorden frostnævnt. De bruge og Skiøtterie i Skougen, og at fiske Lax i Elven. Paa begge Sider af \textit{Kimi}-Elv er Landet slet uden Fielde, med mangfoldig stor og smaa Skoug af allehaande Slags; Nær ved Elven kan sommestedz være Græss, men noget derfra er Myrland, Moese og Lyng.\par
Paa Østre Side af \textit{Kimi}-Elv imod \textit{Rusland} boe og Bønder ved Vande og Elve. ‒\par
Paa Vestre Side fra \textit{Kimi}-Elv boer Jngen hen imod \textit{Torne}-Elv.\par
Oven for eller Norden for \textit{Kittila} skeer ingen Soldater-Udskriving, men strax Sønden derfor ‒\hspace{1em}\par
Sp. 27. 28. Svar: Vej herfra til Grændsen see \textit{pag.} [ ]; Tvistighed ingen.\hspace{1em}\par
Sp. 29. Svar: Nytten af Grændse-Fieldene og der nordenfor værende Landskab er alleene for Field-\textit{Finner;} J \textit{Tana}-Elv og \textit{Karasjok} fanges Lax, og faaes Perler ‒\hspace{1em}\par
Sp. 30. Sv: Grændse-Stederne ere i Kongens Alminding.\hspace{1em}\par
Sp. 31. Sv: De nærmeste \textit{Finner} herfra i Øster ere \textit{Varangers Finner}. ‒\hspace{1em}\par
Efter at dette var forrettet, fremkom en Mand, som og vidste og havde hørt noget om GrændseGangen; Hvilken man derpaa toeg i \textit{Corporlig} Eed, i Overværelse af samme Lensmand og LaugRettes Mænd:\par
\centerline{\textbf{25de Vidne}, \textit{Jon Olsen}, Tienendes hos den \textit{Norske} Field-\textit{Finn}Peder \textit{Nielsen}.}\par
Denne \textit{Jon Olsen} er født i \textit{Arisbye}, og døbt sammestedz; Hans Forældre have været \textit{Arisbye} Field-\textit{Finner}, og om Sommeren med dennem holdet til paa de \textit{Norske} Fielde, Vesten for \textit{Tana}-Elv; Fra samme hans Forældre er han for 4 Aar siden kommet i Tienneste her i \textit{Norge} hos den \textit{Norske} Field-\textit{Finn Peder Nielsen;} Om han fremdeles bliver her og nedsætter sig her i Landet? det veed han ikke endnu; Er 30 Aar gammel, u-gift, gaaer til Gudz Bord i \textit{Arisbye} Kirke hos den Svenske Præst, hvor han sidst afvigte Paaske har betient sig af den hellige \textit{Communion;} Han vidner:\par
Af gamle Folk i \textit{Arisbye} at have hørt: hvor \textit{Karasjok} og \textit{Tana}-Elv først oprinder, der skal Grændse-Skiellet være imellem \textit{Norge} og \textit{Sverrig}; han har og hørt om \textit{Karasjok}, at den opkommer i \textit{Jaurisduoder}, det samme han og har hørt om \textit{Tana}-Elv: men hvad de Steder eegentlig hede, hvorfra disse Elve udrinde? det veed han ikke; Ligeledes har han hørt, at heele \textit{Jaurisduøder} giør Grændse-\textit{Kiølen} imellem \textit{Norge} og \textit{Sverrig:} men selv har han ikke været der; Og om andre Grændse-Merker veed han intet at sige.\par
Hvorpaa Retten paa dette Sted blev slutted, og Lensmand og Vidnerne for Retten betydet: Naar de af den Kongelige \textit{Norske} Øvrighed tilsiges, skal de begive sig til Field- \hypertarget{Schn1_94854}{}25de Vidne i Finmarken. Kiøllefiords Præstegield.\textit{Kiølen} med fleere Folk, som Øvrigheden dennem medgiver, at udviise, og udmerke GrændseStederne. \textit{Leerpollbotten} j Ø. \textit{Tana}Fiord i \textit{Kiøllefiords} Gield i \textit{Østre Finmarken} d. 31 Octob. 1744 \hspace{1em}\centerline{Peter Schnitler}\hspace{1em}Samuel Samuelsen\hspace{1em} L. S. \hspace{1em}LerpollJver Gunderssen\hspace{1em} L. S. \hspace{1em}LangfiordMichel \textit{Nielsen}\hspace{1em} L. S. \hspace{1em}Stangnes\hspace{1em}\par
d. 1 \textit{Novemb.} var Hellig. ‒
\DivII[Nov. 2. Fra Leirpollbotn til Polmak]{Nov. 2. Fra Leirpollbotn til Polmak}\label{Schn1_94934}\par
2den Reiset fra \textit{Peder Nielsens Finn}, hans Bye paa Nordostlige Side af \textit{Leerpolls}Botten, over \textit{Leerpollen} i Søer med Baad til \textit{Samuel Finns} Bye paa Sydvestlige Side av \textit{Leer}- pollen{1/4 Miil} derfra i Kieredster med Reen for, i Vester til imod \textit{Guldholmen}{1/2 ‒} derfra i Søer til \textit{Bonjækas} paa Vestre \textit{Tana} Elv-Bræde, den 1te \textit{Norske Qvæn}{1/2 ‒ _______ Søe Mile 1 1/4 ‒}\hspace{1em}\par
Paa Vejen fra \textit{Samuels} Bye til imod \textit{Guldholmen} saae man paa Syd-vestlige Bræde af \textit{Leerpollen} adskillige smukke Eng-Sletter med Bierk paa, neere ved Søen, tienlige til Rødning og Bebyggelse\par
d. 3die Fra \textit{Bonjækas} i Syden til \textit{Storholmens} Søndre Ende {1/2 ‒} Derfra til \textit{Sørmaukie} det nedre knap {1/4 ‒} Derfra til \textit{Sørmaukie} det Øvre knap {1/4 ‒} saa til \textit{Lodjok}{1/4 ‒} siden til \textit{Moketveie}, den 2den \textit{Norske Qvæn} paa Østere Elv-Bræde {1/4 ‒ _______ Søe Miil 1 1/2 ‒}\hspace{1em}\par
Paa begge ElvBræder ere gode Sletter med Bierk paa, allveis til Rødning.\par
\textit{Qvænen} paa \textit{Moketveie} sagde, navnl. \textit{Clement Samuelsen}, i Paahør af Tolken \textit{Hælset} og den Fieldfinn \textit{Ole Pedersen}, at den Svenske \textit{Arisbye} Præst Ao 1743. med \textit{Sacramentet} har betient 3 Personer, og Ao 1744. 2de Dito i det \textit{Qvænske} Sprog, (som disse Folk ei andet forstaae) Hiemme hos dem.\hspace{1em}\par
d. 4de. Fra \textit{Moketveie} mest i Syd-Syd-vest til \textit{Bolma}, hvor 3die Norske \textit{Qvæn} i RøgTømmerstue boer, {1 1/4 SøeMil}\par
\textit{Qvænen} her, \textit{Hendrik} Eriksen, sagde i Tolkens, og Skolemesterens Overværelse, at \textit{Arisbye}-Præsten i for-Vinter betiente ham og hans Hustrue i deres Stue med \textit{Sacramentet}.\hypertarget{Schn1_95132}{}Schnitlers Protokoller V.
\DivII[Nov. 5.-7. Rettsmøte på Polmakjord i Polmak]{Nov. 5.-7. Rettsmøte på Polmakjord i Polmak}\label{Schn1_95134}\par
\textbf{Ao 1744. d. 5 Novbr:} holdte man paa \textit{Bolma}-Jord, den yderste og Søndreste Pladz af \textit{KiølleFiordz} Præstegield ved \textit{Tana}-Elv, \textit{Examinations}Rett med 3de \textit{Arisbye}-Field\textit{finner}, hvilke den \textit{Norske Missions}-Skolemester over de \textit{Norske Tana-Finner, Peder Tudesen, fra Arisbye} havde ført hid ned med sig; J samme Skolemesters og LaugRettens Overværelse blev Eeden af Lovbogen ei alleene for dennem, men og for en \textit{Varanger-Finn} fortolket, og de alle aflagde deres \textit{Corporlig} Eed, til at sige Sanden, angaaendes Grændse-Gangen:\par
\centerline{\textbf{26de Vidne i Finmarken,}\textit{Ole Pedersen Guttorm}}\par
født i \textit{Arisbye} af \textit{Finne}-Forældre, hiemme-døbt i en \textit{Finne}-Gamme i \textit{Avjevara}, meener at være mod 80. Aar gammel, gift, har 5 Børn, Ved sidst afvigte PaaskeTider været til Gudz Bord i den Svenske \textit{Arisbye} Kirke; er Fieldfinn, levendes af sine Reen, Lax-Fiskerie i den \textit{Norske}\textit{Tana}-Elv, og af andet Fiskerie i Field-Vandene; Om Sommeren sidder han paa Vestre Side af \textit{Tana}-Elv paa \textit{Norsk} Grund, om Vinteren imellem \textit{Otzjok} og \textit{Væke-jaure}.\hspace{1em}\par
Sp. 1. Hvor mange Field-\textit{Finner} i \textit{Arisbye} ere, som svare under den \textit{Svenske}\textit{Arisbye} Kirkes Meenighed?\par
Svar: De ere i alt, med de 2de \textit{Juxbye Finner}, som høre til \textit{Arisbye}-Meenighed, Rige og fattige {28 \textit{Familier}} og er \textit{Arisbye}-Kirke et \textit{Annex} under \textit{Koutokeino}-HovedKirke.\hspace{1em}\par
Sp. 2. Hvor disse \textit{Arisbye-Finner} sig om Sommeren opholde?\par
Svar: Af disse 28. \textit{Arisbye-Finner} sidde om Sommeren imellem de \textit{Norske}\textit{Laxe}- og \textit{Tana}Fiorder{5 [Familier]}\par
Paa Vestre Side af \textit{Tana}-Elv, fra Elv-Munden at reigne i Søer op imod \textit{Bolma} paa \textit{Norsk} Grund {13 ‒}\par
Oven for eller Sønden for \textit{Foss-Holmen} paa fælles Grund {8 ‒}\par
Oven for \textit{Juxbye} ved \textit{Tana}-Elv sidde de fra \textit{Juxbye}{2 ‒ _______ 28 \textit{Fam.}}\hspace{1em}\par
For Pintz-Tid, naar Sneen begynder at tøes, komme de hid Nord, og om Olsmiss, (sidst \textit{in Julio}) fare de herfra til \textit{Arisbye} igien; Medens de ere her i \textit{Norge}, besøges de ikke af de \textit{Norske Missions} Betientere, ei heller søge de til de \textit{Norske} Kirker, eller \textit{Missions} Skolemesterens Forsamlings Steder, hvor han holder Skole, uden naar \textit{Missionairen} ved \textit{St Hans} Tid nogle Søndager præker, eller Skolemesteren læser af \textit{Postillen} om Søndagen, da de nærmeste møde;\par
De søge ikke \textit{Norsk Ting}, enten for Trætters skyld imellem sig indbyrdes, eller for anden Forseelse, her i \textit{Norge} kan begaaes; Thi da søges de for den Svenske Rett i \textit{Arisbye}; Naar nogen døer i \textit{Norge}, bliver Arveskiftet holden af de Svenske Betientere i \textit{Arisbye}; De hielpe og ikke til med at giøre fri Skydz; Nærværende \textit{Missions} Skolemester tilstoed det og.\hypertarget{Schn1_95426}{}26de Vidne i Finmarken. Kiøllefiords Præstegield.\par
Sp. 3. Hvor \textit{Arisbye-Finner} sig om Vinteren opholde?\par
Svar: Om Vinteren sidde langs efter \textit{Otzjok}-Elv, paa begge dens Sider, fra samme Elvs Kæft i Søer op til \textit{Merisjaure inclus}: {9 \textit{Fam.}} herhos berettendes, at \textit{Otzjok} kommer Søndenfra \textit{Suøkas-vara}; hvor de dog ei ligge med deres Reen, men bruge det til VildReen-Skøtterie; Jkke ret langt Sønden for \textit{Otzjoks} Begyndelse er \textit{Saxe-jaure}, men hvor langt? veed ej. Dette Vand skal være 1/4 Miil langt, og et par Bøsse-Skud over bredt, og ligge Sønden for \textit{Merisjaure} imod 1 FieldMiil.\par
Ved \textit{Vækejaure} sidde {5 ‒}\par
Jmellem \textit{Vækejok} og \textit{Bolmajok} have for 3 Aar siden satt sig, som er nærmere Vores Grændser {3 ‒}\par
Jmellem \textit{Foss-Holmen} og \textit{Otzjok}-Munden opholde sig de fattige {8 ‒}\par
Ved \textit{Juxbye} paa Syd-ostlige Side af \textit{Tana}-Elv{2 ‒}\par
Ved \textit{Karasjok} paa Østre Side {1 ‒ _______ 28 \textit{Fam.}}\hspace{1em}\par
Her i \textit{Arisbye} blive de af den Svenske Præst, og den Svenske Øvrighed med Præstelig Tieneste og Retten plejede.\par
Han sluter, det er nu noget over 40. Aar siden denne \textit{Arisbye} Kirke først blev bygged af Furre-Timmer, hvortil \textit{Finnerne} maattet fremføre Timmeret uden Betaling; For den Tid blev der præket i Stue-Gammer af Præsten, som om Vinteren reiste fra \textit{Torne}-Stad, og efter forretted Tieneste foer did tilbage igien i Førstningen.\par
Som han har hørt, har Hr \textit{Anud Curtelius} fra \textit{Torne} Stad været den 1te løse ‒ det er: Omreisende Præst (som hos os kaldes \textit{Missionarius}) der har besøgt dem i \textit{Arisbye}, og har samme \textit{Curtelius} døbt Vidnet; For denne \textit{Curtelius} veed han ikke, ei heller har hørt, at nogen anden Svensk Præst, eller \textit{Missionarius} der i \textit{Arisbye} har været; Dette Vidne, som af \textit{Curtelius} er døbt, er, som meldt, mod 80 Aar gl. ‒\par
Efter \textit{Curtelius} Kom Hr \textit{Johannes Tornberg}, som stadig boede i \textit{Koutokeino}, og giorde Præstelig Tieneste i \textit{Koutokeino, Avjevara}, og \textit{Otzejok}, eller \textit{Arisbye}; Da han havde betient disse \textit{Finner} nogle Aar, fik han \textit{Siergeloft}-Kald, Norden for \textit{Torne}-Stad; og i denne Hr \textit{Johannes Tornberg} sin Tid var endda ikke nogen Kirke bygd, men Tienesten blev forretted i \textit{Finne}- Gammer;\par
Efter \textit{Johannes Tornberg} fuldte Hr. \textit{Anders Tornensis}; Han holdte og stadig til i \textit{Koutokeino}, og var mange Aar Præst hos dennem; J denne Hr. \textit{Anders Tornensis} sin Tid ere begge Kirker i een Vinter, nemlig den i \textit{Arisbye} først, siden den i \textit{Koutokeino} bleven bygde, for nogle og 40 Aar siden, og nogle faa Aar efter at Kirkerne vare bygde, døde denne Hr \textit{Anders Tornensis} i \textit{Koutokeino}.\par
Denne Hr \textit{Anders} blev efterfuldt i Kaldet af Hr \textit{Johannes Tornberg}, den forommeldte Hr \textit{Johannes} sin Søn; som han slutter var Præst hos dennem en 13 Aars Tid, da han siden fik \textit{Jokasjarfs} Kald i \textit{Torne-Lapmark}.\hypertarget{Schn1_95702}{}Schnitlers Protokoller V.\par
Hannem \textit{succederede} Hr \textit{Johannes Junelius}, nu værende Sognepræst til \textit{Koutokeino}, som har betient begge, baade \textit{Koutokeino} og \textit{Arisbye} Kirker, indtil for 3de Aar siden, da en \textit{Capellan} til \textit{Arisbye} er beskikked, navnlig \textit{Anders Hellander}, som holder tillige Skole nu for 4 \textit{Finne}- Børn, hvilke han hos sig haver i husse, samme \textit{Capellan} lønnes af Cronen. (\textit{vide} 4 \textit{Volumen}II 385.)\hspace{1em}\par
Marke- og Tingsteder, har Vidnet hørt, at den Svenske Øvrighed har holdet i \textit{Finmarken} længe for den 1te Præstes \textit{Curtelius} hans Tid; han har og hørt, at de med de Tingsteders Holdelse først have begyndt: men hvor længe det har været for \textit{Curtelius} hans Tid, det kan han ikke sige; Det har han og hørt, da de Svenske begyndte at holde Ting, da begyndte de tillige at optage Skatt;\par
Efter nøyere Tilspørsel svarede han: at det var vel for hans Faders-Faders Tid, at de Svenske med Ting og Skatten have giort Begyndelse.\par
Den Kongel. \textit{Norske} Foged har og alltid, i sin, og sin Faders og Farfaders Tid reiset om hos dennem, og optaget Skatten.\hspace{1em}\par
Sp. 4. Hvilke ere \textit{Arisbyes} Raamerke eller Grændseskiell til de \textit{Jndiagers district?}\par
Svar: De \textit{Arisbyers} yderste Steder ere\par
(1) foromrørte Field \textit{Suøkas}; Sønden for hvilket de \textit{Arisbyer} ej komme;\par
Sønden for \textit{Merisjaure} Een liden FieldMiil er\par
(2) \textit{Saxe-jaure}, dette Vand er et Raamerke imellem dem og de \textit{Jndiager-Finner}, saa at de \textit{Arisbyer} bruge den Nordre- og de \textit{Jndiager} den Søndre Deel deraf;\par
Af dette \textit{Saxe-jaure} løber Aaen \textit{Saxe-jok} i Søer, ved 1 Field-Miil lang i \textit{Gamas-jok}, og denne \textit{Gamas-jok} opkommer Vesten fra af det Field \textit{Røeter}, som han har hørt at være 2 FieldMile i Nord-Vest fra det Sted, hvor \textit{Saxe-jok} i \textit{Gamas} indløber, og at ligge fra \textit{Bais-duoder} ved en liden Dal, 1 Bøsseskud viid, i Søer adskilt.\par
\textit{Roëter}-Field skal være fra Nord i Søer 1 FieldMiil langt, og noget smalere i Bredden. Dette\par
(3) \textit{Roëter} er det yderste Field i Søer, som de \textit{Arisbyer} indehave, imod de \textit{Jndiager} deres Land.\par
Norden for dette \textit{Roëter} ligger, som meldt, \textit{Bais-duoder}, og herfra i Syd-vest 1/2 FieldMiil \textit{Niaug-oive}; Fra \textit{Niaug-oive} i Syd-vest \textit{Muedke-duoder}, hvoraf de \textit{Arisbyer} kun nyde den Nordre Ende; Og disse sidst benævnte Fielde bruge de \textit{Arisbyer} alleene til VildReen-Skøtterie.\par
\textit{Muedkeduoder} er slet ovenpaa, med smaa Skoug sommestedz, og med smaa Houger paa, meget stort, at, naar de ere paa den Nordre Ende, kan de ei oversee det til den Søndre Ende.\par
(4) \textit{Garadas-jok}, som han slutter, at være, 1/2 Miil Vesten for \textit{Roëter}, kommer op ved \textit{Moedkeduoders} Nordre Ende, imellem \textit{Niaugoive} og \textit{Moedke}, dog nærmere til denne, og løber i Nordvest, hvor langt? veed ei, i \textit{Øvre-Tana}-Elv, som herfra begynder at kaldes \textit{Jndiager}-Elv; Til denne \textit{Garadas-jok} gaaer de \textit{Arisbyers district}, og langs med den til \textit{Øvre Tana}-Elv; Sønden for \textit{Garadas} er Landet de \textit{Jndiagers} ‒\hypertarget{Schn1_95983}{}26de Vidne i Finmarken. Kiøllefiords Præstegield.\par
De \textit{Juxbye-Finner} have deres Brug op til denne \textit{Garadas};\hspace{1em}\par
Sp. 5. Hvilket er de \textit{Arisbyers} Raamerke imod \textit{Avjevara}.\par
Svar: Det vidste han ikke: Men efter Tilspørgende, forklarede, at de 3 Svenske \textit{Qvæner} sidde paa den Sydostlige Side af \textit{Karasjok}, ved 2 Mile Sønden for \textit{Karasjoks} Kæften, og imod 2 Mile Norden for \textit{Jetzjoks}-Kæften, saa der bliver imod 4 Mile imellem \textit{Karasjoks} Kæften, hvor denne i \textit{Tana}Elv udgaaer, og \textit{Jetzjoks} Kæften, hvor denne i \textit{Karasjok} indfalder.\par
Disse \textit{Karasjoks Qvæner} høre til \textit{Avjevara} under \textit{Koutokeino}-HovedKirke.\hspace{1em}\par
Sp. 6. Hvilket er Raamerke imellem de \textit{Arisbyer} og de \textit{privative Norske TanaFinner?}\par
Svar: Forrige Amtmand \textit{Lork} har vel villet have\par
(1) \textit{Foss-Holmen} til et Raamerke imellem \textit{Tana}- og \textit{Arisbye-Finner:} men han har hørt, at i gammel Tid de Norske \textit{Varangers}- og de fælles \textit{ArisbyeFinner} have brugt \textit{Skaare-jok}, hver andet Aar om hinanden, men nu bruger Jngen af dennem denne Aae; Og at denne \textit{Skaarejok} er Raamerke.\par
\textit{Skaare-jok} opkommer af \textit{Skaare-jaure}, og løber fra Nord vest i Sydost, hvor langt? veed ej, i \textit{Tana}Elv paa dens Vestlige Side 1/4 Miil oven for den \textit{Norske Qvæns} Jord-pladz \textit{Bolma}; og 3/4 Miil Norden for \textit{Storfossen} og \textit{Fossholmen};\par
\textit{Skaare-jaure} skal være 2 Bøsseskud langt fra Vester i Øster, halv saa bredt, hvorj Øreter fanges. Vesten for \textit{Skaarejaure} et Stykke ligger \textit{Laisduøder}\par
(2) Østen herfor bruge \textit{Arisbye Finner Vækejaure} og \textit{Vækejok:} derimod de \textit{Norske Tana}Field\textit{Finner} indehave \textit{Bolmajaure} og \textit{jok}, og gaae i Søer hen imod \textit{Væke-jaure}\hspace{1em}\par
Sp. 7. Hvad Skatt de til \textit{Sverrig} og ellers betale?\par
Svar: En fuld Skattmand betaler Skatt til \textit{Sverrigs} Crone 1 Rdl., til \textit{Koutokeino} Hoved Præst 40 s. til klokkeren 12 s. til Laugmand 24 s.\par
Til den Kongel. \textit{Norske} Fogd ligeledes 1 Rdl. i Skatt, saa at der svares lige meget i Skatt til begge Croner; En halv og mindre Skattemand svarer efter \textit{proportion} mindre;\par
De 8. Field\textit{finner}, som sidde imellem \textit{Storfossen} og \textit{Otzjoks} Kæften svarer ingen Skatt; thi de ere fattige, men dog betale deres Rettighed til den \textit{Svenske Koutokeino} Præst, efterdi de søge til \textit{Arisbye} som den nærmeste Kirke.\par
Beretter ellers om den Kongel. \textit{Norske} Fogdz Reise fra \textit{Jndiager} i Nordvest til \textit{Arisbye} at være i {2 Dage}\par
Fra \textit{Arisbye} i Syd-vest til \textit{Juxbye}{1 ‒}\par
Fra \textit{Juxbye} i S. V. til \textit{Karasjok} sterk {1 ‒}\par
Fra \textit{Karasjok} i Vester til Nord til \textit{Avjevara}{1 ‒}\par
Fra \textit{Avjevara} til \textit{Koutokeino} i S. S. V. {2 ‒ _______ 7 Dage}\par
Sp. 8. Hvad han om Grændse-Skiellet imellem \textit{Norge} og \textit{Sverrig} veed?\par
Svar: Han veed intet at sige om Grændse-Gangen imellem \textit{Norge} og \textit{Sverrig}; Thi de \textit{Arisbye-Finner} komme ikke længere op i Søer, end til de \textit{Jndiagers} Raamerker; og disse \textit{Jndiager-Finner} saavel som de af \textit{Avjevara} bruge de Fielde Sønden for de \textit{Arisbyer} alt op til \hypertarget{Schn1_96340}{}Schnitlers Protokoller V.\textit{Jauris-duoder}; derfor veed han intet af dette \textit{Jauris-duoder}, ei heller har han hørt noget om dette \textit{Jauris-duoder} eller noget andet Merke at giøre Grændse-skifte imellem \textit{Norge} og \textit{Sverrig}. Hvorpaa \textit{dimittered}.\par
\centerline{\textbf{27de Vidne i Finmarken,}\textit{Peder Jonsen Bidte.}}\par
født i \textit{Arisbye} af \textit{Finne}Forældre, døbt i \textit{Arisbye} Kirke, mod 60 Aar gl. gift, har 3 Børn, været i sidst afvigte Sommer i den \textit{Norske}\textit{Angsnæss} Kirke af \textit{Vadsøe} Præstegield hos Provsten \textit{Angell} til Gudz Bord; om Sommeren sidder han ved den \textit{Norske}\textit{Varanger} Fiord og nærer sig af Fiskerie, om Vinteren ved det Norske \textit{Bolma} Vand, iblant de \textit{Norske} Field\textit{Finner}; Været en \textit{Arisbye}-Fieldfinn indtil for 3 Aar siden, da han for Armod skyld derfra er fløtted, og er nu \textit{Norsk Finn} af \textit{Varangers} Præstegield.\hspace{1em}\par
Til Spørsmaalene fra 1te til 8de \textit{incl.} svarede ligesom næst forrige 26de Vidne i alle maader, undtagen\par
(1) Han er ikke døbt af Præsten \textit{Curtelius}.\par
(2) at \textit{Skaarejok} er fra \textit{Skaarejaure} til \textit{Tana}Elv ved 1 Field-Miil lang.\par
\centerline{\textbf{28de Vidne i Finmarken}\textit{Ole Mortensen}}\par
født i \textit{Jndiager}, eller paa \textit{Finnsk:}\textit{Anar} af \textit{Finne}-Forældre, og døbt sammestedz; 45 Aar gl., gift, har 1 Barn; Ved sidste Pintzetid været i \textit{Arisbye} Kirke til Gudz Bord; Da han var 17 Aar gl., begav han sig fra \textit{Jndiager} til \textit{Arisbye} at tiene; Hvor han siden den Tid har været, og omsider bleven sin eegen Mand, en Field-\textit{Finn}, om Sommeren liggendes Vesten ved \textit{Tana}- Elv, om Vinteren ved \textit{Væke-jaure};\par
Fra 1te til 8de Sp. \textit{incl:} giver han det samme Svar; som den sidste, det 27de Vidne.\par
\centerline{\textbf{29de Vidne i Finmarken}\textit{Niels Clementsen}}\par
født i \textit{Arisbye} af \textit{Finne}-Forældre, døbt i \textit{Arisbye} i en \textit{Finne}-Gamme, førend Kirken var opbygged; 50. Aar gl., gift, har 4 Børn, været ved sidste Kyndels-miss til Gudz Bord i \textit{Arisbye} Kirke; Om Sommeren har han siddet ved \textit{Varanger}Fiord; J afvigte Vinter været ved det \textit{Norske}\textit{Bolma}Vand; Før været en \textit{Arisbye} Field-\textit{Finn}, men for Armod skyld fløttet derfra.\par
Til alle Spørsmaale svarer det samme, som næstforrige 2de sidste Vidner.\par
Og dermed blev Retten paa dette Sted slutted.\par
\textit{Bolma} ved \textit{Tana}-Elv d. 7 \textit{Novbr.} 1744.\hspace{1em}\par
\centerline{L. S. Niels Larsen i \textit{Tana Fielde}}Peter Schnitler\hspace{1em}\centerline{L. S. Niels Olsen i \textit{Tana Fielde}}\hypertarget{Schn1_96634}{}Forskjellige ekspeditioner. Til Vardøe Præstegield.\par
d. 8 \textit{Novbr.} var hellig.\par
d. 9 \textit{Novbr.} 1744. Sendt Tolken Erik \textit{Hælset}, med \textit{Missions} Skolemestern et Støkke op efter \textit{Tana}-Elv, at erfare, om den havde sit Løb, efter Vidnernes Udsagn; og befandtes den fra \textit{Bolma}, hvor vi holdte til, at vende sig j NordVester {1/4 Miil} derefter i Sydvest til \textit{Fossen}{3/4 Miil}\par
Her hvor Elven snoer sig fra Nordvest i Sydvest, der Løber \textit{Skaare}-Aaen, som \textit{ArisbyeFinner} vill have til Raamerke imellem sig, og de Norske \textit{Tana-Finner}.\par
Fra Fossen stikker den sig i Nordvest {1/4 ‒} Saa i Sydvest {1/4 ‒} derefter i Nordvest igien {3/4 ‒} Endelig i Sydvest til \textit{Otzjok}-Kæften{1 ‒ _______ = 3 1/4 Mil.}\par
‒ skrevet samme Dag til Amtmand \textit{Kieldsøn} ang. den \textit{Delinqvent Finn, Morten Olsen}, at forestilles til Kongl. allern: \textit{Moderation}, for i Nødzfall at bruge ham, som Vejviisere i Grændsegangen, hvorj han sagde sig at være kyndig.\par
‒Givet Hr Obriste \textit{Mangelsen Relation} om min Forretnings Fremgang.\par
‒\textit{Communiceret} de Kongl. Norske Hrr GrændseMaalere, hvorvidt jeg med mine \textit{Examiner} paa \textit{Kiølen} er kommet, og i sær til Hr \textit{Commissions Secretairen} berettet, at jeg heri \textit{Finmarken} har faaet bevidnet, at de Svenske fra \textit{Tornestad} først have begynt deres \textit{Mission} i \textit{Finmarken} for omtrent 80. Aar, og bygget de første \textit{Kautokeino}- og \textit{Arisbye} Kirker for nogle og 40 Aar siden.\par
Og som herpaa haft Vidner ligeledes i \textit{Tromsøens} Fogderie i \textit{Nordland}, saa skrev\par
‒\textit{dito dato} til Fogden og Sorenskriverne over \textit{Tromsøe} og \textit{Sennien}, at de Vidner, som derom have \textit{deponeret}, ved de til Grændse-Maalingen \textit{committerede Officerer} deres Ankomst, til Grændsen maatte føres.\par
10 \textit{Novbr:} til \textit{Heike Qvæn} paa \textit{Bolma} levert en skriftlig \textit{ordre} til den beskikkede \textit{Norske} Lensmand i den \textit{Svenske Finne}Bye \textit{Juxbye} over de fælles \textit{ArisbyeFinner}, at tilsige de 2de \textit{ArisbyeFinner}, 26de og 28de Vidner, at møde om 1 eller 2 Aar de Kongelige \textit{Norske} og \textit{Svenske} Grændse-Maalere ved deres Ankomst paa Grændsen, at forklare der deres Vidne; Om Underholdning paa den Vej have de at melde sig hos den Kongel. Amtmand eller Foged i \textit{Finmarken}\hypertarget{Schn1_96849}{}Forskjellige ekspeditioner. Til Vardøe Præstegield.\par
d. 8 \textit{Novbr.} var hellig.\par
d. 9 \textit{Novbr.} 1744. Sendt Tolken Erik \textit{Hælset}, med \textit{Missions} Skolemestern et Støkke op efter \textit{Tana}-Elv, at erfare, om den havde sit Løb, efter Vidnernes Udsagn; og befandtes den fra \textit{Bolma}, hvor vi holdte til, at vende sig j NordVester {1/4 Miil} derefter i Sydvest til \textit{Fossen}{3/4 Miil}\par
Her hvor Elven snoer sig fra Nordvest i Sydvest, der Løber \textit{Skaare}-Aaen, som \textit{ArisbyeFinner} vill have til Raamerke imellem sig, og de Norske \textit{Tana-Finner}.\par
Fra Fossen stikker den sig i Nordvest {1/4 ‒} Saa i Sydvest {1/4 ‒} derefter i Nordvest igien {3/4 ‒} Endelig i Sydvest til \textit{Otzjok}-Kæften{1 ‒ _______ = 3 1/4 Mil.}\par
‒ skrevet samme Dag til Amtmand \textit{Kieldsøn} ang. den \textit{Delinqvent Finn, Morten Olsen}, at forestilles til Kongl. allern: \textit{Moderation}, for i Nødzfall at bruge ham, som Vejviisere i Grændsegangen, hvorj han sagde sig at være kyndig.\par
‒Givet Hr Obriste \textit{Mangelsen Relation} om min Forretnings Fremgang.\par
‒\textit{Communiceret} de Kongl. Norske Hrr GrændseMaalere, hvorvidt jeg med mine \textit{Examiner} paa \textit{Kiølen} er kommet, og i sær til Hr \textit{Commissions Secretairen} berettet, at jeg heri \textit{Finmarken} har faaet bevidnet, at de Svenske fra \textit{Tornestad} først have begynt deres \textit{Mission} i \textit{Finmarken} for omtrent 80. Aar, og bygget de første \textit{Kautokeino}- og \textit{Arisbye} Kirker for nogle og 40 Aar siden.\par
Og som herpaa haft Vidner ligeledes i \textit{Tromsøens} Fogderie i \textit{Nordland}, saa skrev\par
‒\textit{dito dato} til Fogden og Sorenskriverne over \textit{Tromsøe} og \textit{Sennien}, at de Vidner, som derom have \textit{deponeret}, ved de til Grændse-Maalingen \textit{committerede Officerer} deres Ankomst, til Grændsen maatte føres.\par
10 \textit{Novbr:} til \textit{Heike Qvæn} paa \textit{Bolma} levert en skriftlig \textit{ordre} til den beskikkede \textit{Norske} Lensmand i den \textit{Svenske Finne}Bye \textit{Juxbye} over de fælles \textit{ArisbyeFinner}, at tilsige de 2de \textit{ArisbyeFinner}, 26de og 28de Vidner, at møde om 1 eller 2 Aar de Kongelige \textit{Norske} og \textit{Svenske} Grændse-Maalere ved deres Ankomst paa Grændsen, at forklare der deres Vidne; Om Underholdning paa den Vej have de at melde sig hos den Kongel. Amtmand eller Foged i \textit{Finmarken}\hspace{1em}
\DivII[Nov. 11.-12. Fra Polmak til Veines i Nesseby]{Nov. 11.-12. Fra Polmak til Veines i Nesseby}\label{Schn1_97065}\par
11te 9\textit{br.} gav man sig paa Reisen i \textit{Finnernes} Kieredser, fra \textit{Bolma} i NordNordost paa Jisen til \textit{Gollevara}{1/2 Søe Miil} saa over \textit{Gollevara}Field dets Nordre Ende, som var begroet med Bierk, i Øster {1/4 ‒} Siden over en Myr- og Bierkedal med smaa Bakker i, kroged i OstNordost, for at komme imellem \textit{Raude}- og \textit{Sode}-Fielde at fare {1/4 ‒} derfra imellem bemeldte Fielde over Myr- og Bierkedal, med smaa Houger i, i Ost-Syd-ost til \textit{Varanger}-fiordz Østre Botten {1 ‒ _______ = 2 Søe Mile}\hypertarget{Schn1_97115}{}Schnitlers Protokoller V.\label{Schn1_97117} \par 
\begin{longtable}{P{0.7680722891566265\textwidth}P{0.06485943775100401\textwidth}P{0.01706827309236948\textwidth}}
 \hline\endfoot\hline\endlastfoot d. 12te fra Østbotten til Ost-Syd-ost til \textit{Veinæss}-Land, som skiller Østbotten fra \textit{Veinæss}Fiord\tabcellsep 1/4 ‒\\
Over \textit{Veinæss}Land gaaet til \textit{Veinæss}- eller-\textit{Søer}fiord, hvor \textit{Finne} Lensmand boer\tabcellsep 1/4 ‒\\
\tabcellsep _________\\
\tabcellsep \tabcellsep 1/2 ‒\end{longtable} \par
 \par
d. 13de giordt Bud efter Vidner og LaugRetten.\par
d. 14 Ankomme de\par
d. 15. Hellig ‒
\DivII[Nov. 16.-des. 8. Rettsmøte på Veines]{Nov. 16.-des. 8. Rettsmøte på Veines}\label{Schn1_97193}\par
d. 16de Satt Retten i Overværelse af \textit{Varangers Missionaire} Hr \textit{Jæger} og \textit{Missions} Skolemesteren \textit{Niel Peersen, item} Lensmanden \textit{Hendrik Mathisen}, samt de af ham beskikkede LaugRettes Mænd; Den Kongelige \textit{Ordre} til denne Act blev offentlig forkynded, og Eedens Forklaring af Tolken Erik \textit{Hælset} for Vidnerne betyded, som derpaa aflagde deres Corporlig Eed:\par
a. \textit{Vidner}, som om det Nordreste eller yderste Land, nemlig Norden for og om \textit{Varanger}-Fiord, af \textit{Vaardøe} Præstegield vidste Beskeeden, vare\par
\centerline{\textbf{30te Vidne i Finmarken}\textit{Jver Nielsen Vind, Norsk Varanger} Søefinn}\par
fødd i \textit{Alten} i \textit{VestFinmarken} af BønderForældre, 63 Aar gammel, har 4 Børn, næret sig stedse i \textit{Finmarken} som Søefinn, for 14 Dage siden været til Gudz Bord.\par
\centerline{\textbf{31te Vidne i Finmarken,}\textit{Jon Nielsen Akkepeis, Norsk Varanger} SøeFinn,}\par
fødd og døbt i \textit{Koutokeino}-fælles\textit{Finmarken}, hvorfra fløttet ganske ung hid til \textit{Varanger}-Fiord, hvor han er SøeFinn, 51 Aar gammel, havendes 1 Barn; ved sidste Mikkelsmiss været til Gudz Bord.\par
\centerline{\textbf{32te Vidne i Finmarken,}\textit{Sabbe Olsen, Norsk Varanger} Søe-\textit{Finn}, }\par
fødd af \textit{Finne}Forældre, og døbt i \textit{Tanen}, i \textit{Ost-Finmarken}, 33 Aar gammel, u-gift, stedse næret sig, som SøeFinn, Ved sidste \textit{St Hans} Tider gaaet til Gudz Bord.\hspace{1em}\par
b. De som kiende Landet Sønden for \textit{Varanger}-Fiord, imod \textit{Arisbye} og \textit{Neidens} fælles \textit{Finmarken}, ere:\par
\centerline{\textbf{33te Vidne}\textit{Anders Poulsen}, nu \textit{Norsk Varanger}-Søe-Finn,}\par
fødd, og døbt i \textit{Varanger}, før været en \textit{Norsk Varanger} Field-Finn, nu Søe-Finn, 56 Aar gammel, gift, har 4 Børn, ved sidste \textit{Helge}miss været til Gudz Bord.\hypertarget{Schn1_97393}{}Vardøe Præstegield.\par
\centerline{\textbf{34te Vidne i Finmarken}\textit{Gunder Poulsen, Norsk Varanger} SøeFinn,}\par
fødd og døbt i \textit{Varanger}, 42 Aar gammel, gift har 2 Børn, stedse næret sig af Søen, ved Helgemiss været til Gudz Bord.\par
\centerline{\textbf{35te Vidne i Finmarken}\textit{Sabbe Minnesen, Norsk Varanger} SøeFinn,}\par
fødd og døbt i \textit{Varanger}Fiord, 69 Aar gammel, gift, har 3 Børn, stedse næret sig som SøeFinn her \textit{communiceret} ved sidste Mikkelsmiss.\par
\centerline{\textbf{36te Vidne i Finmarken}\textit{Peder Andersen, Norsk} Field-Finn}\par
fødd, og døbt i \textit{Varanger}, 62 Aar gl. gift, har 4 Børn, stedse været Field-Finn omkring \textit{Varanger}-Fiord, og ved St \textit{Hans} Tid gaaet til Gudz Bord.\par
\centerline{\textbf{37te Vidne i Finmarken}\textit{Mathis Mathisen}, nu \textit{Norsk Varanger} SøeFinn,}\par
fødd, og døbt i \textit{Jndiager}, fælles \textit{Finmarken}, derfra i sin Barndom kommet hid til \textit{Varanger}, og næret sig først som FieldFinn, siden udarmed, som Søe-Finn, 60 Aar gl., gift, har 1 Barn, ved sidste Mikkelsmiss \textit{communiceret}.\par
\centerline{\textbf{38. Vidne i Finmarken}\textit{Salomon Mathisen}, nu \textit{Norsk Varanger} SøeFinn, }\par
fødd, og døbt i \textit{Jndiager}, som et Barn fløttet derfra, først været Norsk \textit{Varanger} FieldFinn, siden udarmed, SøeFinn 44 Aar gl. gift, har 1 Barn, ved Mikkelsmiss været til Gudz Bord.\par
\centerline{\textbf{39te Vidne i Finmarken}\textit{Peder Tudesen, Norsk Varanger} SøeFinn,}\par
fødd, og døbt i \textit{Varanger}, stedse været her Søefinn, 60 Aar gl., gift, uden Børn, \textit{communiceret} ved Mikkelsmis.\par
\centerline{\textbf{40de Vidne i Finmarken}\textit{Peder Minnesen, Norsk Varanger} Søe-Finn,}\par
fødd, og døbt i \textit{Varanger}, stedse næret sig af Søen, 66. Aar gammel, gift, har 3 Børn, ved Helgemiss sidst gaaet til Gudz Bord.\hypertarget{Schn1_97582}{} Schnitlers Protokoller V.\par
\centerline{\textbf{41de Vidne i Finmarken}\textit{Morten Pedersen, Varanger} Søe og Fieldfinn,}\par
fødd, og døbt i \textit{Jndiager}, siden han var 9 Aar gammel, været paa \textit{Norsk}\textit{Varanger}-Grund, han selv nærer sig som Søe-Finn, hans Kone vogter Reen til Fieldz om Sommeren, han er 53 Aar gl., har 5 Børn, og været til Gudz Bord ved Pintze Tider.\par
\centerline{\textbf{42de Vidne i Finmarken}\textit{Ole Olsen Mind, Norsk Varanger} SøeFinn,}\par
fødd, og døbt i \textit{Varanger}, 33 Aar gl. gift, har 1 Barn, næret sig stedse, som SøeFinn, ved Mikkelsmiss gaaet til \textit{Communion}.\hspace{1em}\par
c. De som vidne om \textit{Kiølen}, Sønden for fælles \textit{Finmarken}, ere:\par
\centerline{\textbf{43 Vidne i Finmarken}\textit{Hendrik Hendriksen}, nu \textit{Norsk Varanger} SøeFinn,}\par
fødd i \textit{Kittel}, Svensk Markested i \textit{SverrigsKimi-Lapmark} af \textit{Finne}-Forældre, døbt i Jern-BrugsKirken ved \textit{Torne}-Elv, i \textit{Sverrig} udarmed, er han i forleden Aars Sommer hid til den \textit{Norske}\textit{Varangers} Fiord kommet, og nærer sig, som SøeFinn, 67 Aar gl., gift, har 6 Børn, været i den Norske \textit{Angsnæss Capell} ved sidste Helgemiss til Gudz Bord.\par
\centerline{\textbf{44de Vidne i Finmarken,}}\par
\textit{Ole Samuelsen}, nu \textit{Norsk Varanger} SøeFinn, fødd af \textit{Finne}-Forældre i \textit{Jndiagers} Bye, eller paa \textit{Finnsk: Anar-Sidast}, og sammestedz døbt, næret sig før, som FieldFinn, i \textit{Jndiager}, af Reen, og været Een af dem, som haft sin Sommerbye ved \textit{Øvre-Tana}-Elv, og derj fisket Lax, men for A[r]mod skyld, i sidstafvigte Vinter, hid til \textit{Varanger} nedfløttet, og nærer sig, som Søe-Finn, 40 Aar gammel, gift, har 3 Børn, J sidste Sommer \textit{communiceret} i den Norske \textit{Angsnæss Capell}.\hspace{1em}\par
d. Om Landskabet Norden og Sønden for \textit{Kiølen} have udsagt:\par
\centerline{\textbf{45de Vidne i Finmarken}\textit{Thomas Pedersen}, nu \textit{Norsk Varanger} SøeFinn,}\par
fødd i den \textit{Svenske Sombye} i \textit{SverrigsKimi-Lapmark}, døbt i \textit{Siaadekilla}, et Svensk Markested sammestedz; Fra \textit{Sombye}, hvor han har boet, er han for Armod skyld, i forleden Vinter hid over til \textit{Varanger} fløttet, og nærer sig, som SøeFinn, 54 Aar gammel, gift, har 3 Børn, ved sidste Helgemiss været til Gudz Bord i den \textit{Norske Angsnæss Capell}.\hypertarget{Schn1_97816}{}Vardøe Præstegield.\par
\centerline{\textbf{46de Vidne i Finmarken}\textit{Peder Johannsen}, nu \textit{Norsk Varanger} SøeFinn,}\par
fødd i \textit{Sombye} i \textit{SverrigsKimi-Lapmark}, døbt sammestedz, og boet der, indtil i Vinter for eet Aar siden, han formedelst Fattigdom derfra til \textit{Varanger} er nedfløttet, og nærer sig her som Søe-Finn, 57 Aar gammel, gift, har 3 Børn, i sidste Sommer gaaet i det \textit{Norske Angsnæss Capell} til Gudz Bord.\hspace{1em}\par
a. Foretages da først det Nordreste Land af \textit{Vaardøe} Præstegield, af de 3de første Vidner forklaret, i følge de brugte Spørsmaale, som fra \textit{pag.} 221 f. indførte:\hspace{1em}\par
til Sp. 1. Svar: Denne \textit{Veinæss}Fiord, eller \textit{Sørfiord}, hvor Retten holdes, svarer under \textit{Vaardøe} Præstegield\textit{Angsnæss Finne-Capell;}\par
\centerline{\textbf{Vaardøe-Præstegield} har}\par
1) \textit{Vaardøe} Hoved-Kirke af Træ, paa Øen af samme Navn; Derunder svare\par
2) \textit{Makurf-Capell}, 4 Mile Vesten fra HovedKirken, paa det faste Land imod Nord-Søen\par
3) \textit{Kiberg-Annex}Kirke af Træ, ved 3/4 Miil i Syd-Syd-vest fra \textit{Vaardøe}-HovedKirke, liggendes paa Østre Side dens Søndre Ende af \textit{Varangers} Næss, som er imellem \textit{NordSøen} og \textit{Varanger}Fiord,\par
4) \textit{Vadsøe} Kirke af Træ, inde i \textit{Varangers} Fiord paa den Søndre Side af \textit{Varanger}-Næssets faste Land, Vesten for \textit{Kiberg Annex} Kirke 4 Mile; Tilforn har den staaet paa den liden Øe \textit{Vadsøe}, fra det faste yttre Land ved et kort Sund adskilt: men er derfra over Sundet forfløtted ind paa det faste Land, der kalded \textit{Skattøren}, og har siden dog beholdet sit forrige Navn;\par
Efterdi Styrken af Meenigheden boer her inde i Fiorden, boer Sognepræsten ved denne \textit{Vadsøe} Kirke; Herunder \textit{sorterer}\par
5) \textit{Angsnæss Finne-Capell}, af Træ, 3 Mile Vesten for \textit{Vadsøe} Kirke, inde ved FiordBotten, bygged paa det Næss, \textit{Angsnæss}, som stikker fra det Vestre faste Land ud i Øster.\par
\textit{Makur-Capell} er fra \textit{Kiøllefiords} Præstegieldz nærmeste \textit{Omgangs Annex} Kirke i Ost- Syd-ost 4 Mile;\par
Fra samme Præstegieldz \textit{Tana-Capell} er \textit{Angsnæss} i Syd-Syd-ost 3 1/2 Mile.\label{Schn1_98029} \par 
\begin{longtable}{P{0.6058510638297872\textwidth}P{0.054255319148936165\textwidth}P{0.09946808510638298\textwidth}P{0.09042553191489361\textwidth}}
 \hline\endfoot\hline\endlastfoot \Panel{}{label}{1}{l}\tabcellsep \Panel{Normænd}{label}{1}{l}\tabcellsep \Panel{Søe-\textit{finner}}{label}{1}{l}\tabcellsep \Panel{Field-\textit{fin}}{label}{1}{l}\\
\textit{Vaardøe} Kirke har til Meenighed\tabcellsep 14\tabcellsep \tabcellsep \\
\textit{Makurf Capell}\tabcellsep 8\tabcellsep \tabcellsep \\
\textit{Kiberg-Annex} Kirke\tabcellsep 10\tabcellsep \tabcellsep \\
\textit{Vadsøe} Kirke\tabcellsep \tabcellsep 130\tabcellsep 14\\
\textit{Angsnæss Capell}\end{longtable} \par
 \par
Dette \textit{Vaardøe} Præstegieldz Strekning er saaledes:\par
Den 1te Jordpladz eller \textit{Finne}bye paa Vestre Side i Søndre Enden er \textit{Veinæss}Fiordz Pladser, fra næste \textit{Kiøllefiords} sidste Jordpladz, \textit{Bolma} ved \textit{Tana}Elv 1 1/2 Søe Miil i Øster til Norden;\hypertarget{Schn1_98127}{}Schnitlers Protokoller V.\par
Den 1te Jordpladz paa Vestre Side i Nord er \textit{Makur} Jordpladz paa Nordre Side af \textit{Varanger}Næssets faste Land ud til Havet, fra næste \textit{Kiøllefiords} sidste Jordpladz \textit{Omgang} i Ost- Sydost 4 Mile, fra forbem.te \textit{Veinæss}Fiord i Nordost 5 1/2 Miil;\par
Den sidste Jordpladz paa Østre Side i Sønden er \textit{Skoggerøe}-Pladser, fra \textit{Veinæss} Jordpladser i Øster til Sønden (Hvortil er Naboer de \textit{Russisk-Norske Neidens Finner}) ved{3 SøeMile.}\par
\textit{Vaardøe} Pladser paa Østre Side i Nord ere de sidste, fra \textit{Skoggerøe}-Pladser i Nordost over {5 SøeMile,} og fra \textit{Makur}-Pladser i Ost-Syd-ost {4 SøeMile.}\hspace{1em}\par
2 Sp. Svar: \textit{Vaardøe} Præstegield har vel smaa Øer og Holmer inde i Varangerfiord, men ingen udenfor til Havs, uden \textit{Vaardøe}, det øfrige er fast Land.\par
\centerline{\textbf{Vaardøe}}\par
fra det Østre faste Land, og især fra \textit{Varanger}-Næssetz mellemste Næss, \textit{Svartnæss}, som ligger lige oven for, 1/8 Miil i Øster, ved \textit{Bussøe} Sund derfra adskilt;\par
Er den Østerste Øe, som af \textit{Normænd} indehaves i \textit{Finmarken}, Østen for det yttere eller Nordre faste Landz Næss, \textit{Varanger-Næss;} hvilket paa sin Østre Ende haver 3de Særdeles Næsse fra Søer i Nord, navnlig \textit{Kiberg}, i Søer, \textit{Svartnæss} mitt paa, og \textit{KavringNæss} i Nord; Jmod det mellemste \textit{Svartnæss} ligger \textit{Vaardøe} uden for i Øster;\par
Dette \textit{Vaardøe} er paa sin Vestre Side 1/2 Miil lang fra Søer i Nord, og paa Sin Østre Side 3/8 Miil i samme Strekning; Tvert over fra Vester i Øster, der hvor Eidet er, kan den være 1/4 Miil over breed; \textit{Vaardøeen} indskiæres af 2de Vaage, den 1ne fra Søer i Nord, kaldes den \textit{Søndre Vaag;} den 2den fra Nord i Søer gaaer ind imod den \textit{Søndre Vaag}, med det Navn \textit{Vestre Vaag}, (fordj den i Gabet vider sig ud ad Nord-vest).\par
Disse 2de Vaage giøre nu et smalt Eid, hvorved den Vestre Deel af \textit{Vaardøe} med den Østre sammenhænger; Den \textit{Søndre Vaag} gaaer ind i Nord, 1/16 Miil lang, 2 Riffelskud og mindre over breed; Den \textit{Vestre Vaag} stikker i Søer, ved 3/8 Miil lang, imod 1/4 Miil i Gabet, imellem \textit{Skagen-Næss} og \textit{Asell-Næss}, viid, siden ind ad smalere.\par
Eidet imellem begge Vaage er 1 kort Bøsseskud over bredt, i Nord, og ved 2 Riffelskud over langt fra Vester i Øster.\par
Disse 2de Vaage skifte \textit{Vaardøen} i 2de, nemlig \textit{Vestre}- og \textit{Østre} Parter:\par
Den Vestre Part kaldes \textit{Bussøe-Sund}, fordi den ligger langs med \textit{Bussøe} Sund; Dens Søndre Næss heder \textit{Galge-Næss}, og det Nordre Næss \textit{Skage;} Denne Vestre Part af \textit{Vaardøen} har Fæstningen \textit{VaardøeHuus}, Vesten for Eidet, nær ved \textit{Bussøe-Sund;} Denne Vestre Part er uden Skoug, slet med Græss.\par
Den Østre Part har det Navn, \textit{Vaarberget}, berged, med Græss paa, uden Skoug, dens Nordre Næss er \textit{Asell-Næss}, det Søndre, \textit{Guldring-Næss;} Her staaer \textit{Vaardøe} Kirke, imod Eidet i Øster. Paa Eidet ere Handels-Hussene, og boe \textit{Normænd} i Stue-Gammer.\par
\textit{Bussøe-Sund}, hvorved \textit{Vaardøe} fra det faste Land adskilles, er ligesaalangt, som \textit{Vaardøen}, nemlig 1/2 Miil fra Søer i Nord, og 1/8 Miil bredt; hvor vel Skibe om Sommeren kan ligge, men for den sterke Strøm, der gaaer, ingen Vinter-Havn er.\hypertarget{Schn1_98409}{}Vardøe Præstegield.\par
\textit{Holmer} under \textit{Vaardøe} ligge\par
a. \textit{Tyvholm} i \textit{Bussøe-Sund}, imellem Øens Vestere Partes Nordre Deel og det faste Land, dog nærmere til dette, rund, 1 Riffelskud over stor, uden Skoug, slet med Torv-jord, og lidet Græss paa, u-beboed.\par
b. \textit{Reen}-øe, Norden for \textit{Asell-Næss} 1/16 Miil, lang fra Nord-vest i Syd-ost mod 1/4 Miil, halv saa breed, slet med Moese indenpaa, og med lidet Græss ude paa Søe-Brædene, uden skoug; Her faaes Fugle-Egg, og noget lidet Eeder-Duun, u-beboed.\par
c. \textit{Hornøe}, i Ost-Syd-ost fra \textit{Reenøe} liggendes 1 Bøsseskud over, er klipped, uden Skoug, og Græss, dog med Torv-Jord paa, mest rund, 1/4 Miil omkring stor.\hspace{1em}\par
\textit{Øer} og \textit{Holmer} inde i \textit{Varangers}Fiord, ved den Nordre Side af det faste Land, kaldet \textit{Varanger-Næss:}\par
1. \textit{Lill-Ekkerøe}, 1/4 Miil Sønden for det Nordre faste Land, og i sær Sønden for \textit{Trampenæss}, fra \textit{Vaardøe} i Syd-vest 3 1/2 Søe Mile, er fra Nord vest i Syd-ost 1/8 Miil lang, halv saa breed, slet med Torv-Banker, og lidet Græss paa, uden Skoug, nu u-beboed, som Sommeren sidde Folk fra \textit{Stor-Ekkerøe}, som fiske, derpaa.\par
2. \textit{Stor-Ekkerøe}, fra \textit{Lill-Ekkerøe} i Vester 1/8 Miil, fra det Nordre faste Land, nemlig \textit{Varanger-Næss}, og i sær fra \textit{Sollnæss}, et par Bøsseskud i Søer, med hvilket faste Land det sammenhenger ved en Val, som ved høy Flod overflydes, fra det Søndre faste Land, og især fra \textit{Holmgraa-Næss}, 2 Søe Mile over Fiorden i Nord, fra Vester i Øster 1/4 Miil lang, halv saa breed, slet med Græss og Torv-jord paa, uden Skoug, beboed af \textit{Normænd}.\par
3. \textit{Stor-Vadsøe}, fra \textit{Stor-Ekkerøe} i Vester 1 Miil, fra det Nordre faste Land, nemlig \textit{Varanger-Næss}, og især fra \textit{Skattøren} 2 Bøsseskud over et Sund i Søer, fra det Søndre faste Landz \textit{Bugøe} over Fiorden 1 Miil i Nord, er fra Vester i Øster 1/8 Miil lang, halv saa breed, slet med Græss, og indentil med Lyng begroed, u-beboed, bruges af \textit{Normænd} til Høe-Slotte,\par
Paa denne Øe har Kirken før staaet, men for en 30 Aar derfra fløtted, ind paa det faste Land, \textit{Varanger-Næss}, og i sær paa \textit{Skattøren}, Nordenfor \textit{Vadsøe}, nær ved Sundet;\par
Dette Sund imellem \textit{Vadsøe} og \textit{Skattøren} kaldes \textit{Vadsøe-Sund}, fra Vester i Øster saa langt, som \textit{Vadsøe}, et par Bøsseskud over bredt, giør en Vinter-Havn for nogle Skibe.\par
4. \textit{Lill-Vadsøe}, fra \textit{Storvadsøe} i Vester 1/8 Miil, fra det Nordre faste Land, nemlig \textit{Varangernæss}, og i sær fra \textit{Laxebøe-Næss} 1/16 Miil i Søer, fra det Søndre Landz \textit{Bugøe} ligeledes omtrent 1 Miil, er fra Vester i Øster et par Bøsseskud lang, ikke fuld saa breed, slet med Græss og TorvMark paa, uden Skoug, u-beboed; \textit{Normænd} bruge den til Torv-skiæring, og at sætte Faar derpaa.\par
5. \textit{SkiaaHolmen}, 2 1/2 SøeMiil Vesten for \textit{Vadsøe}, liggendes i \textit{Varanger}fiordz\textit{Østre Botten} ved dens Nordre Land-Side, er fra Syd-vest i Nord-ost 1/8 Miil lang, i hver Ende ved 2 Riffelskud, ‒ men mitt paa 1 Riffel-skud over breed, flad, mest steened, med noget Lyng og Græss paa, uden Skoug, u-beboed, bruges af Næst-Boende til at Sætte Faar derpaa om Sommeren.\hspace{1em}\par
\textit{Øer og Holmer} inde i \textit{Varangerfiord}, ved den Søndre Side af det faste Land, kalded \textit{RafteSiden:}\par
6. \textit{Svinøe}, 2 5/8 Miil fra \textit{Østre Bottens} Bond i Øster til Sønden, fra det Søndre faste Land, imod det \textit{Næss Aka-Niarg}, 2 Bøsseskud i Nord, er 2 Bøsseskud lang, halv saa breed, steen\hypertarget{Schn1_98653}{}Schnitlers Protokoller V. berged, mest bratt ud ad Fiorden, deels flad deels houged ovenpaa, med noget Græss og Lyng paa, uden Skoug, u-beboed; \textit{Normænd} i \textit{Bugøe} bruge den at sætte Faar derpaa om Sommeren.\par
7. \textit{Bugøe}, en knap 1/8 Miil fra \textit{Svinøe} i Øster til Sønden, fra det Søndre faste Landz \textit{BugøeNæss} i Nord-ost over 1 RiffelSkud, er i Nordost ved 3 Bøsseskud lang, og hvor videst, 2 Bøsseskud over breed, slet med smaa Houger paa, paa Østre og Søndre Sider noget steened og undlænded, uden Skoug, med Myr- og Senni-Græss ellers begroed, u-beboed; dog ligge om Sommeren \textit{Norske Finner}, som fiske der paa Laug i Fiorden, paa denne Øe i deres Gammer, og de nærmeste \textit{Normænd}, som boe paa \textit{Bugøe-Næss} paa det Søndre faste Land, slaae Græss derpaa, og tage Brænde-Torv derfra.\par
8. \textit{Kiøholm}, fra \textit{Bugøe} i Øster til Syden næsten 1/2 Miil, fra det Søndre faste Land, og i sær fra det Vestre Næss af \textit{Kiøfiord}, navnlig \textit{Brasshavn} 4 Riffel Skud i Nord afliggendes er fra Søer i Nord 2 à 3 Bøsseskud lang, paa Søndre Ende 1 Bøsseskud og på Nordre Ende 2 Bøsseskud breed, flad med Græss og Lyng paa, uden Skoug, sommestedz lidet steened. ‒\par
Denne \textit{Kiøholm} bruges ikke af \textit{privative-Norske} Undersaattere, men om Sommeren af \textit{Neidens}- som ere \textit{Russisk-Norske} fælles \textit{Finner}, saaledes at de om Paasketider komme fra deres Field-Byer ned ad Fiorden at fiske, og ligge Sommeren over paa denne \textit{Kiøholm} i deres Gammer, havendes nogle Faar med sig, som fødes paa Holmen, og om Mikkelsmiss fare de fra Øen og Fiorden op igien til deres Field-Byer.\par
At \textit{Neidens} fælles \textit{Finner} dette saaledes fra gammel Tid have brugt, det og anderledes mindes Vidnerne ikke, ei heller have det anderledes af deres Forældre hørt. Nu om stunder skal der af disse \textit{Neidens} Fælles \textit{Finner} kun en 8 \textit{Familier} sig opholde, men i forrige Tider have de været meer, end engang saa mange, som til det mindre Tall ere bortdøde.\par
9. \textit{Skoggerøe}, fra \textit{Kiøholmen} i Øster til Syden 1/8 Miil, fra det Søndre faste Land med sin Søndre Side 1/8 Miil i Nord, fra hvilket Søndre faste Land det adskilles ved det Sund, kaldet \textit{Kaarsfiord}, fra det Vestre faste Land, hvor \textit{Brasshavns} Næss er, med sin Vestre Side 1/8 Miil og mindre i Øster, fra hvilket Vestre Land det adskilles ved \textit{Kiøfiord}, er paa den vestlige Side fra Søer i Nord, nemlig fra det Søndre Næss \textit{Salagækie} til Nordre Næss \textit{KiesvadNiarg} langs efter \textit{Kiøfiord} 1 3/4 Miil lang; Paa den Østre Side fra det Søndre Næss \textit{Raa-aveNiarg}, til det Nordre Næss \textit{Galle-Niarg} rundvoren ‒ lang 1 1/4 Miil; Paa Søndre Side til \textit{Kaarsfiord} nemlig fra \textit{Salagækie} til \textit{Raa-aveNiarg} fra Vester j N.Øster 3/8 Miil breed; Paa Nordre Side fra \textit{KiesvadNiarg} til \textit{GalleNiarg} fra Vester i Øster 1/4 Miil breed. ‒\par
Paa den Søndre Side strax Østen for det Næss \textit{Raa-aveNiarg}, gaaer en Strøm, navnlig \textit{Leervaag}, ind i Øen i Nord-vest 1/4 Miil lang, 1 Steenkast over breed, men inde ved Botten over 1 Bøsse-Skud viid;\par
Paa Vestre Side fra \textit{Kiøfiord} gaaer en liden Fiord \textit{Riommæ vuodne} ind i Øen i Sydost, 2 à 3 Bøsseskud lang, 1 Bøsseskud over breed, imod Strømmen \textit{Leervaag}, saa at det Eid imellem Strømmen og denne liden \textit{Riommæ}-Fiord er kun 2 Bøsseskud over langt;\par
Paa Østre side, 1/2 Miil i Syd-ost fra \textit{Galleniarg}, gaaer en Strøm \textit{Selfer-Strøm} ind i Øen i Syd-vest 3/4 Miil lang, saa at fra dens Botten i Søer til \textit{Leervaag}-Strømmen er 1/4 ‒ til \textit{Riommævuodne} i Vest-Syd-Vest 1/4 ‒ og til \textit{Kiøfiord} i Vester 1/4 Miil.\hypertarget{Schn1_98870}{}Vardøe Præstegield.\par
Denne \textit{Skoggerøe} i sig selv er ganske klipped, nogle af dens Berge ere bare, nogle med Reen-Moese paa; J Dalene imellem Klippene er Græss og god Bierk, nu ubeboed; Dog bruge \textit{Normænd} fra \textit{Vadsøe} og andenstedz fra, der at slaae Høe om Sommeren, saa der er ligesom en Alminding for de \textit{Norske} til Høe-Slotte. For omtrent 9 Aar siden satt sig der en \textit{Norsk} Field-\textit{Finn} ned med sine Reen, opholdendes sig der i 3 Aar, da han døde; Siden har Øen ligget Øde, indtil for 2 Aar siden, da 4. \textit{Normænd} fløttede did paa Vestre Side ad \textit{Kiøfiord}, nær Østen for \textit{Riømmæ-vuodne} mens forbleve der kun 1 Vinter, da de foere derfra igien; Fordi der falder strengt Vejer, og \textit{Kiøfiord} lægges om Vinteren med Jis.\par
Jngen af de fælles \textit{Russisk-Norske Finner} have siddet, eller sidde paa denne Øe.\par
At agte, at \textit{Kiøfiord} kaldes og \textit{Neidens}Fiord.\par
10. \textit{Kielm-øe}, fra \textit{Skoggerøe} i Øster 1/8 Miil, ligger i Gabet af \textit{Bøg-Fiord}, eller \textit{Pasvig}- Fiord, fra \textit{Holmgraa-Næss} paa det Østre faste Land 1/4 Miil, fra \textit{Ellinghavn-Næss} paa det Sydvestlige Land i Nord Nordost 1 1/4 Miil, fra \textit{Ekkerøe}, som ligger over Fiorden \textit{Varanger} ved det Nordre faste Land lige ovenfor, 2 Søe-Mile i Søer, er fiirkanted, paa hver Side 1/8 Miil lang, mest klipped af temmelig høye og tindede Berge, med Lyng paa, og med smaa Græss-Sletter hist og her i Dalene, har ingen Skoug, uden smaa Vidie-Riis, u-beboed;\par
Her falder nogen Eederduun, Multibær, Egg, benyttes, som en Alminding baade af \textit{Varangers Norske Finner}, og \textit{Normænd}, og af fælles \textit{Passvig-Finner}, til at samle Duun, Multer og Egg; Disse de \textit{PassvigFinner} holde til der paa om Sommeren med deres Reen, og for Fiskeriets Skyld i Havet Østen for Øen.\par
11. \textit{Reen-øe} inde i \textit{Bøg}- eller \textit{Passvig}fiorden, fra \textit{Kielmøe} i Søer 1/2 Miil, fra det Østre faste Land 1 Bøsseskud, fra det Næss af det Vestre faste Land, \textit{Haakierring-Næss}, som skiller \textit{Bøgfiord} fra den Jndfiord, \textit{Langfiord}, 1/4 Miil, langagtig-treekanted, 1/8 Miil i hver Kant, dog paa den Østre Side noget kortere, klipagtig paa Nordre Side, ellers slet med noget Bierk og Græss paa;\par
For omtrent en Halv Snees Aar siden har paa denne \textit{Reen-øe} boet henved 5 \textit{Normænd} en 3 Aars Tid: Men fordi Fiorden der fryser, og intet Fiskerie nær ved Øen vanker, have de fløttet derfra, og nu boer Jngen der: dog har Bøtkeren af den \textit{Norske Vadsøe} Handel sin Bod der om Sommeren hvor han af de fælles \textit{Passvig Finner} imodtager deres Lax, og salter: Men af de fælles \textit{Passvig Finner} holder Jngen der til.\par
Længere i Søer var Jngen her bekient.\hspace{1em}\par
Følger nu Beskrivelsen af \textit{Vaardøe} Gieldz faste Land:\hspace{1em}\par
a. Norden for \textit{Varanger}fiord, som de første 3de Vidner \textit{pag.} 328 benævnte, have forklaret.\par
\textit{Pag:} 286 her er meldet, at det yderste faste Land af \textit{Kiøllefiords} Præstegield i Øster strekker sig til \textit{Løkvig;} derfra imodtager nu \textit{Vaardøe} Præstegieldz \textit{District} i Øster.\par
Det Nordreste Støkke af det faste Land, at regne fra \textit{Tanahornet} i Ost-Syd-ost til \textit{Kafring-Næss} imellem \textit{NordSøen} og \textit{Varangerfiord} kaldes i Almindelighed \textit{Varje-Niarg}, eller \textit{Varanger-Næss} og er paa Nordre Side ud til Havet langt 6 Søe Mile, bredt, at reigne fra Vestre \hypertarget{Schn1_99116}{}Schnitlers Protokoller V. Ende imellem \textit{Tanahornet} i Nord, og \textit{Vest-Botten} af \textit{Varanger}-fiord i Søer er {5 1/2 Søe Mile} Jmellem \textit{Makur} i Nord, og \textit{Skattøren} eller \textit{Vadsøe} i Søer er 2 Dagers Reise, eller {5 Søe Mile} Jmellem \textit{Syltevig} i Nord, og \textit{Ekkerøe} i Søer ved {4 1/2 ‒} Jmellem \textit{Kafring-Næss} i Nord, og \textit{Kiberg-Næss} i Søer god {1 ‒ _______}\par
At beskrive nu Søe-Kusten af \textit{Varanger-Næssets} Nordre Side:\par
Saa er fra \textit{Løkvig,}\textit{pag.}, 286 forklared, i OstSydost til \textit{Kongsøe}fiord{1/2 Søe Miil}\par
Jmellem \textit{Løkvig} og denne \textit{Kongsøe}fiord er en Bugt\par
\textit{Riisbugt}, 1/2 Miil viid i Gabet, halv saa dyb; Landskabet om denne \textit{Riisbugt} bestaaer af bare skallede Fielde, slette med nogle smaa Houger ovenpaa; Jnde i Botten er intet, uden noget Vidie-Riis; Det Østre Næss af denne Bugt heeder \textit{NaalNæss}, som og er det Vestre Næss af \textit{Kongsøe}Fiord, spidz og fladt udstikkendes i Nord, bart oppaa. Det Østere Næss af \textit{Kongsøefiord} er \textit{Baass-næss}, 1/4 Miil over stort i Øster, bratt og barsteenet; Jmellem disse 2de Næsse er \textit{Kongsfiord} sterk 1 Søe Miil i Gabet viid, derfra gaaer Fiorden i Vester til Syden først 1/2 Miil til \textit{Veinæss}, paa Vestre Land-Side, imellem hvilket \textit{Veinæss} og det Østre Land Fiorden er 3/8 Miil breed, siden vender Fiorden fra \textit{Veinæss} sig i Vester 3/8 Miil lang til Strømmen, som er 1 Bøsseskud lang og breed, ved fuld Flod farendes, derefter Fiorden sig stikker fra Strømmen i Syd-vest 1/2 Miil lang til Botten, 1/8 Miil og mindre breed; Paa begge Siders Fiord-Brædder er det lidet slet og Lønggroet, men siden barfieldet; Jnde i Botten, hvor Elven løber j, er Bierke-Skoug paa begge Sider af Elven, et par Bøsse-Skud paa hver Side breed, 1/4 Miil lang opefter Elven;\par
J Botten af denne \textit{Kongsøefiord} kommer fra Syd-vest \textit{Kenje}-Elv, som først oprinder af \textit{Jo-jaure}, og derfra gaaer rundvoren i Vester, og Nord-vest 1/8 Miil lang ind i \textit{Nastjaure}, og omsider heraf i NordNordost 1/8 Miil lang i \textit{Kenje-jaure}, og endelig herfra i samme \textit{linie} 1 stor Miil lang i \textit{Kongsøe}fiord-Botten.\par
\textit{Jo-jaure} er rundvoren, 1 Bøsseskud over vidt; \textit{Nast-jaure} er rundt, 1/16 Miil over stort. \textit{Kenje-jaure} er fiir-oddet, 1 Miil omkring stort; J disse Vande fiskes Røer.\par
Jnde i denne \textit{Kongsøe}fiord imellem \textit{Veinæss} og det Syd-ostlige Land, dog nærmere til dette, er \textit{Kongsøe}, en Holm, rund, 2 Bøsseskud over stor, med Berger paa, som ovenpaa ere slette. Fiorden og Holmen ere u-beboede ‒ Strax Østen om \textit{Baass-Næss} aabner sig\par
\textit{Baass-Fiord}, hvis andet Næss \textit{Raossmaal} ligger fra \textit{Baassnæss} i Syder til Osten 1/2 Miil, derimellem stikker \textit{Baassfiord} i Vester 1 Miil lang til Botten; breed er den i Gabet 1/2 Søe Miil, siden bliver smalere; Naar den har gaaet 1/2 Miil ind, trænges den til 1/8 Miils Bredde, den hun beholder 1/2 Miil lang til Botten; Paa Nordre Side ere skallede Berge, flad-nedgaaende til Fiorden; Paa Søndre Side fra \textit{Rossmaal-Næss} til Botten ere Fieldene steile og bare; Jnde paa Botten er Bierkeskoug, 1/4 Miil lang, halv saa breed; og i Botten kommer Vesten-fra kun en Bæk. Fiorden er ubeboed. Nær Østen for \textit{Rossmaal-Næss} er den Bugt\par
\textit{Makour}, 1/2 Bøsseskud i Gabet viid, 1 Bøsseskud dyb, paa Sidene barsteened, inde i Botten er en Slette, 1 Bøsseskud stor, steened med lidet Græss imellem; Paa denne Slette boe en 8 \textit{Normænd}, som holde kun nogle faa Gieder, og staaer det \textit{Norske Makour-Capell}. Fra \textit{Makour} i Øster er til \textit{Kaarsnæss} 1/2 Miil; Landet derimellem er klippedt, høy-bratt og bart, \hypertarget{Schn1_99367}{}Vardøe Præstegield. hvilke Berge ovenpaa ere slette med lidet Reen-Moese paa; ubeboedt. Dette \textit{Kaarsnæss} er det Nordre Næss af \textit{Sandfiord}, hvis Søndre Næss er \textit{Klubben}; Jmellem disse Næss\par
\textit{Sandfiord} i Gabet er 1/2 Miil lang i Søer, 1/2 Miil dyb i Vester; J hvis Botten kun en Bæk gaar. Landet paa Nordre Side inde i Fiorden er bratt-klippedt, paa Søndre Side noget fladt, dog berget og bart; u-beboedt. Forbemeldte \textit{Klubben} er det Nord-vestlige Næss af \textit{Syltevig}- fiord, hvis Syd-ostlige Næss giøres af \textit{Havningberg};\par
\textit{SyltevigFiord} er imellem de 2 Næss viid 1 Miil i Ost-Syd-ost, og fra \textit{Klubben} 3/4 Miil: men fra \textit{Havningberg-Næss} 1 1/2 Miil dyb i Vester; Landet inde i Fiorden paa Nordre Side er bar-klippedt paa 1/4 Miil nær Botten, hvor Bierk og noget Græss er; Bierkeskougen er til Botten 1/4 Miil lang, 1/8 Miil breed, siden strekker den sig efter Elven 3/4 Miil lang i Syd-vest, og ligesaa breed. ‒\par
I Botten af \textit{Syltevig}fiord stikker af det faste Land fra Vester et Næss ud i Fiorden, ved Navn \textit{Veinæss}; dette \textit{Veinæss} er inde ved det faste Land kun 2 Bøsseskud over smalt, siden breder det sig ud til imod 1/8 Miil, hvor det er videst, men i Enden stikker spidz ud; dette Næss ligger det Nordre 1/16 Miil nær, og fra det Søndre Land over 1/4 Miil. J Botten af denne Fiord kommer fra Syd-vest Aaen \textit{Syltevigfiord}-Elv af et FiskeVand \textit{Aarde-jaure}, og løber i Nordost 2 Mile lang i Fiord-Botten. \textit{Finnerne} kalde Elven \textit{Aardejok}, og Fiorden \textit{Aardevuodne}, som er u-beboed, dog bruges af Norske Field-\textit{Finner. Aarde-jaure} er fra Syd-vest i Nord-ost 1/2 Miil langt halv saa bredt, fiskerigt af Røer, og gaaer Lax op fra Fiorden i Elven. ‒ Strax Østen om Næsset \textit{Havning-berg} er Bugten\par
\textit{Sandfiord-Bugt}, i Gabet 1/4 Miil viid i Syd-ost, halv saa dyb, paa Sidene indentil barklipped, inde i Botten lidet slet med noget Græss paa, u-beboed; J Botten indgaaer en Aae Vesten-fra, \textit{Sand}Elv, paa \textit{Finnsk, Davag}, kommendes fra Field, og løber 2 Mile lang i Øster i Fiord-Botten. Det Syd-ostlige Næss af \textit{Sandfiord} heder \textit{Hamborr-Næss}. Fra dette Næss i Ost-Syd-ost er 1/4 Miil til\par
\textit{Parsfiord}; Landet derimellem er steen-klippedt, og noget bratt, u-beboedt. Det Vestre Næss af \textit{Parsfiord} heder \textit{Seigl}; det Østre \textit{Kafring-næss} derfra i Ost-Syd-ost 3/4 Miil; hvorimellem \textit{Parsfiord} i Gabet 3/4 Miil, som meldt, er viid, dyb fra \textit{Seigl-næss} 1/4 Miil, og fra \textit{Kafring} imod 1 Miil i Vester. Landet paa Nordre og Søndre Sider er klippedt og bart; J Botten er Enge-Slette 1/4 Miil breed, som Botten er viid, 1/8 Miil dyb, med smaa Vidie-Riis paa, u-beboed.\par
\textit{Kafring-Næss} er rundvoren, som en Kafring, skabt. J Botten af \textit{Parsfiord} indløbe 2de, nemlig, Vestre og Østre Elve, og paa Søndre Side af \textit{Parsfiord Præstnæring}-Elv, som alle kun ere Bække; Denne \textit{Præstnæring}Elv løber ind i Fiorden 1/8 Miil Vestenfor \textit{Kafring-Næss}, og Østenfor Elv-Munden ligger \textit{Præstnæring-Næss}, som henger sammen med \textit{Kafring-Næss}, og ligesom \textit{Kafring}, er barklippedt.\par
Fra \textit{Kafring-næss} vender Landet sig i Søer til \textit{Svartnæss} 1/2 Miil; derimellem er en Bugt, 1/8 Miil Nordenfor \textit{Svartnæss}, slet neere ved Havet, en 4 Bøsseskud breed, med noget MyrGræss paa, hvor \textit{Normænd} paa \textit{Svartnæss} have deres Køer og Faar gaaendes; ellers ubeboed. \textit{Svartnæss} er et lavt sort Berg, bart, slet ovenpaa, ind ad Landet bredt, men spidz udstikkendes ad Havet, 1 Bøsseskud over bredt fra Nord i Søer; Af de mange \textit{Normænd}, som her boet, ere kun 2 igien, de Øvrige bortdøde.\hypertarget{Schn1_99580}{}Schnitlers Protokoller V.\par
Fra \textit{Svartnæss} gaaes i Søer en god 1/2 Miil til \textit{Kiberg}, som er det Nordre Næss af \textit{Varanger}fiord; Jmellem disse 2de Næss, dog nærmere \textit{Svartnæss} er det Field \textit{Domen}, høyere, end de andre omkring værende Fielde, over alt skallet, ovenpaa fladt, rundt, 1/8 Miil over stort, steilt ud paa Østre Side ad \textit{Bussøe}Sund, ellers fladagtigt; Paa begge Sider af dette \textit{Domen} er slet Løng-Mark, den Nordre Slette bruge de Folk paa \textit{Svartnæss}, den Søndre Slette de af \textit{Kiberg}. ‒ \par
\textit{Kiberg} er et lavt Field fra Nordvest i Sydost 1/4 Mil langt, halv saa bredt, bart, med smaa Houger ovenpaa; Herfra vender Landet sig i Syd-vest, ad \textit{Varangers} Fiord, og strax Vesten om \textit{Kiberg} er\par
\textit{Kibergs-Vaagen}, 1/8 Miil viid i Gabet, og næsten ligesaa dyb, rundagtig, slet omkring Søe-Brædene et par Bøsseskud, da Landet siden til lave skallede Berge opstiger; Her have i gammel Tid boet mangfoldige \textit{Normænd}, som ere uddøed til 1 nær som boer paa Østre Side af \textit{Vaagen;} Fordj denne Vaag er fiske-riig; Omtrent mitt i Botten staaer \textit{Kiberg-Annex}- Kirke af Træ, paa den Søndre Side ad Elven. J denne Vaag paa dens Nordre Side indkommer \textit{Kiberg}-Elv fra Nord-vest, 1/8 Miil lang af \textit{Kiberg}-Vand, Som er fra Nord-vest i Syd-ost 1/4 Miil langt, halv saa bredt, havendes Øreter. ‒ Det Syd-vestlige Næss af denne \textit{Kiberg-Vaag} heder \textit{Oxebaas-Næss}, saa kaldedt, fordi den har en liden Bugt ind ad, som en Baad kan fare ind i.\par
1 Bøsseskud uden for i Syd-ost er et lavt lidet Skiær, fra Øster i Vester et par Bøsseskud langt, halv saa bredt, navnlig \textit{Kunt-skiær;} Jmellem dette Skiær og \textit{Oxebaas} er \textit{Kuntsund}, et par Bøsseskud langt fra Øster i Vester, 1 Bøsseskud over bredt.\par
Fra \textit{Oxebaas} i Syd-vest ligger 1 Bøsseskud, \textit{Jndre-Kiberg-Næss}, hvorinden eller Vestenfor er en Bugt, \textit{Jndre-Kiberg}, 3 Bøsseskud i Gabet viid, halv saa dyb, u-beboed. Fra \textit{Jndre- Kiberg-Bugt} 1/4 Miil i Syd-vest er\par
\textit{Kramvig}, 1 Riffelskud over viid, halv saa dyb; Omkring Vigen er [en] Slette paa hvilken Græss, 3 Bøsseskud breed, hvor 3 \textit{Normænd} boe; der en Bæk indgaaer. ‒ 1/8 Miil i Syd-vest fra \textit{Kramvig} er \textit{Grundnæss}, rundagtig udgaaendes, lavt, slet og bart, der Vesten-for er\par
\textit{Grundnæss-Bugten}, i Gabet 1/4 Miil over viid, halv saa dyp, slet omkring med Græss paa, 1/8 Miil, dog uden Skoug, beboes af omtrent 4 \textit{Normænd;} Det Vestre Næss af \textit{Grundnæss-Bugten} er \textit{Svartnæss;} Fra \textit{Svartnæss} i Syd-vest til \textit{Langnæss} er 1/8 Miil; derimellem er en Bugt, slet deromkring med noget Græss paa, ved 3 Riffelskud breed indtil Fieldene, u-beboed. ‒\par
Fra \textit{Langnæss} i Syd-vest til \textit{Komag-næss} er 1/2 Miil; Dette \textit{Komagnæss} har en temmelig høy Tinde, er skalled, stikker ud imellem \textit{Komag}Elv paa Østre og \textit{Skall}Elv paa Vestre Side, 1/2 Miil over i Syd-vest. \textit{Komag}-Elv, Østen for \textit{Komagnæss}, oprinder af Field fra Nord-Vest, og løber 2 Mile lang i Syd-ost i Fiorden. \textit{Skall}-Elv udrinder af \textit{Ridjaure} fra Vester i Øster 1/2 Miil, siden i Syd-Syd-ost 1/2 Miil, i alt 1 Miil lang mitt i \textit{Skallvigbotten}. ‒ \textit{Ridjaure} er fra Vester i Øster 3/8 Miil langt 1/8 Miil bredt, havendes RøerFisk. Fra \textit{Komagnæss} fares i Syd-vest til \textit{Skallnæss} 3/4 Miil, Jmellem disse Næsse ligger\par
\textit{Skallvigen}, 3/4 Miil i Gabet viid, 1/4 Miil dyb, slet omkring Vigen paa Østre Side 1/4 Miil ‒ men paa de andre Sider 3/4 Miil breed, førend Landet stiger op til Fieldz, med Græss begroed, uden Skoug, nu u-beboed. ‒ Mitt i denne \textit{Skallvig} løber bem.te \textit{Skall}Elv. \textit{Skallnæssfield} rekker fra Vandet, nemlig fra Nord-vest i Syd-ost 1 1/4 Miil langt, 3/4 Miil bredt, ovenpå fladt og \hypertarget{Schn1_99829}{}Vardøe Præstegield. skallett overalt, dette Næss er lavt, spidz udstikkendis i Syd-ost, 1/4 Miil over bredt. Fra \textit{Skalnæss} i Syd-vest til \textit{Krampenæss} er 1/2 Miil; Landet herimellem ved Fiord-Bræden er slet 1/8 Miil bredt, med Løng og lidet Græss paa, uden Skoug, u-beboed. \textit{Krampenæss} er rundvoren, lavt, et par Bøsseskud over stort, bart, af 2 \textit{Normænd} beboed.\par
Fra \textit{Krampenæss} i Syd-vest til \textit{Sollnæss} er 1/2 Miil; J Landet derimellem, dog nærmere \textit{Krampenæss} ligger en Bugt, 1/8 Miil i Gabet viid, 1 Riffelskud dyb; Landet Østen for Bugten er 1/8 Miil, Vesten for Bugten 1/4 Miil, paa begge Steder Lyng-Mark. 1/4 Miil fra \textit{Krampenæss}- odden ligger \textit{Lill-Ekkerøe}, 2 Bøsseskud fra \textit{Sollnæss} ligger \textit{Stor-Ekkerøe} i \textit{Varanger}-fiord. \textit{Sollnæss} er en lav Bakke fra Nord-vest i Syd-ost lang 1/4 Miil, paa den Østre Side med de andre Berge field-fast, flad med Lyng og lidet Græss paa, af 4 \textit{Normænd} beboed. Fra \textit{Sollnæss} til \textit{Thibye} i Vester er 3/4 Miil; Landet derimellem er slet Lyng-Mark, og myredt, ved 3/4 Miil vidt i Nord, uden Skoug, u-beboedt.\par
\textit{Thibye} er og slet med Lyng og lidet Græss paa, 2 Bøsseskud i Vester lang, 1/2 Miil i Nord, førend Fielde begynde, viid, beboes af 5 à 6 \textit{Normænd}. Fra \textit{Thibye} i Vester til \textit{SkattørenNæss} er 1/4 Miil; Mellem-Rommet er lynget og myret, ubeboet.\par
\textit{Skattøren-Næss} er i Vester 1 Riffel-Skud over bredt, 2 Riffel-skud langt i Nord, en lav Sandøre, udstikkendes i \textit{Vadsøe} Sund; Herpaa staaer \textit{Vadsøe} Kirke af Træ, Præstegaarden, og boe nogle Bønder, \textit{Normænd}. Fra \textit{Skattøren} i Vester til Norden er 1/2 Miil til \textit{Andersbye}; Landet derimellem er slet, over 1/8 Miil i Nord bredt til Fieldz, deels græsset, deels lynget, u-beboet.\par
\textit{Andersbye} er en Slette, Græss-groed, et par Bøsseskud i Vester lang, 1/8 Miil i Nord breed, hvor 4 \textit{Normænd} boe. Fra \textit{Andersbye} i Vester til Nord er 1/4 Miil til \textit{Sandskiær}; Landet derimellem er slet, men kun 1. Riffelskud bredt, indtil Fieldene, græss-groet, ubeboet. \textit{Sandskiær} [er] en slet Pladz og Fiske-vær hvor 2 \textit{Normænd} boe. Fra \textit{Sandskiær} i Vester til Nord er 1/4 Miil til \textit{Padvignæss}; Landet derimellem er slet 1/8 Miil dyb i Nord, græss-groet;\par
\textit{Padovig-Næss} er fladt af Jord- og Myrland, et par Bøsseskud over bredt fra Øster i Vester, ubeboet. Fra \textit{Padovig-Næss} i Vester til Nord er 1/8 Miil til \textit{Findenæss}, derimellem ligger en \textit{Normands} Bye \textit{Carjell}, slet med lidet Græss paa, 1 Bøsseskud dyb i Nord, hvor omtrent 6 \textit{Normænd} boe.\par
\textit{Findenæss} er en Jord-Slette, 2 Riffelskud over i Vester bredt, med Lyng bevoxen, ubeboet. Fra \textit{Findenæss} i Vester til Nord er 3/8 Miil til \textit{Klubbe-Næss}; Landet derimellem er slet med Græss og Lyng paa, men ganske smalt efter Fiordbræden. Herimellem dog nær ved \textit{Findenæss} løber en Aae, \textit{Jakobs}-Elv ud i Fiorden, 1 1/2 Miil lang fra Nord-vest af en VatsKiøn; 1/16 Miil Vesten for Elv-Munden boe omtrent 6 \textit{Normænd}.\par
\textit{Klubbenæss} er et skallet Berg, noget høyt, rundvoren, et par Bøsseskud over stort. ‒ Fra \textit{Klubbenæss} i Vester til Nord er 3/8 Miil til \textit{MortensNæss}, derimellem nær ved \textit{Klubbenæss} ligger \textit{Klubvigen}, 3 à 4 Bøsseskud i Gabet viid, halv saa dyb, omkring Vigen brattklipped, men i Botten 1 Bøsseskud slet, hvor 6. Norske Søe-\textit{Finner} tilholde. Fra \textit{Klubvigen} til \textit{Mortensnæss} er Landet slet med Lyng paa, men small, u-beboet. ‒ Fra \textit{Mortens Næss} i Vester til Nord er mod 1/4 Miil til \textit{Hammer-Næss}; Landet derimellem er mest fieldet, u-beboet. ‒ \textit{Mortensnæss} er et fladt Jord-Næss, ved 4 Bøsseskud over bredt fra Øster i Vester, af 4. Norske \hypertarget{Schn1_100070}{}Schnitlers Protokoller V. Søe-\textit{Finner} beboet. \textit{Hammer-Næss} er en flad Græss-Slette, 1 Bøsseskud over breed i Vester, beboed af 8 \textit{Norske} Søe-\textit{Finner}. ‒ Fra \textit{Hammer-Næss} i Vester til Nord er 2 Riffelskud til \textit{Yttre-Berge-Bye}, en slet \textit{Finne}-Bye, hvor 4 \textit{Norske} Søe-\textit{Finner} om Sommeren sidde, med lidet Græss paa.\par
Fra \textit{Yttre Bergebye} er 1 1/2 Bøsseskud til \textit{Jndre-Berge-Bye}, derimellem et stort, bart Field, steilt ud til \textit{Varanger}-fiord; Denne \textit{Jndre Bergebye} er 1 1/2 Bøsseskud lang, og ligesaa dyb med sin Slette i Nord; Denne \textit{Jndre Bergebye} har ved 8 \textit{Norske} Søe\textit{Finner} til Jndbyggere, den \textit{Norske Missions} Skole og \textit{Missionaire}. Fra \textit{Jndre Bergebye} er i Syd-Vest 3 Riffelskud til\par
\textit{Næsse-Bye}, 2 Bøsseskud lang i Vester, slet i Nord 1/8 Miil, hvor vel 20 Norske Søe-\textit{finner} boe. Fra \textit{Næssebye} til \textit{Ambonæss} 1/4 Miil er i VestNordvest; Landet derimellem er slet med Græss paa, 1/4 Miil og mindre bredt, i Nord til Fieldene med smaa Skoug paa.\par
\textit{Ambonæss} er et Sand-Næss, 1/2 Bøsseskud over bredt, derfra til \textit{Vester-Botten} af \textit{Varanger}fiord i Vester til Søer er 1/2 Miil. Landet derimellem er slet med Græss og Bierke-skoug paa, omtrent 1/16 Miil dyb i Nord til Fieldene. J dette Mellem-Rom holde Norske Søe\textit{Finner} til om Vinteren, for Skougen skyld; men om Sommeren have de deres Sommer-Byer, eller Vaan-Hytter længer ud i \textit{Varanger-}fiord, for Fiskeriets Skyld. Nær ved \textit{Ambonæss}, Vesten derfor, er et temmelig høyt Field, slet og bart ovenpaa, med en flad Tinde paa dets Vestre Ende, nedhældendes i Søer ned ad Fiorden, navnlig \textit{Meitsk}, fra Oster i Vester mod 1/8 Miil lang, ikke fuld saa bredt, deraf merkeligt, at \textit{Finnerne} der i gamle Dage skal have haft deres Offer-Sted.\par
Dette har nu været Beskrivelsen af det Nordre, eller \textit{Yttere} faste Land, kaldet \textit{Varanger-Næss}, efter dets Søe-Kuster baade ud ad \textit{Nord}Søen, og ind ad \textit{Varanger}-Fiord;\par
Det Jndre Land inde paa \textit{Varanger-Næss} er mest slet-fieldet, som \textit{pag}. 316 f. før forklaret.\par
\textit{Varangersfiord} beskrives herefter ved 5 Sp: \textit{pag:} 350.\hspace{1em}\par
Følger nu Forklaring af det Vestre faste Land, at reigne fra \textit{Varanger}-Fiordz Bottener i Vester imod \textit{Tana}-Elv:\hspace{1em}\par
Forommelte \textit{Vester-Botten} af \textit{Varanger}-Fiord stikker ind i Vester til Syden, en kort 1/2 Søe-Miil lang, 1/8 Miil meer og mindre breed, (hvorinde boer 1 \textit{Norsk} Søefinn) og ligger imellem de 2de Næss, \textit{Ambonæss} paa \textit{Varanger}-Næssets faste Land i Nord, og \textit{Angsnæss}-Odden i Søer; Landskabet paa Søndre Side af \textit{Vester-Botten} er slet med noget Bierk, og lidet Græss paa; Mitt i Bonden af denne \textit{Vester-Botten} rinder Sønden-fra en Aae ‒ navnlig\par
\textit{Nidjo-jok}, 1 Miil lang af \textit{Nidjo-jaure}, som fra Syd-Syd-vest i Nord-Nord-ost er 1/4 Miil langt, halv saa bredt; J dette \textit{Nidjo-jaure} paa dens Vestre Side løber en anden Aae Vestenfra, \textit{Saide-jok}, imod 1 Bøsseskud lang af \textit{Saide}Vand; dette \textit{Saide-jaure} er rundt, 1 Bøsseskud over stort; J samme \textit{Nidjo-jaures} Søndre Ende kommer \textit{Savæk-jok}, 3/8 Miil lang, fra Syd-vest af \textit{Savæk-jaure}, som bestaaer af 3 smaa efter hinanden liggende Vande fra Syd-vest i Nord- ost; Det Søndre og Nordre af disse Vande er 1/8 Miil ‒ det Mellemste 1/4 Miil langt; det Søndre og Mellemste ere 1/8 Miil brede, det Nordreste ganske smalt, af hvilket sidste \textit{Savæk-jok} rinder i Nordost i \textit{Nidjo-jaure}.\par
Det Næss, som Vesten-fra af det faste Land stikker ud i \textit{Varanger}fiord, imellem denne \textit{Vester-Botten} i Nord, og \textit{Øster-Botten} i Søer, kaldes \textit{Angsnæss}, langt i Øster til Nord, som \textit{Ves\hypertarget{Schn1_100361}{}Vardøe Præstegield. ter-Botten} er, nemlig 1/2 Miil, bredt imellem begge Bottener 1/4 Miil, og Østerst paa Odden et par Bøsseskud over bredt; Mitt paa Odden af dette \textit{Angsnæss} staaer det \textit{Norske Finne-Capell} af Træ, svarendes under \textit{Vaardøe}Gield. \textit{Angsnæsses} Land er slet med smaa Houger paa, med smaa Bierke-Riis begroed, ubeboed. Strax Sønden om \textit{Angsnæss}-Odden gaaer fra \textit{Varanger}- Fiord ind i Vester\par
\textit{Øster-Botten}, imod 1/2 Miil dyb, i Gabet imellem \textit{Angsnæss}-odden, og \textit{Veinæss}-odden imod 1/2 Miil breed, men siden bliver indad smalere; Landskabet af \textit{Øster}-Botten, paa dens Nordre Søe-Bræde, er ved \textit{Angsnæss} meldt; Paa Søndre Bræde og i Botten er det smaahouged, med Lyng og smaa Bierke-Riis paa; J Botten gaaer kun en Bæk, og boe 2de \textit{Norske} Søe \textit{Finner:} Men paa Søndre Side af \textit{Østbotten} er en Bugt, navnlig\par
\textit{Karlsbugt}, eller \textit{Ræpen}, et par Bøsseskud viid; J denne Bugt løber Sønden-fra \textit{Ræpen}- Elv, 1 Miil lang af \textit{Dierge-vatten}, som fra Søer i Nord er 3/4 Miil langt, og 2 Bøsse-Skud over bredt, Synden for \textit{Angsnæss} er Foromrørte \textit{Veinæss}, stikkendes Vesten-fra af det Vestre faste Land i Øster til Søer, imellem \textit{Øster-Botten}, og \textit{Veinæss}-Fiord, eller \textit{Søer}Fiord, knap 1/4 Miil langt, og halv saa bredt, dog inde ved Botten er det kun 3 à 4 Bøsseskud over vidt; \textit{Veinæss} er slet og myret, ovenpaa med Lyng begroet, paa dets Nordre Side alleene er lidt Græss. Dette \textit{Veinæss} paa den eene ‒ og det Søndre faste Land paa den anden Side giøre en Jndfiord\par
\textit{Veinæss}-Fiord, eller \textit{Sørfiord}, som fra \textit{Varanger}-Fiord stikker 1/4 Miil ind i Vester, vendendes sin Botten lidet til Nord; denne Fiordz Søndre Land-Side er brat og bart, og Botten er slet med smaa Houger paa, med Græss, Lyng og lidet Bierke-Riis begroed; J denne \textit{Veinæss}fiord boe 25 \textit{Norske} Søe\textit{Finner}, og gaaer ingen Aae ind, uden paa dens Søndre Side i Gabet imod \textit{Veinæss}-Odden, der løber Sønden-fra af det Søndre Land, og opkommer af VatsKiønne, navnlig \textit{Ny-elv}, 1/2 Miil lang.\hspace{1em}\par
Landskabet fra \textit{Varangers}fiordz Bottener i Vester ad \textit{Tana}-Elv er nu følgende:\hspace{1em}\par
(a) Fra \textit{Vester-Botten} i Vester til \textit{Moketveje}, en \textit{Norsk} Boemandz Jord-pladz ved \textit{Tana}- Elv er god {1 SøeMiil,} og beskreven \textit{pag.} 317.\label{Schn1_100558} \par 
\begin{longtable}{P{0.7587248322147652\textwidth}P{0.07130872483221477\textwidth}P{0.019966442953020136\textwidth}}
 \hline\endfoot\hline\endlastfoot (b) Fra \textit{Østre Botten} fares i Vester over Myrland og over \textit{Nidjo-jaure} til \textit{Raude-vara}\tabcellsep 1 FieldMiil,\\
Naar man derover er kommet, er slet Myrland med Houger; og Bierk paa, til \textit{Bolma}, en Norsk Boemandz Jord-pladz ved \textit{Tana}Elv, i Vest-Syd-vest\tabcellsep 1 dito\\
\tabcellsep _______\tabcellsep 2 Miil.\end{longtable} \par
 \par
\textit{Raude-vara} er et Field, ikke meget høyt, ovenpaa rund-fladt, med Reen-Moese bevoxen, paa Sidene nedhældendes, med Lyng og Bierk paa, fra Søer i Nord et par Bøsseskud langt, halv saa bredt. Fra \textit{Raudevara} ligger i Nord-Nord-vest\par
\textit{Goldevara}, ved en kort Myr- og Bierkedal derfra adskilt, fra Søer i Nord 1/4 Miil langt, halv saa bredt, ikke ret høyt, rundfladt og moeset ovenpaa, paa Siderne fladtvoren, med Bierk og Lyng begroet. ‒ Nordenfor \textit{Goldevara} er en Myr- og Bierke-Dal, 1/4 Miil viid, hvorefter i Nord følger\hypertarget{Schn1_100619}{}Schnitlers Protokoller V.\par
\textit{Saadde-vara}, ikke høyt, slet med smaa Houger og Reen-Moese paa, fladt paa Sidene, med Bierk og Lyng bevoxen, treekanted 1/4 Miil i hver Kant,\label{Schn1_100627} \par 
\begin{longtable}{P{0.7698706099815157\textwidth}P{0.06284658040665435\textwidth}P{0.017282809611829944\textwidth}}
 \hline\endfoot\hline\endlastfoot (c) Fra \textit{Veinæss}fiordz Botten er Myrland til \textit{Ræpen}-Elv\tabcellsep 1/4 Miil\\
derfra i Vest Syd-vest til \textit{Nidjo}-Elv, over slet Bierke-Land\tabcellsep 1/4 ‒\\
Siden langs ved \textit{Nidjo}-Elv i Syd-vest igiennem slet Land til \textit{Nidjojaure}\tabcellsep 1/4 ‒\\
Derfra over \textit{Nidjo-jaure} igiennem en slet Dal til \textit{Raude-gorre}, eller Skare\tabcellsep 1/4 ‒\\
saa over Myr, og Vande \textit{Savækjaure} i Syd-vest til \textit{Sorme-jaure}\tabcellsep 3/4 ‒\\
Derfra i Syd-vest over et kort field, \textit{Serbak} til Mitten af \textit{Nedre Bolma jaure}\tabcellsep 1/4 ‒\\
\tabcellsep _______\tabcellsep 2 FieldMile\end{longtable} \par
 \hspace{1em}\par
\textit{Sorme-jaure} er rundagtigt, et par Bøsseskud over stort, derfra rinder Aaen i Vester mitt i \textit{Nedre Bolma-jaure}.\hspace{1em}\par
\textit{Serbak}, Sønden for \textit{Sorme}-Aaen, er et fladt field, høyt paa Nordre Ende med en Tind paa, nedhældendes paa Siderne, ovenpaa bart, paa Sidene med Moese og Bierk paa, 1/4 Miil langt fra Søer i Nord, halv saa bredt.\label{Schn1_100740} \par 
\begin{longtable}{P{0.7684151785714285\textwidth}P{0.06261160714285714\textwidth}P{0.018973214285714284\textwidth}}
 \hline\endfoot\hline\endlastfoot En anden Vej Sønden for den næstforrige er Fra \textit{Veinæss}-Botten i Syd-vest til \textit{Ræpen}Elv\tabcellsep 1/4 Miil\\
saa efter \textit{Ræpen}-Elv i Søer\tabcellsep 1 ‒\\
derfra i Vester til \textit{Gorre-Niunes}\tabcellsep 1/4 ‒\\
Dette \textit{GorreNiunes} er efter \textit{Dierje-jaure} fra Søer i Nord 3/4 Miil langt, halv saa bredt; Altsaa fares over \textit{Gorreniunes} i Syd-vest\tabcellsep 3/8 ‒\\
Siden i Syd-vest over Myr- og Bierkeland \textit{item} smaa VatsKiønner\tabcellsep 1/2 ‒\\
\tabcellsep _______\tabcellsep 2 3/8 Mil.\end{longtable} \par
 \hspace{1em}\par
Næstberørte \textit{Dierge-jaure} er fra Søer i Nord 3/4 Miil langt, 2 Bøsseskud bredt; see \textit{p.} 341. Søndenfor dette \textit{Dierge-jaures} Søndre Botten 1/4 Miil ligger\label{Schn1_100828} \par 
\begin{longtable}{P{0.808078231292517\textwidth}P{0.02891156462585034\textwidth}P{0.013010204081632653\textwidth}}
 \hline\endfoot\hline\endlastfoot \textit{Gieksoive}, noget høyt, 1/4 Miil fra Øster i Vester langt, halv saa bredt, slet ovenpaa, fladt paa Sidene, allestedz Reen-moeset, Fra \textit{Dierge-jaures} Søndre Botten imellem dette \textit{Gieksoive} og \textit{Gorreniunes} gaaer et Skare, navnlig \textit{Sara-jugge}, i Syd-vest, som imellem de 2 Fielde er 1/4 Miil bredt. Fra \textit{Sara-jugge} bestaaer Landet i Syd-vest af smaa stillstaaendes Vande og Myrland med noget Bierk paa til en sterk Bierkeskoug\tabcellsep 3/4 Miil\\
Bierkeskougen er i Syd-vest indtil Øvre \textit{Bolma}-Elv, nær hvorved og Furreskoug er\tabcellsep 3/4 ‒\\
\tabcellsep _______\tabcellsep 1 1/2 Mil\end{longtable} \par
 \hspace{1em}\par
Landet imellem \textit{Nedre Bolma-jaure} og \textit{Væke-jaure} i Syd-vest siges her at være 1 3/4 Miil: forklared deels af 36te Vidne, deels af 27de og 29de Vidner, som her for Retten mødte:\hypertarget{Schn1_100890}{}Vidner i Finmarken. Vardøe Præstegield.\label{Schn1_100892} \par 
\begin{longtable}{P{0.7620689655172413\textwidth}P{0.06128526645768025\textwidth}P{0.026645768025078367\textwidth}}
 \hline\endfoot\hline\endlastfoot Mitt fra \textit{Nedre Bolma-jaure} i Sydvest bestaaer Landet af slet Myr- og Bierke-Mark med staaendes Vand i, hen til under \textit{Galdoive}\tabcellsep 1/2 Miil\\
\textit{Galdoive} er over bredt\tabcellsep 1/4 ‒\\
Derefter J Syd-vest er det slet, myred, med smaa staaendes Vande, og Bierke-Riis, alt hen til \textit{Vækejaure}\tabcellsep 1 ‒\\
\tabcellsep _______\tabcellsep 1 3/4 Miil\end{longtable} \par
 \par
\par
\textit{Galdoive} er fra Søer i Nord 1/2 Miil langt, og 1/4 Miil over bredt, slet ovenpaa, paa nordre og Østre Sider noget bratt, paa de andre Sider fladtvoren, overalt med Reen-Moese paa. Dette Land imellem \textit{Nedre Bolma} og \textit{Væke-jaure} bruge Norske \textit{Tana-Finner}, ved Moese-Trang: Men \textit{Vækejaure}, og deromkring have, og bruge \textit{Arisbye}-Fælles\textit{Finner}. Sønden for \textit{Gald-oive} er ved et Lægd derfra adskilt\par
\textit{Auxse-Niunes}, rundt, et par Bøsseskud over stort, med Moese paa; Om Høsten sidde \textit{Arisbye-Finne} her ved paa Vestre Side, men ei om Vinteren; thi her er ingen Skoug til Brændehved.\hspace{1em}\par
\textit{Landet imellem Øvre Bolma-jaure og Vækejarue}.\par
\textit{Øvre Bolma} siges her at ligge fra \textit{Væke} i Øster 1 Miil; ‒ Fra \textit{Dierge-jaure} i Syd-vest 2 Mile, ‒ fra \textit{Nedre Bolmajaure} i Søer til Vesten 1 Miil; Nær Vesten for Øvre \textit{Bolmajaure} er et lavt Field\par
\textit{Jako-vadda}, rundt, 1/2 Miil over stort, slet ovenpaa, og paa Sidene fladt, overalt moeset, uden Skoug; Vesten for dette \textit{Jakovadda} er Landet myret, moeset, med smaa Bierke-Riis, alt hen til \textit{Vækejaure} 1/2 Miil.\par
Om Høsten bruge \textit{ArisbyeFinner} dette Land imellem \textit{Øvre Bolma} og \textit{Vækejaure}, ikke om Vinteren; thi der er ingen Brændehved. ‒\par
Jmellem \textit{Væke-jaure}, og \textit{Otzjok} fra Oster i Vester er 1 FieldMiil.\hspace{1em}\par
Da nu Retten forestillede disse Vidner særdeles 27de og 29de Vidner, at de \textit{Norske Tana}- Field-\textit{Finner} have udsagt pag: [306 f.] at gaae fra \textit{Bolma} til imod \textit{Vækejaure}, og disse deres her giordte \textit{Deposition} siunes at stride derimod; saa forklarede begge nemlig 27de og 29de Vidner det saaledes, at \textit{Norske Tana Finner}, og fælles \textit{Arisbye-Finner} bruge det Land imellem \textit{Bolma} og \textit{Væke-jaure} tilfælles, at forstaae det \textit{Øvre Bolma}-Vand. ‒\hspace{1em}\par
\textbf{Raamerker} imellem de \textit{Norske Varanger-Finner} og \textit{Norske Tana-Finner} fra \textit{Varangers}- Fiord i Vester:\par
36de \textit{Vidne} vandt: at \textit{Varangers} Field\textit{Finner} gaae fra \textit{Varangers} Vestre og Østre Bottene i Vester til imod \textit{Goldevara}, og i Syd-vest hen mod \textit{Nedre-Bolma-jaure} imod de Norske \textit{Tana}- field-\textit{Finner}; J forrige Tider da \textit{Varangers} field\textit{Finner} vare sterke og formuendes, have de gaaet end videre i Syd-Syd-vest over 3 FieldMile til imod \textit{Øvre Bolma-jaure}; Men siden de ere bleven fattige og faa, fare de nu omstunder ikke saa vidt fra \textit{Varanger}-Fiord, hvor de maa søges deres Næring af Søen.\hspace{1em}\par
35de Vidne mindes, at \textit{Varanger}Finner fordum have fisket Lax i den Elv imellem \textit{Øvre} og \textit{Nedre Bolma-jaure}: men for deres Fattigdoms skyld, som næst for meldt, skeer det nu ikke.\hypertarget{Schn1_101175}{}Schnitlers Protokoller V.\par
\textbf{Raamerker} imellem de \textit{Norske Varanger}- og \textit{Arisbye}-fælles-\textit{Finner:}\par
31de 35de 36de og de øvrige Vidner vandt, af deres Forældre og gamle Folk at have hørt, at Raamerket imellem \textit{privative Norske Finner}, og fælles \textit{Arisbye-Finner} har været fra gammel Tid, end og for Amtmand \textit{Lorkes} Tid.\par
1. \textit{Foss-Holmen} i \textit{Tana}-Elv\par
31de 34de 36de Vidner og fleere af tilstædeværende Almue have hørt, at fra \textit{Foss-Holmen} i Syd-ost 1/8 Miil har Raamerket været\par
2. \textit{Sapes-duoder}, fra Nord-vest i Syd-ost 3/4 Field-Miil langt, noget smalere, ovenpaa slet, fladvoren paa Sidene, med Reen-Moese paa overalt; Af dette \textit{Sapesduoder} har den halve, neml. Vestre Deel til \textit{Arisbye} ‒ og den halve, nemlig Østre Deel til \textit{Varangers Finner} hørt. Videre har Raamerket gaaet til\par
3. \textit{Gald-oive}, fra \textit{Sapesduoder} 1 FieldMiil i Syd-ost, før \textit{pag.} 343 beskreven. Derefter Fra \textit{Gald-oive} i Syd-Syd-ost 1/8 Miil til\par
4. \textit{Jako-vadda}, \textit{pag.} 343 her beskreven, saaledes at Hælften af \textit{Galdoive} og \textit{Jakovadda} til de \textit{Norske Varangers} ‒ og Hælften til \textit{Arisbye} fælles \textit{Finner} har hørt. Fra \textit{Jakovadda} i Øster 1/8 Miil til\par
5. \textit{Øvre-Bolma-jaure}, hvor Skiellet har gaaet over dets Nordre Ende.\par
Om \textit{Arisbye}-fælles\textit{Finner} have brugt noget Land Sønden for \textit{Øvre Bolma-jaure?} det vidste de Norske \textit{Varangers} Vidner ikke: men det vidnede de, at \textit{Arisbye-Finner} ei Østen for \textit{Øvre Bolmajaure} i gamle Tider have gaaet, ei heller have Rett til at gaae:\par
35de 36de 37de 38de 41de og 42de Vidner opgav, at i deres Tid have de \textit{Arisbye} fælles \textit{Finner} gaaet Østen for Øvre \textit{Bolma-jaure} ind paa \textit{Varangers privative Norsk} Grund, og for 2 Aar siden har en Deel deraf med deres Huus og Boeskab nærmet sig endnu videre, og gaaet forbi den Norske \textit{Ny-elv} til \textit{Garde-jaure} og \textit{Garde-jok}, hvilket \textit{Gardejaure} ligger kun 1/4 Miil Sønden for denne \textit{Norske}\textit{Varangers}-Fiord paa \textit{Norsk} u-tvisted Grund, hvor de have ligget, og brugt Vild Reen-Skøtterie;\par
Dette samme stadfæstede og de 2de fra \textit{Arisbye} nys komne 27de og 29de Vidner: Over hvilken de \textit{Arisbyers} Jndtrængsel den tilstædeværende Almue sig besværgede, see \textit{p.} 348.\par
Efter Rettens Tilspørgende: Om Vidnerne ikke have hørt, at \textit{Skaar}Aae fordum af Norske \textit{Varangers Finner}, og \textit{Arisbye}-fælles\textit{Finner} har været brugt tilfælles, hvilken \textit{Skaar}Aae 3/4 Miil Norden for \textit{FossHolmen}, gaaer ind i \textit{Tana}-Elv, og \textit{Arisbye} fælles Vidner \textit{pag:} 325 have beraabt sig paa?\par
De \textit{Norske Varangers Finner}-Vidner svarede: Hertil aldrig at have fornommet, eller derom hørt. ‒\par
Hvorpaa Retten fremkaldede de 2de fra \textit{Arisbye} komne Finner, som med de andre \textit{Arisbye-Finner}\textit{pag:} 326 dette Fælleskab i \textit{Skaar}Aaen havde bevidnet, for at \textit{confronteres} med disse nærværende \textit{Varangers} Vidner; hvilke 2de ere de 27de og 29de Vidner her i \textit{protocollen}\textit{pag.} 326 før afhørte;\par
Det 27de Vidne blev ved sit forrige Udsagn, indført \textit{pag.} 326. Men det 29de Vidne erklærede sig saaledes, at han først for 2 eller 3de Aar siden om dette Fælleskab i \textit{Skaar}Aaen havde hørt tale paa det Svenske Ting i \textit{Arisbye}. Førstbem.te 27de Vidne forklarer det \hypertarget{Schn1_101499}{}Vidner i Finmarken. Vardøe Præstegield. Fælleskab at have bestaaet derj, at, som han har hørt, skal \textit{Varangers} Norske og \textit{Arisbye}- fælles\textit{Finner} have brugt denne \textit{Skaar}Aae hver Andet Aar om hinanden, til Bæverfangst, som nu \textit{cesserer}, see \textit{p.} 347.\par
Jmod denne de \textit{Norske} Vidners \textit{Deposition} have fælles \textit{Arisbye} Vidner opgivet deres\par
\textbf{Raa-Merker} imellem dennem, \textit{Arisbye-Finner}, og \textit{Norske Varangers Finner item} andre, som følger:\par
26de 27de 28de og 29de Vidner \textit{pag:} 325 sagt,\par
1. \textit{Skaaraae}; dog har 29de Vidne, ved \textit{Confrontation}, paa en viss Maade fragaaet det; og 27de Vidne ved \textit{Confrontation} forklaret sig, at det Fælleskab i \textit{Skaar}Aae fordum har bestaaet i Bæverfangst (ikke tales om noget videre) hvilket Brug nu \textit{cesserer}.\par
2. \textit{Væke-jaure}, som baade \textit{Norske}, og \textit{Arisbye-Finner} have opgivet \textit{pag.} 325. Men 27de og 29de Vidner have \textit{pag.} 343 tillagt\par
3. \textit{Auxse-Niunes}, ved 1 Miil Østenfor \textit{Væke} ‒\par
4. \textit{Jako-vadda} tæt Norden derfor; forklarendes sammestedz, at Landet imellem \textit{Væke} og \textit{Bolma} er for begge Parter til fælles Brug. (Som de forklare, have forstaaet om Landet imellem \textit{Øvre-Bolma} og \textit{Væke}, da det kan \textit{concilieres} med de Norske Vidner 36. og 35. deres \textit{Deposition}, \textit{pag.} 343 og alle de Øvriges Vidnesbyrd \textit{pag.} 344 etc.)\hspace{1em}\par
\textbf{Arisbyes Raa-merker} imod \textit{Jndiager} i Søer, og \textit{Neidens Finner} i Øster: af 27de og 29de Vidner udsagt:\par
\textit{Merker imod Jndiager} i Søer for \textit{Arisbyerne}\par
1) \textit{Saxe-jaure}\par
2) \textit{Betzoko-jaure}\par
3) Fra \textit{Østre Bolma-jaure} 1 Miils Strekning i Søer, omtrent. ‒\hspace{1em}\par
\textit{Merker imod Neiden} i Øster for \textit{Arisbyerne}.\par
Fra den Eene Miils Strekning Sønden for \textit{Øvre Bolma}, gaae de \textit{Arisbyer} i Øster, indtil de naae \textit{Neidens Finner;} med hvilke de have haft i forrige Tider \textit{Jje-jaure}, hvoraf \textit{Neidens} Elv udrinder, tilfælles: men hvordan nu bruges? vidste Vidnerne ikke.\par
Følger nu Beskrivelsen af \textit{Varanger}-Fiordz\textit{Søndre Land}-Side, som og kaldes \textit{Rafte}-Siden, at reigne fra \textit{Veinæss} Jndfiordz Gab ved \textit{Ny-elv}, hvor man \textit{pag.} 341 opholdte, J Øster, Bevidned af 33de til 42de Vidner, hvis Navne see \textit{pag.} 328 f. ‒\par
Paa denne Søndre Landz Side strax Østen for \textit{Ny-elv} møder\par
\textit{Karekfield}, noget høyt fra \textit{Varangers}-fiord op ad, slet oventil med smaa Houger og Tuver paa, fra \textit{Nyelv} efter \textit{Varanger}-fiord, 1/8 Miil i Øster bredt, og fra Nordvest i Sydost 1/4 Miil langt; bratt ud ad Fiorden i Nord og ad \textit{Nyelv} i Vester, paa de andre Sider fladtvoren, med Reen-Moese paa overalt. 1/4 Miil herfra over en Myr- og Moese-Dal i Øster er det Field\par
\textit{Garde-kiaro}, fra Syd-vest i Nord-ost 1/2 Miil langt, fra Vester i Øster, efter \textit{Varanger}- Fiord, 1/4 Miil bredt, ikke saa høyt, som \textit{Karek}-Field, slet ovenpaa, paa Nord-ostre Side til Fiorden bratt, ellers fladtvoren, med Moese paa alleveigne. Jmellem disse, \textit{Karek}field og \textit{Garde-kiaro} deres Søndre Ender, ligger det Vand\hypertarget{Schn1_101833}{}Schnitlers Protokoller V.\par
\textit{Garde-jaure} fra Nord i Søer 1/2 Miil langt, halv saa bredt, hvoraf Aaen \textit{Garde-jok} løber 1 1/2 Field-Miil, først i Søer, siden i Sydost i \textit{Neidens} Elv. ‒ Paa \textit{Gardekiaro} følger, efter en slet Myr-Dal, hvorj stillstaaendes Vand, imod 1/4 Miil\par
\textit{Vouste-kiaro}, fra Nord i Søer 1 Miil langt, vendendes sig med søndre Ende imod \textit{Bugøe}- fiordz Botten, halv saa bredt i Øster efter \textit{Varanger}-fiord; ikke saa høyt, som \textit{Garde-kiaro}, bratt i Øster, og i Nord til \textit{Varanger}-Fiord, saa og i Søer imod \textit{Bugøe}fiordbotten, men paa Vestre Side fladt, overalt Moese-groet. Med dette \textit{Voustekiaro} er i Øster fieldfast\par
\textit{Belsniunes}, saa høyt, som næstforrige, rundt, naaer i Øster til \textit{Falk-Vatten}, er 1/4 Miil over stort, ovenpaa slett, paa Sidene brattvoren med Reen-Moese paa, uden Skoug. \textit{Falkvatten} er fra Nord i Søer 1/2 Miil langt, 1 Bøsse-skud over bredt; Af dette \textit{Falkvattens} Søndre Ende rinder Aaen \textit{Falk-jok} 1/4 Miil lang, i \textit{Bugøe}fiord paa dens nordre Side, et par Bøsse-skud Østen for Botten. Strax Østen for dette Vatten er det Field\par
\textit{Gierge-vare}, langt i Øster til Syden efter \textit{Varanger}-Fiord 1 1/8 Miil, og rekker hen til \textit{Bugøe}- Fiordz Gab, hvortil den giør det Vestre Næss, bredt fra Nord i Søer, i Vestre Ende 1 Miil, mitt over 3/4 Miil, i Østre Ende 1/2 Miil; høyere, end de næstforrige, med Houger og Tinder ovenpaa, brattvoren paa alle Sider, med Reen-Moese paa allestedz, uden Skoug og Græss. Østen for \textit{Gierge-vare}, over \textit{Bugøe}fiord, møder\par
\textit{Brass-Havnen}, \textit{Bugøe}fiordz Syd-ostlige Næss, Dette \textit{BrassHavn}-field ligger imellem \textit{Bugøe}-Fiord og \textit{Kiø-fiord} (som og kaldes \textit{Neidens}Fiord, fra Nord i Søer næsten saa langt, som \textit{Kiøfiord} er i Søer, hvilket de slutte at være ved 2 SøeMile langt, Fra Øster i Vester naaer det fra \textit{Kiøfiord} eller \textit{Neidens}Fiord til \textit{Bugøe}fiord, med Næsset Spidz udstikkendes i Nord, men siden vider sig ud i Søer saa bredt, som Landet imellem begge Fiordene er, og rekker med sin Søndre Ende Sønden forbi \textit{Bugøe}Fiordz Botten; Ovenpaa nær Næsset er det tindet, men et Støkke derfra i Søer er det slet, paa Vestre, Nordre og Østre Sider til Fiordene bratt, men paa Søndre Side ind ad Landet fladt-nedhældendes, overalt moese-groed. Østen for \textit{Kiøfiord}, eller \textit{Neidens}Fiord paa det Sønd[r]e faste Landz Side, eller \textit{Rafte}-Siden, er\par
\textit{Vakiær}, et Field, som ligger med sin Nordre Ende langs efter \textit{Kaarsfiord} fra Vester i Øster hen til \textit{Bøgfiord}, eller \textit{Pass-vig}-Fiord 1 kort Miil bredt, og skal strekke sig imellem disse Fiorder fra Nord i Søer meget langt, hen imod \textit{Jndiager}-Land, hvor vidt? vidste man ikke, dog bliver det smalere i den Søndre Ende, er ovenpaa slet paa Vestre, Nordre og Østre Sider ad alle 3 Fiorder fladtvoren, overalt moese-groet. ‒ Paa Vestre Side ad \textit{Kiøfiord}, eller \textit{Neidens}- Fiord, og paa Østre Side ad \textit{Bøgfiord} eller \textit{Pasvig}fiord er god Bierke-skoug, paa Fiord-Brædene, paa Nordre Sides Fiord-Bræde til \textit{Kaarsfiord} er kun Bierke-Riis ‒\par
Længer i Øster Jngen her var bekiendt.\hspace{1em}\par
\textit{Landskabet} fra \textit{Varanger-Fiords} Søndre Land-Side, eller \textit{RafteSiden} i Søer imod \textit{Russisk- Norske} fælles \textit{Neidens Finner:}\hspace{1em}\par
Fra \textit{Veinæss} Jndfiord i Søer er Landet for \textit{pag.} 342 beskreven til \textit{Gieksoive inclus:} 1/8 Miil derfra i Søer er det Field\par
\textit{Kistepell-oive}, rundagtigt, et par Bøsseskud over stort, paa Vestre Side noget høyt, paa den Østere, lavere, med Reen-Moese paa. Nær Sønden for dette \textit{Kistepell-oive} er \hypertarget{Schn1_102107}{}33te og fg. Vidner i Finmarken. Vardøe Præstegield.\textit{Nakken-gargas}, 1/2 Miil langt i Søer, halv saa bredt, paa Vestre Side noget høyt med smaa Houger paa, paa Sidene fladt, Moese-groet. 3/8 Miil Sønden for \textit{Nakkengargas} er\par
\textit{Ull-vare}, fra Nord i Søer 3/8 Miil langt, 1/4 Miil over bredt, noget høyt, slet ovenpaa, paa Vestre Side brattvoren, ellers fladt ned ad, med Moese paa, og med Skoug neden under; Paa den Søndre Side herfra begynder Furreskougen, og neden under forbi den Søndre Side gaaer \textit{Neidens} Elv. ‒\par
Er saa i alt fra \textit{Veinæss}-Jndfiord i Søer op til \textit{Neidens} Elv 4 FieldMile.\hspace{1em}\par
Landet imellem \textit{Gieksoive} og \textit{Øvre Bolma-jaure} er før \textit{pag.} 342 beskreven.\hspace{1em}\par
Landskabet Sønden for \textit{Ny-elv} og \textit{Gardekiaro} (see \textit{pag.} 345 f.) bestaaer af smaa Houger med Reen-Moese paa, og af smaa stillstaaende Vande derimellem, vidt {1 FieldMil,} Derpaa følger i Søer slet Myrland, med noget Bierke-Riis paa, hen til \textit{Neidens} Elv, {3/4 ‒} Naar nu Ny-elv fra \textit{Varanger}fiord i Søer er {1/2 ‒ ___________} Saa er paa dette Sted fra \textit{Varanger} til \textit{Neidens} Elv den korteste Vej, nemlig {2 1/4 FieldMil.}\hspace{1em}\par
Landskabet Sønden for \textit{Vouste-kiaro} og \textit{Bels-niunæss} (see \textit{pag.} 346) er af smaa lave rundagtige Houger med Reen-Moese paa bestaaendes, uden Græss og Skoug, førend nær ved \textit{Neidens} Elv, hvor Bierk og smaa Furre staaer,\par
Er saa paa dette Sted Landet fra \textit{Varanger}-fiord, i Søer (som de slutte) til \textit{Neidens}- Elv 2 1/2 F: Mil.\hspace{1em}\par
Landskabet Sønden for \textit{Bugøe}fiord:\par
først Sønden for \textit{Bugøe}-Fiordz Botten: kommer en Aae \textit{Sør-elv} af Myr, og rinder 1/2 Field-Miil lang i \textit{Bugøe}-Fiordz Botten; Fra denne \textit{Sør-elvs} Oprindelse til \textit{Neidens} Elv i Søer 1/2 Miil; Paa begge Sider af \textit{Sør}-Elv er Myrland med Bierk; Samme Slags Land er imellem \textit{Sørelvs} OpKomst indtil \textit{Neidens} Elv, dog nær herved nogen smaa Furre; Saaledes er fra \textit{Bugøe}fiordz Botten til \textit{Neidens} Elv i Søer 1 FieldMiil. Dernæst imellem \textit{Bugøe}-Fiord selv i Nord, \textit{Kiø-Fiord} eller \textit{Neidens}fiord i Øster, og \textit{Neidens}-Elv i Søer bestaaer Landet af \textit{Brasshavn}-Field, beskreven \textit{pag.} 346, som naaer hen til Neidens Elv i Søer: dog er noget Bierk og smaa Furre imellem Fieldet og \textit{Neidens} Elv.\par
\textit{Landet} Sønden og Øster for \textit{Vakiær-vara} er her u-bekiendt, dog vides, at \textit{Pasvig}-Elv løber Sønden og Østen om \textit{Vakiær}, (som \textit{pag.} 346 er beskreven) ind i \textit{Bøg}- eller \textit{Passvig}-Fiord.\hspace{1em}\par
\textit{Raamerker} imellem de \textit{privative Norske Varangers-Finner}, og \textit{Russisk-Norske} fælles \textit{Neidens Finner} i Søer:\par
Den LandStrekning paa Søndre Side af \textit{Varangers}fiord, som \textit{Norske Varangersfinner} tilkommer, er fra Veinæss-Fiordz Botten i Vester til forbj \textit{Bugøe}fiordz Botten i Øster over 3 Field-Mile lang; Fra hvilken \textit{Bugøe}fiordz Botten i Øster \textit{Neidens} fælles \textit{Finner} vel vill tileigne sig Landet; men see 6 \textit{Vol.}\textit{p.} 423 f. Fra \textit{Varangers}fiord i Nord til \textit{Neidens} Elv i Søer er den Landstrekning (see ovenfor) 2 1/4 à 4 Field-Mile breed; ‒ Med denne Landstrekning imellem \textit{Varangers} Norske og \textit{Neidens} fælles \textit{Finner} holdes det nu saaledes:\hypertarget{Schn1_102454}{}Schnitlers Protokoller V.\par
J gamle Tider, da de \textit{Varangers} og \textit{Neidens} Field-\textit{Finner} vare mange og paa Reens-dyr formuendes, have de paa begge Sider med god indbyrdes Villie og Vennskab brugt dette Støkke Land tilfælles, at de \textit{Norske}, med de \textit{Neiden}ske om hinanden, have gaaet i Søer med deres Reen og Skiøtterie op til \textit{Neidens} Elv; og de \textit{Neidensfinner} ligeledes gaaet i Nord til \textit{Dierge-} og \textit{Garde-jaure}, og brugt Fiskerie i samme Vande; Fremdeles naar Fiskeriet i \textit{Kiøfiord} eller \textit{Neidens}Fiord har slaaet feil for \textit{Neidensfinner}, have disse besøgt \textit{Varangers} Hoved-Fiord uden for eller Norden for \textit{Gierge}-Field med deres Fiskerie, men ikke længer i Vester ind i Fiorden: Derimod de \textit{Norske Varanger-Finner}, naar fiskeriet hos dennem har skiortet, have søgt ind i Gabet af \textit{Kiøfiord} eller \textit{Neidens}fiord, og ikke videre ind:\par
Mens nu, efterat de \textit{Norske Varangers}field-\textit{Finner}, og \textit{Neidens} fælles Field-\textit{Finner} paa begge Sider ere bleven faa, og fattige paa Dyr; Saa sidder i denne heele FieldStrekning paa Søndre Land-Side kun 1. Norsk Field-\textit{Finn} ved \textit{Sør-elv}, som har saa faa Dyr, at han ei kommer Sønden for Sør-elv; de \textit{Neidens} fælles \textit{Finner} udarmede, ligge nu omstunder Sønden for \textit{Nei}- dens Elv, og af Fattigdom paa Dyr, komme ei paa Nordre Side af \textit{Neidens} Elv: Dog have de \textit{Norske Varangers} saavel Søe- som Field\textit{finner} vedligeholdet, og endnu holde den Skik, som de fra Arildz Tid brugt have, at (som de ingen anden Furre-Skoug have) saa hugge de af Furre-Skougen nær ved \textit{Neidens} Elv Timmer til deres Baaders og Gammers Bygning.\par
Jmidlertid have \textit{Arisbye} fælles Field-\textit{Finner} Vestenfra \textit{arriperet} Leiligheden til at søge dette \textit{vacante} Field-Rom imellem \textit{Varangers}Fiord og \textit{Neidens} Elv med deres Reen og Reen- Skøtterie, og for 2de Aar siden, uden Forlov af \textit{Varangers-finner}, begyndt, at gaae med deres Familier og Reen til \textit{Garde-jok}, og med deres Skøtterie Østen derfor, til imod 1/4 Miil nær \textit{Sør}- Elv. see pag. 344.\par
\textit{Raamerke} imellem \textit{Norske Varangers-Finner} og \textit{Neidens} fælles \textit{Finner} i Øster Angives at være ved Søe-Siden ad \textit{Varangers}fiord\par
\textit{Bugøe}-Fiord, en Jndfiord af \textit{Varangers} hoved-Fiord, imod 3 Søemile Østen for \textit{Veinæss}- Fiordz Botten, eller 3 Mile Østen for \textit{Varangers Østbotten}, og fra \textit{Hennøerne} i Gabet af \textit{Varangers}- fiord ved dens Søndre Landz Side, omtrent 6 Søe Mile i Vester, saaledes at halve \textit{Bugøe}fiord, nemlig den Østre Deel skulle tilhøre de fælles \textit{Neidens} ‒ og den andre halve nemlig Vestre Deel de Norske \textit{Varangers Finner;} Følgelig tileegne \textit{Neidens} Finner sig \textit{Brasshavn}field, som ligger strax Østen for \textit{Bugøe}fiord, og \textit{Vakiær}field Østen for \textit{Brasshavn}-field, og \textit{Kiøfiord} eller \textit{Neidens}fiord, ligesom \textit{Passvigs}-fælles \textit{Finner} tileegne sig \textit{Bøg}fiord eller \textit{Passvig}-fiord, der Østen for \textit{Neidens}fiord. ‒\par
Dette sagde nogle af de ældste nærværende \textit{Varangers} Vidner, med den Forklaring, at i gammel Tid have \textit{Neidens Finner} saaledes villet have det med \textit{Bugøe}fiord og \textit{Brasshavn}- Field; See meere Forklaring i 6te \textit{Volum.}\textit{p:} [416]. Dog vides derhos, at \textit{Norske} Boemænd, eller \textit{Normænd} fra arildz Tid have taget, og endnu tage deres Brændehved af \textit{Kiøfiords} eller \textit{Neidens}fiordz og \textit{Bøgfiords} eller \textit{Passvig}-fiordz Bierke-Skouge, saa og deres Reen-Moese for deres Creaturer baade af disse \textit{Brasshavns-} og \textit{Vakiær-} saa og af fleere Østenfor værende Fielde, og det uden Betaling, og uden Modsigelse af de fælles Russiske \textit{Finner}. ‒\par
At agte, at forberørte \textit{Vakiær}-Field, som ligger imellem \textit{Neidens} og \textit{Passvig}fiorder, giør Bøyde-Skiellet imellem \textit{Neidens} og \textit{Passvigs Finne}-Bøyder.\hypertarget{Schn1_102829}{}33te og fg. Vidner i Finmarken. Vardøe Præstegield.\par
\centerline{\textbf{Øerne}}\par
ved \textit{Rafte} Siden, eller den Søndre Landz Side i \textit{Varangers}fiord, hvorledes de imellem de \textit{privative-Norske} Undersaattere, og fælles Russ[i]ske \textit{Finner} findes brugte og deelte:\par
\textit{Bugøe} straks uden for det Vestre Næss af \textit{Bugøe}fiord benyttes alleene af \textit{Norske Varangers} Undersaattere, see herom \textit{pag.} 334.\par
\textit{Kiøholm}, ved 1/2 Miil Østenfor \textit{Bugøe}, strax uden for det Østre Næss af \textit{Bugøe}-Fiord, bruges alleene af \textit{Neidens} fælles \textit{Finner} om Sommeren, see \textit{pag.} 334.\par
\textit{Skoggerøe}, fra \textit{Kiøholmen} i Øster til Sønden 1/8 Miil, liggendes imellem \textit{Neidens}- og \textit{Passvigs}-Fiorder, har før været beboed af \textit{privative Norske} Undersaattere, men nu øde; dog bruges den endnu som en Alminding, alleene af de \textit{Norske} til Høe-Slotte. See \textit{pag.} 334 f.\par
\textit{Kielmøe}, fra \textit{Skoggerøe} i Øster 1/8 Miil, liggendes i Gabet af \textit{Bøgfiord}, eller \textit{Passvig}fiord, besiddes om Sommeren af \textit{Passvigs} fælles \textit{Finner:} Dog benyttes derhos, som en Alminding, baade af \textit{Norske} Undersaattere, og af fælles \textit{Passvigs Finner}, til at tage Eederduun, Multer og Fugle-Egg der.\par
\textit{Reenøe} inde i \textit{Passvig}-fiord, fra \textit{Kielmøe} i Søer 1/2 Miil, har for en halv Snees Aar siden af \textit{privative Norske} Undersaattere været beboed, men nu er øde: Dog har Bøkkeren af den \textit{Norske}\textit{Vadsøe}-Handel derpaa sin Stue staaendes, og \textit{Passvigs} fælles \textit{Russe-Finner} have ei holdet her til.\par
Efter dette tilkom den Forklaring, rørendes \textit{Bugøe}-Fiord, som hører til \textit{pag.} 348.\par
Omtrent 8 \textit{Norske Varangers} Søe-\textit{Finner} boe i Botten, og paa Fiord-Brædene af \textit{Bugøe}- Fiord, som for 4. 5. à 9 Aar sig der have nedsatt, hvoraf de 2de om Sommeren fløtte ud til Gabet paa Østere Side af \textit{Bugøe}fiord. ‒\par
For 3 à 4 Aar siden er bleven omtalt, og fastsatt, til hvem \textit{Bugøe}Fiord skulle være deelt, og hvorledes dermed skulle holdes ‒ see 6.te \textit{Volum:}\textit{p:} [423]\par
J \textit{Bugøe}Fiordz Botten kommer Sønden-fra \textit{Søer-elv}, 1/2 Miil lang fra Søer i Nord. Ved denne \textit{Søer}-Elv sidder en \textit{Norsk} Field-\textit{Finn}. Herom see nærmere Forklaring i 6te \textit{Volumens}\textit{pag:} [423]. Men Østen for hannem er ingen af de \textit{Norske Finner}, som tilholder, at forstaae paa den Søndre Landz Side.\hspace{1em}\par
Til Sp. 3die Svar: Fisk falder i \textit{Varanger} fornemmelig Torsk til Rodskiær, Sey, Hyse, Steen-Bid, Ourer, Lange, Søe-Brassem, Flyndre, Noget af Hveete, Hval, lidet af Elv-Lax og Øreter. ‒ Sild i Havet er nok, men de \textit{Norske} Jndbyggere have ei Sildgarn; ‒ \par
J \textit{Bøg}- eller \textit{Passvigs}Fiord gaaer Silden jævnlig til hvert Aar, og fanges af \textit{Passvigs}-fælles\textit{Finner} med Nod-garn; Did komme og \textit{Russerne} fra \textit{Cola} næsten aarlig med 4. 6. à 7 Jægter, hver 5 à 6 Læster drægtig, og kiøbe Silden af \textit{Passvigs Finner}.\par
Merkværdigt er her, at en liden smal Fisk, kaldes Lodden, næsten hvert eller hvert andet Aar i denne \textit{Varanger}fiord frem for andre \textit{Norske} Fiorder om Paaske Tider mangfoldig indkommer, som trækker Torsken og anden stor Fisk i utallig Hob efter sig; Paa hvilken Tid en Fiske-riig Fangst for Jndbyggerne falder; Paa den Tid at Elvene af Jisen løsnes, siger man her, at Lodden forlader Fiorden, og den anden store Fisk drager af med den.\par
Fugl er her, som i \textit{Alten}, \textit{pag.} 226.\hypertarget{Schn1_103148}{}Schnitlers Protokoller V.\par
Sp. 4. Sv: Vinter-Havn for 3. à 4 Skibe er kun i \textit{Vadsøe}Sund, inde i \textit{Varanger}fiord, men om Sommeren rømmes der flere.\par
J \textit{Bussøe} Sund imellem \textit{Vaardøe} og det faste Land er alleene Sommer- men ingen VinterHavn for Skibe, for den sterke Strøm, der gaaer. ‒\par
J \textit{Syltevig}Fiord paa Nordre Land-Side ad \textit{Nord}-Søen paa Søndre Side af Næsset er en Havn for Jægter, og et par Skibe, men alleene om Sommeren; thi om Vinteren fryser den til.\par
J \textit{Kiøfiord} eller \textit{Neidens}fiord er \textit{Brasshavn} for Jægter.\hspace{1em}\par
Sp. 5. Svar: \textit{VarangersFiord} regnes i Gabet imellem \textit{Vaardøe}, og i Syd-Syd-ost derfra \textit{Henn-øerne}, 8 Søe-Mile over breed; Stikker mitt fra Gabet i Vester, dog med sin Botten i Vester til Norden fra \textit{Vaardøe} sterke 7 Søe-Mile, men fra \textit{Henn-øerne}, meenes 9 Søe Mile lang;\par
Det Nordre Næss af \textit{Varangers} faste Land heder \textit{Kiberg}, for \textit{pag.} 338 beskreven. Det Søndre Næss af det faste Land vides ej her: see \textit{pag.} [416].\par
\textit{Varangers}fiord, efter sin Nordre Land-Side skiær fra \textit{Kiberg} i Vest-Syd-Vest til \textit{Ekkerøe}{3 Søe Mile,} Siden i Vester til Norden hen til \textit{Østbotten}{4 ‒ _______ 7 Søe Mile}\par
Fra \textit{Hennøerne} til \textit{Holmgraa-Næss}, det Østre Næss af \textit{Pass-vig}fiord, som ligger lige imod \textit{Stor-Ekkerøe}, meenes, i Vest Sydvest {5 Søe Mile} Derfra til \textit{Øster-Botten}{4 ‒ ___________ giør 9 Søe Mile}\par
\textit{Bredden} af \textit{Varangers}fiord er, som meldt, i Gabet, 8 Søe Mile. Siden trænges den sammen, saa at imellem \textit{Stor-Ekkerøe}, ved Nordre Land-Side 3 Mile fra \textit{Kiberg-Næss}, og \textit{Holm}- graa-Næsset paa Søndre Landz Side er 2 Søe Mile. Jmellem \textit{Stor-Vadsøe} ved Nordre LandSide, 1 Miil fra \textit{Ekkerøe}, og \textit{Bugøe} ved Søndre Land-Side er 1 Søe Miil. Jmellem \textit{Mortens Næss} paa Nordre LandSide 2 Mile fra \textit{StorVadsøe}, og \textit{LatNæring} paa Søndre Land er 3/4 Søe Miil. Jmellem \textit{Angsnæss}, og \textit{Veinæss}-Odden 1/2 Søe Mil. Fra disse Næsse bliver Fiorden smalere alt indtil Bonden af \textit{Øster-Botten.}\par
\centerline{\textbf{Jndfiorder af Varangerfiord:}}\par
Paa Nordre Land-Side er Jngen.\par
Paa Søndre Landz Side, at reigne fra \textit{Bøg}- eller \textit{Passvig}-Fiord, (videre i Øster Folket her ej vare bekiendt);\par
\textit{Passvig}-Fiord, eller \textit{Bøgfiord} er imellem det Østre Næss, \textit{Holmgraa-Næss}, og det Vestre Næss paa \textit{Skogger-øe, Galle-Niarg}, i Gabet 1/2 Miil over breed; Lang er den i Søer til det Næss, \textit{Haakierring-Næss}, som skiller den Jndfiord, \textit{Langfiord}, fra \textit{Passvig}-Fiord 1 1/4 Søe Miil, siden stikker den i Søer, hvorlangt? vides ej her, see \textit{pag.} [416]. \textit{Passvig}-fiord er imellem \textit{Haa-kierring-Næss}, og det Østre faste Land 1/8 Miil viid. Denne \textit{Passvig}-fiord har paa Vestre Side en Jndfiord,\hypertarget{Schn1_103440}{}33te og fg. Vidner i Finmarken. Vardøe Præstegield.\par
\textit{Langfiord}, hvis Østre Næss er bem.te \textit{Haakierring}, det Vestre Næss er \textit{Vakier-Næss}, paa Norsk, \textit{Ellinghavn-Næss}, derimellem er \textit{Langfiord} imod 1/4 Miil breed, stikker først i Sydvest 1/4 Miil ‒ siden i Søer 1 Miil, saa i alt 1 1/4 Miil lang. Vesten for \textit{Passvig}-fiord 1 Miil, uden om og forbi \textit{Skoggerøe} er\par
\textit{Neidens}- eller \textit{Kiøfiord}, dens Østlige Næss er \textit{Kiesvad-Niarg}, paa \textit{Skoggerøe}, dens Vestre Næss \textit{Brasshavn-Næss} paa det faste Land, hvorimellem \textit{Neidensfiord} i Gabet er 1/8 Miil breed, og stikker ind i Søer 2 1/4 Miil lang. 1/2 Miil Norden for Botten stikker en Tværfiord,\par
\textit{Kaarsfiord} i Ost-Nord-Ost imellem \textit{Skoggerøens} Søndre LandSide, og det Jndre faste Land, 1/8 Miil breed, hen til \textit{Vakier-Næss}, ved 1 Miil lang, da den udgaaer i \textit{Passvig} eller \textit{Bøg}- fiord. Strax Vesten om \textit{Brasshavn-Næss} indgaaer\par
\textit{Bugøe-Fiord}, hvis Østre Næss er samme \textit{Brasshavn}, det Vestre Næss heder \textit{Bugøe-Næss}, derimellem er \textit{Bugøe}-Fiord 1/2 Miil breed, og gaaer i Syd-vest 1 1/8 Miil lang; 1 Miil i Vester til Nord møder\par
\textit{Gandvig-Bugten}, 1/4 Miil i Gabet viid, en god 1/4 Miil dyb, dens Østre Næss heder \textit{Gandvig-klubben}, dens Vestre Næss \textit{Kuknæss}, hvor 5 \textit{Norske Søefinner} boe. Fra \textit{Kuknæss} til \textit{Veinæss} Jndfiord, eller \textit{Søerfiord} er i Vester 1 Miil, at forstaae til \textit{Veinæss}-Gab:\par
\textit{Veinæssfiord} dens Nordre Næss er \textit{Veinæss}-Odden, dens Søndre Merke er \textit{Ny-elvs} Mund paa Søndre Landz Side, hvorimellem \textit{Veinæss}-Fiord er i Gabet breed 1/4 Miil og i Vester ligesaa dyb. ‒\par
\textit{Varanger-Fiord} har 2 Bottener, nemlig den Østre og Vestre, som ved \textit{Angsnæss} adskilles;\par
\textit{Øster-Botten} er \textit{Varangers} Hoved-Botten, og stikker 1/2 Miil dyb i Vester.\par
\textit{Vester-Botten} er Nordenfor \textit{Øster-Botten}, og har til Søndre Næss bemældte \textit{Angsnæss}, og til Nordre Næss \textit{Ambonæss} paa det Nordre faste Land; Denne Vester-Botten er imellem disse 2de Næss et par Bøsseskud breed, og stikker i Vester til Søer noget over 1/2 Miil.\hspace{1em}\par
Sp: 6. Svar: Elve og Aaer i \textit{Varanger}fiord udløbende fra \textit{Nordre} Land-Side:\hspace{1em}\par
1. \textit{Jacobs}Elv nær ved \textit{Findenæss} see \textit{pag.} 339.\par
2. \textit{Bergebye}-Elv, Vesten for \textit{Jndre-Bergebye} 1 Bøsseskud, oprinder af en VatsKiøn Nordenfra, 1 Miil lang i \textit{Varanger}fiord. J denne Elv gaaer Lax og Øreter op, men ej længer, end til Fossen, som er ved 4 Bøsseskud fra Fiorden; J Vattenet, hvoraf den kommer, fanger [man] Røer. Fossen i denne Elv falder bratt ned af et Berg, 3 Favn over breed, og 4 Favner høy.\par
Fra Søndre Landz Side:\par
3. \textit{Passvig}- eller \textit{Bøg}-Elv udrinder af \textit{Jndiager}-Vand, eller \textit{Anar-jaure} Søndenfra, og løber i \textit{Passvig}- eller \textit{Bøgfiords} Botten. J denne El[v]s Mund er en stor Foss. som ei oplade Laxsen.\par
J \textit{Langfiords} Botten kommer kun en Bæk.\par
4. \textit{Neidens} Elv kommer Vesten-fra af \textit{Ji-jaure}, og løber Øst-lig i \textit{Neidens}- eller Kiøfiord, 3/8 Miil Norden fra Botten; Denne er den Lax-rigeste.\par
5. \textit{Søer}-Elv rinder Sønden-fra af Myr 1/2 Miil lang i Botten af \textit{Bugøe}-Fiord.\par
6. \textit{Falkjok} see \textit{pag.} 346.\hypertarget{Schn1_103768}{}Schnitlers Protokoller V.\par
7. \textit{Garde-jok} see \textit{pag.} 346.\par
8. \textit{Ny-elv}, fra \textit{Bugøe}fiordz Gab 2 1/2 Miil i Vester, og fra \textit{Veinæss}botten 1/4 Miil i Øster, rinder Sønden-fra 1/2 Miil lang i Gabet af \textit{Veinæs} Fiord, hvorj Lax gaaer op.\par
J \textit{Veinæss}fiord gaaer ingen Elv.\par
9. \textit{Ræpen}-Elv løber i \textit{Karlsbugten} af \textit{Østbotten}, see pag. 341. J denne Elv ere adskillige Fosser, hvoraf de betydeligste\par
(a) \textit{Røuvde-gordze} i ElvMunden, bratt nedstubendes af Berget, 2 Favner høy, og 3 Favner breed; hvorover ingen Lax slipper op i Elven.\par
(b) \textit{Gako-gordze}, et paa Bøsseskud Sønden for næstforrige, brat nedfaldendes af et BergSkare, et par Favner høy, lidet smalere.\hspace{1em}\par
Sp. 7. Svar: Myre ere besynderligen imellem \textit{Tana}-Elv og \textit{Varangers} Botten\par
(a) \textit{Russægge}, strax Østenfor \textit{Gollevara}, 1/8 Miil Østen herfor\par
(b) \textit{Loake-ægge}, nær ved \textit{Varangers} Vester Botten.\par
Dale gives her ingen, som er at sige af.\hspace{1em}\par
Sp. 8. Svar: \textit{Bønder}Gaardene her, som \textit{Normænd} og Søe\textit{finner} boe i, ere Torv-Gammer, med Bierke-Stænger, eller Bræder indentil fastgiorde eller beklædde; de som Field\textit{finnerne} holde til i, kaldes Tielder, eller Tield-Gammer, (som er det samme som hos os Telter) bestaaende af nogle opreiste Bierke-Stænger, omklædde med Vadmel, Lærret, eller Green-Dekkener, rundagtige nedentil, og opad spidz, med et aabent Hull i, hvorigiennem Røgen opgaaer. Hvor nu Een eller nogle enten Søe- eller Field-\textit{Finner} have deres Gamme, det kaldes en \textit{Finne} Bye af den Mandz Navn, som holder derj til.\par
\textit{Normænd} fløtte ikke. Men Søe\textit{finnerne} have gemeenlig 2 à 3 Byer, eller Gammer, som de i eet Aar omskifte, nemlig Sommer- og Vinterbye, en Deel have og Høste-Bye; SommerBye have de ude i Fiorden for Fiskeriets Skyld; Vinter-Bye for Brændehved, og Moese-Skyld for deres Creaturer til Fieldz; Høstebye haves i Fiord-Bottene, for Græssets Skyld. Field- \textit{Finnerne} fløtte deres Byer, eller Tielder hver 2 eller 3die Uge, at, naar Reenen har opbeedet et Sted, fare de andenstedz hen med Reenen.\hspace{1em}\par
Sp. 9. \textbf{Skoug} i \textit{Vaardøe} Præstegield:\par
Paa Nordre faste Land, kalded \textit{Varanger-Næss}, imellem \textit{Nord}Søen og \textit{Varanger}fiord, dets nordre Siø-Bræde, \textit{item} inde paa Næsset ovenpaa Fieldene, saa og paa Søndre FiordBræde alt fra \textit{Vaardøe} hen til imod \textit{Varangers} Botten er ingen Slags Skoug, ikke engang til Brænde-fangst: Men \textit{Normænd} paa Nordre Land-Side brænde Torv, og de paa Søndre FiordBræde eller paa Øerne tage deres Hved deels af \textit{Varangers} VesterBotten, deels af den Søndre LandzSide, neml. af \textit{Bugøe- Neidens}- og \textit{Passvig}-Fiordene, som høre vel \textit{Russisk-Norske} fælles Finner til, men de \textit{Norske} have fra Arildz Tid brugt, at hente deres Hved derfra: dog er i \textit{Syltevig}fiord noget Bierk. \textit{pag.} 337. J \textit{Varangers} Vestre Botten er den ligeste Bierke-Skoug, men i den anden \textit{Østerbotten} og \textit{Veinæss}-Jndfiord er kun Bierke-Riis.\par
Paa Søndre Land-Side eller \textit{Rafte}-Siden, som tilhører de \textit{Norske}, dens Fiord-Bræder, nemlig fra Botten i Øster til \textit{Bugøe}-Fiord, er slet ingen Skoug, førend en 3 à 4 Mile fra Fiorden \hypertarget{Schn1_104018}{}33te og fg. Vidner i Finmarken. Vardøe Prestegield. i Søer nær ved \textit{Neidens}Elv, som \textit{Neidens} fælles \textit{Finner} tilhører. Hvorfra og de \textit{Norske} Søe- \textit{finner} fra Arildz Tid have været vandte til at hente deres Furre-Planter eller Bræder til deres Baaders Bygning. J \textit{Bugøe- Neidens- Passvig}-Fiorder, som hører \textit{Russisk-Norske Finner} til, er Bierk til Brændefangst, hvorfra de Norske Betientere og Bønder af \textit{Nordre} Land-Side pleje fra Arildz Tid at tage deres Hved. Furre-skoug er ingen paa \textit{Norsk} Grund, men begynder Sønden for \textit{Varanger}fiord ved \textit{Neidens} Elv, hvorfra den vedholder i Øster og Syd-ost til over \textit{Passvig}- og \textit{Peisens} Elve alt hen til \textit{Kola} og \textit{Rusland}, og desuden i Søer igiennem \textit{Jndiagers}-Land alt op imod field\textit{Kiølen:} Men denne Furre-Skoug er paa fælles \textit{Norsk-Russisk} og \textit{Svensk} Grund og kan ei naaes til, eller komme til Nytte for Norske \textit{Varangers} Undersaattere, uden den nærmeste ved Fiorden.\par
Merkeligt er, at \textit{Normænd} behielpe sig med Torv til at brænde: Men hverken Søe- eller Field-\textit{Finn} trives ved den Lugt, hvorfor de kan ej være, hvor Brændehved fattes. ‒\hspace{1em}\par
Sp. 10. Svar: \textit{Vildt} og \textit{Jnsecter}, som i \textit{Alten}\textit{pag.} 228 undtagen Tiødere og Eeghorn, gives her ej.\hspace{1em}\par
Sp. 11. Svar: \textit{Leilighed} til \textit{Rødning} eller \textit{peuplering} er her næsten allevegne; Thi det er Havet og Fiorden, som er fiskeriig, der underholder, og lidet Rom behøves til at sætte en Gamme eller Hytte paa. ‒ Da Ny-Byggere, som fra \textit{Sverrigs}\textit{Kimi}- og \textit{Torne-Lapmark}, fra \textit{Storfinnland} og fra \textit{Arisbye}- og \textit{Jndiagers} Fieldland hid ere komne, og komme, begaae sig ret vel.\par
Fra \textit{Russisk-Norsk} fælles Grund, som \textit{Neiden, Pasvig, Peisen-Kola etc}. kommer Jngen at nedsætte sig; Formodentlig derfor, at den \textit{Russiske} Geistlighed af \textit{politiske} Aarsager indbilder den gemeene Mand, at de \textit{Norske} have ei saa rett en Troe, som \textit{Russerne}. ‒ see følgende næste Spørsm.\hspace{1em}\par
Sp. 12. \textit{Fielde} her omkring ere beskrevne \textit{p.} 335 f. Paa Fieldene omkring \textit{Varangers} fiord kunde være Leilighed vel for en Snees Field\textit{finner} at nære sig.\hspace{1em}\par
Sp. 13. Svar: \textit{Toldsted} ingen; Den \textit{Octroyerede} Handel paa \textit{Vaardøe} og \textit{Vadsøe} forsiuner Landet.\hspace{1em}\par
Sp. 14. \textit{Næring} bestaaer næsten eene i Fiskerie særdeles af Torsk til Rodskier og Sej.\hspace{1em}\par
Sp. 15. Svar: \textit{Mineralia} vides ei her af\hspace{1em}\par
Sp. 16. ‒ Nærmeste Jord-Pladser paa Vestre Side i Sør af \textit{Kiøllefiords} Præstegield er \textit{Bolma} ved \textit{Tana}-Elv, 2 Mile i Vester til Syden fra denne \textit{Veinæss}Fiord; J Nord ere \textit{Omgangs} Jord-Pladser, 4 Mile Vesten for \textit{Makour}-Pladser; Paa Østre Side i Søer er \textit{Bomeni}, en \textit{Russisk}\textit{Finne}-Bye, 3 1/2 Dags Køer om Vinteren i Ost-Syd-ost fra \textit{Bugøe}Fiord i \textit{Varanger}; ‒ J Nord ender NordSøen Landet.\hspace{1em}\par
Sp. 17. Svar: \textit{Vej} fra \textit{Varanger}-Botten til \textit{Arisbye} i Syd-vest fares over Land til \textit{Bolma}, siden efter \textit{Tana}-Elv; Om Vinteren i Kieredster, om Sommeren roes did paa Elven.\par
Til \textit{Jndiager} i Søer;\hypertarget{Schn1_104338}{}Schnitlers Protokoller V.\par
Til \textit{Bomen} i Sydost: dog kun om Vinteren i Kieredster; Jkke om Sommeren, thi Hester haves ej, ej heller kan bruges for de mange Vande, Elve og Myrer i de Udørkener.\par
Mellem \textit{Russisk Bomeni} og \textit{Varanger} er ingen kommet nogen Tid over Land;\par
Til \textit{Arisbye} og \textit{Jndiager} kommer Jngen af de Norske, Thi der er intet for dennem at hente, eller faae: Men \textit{Finnerne} fra disse Stæder komme undertiden endog om Sommeren hidgaaendes, som trænge til Søefisk.\par
Først i \textit{Februario} holdes fra gl. Tid i \textit{Varangers} Botten det \textit{Varangers} Market, hvortil og søge \textit{Torne}Stadz Borgere, som her i Landet kaldes \textit{Qvæner}, med Kram-Varer, og kiøbe Sej-Fisk og Reen-Skind ‒\hspace{1em}\par
Sp. 18. før besvaret, hvor mange \textit{Norske Finner} her, \textit{pag:} [337 f.]\hspace{1em}\label{Schn1_104409} \par 
\begin{longtable}{P{0.6968944099378882\textwidth}P{0.07391304347826086\textwidth}P{0.07919254658385093\textwidth}}
 \hline\endfoot\hline\endlastfoot Sp. 19. Svar: \textit{Fælles Finner}, saavidt her vides, i \textit{Jndiagers} Field-Land skal være ved\tabcellsep 46.\tabcellsep \textit{Famil.}\\
i \textit{Neidens district} nu\tabcellsep 8.\\
\tabcellsep \multicolumn{2}{l}{_________}\end{longtable} \par
 \hspace{1em}\par
Sp. 20. Svar: J \textit{Jndiager} Land ere Field-Markene og Fiske-Vandene, endogsaa det store \textit{Jndiager}-Vand imellem dets \textit{Finner} skiftede, saa at hver har sit visse Romm. J \textit{Neiden, Passvig} og \textit{Peisen} er Field-Marken tilfælles, men Elvene og Vandene skiftede.\hspace{1em}\par
Sp. 24. \textit{Om Kiølen} eller \textit{Field Ryggen} vidne \textit{43de og 44de Vidner}, hvis \textit{personalia} før \textit{pag.} 330 ere indførte, og i sær\par
\centerline{\textbf{43de Vidne,}}\par
som i forleden Aars Sommer fra \textit{Kimi-Lapmark} er gaaet over \textit{Jaurisduøders} Østre Ende, der hvor \textit{Piil} Elv opkommer, og har seet \textit{Jaurisduøder} at bestaae af deels høye deels lave Fielde, og at Vande derfra rinde baade til \textit{Norge} og \textit{Sverrig}: dog veed ei, at navngive dette \textit{Jaurisduøders} særdeles Fielde; Der hvor han gik over \textit{Jaurisduoder}, var en Moese- og SteenDal med noget Bierk i, imellem 2 Fielde, hvis Navne han ei veed; Jmellem hvilke Fielde Dalen er 1/8 Miil viid; Foromtalte \textit{Piil}-Elv har sin Oprindelse af Kiøn, og, som han slutter, løber i Nord-vest til i \textit{Tana}-Elv, saa lang en Vej, som en løs Karl i een kort Dag gaaer, (hvilket vill holdes for 2 Field-Mile)\par
Efter Rettens Tilspørgende: Om Vidnet ikke vidste noget af \textit{Masel-jok}, eller \textit{Skiekkem}- Elv, som ligeledes skulle komme fra \textit{Jaurisduøder} der omtrent paa Laug? Svarede han: Nej: det vidste han ikke, ei heller havde han vidst Navn paa den opgivne \textit{Piil}-Elv, men efterat han var kommet ned til \textit{Jndiager-Bye}, hørte han af Folkene der, at det var \textit{Piil}-Elv, hvorved han var kommet over \textit{Kiølen}; dog saae han en anden Elv, hvis Navn er hannem u-bekiendt, kommer Østen-fra i denne \textit{Piil}-Elv.\hspace{1em}\par
Hvad Grændse-Skiellet imellem \textit{Norge} og \textit{Sverrig} monne være? det kunde Vidnet ikke si ge: men det vidste han vel, at \hypertarget{Schn1_104594}{}43de og 44de Vidner i Finmarken. Vardøe Præstegield.\textit{Kiølen} eller \textit{FieldRyggen} giøre\par
\centerline{1. \textbf{Jaurisduøder.}}\par
Hvorlangt dette strekker sig, vidste han ikke, at forstaae i Længden fra Vester i Øster; Landskabet paa den Nordre Side af Jaurisduoder der hvor han gik, var saaledes: Fra Piiljoks Oprindelse 1 Field-Miil i Nord-vest er Bierk, derpaa følge nogle bare deels slette deels spidse Fielde paa begge Sider af Elven, med Bierk i Dalene derimellem, og paa Elv-Brædene, indtil den anden forberørte Elv Østen-fra i denne \textit{Piil}-Elv indløber; Hvorefter deels Bierk, deels Furre ved Elven findes alt hen til \textit{Tana}-Elv.\par
\centerline{2. \textbf{Beldo-vadda}}\par
Som \textit{Finnerne} og kalde \textit{Laddegein-Nodak}, fordi de \textit{Svenske Torne}-Borgere fare denne Vei herover til \textit{Jndiager}-Bye; Thi \textit{Ladde} bemerker \textit{Svensk, Gein} en Vei, \textit{Nodak} vanlig. ‒ Dette \textit{Beldovadda} er en slet Mark uden Skoug; Hvor lang, og hvor breed den er? kunde Vidnet ei sige; Thi han har ei været derpaa, men faret Vesten der forbi: Dog har han iagttaget, at \textit{Beldo-vadda} ligger fra \textit{Jaurisduøder} i Øster saa langt, at man herfra kunde see \textit{Beldo-vadda}, og er imellem \textit{Jaurisduøder} og \textit{Beldo-vadda} deels Bierke-Skoug, deels fieldet Land; \textit{Jtem} at \textit{Beldo-vadda} har sin Strekning fra Vester i Øster, og at derfra rinde Vande til begge Sider i Nord og Søer; Saaledes løber fra dets Østre Endes Nordre Side af Myr-Kiønne\par
\textit{Afvel-jok} først i Nord-ost, siden i Nord-vest rundagtig i \textit{Jndiager}-Vand paa dets Søndere Side nær ved dets Vestlige Botten, hvor lang? veed ej. Fra den Søndere Side af dette \textit{Beldovadda} stikker en anden Elv, navnlig\par
\textit{Beldo-jok} i Sør ad \textit{Sverrigs}\textit{Vuones-jok}. Dette \textit{Beldo-vadda}, siger Vidnet, er Enden paa \textit{Kiølen} eller paa de høye skallede Fielde; Thi Østen derfor er en u-endelig Furre-Skoug, som rekker hen til \textit{Rusland}. ‒ Ligesaa er \textit{Afvel-jok} den Østerste Elv, som rinder nordlig til \textit{Jndiager}-Vand; Thi de Elve Østen for \textit{Afvel-jok}, som opkomme i den u-endelig Furre-Skoug, vende sig ad \textit{Russisk} Land. Paa Nordre Side af \textit{Beldo-vadda}, nær \textit{Kiølen}, er Bierk; Paa Søndre Side Furre-Skoug, siden Gran, u-endelig viid i Søer ind i \textit{Sverrig}.\par
\centerline{\textbf{44de Vidne}}\par
kan ei heller \textit{determinere} Grændse-Skiellet imellem \textit{Norge} og \textit{Sverrig}, men, som næst-forrige, vidner, at \textit{Kiølen}, eller Field-Ryggen giøre\par
\centerline{1. \textbf{Jaurisduøder}}\par
hvorfra Elve rinde til \textit{Norge}, og \textit{Sverrig}, hvor \textit{Tana}-Elv fra \textit{Borv-oive}, eller udaf \textit{Gaune-jaure} oprinder? Veed han ikke: Men om Vinteren har han været ved \textit{Piil}-Elven, eller paa \textit{Finsk}, \textit{Niules-jok}, et Støkke oven for eller Sønden for der, hvor denne \textit{Piil}-Elv med en anden, navnlig \textit{Skiekkem}-Elv sammenløber, saa han ikke kan sige, om denne \textit{Piil}-Elv kommer fra \textit{Jaurisduøder}, eller Østen derfor? Om af et Vand, eller Myr? Thi han har ikke været saa langt Søer, at han har været oven paa \textit{Jauris-duøder:} Men \textit{Piil}-Elv derfra, hvor han [har] seet, stikker \hypertarget{Schn1_104856}{}Schnitlers Protokoller V. først i Nord, indtil den gaaer ind i \textit{Skiekkem}-Elv; Der faaer den det Navn \textit{Skiekkem}, eller som den her udtales, \textit{Skietzem}, og rinder i Nord-vest en kort 1/2 Dags Reise, eller 1 god Miil lang i \textit{Tana}-Elv.\par
\centerline{2. \textbf{Beldo-vadda,} eller \textbf{Ladde-gein}}\par
som han har været ved; Det er en slet Mark med Moese paa, uden Skoug og Græss, og, som ham siunes, ligger fra Vester i Øster, hvor langt, og hvor bredt? veed ei. ‒ Fra dets Vestre Ende i Øster omtrent 1/4 Miil ligger et Vand \textit{Skietzem-jaure}, 1/4 Miil langt fra Vester i Øster, 3 à 4 Bøsse-Skud bredt; Af dette Vandz Vestre Ende gaaer\par
\textit{Skietzem-jok} (som andenstedz er udtalt \textit{Skiekkem-jok}) omtrent 1/2 Miil lang i Nord til Vesten i \textit{Piil}-Elv, hvorefter den med samme Navn \textit{Skietzem} fortløber i Nord-vest 1 god Miil lang i \textit{Tana}-Elv; Og denne \textit{Skietzem} er den Østerste Aae, som har sit Løb i \textit{Tana}-Elv. Og her, hvor \textit{Beldo-vadda} er, der er Enden paa \textit{Kiølen;} Thi Østen derfor er en u-endelig FurreSkoug, som og 43de V. \textit{pag.:} 355 har udsagt, stadfæster og det samme om \textit{Afvel-jok}, som bemeldte Vidne sammestedz har forklaret. Fra samme \textit{Beldo-vaddas} Nordre Side 1/4 Miil i Nord, og fra \textit{Skietzem-jaure} 1/4 Miil i Øster opkommer\par
\textit{Vasko}-Elv, og løber 3 Dagers Reise lang (som han slutter) i Nord-ost i \textit{Jndiager} Vands Vestlige Botten: dog gaaer den tilforn i et Vand \textit{Badar-jaure}, 2 Field-Mile Sønden-for \textit{Jndiager}- Vandz Vestre Botten; Af hvilket \textit{Badar-jaure} naar den udgaaer, faaer den det Navn \textit{Joudajok}. ‒ \textit{Badar-jaure} er rundagtigt, 1/4 Miil over stort: Jmellem dette \textit{Badarjaure} og \textit{Baldovadda}, dog nærmere til denne, tager \textit{Vasko}-Elv til sig fra Syd-ost en anden Aae,\par
\textit{Bost-jok}, kommendes af Myr, der er 1 1/2 Miil lang. Om Landskabet Sønden- og Norden for \textit{Kiølen} siger det samme, som næste 43 Vidne \textit{p:} 355.\par
\centerline{\textbf{45de og 46de Vidner}}\par
hvis \textit{personalia}\textit{pag.} 330 f. ere forklarede, vide intet af \textit{Kiølen} at sige; Thi ihvorvel de i \textit{Sverrigs}\textit{KimiLapmark} i \textit{Sombye} ere fødde, saa have de dog ej taget Veien over \textit{Kiølen}, men Østen forbi \textit{Beldovadda} faret igiennem den store Furre-Skoug, i Kieredster, om Vinteren, fra \textit{Sombye} i Nord til Vest, til \textit{Jndiagerbye}, den 1te i 5 ‒ den Anden i 4 Dage, hvorfra de siden have begivet sig hid til \textit{Varanger}.\par
Og denne Vej fra \textit{Sombye} i \textit{KimiLapmark} igiennem Skougen fare de Svenske Field- Betientere om Vinteren til \textit{Jndiager} efter Skatten, sommetidz i 3 ‒ og sommetidz i 4 Dage, ligesom Vejret falder til; Undervejs fra \textit{Sombye} 9. Field-Mile i Nord til Vesten møder et Field \textit{Siolosielge}, som strekker sig fra Syd-Vest i Øster, hvor langt? vides ej, 2 Field-Mile over bredt, paa begge Ender høyt og skallet, med et Skare i mitt paa, hvorigiennem fares; Siden reises fremdeles i Nord til Vesten til bem.te \textit{Jndiager;} hvilket er fra \textit{Sombye} did den korteste Vej.\hspace{1em}\par
Sp. 25. Svar: \textit{Landskabet} paa den Nordre Side af \textit{Kiølen} er (a) \textit{Jndiagers} Field-Land, (b) Norden derfor \textit{Neidens} fælles Land, og (c) Norden herfor det \textit{Norske Varangers} Land, eller \textit{Vaardøe} Præstegield; Vidden af dette Landskab Norden for \textit{Kiølen} bereignes saaledes:\hypertarget{Schn1_105128}{}45de og 46de Vidner i Finmarken. Vardøe Præstegield.\par
Fra \textit{Beldovadda} (Enden af \textit{Kiølen}) fares i Nord-ost, som meenes, 3 Dagers Reise til \textit{Jndiager}-Bye; Siden herfra i Nord til Vesten 3 sterke Dagers Reise til \textit{Varanger}fiordz ØstreBotten, hvilke 6 Dagers Reise man vill \textit{taxere} for omtrent {18 FieldMile.}\label{Schn1_105155} \par 
\begin{longtable}{P{0.8120446533490011\textwidth}P{0.037955346650998825\textwidth}}
 \hline\endfoot\hline\endlastfoot Dette nu nærmere at forklare, saa reignes fra \textit{Varangers}fiord i Søer til \textit{Neidens} Elv (see \textit{pag.} 347) det fælles Land som de \textit{Norske} med \textit{Neidens Finner} have haft i Fælleskab ved\tabcellsep 3 Mile.\\
\textit{Neidens Finners} Land Søndenfor \textit{Neidens} Elv gaaer i Søer til \textit{Guezejaure}, som meenes\tabcellsep 3 ‒\\
hvilket \textit{Guetze}-Vand bestaaer af 3de efter hinanden liggende Vande fra Sydvest i Nord-ost 3/4 Miil lange, et par Bøsseskud breede, og rinder Elven heraf dets Østlige Ende Nordlig i \textit{Neidens} Elv; Samme \textit{Gueze-jaure} ligger imellem \textit{Neidens} Elv, og \textit{Jndiager}-Vand, dog nærmere til dette, og fra \textit{Jndiagers} Bye i Nord-ost. Bliver saa \textit{Jndiagers District} fra \textit{Gueze-jaure} i Syd-ost til \textit{Beldovadda-Kiølen} ved\tabcellsep 12 ‒\\
\tabcellsep ___________\\
Som giør de\tabcellsep 18 FieldMile.\end{longtable} \par
 \par
(Denne Udreigning angav andre Vidner)\hspace{1em}\par
Længden af \textit{Jndiagers} Land begynder i Vester fra \textit{Øvre Tana}-Elv, hvor 2de \textit{Jndiagers}\textit{Finne-Familier}, og fleere ikke, have deres Sommer-Bye paa Østre Side af \textit{Tana}-Elv som hørt, 6 FieldMile Sønden for \textit{Karasjok} Kæften, og paa den Tid fiske Lax i \textit{Tana}-Elven, stængendes Elven heel over; Thi oven for disse 2de \textit{Familier} er Jngen meer, som bruger Fiskerie i Elven; Fra denne Sommer-Bye gaae samme 2de \textit{JndiagersFinner} langs efter den Østre Side af \textit{Øvre Tana}-Elv med deres VildReen-Skøtterie i Søer op til \textit{Skietzem}-Elvs Søndre Mund; Herfra i Søer ere der andre \textit{Jndiager-Finner}, som deres Reen Skiøtterie alt op til \textit{Kiølen} have. Og videre end til den Østre Side af Øvre \textit{Tana}-Elv og af \textit{Skietzem}Elv have \textit{Jndiagers Finner} ei Rett at gaae; Thi Vesten for samme Elve ere de \textit{Avjevara-Finner}, som have deres Brug til de Elvers Vestre Side; Vel hender det sig, at de \textit{Jndiagerer} gaae undertiden over Elven i Vester: men da skeer det med de \textit{Avjevarers} Forlov og Tilladelse.\par
Den som dette forklarede, var det 44de Vidne; see \textit{pag.} 330 og 355 og har været Een af de selvsamme \textit{Jndiager-Finner}, som fra Barns Been har siddet i den \textit{Jndiagers} Sommerbye Østen for \textit{Øvre Tana}-Elv, indtil j afvigte Vinter, da han til denne \textit{Varangers}fiord er nedfløtted, hvis Forældre og for hannem have haft samme Sommer-Bye.\par
Hvorlangt \textit{Jndiagers} Land strekker sig i Længden imod det \textit{Russiske Finnebye, Søndergield} i Øster? vidste her Jngen; ei heller forbem.te \textit{Jndiager Finn;} Thi deres Field-Mark er imellem dennem skifted, og følgelig bliver hver gerne i sit Fieldleje.\par
Dette Landskab norden for Kiølens \textit{Beldovadda} indtil \textit{Jndiager-jaure} bestaaer af Bierk, Furre og noget Gran, iblant hinanden, sommestedz tæt, og sommestedz tønn-adspredt; Hvilket de \textit{Jndiagerer} saaledes bruge, at til \textit{Baderjaure} og et Støkke Sønden derfor ligge de med deres tamme Reen, men den \textit{district} Sønden derfor forbeholdes til VildReen-Skøtterie. ‒\par
J denne Skoug er bare smaa Skoug-groede Houger, og ingen skallede Fielde, uden imel\hypertarget{Schn1_105417}{}Schnitlers Protokoller V. lem \textit{Jndiagers} Bye og \textit{Beldovadda}, dog nærmere til dette, tæt ved den Østre Side af \textit{Vasko-jok} paa Nordre Side af \textit{Bostjok}, et, navnl. \textit{Marasti} rundagtigt, bart, med Reen-Moese paa, hvor stort? vidstes ej, som bruges til VildReen-Skøtterie.\par
Landet Norden for \textit{Jndiager}-Vand til Elven \textit{Neiden} bestaaer af Myr, FurreSkoug og Vande. ‒ Landet Vesten for \textit{Jndiager}Vand har Furre og Bierk med smaa lave Houger og stillstaaende Vande i. ‒ Landet Østen for \textit{Jndiager}Vand har stor Gran- og Furreskoug med Vande, og lave Houger imellem, uden høye skallede Fielde. ‒\par
Landet Østen for \textit{Kiølen Laddegein}, eller \textit{Beldovadda} bestaaer af den foromtalte u-endelige Furre-Skoug med Vande i, alt hen til \textit{Rusland;} Hvilken Skoug all kommer ikke, ei heller kan komme til noget Brug, uden for \textit{Finnerne} til HuusBehov. ‒\par
\textit{Jndiager-Vand}, paa \textit{Finnsk, Anar-jaure} er det største af \textit{Finmarkens} ferske Vande, som sluttes, fra Vester i Oster til Nord 2de Dagers Reise langt (som man vill holde for 8 FieldMile) halv saa bredt, og gaaer smal ud i den Vestre Botten, med u-tallige store og smaa Holmer i, deels med Furre bevoxne, deels Reen-mosede, hvorpaa \textit{Finner} og om Sommeren tilholde, Fiske-rigt; \textit{Jndiager Finne}-Bye, paa \textit{Finnsk:}\textit{Anar-Sid} med Kirken staaer ved dets Vestre Ende, 1/2 Miil Norden for \textit{Jouda}-Elvs Udløb i \textit{Jndiager}Vand, og norden for dette samme \textit{Jndiger}Vand. ‒ Af dette \textit{Jndiager} Vandz Østre Ende udrinder \textit{Passvig}-Elv nordlig i \textit{Passvig}- Fiord, en Jndfiord af \textit{Varanger}-Fiord. ‒\par
Midlertid man med dette Forhør var \textit{occupered}, Kom Posten Hvorfore man maatte \textit{licentere} Retten, paa nogle Dage, da man \textit{expederede Relations} og andre fornødene Skrivelser til fornødne Steder, som siden efter Forhørets Slutning ommeldes. \textit{p.} 361 f. Efter at det var bestilt, foretoeg man\hspace{1em}\par
d. 3 \textit{Decembr.}\textit{Examinationen} over Landet Sønden for \textit{Kiølen} paa den Kongelig Svenske Side:\par
Dette Søndre Land bestaaer af Skoug og Vande, først Bierk, siden Furre, og omsider Gran med lave Houger i, foruden høye skallede Fielde; Ved Elvene, som rinde af Vandene, er noget Græss.\par
Nær den Østre Ende af \textit{Jaurisduøder}, tæt Sønden for \textit{Kiølen} ligger det Vand \textit{Spørkojaure}, fra Nord-ost i Sydvest 1/8 Miil langt, halv saa bredt, deraf udgaaer Aaen \textit{Spørko-jok} i Søer med mange Kroger i Elven \textit{Gækel}, som ligeledes oprinder paa \textit{Jaurisduøder}, Vesten for \textit{Spørko;} Efter denne Foreening kaldes Elven \textit{Gækel}, og løber i Søer 8 Field-Mile lang i \textit{Beldojok}, som kommer fra Nord-Nord-ost af \textit{Kiølens}\textit{Beldo-vadda} dets Østre Ende; Ved samme Sted, hvor \textit{Beldo-jok} løber sammen, kommer fra Nord-vest den Aae \textit{Vonæss-jok}, og løber i Søer 30. Field-Mile i \textit{Kimi}-Elv. \textit{Vonæss-jaure} ligger omtrent fra Vest-Nord-Vest 1 Miil langt, 1/4 Miil bredt, Vesten for \textit{Gækel}. \textit{Kimi}-Elv kommer Østen fra af det Vand \textit{Kimi-jaure;} hvor- langt der er imellem, og hvor stort \textit{Kimi-jaure?} vidstes ikke. Ved denne \textit{Vonæs-jok}, 10 Mile Sønden derfra, hvor den med \textit{Beldojok} sig har foreenet, paa Vestre Side af Elven, er det Markested\par
\textit{Kittel}, hvor fordum imod 40. \textit{FinneFamilier} have boet, men for Fattigdom er en Deel derfra til \textit{Norge} fløttede; de Folk der have baade Reen, og Jordbrug; Dette \textit{Kittel} er det sidste \hypertarget{Schn1_105691}{}46de og 47de Vidner i Finmarken. Vardøe Præstegield. Sted fra \textit{Kiølen} at reigne, som er Knegt-fri og Sønden derfor er \textit{Kvoller}Bye det første paa Vestre Siden, der har Soldater-Udskrivning. Østen for \textit{Kittel} 12 Field-Mile ligger det\par
\textit{Saaddekill}, et Markested, hvor Kirken staaer, og fleere Folk, end i \textit{Kittel}, have boet af samme Nærings Brug. ‒ \textit{Saaddekill} ligger ved den Elv \textit{Kittes} paa Vestre Side; ‒ Denne \textit{Kittes-jok} løber i Søer i \textit{Kimi-jaures} Østere Ende. ‒ 8. Mile Østen for \textit{Saaddekill} ligger\par
\textit{Sombye}, der har 19 \textit{Finne-Familier}, der nære sig alleene af Reen, og 8 Bønder, som bruge Jord. Folkene af dette \textit{Sombye}, saa og de af \textit{Kittel} søge Kirke i \textit{Saaddekill}. 6 FieldMile i Syd-ost fra \textit{Sombye} Markested er det Markested\par
\textit{Keme-Bye} noget nær Østen for \textit{Keme-jok}, som rinder i Syd-vest i \textit{Kimi-jaure}; J denne \textit{Kemebye} ere 8 \textit{FinneFamilier}, som have Reen og Jordebrug, og søge til \textit{Saaddekill} Kirke. J Sydost fra \textit{Kemebye} er det Markested\par
\textit{Gvellojaure} ved et Vand af samme Navn, hvorfra Elven rinder i \textit{Kimi-jaure}; Her boe \textit{Finner}, som søge til \textit{Kimi-jaures} Kirke. Omtrent i Øster fra \textit{Gvellojaure} ligger\par
\textit{Kusan}, en Finnebye og Markested, hvor Folkene holde Reen og andre Creaturer til Jordebrug; Her er en Kirke, hvis Præst tillige betiener \textit{Jndiager-Finner}.\par
\textit{Kimijaure}bye ligger Vesterlig fra \textit{Gvellojaure}, havendes Kirke og Bønder, og er den 1te Bønde-Bye paa Østre Side, som har Soldater Udskrivning.\par
Af disse opreignede \textit{Finne}-Byer ere fælles Russisk-Svenske (1) \textit{Kusan}. (2) \textit{Gvellojaure}, (3) \textit{Kemebye}, (4) \textit{Sombye}, og (5) \textit{Saaddekill}, som ere de Østerste, hvor Sverrigs Crone har Geistlig og Verdzlig \textit{Jurisdiction}. ‒\par
Da dette forrettedes, ankom fra \textit{Jndiager} Land en Fieldfinn, som har sin Sommerbye ved \textit{Øvre Tana} Elv; Denne blev da for Retten kalded, at vidne sin Sandhed, som hannem bevist kunde være angaaendes Grændse- og Raa-merkerne imellem de \textit{Norske} og \textit{Svenske:}\par
\centerline{\textbf{47de Vidne}\textit{Samuel Samuelsen}}\par
fødd i \textit{Jndiager}, 41 Aar gl., gift, har 5 Børn, har sin Sommerbye ved den Østre Side af \textit{Øvre Tana}Elv, hid nu til \textit{Varanger} kommet i sit eeget Ærende;\par
Om Grændse-Skiellet imellem \textit{Norge} og \textit{Sverig} kan han intet sige, men han veed, at \textit{Kiølen} giører\par
\centerline{(1) \textbf{Jaurisduoder} og (2) \textbf{Beldovadda.}}\par
\textit{Jndiagers} Land gaaer i Vester til den Østre Side af \textit{Øvre-Tana}-Elv, og til den Østre Side af \textit{Skietzem} Elv, som kommer af \textit{Skietzem-jaure} ved den Vestre Ende af \textit{Beldo-vadda}, og videre forklares, som 44 Vidnet pag. 356 sagt haver; denne \textit{Skietzem} Elv er den Østerste, der løber til \textit{Tana}-Elv; Over denne \textit{Øvre Tana}Elv og \textit{Skietzem}Elv have de \textit{Jndiagerer} ingen Rett at gaae, med mindre det skeer med \textit{Avjevaras} og \textit{KoutokeinoFinners} Tilladelse; Ei heller gaae de over \textit{Kiølen} i Søer. Hans Sommerbye ved \textit{Øvre Tana} kan han ikke sige for vist, men slutter; at den ligger 3 Mile Sønden for \textit{Karasjoks} Kæften, og 3de korte Dagers Reise fra \textit{Skietzemjaure} eller \textit{Beldovadda} i Norden til Osten.\hypertarget{Schn1_106015}{}Schnitlers Protokoller V.\par
Om \textit{Jndiagers} Land forklarer han, at \textit{Jndiager} Bye, eller \textit{Anar-Sid} ligger fra \textit{BeldovaddaKiøl} i Nordost 3 korte Dagers Reise, og fra \textit{Varangers} Østre Botten, som han meener, 3 sterke DagsReise i Søer til Osten; LaugRetten \textit{taxerer} 1 kort DagsReise for 3 ‒ og 1 sterk DagsReise for 4 Field-Mile.\par
Fra \textit{Anar-Sid} fares i Syd-Syd-ost til \textit{Sombye} i \textit{Kimi LapMark} i 3 sterke Dagers Reise over Vande og igiennem Furre-skoug.\par
Den \textit{Svenske} Præst i \textit{Kusan} betiener den i \textit{Jndiager}, og kommer eengang om Aaret, nemlig om Vinteren, og er hos dem 2 ugers Tid.\par
Skatt til \textit{Rusland} betale de tilsammen 30 Rdl. \textit{Spesies}, til \textit{Sverrig} engang saameget, som til \textit{Norge}, hver Mand: dog siger det ikke til visse. Folket nærer sig af deres Reen, Fiskerie, Vild Reenskøtterie og Bæver-Fangst, og er FieldMarken og Vandene dennem skiftede imellem.\par
Landskabet imellem \textit{Kiølen} og \textit{Jndiagerjaure} beskriver han, som næste Vidne, og at der er noget Gran iblant den anden Skoug; Ligesaa siger om \textit{Vasko-jok} og \textit{Bostjok} det samme.\par
\textit{Ji-jaure}, hvoraf \textit{Neidens} Elv rinder, ligger 2 sterke Dagers Reise fra \textit{Varangers} Botten, saaledes at fra \textit{Ji-jaures} Østlige Ende kan være til \textit{Dierje-jaures} Søndre Botten ved {3 Mile,} og herfra til \textit{Varanger}Botten, alt i Nord {2 ‒ _______ 5 Mile}\hspace{1em}\par
\textit{Ji-jaure} strekker sig fra Syd-Syd-vest i Nordnordost 2 Mile langt, 1/4 Miil bredt, og tilhører \textit{Jndiager-Finner} alleene, saa at Raamerket imellem \textit{Jndiagers} og \textit{Neidens Finner} er i den Nordre Ende af \textit{Ji-jaure}; som kaldes \textit{Ordogiorgo:} dog tillade de \textit{Jndiagerer} de \textit{Neidens- Finner} at fiske i den Nordre Deel af samme Vand. At \textit{Arisbye-Finner} have Deel eller Fælleskab i dette \textit{Ji-jaure}, det veed han ikke, ej heller hørt har. Fra \textit{Ji-jaures} Søndre Ende kan være til \textit{Jndiager}Bye 1 ‒ men til Varangers Botten 2 sterke Dagers Reise.\par
Retten tilspurdte de tilstædeværende Vidner og Almue: Om de vidste, eller have hørt: Naar den \textit{Jndiagers} Kirke først er bleven bygged af de Svenske? De svarede eenstemmig, at Kirke har været til fra u-mindelig Aar, og have de ei andet hørt, end at der stedse har været Kirke.\par
\textit{Pag.} 345 er af det 27de Vidne sagt, at \textit{Norske Varangers} og \textit{Arisbye} fælles \textit{Finner} have brugt \textit{Skaar}Aaen hverandet Aar om hinanden til Bæver-Fangst; ‒ Dette vedbliver nu samme Vidne vel, mens forklarer derhos, ei at have hørt, at de \textit{Arisbyer} skall have haft Fælleskab med de Norske fra gammel Tid i \textit{Tana} Elv imellem \textit{Skaar}Aaen og \textit{Foss-Holmen}, til Lax-Fiskerie.\par
Hvorefter Retten paa dette Sted blev slutted, og bekiendtgiordt,: naar de til GrændseMaalingen \textit{Committerede Officerer} om 1. eller 2 Aar efterkomme, maa enhver, som tilsiges, være dennem \textit{assisterlig}; J sær beordres de Vidner 43. \textit{Hendrik Hendriksen}, og 44de \textit{Ole Samuelsen} til den Tid at være i Beredskab, om de skulle behøves, at fare til Field-\textit{Kiølen}, at anvise, den Østre Ende af \textit{Jaurisduoder} og \textit{Beldo-vadda}, hvorfra \textit{Skietzem}- eller \textit{Skiekkem-jok} og \textit{Piil}- Elv rinde; hvilke Vidne med fleere, som skulle behøves, \textit{Missions}Skolemester did fører; Om deres Underholdning paa denne deres Vej have de sig at melde hos Hr Amtmanden, eller Fogden.\hypertarget{Schn1_106292}{}Examinationens Slutning i Vaardøe Præstegield.\par
\textit{Veinæssfiord} i \textit{Varanger} d. 8 \textit{Decbr.} 1744.\hspace{1em}\par
\centerline{Peter Schnitler.}\hspace{1em}\centerline{L.S. \textit{Sivert Pedersøn} NyElf}\centerline{L.S. \textit{Anders Nielsøn} Næseby}\centerline{L.S. \textit{Melchor Olsøn} Næseby}\par
[Vol. V pag. 345‒361 er avskrevet i vol. VI, avskriften ikke trykt her.]\hspace{1em}\par
Efter Formeld paa \textit{pag.} 358\textit{expederede Major Schnitler} med \textit{Finmarkens Post}, imellem Forhørets Holdelse, følgende Breve:\par
\textit{Novbr}. 28de givet den Kongel. Grændse-\textit{Commissaire}, Hr Obriste \textit{Mangelsen} en \textit{Relation} om min \textit{Examinations} Fremgang til den Aae \textit{Skiekem} eller \textit{Skietzem}, som den Østerste Elv, der rinder i \textit{Øvre-Tana}-Elv fra \textit{Kiølen}, hvorhen-til de \textit{Jndiagerers District}, (hvorj \textit{Rusland interesserer}) rekker, og hvorhen-til, og ikke videre efter min u-forbribelig Meening, den anordnede \textit{Norske-Svenske} Grændsens Opmaaling kan gaa, helst \textit{Beldovadda}, eller \textit{Laddegein}, hvorfra bem.te \textit{Skietzem}-Aae kommer, giør Ende paa \textit{Kiølen} eller FieldRyggen ‒. Tillige og \textit{relateret} Velbem.te Hr Obrister de Vanskeligheder, der møde baade den \textit{provisionelle} Grændse-Befaring, og den Kongelig \textit{Norske} Grændse Maaling, udaf Hr Amtmand \textit{Kieldsons} Brev til \textit{Majoren} af 15 \textit{Octobr}. sidstl., hvoraf sendt ham \textit{Copie}, saavelsom af det Svar, \textit{Majoren} Velbem.te Hr Amtmand derpaa har givet; Hvad og \textit{Majoren} til de \textit{Nordlandske Civile} Betientere under denne Dags \textit{dato} skriver, deraf \textit{communiceret} Hr Obristen Gienpart.\par
‒ Samme \textit{dato} skrevet til Hr Amtmand \textit{Kieldson}: Siden Field-\textit{Kiølen} for \textit{Finmarken}, naar man vill regne fra \textit{Korsevara} til \textit{Beldovadda}, ei bliver saa lang, saa vill den \textit{provisionelle} Befaring med videre for \textit{Finmarkens} Norske Jndbyggere ei blive saa tung; og naar i Sommer taler med hannem, skal vi \textit{Conferere} derom til beste og letteste Maade for Undersaatterne. ‒\par
‒ \textit{dito dato} skrevet til de Kongel. Fogder og Sorenskrivere i \textit{Tromsøens- Senniens-} og \textit{Saltens} Fogderier, (thi i \textit{Helgelands} Fogderie holdes \textit{Lappeting} før) og formanet dennem, at \textit{continuere} med de foranstaltede \textit{Lappeting} i næste og følgende Aarer med Alvor og Høytidelighed; Hvorfor ved Sammenkomst skulle viise dennem \textit{Autoritet}.\par
\textit{Novbr}. 30de skrevet til Amtmanden over \textit{Nordland}, Hr \textit{Etats}-Raad \textit{Scheldrup}, og \textit{specificeret} de saa ofte giordte Anmodninger og Erindringer til ham, og særdeles til Fogden \textit{Ursin} over \textit{Tromsøen etc:}, angaaende den \textit{provisionelle} Grændse-Befaring, den Norske Søe\textit{finn Mikkel Pellegs} Tilladelse til at rødde og bygge i Almindingen mellem Grændsen og Fiorden, \textit{Lappe}Tingenes Fortsættelse i næste og følgende Aarer, Med Forestilling, hvad Ansvar der i vidrig Fald, om \textit{abrumperes}, ville følge.\par
‒ Sendt dette brev til Hr Obriste \textit{Mangelsen}, at lade til Amtmanden befordre, og givet ham \textit{part} deraf.\par
‒ Givet Fogden og Sorenskriverne i \textit{Tromsøens} og \textit{Senniens} Fogderier, hvilke det særdeles pa[a]gieldede, \textit{Copie} af forberørte Brev til Hr Amtmanden, til deres Efterretning.\hypertarget{Schn1_106610}{}Schnitlers Protokoller V.\par
‒ skrevet til Lensmanden i \textit{Skiervøe} Sogn, \textit{Clement Rasmussen Oderup} i \textit{Rotsund}, at betyde de 3de Vidner, \textit{Mikkel Pelleg, Jacob Nyensted}, og \textit{Lars Spen:} Om de ikke før have gaaet, da til Vaar, saasnart Jorden bliver bar, at give sig med den Norske \textit{Qvænangens Missions} Skolemester og Besigtelses Mænd til den \textit{provisionelle} Befaring; Da jeg (\textit{Majoren}) lover \textit{Mikkel PellegVestre Rostojaure} til Bøxel: Derimod, hvis det ikke skeer, skal jeg (\textit{Majoren}) mage det, at de skal dømmes og jages af Landet.\par
‒ De Hrr Grændse-Maalere, \textit{communiceret} den Underretning, \textit{Majoren} af \textit{Jndiagers Finner} i \textit{Varanger} har faaet, at \textit{Jndiagers district} gaaer i Vester paa Nordre Kanten til \textit{Øvre- Tana}, som der og kaldes \textit{Jndiagers} Elv, og paa Søndre Kanten ved \textit{Kiølen} til \textit{Skiekkem}, eller \textit{Skietzem}, den Østerste Aae, der fra \textit{Kiølen} rinder i bemeldte \textit{Øvre Tana} Elv; og at \textit{Beldovadda} eller \textit{Laddegein}, hvorfra bem.te \textit{Skietzem} løber, giør Enden paa \textit{Kiølen}.\par
\textit{Decembr}. 1.\textit{Relateret} Hr Obriste \textit{Mangelsen}, hvad \textit{Majoren} til Lensmanden i \textit{Skiervøe} d. 30 \textit{Novbr}. næst tilforn har skrevet; og forklaret for hannem en og anden Vanskelighed i \textit{Nordland}, som vill møde Grændse-Maalingens Fremgang.\par
‒ Samme \textit{dato} har \textit{Majoren} skrevet paa ny til Lensmanden i \textit{SkiervøeClement Oderup:} Om de 3de Vidner, \textit{Pelleg, Nyensted} og \textit{Spen} med Skolemesteren og BesigtelsesMænd have faret, eller fare til Fieldz, ad udviise Grændse-Løbet fra \textit{Kalkogaabb} til \textit{Pitsekiolme}, saa kunde de andre Vidner fra \textit{Ulfs- Reisens-} og \textit{Qvænangens} Fiorder (hvilke Støkke-viis have bevidnet samme Grændser) for den provisionelle Befaring forskaanes; Som han Fogden kunde sige. ‒\par
‒ d. 2 \textit{Decbr}. skrevet til Hr \textit{Etats}Raad og Amtmand \textit{Scheldrup}, at (som jeg af den \textit{provisionelle} BefaringsForretning over \textit{Vefsens} Fielde seer,) den \textit{Norske Øst-Lap}-Vidne, \textit{Thomas Siursen Nørtemand}, efter bekomne Budskab, ei har villet følge de Norske Befarings Mænd, og at disse, for Mangel af Vej-viisere, ei have kundet fortsætte deres Befaring til \textit{Østre Brak}- Field, det han alleene har vidst, i Grændsen at udviise; Thi begierer, at bem.te \textit{Nørtemand} for en \textit{Extra} Rett derfor \textit{fiscaliter} tiltales, og andre til Skrek \textit{exemplariter} straffes.\par
‒ \textit{Relateret} det til Hr Obriste \textit{Mangelsen}.\par
‒ \textit{Communiceret} det til Fogden Hr (?) \textit{Broch Angel}, at han i Tide derom nærmere ville erkyndige sig. ‒\par
d. 3die\textit{Reassumeret Examinations}Retten i \textit{Veinæss}-Fiord i \textit{Varangers} Botten, som varede til\par
d. 8de\textit{inclusive}.
\DivII[Des. 9.-11. Fra Veines til Vardøhus]{Des. 9.-11. Fra Veines til Vardøhus}\label{Schn1_106894}\par
d. 9de Reiset derfra igiennem \textit{Varangers}fiord i Øster til \textit{Vadsøe}{3 Søe Mile}\par
d. 10de Faret i Kieredzer over Land, som var slet-fieldet i Nord-ost til den \textit{Finne-Bye, KomagNæss} mod {3 ‒}\par
d. 11te Fra \textit{KomagNæss} i Nord-ost over Land, ligeledes slet og bart, til \textit{Svartnæss} ved \textit{Bussøe} Sund{1 7/8 ‒} over \textit{Bussøe}Sund i Øster til Vaardehuus{1/8 ‒ ___________ 5 Søe Mile}
\DivII[Des. 11.-12. Eksaminasjoner på Vardøhus]{Des. 11.-12. Eksaminasjoner på Vardøhus}\label{Schn1_106957}\par
‒ Begiert skriftlig af \textit{Commandanten}, Hr Obrist\textit{Lieutenant Passau}, at \textit{communicere} mig af \textit{Archivet} hvad til min anbefalede Grændse-Forretning kunde henhøre, med videre.\hypertarget{Schn1_106976}{}Forretted i Vardøe Præstegield.\par
d. 12te Faaet af \textit{Commandanten} det Svar, at han intet af \textit{Archivet}, som angik \textit{Finmarkens} Amt, havde;\par
‒ Foranstaltet \textit{Examinations}Retten paa \textit{Vaardøe}, angaaendes de \textit{Russiske} Grændser, til næste Ørkensdag.\par
‒ Talet med den Kongel. \textit{Norske} Foged paa \textit{Vaardøe}, om en \textit{Officer} kunde udrette nogt i \textit{Cola}, om han reiste did, til Grændsernes Udforskning? Han svarede mig, som \textit{Provsten} paa \textit{Vadsøe} før havde forstændiget mig, at fra \textit{Vaardøe} over Land om \textit{Varanger}-Botten var Vejen til \textit{Cola} en 50 Field-Mile, bekostelig og meget besværlig igiennem luter Udørkener; Jngen \textit{Norsk} Mand, uden Fogden farer herfra til \textit{Cola} nogen Tid, og derfor ville det forekomme \textit{Russerne} sælsomt og betænkeligt, om en fremmed \textit{Officer} did skulle komme, uden noget udtrykkeligt Ærende did at have.\par
d. 14de Holdet Forhør paa \textit{Vaardøe} angaaendes \textit{Ruslands} Grændser, men, som \textit{Ordren} lyder, at det under Haanden skee skulle, uden Eed: dog med Betydende: naar det forlangedes, skulle \textit{Deponenterne} stadfæste deres Udsagn med \textit{Corporlig} Eed; Forhøret varede til\par
d. 19de \textit{Decbr.} næstefter.[Referert i vol. VI pag. 415‒422.]\par
d. 20de  Hellig
\DivII[Des. 21.-22. Fra Vardøhus til Vadsø]{Des. 21.-22. Fra Vardøhus til Vadsø}\label{Schn1_107095}\label{Schn1_107097} \par 
\begin{longtable}{P{0.7274418604651163\textwidth}P{0.0869767441860465\textwidth}P{0.03558139534883721\textwidth}}
 \hline\endfoot\hline\endlastfoot d. 21de Reiset fra \textit{Vaardøe} over \textit{Bussøe}-Sund, og siden over Land i \textit{Kieredzer} i Syd-vest til \textit{KomagNæss} mod\tabcellsep 2 Søe Mile\\
d. 22de derfra ligeledes til \textit{Vadsøe}\tabcellsep 3 ‒\tabcellsep 5 S. Mile\\
\tabcellsep _________\end{longtable} \par
 \par
hvor man blev Julen over ‒\par
d. 28de og 29de holdet paa \textit{Skattøren} ved \textit{Vadsøe Examen} med 2de \textit{Bugøe}fiordz\textit{Norske} Søe \textit{Finner} over de Russiske Grændser ved \textit{Varanger}fiord. [Se vol. VI pag. 422‒424.]\par
d.30te Skrevet til \textit{Finmarkens} Foged om hans Betænkning, ang. \textit{Vaardehuus}-Fæstnings\textit{gva[r]nisons} Vedligeholdelse af \textit{Finmarkens nationale} Mandskab, med meere.\par
\centerline{\textbf{1745.}}\par
\textit{Jan}:4. og 5.\textit{Reassumeret} Forhøret med 2de andre ankomne \textit{Bugøefiords\textit{Norske}Søe- \textit{finner},ang. de Russiske Grændser. [Se vol. VI pag. 424 f.]}\par
d.6te Hellig\par
d.7de\textit{seqv}. Forfattet en \textit{Jnstrux} for Fogden over hvis han paa sin \textit{Malmis}-Reise angaaendes de Russiske Grændser havde at \textit{observere}, og givet hannem Tolken \textit{Hælset}, som Fogd-Hovmand, med.\par
d.14de Satt Retten paa \textit{Skattøren, examinerendes} Folk af \textit{Vadsøe}-Sogn over dette Landz Grændser \textit{ad Rusland}; Det til\par
d. 19de varede. [Se vol. VI pag. 425‒431.]\par
Som af et gammelt Skrift fornam, at de \textit{Kareler} og \textit{Karlstrand} skall have været paa \textit{Finmarkens} Grændser, saa foranstaltede derfor et Forhør paa \textit{Skattøren},\par
d.22de detsangaaende, som varede til\hypertarget{Schn1_107321}{}Schnitlers Protokoller V.\par
d. 23de\textit{inclusive} [se vol. VI pag. 431‒433.]\par
d. 24de Hellig.
\DivII[1745 Jan. 25.-febr. 3. Korrespondanse på Vadsø]{1745 Jan. 25.-febr. 3. Korrespondanse på Vadsø}\label{Schn1_107341}\par
d. 25. og\par
d. 26de\textit{Expederet} et \textit{Deductions} Skrivelse til Provsten over Øst-\textit{Finmarken} Hr \textit{Angel} anlangende \textit{Vaardehuuses} Besætning af \textit{Finmarkens} eegene \textit{nationale} Folk;\par
d. 28de Herpaa faaet Velbem.te Provstes udførlige Svar.\par
d. 29de\textit{Expederet} herover til \textit{Generalitet}[et] i \textit{Norge} min \textit{Relation}.\par
Som af \textit{Nordlands} og \textit{Finmarkens} Vidner erfaret, at de Svenske for omtrent 80 Aar have begyndt deres Christlige \textit{Mission} fra \textit{Tornestad} til \textit{Finmarkens}Søer-\textit{Fielde}. og for nogle og 40 Aar først oppbygget de \textit{Svenske} Kirker i \textit{Koutokeino}, ved \textit{Altens} Elv, og i \textit{Arisbye} ved \textit{Otzjok;} Saa har\par
\textit{dito Dato} til Amtmanden udferdiget en \textit{Deduction} derover, med Begier, i Amtets \textit{Archiv} og \textit{JustitzProtocollen} at lade eftersee: Om derj noget Beviis eller Spoer findes, at forberørte \textit{Finmarkens Søer-Field-Finner} for den Tid af de \textit{Norske Civile} Betientere med Retten, eller af den \textit{Norske} Geistlighed, som den nærmeste og \textit{competente}, ere bleven med Præstelig Tieneste forsiunede? da derover mig et Ting-vidned \textit{Document} maatte tilsendes. ‒ Samme \textit{Deducton} var tillige indretted til Provsterne i \textit{Finmarken}. ‒\par
d. 30 \textit{Januarj} Forfattet et ligelydende til Provsten i \textit{Ost-Finmarken}, Hr \textit{Albrecht Angell} i sær.\par
d. 31te Hellig.\par
\textit{Febr:}1te Til Hr Amtmand \textit{Kieldson}, og Provsten i \textit{Ost-Finmarken} Hr \textit{Angell} givet tilkiende, at den \textit{Svenske Arisbye Capellan} og \textit{Missionaire} paa \textit{privative Norske} Grund forretter \textit{Ministerialia} ved \textit{Tana}Elv i \textit{Kiøllefiords} Gield, forestillendes, at det i Tide vel vill hæmmes.\par
d. 2den Kyndelsmiss\par
d. 3die Et ligelydende af næstforrige meddeelt Provsten \textit{Angell.}
\DivII[Febr. 4.-8. Fra Vadsø til Varangers Østbotten]{Febr. 4.-8. Fra Vadsø til Varangers Østbotten}\label{Schn1_107574}\par
d. 4de Givet mig i Kieredser paa Vejen fra \textit{Vadsøe} i Vester \textit{ad Varangers} Botten til \textit{Varanger}-Market, hvor \textit{Svenske Qvæner} og Russisk-fælles \textit{Finner} kunde ventes at komme; den Dag kørt til \textit{Bergebye}{2 1/2 Søe Mile}\par
d. 5te i Vester til \textit{Varangers}Østbotten{1 1/4 ‒} og siden til en \textit{Finnes} Vinter-Bye {1/4 ‒ _________}\hspace{1em}\par
d. 6te Foranstaltet ved \textit{Finne}-Lensmanden Retten til i næste Uge, ved Markets Tider.\par
d. 7de Hellig.\par
d. 8de Begivet mig til \textit{Varanger}Market i \textit{Østbotten}, hvor Dagen næstefter 3de \textit{Qvæner}, eller \textit{Torne}-Borgere ankomme med en 20 Kieredzer, medhavendes Vadmel, groft Lærret, groft Klæde, Kartun, Smør og Brændeviin, lidet af hvert Slags; Hvorimod de indkiøbte Sej-Fisk, og frossen Torsk. Market stoed i 2de Dage.\par
d. 11te Foer jeg fra Market til \textit{Peders Finne}bye\par
d. 12. til\par
d. 15de holdet der \textit{Examen}. [Se vol. VI pag. 433 f.]\hypertarget{Schn1_107696}{}Forretted i Vardøe Præstegield.
\DivII[Febr. 17.-19. Fra Varangers Østbotten til Vadsø]{Febr. 17.-19. Fra Varangers Østbotten til Vadsø}\label{Schn1_107698}\par
d. 17de Faret fra \textit{Finne}-Byen til \textit{Bergebye}{1 1/2 Søe Miil}\par
d. 18de Fra \textit{Bergebye} til \textit{Carjel} i Øster {1 1/2 ‒}\par
d. 19de derfra til \textit{Vadsøe}{1 ‒ ___________}
\DivII[Febr. 20.-mars 15. Korrespondanse og protokollarbeide på Vadsø]{Febr. 20.-mars 15. Korrespondanse og protokollarbeide på Vadsø}\label{Schn1_107742}\par
d. 20de J Anledning af et \textit{Manuscript} efter afdøde Foged \textit{Soelgaard}, skrevet til Amtmand \textit{Kieldson} angaaendes et Sølv- og Svovel-Berg, i \textit{Vest-Finmarken} skal være; Hvoraf han ville see at forskaffe mig Prøver.\par
d. 21de Hellig.\par
d. 22de Erholdte mine Breve med Posten Sønden- og Vestenfra. ‒ Derfra til\par
\centerline{\textbf{Martij}}\par
2den\textit{expederet} mine Breve i sær til Hr Obriste \textit{Mangelsen}, de Kongel. \textit{Norske Commissions}Betiente, Amtmand \textit{Kieldsen}, Fogder og andre.\par
d. 3die til 6te\textit{Confereret} med Fogden \textit{Wedege} om hvis han efter mine \textit{Jnstructions} Poster paa sin \textit{Malmis}Reise ang. den fra den \textit{Norske} Side paafodrende \textit{Nordfield}-Skatt har \textit{observeret}.\par
d. 8de\textit{Extraheret} af \textit{Finmarkens Examinations Protocoll Tabeller} over Grændse-Merkene, med Beskrivelse af Grændsens Gang, og Forteignelse af Vidnerne og deres Udsagn; Hvoraf Amtmanden, Fogden og hver Provst i \textit{Finmarken}, hver skulle have et \textit{Exemplar}, deraf at \textit{instruere Missions} Betiente og Besigtelses Mænd til den \textit{provisionelle} Befaring.\par
d. 15de Tilstillet heraf Provst \textit{Angell} i \textit{Ost-Finmarken}, og Fogden \textit{Wedege} hver et \textit{Exemplar} med en \textit{Geographisk} Afteigning over Grændse-Gangen.
\DivII[Mars 16.-23. Fra Vadsø til Karasjok]{Mars 16.-23. Fra Vadsø til Karasjok}\label{Schn1_107884}\par
d. 16de Givet mig til Field-Reisen, og kommet den Dag igiennem \textit{Varanger}fiord i Syd- Syd-vest til \textit{Ny-elv} paa \textit{Rafte}- eller Søndre Landz Side{2 1/2 Søe Mile}\par
d. 17de Ventet her efter \textit{Jndiagers Finner}, som ei Komme.\par
d. 18de Givet mig i Kieredzer til Fieldz, og Kommet i Syd-vest til \textit{Bolma-Aae}, nær Øvre \textit{Bolma}-Vatten, paa Grændsen af \textit{Jndiager} og \textit{Arisbye;} Hvor \textit{Jndiagers Finner} ei heller indfandt sig, som Kan være 2 1/2 Søe Mile, eller 3 1/2 Field Mile. Hvilke \textit{Jndiagerer} havde ladet mig vide: Om de af deres den Svenske Øvrighed fik Forlov, ville de komme ned til mig, paa min Field-Reise: Men, som meldt, skeede ej.\par
d. 19de Faret i VestSydvest til \textit{Arisbye} den \textit{Svenske Lappe}Kirke ved \textit{Otzjok}-Aae{3 1/2 Field Mile} Den Svenske Præst fortalte mig her, at fra LandzHøvdingen var kommet hannem \textit{ordre} til, til \textit{publication}, at \textit{Finnerne} selv, og med deres Reen skulle være til reede at gaae de Svenske Grændse-Maalere i tilstundende Sommer, da de her kunde ventes, til Haande, for Betaling.\par
d. 20de Fra \textit{Arisbye} over Field i VestSyd-Vest til \textit{Juxbye}{3 Field Mile} Saa efter \textit{Tanen}Elv mest i Søer til en \textit{Arisbye-Finns} Vinterbye paa Østre Side af \textit{Tanen}{1 1/2 ‒ ___________ = 4 1/2 F. Mil}\hypertarget{Schn1_108014}{}Schnitlers Protokoller V.\par
d. 21de Hellig.\par
22de Efter \textit{Tanen} mest i Søer til en \textit{Norsk Porsanger-Finns} Tield paa Vestre Side \textit{Tanen}, ved \textit{Karasjoks} Kæften, hvilken Tield, aaben oventil, og paa Sidene hulled, naar Jlden mod Natten slukkedes, ei var synderlig varm, at ligge i.\par
d. 23de Herfra opefter \textit{Karasjok}-Elven til det Svenske Markested \textit{Karasjok} i Syd-Vest til Syden {2 Field Mile}\par
d. 24de Foranstaltet Retten her i \textit{Karasjok} over Norske \textit{Finner}.\par
d. 25de Hellig. ‒
\DivII[Mars 26.-30. Rettsmøte i Karasjok]{Mars 26.-30. Rettsmøte i Karasjok}\label{Schn1_108079}\par
\centerline{\textbf{Anno 1745. den 26de Martii}}\par
blev \textit{Examinations} Retten satt i en fælles \textit{Qvæns} Stue-Huus ved \textit{Karasjok}-Elv paa dennes Syd-ostlige Bræde, der hvor de Svenske holde deres Markested. Efter den \textit{Norske} Fogedz Foranstaltning mødte her \textit{privative-Norske Porsangers} Field-\textit{Finner}, som Vidner; Ved Retten var tilstede den \textit{Norske Missions} Skolemester \textit{Nicolaus Mortensen}, og Som LaugRettes Mænd gamle \textit{Peder Juxsen} og \textit{Niels Pedersen}, begge \textit{Norske Porsangers} Field-\textit{Finner;} J deres Overværelse blev Eeden for de Vidner, som ei før været afhørte, forklared, og Corporligen aflagd; De andre Vidner, som i \textit{Alten} tilforn have giort Eed, bleve kun formanede, at give deres Forklaring om, hvis de tilspørges, i Kraft af deres forrige aflagde Eed.\par
Derpaa blev først fremkalded i Ordenen\par
\centerline{\textbf{det 48de Vidne i Finmarken}\textit{Johannes Pedersen}}\par
fødd paa SøeKanten af den \textit{Norske}\textit{Porsangers} Fiord af \textit{Norske} Søe-\textit{Finn}-Forældre, døbt i den \textit{Norske}\textit{Kielvigens} Kirke, en 50 Aar gammel, gift, har 2 Børn, sidst afvigte Høst været i \textit{Kistrands Norske Capell} paa \textit{Porsangers} Fiord-Bræde til Gudz Bord, nærer sig, som FieldFinn, siddendes om Vinteren med sine Reen ved \textit{Karasjok}-Elv, om Sommeren ved Søe-Kanten paa \textit{Refsnæss} ved \textit{Refsbotten}\hspace{1em}\par
Sp. 1. Svar: Dette Sted, hvor Retten nu holdes, kaldes af Elven, paa hvis Syd-ostlige Bræde det ligger, \textit{Karasjok}, beboed af 5 saakaldede \textit{Qvæner}, eller \textit{Torne}-Bønder, som svare Skatt baade til \textit{Norge} og \textit{Sverrig}, og søge undertiden den \textit{Svenske}\textit{Koutokeino}-Hoved-Kirke, hvorhen de have 17 ‒ undertiden den \textit{Svenske}\textit{Arisbye Annex}Kirke, hvortil de have 9 FieldMile; (hvilket for de Svenske Præster kan være lige meget; Thi Hoved-Præsten i \textit{Koutokeino} oppebærer eene all præstelig Rettighed, og \textit{Capellanen} til \textit{Arisbye} lever af sin \textit{Missionarii} Løn fra \textit{Sverrigs} Crone).\hspace{1em}\par
Sp. 2. Svar: Dette \textit{Karasjok} ligger paa det \textit{Jndre} faste Land, og kan reignes til \textit{Vest- Finmarken}, fra den \textit{Norske}\textit{Porsangers}fiordz Botten, naar der gaaes Vesten forbi \textit{Gaisak}, og Østen forbi \textit{Vorieduder}-Fielde, i Søer 2de Dagers Reise, som meenes at være ved 7 à 8 Field-Mile.\hspace{1em}\par
Sp. 3. til 5te \textit{cesserer}.\hypertarget{Schn1_108298}{}48de Vidne i Finmarken. Karasjok.\par
Sp. 6. Svar: Elven \textit{Karasjok}, hvorved dette Sted ligger, veed dette Vidne ingen Beskeeden om. (Til Rettens Oplysning tiener, at den for \textit{pag.} 237 f. her er beskreven) Dog er ham bekiendt, at herj fanges Lax.\par
Elv-Brædene af denne \textit{Karasjok} have temmelig gode Sletter, med Bierk bevoxne; og naar Landet stiger op i Vejret til Fielde, have de skikkelig god Furre, dog ikke af største Slag; Fra \textit{Karasjok}Kæften i Sydvest til \textit{Jietzjok}Kæften, en 4 Field-Miile vejs er Furre-skougen ligest, siden bliver den tynnere; hvilken Furre er paa Sidene af Fieldene, men ovenpaa ere disse skallede, med Reen-Moese paa: Dog voxer her ikke Korn.\hspace{1em}\par
Sp. 8. Svar: Folk, som boe stadig ved denne \textit{Karasjok}, ere de forberørte 5 \textit{Qvæner}, som nære sig af Laxe-Fiskerie i Elven, og af deres Qvæg, men Kornet maa de kiøbe. Af \textit{Finner} komme vel Norden-fra de \textit{Norske PorsangersFinner}, og fra Syd og vest de fælles \textit{Avjevara- Finner:} men de \textit{Norske} og de fleste af \textit{Avjevara-Finner} sidde om Sommeren ved den \textit{Norske} Søe-Kant, og have hine deres Fiskerie i Havet disse Reen-Moese der. Dog nogle faa af de \textit{Avjevarer} ere om Sommeren her ved \textit{Karasjok}, og fiske Lax.\hspace{1em}\par
Sp. 9. Om Skougen er talt ved 6te Spørsmaal.\hspace{1em}\par
Sp. 10. Svar: Vildt er i Fieldene Ræver, Ulve, Vilde-Reen op imod Landz\textit{kiølen}, undertiden Biørn, Harer, af Fugle Tyder, Riper ‒ \textit{item} Eeghorn, Hermelin, Ottere.\hspace{1em}\par
Sp. 11. Svar: Ved \textit{Karasjok}- og \textit{Jetzjok}-Elve ere gode Sletter med smaa Skoug begroede; J denne \textit{Jetzjok} fanges og Lax, og naar Elv-Brædene stige op til Berge, findes der og nogen Furre fra \textit{Jetzjoks} Kiæft til hen imod \textit{Avjevara;} Hvorfore der siunes Leilighed til Rødning, for Folk, hvilke, som berørte her boende Qvæner, kan nære sig af Lax-Fiskerie og deres Creature.\hspace{1em}\par
Sp. 12. Svar: Han veed ikke at navngive Fieldene ved Elvene, undtagen ved \textit{Tanen} paa dens Vestre Side Sønden for \textit{Vall-jok} er bekiendt det Field\par
\textit{Vaudevara}, 1 Field-Miil i Søer efter \textit{Tanen} langt, og en 3 à 4 Bøsseskud over bredt, havendes paa begge Ender 2de bare skallede Topper, og derimellem lidet smaa Bierk. 1/4 Miil Sønden for \textit{Vaudevara} er\par
\textit{Niargasaiks}, ikke saa høy, som \textit{Vaudevara}, slet og bart ovenpaa, paa Østre og de øvrige Sider med Furre bevoxen, 1/2 Miil langt i Søer efter \textit{Tanen}, og lidet bredere, end \textit{Vaudevara} i Vester; dette \textit{Niargasaiks} rekker hen til \textit{Karasjok}-Kæften, og giør deraf det Nordre Næss. Jmellem disse \textit{Vaudevara} og \textit{Niargasaiks}-Fielde kommer Vesten-fra op af Jorden, som af en Kilde, \textit{Saukadasjok}, en Aae, som 1 god Miil lang rinder i Øster ud i \textit{Tanen}-Elv. Fra \textit{Niargasaiks}, op efter \textit{Karasjok}-Elv, paa dennes Nordre Side, 1/2 Miil ligger det Field\par
\textit{Gaarmik}, paa Sidene steilt, med Bierk paa, Oventil med 3 Topper paa fra Nord i Søer, og bart, langt fra Søer i Nord 1/4 Miil, nogle Bøsse-Skud over bredt i Vester. 1/4 Miil i Vester fra \textit{Gaarmik}, paa \textit{Karasjoks} nordre Side følger det Field\par
\textit{Kiarkie-vara}, rundt og bart oventil, paa Søndre Side bratt, paa de andre Sider fladtvoren, med Bierk og noget Furre paa, 1/8 Miil stort efter Elven. Paa dette \textit{Karkie-vara} følger Bierke-Skoug og Dale, ved 2 Field-Mile vide hen i Vester til Sønden, til \textit{Vorieduder}, før beskreven i \textit{Alten}\textit{pag.} 225.\hypertarget{Schn1_108527}{}Schnitlers Protokoller V.\par
Paa Syd-ostlige Side af \textit{Karasjok}, ved dennes Mund imod \textit{Niargasaiks} findes\par
\textit{Dillskaite}, ikke ret høyt, 1 Field-Miil langt efter \textit{Karasjok}-Elven i Syd-vest, skal være 1/4 Miil over bredt, bart og slet ovenpaa, med en liden Top paa Søndre Ende, paa Sidene fladtvoren, mest med Bierk paa, neden under ad Elvene er der Furre og Bierk. Efter en kort Dal, et par Bøsse-skud viid, med Bierk i, følger i Syd-vest\par
\textit{Dagter-ove}, saa høyt, som \textit{Dillskaite}, 1 Field-Miil langt i Syd-vest hen til \textit{Karasjok}-Markested, 1/8 Miil over bredt, fladt og bart ovenpaa, paa Østre og Nordre Side ad Elven fladtvoren, paa de andre Sider høybakket, med Bierk og Furre paa. 1/8 Miil i Syd-vest fra \textit{Dagterove} forekommer\par
\textit{Bakiel-vare}, lavere, end næst forrige, rundt, 1/8 Miil stort i Syd-vest, slet og bart ovenpaa, fladt paa Sidene; Paa Østre Side er Bierk, paa Vestre Bierk og Furre, paa Nordre Side ligesaa. Jmellem \textit{Dagterove} og \textit{Bakielvare} gaaer en Aae \textit{Bakiel-jok} Sønden-fra af Jorden opkommendes, i Nord i \textit{Karasjok}, ved 2 Field-Mile lang. Videre ei bekiendt.\hspace{1em}\par
Sp. 15. Svar: Om \textit{Mineralia} veed intet at sige; At Perler i \textit{Karasjok} er faaet, har han for lang Tid hørt, men nu nylig ej fornommet dertil.\hspace{1em}\par
Sp. 16. Svar: De nærmeste Steder og Pladser fra dette \textit{Karasjok}-Field-Bye er i Vester Til Sønden \textit{Avjevara}, ved \textit{Jetzjok}, et \textit{Svensk} Ting- og Markested {5 Field Mile} Saa fra \textit{Avjevara} i Syd-vest til \textit{Kautokeino}, et Svensk Ting- og Markested, hvor og den \textit{Svenske} HovedKirke er, {12 ‒} De Steder Østen for \textit{Karasjok} er det nu øde \textit{Juxbye}, i Nordost derfra ved \textit{Tanen}{6 ‒} Fra \textit{Juxbye} i Ost-Nord-ost over \textit{Gieskadam}-Field til \textit{Arisbye}, et \textit{Svensk} Ting- og Markested ved \textit{Otzjok}-Aaen, hvor og den \textit{Svenske Annex}Kirke staaer, er. {3 ‒}\par
Fra \textit{Karasjok} i Nord til den \textit{Norske}\textit{Porsanger}-Fiordz Botten er 2 Dagers Reise, eller 7 à 8 Field Mile\hspace{1em}\par
Sp. 17. Svar: Vej fra \textit{Porsanger}Fiord til dette \textit{Karasjok} tages Vesten forbj \textit{Gaisak} og Østen forbj \textit{Vorieduder} i Søer: Vejen fra \textit{Karasjok} videre i Søer har han kun engang om Vinteren faret op efter \textit{Karasjok}-Elven i Søer til imod \textit{Kiølen}.\hspace{1em}\par
Sp. 18. Svar: Hvormange \textit{Norske} Field\textit{finner} af \textit{Kistrand-Capell} i \textit{Porsanger}fiord ere? findes før \textit{pag.} 269 antegnet.\hspace{1em}\par
Sp. 19. Svar: Hvormange fælles Field-\textit{Finner} til \textit{Koutokeino}-Kirke-Sogn høre? er for \textit{pag.} 230 angivet.\par
Om hvilke Mandtalle Vidnet ej viss Beskeed vidste.\hspace{1em}\par
Sp. 20. Svar: Field-Landet imellem \textit{Finnerne} er ei skiftet hverken paa \textit{privative-Norsk} Grund, ei heller paa fælles \textit{Søerfielde} i \textit{Koutokeino}-Gield.\par
Sp. 21. Svar: Hvorvidt de \textit{Svenske} fælles Field-\textit{Finner} deres \textit{district} gaaer i Nord til \textit{Norges privative}-eeget Land? derom giver han følgende Beskeeden: Hvorlangt de fra \textit{Koutokeino} gaae i Nord imod \textit{Masi Capell} i \textit{Altens}Fielde? Det veed han ei: Men de fra \textit{Avjevara}\hypertarget{Schn1_108832}{}48de Vidne i Finmarken. Karasjok. pleje at søge, og holde til ved \textit{Sios-jaure} (som han hørt har) paa Vestre og Søndre Side, og fiske i samme Vand; Videre holde de til ved \textit{Jetzjok}, som løber igiennem \textit{Siosjaure} i Øster paa dens Søndre Side: men over denne \textit{Jetzjok} ere de ei komne i Nord at ligge med deres Reen om Vinteren: Dog er dette ei at forstaae om de \textit{Avjevarers} Fløtningsfærd om Vaaren til ‒ og om Høsten fra Søe-Siden; thi da gaae de baade Norden om \textit{Sios-jaure} og \textit{Jetzjok}.\par
J Ost-Nord-ost fra \textit{Jetzjoks} Mund paa Søndre Side af \textit{Karasjok}-Elven hen til dennes Udløb i \textit{Tana}-Elv sidder ingen \textit{Finn:} Mens for en 20 Aar siden have der nedsatt sig nogle \textit{Svenske Torne}-Bønder, som kaldes \textit{Qvæner}, hvoraf nu 5 boe paa Søndre Elv-Bræde af \textit{Karasjok}, og have siden deres Ankomst bygget deres Sommer-Sæter tæt Norden for \textit{Jetzjoks} Mund. Bemeldte \textit{Avjevara-Finner} bruge til deels, at fiske Lax i \textit{Jetzjok}, ligesom de berørte \textit{Qvæner} i \textit{Karasjok} om Sommeren.\par
Fra \textit{Karasjoks} Kæften langs med den Søndre Side af \textit{Tana}-Elv i Ost-Nord-ost forbi forrige \textit{Juxbye} og forbi \textit{Otzjok} hen til \textit{Fossholmen} har han hørt, at \textit{Arisbye}-fælles\textit{Finner} have besiddet det Land Østen for \textit{Tanen}-Elv, men ei kommet derover, uden i deres FløtningsFærd til og fra Søe-Kanten; Samme \textit{Arisbye-Finner} have og brugt Laxe-Fiskerie i denne Streknings \textit{Tana}-Elv.\par
(Retten til Underretning meddeeles, at de Vidner i \textit{Alten}\textit{pag.} 230 f. have udsagt de \textit{Koutokeino-Finners district} imod \textit{Altens} Bøyd)\par
Sp. 22. Hvorvidt gaaer de \textit{privative Norske Finners District} i Søer fra SøeKanten op imod de \textit{Svenske} fælles \textit{Finner?}\par
Svar: Hvorvidt de \textit{Altens Norske Finner} gaae i Søer imod \textit{Koutokeino-Finner} efter \textit{Altens} Elv? veed Vidnet ikke: men nogle af bem.te \textit{Altens Finner} sidde om Vinteren iblant \textit{Avjevara-Finner} baade Vesten og Sønden for \textit{Sios-jaure}.\par
De \textit{Norske Repperfiord-Finner} af \textit{Hammerfest} Gield pleje at sidde, nogle Vesten- og Sønden for berørte \textit{Sios-jaure} iblant \textit{Avjevara} og \textit{Altens Finner}, (hvorlangt i Søer? vides ej) nogle Norden- og Sønden for \textit{Jetzjok}, der hvor \textit{Avjevara}-Markested er, og det (nemlig Sønden for \textit{Jetzjok}) iblant \textit{Avjevara-Finner}.\par
De \textit{Norske Refsbottens} Field-\textit{Finner} af \textit{Jngøens} Gield sidde om Vinteren ved \textit{Karasjok} baade paa den Søndre og den Nordre Elv-Bræde.\par
De \textit{Norske Porsangers} Field-\textit{Finner} af \textit{Kielvigens} Præstegield have holdet sammestedz til, nemlig paa Nordre Side af \textit{Karasjok}, og paa dens Søndre Elf-Bræde; Paa hvilket sidste Sted de for en 10 Aar have begyndt at sidde.\par
De \textit{Norske Laxefiord} Field-\textit{Finner} af \textit{Kiøllefiords} Gield have for nogle faa Aar siden, da de vare fleere i Tallet, brugt Landet Norden for og indtil \textit{Tanen}-Elv, og naar Reen-Moesen har skiortet dennem her, have de og gaaet Søer over \textit{Tanen}-Elv ind under \textit{Gieskadam}-Field, og ved \textit{Nulli-jok:} men siden disse \textit{Laxefiord-Finner} ere blevne faa, saa have de slaaet sig i Selskab med \textit{Tanens Norske Finner}. (\textit{Conferat.}\textit{pag.} 221.)\par
Disse \textit{Norske} Field-\textit{Finner} have ei brugt, at fiske i nogen af forberørte Elve om Sommeren Lax; Thi da have de siddet ved Søen, og haft deres Fiskerie der: men hvem som af dem har villet, har om Vinteren gaaet i Søer til \textit{Jaurisduøder} eller Field-\textit{Kiølen} paa vildReen-Skøtterie; Og dette Skøtterie have de fra Arildz tid brugt, og bruge, hvem som vill.\hypertarget{Schn1_109166}{}Schnitlers Protokoller V.\par
Sp. 23. Naar de Svenske fælles Field-\textit{Finner} nedkomme til SøeKanten, og naar de derfra tilbage fløtte? hvad de nyde og bruge ved SøeKanten? om de i den Tid de her sidde, søge \textit{Norsk} Ting, Kirker, eller \textit{Missions}-ForsamlingsSteder? og hvor de sidde?\par
Svar: Om de \textit{Koutokeino-Finner} veed han intet; De \textit{Avjevara-Finner} begynde at fløtte om \textit{Korsmiss} nord-efter, og komme ned til Søe-Kanten om \textit{St. Hans} Tid; Om \textit{St. Ols} Tid fare de fra SøeKanten op igien, og fløtte langsom i Søer tilbage, at de om \textit{Helgemiss} Tider komme Sønden for \textit{Sioss-jaure} og \textit{Jetzjok} til deres Hiemstad.\par
Disse fælles Field-\textit{Finner}, medens de opholde sig ved Søe-Kanten, fiske ikke i Søen, men benytte alleene Fieldene til Reen-Beete; J den Tid de ere ved den \textit{Norske} Søe-Kant, søge de ikke \textit{Norske} Ting, ei heller til \textit{Norske} Kirker, eller til \textit{Norske Missions} Forsamlings Steder; hvilket nærværende \textit{Norske Missions} Skolemester \textit{Nicolaus Mortensen} bekræftede, og sagde, ikke at have med dem at bestille.\par
Af \textit{Avjevara-Finner} sidde om Sommeren nogle imellem \textit{Korsfiord} og \textit{Komagfiord}, norden for \textit{Altens} Hoved-Fiord, \textit{item} ved \textit{Lærretz}Fiord under \textit{Talvigens} Ting; Nogle ved \textit{Næverfiord} og Vesten for \textit{Repper-fiord} i \textit{Hammerfest}-Tinglaug; Endeel paa Vestre Side af \textit{Refsbotten} i \textit{Maasøe} Tinglaug; Een \textit{Arisbye}-FieldFinn sidder om Sommeren ved \textit{Veinæss} i \textit{Laxefiord} i \textit{KiølleFiords} Tinglaug.\par
Her fremstoed den \textit{Norske Porsangers} FieldFinn \textit{Peder Nielsen}, og klagede paa eegen og de \textit{Norske} Søe-\textit{Finner} ved \textit{Veinæss} deres vegne, at denne \textit{Arisbye} FællesFinn ved \textit{Veinæss} om Sommeren kommer dem for nær, dem til Skade.\par
Den \textit{Arisbye} Finn Anders Aslaksen var just her tilstede, og nægtede, at have giort nogen af \textit{Veinæss Norske Finner} Skade; Han har siddet der en 20. Aars Tid, siden han er bleven gift med en \textit{Norsk Finne}-Kone, og have \textit{Laxefiords Finner} givet ham Lov at sidde der ved \textit{Veinæss}, naar han ikke komm deres England med sine Reen for nær. ‒\par
Sp. 24. Svar: Om Landz\textit{kiølen} imellem \textit{Norge} og \textit{Sverrig} veed han intet.\par
Sp. 25. Svar: Om Landskabet Sønden for \textit{Kiølen} veed intet.\par
Sp. 26. Svar: Ei heller veed om Landskabet nær Norden for Landz\textit{kiølen}.\par
Sp. 27. Hvad Vej til \textit{Kiølen}, og hvor langt did? Svar: Det veed ej.\par
Sp. 28. Hvad Vej, og hvor langt til fælles Land? Svar: Fra \textit{Repperfiord} i Søer til \textit{Jetzjok}, der hvor \textit{Avjevara} ligger, er en 4 Dagers Reise, eller en 9 \textit{Finmarkske} Søe-Mile, som holdes for en 12 Field-Mile. ‒ Vejen fra \textit{Repperfiord} gaaes langs med \textit{Basten-jok, Rauds-jok}, saa Vesten for \textit{Vorieduder}, alt i Søer til \textit{Jetzjok} ved \textit{Avjevara}.\par
De fra \textit{Refsbotten} fare samme Vej, og have omtrent ligesaa langt ‒\par
De fra \textit{Porsangers} Botten gaae i Søer Vesten forbi \textit{Gaisak}, siden Østen forbi \textit{Vorieduder} til \textit{Karasjok}, i 2de Dage, som Kan giøre omtrent 8. Field-Mile. Fra \textit{Porsangers}Botten gaaes over \textit{Gaisak}-Field i Søer til \textit{Tanen}-Elv, hvor Juxbye har været, i 1 1/2 Dag, som meenes at giøre 4. à 5 Mile. (\textit{conferat.}\textit{pag.} 230. 240. 313.)\par
Sp. 29. Om været nogen Tvistighed imellem \textit{Porsangers} de \textit{Norske} ‒ og \textit{Avjevaras} fælles \textit{Finner?}\par
Svar: Nej, hidindtil ikke; Thi de forligtes vel.\par
Sp. 30. Hvad Nytte ved de Bøyde-Raamerker imellem \textit{Porsangers}- og \textit{Avjevara-Finner?}\hypertarget{Schn1_109513}{}48de og 49de Vidner i Finmarken. Karasjok.\par
Svar: De \textit{Norske Finner} fløtte til Raamærkene ikke for andet, end for Moesen skyld til deres Reen, og ere de Kongens Alminding.\par
Sp. 31. Hvorledes det bliver fremkommendes for de Kongel. \textit{Norske} Hrr \textit{Officerer}, der skal lade opmaale Landz\textit{kiølen} med deres Folk,\textit{bagage} og \textit{provision?}\par
Svar: Om Sommeren vil det blive meget vanskeligt for de Hrr Grændse-Maalere at komme frem langs efter Lands\textit{kiølen;} Thi(1) ere alle Field\textit{Finnerne}, saavel de Svenske-fælles, som de \textit{privative-Norske}, fra Fieldene afdragne ned til Søe-Siden, en 20 Mile og meere fra Landz\textit{kiølen}. (2) Hester haves ei her i Landet (3) \textit{Finnerne} fløtte med deres Reen ned til Søe-Kanten paa Snee-Føret, og om Sommeren skulle de vanskelig komme frem fra Søe-Kanten til Landz\textit{kiølen} med deres Reen, formedelst de mange Vande og Elve, som Reenen ei kan svømme over, med mindre \textit{Finnen} har Baad, og efter den leeder, eller trækker Reenen over Elven, eller Vandet. (4) Reenen om Sommeren bærer i Kløv kun 1. Vaags Tynge (5) Skulle den vanskelig begaae sig paa Landz\textit{kiølen} for Mygg og Varme; Thi for den Aarsag Skyld fløtte alle \textit{Finnerne} derfra ned til Søe-Kanten:\par
Derimod om Vinteren ere Field-\textit{Finnerne}, saavel \textit{Norske}, som fælles \textit{Svenske} med deres Reen paa Fieldene, nærmere Landz\textit{kiølen} (2) De køre da med Reenen over alt den korteste Vej, og kan lettest komme did. (3) Een Reen fører da i en Kieredz 4 Vaager, og derover; Deri kunde de Herrer \textit{Officerer} selv køre, og faa deres Tield, \textit{provision} og \textit{bagage} magelig fremført. (4) Den beleiligste Tid var, at begynde strax efter Hellig Tre Konger, (da Dagen begynder at længes) og med Forretningen fortfare, saalænge Snee-Føret varede, omtrent til \textit{Aprilis} Udgang; J hvilken Tid da meere kunde \textit{avanceres} og vindes, end om Sommeren i 2 1/2 Maaneder, da der kan være bart.\par
Det \textbf{6te Vidne}\textit{Hendrik Povelsen} blev herpaa tilspurdt: Om han trøstede sig til om Vinteren at udviise de bevidnede Grændse-Merker paa Landz\textit{kiølen?} Han svarede dertil: Ja; Han meente, at baade han, og de andre Vidner, som have været paa \textit{Kiølen}, skulle kunde anviise dem om Vinteren.\par
Det \textbf{7de Vidne}\textit{Ole Mortensen} sagde det Samme.\par
Det \textbf{48 Vidne}\textit{Johannes Pedersen} siger og: Naar visse Vidner i forestaaende Sommer skulle vorde opsendt til \textit{Kiølen, provisionaliteter} at udsee, og udmerke Grændse- Skiellet paa \textit{Kiølen}, saa kunde af dennem saadanne Merker sættes, eller tages, at de om Vinteren derefter des lettere kunde kiendes. Hvorpaa \textit{dimittered}.\par
\centerline{\textbf{49de Vidne i Finmarken}}\par
\textit{Peder Jonsen lille, Norsk Porsangers} Søe-Finn fødd i \textit{Arisbye} af \textit{Arisbyes} Field-\textit{Finne}- Forældre, døbt i \textit{Arisbye}-Kirke, 46 Aar gammel, gift, har 6. Børn, sidste Høst været i \textit{Kistrands Norske Capell} til Gudz Bord; For 3 Aar siden fløttet med Kone og Børn, for Fattigdoms Skyld, fra \textit{Arisbye} ned til den \textit{Norske}\textit{Porsanger}-Fiord, hvor han nærer sig som Søe-Finn, og er Post-Finn.\par
Tilspurdt, ligesom næstforrige 48de Vidne, til\par
Sp. 1. Svarer det samme \hypertarget{Schn1_109698}{}Schnitlers Protokoller V.\par
Sp 2. 6. 8. 9. 10. 11. 12. 15. 16. Svarer ligesom næstforrige.\par
Sp. 17. Svarer, som næstforrige: Men til \textit{Kiølen} ei faret.\par
Sp. 18. og 19. Svarer: Han veed ikke.\par
Sp. 20. Svarer det samme, som næstforrige 48de Vidne.\par
Sp. 21. Svar: Fra \textit{Karasjoks} Kæften langs med den Søndre Side af \textit{Tanen}-Elv i Ost- Nord-ost forbj forrige \textit{Juxbye} og forbi \textit{Otzjok} hen til \textit{Fossholmen} veed han, at \textit{Arisbye} Fælles- \textit{Finner} have besiddet det Land Østen for \textit{Tana}-Elv, men ei kommet derover, uden i deres FløtningsFærd til og fra Søe-Kanten; Samme \textit{Arisbye-Finner} have og brugt Laxe- Fiskerie i denne Streknings Tana-Elv.\par
Sp. 22. Svarer: Deraf veed han, at \textit{Refsbottens Norske} Field-\textit{Finner} om Vinteren sidde ved \textit{Karasjok} baade paa Nordre Side, og paa Søndre Elv-Bræde. \textit{Porsangers Norske Finner} ligeledes sammestedz. Om \textit{Laxefiords Norske} Field-\textit{Finner} siger han det samme, som næst- forrige 48de Vidne, med det Tillæg, at han for en 7 Aars Tid har seet samme \textit{Laxefiords Finner} ligge nær ved \textit{Tanen} Elvs Nordre Side imod \textit{Otzjok} Kæften.\par
De \textit{Norske} Field\textit{Finner} have ei brugt, at fiske i Elven: men hvem af dem har villet, har faret til Field-Kiølen eller derimod paa VildReen-Skøtterie.\par
til Sp. 23. 24. 25. 26. og 27. Veed intet at svare.\par
Sp. 28. Svar: Hvorlangt, og Vejen fra \textit{Porsangers} Botten i Søer til \textit{Karasjok}, og til \textit{Tanen}- Elv imod \textit{Juxbye}, forklarer han, ligesom det 48. Vidne; det øvrige veed han ei.\par
Sp. 29. Svarer: Han veed ei af nogen Tvist imellem Grændse-Boerne.\par
Sp. 30. Sv: Ligesom næstforrige 48. Vidne.\par
Sp. 31. Svarer det samme, som næstforrige.\par
\centerline{\textbf{50de Vidne i Finmarken}\textit{Unge Peder Nielsen, Norsk FieldFinn i Porsangers Fielde}}\par
fødd i \textit{Porsangers} Fielde Østen for Fiorden af \textit{Norske} Field-Finn-Forældre, døbt i \textit{Kielvigs Norske} Kirke, 43. Aar gammel, gift, har 6 Børn, sidste Høst været i \textit{Kielvigs} Kirke til Gudz Bord, lever af sine Reen, som Field-Finn, dog fisker lidet i \textit{Porsangers} Fiord. til\par
Sp: 1. 2. 6. 8. 9. og 10. Svarer det samme, som næstforrige.\par
Sp. 11. Svar: Ved \textit{Karasjok}Elv ere Sletter, som næste Vidner forklaret have: Men om \textit{Jetzjok} veed intet.\par
Sp. 12. 15. 16. 17. 18. 19. og 20 Svarer, som næste 49de Vidne.\par
Sp. 21. Svarer som 49de Vidne.\par
Sp. 22. Svarer, som 49de Vidne.\par
Sp. 23. 24. 25. 26. og 27. \textit{item} 28. Svarer, som næste 49de Vidne. mdash;\par
Sp. 29. Svarer som samme 49de Vidne: dog har han hørt, at de Svenske fælles \textit{Finner} for nogle Aarer have \textit{prætenderet}, at de \textit{Norske} Field\textit{Finner}, som søge til Landz\textit{kiølen} paa VildReen Skøtterie, skulle giøre fri Skydz til den Svenske Field-Øvrighed: men derom er det nu stille.\par
Sp. 30. og 31. Svarer det samme, som næstforrige.\hypertarget{Schn1_109935}{}51de og 52de Vidner i Finmarken. Karasjok.\par
\textbf{51de Vidne i Finmarken}\par
\textit{Ole Povelsen, Norsk Porsangers} FieldFinn fødd i \textit{Beldo-vuøbme}, eller \textit{Belbo}-Skoug, en 3 Mile Sønden for Landz\textit{kiølen} af \textit{Svenske Enotekies FieldFinne}-Forældre, døbt i \textit{Enotekies}\textit{Lappe}-Kirke, 50 Aar gammel, gift med en Norsk FieldFinn-Qvinde, har 1 Barn, for 40 Aar siden med sine Forældre for Fattigdom skyld, fløttet fra \textit{Enotekies} over til \textit{Norge}, og nærer sig som FieldFinn, dog fisker lidet om Sommeren i \textit{Porsangers}Fiord.\par
til Spørsm: 1. 2. 3. 8. 10. 11. 12. 15. 16. 17. 18. 19. og 20. Svarer ligesom 49de Vidne.\par
Sp. 21. Svarer han: De \textit{Avjevara-Finner} sidde omkring \textit{Sios-jaure}, hvor Skoug er, end og paa dets Østre Side, dog ikke langt derfra; Videre sidde disse \textit{Avjevarer} paa den Søndre Side af \textit{Jetzjok}, og komme ei herover at ligge, uden i deres FløtningsFærd, som 48de Vidne \textit{pag.} 369 forklaret har. Hvad videre bem.te 48de Vidne om \textit{Karasjoks} Beboelse, og en Deel \textit{Avjevarers} Fiskerie i \textit{Jetzjok}, ‒ \textit{item} at \textit{Arisbye-Finner} sidde til den Østre Side af \textit{Tanen}- Elv fra \textit{Karasjok}-Kæften af til \textit{Fossholmen}, har vudnet, det samme stadfæster han.\par
Sp. 22. Svarer han ligesom 49de Vidne \textit{pag.} 372. Dog siger derhos, at \textit{Repperfjords Norske} Field\textit{Finner} sidde med de \textit{Norske Altens} Field-\textit{Finner} omkring \textit{Sios-jaure} iblant \textit{Avjevaras} fælles \textit{Finner}, af samme \textit{Repperfiords} Field\textit{Finner} sidde og nogle Norden- og Sønden for \textit{Jetzjok}, ligesom 48de Vidne \textit{pag.} 369 det forklaret har.\par
Om de Norske \textit{Porsangers-} og \textit{Laxefiords} Field\textit{Finner} siger han det samme, som 48de Vidne \textit{p:} 369 vidnet, saa og at hvem af de \textit{Norske} har villet, de have gaaet til \textit{Jaurisduøder} paa VildReen-skøtterie.\par
Sp. 23. Svarer det samme som 48de Vidne, undtagen han veed ej hver Sted ved den \textit{Norske} SøeKant, hvor de \textit{Avjevarer} om Sommeren sidde, uden ved \textit{Refsbotten}, der 4. af de \textit{Avjevarer} sig opholde.\par
Sp. 24. 25. 26. og 27. svarer han, at han ikke det veed.\par
Sp. 28. Svar: Han veed alleene Vejen fra \textit{Porsangers} Botten hid til \textit{Karasjok}, som han siger, at være, efter 48de Vidne dets Forklaring \textit{pag.} 370.\par
Sp. 29. Svarer som 50de Vidne \textit{pag:} 372.\par
Sp. 30. og 31 Svarer, som forrige \textit{pag:} 370 f.\par
\centerline{\textbf{52de Vidne i Finmarken,}\textit{Hans Hansen, Norsk Porsangers} Field-\textit{Finn},}\par
fødd i \textit{Jndiager}, 45 Aar gl., gift, har 4 Børn, forleden Høst været til Gudz Bord i \textit{Kielvigens} Kirke, for 12 Aar siden giftet sig i den \textit{Norske Porsanger-Fiord}, og siden bleven her i Landet: Dog har han været tilforn længe en \textit{Arisbye-Finn;} Veed kun i Særdeleshed til\par
Sp. 21. at svare: At han har seet, og veed, at \textit{Arisbye-Finner} have ligget Østen for \textit{Tanen}- Elv, at reigne fra \textit{Karasjok}Kæften til \textit{Fossholmen}, og ei kommet derover, uden i deres FløtningsFærd, og have der i \textit{Tanen} fisket.\par
Sp. 22. Svar: De \textit{Norske Refsbottens} saa og \textit{Porsangers}-Field-\textit{Finner} sidde baade Sønden og Norden for \textit{Karasjok}, og haver han (Vidnet) i nogle Aar tilholdet 1 Miil Sønden for \textit{Karasjok} paa Vestre Side af \textit{Øvre-Tanen}.\hypertarget{Schn1_110265}{}Schnitlers Protokoller V.\par
Om \textit{Laxefiords Finner} sige det samme, som 48de Vidne \textit{pag.} 369 og at de \textit{Norske Finner}, hvem som har villet, har gaaet efter VildReen op til \textit{Kiølen}.\par
Sp. 23 Sv: Om \textit{Arisbye-Finner} siger han det samme, som 48de Vidne, og at de ligge om Sommeren ved \textit{Tanen}-Elv, og imellem \textit{Tanen}- og \textit{Laxefiordene}; Om de andre \textit{Svenske} fælles Vidner Vesten-for veed han intet.\par
Sp. 28. Svar: Om Vejen fra \textit{Porsanger} til \textit{Karasjok} og \textit{Tanen} siger det samme, som 48de Vidne \textit{p.} 370.\par
Sp. 29. Svarer, som 50 Vidne \textit{pag.} 372. ‒\par
Sp. 30. Svar: Svarer, som 48de Vidne \textit{pag.} 370 f.\par
Derefter bleve fremkaldede de 5te 6te og 7de Vidner, som \textit{Ao} 1744 \textit{Aprilis} 6.\textit{s.} i \textit{Alten} efter \textit{pag.} 236 f. for ere afhørte, at svare i Kraft af deres tilforn aflagde Eed, nærmere til Spørsmaal:\par
til Sp. 1 og 2. Svare de, som 48 Vidne her \textit{pag.} 366.\par
Sp. 6. Svare de 2de Vidner, ligesom 48de Vidne, men det 6te Vidne vidste at forklare \textit{Karasjok}, ligesom 8de Vidne \textit{p.} 237 dog sagde, at fra \textit{Karasjok}-Kæften til \textit{Juxbye} er kun 4 Mile. (hvilket man og paa denne FieldReise har befundet)\par
Fra 8de til 21 Sp: Svare det samme, som 48de Vidne: dog sige ved det 21de Sp: at de \textit{Avjevarer} sidde trint omkring \textit{Sios-jaure}, som 51 Vidne \textit{pag.} 373 forklaret.\par
Sp. 22. Svare, som 48de Vidne \textit{pag.} 369 tillæggendes, at nogle \textit{Norske Porsangers Finner} have i nogle Aar siddet, og nu sidde ved denne \textit{Karasjoks} Elv paa dens Vestre Side, 2 Mile Søndenfor \textit{Jetzjok}, og 4 Mile i Syd-Syd-vest fra dette \textit{Karasjoks} Markested.\par
Sp. 23. Sv: som 48de Vidne \textit{pag.} 370.\par
Fra 24. til 26de Sp. \textit{incl:} Svare: Hvad de om \textit{Kiølen} vide have de før \textit{pag.} 236 f. udsagt; 6te og 7de Vidner tillægge, at Landet Søndenfor \textit{Kiølen} paa 1 Miil er fieldet, siden skouget, først af Bierk, siden Furre, og omsider Gran, saa vidt, at den seer ud som et Hav, med Myrsletter, Elve og nogle Fieldtopper i.\par
Sp. 27. Svar: De have faret om Vinteren baade paa Vestre og Østre Sider af \textit{Karasjok} op til \textit{Kiølen}, og sige derhen at være fra \textit{Karasjok}-Kæften 12 Field-Mile.\par
Sp. 28. til Enden \textit{inclus.} Svare, som 48de Vidne.\par
Efter \textit{pag:} 268 her, vare som kyndige Vidner \textit{Sara Ammondsen} og \textit{Ammond Olsen} angivne; disse bleve nu paaraabte; Den første \textit{Sara Ammondsen}, enddog man hørte, han af Fogden var tilsagt, var imod Søe-Kanten herfra dragen; den anden fremstoed, neml.\par
\centerline{\textit{Ammond Olsen, Norsk Porsangers FieldFinn} som \textbf{53de Vidne i Finmarken}}\par
fødd i \textit{Arisbye} af FieldFinn-Forældre, døbt i \textit{Arisbye}-Kirke, 36 Aar gammel, gift med en \textit{Norsk Porsangers Finne}-Kone, uden Børn; For 5 Aar siden og noget derover fløttet fra \textit{Arisbye} hid neer til \textit{Porsanger}, da han giftede sig, og som han fik en Deel Reen med sin Hustrue, som ei vare vannte til \textit{Arisbyes} Fielde, saa er han med dennem bleven tilbage paa \textit{privative-Norsk} Grund; Været i afvigte Sommer i den \textit{Norske}\textit{Maasøe} Kirke sidst til Gudz Bord. Efter aflagde Eed\hypertarget{Schn1_110560}{}53de Vidne i Finmarken. Karasjok.\par
til Sp. 21. Svarer: at \textit{Arisbyes Finner} siddet paa den Østre Side af \textit{Tanen}-Elv, at forstaae fra \textit{Karasjok}-Kæften til \textit{Fossholmen}, og ei kommet derover uden i deres FløtningsFærd, med videre, som 48de Vidne \textit{pag.} 369 forklaret.\par
Efter Tilspørgende om Raamerket i \textit{Tanen}-Elv imellem de \textit{Norske Tanens-} og fælles \textit{Arisbyes Finner} svarer han: De \textit{Arisbye-Finner} sige, at \textit{Skaar}-Aae, og \textit{Tanens Finner} paastaae, at \textit{Fossholmen} giør Skiellet.\par
Sp. 22. Svar: Han veed om de \textit{Norske Refsbottens-} og \textit{Porsangers} Field-\textit{Finner}, at de sidde baade Norden og Sønden for \textit{Karasjok}-Elven, som 48de Vidne \textit{p.} 369 og de andre Vidner \textit{pag.} 374 ved dette Spørsmaal have udsagt. De \textit{Norske Laxefiords} Field-\textit{Finner}, da de vare fleere i Mandtallet, have indtil for omtrent 6 Aar siden haft deres Leje til \textit{Tanen}-Elv omtrent imod \textit{Nulli-jok, Gieskadam} og \textit{Otzjok}-Munden, hvor han har seet dennem; Og at de samme have gaaet over \textit{Tanen} med deres Reen til \textit{Nullijok}, og til under \textit{Gieskadam}, naar Moesen paa Nordre Side har skiortet, det har han hørt.\par
Sp. 23. Svar: Som 48de Vidne, saaledes de \textit{Arisbye-Finner} søge ikke til \textit{Norske} Kirker eller \textit{Missions}Forsamlinger, uden naar det høver sig, at de ere der nærmest ved. De øvrige Spørsrmaale, særdeles om \textit{Kiølen} veed intet at svare til.\par
Sp. 31. Svarer han angaaendes de Norske Hrr \textit{Commissions}-Betienteres Fløtning, som de andre \textit{pag.} 371.\par
Saa fremkom og det \textbf{8de Vidne,}\textit{Porsangers Norsk FieldFinn}, som i forleden Vaar 1744 d. 6 \textit{April}. \textit{s}. i \textit{Alten} er \textit{examinered}, efter \textit{pag.} 237 dens Udviis; Hvilken, i Kraft af sin forrige aflagde Eed, foruden det, han tilforn vidnet har, nu, efter\par
Sp. 21. forklarede sig nærmere, som 48de Vidne \textit{pag.} 368 f. og de andre Vidner \textit{pag.} 374 udsagt have: dog med den Forskiell:\par
Da han for en halv Snees Aar har faret i \textit{Koutokeino}-Fielde, veed han, at \textit{Koutokeino}- fælles\textit{Finner} ikke komm, eller laae om Vinteren længere Nord imod \textit{Altens} Fielde, end til \textit{Kalbo-jok inclusive}, over 3 Field-Mile Sønden for \textit{Masi-jok}, eller \textit{Masi-Capell}.\par
Om \textit{Avjevaras} fælles \textit{Finner} siger det samme, som 5te 6te og 7de Vidner her \textit{pag.} 374 vidnet have.\par
Om \textit{Arisbye}-Fælles\textit{Finners} Leje veed intet.\par
Sp. 22. Svarer: For en 10 Aar siden, da han foer der, har \textit{Altens Norske Finner} gaaet imod de \textit{Koutokeiner} til hen imod \textit{Kalbo-jok}.\par
Om de andre \textit{Norske Finners} Gang i Søer siger det samme, som Bem.te 5te 6te og 7de Vidner \textit{pagina} 374 her forklaret.\par
Om \textit{Laxefiords Norske Finner} veed intet.\par
Sp. 23. Svarer, som de andre \textit{pag.} 370 dog veed intet af \textit{Arisbye Finnen} paa \textit{Veinæss}.\par
Til de Øvrige Spørsmaale indtil Enden svarer han som de berørte 3de Vidner.\par
Hvorpaa Retten paa dette Sted blev slutted.\hspace{1em}\par
\textit{Karasjok} d. 30 \textit{Martij} 1745. \hspace{1em}\centerline{Peter Schnitler.}\centerline{Peder (L.S.) Joxsen}\centerline{Niels (L.S.) Pedersen}\hypertarget{Schn1_110908}{}Schnitlers Protokoller V.\par
Endnu tilkom om de saakaldede \textit{Svenske} Ting- og Markesteder deres Beskaffenhed og Omstændigheder følgende Forklaring:\par
\textit{Pag:} 369 er meldt, at ved denne \textit{Karasjok}-Elv fra \textit{Jetzjok}-Kæften til \textit{Karasjoks}-Kæften holder ingen af de fælles \textit{Svenske} Field\textit{Finner} til: men for omtrent 20 Aar siden have nogle \textit{Svenske} saa kaldede \textit{Qvæner} eller \textit{Torne}-Bønder sig paa Syd-ostlige Elv-Bræde af denne Streknings \textit{Karasjok} nedsatt;\par
Efter Tilspørsel forklarede nu \textbf{5te Vidne}\textit{Peder Joxsen}, og \textbf{6te Vidne}, \textit{Hendrik Povelsen}, at denne Herre-Stue af Timmer, hvor Øvrigheden holde til i ved \textit{Karasjok}, hører den her boende \textit{Qvæn Mathies Eriksøn} til, hvis Fader haver bygget den. ‒\par
Den saakaldede \textit{Kongs}Gamme i \textit{Avjevara} paa Søndre Side af \textit{Jetzjok} skal for en 11. à 12 Aar siden være bygget af Timmer paa \textit{Avjevaras Finne}-Almues Bekostning for de 2/3 Parter, da der kun tilforn har været en Torv-Gamme, hvori de Kongelig-\textit{Svenske} Betientere Ting og \textit{Torne}Borgere Market have holdet; Derved sidde ikke \textit{Avjevara-Finner}, uden ved Ting- og MarketsTiden; Thi om Sommeren ere de mest ved den \textit{Norske} Søe-Kant, om Vinteren ved \textit{Sios-jaure}, og oppe ved \textit{Kiølen}, saaledes at naar de 3 à 4 Uger paa et Sted have ligget, fløtte de derfra andenstedz hen.\par
Ligesaa ved \textit{Koutokeino}-Kirke kan nogle faa, ved \textit{Kalbojok} nogle, ved \textit{Rigna-jaure} op ved \textit{Kiølen} andre ligge, hvilke, ligesom om næstforrige er meldt, iidelig fløtte.\par
Paa samme Maade ved \textit{Arisbye} Kirke komme \textit{Finnerne} nu ikke sammen, uden til Prædiken, som siges, hver 3die Helgen, og til Ting- og Marke-Tiden; Thi om Vinteren ligge en Deel ved \textit{Tanen}, en Deel ved \textit{Merisjaure} og efter \textit{Otzjok}-Aaen, andre ved \textit{Vækie-jaure} og \textit{-jok} og fløtte med deres Reen fra et til andet Sted; Om Sommeren ere de mest alle ned ved den \textit{Norske} Søe eller \textit{Norske} Elve. Saaledes forklaret, \textit{testerer} med LaugRettes Mænd, \textit{datum ut supra}\hspace{1em}\par
\centerline{Peter Schnitler.}\centerline{\textit{Nicolaus} (L.S.) \textit{Mortensen} (\textit{Nicolaus Mortensen})}\centerline{\textit{Peder} (L.S.) \textit{Olsen}}\hspace{1em}
\DivII[Mars 31. Fra Karasjok til Avjovarre]{Mars 31. Fra Karasjok til Avjovarre}\label{Schn1_111118}\par
d. 31te Reiset fra \textit{Karasjok} Markested paa \textit{Karasjok}-Elven i Syd-vest til \textit{Jetzjok}-Kæften 2 FieldMile. Siden paa \textit{Jetzjok} omtrent i Syd-vest til \textit{Avjevara} ‒ som ligger paa Søndre Side af \textit{Jetzjokelven}, 3 FieldMile. Her omkring var Jngen af de fælles \textit{Avjevara-Finner} tilstede, men hørtes, at ligge i Søer herfra op paa \textit{Jaurisduøder}. Her mødte
\DivII[April 1. Rettsmøte i Avjovarre]{April 1. Rettsmøte i Avjovarre}\label{Schn1_111160}\par
\centerline{\textbf{54de Vidne i Finmarken}\textit{Lasse Pedersen, Norsk Porsangers FieldFinn}}\hspace{1em}\par
fødd i Fieldene oven for \textit{Porsangers} Botten, døbt i \textit{Avjevara} ‒ i en Gamme af en \textit{Svensk} Præst, 26 Aar gammel, u-gift, altid opholdet sig hos sine Forældre om Vinteren i de \textit{Norske}\textit{Porsangers}Fielde, om Sommeren ved \textit{Porsangers}Fiord, sidste gang været til Gudz Bord ved sidstafvigte Mikkelsmiss i den \textit{Norske Kistrands Capell}. Dette Vidne i LaugRettes Mændz Overværelse aflagde sin Corporlig Eed, at sige, hvad ham bevist var, sandferdeligen; og vidnede:\hypertarget{Schn1_111213}{}54de Vidne i Finmarken. Karasjok.\par
Han har vel hørt, at \textit{Jaurisduøder} eller \textit{Kiølen} giør Grændse-Skiellet imellem \textit{Norge} og \textit{Sverrig}, har og været der engang om Vinteren: men kan ikke udviise Grændse-Merkerne.\par
Han veed, at \textit{Norske Porsangers} Field-\textit{Finner} have i nogle Vintre siddet, og nu sidde ved \textit{Karasjoks} Elv, 1 à 2 Mile Sønden for \textit{Jetzjok}-Kæften, og 3 à 4. Mile i Syd-Syd-vest fra \textit{Karasjok} Markested, og det ei alleene paa den Vestre ‒ men og paa den Østre Side af \textit{Karasjok}.\par
Fremdeles veed han, saalænge han kan mindes, har \textit{Norske Porsangers Finner} om Vinteren siddet ved \textit{Karasjok}-Elv paa dens Vestre og Østre Side 2 Mile Sønden for \textit{Karasjok}- Markested, nær norden for \textit{Jetzjok}-Kæften, og sidder endnu den \textit{Norske PorsangersFinn Niels Nielsen} med andre fleer \textit{Norske Finner} sammestedz baade Norden og Søndenfor \textit{Jetzjok}- Kæften.\par
Denne \textit{Niels Nielsen}, som er \textbf{8de Vidne i Alten} efter \textit{pag.} 237 her, var tilstede, og i Kraft af hans forhen aflagde Eed stadfæstede næstforrige Vidnets Udsagn, at han med fleere \textit{Norske Porsangers Finner} om Vinteren har ligget ved \textit{Karas-jok} paa dens Vestre og Østre Side norden- og Sønden for \textit{Jetz-jok}-Kæften; og for hans Tid har andre \textit{Norske Finner} ligget sammestedz norden for \textit{Jetz-jok}-Kæften.\par
Bemeldte \textit{Nielsen} har for en 4 Aars Tid været en fælles \textit{Avjevara-Finn}, der tilforn har betalt Skat baade til den Kongelig \textit{Norske} Foged, og den Svenske Field-Øvrighed, men for 4 Aar har han begivet sig fra \textit{Avjevaras} fælles Land ned til \textit{privative Norsk} Land, hvor han om Sommeren sidder ved den \textit{Norske}\textit{Porsangers}Fiord, og om Vinteren ved \textit{Karasjok} iblant andre \textit{Norske} Field-\textit{Finner} paa \textit{Norsk} Grund, og holder sig nu alleene til den \textit{Norske} Kirke og det \textit{Norske} Ting; Efter denne sin Fløtning over til \textit{Norge}, sender den Kongelig \textit{Svenske} Foged hannem Bud, at han alligevel skal vedblive at betale sin Skatt til \textit{Sverrig;} Thi spørger han sig for: Om han til den Kongel. Svenske Foged skal betale Skatt, enddog han har givet sig til at være en \textit{Norsk} FieldFinn, og opholder sig iblant andre \textit{Norske Finner} paa eene \textit{privative Norsk} Grund?\par
Retten gav hannem herpaa den Beskeeden:\par
Efterdi han fra \textit{Avjevara}-fælles Land har begivet sig over til \textit{Norges} eeget \textit{Finmarken}, og bruger Vinter og Sommer alleene \textit{privative Norsk} Grund, holder sig og eene til den \textit{Norske} Kirke og det \textit{Norske} Ting, saa har han til den Kongel. \textit{Svenske} Foged ingen Skat at betale, førend den anordnede Kongelige Grændse-\textit{Commission} har naaet sit Udfald.\par
Vidnet vidste ei meere; Thi blev Forhøret her slutted.\hspace{1em}\par
\textit{Avjevara} ved \textit{Jetzjok} d. 1 \textit{April.} 1745. \hspace{1em}\centerline{Peter Schnitler.}\hspace{1em} L. S. Peder Nielsen\textit{Oster Porsanger} finn L. S. Jakob Jonsen Vestre Porsanger fin L. S. Lars Nielsen Vestre Porsanger finn\hypertarget{Schn1_111501}{}Schnitlers Protokoller V.\par
\textit{Kongelig-Norske Finner}, som opholde sig Vinter og Sommer paa \textit{privative-Norsk} Grund, og holde sig eene til \textit{Norske} Kirker og Ting, have til den Kongel. \textit{Svenske} Foged ingen Skatt at betale: Forinden den anordnede Kongelige Grændse-\textit{Commission} har naaet sit Udfald.\hspace{1em}\par
Ved \textit{Jetzjok} d. 1 \textit{April.} 1745. {Peter Schnitler.}\hspace{1em}\par
Fo[r]bem.te 8de Vidne \textit{Niels Nielson} blev ligesom de 2de Andre \textit{pag.} 371 tilspurdt: Om han trøestede sig om Vinteren til at udviise Grændse-Merkerne paa \textit{Kiølen?} Han svarede i LaugRets mændenes Paahør: At han trøstede sig dertil der hvor de store Elve, som \textit{Karasjok} af \textit{Auske Suppetok}, og \textit{Tana}-Elv ved \textit{Borv-oive} og \textit{Gaune-jaure, opkomme;} men hvor Merkerne ere imellem andre smaa Bække, vidste han ikke, om om Vinteren saa lige kunde udfinde og anviise.\hspace{1em}\par
\textit{Datum supra}{Peter Schnitler.}\hspace{1em}
\DivII[April 2. Fra Avjovarre til Masi]{April 2. Fra Avjovarre til Masi}\label{Schn1_111591}\label{Schn1_111593} \par 
\begin{longtable}{P{0.8041079812206573\textwidth}P{0.03192488262910798\textwidth}P{0.013967136150234743\textwidth}}
 \hline\endfoot\hline\endlastfoot d. 2den Reiset fra \textit{Avjevara} paa \textit{Jetzjok}, hvor paa Elv-Brædene kun Bierk og smaa Sletter vare, i Vester til Sønden til \textit{Siosjaure}\tabcellsep 3 Mile\\
Herfra over slette bare Fielde, med noget Bierkeriis imellem, i Vester til Norden til \textit{Norske Masi Capell}, som er paa Østre \textit{Altens} Elv-Bræde, imod \textit{Masi}-Aae, som paa Vestre Side i \textit{Alten} indfalder, lidt Norden derfor\tabcellsep 3 ‒\\
\tabcellsep _______\tabcellsep 6 Mile.\end{longtable} \par
 \hspace{1em}
\DivII[April 3.-5. Eksaminasjoner ved Masi]{April 3.-5. Eksaminasjoner ved Masi}\label{Schn1_111647}\par
d. 3die \textit{April}: kaldet til mig i en Gamme ved \textit{Masi-Capell} den \textit{Koutokeino}-fælles\textit{Field Finn}, \textit{Mikkel Aslaksen}, som i forleden Vaar i \textit{Finmarkens}\textit{Altens} Gield har været 2det Vidne, efter \textit{pag.} 233 og tilspurdt ham, i LaugRettes Mændz Paahør, i Kraft af hans forhen aflagde Eed\par
1. Om han trøstede sig, om Vinteren at udviise de Merker paa \textit{Kiølen}, som han i fior har bevidnet, at have seet, nemlig fra \textit{No} 17. \textit{Salvasvadda}, til \textit{No} 33 \textit{incl. Laddegein?}\par
Han svarede: Ja.\par
2. Om det ikke siunes ham bedre og lettere for de Hrr \textit{Committerede} til Grændse-Maalingen, at de befare \textit{Kiølen} om Vinteren, end om Sommeren? Og paa hvad Tid?\par
Svarede: Ja; det var bedre, og at dermed begyndtes efter Hell. 3 Konger.\par
3. Hvor de \textit{Koutokeino-Finner} om Sommeren ved den \textit{Norske} SøeKant? og hvormange paa hver Sted?\par
Svarede: J den \textit{Norske Langfiord}, en Jndfiord af \textit{Al[t]ens} Hoved-Fiord, ligge af \textit{Koutokeino-Finner}{3 Familier} hvoriblant han, Vidnet \textit{Mikkel Aslaksen er}. J \textit{Talvigen}{5 ‒ ___________ giør 8 \textit{Familier}}\hspace{1em}\par
De Øfrige af \textit{Koutokeino} søge ned til \textit{Tromsøens Norske} Fiorder nordest i \textit{Nordland}, nemlig i \textit{Qvænangen, Oxfiord, Strømsfiord}, \textit{Reisens}Fiord; hvormange paa hvert Sted? vidste ikke.\hypertarget{Schn1_111798}{}Vidner i Masi FieldBye.\par
4. Hvorlangt \textit{privative Norske Altens} Field-\textit{Finner} ere om Vinteren vandte til, at gaae i Søer imod \textit{Koutokeinos} Fælles\textit{Finner?}\par
Svar: J gammel Tid for en 5 à 6 Aar siden har \textit{Altens Norske Finner} et par Aar siddet med deres \textit{Familier} og Boeskab en 6 Mile Sønden for \textit{Masi-Capell} Østen for \textit{Altens} Elv: Siden den Tid nu have de siddet Omkring \textit{Masi-Capell}, neml. baade Sønden- østen- og Norden derfor.\par
5 Hvorlangt \textit{Koutokeinos} fælles \textit{Finner} ere vandte til at sidde i Nord imod \textit{Altens Norske Finner} om Vinteren?\par
Svar: J forrige Tider have de \textit{Koutokeino-Finner} ligget iblant \textit{Altens Finner}, da disse gienge saa vidt i Søer; Ellers pleje de at ligge en Deel omkring \textit{Koutokeino} ved \textit{Altens} Elv, 2 à 4 Mile Norden for \textit{Koutokeino}-Kirke, som er 4 Mile Sønden for Norske \textit{Masi-Capell}, en Deel tilholde paa og ved \textit{Jaurisduøder}, eller Kiølen. \par
Vidnet blev da betydet, at komme ned til \textit{Majoren} ved Elv-Bakken paa \textit{Altens}Fiordz Botten, tillige med \textit{Koutokeino}-Lensmand \textit{Rasmus Siversen}, og den Finn Erik \textit{Nielsen}, inden 10. à 12 Dage, for at aftale tillige med \textit{Finmarkens} Amtmand, om den forestaaende \textit{provisionelle} Befaring med dennem.\par
Hvorpaa \textit{Dimittered}.\par
\textit{Jon Nielsen}, \textit{Koutokeino}-FieldFinn, \textbf{1te Vidne} i \textit{Altens} Gield, efter \textit{pag.} 230 var til stæde, og i kraft af sin aflagde Eed stadfæstede sit forrige Udsagn, og derefter sagde alt det samme, som næstforrige \textit{Mikkel Aslaksen} har forklaret.\par
Dette saaledes rigtig \textit{passeret, testere}, ved \textit{Masi}-Elv, paa \textit{Altens} Østre Side, der hvor \textit{Masi Capell} staaer, d. 3 Apr. 1745.\hspace{1em}\par
\centerline{Peter Schnitler}\hspace{1em}\textit{Nils} (L.S.) \textit{Pedersen}. \textit{Anders} (L.S.) \textit{Iversen}\textit{Aslak} (L.S.) \textit{Pedersen}\hspace{1em}\par
d. 4de April: næstefter ankom \textit{Koutokeino} Lensmand \textit{Rasmuss Siversen}, \textbf{3die Vidne {i}}\textit{Alten}, efter \textit{pag:} 235 som forklarede Dagen efter\par
d. 5te: med de 2de \textit{Qvænangenske} Vidner af \textit{Tromsøens} Fogderie i \textit{Nordland}, navnlig \textit{Raste Rastesen Pundskei}, det 17de Vidne i \textit{Tromsøens} Fogderie, efter 4de \textit{Volumen}, II 388, og \textit{Jon Mikkelsen}, det 20de \textit{Vidne} sammestedz, efter 4de \textit{Volumen}, II 388, 3de Uger for Mikkelsmiss omtrent, at have befaret Field-\textit{Kiølen}, i fior Ao 1744. i Selskab med den \textit{Norske Qvænangens} Skolemester, \textit{Mons Jakobsen}, og 2de andre \textit{Qvænangens Norske} Søe-\textit{Finner}, navnlig \textit{Hendrik Nielsen}, og \textit{Jon Olsen}, fra de Fielde \textit{Kieldevadda} til \textit{Paresoive}, og der, hvor GrændseGangen imellem Elvenes Fald er, nedsatt og med Steene neden omkring fastgiort høye BierkeStænger, hvorpaa Skolemesteren med Kniv har udskaaret Mændenes Navne og Aarstallet; Derfor han 4 Mark danske havde faaet paa Haanden, og haabede, at faae det meere, som ham med Rette kan tilkomme.\par
Derefter blev hannem forelæst, hvad igaar Vidne \textit{Mikkel Aslaksen} havde udsagt, og til\par
Sp. 1. svarede, ligesom næstforrige; ja\hypertarget{Schn1_112137}{}Schnitlers Protokoller V.\par
Sp. 2. Svarede det samme, dog meente han, det var for den sterke Kuld i \textit{Januario}, at med Opmaalingen først efter Kyndelsmiss best kunde begyndes, da dermed kunde fortfares til omtrent Kaarsmiss-Tid.\par
Videre sagde denne \textit{Rasmuss Siversen}, det var tienligst, at nu i Sommer \textit{Kiølen} for i Vejen blev befaren, og udmerket af Vidner og BesigtelsesMænd, saa kunde man Vinteren derefter des bedre og fortere \textit{avancere} med Forretningen; og, om han for sin i fior giorde Befaring bliver betalt med den efterstaaende Rest, tilbyder han sig at fuldføre den Øvrige \textit{provisionelle} Befaring i tilstundende Sommer, fra \textit{Parse-oive} til \textit{Laddegein} eller \textit{Beldovadda} (Enden af \textit{Kiølen}). Til den Ende, han forsikrer at være hiemme ved \textit{Koutokeino}-Kirke just \textit{Bartholomei} dag, gammel stiil, som er 12 Dage efter den \textit{Norske Bartholomæi}, ny stiil, da den \textit{Norske Altens} Skolemester maatte møde ham der paa samme Dag, at følge med ham. ‒\par
\textit{Majoren} lovede ham rigtig Betaling for den i fior forrettede \textit{provisionelle} Befaring, og gav ham til den Ende en skriftlig Anviisning til den Kongelig \textit{Tromsøens} Foged, at see hannem endelig for sin hafde Umage tilfredzstilled, siden man i tilstundende Sommer havde hans Tienneste til samme \textit{provisionelle} Befaring for \textit{Finmarken} fornøden.\par
Efter Tilspørgende begierede han for sin Person daglig 24 s. og for hver 1 Reen 16 s. daglig, det de andre \textit{Norske} Besigtelses Mænd og ville have. ‒\par
\textit{Majoren} spurde de \textit{Koutokeino-Finner:} om de kunde om Vinteren møde de Herrer \textit{Jngenieurer} ved og Sønden for \textit{Halde}, og hvad de for 1 Reen og Kieredz med Karl begierede i Leje, og hvad en Kieredz kunde føre?\par
De Svarede: de kunde møde Sønden for \textit{Halde}, og meene at faae for 1 Reen med Kieredz og Reedskab dertil 1 Mark daglig, fra den Tid, de bruges, og for en Karl, som fører 5 à 6 Reen, 24 s daglig: dog naar Herrerne ligge stille, begieres for de hvile-dage intet for Reenen, men Kør-Svendenes Dag-Penge vill gaae fort hver Dag, hvad enten der fares, eller hviles. Een Kieredz føres med 4 Voger.\par
Sp. 3. Svarer, som næstforrige \textit{pag.} 378, hvor de \textit{Koutokeiner} om Sommeren ligge; Af dennem ligger den FieldFinn \textit{Erik Nielsen} hans Reen ved \textit{Talvigen}, hvor de komme strax efter \textit{Sanct Hans} Tid, men for sin Person plejer han at ligge ved \textit{Masi}, og fisker i \textit{Altens} Elv. ‒ og Lensmand \textit{Rasmuss Siversen} holder om Sommeren til ved \textit{StrømsFiord}, lidt for \textit{St Hans} Tid.\par
Sp. 4. Svar: De \textit{Norske Altens Finner} i de Aarer, at vildReenskøtterie gaaer til i \textit{Jaurisduøder}, ere vandte til at ligge 5 à 6 Mile Sønden for \textit{Masi} paa Østre Side af \textit{Altens} Elv baade i gammel Tid og nu, at forstaae de \textit{Altens Finner}, som ere Skøttere efter vild-Reen.\par
næstforrige Vidne \textit{Mikkel Aslaksen}, som \textit{pag.} 378 siunes ikke hermed at komme overeens, blev \textit{confrontered}, og forklarede sig, med sit forrige Udsagn at have forstaaet den Vestre Side af \textit{Altens Elv}: Thi om dens Østre Side siger det samme, som \textit{Rasmuss Siversøn;}\par
Sp. 5. Svarer det samme, som næstforrige \textit{pag.} 379. Derpaa Forhøret blev sluttet, og af LaugRettesMænd underskreven og forseglet. ved \textit{Masi-Capell} d. 5 \textit{Apr.} 1745.\hspace{1em}\par
\centerline{Peter Schnitler.}\hspace{1em}\textit{Anders} (L.S.) \textit{Jversen}\textit{Niels} (L.S.) \textit{Pedersen}\textit{Niels} (L.S.) \textit{Jonsen}\hypertarget{Schn1_112387}{}Forrettet i Alten.
\DivII[April 5.-6. Fra Masi til Elvebakken i Alta]{April 5.-6. Fra Masi til Elvebakken i Alta}\label{Schn1_112389}\par
Samme Dato, Reiset derfra Fieldz-Leedz til en Field-Bye ved \textit{Sollojaure}{2 Field Mile}\par
d. 6te \textit{Apr:} derfra videre over Fieldene i Nord til \textit{Elvbakken} ved \textit{Altens} Elv{6 ‒ _________}
\DivII[April 7.-12. Ekspedisjonsarbeide i Alta]{April 7.-12. Ekspedisjonsarbeide i Alta}\label{Schn1_112418}\par
fra 7de til 12te \textit{Dito}, \textit{expederet} Grændse\textit{tabellerne} med \textit{geographisk} Afteigning over \textit{Finmarken} til Amtmand \textit{Kieldson}, og Provsten i \textit{Vest-FinmarkenFalk}, derefter at foranstalte den \textit{provisionelle} Befaring af dets øvrige Grændse-Merker.\par
\textit{Reqvireret} og af Velbemeldte Amtmand de andre behøvende \textit{Documenter} af \textit{Archivet}, samt \textit{confereret} med ham om Grændse-Maalingens muelig-skyndeligste Tilendebringelse, som gik derpaa ud, at den om Vinteren maatte skee. ‒\par
Udferdiget derover min Forestillings \textit{Deduction} til Hr Obriste \textit{Mangelsen}, som Kongelig Grændse-\textit{Commissaire}, med hosføyede \textit{Attestatis}, og en ligelydende Gienpart til de Kongelige \textit{Norske Jngenieurer}, at Grændse-Maalingen, om den skal have nogen Fremgang, herefter om Vinteren vill bestilles.\par
De \textit{Documenter} af det \textit{Vaardehuusske Archiv}, jeg af Amtmand \textit{Kieldson} før havde faaet til \textit{Vadsøe} tilsendt, vare et gammelt \textit{Privilegium} af den Russiske \textit{StorFørste} for \textit{Lapperne}, og nogle gamle Kongelige \textit{Ordres} Grændsen angaaendes; De, jeg nu i \textit{Alten} derefter erholdede, vare et \textit{Deductions} Skrift paa Tysk, angaaendes de Russiske Grændsers Tvistighed \textit{etc.}
\DivII[April 12.-23. Fra Alta til Lyngseidet]{April 12.-23. Fra Alta til Lyngseidet}\label{Schn1_112526}\par
d. 12te Reiset fra \textit{Elv-bakken}, og\par
d. 13de igiennem \textit{Altens}Fiord, forbi \textit{Tallvig} og \textit{Langnæss}, siden igiennem \textit{Lang}Fiord{3 3/4 Søe Mil.}\par
d. 14de Gaaet over \textit{Alt-eidet} i Vester til Søer til \textit{Altfiord}{3/4 ‒ Miil ___________}\par
d. 15de og 16de Hellig.\par
d. 17de Faret igiennem \textit{Altfiord} i Vester, saa igiennem \textit{Qvænangen} i NordVest til \textit{Meiland}, siden igiennem \textit{Reisens}Fiord til \textit{Strøms}Fiord i Søer til Vesten {3 Mile.}\par
‒ ‒ Afferdiget min Skriver-Tiener, som \textit{Express} til de Kongel. \textit{Norske Commissions} Betiente, med mit Forslag (angaaendes Vinter-Maalingen) at fare paa først-forefindende Nordfar-Jægt til \textit{Bodøe} dermed, paa det at bemeldte \textit{Officerer} kunde erholde det, førend de gave sig i Vejen til Fieldz, og at de med samme Post, som bringer et lige Forslag fra mig til Obriste \textit{Mangelsen}, kunde tillige sende til næst Velbemeldte Obrister deres Betænkning derover: Om? og hvad de derimod kunde have at erindre.\par
d. 18.19. og 20de Var Paaske-Helgen.\par
d. 21de Søgt ud ad \textit{Reisens}Fiord efter \textit{Missions} Skolemestern, at drive hannem med Vidnerne, saasnart Jorden skulle blive bar, til den \textit{provisionelle} Befaring, mest i Vester 1/2 Mil.\par
d. 22de Gaaet over \textit{Ravels}Eid i Vester til \textit{Rotsund}, siden faret i Syd-Vest til Lensmanden, at tale med ham om de fornødne Anstalter til den \textit{provisionelle} Befaring 1 1/2 Mil.\par
d. 23de derfra forbi \textit{Spaakenæss} igiennem \textit{Løngens}Fiord til \textit{Løngs}Eidet, der at tale med \textit{Missions} Skolemestern og Vidnerne om den igienstaaende \textit{provisionelle} Befaring i \textit{Løngens}- Fielde, saa og at oppebie her Posten, som ventedes, førend Jeg droeg ind ad \textit{Løngens}Fielde 2 Mile.
\DivII[April 24.-mai 9. Ekspedisjonsarbeide på Lyngseidet]{April 24.-mai 9. Ekspedisjonsarbeide på Lyngseidet}\label{Schn1_112722}\par
Som jeg paa min FieldReise af den \textit{Svenske} Præst i \textit{Arisbye} hørte, at den \textit{Svenske} Landz Høvding havde sendt hannem \textit{ordre} til \textit{publication} for \textit{Lapperne}, at de de \textit{Svenske} Grændse\hypertarget{Schn1_112749}{}Schnitlers Protokoller V. Maalere, naar de der i Egnen ankomme, ei alleene selv, men og med deres Reen for Betaling skulle gaae til Haande; Saa udferdigede\par
d. 30de jeg herfra \textit{Eidet} samme Slags \textit{circulaire} Breve til de \textit{Norske} Provster i Finmarken, i deres Provsties, og særdeles i Grændse-Kirkerne at foranstalte, at Præsterne fra PrædikeStoelene oplæse og \textit{publicere} samme mit Brev, hvorj jeg betyder, at \textit{Lapper} og \textit{Finner} saavel \textit{privative-Norske} som fælles Undersaattere af \textit{Koutokeino}-og \textit{Arisbye}-Sogner skulle alle, som af den Kongelig \textit{Norske} Øvrighed tilsiges, strax begive sig til Fieldz, til den \textit{provisionelle}\textit{Kiøls} Udmerkelse; Hvortil Folkene, efter Høylofl. Rente-Cammeretz Anordning af den Kongel. Foged skulle nyde et Forskud til deres Underhold, indtil med dennem efter Tilbagekomsten kunde giøres Afreigning; Siden skulle de samme BefaringsMænd med fleere Vidner og Arbeidere, som af velbem.te \textit{Norske} Øvrighed befales, forføye sig til den udmerkede Grændse- \textit{kiøl}, ved de Kongelige Grændse-Maaleres Ankomst, baade for den der anordnede begge høye Siders Field-Rett nærmere at forklare deres Vidne, og at hielpe Velbem.te Herrer \textit{Officerer} igiennem \textit{Districten} saa vel selv, som med deres Reen.\par
‒ ‒ Af det samme givet Amtmand \textit{Kieldson part}, og tillige bedet, ved Sorenskrivern at lade afhøre en \textit{Koutokeino-Lap}, som jeg ei kundet faae fatt paa, naar han i Sommer paa \textit{Norsk} Grund skulle komme, navnlig \textit{Erik Næver} ‒\par
‒ ‒ Skrevet og samme Tid til den Kongel. \textit{Finmarkens} Foged, at han for ingen Ting maatte negte de \textit{provisionelle} Grændse-Befarere den fornødne Forstrekning; Thi ellers derved Grændse-Verket et heelt Aar ville standses.\par
d. 30 ‒ Ankom paa \textit{Løngs-Eidet} den forventede Post Sønden-fra.\par
Til d. 8 \textit{Maj:} næstefter\textit{expederede} jeg forberørte mine Skrivelser, med endnu ydermeere Forklaring over den foreslagne Vinter-Maaling til Hr Obriste \textit{Mangelsen} og de Kongelige \textit{Norske Commissions} Betiente.\par
‒ Skrevet og \textit{circulaire} Breve til de Kongel. Fogder og Sorenskrivere i \textit{Nordland}, at med de i fior foranstaltede \textit{Lappe}Ting i Aar, og saa videre fortfares; Af Sorenskriverne i sær \textit{reqvireret} Gienparter af deres holdene Field\textit{protocoller}, for at nedsendes.\par
d. 9de Hellig ‒
\DivII[Mai 10. Fra Lyngseidet til Lyngsbotn]{Mai 10. Fra Lyngseidet til Lyngsbotn}\label{Schn1_112893}\par
d. 10de Reiset fra \textit{Løngs}Eidet til \textit{Løngens}Botten{2 1/2 Mil.}\par
d. 11te Giort Bud i Fieldene til \textit{Enotekies Svenske Lapper}, at komme, og tale med mig. Jmidlertid
\DivII[Mai 12.-28. Eksaminasjoner i Lyngsbotn]{Mai 12.-28. Eksaminasjoner i Lyngsbotn}\label{Schn1_112924}\par
fra 12te til 15de igiennemgaaet de \textit{Documenter}, af det \textit{Vaardehuusske Archiv} ved Amtmand \textit{Kieldson}, og de af det \textit{Nordlandske Archiv} ved \textit{Etats}Raad og Amtmand \textit{Scheldrup} mig \textit{communicerede}; Og som af et \textit{Rescript} fra salig og høylofl. Konge \textit{Christiano 4to} til Amtmand \textit{Bille} i \textit{Nordland} af Ao 1609. fornam, at de \textit{Svenske} Betientere toege Skatt af de \textit{Finne}-Byer i \textit{Nordland} oppe imod \textit{Kiølen}, navnlig \textit{Sigvar, Eingvar} og \textit{Raufuholt}; Saa kaldede til mig de 2de Vidner, nemlig det 11te og 12te i \textit{Tromsøens} Fogderie\textit{Nordlands} Amt, efter 4de \textit{Vol:}II 359 og 362, som de Bekienteste, og de i deres Ungdom med smaat Kiøbmandskab i \textit{LapMarken} meget have faret, ved Navn \textit{Mikkel Qvæn Pælleg}, og \textit{Jakob Nyested}, og i Kraft af deres Vidne-Eed tilspurde dem om det, saavel som andre Poster:\par
Sp. 1. Om de kiendte de \textit{Lappe}-Byer oppe imod \textit{Kiølen, Sigvar, Eingvar}, og \textit{Raufuholt?}\hypertarget{Schn1_113044}{}J Løngens Botten og Fielde.\par
De Svarede: Nei; de kiendte dem ikke, ei heller havde hørt noget om dem.\par
Sp. 2. Om have hørt, at de \textit{Svenske} Betientere have givet nogen \textit{SvenskLap} Forlov, at sidde ved \textit{Vestre Rostojaure?} Eller taget Skatt derfor af Nogen?\par
\textit{Pælleg} svarede: For en 40 Aar siden have de \textit{Svenske} Betientere givet en \textit{Lap Anders Larsen} Lov, at sidde der, som laae der om Vaaren og Høst for Fiskeriets Skyld, saa længe han levede, og svarede Skat deraf til \textit{Sverrig}; Efter ham har en anden \textit{Svensk Lap} faaet Lov paa dette Vand, og staaer i den \textit{Svenske Matrikul} ansatt, at have \textit{Rosto-Vagge}, hvis Navn er \textit{Niels Lorwig}, siddendes om Sommeren i den \textit{Norske Ulfsfiords} Botten, som og af dette \textit{Rosto-vagge} i en 20 Aars Tid til \textit{Sverrig} har svaret Skat. ‒ Denne \textit{Niels Lorvig} skal i denne Vinter paa \textit{Enotekies-Ting} til de \textit{Svenske} Betientere have opsagt dette Field-Sæde ved \textit{Rosto-jaure}, efter di det ligger en 3 Mile Vesten for \textit{Kiølen}, og vilde falde \textit{Norge} til; Men de \textit{Svenske} Betientere havde svaret ham, han skulle alligevel blive ved, at betale Skatten deraf til \textit{Sverrig}. ‒\par
Paa lige Maade berettede \textit{Pælleg}, at de \textit{Svenske} Betientere have inddeelt Fieldene Vesten for \textit{Kiølen} iblant deres de \textit{Enontekis-Lapper}. Ellers vedbliver denne \textit{Pælleg} sin forrige Søgning, om at faae dette \textit{Vestre Rostojaure} fra de \textit{Norske} Betientere bøxlet, og sagde derhos, at de \textit{Svenske Enontekis Lapper} gerne skulle see det; Thi de kan dog ikke ligge der om Sommern med deres Reen for Myggen Skyld, og at der ingen ReenMoese er.\par
Sp. 3. Hvorpaa Laug de \textit{Svenske Østlapper} om Sommeren her i \textit{Nordland} ligge?\par
Svar: Norden-fra at reigne, saa komme Part af de \textit{Enontekis-Lapper} om Vaaren neer til \textit{Reisens-} Part til \textit{Løngens-} Part til \textit{Ulfs-} og en Part til \textit{Bals}-Fiordene i \textit{Tromsøens} Fogderie; Paa hvilket sidste Sted de Rigeste af \textit{Lapperne} skal tilholde; Fra \textit{Jokasjarfs Lappe}-Hoved-Sogn ligge og nogle ved sidstbenævnte \textit{Balsfiord}, nogle ved \textit{Malangens}Fiord, andre ved \textit{Reisens}Fiord, og saa videre Søer ad i \textit{Senniens} Fogderie. ‒ Om Vintern 5 Uger for Juul søge de \textit{Koutokeiner} de \textit{Norske} Markeder i \textit{Qvænangen} og \textit{Reisen;} de fleeste af \textit{Enontekis} og mange af \textit{Jokasjarf}, \textit{Skibots}-Market i \textit{Løngen}.\par
Sp. 4. Om \textit{Sverrigs Lapmark} Østen for \textit{Kiølen} gave de den Beskeeden: \textit{Torne-Lapmark} strekker sig Søndenfra i Nord til \textit{Finmarkenskiøl}, alt hen til \textit{Beldovadda, Kiølens} Ende i Øster, hvorfra \textit{Lapperne} søge Kirke til \textit{Enontekis}, et \textit{Annex} under \textit{Jokasjarfs} Hoved-Kirke i \textit{Torne- Lap-mark;} Og i Øster naaer dette \textit{TorneLapmark} hen imod \textit{Kimi}-Elv: Dog søge de Bønder i \textit{Kittil} til \textit{Saaddekill}-Kirke i \textit{KimiLapmark}.\par
\textit{Tornestad} er den yngste af Kiøbstæderne ved den Botniske Søe, og førend den dertil blev opphøyed, have \textit{Lapperne} i gl. Tider med deres Boliger gaaet i Søer hen imod den Botniske Søe, saa at ikke langt fra \textit{Tornestad}, siges der, at skal findes, og endnu kiendes \textit{Rudera}, og Kiende-Tegn af Jld-Steene i Jorden, som \textit{Lapperne} fordum i deres Tielde der haft have.\par
De have og hørt, at \textit{Mononiska} ved \textit{Enontekis}-Elv, 30 Mile Norden for \textit{Tornestad}, i gamle Tider har hørt \textit{Lapperne} til, men nu beboes alleene af Bønder, komne fra \textit{StorFinland}. ‒\par
J de Bonde-Bøyder, som i Lapland have \textit{etableret} sig, har og \textit{Sverrig} Soldater-Udskriving, hvoraf de yderste, eller Nordreste ere\par
\textit{Terrando} ved \textit{Galasjok} imellem \textit{Lula}- og \textit{Torne}-Elve, 19 Mile fra den Botniske Søe, og ved 6. Mile i Syd-vest fra \textit{Køngeme}-Jernverk.\par
\textit{Laudagorsky} ved \textit{Torne}-Elv, 2 Mile Sønden for \textit{Mas-Ovnen}, 21 Mile fra den Botniske Søe\hypertarget{Schn1_113436}{}Schnitlers Protokoller V.\par
\textit{Kangos} ved \textit{Laimi}-Elv 2 Mile Norden for dens Udløb i \textit{Torne}-Elv.\par
\textit{Mononiska} ved \textit{Enontekis} Elv, 30 Mile fra den Botniske Søe, som har 18 Bønder, og\par
\textit{Kætkæ-soando}, 2 Mile Norden for dette \textit{Mononiska}, der har 2 Bønder.\par
\textit{Qvollerbye}, 10 Mile Sønden for \textit{Mononiska}, der har 5 Bønder.\par
\textit{Kimi-jaure} i \textit{Kimi Lapmark}, 26 Mile fra den Botniske Søe.\par
Derimod er den Bonde-Bye \textit{Kittil} ved \textit{Vonessjok}, 26 Mile fra \textit{Kimi} Markested ved Botniske Søe, og de Norden derfor værende Bonde-Bøyder Knegt-frie.\par
5. Om Kirkernes Ælde i \textit{Sverrigs Lapmark} berettede en gammel \textit{Lap} fra \textit{Enontekis, Aslak Andersen}, 80 Aar gl., at det var for u-mindelige Aar siden, at Hovedkirkerne, nemlig \textit{Jokasjarf, Jokomok, Ariplo} og \textit{Lykselle} ere til, og vilde han slutte, at de vel ere 5 Mandz Alder, hver til 80 Aar regned, følgelig ved 400 Aar gamle: Men \textit{Annex}Kirkerne nærmere \textit{Kiølen}, navnlig \textit{Enontekis- Hytte-} og de øvrige vare yngre bygde. (Dette \textit{Pælleg} hørte.)\par
6. Om \textit{Kimi Lappe}-Byer gav \textit{Pælleg} samme Beretning som de Vidner i \textit{Varanger}\textit{pag}. 358 f. især om \textit{Gækel-Elv, Beldo-jok, Vonessjok, Kittil, Saaddekill:} dog sagde at \textit{Saaddekill} ligger fra \textit{Kittil} i Nordost en 15 Field-Mile, og Kittil fra den Botniske Søe en 30 Mile; og at \textit{Sombye} fra \textit{Saaddekill} ligger 8 Mile i Ost-Nord-ost. Om \textit{Kæme}-Bye berettede det samme, som \textit{pag.} 359 ‒ men at \textit{Kæmejok} rinder i Syd-vest i \textit{Kimi-jaure}. Om \textit{Gvellojaure} vidste intet.\par
\textit{Kusan} skulle ligge i Sydost fra \textit{Kæme}-Bye, saa at denne \textit{Kæme} er saa got, som imellem \textit{Sombye} og \textit{Kusan}, dog noget længere fra denne, nemlig \textit{Kusan}. Fra \textit{Kusan} rinder \textit{Kusanjok} syderlig ad \textit{StorFinland}, og omsider ved \textit{Ulastad} i den Botniske Søe.\par
\textit{Kimi}-Elv kommer fra \textit{Kimi-jaure}, og løber i Sydvest i \textit{Vonessjok}, med hvilken den foreener sig, omtrent 10 Mile fra den Botniske Søe; denne 6te Post stadfæsted og \textit{Nysted}.\par
7. J \textit{StorFinland} er ingen \textit{Lapper:} dog gives der vildReen, hvorefter Bønder Øster i Landet jage. Bønderne boe der efter hinanden 1/4. à 1 Miil den Eene fra den Anden, indtil nær de \textit{Russiske} Bønder.\par
Det \textit{StorFinlandske} Sprog, som Nord i \textit{Norge} og kaldes det \textit{Karelske}, eller det \textit{Qvænske, differerer} fra det \textit{Lappiske}, at den ene \textit{Nation} ei forstaaer den anden; \textit{Dialecten} selv i \textit{Stor- Finland} er saa adskillig, at de Folk inderst, eller Østerst i \textit{StorFinland} neppe forstaaer dem ved \textit{Tornestad}, ligesom de \textit{Lapper} nordest i \textit{TorneLapMark} eller i \textit{Finmarken} ikke skal kunde forstaae dem der Sønderst i \textit{Uma}.\par
8. Ang. de \textit{Russiske} Grændser fra \textit{Sverrig} af, saa støder \textit{Kimi Lapmark}, og \textit{Cajanen} i \textit{Stor- Finland} til det \textit{Russiske territorium} i Øster.\par
Den nærmeste \textit{Russe}-Stad ved Havet skal være \textit{Kæme}, hvorfra Russiske Kræmmere aarlig til de \textit{Svenske} Markeder komme, nemlig til \textit{Cajanen} og fleere \textit{Finlandske} Steder med groft Lærret, Vadmel, Salt og anden smaa Kram; Til \textit{Kusan} og de andre \textit{Kimi-Lappe}-Steder desuden med Meel. Denne \textit{Kæme}-Stad skal ligge fra \textit{Kusan} (som Vidnet \textit{Pælleg} hørt) i Øster til Søer 13 Mile; det og Vidne \textit{Nysted} siger.\par
Fra den Botniske Søe naar Vejen fares efter \textit{Torne}-Elv igiennem \textit{Qvoller}Bye, \textit{Saaddekill, Sombye} til \textit{Kusan}, som Kræmmerne tage kroged, er det til \textit{Kusan}{60 Field Mile} fra \textit{Kusan} til den Russiske Stad \textit{Kæme}{13 ‒}\par
Bliver det da 73 Field Mil. som om Vinteren fares med Reen.\hypertarget{Schn1_113900}{}J Løngens Botten og Fielde.\par
En Anden Vej fra \textit{Kæmestad} fares ligeledes kroged over Field 3 Mile med Reen, siden over Vande og Elve med Baad til \textit{Tornestad} 72 F. Mile. Endnu anden Vej tages af de Russiske Kræmmere om Vinteren med Hester og Reen fra \textit{Kæmestad} til \textit{Cajanen i Stor- Finland}{30. Mile} fra \textit{Cajanen} til \textit{Tornestad} kroged {30. ‒ _______ = 60 Mile.}\par
Om nu denne \textit{Kæmestad} er den samme, som paa \textit{Carterne} betegnes med det Navn \textit{Candalax}, det kunde Vidnet ei for vist sige, men meente, at det kunde vel være.\par
Det om Vejene berettede \textit{Pælleg, Nysted} vidste og det samme, at sige, undtagen om Vejen fra \textit{Cajanen} til \textit{Kæme Russe-}Stad, hvor lang den var, havde han ei hørt. Begge komme derj over eens, at med Hester og Slæder om Vinteren kan fares fra \textit{Cajanen} paa 3 Mile nær \textit{Kæmestad}, der et Field skal ligge, hvorover med Reen maa køres, som de Russiske \textit{Lapper} have.\par
9. Om et særdeles Merke i \textit{Jndiager} berettede Vidne \textit{Nysted}, at paa en stor Holm i \textit{Jndiager}-Vand har et gammelt stort Furre-Træ staaet, hvorpaa med Kniv [har] været udskaaret de 3de \textit{Riger Norge, Sverrig} og \textit{Rusland}, og kaldes denne Holm deraf 3 Kongers Holm; Dette Vidne \textit{Nysted} er nu 46 Aar gl., og har seet dette Merke-Træ, der han var 12 Aar gl., men om det er til endnu? veed ej.\par
Dette saaledes, som forestaaer, at være, have vi til deels hørt, til deels vide.\hspace{1em}\par
\textit{Kielengen} d. 15 \textit{Maj.} 1745. \hspace{1em}\centerline{\textit{Jacob Nysted.}}\centerline{M N P (Nikkel Nielsen \textit{Pælleg}.)}\hspace{1em}\par
Til Vitterlighed underskrive dette ‒ \centerline{N M s}\centerline{E M S}\hspace{1em}\par
Om Grændse-Maalingen forklarede de begge Vidner, at den ei lader sig giøre fra \textit{Kalkogaabb} til \textit{Halde}, ei heller fra \textit{Halde} til \textit{Korsevara} om Sommeren; Thi der er da ikke frem-kommendes med Heste deels for store Steen-Urer, deels for Mangel paa Græss; Med Reen kommer man ei heller frem om Sommeren, fordi da hverken Reen eller Folk kan begaae sig for Myg og Klæg; Reenen er da haarløs og som svagest, at han ganske lidet bærer; Til med skal Field- \textit{Finnen} ei ville leje sine Reen bort om Sommeren, thi da bliver den bortskæmt; Meene derfor, at MaalingsForretningen, efter Kyndelsmiss, om Vinteren vill foretages, da baade Reenen er ved Magt, og kan fremføre noget i Kieredsen.\hspace{1em}\par
\centerline{\textit{Jacob Nysted}.}\centerline{M. N. P:}\hspace{1em}\par
fra 16de til 28 \textit{Maij} komme til \textit{Majoren} en Deel \textit{Enontekis Lapper}, tid efter anden, en 18 i Tallet, med hvilke man talede angaaendes de Kongelige \textit{Norske Jngenieurers} Fløtning langs med \textit{kiølen}, naar de i denne Egn skulle ankomme; De svarede, at om Sommeren ville \hypertarget{Schn1_114131}{}Schnitlers Protokoller V. det blive vanskeligt for Folk og Dyr at fremkomme for Myg og Klæggens Skyld, ej heller kunde de da lade deres Reen til de Herrers Fodringskab paa den Tid: men om Vinteren ville de vel skydse de ankommende \textit{Norske} Hrr \textit{Officerer}, og begierede de da overhovedet paa den Vej fra \textit{Kalkogaabb} til \textit{Pitzekiolme}, som de med sine Kroger ville regne for 20 Field Mile, for 1 Reen allvejs med Kieredz 48 s. og for 1 Raider, eller Reen-Kører i alt 3 r\textit{d.} 32 s. forklarendes derhos, at til en 50 Reen behøvedes kun 4 saadanne Raider ‒ Skulle de skydse længere end til \textit{Pitzekiolme}, saa fikk de have ny Betaling efter \textit{accord}.\hspace{1em}\par
Saaledes\label{Schn1_114163} \par 
\begin{longtable}{P{0.06538461538461539\textwidth}P{0.7009838998211091\textwidth}P{0.034973166368515204\textwidth}P{0.04865831842576029\textwidth}}
 \hline\endfoot\hline\endlastfoot for 1.\tabcellsep \textit{Lap}\textit{Thomas} Nielsen lovede \textit{Jakob Nysted}\tabcellsep 10\tabcellsep Reen med Kiereds.\\
2.\tabcellsep ‒ \textit{Peder Olsen} udlovet selv\tabcellsep 3\tabcellsep ‒\\
3.\tabcellsep for \textit{Morten Hermelin} udlovede \textit{Nysted}\tabcellsep 10\tabcellsep ‒\\
4.\tabcellsep ‒ Jon Ammondsen selv lovet\tabcellsep 2\tabcellsep ‒\\
5.\tabcellsep ‒ \textit{Aslak} Andersen\tabcellsep 6\tabcellsep ‒\\
6.\tabcellsep ‒ \textit{Lars Larsen}\tabcellsep 1\tabcellsep ‒\\
7.\tabcellsep ‒ \textit{Peder Thomasen}\tabcellsep 2\tabcellsep ‒\\
8.\tabcellsep fattig\\
9.\tabcellsep fattig\\
10.\tabcellsep fattig\\
11.\tabcellsep sagde at have ingen Reen.\\
12.\tabcellsep \textit{Anders} Jonsen\tabcellsep 3\tabcellsep ‒\\
13.\tabcellsep \textit{Hendrik} Guttormsen\tabcellsep 3\tabcellsep ‒\\
13.\tabcellsep \textit{Ole Nielsøn}\tabcellsep 3\tabcellsep ‒\\
15.\tabcellsep Lars Andersen, ville være Raider, og skaffe\tabcellsep 2\tabcellsep ‒\\
\tabcellsep \tabcellsep \multicolumn{2}{l}{_______}\\
\tabcellsep \tabcellsep 45\tabcellsep Reen\\
16.\tabcellsep 17. og 18. tilkomme siden, udlovendes hver 1. Reen, giør\tabcellsep 3\tabcellsep ‒\end{longtable} \par
 \hspace{1em}\par
Disse \textit{Enontekis Lapper} bleve nu befalede, at møde paa \textit{Norsk} LandzTing i \textit{Rotsund} d. 10 Junj førstkommende, der at faae videre Beskeeden af den Kongel. \textit{Norske } Øvrighed.\par
Ellers er disse \textit{Enontekis Lappers} Adfærd denne: De fare Østen fra over Field\textit{kiølen} neer til den \textit{Norske} SøeKant en Deel om Kaarsmiss, en Deel mitt i \textit{Maj.}, endeel om PintzeTid for og derefter alt til \textit{St Hans} Tid; paa Vejen fra \textit{Kiølen} til Søen tilbringe de 2 à 3 Samdynge, og ligesaa lang Tid paa TilbageReisen; En Deel af dem blive nu ved den \textit{Norske} SøeKant til \textit{St Olufs} Tid, en Deel til \textit{Bartholomæi} og derefter til sidst \textit{in augusto;} Medens de ere her neere, nære de fattige sig mest af Søen, de Rigere blive ved deres Reen i Fieldene, hvor der er Græss, og mindre Utøyg af Mygg og Klægg, for hvis Skyld de ei kan begaae sig i \textit{Sverrig}.\par
Saaledes at være \textit{testerer}, og har jeg Nysted lovet de 20. Reen med Kieredser, som meldt.\hspace{1em}\par
d. 28. \textit{Maj} 1745. \textit{Jacob Nysted}.\hspace{1em}\par
M N P (Mikkel Nielsen \textit{Pælleg})\hypertarget{Schn1_114464}{}Paa Reisen i Tromsøen og Sennien.
\DivII[Mai 29.-juni 4. Ekspedisjonsarbeide]{Mai 29.-juni 4. Ekspedisjonsarbeide}\label{Schn1_114466}\par
Fra 29de \textit{Maj}: til 4de \textit{Junij}\textit{inclus: Expederet} min \textit{Deduction} og udførlige Forklaring over \textit{Finmarkens} Grændser ad \textit{Sverrig} og \textit{Rusland} til Amtmand \textit{Kieldson}, og især angaaendes den yderste Grændse-\textit{Lappe}-Bye \textit{Peisen}, som for \textit{Rusland} og \textit{Norge} er fælles, givet til \textit{Commandanten} paa \textit{Vaardehuus}, Obrist\textit{Lieutenant Passau} min u-forgribelig Betænkning til kiende, begge saavel Amtmanden, som \textit{Commandanten} til nøyere Overlæg og Efterretning.
\DivII[Juni 5.-juli 25. Mindre reiser i Tromsø og Senja fogderier, bygdeting, lappeting, eksaminasjoner]{Juni 5.-juli 25. Mindre reiser i Tromsø og Senja fogderier, bygdeting, lappeting, eksaminasjoner}\label{Schn1_114532}\par
d. 5te Begivet mig fra \textit{Løngens}Botten i Nord-Nord-ost igiennem \textit{Løngens}Fiord til \textit{Spaakenæss}{3 3/4 Søe Mile.}\par
d. 6te7de og 8de Vare PintzeHelgen ‒\par
d. 9de Fra \textit{Spaakenæss} til Lensmand \textit{Oderup} i \textit{Rotsund}{1/2 Miil} hvor \textit{ordinaire} Bøyde-Ting af de Kongel. \textit{Norske} Betientere skulle holdes. Dette BøydeTing\par
til d. 12te Varede; Paa hvilket de \textit{provisionelle} Grændse-Befarere saavel de af \textit{Qvænangen}, som \textit{Lappe}-Lensmanden fra \textit{FinmarkensKoutokeino}, for deres Field-Gang i fior, af Fogden, efter Amtmandens Foranstaltning, bleve betalte; \textit{Jtem} de Befaringsmænd af \textit{Løngens}Fiord opnævnte, og med en Forstrekning forsiunede til den \textit{provisionelle} Grændse-Gang fra \textit{Kalkogaabb} til \textit{Pitze-kiolme}. Forbesagde \textit{Koutokeino}-Lensmand lovede derpaa, at fuldføre sin \textit{provisionelle} Befaring med de \textit{Norske} Besigtelses Mænd af \textit{Finmarken}, fra \textit{Pares-oive}, igiennem \textit{Finmarken}, til \textit{Beldo-vadda}, \textit{Kiølens} Ende.\par
Den \textit{Norske} Bonde \textit{Pælleg}, fornam jeg og, at have faaet Tilladelse, at bygge i Kongens Alminding. Jmellem de Kongel. Betiente og \textit{Lapperne} blev her aftalt, at \textit{Lappe}-Ting (som fra gammel Tid været Brug) fremdeles herefter skulle holdes, dog beleiligst i \textit{Rotsund}. Man foer derefter til \textit{Langesund}, hvor Bøyde-Tinget ‒\par
til 16de varede, og de \textit{provisionelle} Befarere af \textit{Balsfiord} frembragte deres \textit{Relation} over deres Grændse-Gang fra \textit{Stokkeborre} til \textit{Kapo-vara}, og af Fogden bleve betalte. Derefter været ved Tinget paa \textit{Bredstad}\par
til 19de hvor \textit{Lappe}-Ting, efter gammel Brug blev beskikket, at skulle \textit{reassumeres} paa et for \textit{Lapperne} beleiligst Sted. ‒ Siden faret til \textit{Wang} i \textit{Senniens} Fogderie, hvor \par
d. 25de Øst-\textit{Lapperne} af \textit{Malangens} Fielde forklarede, at de om Vintern ikke vel kan skydse de efterkommende Hrer \textit{Officerer} langs efter \textit{Kiølen;} Thi 1. ligge de paa den Tid langt Østen for \textit{Kiølen}, omtrent ved \textit{Jokas-jarfs Lappe}-Kirke, en 20 Mile meer og mindre fra \textit{Kiølen}, for det 2. Er paa \textit{Kiølen} hvor deres \textit{Districtes} Grændse er, slet ingen Skov, at Folk om Vintern kunde varme, eller opholde sig ved: Men for dem (\textit{Lapperne}) kunde det være beleiligst, at hielpe velbemeldte \textit{Officerer} frem baade med deres Personer, og Reens-Dyr om \textit{Bartholomæi}, som er en 5 Uger for \textit{Mikkelsmiss;} Og kunde de begynde, at tage dem fra \textit{Kamasmutkie}, et Eid imellem \textit{Østre}- og \textit{Vestre Kamas-jaurer}, nord-efter, saa vidt de bekiendt ere.\par
Disse \textit{Jokasjarfs Lapper} fare Østen-fra om \textit{Kaars}misse, og nogle derefter, over Field- \textit{kiølen}, og holde sig der nogle Uger nær ved \textit{Kiølen} paa dens Vestre Side, siden flytte de neer, imod Fiord-Botten, en 8 Dage omtrent for \textit{St Hans} Tid, blive en 4 Uger meer og mindre, herneere, og flytte op imod \textit{Kiølen} igien, og ligge der en Tid lang, men 8de Dage omtrent for \textit{Mikkels}miss fare de over \textit{Kiølen} tilbage i \textit{Sverrigs Lapmark}.\par
Hvilket saaledes blev berettet i Overværelse af \textit{Jsak Kolling}, Bonde-Lensmand, og \textit{Lars Pedersen, Malangens Lappe}-Lensmand. ‒\hypertarget{Schn1_114848}{}Schnitlers Protokoller V.\par
til 26de varede her Bøyde-Tinget; da \textit{Lappe}-Market, efter gammel Skik og \textit{Lappernes} Begier blev berammet i \textit{OuversFiord}, og Ting paa \textit{Sultenvig}.\par
‒ Jmidlertid har jeg til denne \textit{dato} expederet mit \textit{Deductions} Skrivelse til Hr Obriste \textit{Mangelsen}, i Giensvar paa de forelagde Spørsmaale, hvoraf\par
1te er, at de \textit{Norske} have opgaaet en eensidig Grændse-\textit{linie} i \textit{Lula-Lapmark?}\par
2. At den Kongel. \textit{Norske} Foged \textit{Randulf} ved Stevnemaal nedkaldet nogle paa Fieldet da befindende, men under \textit{Sverrigs} Crone liggende \textit{Lapper} til \textit{Norsk} Ting? og\par
3. J Anleedning af saadan opgaaed eensidig Grændse-\textit{linie} Skat af dennem krævet?\par
J denne min \textit{Deduction} beviiser nu\par
til det 1te At de \textit{Norske} ei i \textit{Lula-} eller andre \textit{Sverrigs Lapmarker}, men ret i GrændseGangen imellem \textit{Lula-} og andre \textit{Sverrigs Lapmarker} paa den eene ‒ og \textit{Norges Lapmark} paa den anden Side, selv efter \textit{Svenske} Undersaatteres eeget Vidnesbyrd og Udviisning, have gaaet denne \textit{provisionelle} Grændse-Gang; Hvilket paa den \textit{Norske} Side har maattet skee, paa det at, i fald Øst\textit{Lappe}Vidnerne, som et \textit{vagabond}Folk, ei skulle være at finde, ved de Hrer \textit{Jngenieurers} Ankomst, der kunde da være de, som saavel for de Kongelig-\textit{Svenske}, som \textit{Norske Commissions} Betiente kunde anviise den bevidnede Grændse; Saa at denne de \textit{Norskes præliminaire} Gang ei alleene har været fornøden til begge Kongelige Maj.ters samdrægtige Villies Fuldbyrdelse, men og i sig selv er en u-skyldig og u-skadelig Gierning; Thi om disse \textit{Norske} Grændse-Gangere, til deres eegen Efterretlighed, og deres Samvittighedz Betryggelse, i Grændsen have enten taget, eller sat noget vist Merke, hvorved de siden, ved Jgienkomsten, des bedre kunde kiende Løbet af \textit{Kiølen;} Saa kan disse deres \textit{particuliere} og eensidig-opreiste Merker dog ikke vahre længere, end til at de den samme rette Grændse-Gang for begge høye \textit{Deelers Commissions} Betiente have anviist: da samme Hrer Betiente andre deres \textit{publiqve} og stedzvarende Grændse-Merker saavel i \textit{Limiten}, som paa \textit{Carterne} og i \textit{Protocollerne} fastsætte. ‒\par
til 2det og 3die: At den Kongl. \textit{Norske} Foged har kaldet nogle paa Fieldene, under \textit{Sverrigs} Crone liggende \textit{Lapper} til \textit{Norsk} Ting og Skattens sædvanlige Ydelse? det falder hen af sig selv, naar skielligen overvejes, at \textit{Lapperne} om Sommern Vesten for \textit{Kiølen} i \textit{Norsk Lapmark} liggende, fra Arildz Tid have været, og ere sande \textit{Norske} Undersaattere, ligesom samme \textit{Lapper} om Vintern Østen for \textit{Kiølen} i \textit{Sverrigs Lapmark} ere Kongelige \textit{Svenske} Undersaattere. De \textit{Lapper} nu, som den \textit{Norske} Foged har nedstevnt, have ikke været de, paa den Tid de om Vintern i \textit{Sverrigs Lapmark} have ligget, men de, paa den Tid de om Sommern Vesten for Lands \textit{kiølen} i \textit{Norges Lapmark} have været; Til hvilken Jndstevning da den \textit{Norske} Foged har Medhold ei alleene af Folke-Retten, men og af en ældgammel Brug og \textit{praxi} i \textit{Norge;} det med mangfoldige Exempler er beviist. Saa at disse Sagens sande Omstændigheder af de første \textit{Referentere} til høyere Steder ei tilbørligen, som det burdet skee, maa være bleven andragne, eller og for \textit{Referenterne} selv have været ubekiendte.\par
d. 28de er mig til den Ende en \textit{Lappe}Tings \textit{Protocoll} for \textit{Senniens} Fogderie af Sorenskriver \textit{Kiergaard} bleven \textit{communiceret}.\par
Til d. 30de bivaanet Bøyde-Tinget paa \textit{Dyrøe}, og\par
til 6 \textit{Julij}, Tinget paa \textit{Rolløen}, hvor de \textit{provisionelle} Grændse-Befarere for deres Fieldgang fra \textit{Hierta} paa \textit{Duøderek} til \textit{Stokkeborre} bleve betalte.\hypertarget{Schn1_115148}{}Paa Reisen i Tromsøen og Sennien.\par
Paa disse mine Reiser fik fornemmelig udrettet Følgende: At, som de Kongelige Betientere i \textit{Nordland} boe alle langt borte fra Fiord-Bottene og Fieldene, og en stor Deel af dennem, som Fogder og Bonde-Lensmænd om Sommern fare til \textit{Bergen} i deres Forretninger, saa blev paa min \textit{Reqvisition}, af den Kongl. Foged en beqvem Mand inde i hver Fiord, nærmest Fieldene, af \textit{Tromsøens} og \textit{Senniens} Fogderier til Anfører eller Hovedzmand opnævnt, som ved de Kongelige \textit{Norske Commissions} Betienteres Ankomst paa hver stedz Grændser, skulle tilsige, samle og sætte i Vejen Grændse-Vidnerne og BesigtelsesMænd, med videre, saasom Norden-fra at regne:\par
\centerline{J \textit{Tromsøens} Fogderie}\par
J \textit{Qvænangens}Fiord\textit{Jørgen Gams}, en boesiddendes \textit{Normand} inde i \textit{Qvænangen} paa den Syd-vestlige Side, paa Gaarden \textit{Jndre-Valen}.\par
J \textit{Løngens}Fiord\textit{Peter Østeberg}, boesiddendes inde i \textit{Løngen} paa dens Vestre Side paa Gaarden \textit{Qvalvigen}, der tillige er En af de \textit{Norske} Besigtelses-Mænd, og følger Vidnerne til \textit{Kiølen}.\par
J \textit{UlfsFiordJohann Wormhuus}, boesiddendes \textit{Normand} inde i \textit{Ulfs}Fiord.\par
J \textit{BalsFiordNiels Olsen Kardin}, boesiddendes inde i \textit{Bals}Fiord.\par
\centerline{J \textit{Senniens} Fogderie}\par
For \textit{Malangens} og \textit{Reisens} Fiorder, \textit{Niels Nielsen, Norsk} SøeFinn, tilholdendes i \textit{Finn}- Fiordz Botten, strax Norden for \textit{Reisens}Fiord.\par
For dem i \textit{Grøtangs}Fiord i \textit{Astafiords} Præstegield, \textit{Ole Eriksen, Norsk} Søe-Finn, boendes i \textit{Grøtangs} Botten.\hspace{1em}\par
Til hvilke beskikkede Høvedzmænd, som nærmeste, de Hrer \textit{Commissions} Betiente, ved Ankomsten, kunde \textit{addressere} deres Bud. ‒\hspace{1em}\par
d. 7de Reiset Nord-efter til \textit{Nordstrømmen} i \textit{Reisen}, at arbeide paa min allerunderdanigste \textit{Relation} over min Reise og Forretning fra \textit{August} i afvigte 1744de Aar til \textit{dato}, og paa Bilagene til denne \textit{Protocoll}.\par
d. 16de Faret til \textit{Korvig} i \textit{Gisund;} hvor\par
d. 17de\textit{expederet Reqvisitions} Breve til \textit{Capellanen} paa \textit{Tranøe, Missionaire} i \textit{Asta}- Fiordene, \textit{Peder Krog Hind}, med sin Nærværelse i \textit{Asta}Fiordene og \textit{Reisens}Fiordz Vestre Side at \textit{assistere} den opnævnte Høvedsmænd (!), til at faae \textit{Missions} Skole-Mestern og de \textit{Norske} Besigtelses Mænd med Vidnerne og Arbeidere, om forlanges, i Vejen satt til Fieldz, ved de Kongelige Hrer \textit{Jngenieurers} Ankomst paa den \textit{Districts} Grændser.\par
‒ ‒ Ligesaa til Hr \textit{Johan Arnd Mejer}, som har \textit{Jnspection} over \textit{Missions}-Væsenet i \textit{Reisens}Fiord paa dens Østre Side, og i \textit{Malangen} paa dens Vestre Side, at giøre paa den Tid Høvedzmanden der samme \textit{assistance} og Tienneste.\par
d. 22de J \textit{RavnFiord} i \textit{BalsFiord} haft \textit{Jokasjarfs Lapper}, navnlig \textit{Hans Larsen Gunner} og \textit{Aamund Jonsen Lappe}, og fra \textit{EnontekisThomas Johansen}, Tolv-mand i \textit{Sverrig} hos mig samlede, som sagde: at det for de Kongelige \textit{Norske Officerer} ville blive meget lettere, og bedre, \hypertarget{Schn1_115495}{}Schnitlers Protokoller V. at komme frem langs med \textit{kiølen} om Vintern, end om Sommern, foreslaaendes den Tid af \textit{Mariæ} Bebudelse den 25de \textit{Martij}, da Vejret forlindres, og Dagene længes; Thi den Vej fra \textit{Stokkeborre} til \textit{Kapo} bestaaer af steenede og moesede Fielde, hvor intet Græss er. BrændeVeeden kunde de føre med sig i Kiøredzer paa dette Støkke, som er deres \textit{District}. Den Strekning Sønden-for, nemlig fra \textit{Torne}-Vatten til \textit{Stokkeborre} kunde andere \textit{Jokasjarfs Lapper}, som bekiendte ere, overskydse paa samme Tid om Vintern, og er der Skov at faae. ‒\par
Disse nærværende \textit{Lapper} ligge om Vintern ved \textit{Jokasjarfs} Markested, som er en 11ve Field-Mile Østen for \textit{Kiølen}, og ville gerne paa \textit{Kiølen} møde, naar de faae Bud derom. Fremdeeles forklare, at \textit{Jokasjarfs} Hoved \textit{Lappe}-Sogn har omtrent imod 200 Skatte- \textit{Lapper}. ‒\par
Om \textit{Sigvar- Eingvar}- og \textit{Raufuholt}-Field-Byer oppe imod \textit{Kiølen}, hvorom \textit{pag.} 382 her er meldt, vidste disse heller ikke noget af, at sige. Det er brugeligt i \textit{Norge}, at en \textit{Lappe}- Bye kaldes af \textit{Lappen}, som der sidder; naar den nu døer ud, eller en Anden sætter sig der, forandrer Byen sit Navn, efter den \textit{Lap}, som da har den i Besiddelse, og deraf maa det være, at forbem.te Navne ere nu udgaaet.\par
Dette blev berettet i \textit{Niels} og \textit{Ole Kardins} Overværelse.\par
d. 24de \textit{Julj} Havde hos mig i \textit{Løngens}Fiord af Vidner og Besigtelsses Mænd, som fra \textit{Løngens}Fiord vare gaaet den \textit{provisionelle} Grændse-Gang fra \textit{Gaabb} til \textit{Sierte-vara} GrændseMerker; Den Vej, de sagde, at være ved 10 Field-Mile lang; Videre de Mænd for dengang ei udstaaet at gaae, fordi deres Kost var fortæred, og Skoene forslidte.\par
Vidner, som op været, vare efter \textit{Protocollens Nomere}\par
10de \textit{Lars Spen}\par
12te \textit{Jakob Nysted}.\par
22de \textit{Hendrik Guttormsen} og\par
25de \textit{Niels Anudsen Kok}.\par
De af Betienterne beskikkede Besigtelses Mænd have med-været\par
1. forrige \textit{Missions} Skolemester fra \textit{QvænangenMons Jakobsen}\par
2. \textit{Ole Larsen Oxvig, Norsk} Søe-\textit{Finn} i \textit{Løngens}-Fiord.\par
3. \textit{Esaias Madsen, ibidem}, i steden for 11te Vidne \textit{Mikkel Qvæn Pælleg}, en gammel Mand.\par
4. \textit{Erik Ragte}, Huusmand \textit{ibidem}.\par
5. \textit{Christopher Pedersen, ibidem}.\par
Den Øvrige igienstaaende Vej fra \textit{Siertevara} til \textit{Pitse-kiolme}, som de for Retten have opgivet, sagde de, at skulle \textit{provisionaliter} befare og afmerke strax efter Mikkelsmiss, og samles den 3die \textit{Octob}. ved \textit{Sierte-vara}.
\DivII[Juli 25.-sept. 13. Avsluttende protokollarbeide i Straumfjord i Skjervøy]{Juli 25.-sept. 13. Avsluttende protokollarbeide i Straumfjord i Skjervøy}\label{Schn1_115748}\par
Den 25 \textit{Julj} skrevet til Provsten \textit{Junghans} paa \textit{Tromsøe}, at lade \textit{Missions} Skolemestern paa \textit{Malangens}Fiordz Østre Side, hvor \textit{Niels Nielsøn} SøeFinn i \textit{Finne}Fiord-Botten bliver Anførere, og i \textit{Bals}Fiord, hvor \textit{Niels Kardin} anfører, være disse Anførere behielpelig til at faae Vidnerne og Besigtelses Mænd med videre til rette Tid i Vejen satte. ‒\par
Derefter reiset til \textit{Strøms}Fiord, og arbeidet paa Bilagene og et \textit{Geographisk Carte} til dette \textit{Volumen}.\hypertarget{Schn1_115807}{}Paa Reisen i Tromsøen og Sennien.\par
\textit{Aug}. 8de Bragte \textit{Nordlands}-Posten mig mine Breve fra Hr Obriste \textit{Mangelsen}, og andre. Hvorpaa udferdigede min allerunderdanigste \textit{Relation} over min Reise og Forretning fra \textit{August}-Maanedz Udgang 1744. hvorhen-til min Forrige var gaaet, til\par
d. 20de [Aug.] 1745.\par
d. 21de Bragte Skolemestern \textit{Mons Jakobsen} fra \textit{Qvænangen} sin \textit{Relation} over den \textit{provisionelle} Befaring fra \textit{Gaabb} til \textit{Sierte-vara}, hvorom meldt paa næst-forrige Side, under 24de \textit{Julij}.\par
De \textit{Relationer} fra Befarings-Mændene fra \textit{Hierta} paa \textit{Duøder}-Field strax Norden for \textit{Torne}-Vandet i \textit{Senniens} Fogderie til \textit{Kalko-vadda} i \textit{Finmarken} fra \textit{No} 1. til 6. \textit{inclusive}, som jeg har indfaaet, sender jeg, de Hrer \textit{Commissions} Betientere til Efterretning, bag efter det \textit{Alphabetiske} Register vedhæftede: J fald BefaringsMændene, efter min Foranstaltning, ei skulle have Ligelydende op med sig til de gode Herrer. [Disse bilagene er ikke trykt. De finnes ikke i vol. V.] Til\par
\textit{Sept:} 13te faaet Bilagene med et ungefærlig \textit{Geographisk Carte} til dette 5te \textit{Volumen} af Grændse-\textit{Examinations Protocollen} ferdig; Hvorefter jeg begav mig Søer-ad til Sønderst i \textit{Senniens} Fogderie, for at see det til de H. \textit{Commissions} Betientere, som nu fra Field-\textit{kiølen} kunde ventes, at nedkomme, tilstillet.\hspace{1em}\par
\textit{StrømsFiord} d. 13 \textit{Septb}. 1745. \hspace{1em}\centerline{Peter Schnitler.}
\DivII[Merkverdigheter i Finnmark]{Merkverdigheter i Finnmark}\label{Schn1_115961}\par
\centerline{\textbf{Merkværdigheder i Finmarken}}\par
1. \textit{Vaardøehuus} Amt eller \textit{Finmarken} har 4re store og \textit{navigables} Fiorder\par
a. \textit{AltensFiord} i \textit{Vest-Finmarken}, 6 Mile lang i Ost-Syd-ost størstedeels, see \textit{pag.} 214.\par
b. \textit{Porsanger-Fiord}, som skiller \textit{Vest-Finmarken} fra \textit{Ost-Finmarken}, imod 8. Søe-Mile lang i Syd-Syd-vest. see \textit{pag.} 276.\par
c. \textit{Tana-Fiord} i \textit{Ost-Finmarken}, imod 5 Søe-Mile lang i Syd-Syd-vest. see \textit{pag.} 301.\par
d. \textit{VarangerFiord} sammestedz, over 7. Søe-Mile lang i Vester. see \textit{pag:} 350.\hspace{1em}\par
2. \textit{Finmarken} har 2de Hoved-Elve,\par
a. \textit{Altens} Elv, opkommer nær Norden for \textit{Field-Ryggen} eller \textit{Landskiølen}, en Deel Aaer rinde og deri fra \textit{Kiølen}; denne \textit{Altens} Elv løber mest i Nord ved 19 Field-Mile lang i \textit{AltensFiords} Botten. see \textit{pag.} 224.\par
b. \textit{Tana}-Elv oprinder fra \textit{Kiølen}, og løber omtrent i Nord-Nord-ost 24 3/4 Field-Mile lang i \textit{Tanafiords} Botten, see \textit{pag.} 304 f. Begge disse Elve ere Lax-riige. ‒\hspace{1em}\par
3. \textit{Finmarkens Præstegield} have, hver sin særdeles Herlighed og \textit{Producter:}\par
a. \textit{Loppens} ‒ det første og Vesterste Præstegield har i \textit{Brevig} under \textit{Sørøe} et almindelig Fiskevær af Torsk, fra Kyndelsmiss til Paaske, og Kobbefangst, see \textit{pag:} 250.\par
b. \textit{Altens} Præstegield har ovenfor \textit{Altens}Fiordz Botten god Furre-Skoug, som er den eeneste i \textit{privative Norsk Finmarken}, skiøn Laxe-Fiskerie i \textit{Altens} Elv foruden anden \textit{ordinaire} Fiskerie \textit{pag.} 215. 226.\hypertarget{Schn1_116172}{}Schnitlers Protokoller V.\par
c. \textit{Hammerfest}-Præstegield\textit{producerer} fornemmelig Sey-Fisk. \textit{pag.} 255.\par
d. \textit{Jngøens} Præstegield fremfører overflødig Lange, Torsk, og i visse Aarer Qveete; dog holdes Qveeten i \textit{Finmarken} ikke saa god, som den i \textit{Senniens} Fogderie i \textit{Nordland}; Thi i Vinter-Maanederne for Paaske bliver Qveeten i \textit{Finmarken} af Kulden hvid og tørr; Strax efter Paaske er den vel som best, men behøver tørt og klart Vejer, mens Jndvaanerne her i Landet have ikke den Leilighed til Skoug at bygge Skiur af, til at tyrke den derunder; Efter \textit{St Hans} Tid er Luften taaged, og for den Fugtighedz Skyld tørkes Qveeten ei saa lett, men, naar den bliver u-salted, suurner den, og antager Mark, og naar den saltes, fortærer Saltet en stor Deel af Fædmen.\par
e. \textit{Kielvig}-Præstegield har Torsk og Lange, noget Eeder-duun, og i \textit{Porsanger}Elv en Deel Lax. \textit{p.} 275.\par
f. \textit{KiølleFiords} Præstegield er begavet med Torsk, og med det riigeste Lax-Fiskerie i \textit{Tana}-Elv; Dets Halv-Øe \textit{Omgang} har skiøn VildReen-Skytterie, og \textit{Hops}-Eidet derved \textit{Helleborus} (Nys-Krudt) \textit{Angelica, Charlottes, Pettersille}, Reenfan og \textit{Cochleare} i saadan Mængde, at brave Folk af \textit{condition} sylter den ned i Tønde-Tall, og bruger den, som Kaal, Vinteren over, \textit{item} der og fleere Steder Zellerie, purløg i Overflødighed.\par
g. \textit{Vaardøe} Præstegield berømmes for Rotskiær, har ellers og Torsk og Sej \textit{pag:} 349.\hspace{1em}\par
4. \textbf{Sild} gives og i Fiordene og imellem Øerne, see \textit{pag.} 215. 275. Saaledes er overflødig Sild i \textit{Varangers}Fiord, og fange Russisk-fælles \textit{Passvig-Finner} den i Mangfoldighed i \textit{Passvig}- Fiord, hvortil \textit{Russerne} fra \textit{Cola} bringe dennem Salt og Træ, og hente derfra 4. til 7. smaa Jægter med Sild ladde. \textit{pag.} 349. Men Norske Undersaattere bruge ikke Sild-fangsten her i Landet, deels fordi de have ei Garn og Reedskab, og ei ere vandte dertil, deels vel og at Skoug paa fleeste Steder fattes til at giøre Tønde-Træer af. \textit{ibidem}. Og anden stor Fisk-Fangst nok her haves. ‒\hspace{1em}\par
5. \textbf{Lodden} er en besynderlig Slags Fisk, saa lang, som en Sild, men smalere, spidzagtig fremmentil og i Stærten, deels rund, deels fiirkanted, hvid-blank, som Sild, slet uden Resp, giver en sterk vederstyggelig Stank fra sig, for Folk u-ædelig; Denne Lodde er af u-troelig Mængde i Havet, og jages af den store Fisk som er begierlig efter den, paa Land saa tyk som Sand; Der kan vare 20 Aar meer og mindre imellem, at Lodden indfinder sig Nord i \textit{Nordlandene}, men tidere i \textit{Vest-Finmarken}, og som oftest, saasom hvert Eet, eller 2det 3die eller 4de Aar i \textit{Ost-Finmarken}; Hvor nu denne \textit{Lodde} kommer under Land, trækker den Torsk og Qveete, derefter Sejen og Hvalen efter sig i utallig Mængde, da der vanker et overflødigt riigt Fiskerie af stor Fisk for Jndbyggerne, medens Lodden opholder sig paa det Sted; Det er \textit{in Martio} gemeenlig, at den ankommer, og bliver der saa staaendes en 4. Ugers Tid, hvorefter den begiver sig tilbage i Havet, og drager den store Fisk fra Landet efter sig, saa at den Slags Fiskerie for det Aar ophører: dog sender Gud siden anden Slags SommerFisk ind til Landet, til Folkets Underhold; Aarsagen til denne Loddes Tilbagevendelse ad Havet, angive Jndbyggerne, at være, at naar Elvene og Bækkene begynde at give fersk Vand ud i Fiordene, skal Lodden ei fordrage det, men søger da ud til Havs,\hypertarget{Schn1_116354}{}Merkværdigheder i Finmarken.\par
6. J \textit{Finmarken} saaes ikke, uden ved \textit{Altens} Elv, hvor en Deel \textit{Qvæner}, det er: \textit{Svenske} Bønder fra \textit{StorFinland} og \textit{Torne}-Land i forrige Krigers Tid have nedsatt sig, og dyrket Jorden; Ved \textit{Tana}Elv have og vel samme Slags overkomne Folk begyndt, at saae Korn: men det har endnu ej lykkets, at være bleven moedt. see \textit{pag.} 310. Undersaatterne holde og kun faa Creaturer, som Køer, Faar: dog siunes det, at være til deels Jndvaanernes eegen Skyld, som ikke ere de beste Jord-Dyrkere; Man spinder og væver ikke her i Landet, og er Fattelse paa fornødne \textit{Haandverks}Folk. see \textit{pag.} 282. ‒\hspace{1em}\par
7. \textbf{Meitsk} er et temmelig høyt Field, slett og bart ovenpaa med en flad Tinde paa dets Vestre Ende, nedhældendes i Søer ned ad \textit{Vester-Botten} af \textit{Varangers}Fiord i \textit{Vaardøe} Præstegield; merkeligt deraf, at \textit{Finnerne} deraf have giort sig i gamle Dage en Afgud, hvortil de have offret. see \textit{pag.} 340. ‒\hspace{1em}\par
8. En \textbf{Sundheds Kilde} er i \textit{Varanger}Fiordz\textit{Østre Botten}, opvældendes af et Steen-Berg med Torv-Jord ovenpaa, som om Vinteren opspringer 1/2 a 3/4 Alen høy, men om Sommeren er lavere; dets Vand om Sommeren er kaldt, om Vinteren halv-lunken, meget sundt at drikke.\hspace{1em}\par
9. En Furre-\textit{Bielke} paa \textit{Vaardehuus} Fæstning bevares, 4 3/4 Alen lang, fiirkanted, 1/4 Alen i hver Kant breed, som i gamle Tider har ligget i en \textit{Norsk} Fiskers Stue-Gamme paa \textit{Vaardøe;} hvor salig Kong \textit{Christian} den IVde høyloflig Jhukommelse, paa sin Nord-Reise, var kommet ind, da Konen laae i Barsel-Seng, og det spæde Barn i Vuggen; Allerhøystberømte Konge har da stukket en Deel \textit{Ducater} til Barnet i dets Bindel-Tøyg, og selv skrevet med Krid paa Bielken i Gammen, som siden med Kniv er bleven indskaaren: \centerline{\textbf{Anno 1599. 29 Mai, erat hac in domo Christianus 4. RexDaniæ et Norvegiæ}}\hspace{1em}\par
10. Det Sydostlige Næss af \textit{Varangers}Fiord giøre \textit{Hen-øerne;} Paa den mindre Øe af disse \textit{Hen-øer} har fra u-mindelige Aar staaet, og endnu staaer 3de Steen-Støtter, af flade Steene \textit{en Pyramides} opsatte, som end denne Dag af \textit{Norske} og \textit{Russer} kaldes de \textit{Tree Konger:} Den Vestre deraf skal bemerke Kongen af \textit{Norge;} den Østre \textit{StorFørsten} (nu Kaiser) af \textit{Rusland}, og den Søndre og mindste Kongen af \textit{Sverrig;} Hvilke 3de Støtter holdes af \textit{Norske} og \textit{Russiske} Nærboende for et Grændse-Merke mellem de 3de Riger. ‒6 \textit{vol:}pag. [421].\hspace{1em}\par
11. Det Nord-vestlige Næss af \textit{Varanger}Fiord giør \textit{Kiberg} paa \textit{Varanger-Næss}, hvorudenfor \textit{Russerne} fra \textit{Archangel} for 3de Aar siden have begyndt, at benytte sig af Fiskerie i Havet, 1 ‒ 2 ‒ 3 ‒ à 4 Mile fra Landet, efter Torsk, Lange og særdeles Qveete; Det 1te Aar ere de \textit{Russer} kommet med en 40 ‒ det 2det med en 150 ‒ og det 3die sidstforløben Aar med en 30 Baader, 4 Mand paa hver Baad; de roe ud om Dagen Østen og Norden for det \textit{Norske Varanger-Næss}, og naar de søge Land at sove eller koge, legge de sig ind hos \textit{Norske} Undersaattere paa \textit{Vaardøe}. 6.\textit{ Vol:}\textit{pag.} [416].\hypertarget{Schn1_116600}{}Schnitlers Protokoller V.\par
12. Ved denne Leilighed af \textit{Russernes} Fiskerie vill og meldes noget om \textit{Russernes} Seiladz og Brug paa \textit{Bæren-Eiland} og \textit{Spidtsbergen;}\par
\textit{Bæren-Eiland} bestaaer af mange smaa og lave Holmer, dog Een deraf er noget høy og stor, hvoraf Eilandene kiendes, de skal ikke være videre i Begreb, end at de næsten i Een lang Dag skal kunde omroes; de ere uden Skoug, saa at Jngen der kan overvintre; Hvorfore \textit{Russerne} naar de tage Feil af \textit{coursen}, eller den rette Vind ei staaer bi, vende de til \textit{Finmarken} eller Nord til Nordlandene tilbage, og blive der Vinteren over liggende.\par
Fra \textit{Cola} eller \textit{Archangel} fare de først om Vaaren til \textit{NordCapen} af \textit{Finmarken}, og bie enten der, eller under \textit{Sørøen} af \textit{Loppens} Præstegield, indtil de see at have en stadig Syd-Syd- ostlig Vind, hvormed de da give sig J Vejen.\par
Deres Fartøyer ere Skuder, omtrent med 10 Mand bemandede.\par
To Samdynger bringe de til paa Reisen fra \textit{NordCapen} eller \textit{Sørøen} til \textit{Bæren-Eiland}, som ligger fra \textit{NordCapen} i Nord-Nord-vest; Og forblive der paa \textit{Bæren-Eiland} en 6. à 8 Ugers Tid; hvor de faae \textit{Rossmaal}, eller \textit{Vall-ross}, eller Hav-Heste, som ere 2 à 3 Favner lange, saa tykke, og tykkere, som en Hest; Deraf føre de Tænderne, og Spækket hiem til \textit{Archangel}, som brændes til Tran; Skindet deraf er saa tykt, som en Tommelfinger, og seer man det bruges i \textit{Nordland} til Hammel-Baand i Baader;\par
Fra \textit{Bæren-Eiland} fares i Nord i 1 Samdynge til \textit{Spidtsberg-Huken;} Dette\par
\textit{Spidtsberg} har hvide Biørne, vilde Reen, sorte Ræve, et andet Slags hvide-blaa, kalded Mæl-Rakker, Fugl i Mangfoldighed, Eederduun, Vall-Ross og Kobbe, eller Sæl-hunde.\par
\textit{Russerne} bruge den Østlige og \textit{Hollænderne} den Vestlige Deel af \textit{Spidtsberg}, saa de ei komme til hinanden.\par
Naar \textit{Russerne} fare til \textit{Spidtsberg}, blive de der Vinteren over, og føre til den Ende Brændehved og Husse af Timmer med sig; de have og paa \textit{Spidtsberg} staaendes 4 Huusse, 1 à 2 Mile fra hinanden, af \textit{Russerne} opsatte.\par
Paa \textit{Spitzberg} er det 15 Uger, At de ei see Dag\par
Saadan en Russisk Skude, som bliver der Vinteren over, kan bekomme til at føre hiem med sig, ved\par
100 hvide Biørner, meer og mindre, som ere der i Landet meget tamme, og, som ingen Folk der boe, ei Folke-sky, og meget feede, at de have 2 à 3 Fingre-Tyk Spæk paa sig, som brændes til Tran.\par
100. VildReen, meer og mindre, hvilke af forbem.te Aarsage ei skye Mennisker. ‒\par
100 til 200 Rossmaal eller VallRosser.\par
100 til 150 Vaager Eederduun, u-renset.\par
En Haaben Kobber eller Sælhunde\par
En Haaben sorte Ræver og MælRakker.\par
En Russisk Skude med durchstaaendes god Vind er fra \textit{Archangel} til \textit{Cola} paa Reisen {2 Samdynge} fra \textit{Cola} til NordCapen{1 ‒} fra \textit{NordCapen} til \textit{Bæren-Eiland} ved {2 ‒} fra \textit{Bæren-Eiland} til \textit{Spidtzbergen}{1 ‒ ___________}\hypertarget{Schn1_116826}{}Merkværdigheder i Finmarken.\par
Dette berettede mig en Russisk Skipper fra \textit{Archangel}, som mange gange havde faret paa \textit{Bæren-Eiland} og \textit{Spidtsbergen}, i \textit{Talvigen} i \textit{Altens} Præstegield i \textit{Vest-Finmarken}, som med sin Skude \textit{Ao}1743. skulle have gaaet til \textit{Bæren-Eiland}, men for ModVind skyld maattet overvintre i bemeldte \textit{Talvigen}, da jeg var der \textit{in Martio Ao}1744.
\DivII[Summarisk forklaring over Finnmark]{Summarisk forklaring over Finnmark}\label{Schn1_116875}\par
\centerline{Følger nu efter vedtagen Skik \textbf{Summarisk Forklaring over Finmarken,}}\par
at reigne fra Vester i Øster hen imod \textit{Jndiager}-Land, hvor \textit{Rusland} tager Deel i Skatten; Saa vidt hen, og ikke videre Grændse-Gangens Undersøgning skee maa.\par
\textit{Vaardehuus} Amt, ellers kalded \textit{Finmarken}, Vesten-fra at reigne, støder an, og grændser i \textit{Søer} først til \textit{Nordlands} Amtes\textit{Tromsøens} Fogderie, dernæst til \textit{Sverrigs}\textit{Torne}- og \textit{Kimi- Lapmark}, og omsider til Russisk \textit{Lapmark}.\par
Grændserne imod \textit{NorgesNordland} giøre, Vesten-fra at reigne:\par
\textit{Brynnel} ‒ en Holm,\par
\textit{Sokammer}, en Berg-Houg\par
Mittvejs imellem \textit{Rikasjaure} og \textit{Reisens-jaure}, til Fieldz\par
Mittvejs imellem \textit{Teno-motkie}, og \textit{Korsevara}, i Grændse-\textit{Limiten}.\par
Hvilke Steder i det 4de \textit{Examinations Volumen}II 411 f. vidtløftigere ere forklarede.\par
Grændse-Merkerne imod \textit{Sverrigs}\textit{Torne}- og \textit{Kimi-Lapmark} giøre, naar \textit{Finmarken} skal reignes at tage sin Begyndelse, Vesten-fra, paa Landz\textit{kiølen}, Mittvejs imellem \textit{Teno-motkie}, hvorfra Aaen \textit{Nallauksjok} stikker i Nord i den \textit{Nordlandske}\textit{Reisens} Elv, og det Field \textit{Korsevara}, hvorfra Aaen \textit{Vutskieka} løber i Øster i den \textit{Finmarkske Rikasjok} og omsider i \textit{Altens} Elv,\par
Halvvejs fra det Grændse-Merke \textit{TenoMotkie}, til det 1te \textit{Finmarkens} Grændse-Field \textit{Korsevara}, skal være {3/8 Field Mile}\textit{Korsevara} er stort {1/4 ‒} Derfra i Øster til næste GrændseField er {1/4 ‒} samme GrændseField heder \textit{Aakie-vara}, og er i Øster stort {1/4 ‒} Herfra i Øster {1 ‒} er \textit{Koskat-mutkie}, en Dal, viid {1/8 ‒} Saa møder Sapasmaras, en slet Mark, derfra i Øster er {1/2 ‒} til \textit{Nerrevarda}, et Grændsefield, bredt {1/8 ‒} Herfra {1/4 ‒} er \textit{Supsa-vara}, en Bakke, herfra {1/4 ‒} er \textit{Posa-vara}, et Grændse-Field, bredt i Ø. {1/8 ‒} Fra \textit{Posavara} i Øster {1/4 ‒} er \textit{Urtevara}, et Field, stort i Syd-ost {1/4 ‒} Derefter i Øster er en \textit{Bakke} imellem \textit{Masel}-Vand, og en \textit{Svensk} Bæk, der giør Skiellet. Fra \textit{Urtevara} i Øster {1/4 ‒} er \textit{Skier-oive}, eller \textit{Skiervoive}, herfra {1/8 ‒}\hypertarget{Schn1_117165}{}Schnitlers Protokoller V.\par
er \textit{Pitsekiolme}, et lidet fladt Field, Herfra i Øster 1/2 Miil, andre sige {1 F: Mile} er \textit{Kildevadda}, et Grændse-Field, langt i Ø {1 ‒} Derpaa følger \textit{Salvasvadda}, et slet moeset Land, langt {1 ‒}\textit{Kierres-vara}, et Grændse-Field, stort imod {1/4 ‒}\textit{Seuris}- eller \textit{Keuris-vara} ligger derfra {3/4 ‒} og er fra Sydvest i Nord-ost langt imod {1 ‒} Herfra i Øster {1/2 ‒}\textit{Tirmesvara}, et lidet Grændse-field, herfra i Øster {1/2 ‒} er \textit{Bevresmutkie}, eller \textit{Bajas-mutkie}, et kort Eid, derfra i Øster {1/2 ‒}\textit{Rov-oive}, stort {1/4 ‒} Nær Østen herfor er \textit{Mader-oive} herfra {1 ‒} er \textit{Modtatas-oive}, fra Vester i Øster langt {1/2 ‒} Herfra i Øster {1/4 ‒} er \textit{Kalko-vadda}, en Slette, stor {1/4 ‒} Strax Østen herfor er \textit{Pares-oive}, eller \textit{Parse-oive} ‒ derfra i Øster {1/6 ‒}\textit{Borvoive}, fra Vester i Øster langt {3/4 ‒} Herpaa i Øster følger \textit{Gaikkem}, et Grændse-Field; Hvorfra i Øster {3/4 ‒} er \textit{Raudo-oive}, et kort Field; Herfra i Øster {1/6 ‒} er \textit{Seide-kierro}, fra Vester i Øster langt {1 ‒}\textit{Maselg-aukie}, en Dal, er strax Østen derfor; og derpaa følger \textit{Maselg-oive}, fra V. i Ø. langt {3/4 ‒ ___________ = 17. F: Mile}\hspace{1em}\par
For \textit{specificerede} Grændse-Merker kaldes i \textit{Ost-Finmarken} med et allmindeligt Navn, \textit{Jauris-duøder}, som bemerker saa meget, som Fielde, hvorfra Elve rinde til begge Sider baade til \textit{Norge} og \textit{Sverrig}. ‒ See meere herefter \textit{p:} [397 f.].\par
Østen for \textit{Jaurisduøder} er\par
\textit{Beldo-vadda}, eller \textit{Laddegein}, en slet Mark uden Skoug; Ved dens Vestre Ende ligger det Vand \textit{Skiekkem}, eller \textit{Skietzem-jaure}, hvorfra Aaen rinder i Nord i den \textit{Norske}\textit{Tana}-Elv; Og denne \textit{Skietzem}-Aae er den sidste og Østerste, som løber ad \textit{Tana}Elv;\par
Dette \textit{Beldo-vadda} giør Enden paa den \textit{fameuse Landskiøl}; Thi Østen derfor er en u-endelig Furre-Skoug, rekkendes hen til \textit{Rusland}.\par
\textit{Grendse}-Merkerne imod \textit{Russisk Lapmark}, saavidt som denne \textit{Commissions}-Forretning kan vedkomme, findes paa følgende \textit{pag:} [415 ff.].\par
Dette er nu Grændsen af \textit{Vaardehuus} Amt i Søer; Paa Vestre, og Nordre Sider, \textit{item} for en Deel af dets Østre Side, nemlig Øster for \textit{Varanger-Næss} og Fiord, omskylles det af \textit{Nord}Søen; Det Jndre faste Land Sønden for \textit{Varanger}Fiord er i Syd-ost landfast med Russisk \textit{Lapmark}. ‒\hypertarget{Schn1_117444}{}Summarisk Forklaring over Finmarken.\par
\textit{Længden} af \textit{Vaardehuus} Amt, eller \textit{Finmarken}, ved Hav-Siden imod Nord-Søen, er Vesten-fra at reigne fra \textit{Andsnæss} i Ost-Nord-ost til \textit{Nord-Capen}{13 Søe Mile.} fra \textit{NordCapen} i Ost-Syd-ost til \textit{Vaardøe}{15 ‒ ___________ store 28 Søe Mile}\par
At merke, at Een \textit{Finmarkisk} Søe Miil holdes for at være saa lang, som næsten 2de Field- eller Danske Mile; Og saaledes ere de af Field-\textit{Finnerne} angivne Fieldmile herefter at forstaae, at næsten 2de heraf udgiøre 1 \textit{Finmarkisk} Søe Miil.\par
\textit{Længden} af \textit{Finmarken} til Fieldz ved Grændsen ad \textit{Sverrig}, efter Landz\textit{kiølen}, fra \textit{Korsevara} omtrent at reigne, til \textit{Beldo-vadda}, Enden af \textit{Kiølen}, er omtrent i Øster 17 FieldMile, hvilket paa næstforrige \textit{paginis} nøyere ere forklaret, og paa følgende Grændse-\textit{tabell}\textit{p:} [402 ff.] meere udført.\par
\textit{Bredden} af \textit{Vaardehuus} Amt fra Nord i Søer J \textit{Vest-Finmarken} fra \textit{Altens}Fiordz Botten i Søer til det \textit{Norske Masi Capell}, saavidt til, det \textit{Privative Norske Finmarken} rekker og derover {8 Field Mile.} fra \textit{Masi Capell} til den \textit{Svenske}\textit{Koutokeino-Finne}-Kirke i fælles \textit{Finmarken}{8 ‒} Herfra til \textit{Pitsekiolme} er imod 5 Mile, men lige i Søer til \textit{Seurisvara} kan blive {3 ‒ ___________ 19 F. M.}\par
J \textit{Ost-Finmarken} fra \textit{TanaFiords} Botten til \textit{FossHolmen}, hvorhen-til de \textit{Norske} vill reigne \textit{privative Norsk} Grund i Syd-Syd-vest ‒ (som ere SøeMile) {4 3/4 ‒} herfra til \textit{Otzjok}, hvor \textit{Arisbye} Svenske Kirke staaer ved, dens Mund {2 ‒} derfra til \textit{Valljoks} Mund i Syd-Syd-Vest {4 ‒} ‒ til \textit{Karokjoks} Mund {2 ‒} ‒ til \textit{Jndiagers} Sommer-Bye{3 ‒} Siden til \textit{Kiølen} 3 Dagers Reise, som \textit{taxeres} for {9 ‒ ___________ giør i S. S. V. 24 3/4 ‒} Men fra \textit{Tana}-Botten ligefrem til \textit{Beldo-vadda}, kan det regnes for 19 a {20. F. M.} Fra \textit{Varanger}-Botten i Syd-Syd-vest til \textit{Beldo-vadda} igiennem \textit{Jndiager}Land kan det blive ved {18. F. Mile.} Uregnet Øerne Norden for det faste Land.\hspace{1em}\par
\textit{Vaardehuus} Amt, eller \textit{Finmarken} deeles,\par
1. J Henseende til sin Leje (a) i \textit{Vest-Finmarken}, og \textit{Ost-Finmarken}, hvorimellem \textit{Porsanger}Fiord giør Skiellet. (b) i \textit{Øer}, og det \textit{Faste} Land; hvilke i \textit{Protocollen}, hver i sær, ere beskrevne.\par
2. J Henseende til dets Jndvaanere, at det beboes (a) af \textit{Normænd}\hypertarget{Schn1_117735}{}Schnitlers Protokoller V.\par
(b) af \textit{Norske} Søe-\textit{Finner} (c) af \textit{privative Norske} Field\textit{Finner} (d) af Fælles Field\textit{Finner}\par
Hvortil kan lægges (e) \textit{Qvæner}, hvilke dog beqvemmeligst til \textit{Norske} Søe-\textit{Finner} kan reignes.\par
Liste herover findes \textit{pag:} [400] herefter ‒\par
3. J Henseende til dets høye \textit{Jurisdictioner:} (a) i \textit{privative Norsk Finmarken} (b) i Fælles \textit{Finmarken} (c) i det \textit{Finmarken}, som af \textit{Rusland} eene indehaves, og af \textit{Norges} Crone \textit{prætenderes} paa. ‒\par
Til (a) det \textit{privative Norske Finmarken} høre (1) alle beboede Øer, (2) det Nordreste Støkke af \textit{Finmarkens} faste Land, hvis Længde før \textit{pag:} 397 er regned at være fra \textit{Andsnæs} forbj \textit{NordCapen} til \textit{Svartnæss}, imod \textit{Vaardøen}, fra Vester omtrent i Øster = 28 Søe Mile.\par
Dets Bredde fra Nord i Søer imod det fælles \textit{Finmarken} regnes paa følgende Maade:\par
J \textit{Vest-Finmarken} fra \textit{Altens}botten i Søer til det \textit{Norske Masi-Capell inclusive}, see \textit{pag.} 230 f. her er 8 FieldMile endnu 1 Mil S. for \textit{Kalbjok}{4 ‒}\par
Derfra i Øster til Nord til den Aae \textit{Jetzjok,}\textit{p:} 266. 313.\par
fra \textit{Refsbotten} ved Søe-kanten i Søer til \textit{Jetzjok}, tæt Østen for \textit{Siosjaure}, er ved 7 FieldMile\par
Fra bem.te \textit{Siosjaure} i Øster til Søer til det \textit{Svenske} Markested \textit{Avjevara}, paa Søndre Side af \textit{Jetzjok;}\textit{pag.} 230. 266. 312 f.\par
Fra \textit{Porsangers}Fiordz Botten i Søer til imod \textit{Avjevara}, er mod 5 Field Mile\par
Fra \textit{Avjevara} (at forstaae Norden derfor) gaaer \textit{Norsk} Grund i Øster hen til \textit{Karasjok}, en Elv, paa dens Nordre og Syd-vestlige Sider, see \textit{pag.} 230. 266. 312 f. 314.\par
Fra \textit{Laxe-Fiord}-Botten i \textit{Ost-Finmarken} i Søer til Vesten til \textit{Karasjok}, forrige \textit{Svensk} Markested ved \textit{Karasjok}-Elv, er imod 10 FieldMile\par
Fra \textit{Karasjok}-Bye i Nord-ost til imod \textit{Juxbye, Svensk} Markested ved \textit{Tana}-Elv, see \textit{pag:} 292 f. og 314. ‒\par
Fra samme \textit{LaxeFiord}-Botten i Søer til dette \textit{Juxbye} er {5 FieldMile.}\par
Fra \textit{Juxbye} i Nord-ost gaaer \textit{Norsk} Grund til \textit{Fossholmen} i \textit{Tana}-Elv, see \textit{pag.} 313 f.\par
Fra \textit{Tana}Fiordz Botten i Syd-Syd-Vest er efter \textit{Tana}Elv{4 3/4 SøeMile,} som man vill holde for {6 FieldMile}\par
Fra \textit{Fossholmen} vender sig den \textit{Norske} Grund i Syd-ost til \textit{Sapesduøder}-Field dets Mitte, see \textit{pag.} 344.\par
Fra \textit{Tana}Fiord-Botten i Søer til \textit{Sapes-duøder} er ved {6 FieldMile}\par
Fra \textit{Sapesduøder} i Syd-Syd-ost til \textit{Galdoive}-Field dets Mitte, see \textit{pag.} 344.\par
Fra \textit{Tana}Fiordz Botten i Søer til \textit{Galdoive} er 7 FieldMile. Fra \textit{Galdoive} fremdeles i Syd-Syd-ost til \textit{Jako-vadda}, mitt derover, see \textit{pag.} 344. ‒\par
Fra \textit{Tana}Fiordz Botten i Søer til \textit{Jakovadda} er {7 F. Miile}\hypertarget{Schn1_118148}{}Summarisk Forklaring over Finmarken.\par
Fra \textit{Jakovadda} gaaer videre \textit{Norsk} Grund i Øster til \textit{Øvre-Bolma}-Vand over dets Nordre Ende. ‒ see \textit{pag.} 344.\par
Fra \textit{Tana}Fiord-Botten i Sør til dette \textit{Øvre-Bolma}-Vand er ligeledes {7 FieldMile.}\par
Her \textit{cesserer} det Svenske \textit{Koutok[e]ino} Præstegieldz fælles \textit{district} i Øster. ‒\hspace{1em}\par
(b) \textit{Fælles Finmarken} ligger Sønden for \textit{privative-Norsk Finmarken}, og deeles i Almindelighed i \textit{Søer-Fieldet}, og \textit{Nord-Fieldet:}\par
\textit{Søerfieldet} strekker sig i Søer op til \textit{Kiølen}, og dertil høre: de Finne-Byer\label{Schn1_118231} \par 
\begin{longtable}{P{0.40948905109489053\textwidth}P{0.4405109489051095\textwidth}}
 \hline\endfoot\hline\endlastfoot 1. \textit{Koutokeino}-Kirkebye\tabcellsep  Disse 5. Finne-Byer haver \textit{Norge} med \textit{Sverrig} alleene tilfælles i Skatten: Men \textit{Sverrig} øver \textit{jurisdictionen.}\\
2. \textit{Avjevara Finne}-Bye\\
3. \textit{Karasjok} ‒\\
4. \textit{Juxbye} ‒\\
5. \textit{Arisbye} Kirke-Bye\end{longtable} \par
 \par
6. \textit{Jndiager}-Kirke-Bye Denne Bye har \textit{Norge} tilfælles med \textit{Sverrig} og \textit{Rusland} i Skatterne: men \textit{Sverrig} øver \textit{jurisdictionen}.‒\par
Til \textit{Nordfieldet} høre de fælles Byer, som ligge paa Søndre Landz Side af \textit{Varanger}Fiord:\label{Schn1_118326} \par 
\begin{longtable}{P{0.2582278481012658\textwidth}P{0.5917721518987342\textwidth}}
 \hline\endfoot\hline\endlastfoot 7. \textit{Neiden}\tabcellsep Disse 3de \textit{Finne}-Byer har \textit{Norge} tilfælles med \textit{Rusland} alleene: Men \textit{Rusland} øver \textit{privative}-eene \textit{Jurisdictionen}. ‒\\
8. \textit{Passvig}-Kloster Kirkebye\\
9. \textit{Peisen}\end{longtable} \par
 \hspace{1em}\par
Endnu hører til \textit{Nordfieldet} den Deel af\par
(c) \textit{Finmarken}, som af \textit{Rusland} eene indehaves, og af \textit{Norge prætenderes} paa, saasom\label{Schn1_118398} \par 
\begin{longtable}{P{0.453468899521531\textwidth}P{0.3965311004784689\textwidth}}
 \hline\endfoot\hline\endlastfoot 10. \textit{Biarmini}\tabcellsep Som \textit{Rusland} sig tilholder, løbende i Syd-ost ad den \textit{Hvide Søe}, paa hvis Skatt aarlig til \textit{Malmis prætenderes}.\\
11. \textit{Trinis}\\
12. \textit{Søndergield}\tabcellsep \textit{Nordfields Finne}-Byer, hvor Skatten tilbage holdes.\\
13. \textit{Nøttjager}\\
14. \textit{Perisaur}\\
15. \textit{Lajaur}\\
16. \textit{Nergis}\\
17. \textit{Trinnis}\\
18. \textit{Lantrot}\\
19. \textit{Manami}\\
20. \textit{Malmis}\end{longtable} \par
 \par
Som nu \textit{N.} 6 \textit{Jndiager-Finne}-Bye er den sidste paa \textit{SøerFieldet}, liggendes Østen for \textit{N.} 1. \textit{Koutokeino- N.} 2. \textit{Avjevara-} og \textit{N.} 5. \textit{Arisbye} deres \textit{District}, og samme \textit{Jndiager} er den første, hvorj \textit{Rusland interesserer} i Skatten; Saa ere Raamerkerne og Skiellet imellem bem.te Vestre Byer \textit{Koutokeino, Avjevara} og \textit{Arisbye}, saa og den Østre Bye \textit{Jndiager}, følgende, at reigne fra Landz\textit{kiølen} i Søer til \textit{Suøkas} Field i Nord, efter \textit{pag:} 357. 360.\par
Ved \textit{Beldo-vadda}, som giør Ende paa \textit{Kiølen}, dets Vestre Ende paa Nordre Side er \textit{Skiekkem}, eller \textit{Skietzem}-Vand, deraf rinder \textit{Skiekkem}, eller \textit{Skietzem}Elv i Nord i \textit{Øvre}\textit{Tana}-Elv, som og kaldes \textit{Jndiagers}-Elv; Denne \textit{Øvre-Tana}-Elv, løber nu Nordlig Vesten forbj de \textit{Jndi\hypertarget{Schn1_118617}{}Schnitlers Protokoller V. agerers} Sommer-Bye hen til Munden af \textit{Garradas}-Aae; derfra gaaer \textit{Raa}-Merkene Østerlig langs med \textit{Garradas-jok}, tæt Sønden forbi \textit{RøeterField}, \textit{Suøkas}-Field til i den Nordre Deel af \textit{Saxejaure;} see \textit{pag.} 324. 357. 360.\par
Paafølgelig gaaer de Vestre \textit{Finne}Byers \textit{district} til den Vestre Side af \textit{Skiekkem}-Vand, og \textit{Skiekkem-jok}, samt \textit{Øvre-Tana}-Elv. Siden til den Nordre Side af \textit{Garradas-jok;} de indehave og \textit{RøeterField}, \textit{Suøkas}Field og den Nordre Deel af \textit{Saxe-jaure}.\par
Derimod de \textit{Jndiagerer} med deres Brug gaae til den Østre Side af \textit{Skiekkem}-Vand, \textit{Skiekkem}-Aae, og \textit{Øvre-Tana} Elv, siden til den Søndre Side af \textit{Garradas-jok}, og til mod \textit{Røeter}- Field, \textit{Suøkas}Field, og indehave den Søndre Deel af \textit{Saxe-jaure}. ‒\par
\textit{Jndiager-Neiden-Passvig-} og \textit{Peisen}, og de Øvrige ere Finne-Byer, hvorj \textit{Rusland} haver \textit{Jnteresse}, og derfor ei henhøre til dette \textit{Volumen}.\par
\textit{Tingsteder} i \textit{Vaardehuus Amts privative-Norsk Finmarken}\par
\textit{Hassvig}.\par
\textit{Loppen}\par
\textit{Alten}\par
\textit{Hammerfest}\par
\textit{Maasøe}\par
\textit{Kielvig}\par
\textit{KiølleFiord}\par
\textit{Omgang}\par
\textit{Vaardøe}\par
\textit{Vadsøe}.\hspace{1em}
\DivII[Ekstrakt av tabell over Finnmarks innbyggere]{Ekstrakt av tabell over Finnmarks innbyggere}\label{Schn1_118823}\par
\centerline{\textbf{Extract}\textit{af Finmarkens Jndvaanere 1. privative-Norske Undersaattere}.}\label{Schn1_118837} \par 
\begin{longtable}{P{0.29612903225806453\textwidth}P{0.2613978494623656\textwidth}P{0.04021505376344086\textwidth}P{0.0786021505376344\textwidth}P{0.060322580645161286\textwidth}P{0.056666666666666664\textwidth}P{0.056666666666666664\textwidth}}
 \hline\endfoot\hline\endlastfoot \Panel{\textit{Præstegield}}{label}{1}{l}\tabcellsep \Panel{\textit{Kirke-Sogner}}{label}{1}{l}\tabcellsep \Panel{\textit{Normænd}}{label}{1}{l}\tabcellsep \Panel{\textit{Norske Søe-Finner og Qvæner}}{label}{1}{l}\tabcellsep \Panel{\textit{Norske FieldFinner}}{label}{1}{l}\tabcellsep \Panel{\textit{Summa Familier}}{label}{1}{l}\tabcellsep \Panel{\textit{Har ungt Mandkiøn}}{label}{1}{l}\\
i \textit{Vest-Finmarken}\\
1. \textit{Loppens} Gield\tabcellsep \textit{Loppens} Hoved Sogn\tabcellsep 25\tabcellsep 5\tabcellsep 2\tabcellsep 32\tabcellsep 8\\
\textit{Hasvigs Annex}\tabcellsep 23\tabcellsep \tabcellsep \tabcellsep 23\tabcellsep 5\\
2. \textit{Altens} ‒\tabcellsep \textit{Altens} Hoved Sogn\tabcellsep 30\tabcellsep 80\tabcellsep \tabcellsep 110\tabcellsep 27\\
\textit{Masi-Finne Capell}\tabcellsep \tabcellsep \tabcellsep 14\tabcellsep 14\\
3. \textit{Hammerfest} ‒\tabcellsep \textit{Hammerfest} Hoved-Sogn, med \textit{Akkereide Finne-Capell}\tabcellsep 35\tabcellsep 55\tabcellsep 4\tabcellsep 94\tabcellsep 26\end{longtable} \par
 \hypertarget{Schn1_118968}{}Extract af Finmarkens Jndvaanere.\label{Schn1_118970} \par 
\begin{longtable}{P{0.21124260355029584\textwidth}P{0.47655325443786983\textwidth}P{0.05909763313609467\textwidth}P{0.028920118343195265\textwidth}P{0.012573964497041418\textwidth}P{0.03520710059171598\textwidth}P{0.02640532544378698\textwidth}}
 \hline\endfoot\hline\endlastfoot 4. \textit{Jngøens} ‒ \tabcellsep \textit{Jngøens} HovedSogn\tabcellsep 21\tabcellsep 8\tabcellsep \tabcellsep 29\tabcellsep 9\\
\tabcellsep \textit{Jelmesøes Annex}\tabcellsep 10\tabcellsep \tabcellsep \tabcellsep 10\tabcellsep 2\\
\tabcellsep \textit{Stappens-Annex}\tabcellsep 5\tabcellsep \tabcellsep \tabcellsep 5\tabcellsep 2\\
5. \textit{Kielvigens} ‒ \tabcellsep \textit{Kielvigs} Hoved Sogn\tabcellsep 32\tabcellsep \tabcellsep \tabcellsep 32\tabcellsep 36\\
\tabcellsep \textit{Maasøe-Annex}\tabcellsep 17\tabcellsep 8\tabcellsep 2\tabcellsep 27\\
\tabcellsep \textit{Kistrand Finne Capell}\tabcellsep \tabcellsep 34\tabcellsep 13\tabcellsep 47\\
J \textit{Ost-Finmarken:}\tabcellsep \tabcellsep \tabcellsep \tabcellsep \tabcellsep \tabcellsep \\
6. \textit{KiølleFiords} Gield\tabcellsep \textit{Kiøllefiords} Hoved-Sogn\tabcellsep 21\tabcellsep \tabcellsep \tabcellsep 21\tabcellsep 10\\
\tabcellsep \textit{Omgangs Annex}\tabcellsep 7\tabcellsep \tabcellsep \tabcellsep 7\tabcellsep 4\\
\tabcellsep \textit{Lebesbye Finne-Capell}\tabcellsep \tabcellsep 28\tabcellsep 4\tabcellsep 32\tabcellsep 16\\
\tabcellsep \textit{Tana-Finne-Capell}\tabcellsep \tabcellsep 19\tabcellsep 11\tabcellsep 30\tabcellsep 21\\
7. \textit{Vaardøe} ‒ \tabcellsep \textit{Vaardøe} Hoved Sogn\tabcellsep 14\tabcellsep \tabcellsep \tabcellsep \tabcellsep \\
\tabcellsep \textit{Kiberg-Annex}\tabcellsep 10\tabcellsep \tabcellsep \tabcellsep \tabcellsep \\
\tabcellsep \textit{Makur-Annex}\tabcellsep 8\tabcellsep \multicolumn{2}{l}{tilsammen}\tabcellsep 32\tabcellsep 10\\
\tabcellsep \textit{Vadsøe}-Sogn\tabcellsep 57\tabcellsep \tabcellsep \tabcellsep 57\tabcellsep 20\\
\tabcellsep \textit{Angsnæss FinneCapell}\tabcellsep \tabcellsep 130\tabcellsep 14\tabcellsep 144\tabcellsep 50\\
\tabcellsep \tabcellsep \multicolumn{5}{l}{_________________________}\\
\multicolumn{2}{l}{\textit{Summa privative-Norske} Undersaatter}\tabcellsep 315\tabcellsep 367\tabcellsep 64\tabcellsep 746\tabcellsep 246\end{longtable} \par
 \par
\centerline{2. \textit{Fælles Field Finner}.}\label{Schn1_119265} \par 
\begin{longtable}{P{0.20520965692503176\textwidth}P{0.1846886912325286\textwidth}P{0.18144853875476494\textwidth}P{0.044282083862770015\textwidth}P{0.2343710292249047\textwidth}}
 \hline\endfoot\hline\endlastfoot \Panel{\textit{Field-tract}}{label}{1}{l}\tabcellsep \Panel{\textit{Field-Byer}}{label}{1}{l}\tabcellsep \Panel{}{label}{1}{l}\tabcellsep \Panel{Fælles Field \textit{Finner}}{label}{1}{l}\tabcellsep \Panel{}{label}{1}{l}\\
\textit{Sørfieldet}\tabcellsep \textit{Koutokeino} Kirkebye\tabcellsep Som \textit{Norrig} har tilfælles med \textit{Sverrig}\tabcellsep 28\tabcellsep \\
\textit{Avjevara}\tabcellsep 10\tabcellsep \\
\textit{Karasjok}\tabcellsep 6\tabcellsep \\
\textit{Juxbye}\tabcellsep 2\tabcellsep \\
\textit{Arisbye} KirkeBye\tabcellsep 26\tabcellsep \\
\textit{Jndiager} Kirkebye\tabcellsep Som \textit{Norge} har tilfælles med \textit{Sverrig} og \textit{Rusland} efter Wedeges K[art]\tabcellsep 36\\
\tabcellsep \tabcellsep \tabcellsep _____\tabcellsep 108\hypertarget{Schn1_119374}{}\footnote{\label{Schn1_119374}[Tallene klusset med. Summen stemmer ikke.]}\\
\textit{Nordfieldet}\tabcellsep \textit{Neidens} Bye\tabcellsep Som \textit{Norge} har tilfælles med \textit{Rusland}. [efter Wedeges K[art]]\tabcellsep 8\tabcellsep \\
\textit{Passvig}\tabcellsep 24\tabcellsep \\
\textit{Peisen}\tabcellsep 12\\
\tabcellsep \tabcellsep \tabcellsep _____\tabcellsep 44\\
\tabcellsep \tabcellsep \tabcellsep \tabcellsep _____\\
\tabcellsep \tabcellsep \textit{Summa} fælles FieldFinner =\tabcellsep \tabcellsep 162\hypertarget{Schn1_119448}{}\footnote{\label{Schn1_119448}[Tallene klusset med. Summen stemmer ikke.]}\end{longtable} \par
 \hypertarget{Schn1_119452}{}Schnitlers Protokoller V.
\DivII[Tabell over bevidnede grensefjell]{Tabell over bevidnede grensefjell}\label{Schn1_119454}\par
\centerline{\textbf{Tabell} over de bevidnede Grændse-Field og Merker imellem}\par
\textit{Vaardehuus} Amt eller \textit{Finmarken} paa den ene ‒ og \textit{SverrigsTorne-Lapmark} samt \textit{Kimi-Lap mark} paa den anden Side, at reigne fra Vester i Øster,\par
\textit{extrahered} af \textit{ExaminationsProtocollens} 5te \textit{Volumen}\par
Begyndt fra det Grændse-Field \textit{Halde}, som er det 24de Grændse-Merke efter \textit{Tabellens} Orden i \textit{Nordlands Protocolls} det 4de \textit{Volumen}\par
tilstillet Provst \textit{Angell} i \textit{OstFinMarken}, og Fogden \textit{Wedege} d. 15 \textit{Mart:} 1745, og Hr Amtmand \textit{Kieldson} og Provst \textit{Falk} i \textit{VestFinMarken} d. 12 \textit{Ap:} derefter.\hspace{1em}\par
\textit{Grændsens-gang imellem en Deel af det Nordlandske Tromsøens Fogderie samt Finmarkens Begyndelse paa den eene ‒ og SverrigsTorne Lapmark} paa anden Side.\label{Schn1_119566} \par 
\begin{longtable}{P{0.08388157894736842\textwidth}P{0.11184210526315788\textwidth}P{0.11184210526315788\textwidth}P{0.1006578947368421\textwidth}P{0.11184210526315788\textwidth}P{0.11184210526315788\textwidth}P{0.10625\textwidth}P{0.11184210526315788\textwidth}}
 \hline\endfoot\hline\endlastfoot 1\tabcellsep 2\tabcellsep 3\tabcellsep 4\tabcellsep 5\tabcellsep 6\tabcellsep 7\tabcellsep 8\\
\textit{Halde}\tabcellsep \textit{Nappetiøve}\tabcellsep \textit{Samas-oive}\tabcellsep \textit{Vardoive}\tabcellsep \textit{Biertevara}\tabcellsep \textit{Tenomotkie}\tabcellsep \textit{Korsevara}\tabcellsep \textit{Aakie-vara}\end{longtable} \par
 \par
1. \textit{Halde}, forklared i \textit{Nordlands Protocolls} 4de \textit{Volumen}II 352 og bagen til i Grændse- \textit{Tabellen No} 24. ‒ over dets Mitte Grændse-\textit{Linien} vill gaae.\par
2. \textit{Nappetiøve}, en Dal med en Houg i, over hvilken Houg \textit{limiten} gaaer. See dens Beskrivelse i 4de \textit{Volum:}II 353 og i Grændse-Tabellen sammestedz \textit{No} 25.\par
3. \textit{Samas}- eller \textit{Saamas-oive}, et Field, Mitt derover; Beskreven i 4de \textit{Volum:}II 353 og i Grændse-\textit{Tabellen ibidem No} 26.\par
4. \textit{Vard-oive}, et Field, mitt derover; forklared i 4de \textit{Volum}. II 353 og i Grændse- \textit{Tabellen ibidem No} 27.\par
5. \textit{Bierte-vara}, i \textit{Nordland} udtalt: \textit{Siertevara}, et Field, mitt derover; Beskreven i 4de \textit{Volumen}II 353 f. og i Grændse-\textit{Tabellen ibidem No} 28.\par
6. \textit{Teno-motkie}, en Dal, hvori et lidet Eid imellem 2de Aaer; Mitt over Eidet er \textit{limiten}; See 4de \textit{Volumen}II 354 og i Grændse-\textit{Tabellen} sammestedz \textit{No} 29. ‒ Af denne Dal \textit{Teno-motkie} er det den Østerste og sidste Aae, \textit{Nallauks-jok}, som rinder Nordlig i den \textit{Nordlandske Reisens-jok}\par
7. \textit{Korsevara}, et Field, mitt derover; beskreven i 4de \textit{Volum:}II 354 og i Grændse- \textit{Tabellen} sammestedz \textit{No} 30. ‒ Fra dette \textit{Korse-vara} dets Nordre Side flyder Aaen \textit{Muretz-jok}, som den første, Østerlig i den \textit{Finmarkske}\textit{Altens}Elv; Hvoraf riimeligen siges kan, at mitt imellem \textit{Teno-motkie} og \textit{Korsevara} vill Raamerket og Skiellet blive imellem \textit{Nordlands}- og \textit{Vaardehuus} Amter, at forstaae i Grændse-\textit{Limiten}. ‒\par
8. \textit{Aakie-vara}, et Field, mitt derover, beskreven i 4de \textit{Volum:}II 354 og i Grændse- \textit{Tabellen} sammestedz \textit{No} 31. ‒\hypertarget{Schn1_119829}{}Tabell over Grændsens Gang.\label{Schn1_119831} \par 
\begin{longtable}{P{0.06343283582089551\textwidth}P{0.0608955223880597\textwidth}P{0.05582089552238806\textwidth}P{0.05582089552238806\textwidth}P{0.0608955223880597\textwidth}P{0.05835820895522388\textwidth}P{0.06597014925373133\textwidth}P{0.06343283582089551\textwidth}P{0.06343283582089551\textwidth}P{0.06343283582089551\textwidth}P{0.0608955223880597\textwidth}P{0.0608955223880597\textwidth}P{0.06597014925373133\textwidth}P{0.050746268656716415\textwidth}}
 \hline\endfoot\hline\endlastfoot 9\tabcellsep 10\tabcellsep 11\tabcellsep 12\tabcellsep 13\tabcellsep 14\tabcellsep 15\tabcellsep 16\tabcellsep 17\tabcellsep 18\tabcellsep 19\tabcellsep 20\tabcellsep 21\tabcellsep 22\\
{\textit{Koskat-mutkie}}\tabcellsep {\textit{Nerre-varda}}\tabcellsep {\textit{Posa-vara}}\tabcellsep {\textit{Urde-vara}}\tabcellsep {\textit{Masel-jaure}}\tabcellsep {\textit{Skier-oive}}\tabcellsep {\textit{Pitse-kiulbme}}\tabcellsep {\textit{Kielde-vadda}}\tabcellsep {\textit{Salvas-vadda}}\tabcellsep {\textit{Kierres-vara}}\tabcellsep {\textit{Keuris-vara}}\tabcellsep {\textit{Tirmes-vara}}\tabcellsep {\textit{Bevres-mutkie}}\tabcellsep {\textit{Rovoive}}\end{longtable} \par
 \par
9. \textit{Koskatmutkie}, ellers i \textit{Nordland} udtalt: \textit{Kaakasmutkie}, et Eid i en Dal imellem 2de Vand-fald; Mitt herover; See Forklaringen i 4de \textit{Volum:}II 354 og i Grændse-\textit{Tabellen} sammestedz \textit{No} 32. (om \textit{SapasMaras} vides ej her).\par
10. \textit{Nerre-varda}, et Field, mitt derover; Beskreven i 4de \textit{Volum:}II 355 og i Grændse- \textit{Tabellen ibidem No} 34. (om \textit{Supsavara} vides ej i \textit{FinMarken}.)\par
11. \textit{Posavara}, et Field, over dets Nordre Ende; Beskreven i 4de \textit{Volum:}II 355 og i Grændse-\textit{Tabellen No} 36.\par
12. \textit{Urdevara}, et Field, over dets Søndre Ende; Beskreven i 4de \textit{Volumen}II 355 og i Grændse-\textit{Tabellen} sammestedz \textit{No} 37.\par
13. \textit{Masel-jaure}, et Vand imellem næstforrige \textit{Urdevara}, og næstfølgende \textit{Skier-oive}, tæt Sønden derfor over en Bakke; Beskreven i 4de \textit{Volum:}II 355 og i Grændse-\textit{tabellen}\textit{No} 38.\par
14. \textit{Skier-oive}, i \textit{Nordland} udtalt: \textit{Skierv-oive}, et Field, mitt derover; Beskreven i 4de \textit{Volum}. II 356 og i Grændse-\textit{Tabellen No 39}.\par
15. \textit{Pitsekiulbme}, i \textit{Nordland} udtalt: \textit{Pitsekiolme;} forklared i 4de \textit{Volum:}II 356 ‒ og i dette 5te \textit{Volum:}\textit{pag.} 232 f. samt i 4de \textit{Volumens} Grændse\textit{tabell No} 40. ‒\par
16. \textit{Kielde-vadda}, et Field med et Vand, \textit{Kirra-jaure}, paa; Mitt over Fieldet og Vandet gaaer Grændse-Skiellet; See 4de \textit{Volum:}II 383 og Grændse-\textit{Tabellen} sammestedz \textit{No} 41. ‒\par
17. \textit{Salvas-vadda}, et slet moeset Land, mitt imellem 2de Vande, derover gaaer \textit{Limiten}. See 4de \textit{Volum:}II 383 og Grændse-\textit{Tabellen No} 42. ‒\par
18. \textit{Kierresvara}, et Field, hvorover imellem 2de Vande \textit{Limiten} gaaer. See 4de \textit{Volum:}II 383\textit{et} Grændse-\textit{Tabellen} sammestedz \textit{No} 43. ‒\par
19. \textit{Keurisvara}, i \textit{Nordland} udtalt \textit{Seuris-vara}, mitt derover, hvorfra Bække nedfalde til begge Sider, gaaer Grændse-\textit{Linien;} See 4de \textit{Volumen}. II 383 og Grændse\textit{tabellen No} 44.\par
20. \textit{Tirmes-vara}, et Field mitt derover. See 4de \textit{Volumen}II 383 og Grændse-\textit{Tabellen No} 45. ‒\par
21. \textit{Bevres-} eller, som det og her kaldes: \textit{Bajas-mutkie}, imellem 2de Vande, navnlig \textit{Bajas} hvoraf Aaen rinder til \textit{Norge}, og \textit{Bevres-jaure}, hvoraf Aaen stikker til \textit{Sverrig}, et Eid, mitt derover er Grændse-gangen; See 4de \textit{Vol:}II 384 og Grændse-\textit{tabellen No} 46. ‒\par
22. \textit{Rov-oive}, et Field, 1/2 Miil Østen for \textit{Bevres-mutkie} liggendes, rundagtigt, spidzt oventil, ikke høyt, 1/4 Miil over stort, bart uden Græss og Skoug, moeset; Fra dette \textit{Rovoive} paa den Nord-Vestlige Side rinder en Bæk \textit{Akenæss-jok} til \textit{Norge} i \textit{Karas-jok}-Elv; Fra den anden Søndre Side af dette \textit{Rovoive} gaaer en Bæk, uden Navn, i Søer ad \textit{Sverrig} til \textit{Børesjaure}. Mitt over dette \textit{Rov-oive} meenes Grændse-\textit{linien} at skulle gaae. 5 \textit{Volumen}\textit{pag.} 233. ‒ 1 Bøsse-Skud Østen for \textit{Rovoive} ligger\hypertarget{Schn1_120317}{}Schnitlers Protokoller V.\label{Schn1_120319} \par 
\begin{longtable}{P{0.11805555555555555\textwidth}P{0.13576388888888888\textwidth}P{0.12395833333333334\textwidth}P{0.12395833333333334\textwidth}P{0.11805555555555555\textwidth}P{0.10625\textwidth}P{0.12395833333333334\textwidth}}
 \hline\endfoot\hline\endlastfoot 23\tabcellsep 24\tabcellsep 25\tabcellsep 26\tabcellsep 27\tabcellsep 28\tabcellsep 29\\
\textit{Maderoive}\tabcellsep \textit{Modtatasoive}\tabcellsep \textit{Kalkovadda}\tabcellsep \textit{Parse-oive}\tabcellsep \textit{Borv-oive}\tabcellsep \textit{Gaikkem}\tabcellsep \textit{Roudo-oive}\end{longtable} \par
 \par
23. \textit{Maderoive}, rundt, spidzagtigt oven til, 1 Bøsse-skud over vidt, ikke ret høyt, bart, og moeset. Grændse-\textit{limiten} meenes at gaae mitt derover. 5 \textit{Volum:}\textit{pag.} 233. Østen for \textit{Maderoive} 1 Miil er\par
24. \textit{Modtatas-oive}, fra Vester i Øster 1/2 Miil langt, ikke fuld 1/4 Miil bredt, hvor det er videst, slet ovenpaa, noget høyt, fladtvoren på Sidene, og moeset, uden Græss og Skoug; Fra dette Field rinde smaa Bække til begge Sider, baade ad \textit{Norge} og \textit{Sverrig;} Landet imellem \textit{Maderoive} og dette \textit{Modtatasoive} bestaaer af Myr, Bierke-Kratt, Sennie-Græss, og smaa Vande; Et Vand, hvoraf Bækken rinder til \textit{Norge} i \textit{Karasjok}, heder \textit{Matek-jaure}, rundt, 1 Bøsse-skud over stort, og Bækken, deraf stikker Nord, kaldes \textit{Nulles-jok;} Et andet Vand, hvoraf Aaen vender i Søer ad \textit{Sverrig}, heder \textit{Shomalopel}. Mitt imellem Vandene, og Bækkene over dette \textit{Modtatas-oive} meenes \textit{Linien} at gaae. 5 \textit{Vol:}\textit{pag:} 233 efter det 2det Vidne i \textit{FindMarken} dens Beskrivelse. Men 9de Vidne \textit{pag.} 242\textit{ibidem} forklarer det at være fra Søer i Nord 1/8 Miil langt, og 2 Bøsse-Skud bredt, paa den Søndre Ende høyere, end ellers.\par
25. \textit{Kalko-vadda}, en Slette fra \textit{Modtatas-oive} i Øster til Sønden 1/4 Miil, og 1/4 Miil lang og breed, moesed, paa Søndre Side Bierke-groed; J den Østre Ende af denne Slette er et Myrvand, 1 Bøsseskud langt fra Søer i Nord, imellem 2de Berge ganske smalt, som her i \textit{FinMarken} kaldes \textit{Ausche-suppetok}, men i \textit{Nordland}\textit{Landskift}-Vandet, see 4de \textit{Volum:}\textit{Grændse- Tabell No} 48. Af dette \textit{Auschesuppetok} har den betydelige Elv \textit{Karasjok}, der rinder i \textit{Tana}- Elv, i Nord-Nord-ost, sin første Oprindelse; Og Grændsen gaaer over \textit{Kalko-vadda} tæt Sønden for dette \textit{Ausche suppetok}. See 5te \textit{Volum:}\textit{pag.} 234. 242 og 4 \textit{Volum:}II 384.\par
26. \textit{Parse-oive}, eller som i \textit{Nordland} udtalt er, \textit{Pares-oive}, ligger strax Østen for Vandet \textit{Ausche-suppetok}, er et noget høyt Field, rundt, 4 Bøsseskud over stort, fladt med Moese ovenpaa, paa Søndre Side med Bierk begroet; Mitt over det Høyeste af dette \textit{Parseoive} gaaer \textit{Limiten}. 5 \textit{Volum:}\textit{pag.} 234.\par
27. \textit{Borvoive}, et Field, lige i Øster fra \textit{Parse-oive}, ved en slet moesed Dal adskilt langt fra Vester i Øster 3/4 Miil, 1/4 Mil over bredt, ovenpaa, og paa Sidene fladtvoren, moeset og myret, uden Skoug og Græss; Fra dets Østre Ende nedfalder en Bæk, som løber i Øster til Norden, og \textit{Formerer}\textit{Tana}-Elv; Fra den Søndre Side af dette Field rinder en Aae, navnlig \textit{Kiulam}, i Søer ad \textit{Sverrig} igiennem \textit{Beldo}-Vand i \textit{Kimi}-Elv; Landz\textit{kiølen} gaaer mitt over dette \textit{Borvoive}. 5 \textit{Volum:}\textit{p:} 234.\par
28. \textit{Gaikkem}, et Field, ligger fra \textit{Borv-oive} ved en kort Myr-Dal adskilt i Øster, rundvoren, nogle Bøsseskud stort, bart og Steen-uret over alt; Fra dets Nordre Side løber en Bæk i Nord i \textit{Gaune-jaure;} Fra den Søndre Side gaaer en Bæk i Søer i \textit{Kiulam}-Elv. Mitt over \textit{Gaikkem} er det, at \textit{Kiølen} gaaer. 5 \textit{Volum:}\textit{pag.} 234. ‒\par
29. \textit{Raudo-oive} ligger i Øster fra \textit{Gaikkem} 3/4 Miil, hvorimellem en myred og steened Dal uden Skoug er; Langt fra Søer i Nord 1 Miil, nogle Bøsseskud over bredt, bart og rundvoren ovenpaa, paa Sidene fladt og steenet; Paa Østre Side af dette Field ligger et Vand,\hypertarget{Schn1_120667}{}Tabell over Grændsens Gang.\label{Schn1_120669} \par 
\begin{longtable}{P{0.16260869565217392\textwidth}P{0.17\textwidth}P{0.16260869565217392\textwidth}P{0.35478260869565215\textwidth}}
 \hline\endfoot\hline\endlastfoot 30\tabcellsep 31\tabcellsep 32\tabcellsep 33\\
\textit{Seidekierro}\tabcellsep \textit{Maselg-aukie}\tabcellsep \textit{Maselg-oive}\tabcellsep \textit{Beldo-vadda} eller \textit{Ladde-gein}\end{longtable} \par
 \par
\textit{Raudo-jaure}, 1/2 Miil fra Søer i Nord langt, og nogle Bøsseskud bredt, hvoraf Aaen rinder i \textit{Tana}-Elv, Østen for \textit{Østre Bosmed-jaure;} dette \textit{Raudo-jaure} er paa \textit{Raudo-oives} Østre Sides Nordre Deel; Paa samme Side dens Søndre Deel ligger et andet Vand \textit{Sperko-jaure}, noget langagtigt et par Bøsseskud, hvoraf Aaen rinder i Søer ad \textit{Sverrig} ind i \textit{Kimi}-Elv. Eidet imellem disse 2de Vande ligger Østen for \textit{Raudo-oive}-Field ved dets Søndre Deel, og er imellem Vandene 2 à 3 Bøsseskud bredt, med Bierke-Kratt i, og steenet; Landz\textit{kiølen} gaaer over \textit{Raudo-oive} dets Søndre Deel, hvor dette Eid støder an paa dets Østre Side. ibid. p. 234 f.\par
30. \textit{Seidekierro}, et Field, ligger 1/6 Miil i Øster med sin Søndre Ende fra næst bem.te Eid, fra Søer i Nord 1 1/2 Miil langt, saa at dets Nordre Ende naaer hen til \textit{Tana}-Elv, 1 Miil bredt, ovenpaa slet og bart, paa Sidene fladt med smaa Kratt paa, og myret med Sennie-Græss i; Paa Søndre Side løber en liden Aae fra Fieldet i Søer ad \textit{Sverrig} i \textit{Kimi}-Elv (uden Navn). Fra Nordre Side rinde vel smaa Bække i Nord ad \textit{Norge}, men de optørkes om Sommeren. \textit{Kiølen} gaaer fra Eidet over \textit{Seidekierro} dets Søndre Ende, hvor Eidet støder an paa. 5te \textit{Volum.}\textit{pag.} 235.\par
31. \textit{Maselg-aukie}, en dyb Dal, er tæt Østen for \textit{Seidekierro}, med Fielde paa begge, nemlig Vestre og Østre Sider, 1 Bøsseskud over fra Vester i Øster bred, fra Sønden i Nord 1 Miil lang, steen-ured; J denne Dal opkommer en Aae \textit{Maselg-jok}, og stikker i Nord-Nord-ost i \textit{Skiekkem}, eller \textit{Skietzem}-Elv og dermed i \textit{Tana}-Elv; 1 Steen-Kast derfra opstiger en anden Aae, og rinder i Søer ad \textit{Sverrigs}\textit{Kimi}-Elv; det kan være 2 Mile, at \textit{Maselg-jok} løber til \textit{Skiekkem}-Elv, der hvor der er 2 Mile fra denne \textit{Skiekkem}-Elvs Udløb i \textit{Tana}-Elv; Jmellem disse 2de Aaer er Eidet moeset, og ligger Østen for \textit{Seidekierro} dets Søndre Deel; Hvorfor \textit{Kiølen} gaaer over \textit{Seidekierros} Søndre Deel, og derfra over dette Eid. 5 \textit{Volum,}\textit{pag.} 235.\par
32.\textit{Maselg-oive}, et Field, ligger tæt Østen for \textit{Maselg-aukie}, langt fra Vester i Øster 3/4 Miil, 1/8 Miil bredt, ovenpaa slet, paa Søndre Side fladt, paa Nordre steilt, moeset, uden Græss og Skoug, men neden under er Bierke-Kratt paa begge Sider. Fra dets Østre Side rinder en Bæk af et Vand \textit{Storkal-jaure}, som fra Søer i Nord er 1/4 Miil langt, 1 Bøsseskud bredt, deraf Bækken rinder 1/2 Miil Nord i \textit{Maselg-jok}, og med den i \textit{Tana}-Elv; Fra Fieldets Søndre Side nedfalder en Bæk i Søer ad \textit{Sverrig} (uden Navn) som med de andre endes i \textit{Kimi}-Elv. Over dette \textit{Maselgoive} til dets Søndre Side gaaer \textit{Kiølen}, tæt Sønden forbi \textit{Størkel-jaure}, i Øster. 5 \textit{Volum.}\textit{pag.} 235.\par
33. \textit{Beldo-vadda}, eller \textit{Ladde-gein} ligger nær Østen for \textit{Maselg-oive}, er en slet Field-Mark, med Moese paa, uden Skoug og Græss; Fra dets Vestre Ende i Øster 1/4 Miil, er et Vand \textit{Skiekkem} eller (som her udtales) \textit{Skietzem-jaure}, 1/4 Miil langt fra Vester i Øster, 3 à 4 Bøsseskud bredt; Af dette Vandz Vestre Ende gaaer \textit{Skiekkem}- eller \textit{Skietzem-jok} omtrent 1/2 Miil lang i Nord til Vesten i \textit{Piil}-Elv, (paa \textit{Finnsk} kalded \textit{Niules}-jok) hvorefter den med samme Navn \textit{Skietzem} fortløber i Nordvest 1 god Miil, eller som ved foregaaende \textit{N.} 31. er sagt, 2 Mile lang i \textit{Tana}-Elv; Og denne \textit{Skietzem} er den Østerste Aae, som har sit Løb i \textit{Tana}-Elv, og til denne \hypertarget{Schn1_120985}{}Schnitlers Protokoller V. Aaes Østere Side er det, at \textit{Jndiagers} Field-\textit{Finner}, hvorj \textit{Rusland} begynder at tage Deel i Skatten, med deres Brug og Næring gaae; Og her hvor \textit{Beldo-vadda} er, der er Enden paa \textit{Kiølen;} Thi Østen for \textit{Beldo-vadda} er en u-endelig Furre-Skoug, som rekker hen til \textit{Rusland}. 5 \textit{Volum:}\textit{pag.} 235. 355. 356.\par
Af dette \textit{Beldo-vadda}, som det sidste Field af \textit{Kiølen}, udviise de opnævnende Vidner til de \textit{Norske} Besigtelses Mænd, hvilke Øvrigheden dertil \textit{ordinerer}, ikke meere, end til \textit{Skiekkem}- eller \textit{Skietzem-jaure}, som hvorhentil, og ikke videre de fælles \textit{Norske-Svenske Koutokeino-} og \textit{Avjevaras} Field-\textit{Finner} med deres Nærings Brug skal gaae. ‒\par
Endelig til Efterretning meldes, at for \textit{specificerede} i Særdeleshed opreignede GrændseFielde kaldes med et allmindeligt Navn, \textit{Jauris-duøder}, som bemerker saa meget, som \textit{Fielde}, hvorfra Vande til begge Sider baade i Nord og Søer rinde. ‒\par
Grændserne fra Søer i Nord imod \textit{Jndiager}-Land ere forklarede før i den \textit{Summariske} Forklaring \textit{pag:} [395 ff.], nemlig de af \textit{Kautokeino}, \textit{Avjevara} og \textit{Arisbye} gaae til Vestre Side af \textit{Skietzemjok}, og \textit{Øvre Tana} Elv, siden langs med \textit{Garradas-jok}, tæt Sønden forbi \textit{Røeter-Field}, og \textit{Suokas}Field til i den Nordre Deel af \textit{Saxe-jaure}.\hypertarget{Schn1_121124}{}Forteignelse paa Vidner over Grændse-Gangen.
\DivII[Fortegnelse over vidnene i Finnmark, med kommentarer]{Fortegnelse over vidnene i Finnmark, med kommentarer}\label{Schn1_121126}\par
\centerline{\textbf{Forteignelse}}\par
paa \textit{Finmarkens} Vidner, hvilke Lands\textit{kiølen} imellem \textit{Norge} og \textit{Sverrig} have bevidnet, og Grændse-Gangen, som for-\textit{specificered} staaer, fra \textit{N} 1. til 33. \textit{inclusive} deels hørt, deels seet have: Efter \textit{Examen} i \textit{Altens} Botten:\hspace{1em}\par
1 \textit{Vidne}, \textit{Jon Nielsøn}, \textit{FieldFinn} fra \textit{Koutokeino}\label{Schn1_121184} \par 
\begin{longtable}{P{0.054472630173564754\textwidth}P{0.19916555407209613\textwidth}P{0.5373497997329774\textwidth}P{0.02212950600801068\textwidth}P{0.03688251001335113\textwidth}P{0\textwidth}}
 \hline\endfoot\hline\endlastfoot har seet\tabcellsep \textit{No} 1.\tabcellsep \multicolumn{2}{l}{\textit{Halde}, men ei været derpaa}\tabcellsep efter 5 \textit{Volum:}\textit{pag.} 232\\
‒ hørt om\tabcellsep \textit{N.} 2.\tabcellsep \multicolumn{2}{l}{\textit{Nappetiøve}}\\
‒ seet\tabcellsep \textit{N.} 3.\tabcellsep \multicolumn{2}{l}{\textit{Samasoive}, ei været derpaa}\\
‒ hørt om\tabcellsep \textit{N.} 4.\tabcellsep \multicolumn{2}{l}{\textit{Vardoive}.}\\
‒\tabcellsep \textit{N.} 5.\tabcellsep \textit{Biertevara},\tabcellsep efter 5 \textit{Vol:}\tabcellsep \textit{pag.} 232.\tabcellsep \\
‒\tabcellsep \textit{N.} 6.\tabcellsep \multicolumn{2}{l}{\textit{Tenomutkie}}\tabcellsep ‒\\
‒\tabcellsep \textit{N.} 7.\tabcellsep \multicolumn{2}{l}{\textit{Korsevara}}\tabcellsep ‒\\
‒\tabcellsep \textit{N.} 8.\tabcellsep \multicolumn{2}{l}{\textit{Aakievara}}\tabcellsep ‒\\
‒\tabcellsep \textit{N.} 9.\tabcellsep \multicolumn{2}{l}{\textit{Koskatmutkie}}\tabcellsep ‒\\
‒\tabcellsep \textit{N.} 10.\tabcellsep \multicolumn{2}{l}{\textit{Nerrevarda}}\tabcellsep ‒\\
‒\tabcellsep \textit{N.} 11.\tabcellsep \multicolumn{2}{l}{\textit{Posavara}}\tabcellsep ‒\\
‒ seet\tabcellsep \textit{N.} 12.\tabcellsep \textit{Urdevara},\tabcellsep men ej været der\tabcellsep ‒\\
‒ hørt om\tabcellsep \textit{N.} 13.\tabcellsep \textit{Maseljaure}\tabcellsep ‒\\
‒\tabcellsep \textit{N.} 14.\tabcellsep \textit{Skier-oive}\tabcellsep ‒\\
‒ seet\tabcellsep \textit{N.} 15.\tabcellsep \multicolumn{3}{l}{\textit{Pitse-kiulbme}, og gaaet over \textit{Pitse-jok}, som rinder Nord til \textit{Altens} Elv; hvilket \textit{Pitsekiulbme} ligger halv-vejs imellem \textit{Enotekies}- og \textit{Koutokeino}- Kirker, dog nærmere til denne. 5 \textit{Volumen}\textit{pag.} 232 f.}\\
har seet\tabcellsep \textit{N.} 16.\tabcellsep \multicolumn{3}{l}{\textit{Kieldevadda}, dog veed ej af Vandenes Fald, som han om Vinteren derover har faret ‒ \textit{ibid}.}\\
‒\tabcellsep \textit{N.} 17.\tabcellsep \multicolumn{3}{l}{\textit{Salvasvadda}, men ei været der.}\tabcellsep \\
Veed intet af\tabcellsep \textit{N.} 18.\tabcellsep \multicolumn{3}{l}{\textit{Kierresvadda}}\tabcellsep \\
Har seet\tabcellsep \textit{N.} 19.\tabcellsep \multicolumn{3}{l}{\textit{Keurisvara}, i \textit{Nordland} udtalt \textit{Seurisvara}, ei været der.}\tabcellsep \\
‒\tabcellsep \textit{N.} 20.\tabcellsep \multicolumn{3}{l}{\textit{Tirmesvara}.}\tabcellsep \\
‒\tabcellsep \textit{N.} 21.\tabcellsep \multicolumn{3}{l}{\textit{Bevresmutkie}, eller \textit{Bajasmutkie} ‒}\\
‒\tabcellsep \textit{N.} 22.\tabcellsep \multicolumn{3}{l}{\textit{Rovoive}}\tabcellsep \\
‒\tabcellsep \textit{N.} 23.\tabcellsep \multicolumn{3}{l}{\textit{Maderoive}. pag. 233.}\tabcellsep \\
‒\tabcellsep \textit{N.} 24.\tabcellsep \multicolumn{3}{l}{\textit{Modtatasoive}}\tabcellsep \\
\tabcellsep \end{longtable} \par
 \hspace{1em}\par
2det \textit{Vidne Mikkel Aslaksøn}, Field\textit{finn} fra \textit{Koutokeino} vidner det samme, som 1te Vidne om Grændse-Merkerne,\label{Schn1_121599} \par 
\begin{longtable}{P{0.04396551724137931\textwidth}P{0.4396551724137931\textwidth}P{0.36637931034482757\textwidth}}
 \hline\endfoot\hline\endlastfoot fra\tabcellsep \textit{No} 10.\tabcellsep \textit{inclusive}, og om\\
\tabcellsep \textit{N.} 11.\tabcellsep \end{longtable} \par
 \hypertarget{Schn1_121621}{}Schnitlers Protokoller V.\label{Schn1_121623} \par 
\begin{longtable}{P{0.34672754946727546\textwidth}P{0.32343987823439874\textwidth}P{0.15007610350076103\textwidth}P{0.029756468797564686\textwidth}P{0\textwidth}}
 \hline\endfoot\hline\endlastfoot \tabcellsep \textit{No} 12.\tabcellsep \tabcellsep \\
\tabcellsep \textit{No} 15.\tabcellsep \tabcellsep \\
\tabcellsep \textit{No} 16.\tabcellsep \tabcellsep \\
{1 NB. Seet fra \textit{No} 17 til Enden ‒}\tabcellsep \textit{No} 17.\\
\textit{No} 19.de\tabcellsep Grændse-Merke\\
\textit{No} 20.\tabcellsep ‒\\
\textit{N} 21.\tabcellsep ‒\\
\textit{N} 22.\tabcellsep ‒\\
\textit{No} 23.\tabcellsep ‒\\
\textit{No} 24.\tabcellsep ‒\tabcellsep \tabcellsep Fremdeles de Gr: Merker\\
\textit{No} 25.\tabcellsep \textit{Kalko-vadda},\tabcellsep see 5 \textit{Vol:}\textit{pag.} 234.\tabcellsep \\
\textit{No} 26.\tabcellsep \textit{Paresoive}\tabcellsep ‒ \textit{ibidem}\\
\textit{No} 27.\tabcellsep \textit{Borvoive}\tabcellsep ‒ ‒\\
\textit{No} 28.\tabcellsep \textit{Gaiktem}\tabcellsep ‒ ‒\\
\textit{No} 29.\tabcellsep \textit{Raudo-oive}\tabcellsep ‒ ‒\\
\textit{No} 30.\tabcellsep \textit{Seidekierro}\tabcellsep ‒ \textit{pag.} 235.\\
\textit{No} 31.\tabcellsep \textit{Maselg-aukie}\tabcellsep ‒ ‒\\
\textit{No} 32.\tabcellsep \textit{Maselg-oive}\tabcellsep ‒ ‒\\
\textit{No} 33.\tabcellsep \textit{Beldovadda},\tabcellsep \multicolumn{2}{l}{eller Laddegein-oivep: 235.}\tabcellsep \\
\tabcellsep \tabcellsep \end{longtable} \par
 \hspace{1em}\par
\textit{3die Vidne Rasmus Siversen}, FieldFinn fra \textit{Koutokeino, Norsk} beskikked \textit{Finne}-Lensmand, bevidner de samme Grændse-Merker, som næstforrige 2det Vidne, og {2 NB.} har seet alle fra \textit{No} 10. \textit{Nerrevarda}, indtil \textit{No} 33. \textit{Beldo-vadda}, efter 5 \textit{Volum:}\textit{pag.} 235.\hspace{1em}\par
\textit{4 Vidne Ole Olsøn}, FieldFinn fra \textit{Koutokeino}, har\label{Schn1_121907} \par 
\begin{longtable}{P{0.03913443830570902\textwidth}P{0.25828729281767954\textwidth}P{0.5525782688766114\textwidth}}
 \hline\endfoot\hline\endlastfoot alleene hørt\tabcellsep \textit{No} 15.\tabcellsep at være Grændse-Merker\\
\textit{item}\tabcellsep \textit{No} 16.\tabcellsep  Disse \textit{Finmarkiske} første 4re Vidner, nemlig det 1te 2det 3die og 4de Vidne fra \textit{Koutokeino} have, ligesom de \textit{Nordlands} Vidner i 4de \textit{Volumen}\textit{pag.} 281. vidnet om den \textit{Svenske Missions} Begyndelse i \textit{Finmarken}, og \textit{Koutokeinos}- og \textit{Arisbyes} Kirkers Opbyggelse.\\
\tabcellsep \textit{No} 17.\\
\tabcellsep \textit{No} 19.\\
\tabcellsep \textit{No} 20.\\
\tabcellsep \textit{No} 21.\\
\tabcellsep \textit{No} 23.\\
\tabcellsep \textit{No} 24.\\
\tabcellsep \textit{No} 30.\\
\tabcellsep \textit{No} 31.\\
\tabcellsep \textit{No} 32.\end{longtable} \par
 \par
hvoraf han kun har seet \textit{No} 19. ‒ see 5 \textit{Vol:}p. 236.\hspace{1em}\par
\textit{5 Vidne Peder Joxsøn, Norsk Porsangers} Field\textit{Finn}, da \textit{Finne}-Lensmand\label{Schn1_122056} \par 
\begin{longtable}{P{0.20777777777777776\textwidth}P{0.6422222222222221\textwidth}}
 \hline\endfoot\hline\endlastfoot har hørt om\tabcellsep \textit{No} 19. ‒\\
\tabcellsep \textit{No} 20. ‒\end{longtable} \par
 \hypertarget{Schn1_122073}{}Forteignelse paa Vidner over Grændse-Gangen.\par
\textit{No} 21.‒\par
\textit{No} 24.‒\par
\textit{No} 26.‒\par
\textit{No} 30.‒\par
\textit{No} 31.‒\par
\textit{No} 32.‒ og har kun seet \textit{No} 33.\hspace{1em}\par
\textit{6te Vidne Hendrik Poulsøn, Norsk Porsangers Field-Finn}, ligesom næstforrige Vidne\label{Schn1_122121} \par 
\begin{longtable}{P{0.10431818181818181\textwidth}P{0.7456818181818182\textwidth}}
 \hline\endfoot\hline\endlastfoot har hørt om\tabcellsep \textit{No} 19. ‒\\
\tabcellsep \textit{No} 20. ‒\\
\tabcellsep \textit{No} 21. ‒\\
og seet de\tabcellsep \textit{No} 24. ‒\\
\tabcellsep \textit{No} 26. ‒\\
3. NB.\tabcellsep \textit{No} 30. ‒\\
\tabcellsep \textit{No} 31. ‒\\
\tabcellsep \textit{No} 32. ‒\\
\tabcellsep \textit{No} 33. ‒ efter 5 \textit{Vol:}\textit{pag:} 237.\end{longtable} \par
 \hspace{1em}\par
7de \textit{Vidne Ole Mortensøn, Norsk Porsangers-Field-Finn}, kommer over Eens med næst- forrige 5te og 6te Vidner,\label{Schn1_122208} \par 
\begin{longtable}{P{0.20744047619047618\textwidth}P{0.6425595238095237\textwidth}}
 \hline\endfoot\hline\endlastfoot og hørt desuden\tabcellsep \textit{No} 27. ‒\\
4. NB.\tabcellsep \textit{No} 28. ‒\\
\tabcellsep \textit{No} 29. ‒\\
og seet deraf fra\tabcellsep \textit{No} 24. ‒ ind-\\
til\tabcellsep \textit{No} 33.‒ efter 5 \textit{Vol.}\textit{p}. 237.\end{longtable} \par
 \hspace{1em}\par
8de \textit{Vidne Niels Nielson, Norsk Porsangers} Field-\textit{Finn} bevidner de samme Grændse-Merker, som 5te Vidne,\label{Schn1_122272} \par 
\begin{longtable}{P{0.30546875\textwidth}P{0.54453125\textwidth}}
 \hline\endfoot\hline\endlastfoot og deraf seet fra\tabcellsep \textit{No} 24. ‒ indtil\\
5. NB.\tabcellsep \textit{No} 33. ‒\end{longtable} \par
 \par
som 7de Vidne. See 5 \textit{Volum:}\textit{pag.} 237 f.\hspace{1em}\par
J \textit{Hammerfest} afhørte:\par
\textit{9de Vidne Niels Andersøn, Norsk Hammerfest-Field-Finn} veed alleene af \textit{No} 24 ‒og derved liggende \textit{Auskesuppetok}, hvor han om Vinteren har været. See 5 \textit{Vol:}\textit{pag.} 242.\hspace{1em}\par
\textit{10de Vidne Anders Olsøn, Norsk Hammerfest} Field-\textit{Finn}\label{Schn1_122345} \par 
\begin{longtable}{P{0.0875\textwidth}P{0.7625\textwidth}}
 \hline\endfoot\hline\endlastfoot seet de\tabcellsep \textit{No} 24.\\
\tabcellsep \textit{No} 25. neml. \textit{Auske-suppetok}.\end{longtable} \par
 \hypertarget{Schn1_122366}{} Schnitlers Protokollor V.\label{Schn1_122368} \par 
\begin{longtable}{P{0.04047619047619047\textwidth}P{0.8095238095238094\textwidth}}
 \hline\endfoot\hline\endlastfoot \tabcellsep \textit{No} 26.\\
\tabcellsep \textit{No} 27.\\
\tabcellsep \textit{No} 28.\\
6. NB.\tabcellsep \textit{No} 29.\\
\tabcellsep \textit{No} 30.\\
\tabcellsep \textit{No} 31.\\
\tabcellsep \textit{No} 32.\\
\tabcellsep \textit{No} 33.\end{longtable} \par
 \par
og været der om Vinteren. See 5 \textit{Vol:}\textit{pag.} 243.\hspace{1em}\par
\textit{11te Vidne Joen Mathison, Norsk Hammerfest} Field-Finn, {7 NB.} siger det samme, som næstforrige 10de Vidne ‒ see \textit{Protocoll:} 5 \textit{Volum}. pag. 244 f.\hspace{1em}\par
\textit{Vidner} i \textit{Jngøens} Gield afhørte:\par
12 \textit{Vidne Anders Ammondsøn, Norsk Refsbottens Søe Finn}, den ældre\label{Schn1_122484} \par 
\begin{longtable}{P{0.024999999999999998\textwidth}P{0.5453125\textwidth}P{0.2796875\textwidth}}
 \hline\endfoot\hline\endlastfoot har kun hørt om \tabcellsep \textit{No} 1. ‒\tabcellsep {Disse \textit{Finmarkens}, nemlig det 12te 13de 14de 15de og 16de Vidner have ligeledes vidnet om de første \textit{Koutokeinos}- og \textit{Arisbyes Svenske} Kirkers Opbyggelse}\\
\tabcellsep \textit{No} 2. ‒\\
\tabcellsep \textit{No} 3. ‒\\
\tabcellsep \textit{No} 4. ‒\\
\tabcellsep \textit{No} 5. ‒\\
\tabcellsep \textit{No} 6. ‒\\
\tabcellsep \textit{No} 8. ‒\\
\tabcellsep \textit{No} 9. ‒\\
\tabcellsep \textit{No} 10. ‒\\
\tabcellsep \textit{No} 11. ‒\\
\tabcellsep \textit{No} 12. ‒\\
\tabcellsep \textit{No} 13. ‒\\
\tabcellsep \textit{No} 14. ‒\\
\tabcellsep \textit{No} 15. ‒\\
\tabcellsep \textit{No} 16. ‒\\
\tabcellsep \textit{No} 17. ‒\\
\tabcellsep \textit{No} 18. ‒\\
\tabcellsep \textit{No} 19. ‒\\
\tabcellsep \textit{No} 24. ‒\\
\tabcellsep \textit{No} 29. ‒\\
\tabcellsep \textit{No} 33. ‒\end{longtable} \par
 \par
De Mellem-savnede \textit{Nommer} veed ei af, og været kun paa \textit{No} 15. See 5 \textit{Vol:}\textit{pag.} 267.\hspace{1em}\par
\textit{13de Vidne Anders Ammondsøn}, den Yngre, \textit{Norsk Refsbottens} Søe-Finn, siger, som næste 12te Vidne. 5 \textit{Volum.}\textit{pag.} 268. ‒\hypertarget{Schn1_122688}{}Forteignelse paa Vidner over Grændse-Gangen.\par
J \textit{Kiøllefiord} afhørte:\par
\textit{14de Vidne, Erik Clementsøn}, Norsk \textit{Kiøllefiords} Søe-Finn,\par
har kun hørt om \centerline{\textit{Jaurisduøder} og \textit{Beldo-duøder} See 5. \textit{Volum:}\textit{pag.} 293.}\hspace{1em}\par
\textit{15de Vidne Peder Olson, Norsk Kiøllefiords} Field-\textit{Finn} vidner det samme, som næst forrige. 5 \textit{Vol.}p. 295.\par
\textit{16de Vidne Johen Andersen, Norsk Kiøllefiords} SøeFinn lige det samme ‒ efter 5 \textit{Vol:}\textit{pag.} 295.\par
\textit{17de Vidne Hans Jonsen, Norsk Tana} SøeFinn, vidste af Grændsen Jntet. 5 \textit{Vol:}\textit{p.} 297.\label{Schn1_122785} \par 
\begin{longtable}{P{0.6387573964497041\textwidth}P{0.21124260355029584\textwidth}}
 \hline\endfoot\hline\endlastfoot \textit{18. Vidne Peder Nielson}\tabcellsep \textit{Norske Tana}-Field-\textit{Finner}\\
\textit{19de Vidne Ole Olsøn}\\
\textit{20de Vidne Peder Andersøn}\\
\textit{21de Vidne Ole Pedersen}\end{longtable} \par
 \par
have forklaret \textit{Norges privative} indehavende Land imod \textit{Sverrigs Detinerende} fælles \textit{Finmarken} ‒ 5 \textit{Vol:}\textit{pag.} 299. ‒\par
\textit{22de Vidne}\textit{Clement Nielson, Norsk Tana} SøeFinn\par
\textit{23de} ‒ \textit{Mathies Mathiesen, Norsk Tana} FieldFinn\par
\textit{24de} ‒ \textit{Erik Bonjækas}, \textit{Norsk Tana} SøeFinn have beskrevet deels \textit{privative Norsk} Land deels \textit{Sverrigs} indehavende fælles \textit{Finmarken;}\par
23de \textit{Vidne} har desforuden hørt, hvor \textit{Karasjok} og \textit{Tana}-Elv oprinde, i \textit{Jaurisduoder}, der skall Landeskifte være imellem \textit{Norge} og \textit{Sverrig}. ‒\par
24de Vidne i sær har været ved\label{Schn1_122925} \par 
\begin{longtable}{P{0.017708333333333333\textwidth}P{0.8322916666666667\textwidth}}
 \hline\endfoot\hline\endlastfoot \tabcellsep \textit{No} 11. \textit{Salvasvadda}\\
\tabcellsep \textit{No} 21. \textit{Bevresmutkie}\\
8. NB.\tabcellsep \textit{No} 25. \textit{Auskesuppetok}, et Vand\\
\tabcellsep \textit{No} 26. \textit{Parse-oive}\\
\tabcellsep \textit{No} 27 \textit{Gaune-jaure}, ved \textit{Borvoive}, hvor \textit{Tana}Elv opkommer.\\
\tabcellsep \textit{N.} 33. \textit{Beldo-vadda}\end{longtable} \par
 \par
see 5 \textit{Vol:}\textit{pag.} 314 f. ‒\par
\textit{25de Vidne Jon Olsen}, tienendes hos en \textit{Norsk Tana}-Field\textit{Finn Peder Nielsøn} har hørt, hvor \textit{Karasjok} og \textit{Tana}-Elv opkomme, der er Grændse-skiell mellem \textit{Norge} og \textit{Sverrig} ‒ 5 \textit{Vol:}\textit{pag.} 320. ‒\par
\textit{26de Vidne Ole Pedersen Guttorm, Arisbye}-fælles Field-\textit{Finn}\par
\textit{27de} ‒ \textit{Peder Joensen Bidte}, nu \textit{Norsk Varanger} SøeFinn.\par
\textit{28de} ‒ \textit{Ole Mortensøn}, \textit{Arisbye} Felles Field\textit{Finn}\par
\textit{29de} ‒ \textit{Niels Clementsøn}, nu \textit{Norsk Varanger} SøeFinn vidnede, naar de Svenske \textit{Koutokeino}- og \textit{Arisbye}-Kirker først ere bygde, og naar den \textit{Svenske Missions}-Tieneste i fælles \textit{Finmarken}, har taget sin Begyndelse, 5 \textit{Vol:}\textit{pag.} 322 f.\par
\textit{30te Vidne Jver Nielsøn Vind, Norsk Varanger} SøeFinn, i \textit{Vaardøe} Gield afhørt,\hypertarget{Schn1_123140}{}Schnitlers Protokoller V.\par
\textit{31te Vidne Joen Nielsøn Askepeis, dito ibidem}\par
\textit{32te} ‒ \textit{Sabbe Olsen dito ibidem}, forklarede det yttere faste Land Norden for \textit{Varanger}-Fiord. 5 \textit{Vol:}\textit{pag.} 328.\label{Schn1_123172} \par 
\begin{longtable}{P{0.22186440677966102\textwidth}P{0.3342372881355932\textwidth}P{0.29389830508474574\textwidth}}
 \hline\endfoot\hline\endlastfoot \textit{33te Vidne}\tabcellsep Anders Povelsen\tabcellsep \textit{Norske Varangers} Søe Finner\\
\textit{34te} ‒\tabcellsep \textit{Gunder Povelsøn}\\
\textit{35te} ‒\tabcellsep \textit{Sabbe Minnesøn}\\
\textit{36te} ‒\tabcellsep \textit{Peder Andersøn,}\tabcellsep \textit{Norsk Varanger} FieldFinn\\
\textit{37te} ‒\tabcellsep \textit{Mathies Mathiesøn}\tabcellsep \textit{Norske Varangers} Søe Finner\\
\textit{38te} ‒\tabcellsep \textit{Salomon Mathiesøn}\\
\textit{39te} ‒\tabcellsep \textit{Peder Tudesøn}\tabcellsep \textit{Norske Varangers} Søe-Finner\\
\textit{40de} ‒\tabcellsep \textit{Peder Minnesøn}\\
\textit{41de} ‒\tabcellsep \textit{Morten Pedersøn,}\tabcellsep \textit{Varangers} Søe- og FieldFinn.\\
\textit{42de} ‒\tabcellsep \textit{Ole Olsøn Mind,}\tabcellsep \textit{Varangers} SøeFinn\end{longtable} \par
 \par
have bevidnet \textit{Norges privative} Land imod \textit{Arisbye} og \textit{Neidens} Fælles Finne-Byer. ‒ 5 \textit{Vol.}\textit{pag.} 328 f. og 345 f. 347 f.\par
\textit{43de Vidne Hendrik Hendriksen}, nu \textit{Norsk Varanger SøeFinn}, {9. NB.} er gaaet over \textit{Kiølen Jaurisduøder}, der hvor \textit{Piil}-Elv rinder fra, og har seet \textit{Beldovadda}, som Enden paa \textit{Kiølen.}\textit{pag.} 355\textit{ibid.}\par
\textit{44de Ole Samuelsøn}, nu \textit{Norsk Varanger} SøeFinn, {10. NB.} har været ved \textit{Piil}-Elv, som løber i \textit{Skietzem}-Elv, den Østerste, der løber Nord i \textit{Tana}-Elv. har og været ved \centerline{\textit{Beldovadda}, og veed af \textit{Skietzem-jaure}, og \textit{Skietzem-jok}.} see 5 \textit{Vol:}\textit{pag.} 355 f.\label{Schn1_123425} \par 
\begin{longtable}{P{0.22578125\textwidth}P{0.31210937499999997\textwidth}P{0.31210937499999997\textwidth}}
 \hline\endfoot\hline\endlastfoot \textit{45de Vidne}\tabcellsep \textit{Thomas Pedersøn}\tabcellsep nu \textit{Norske Varangers} Søe\textit{Finner}\\
\textit{46de} ‒\tabcellsep \textit{Peder Johansøn}\end{longtable} \par
 \par
vidste at forklare Landet, Sønden for \textit{Kiølen}. 5. \textit{Vol:}\textit{pag.} 331. 356 f.\par
\textit{47. Vidne Samuel Samuelsøn}, en \textit{Jndiagers} fælles Finn ‒ hvormed man intet har at bestille i hans Land. ‒\hspace{1em}\par
Om forbenævnte Vidner er nu at agte:\par
Naar de Kongelige \textit{Norske} til GrændseMaalingen \textit{Committerede} Hrer \textit{Officerer} om et par Aar paa Grændsen af \textit{Finmarken} efterkomme, maa være til reede at møde dennem sammestedz 1te 2det 3die 4de 5te 6te 7de 8de 9de 10de 11te 12te 13de 14de 15de og 16de \textit{item} 23de 25de 26de 27de 28de 29de 43de og 44de \textit{Vidner}, hvilke om Grændsen og Landz\textit{kiølen} vide nogen Beskeeden, hver paa sit vedkommende Sted; disse Vidner af \textit{Missions}Skolemestere, som \textit{competente} efter deres \textit{districter} ere, best til Grændsen føres, og af de \textit{Norske} Besigtelses Mænd, som af Landets Øvrighed opnævnes, didhen geleides.\par
Skulle de Hrer \textit{Jngenieurer} forlange de Vidner, som vide Skiellet imellem nu værende \textit{privative Norsk} Grund, og \textit{Sverrigs detinerende} Støkke af \textit{Finmarken}, da efter de H. \textit{Officerers reqvisition}, vill tilsiges, 33te 34te 35te 36te 37te 38te 39te 40de 41de og 42de Vidner ‒\hypertarget{Schn1_123542}{}Den provisionelle Grændse-Befaring.\par
Jeg seer ikke, at 17de 30te 31te og 32te Vidner have nødig at møde nogenstedz.\par
18de 19de 20de 21de og 22de Vidner vidste alleene \textit{privative Norsk} Land; Hvorfore stilles derhen, om de skulle behøves, at vorde paakaldede.\par
Men som en \textit{provisionell} Grændse-Befaring allernaadigst er \textit{decernered}, for ud at skulle foretages, i Overværelse af behørige \textit{Norske} Undersaattere, som Besigtelses Mænd, til den Ende, at, om ved de Kongel. \textit{Norske} Hrer \textit{Jngenieurers} Ankomst Een eller anden af Vidnerne skulle udeblive, den anordnede Grændse Maaling da, i Mangel af Vejviisere og Vidner ei skulle geraade i nogen Slags Ophold; Saa ville af de opnævnte, særdeles de 10 \textit{Notabenerede} Vidner, som ere det 2det 3die 6te 7de 8de, 10de 11te 24de 43de og 44de Vidner i det mindste til denne \textit{provisionelle} Grændse-Befaring tages, hver sine Bevidnede Grændse-Merker, saasnart giørligt, naar Fieldene for Sneen blive bare, og Vandfaldene kiendes, til den \textit{Norske Missions} Skolemester og behørige Besigtelses Mænd at anviise;\par
Hvilke \textit{Missions}Skolemestere, hver for sin \textit{district} med hoshavende \textit{Norske} Besigtelses Mænd, tage de af Vidnerne anviisende Grændse-Merker i nøye Eftersiun, beskrive hvert Merke efter dets Høyde, Længde, Leje efter \textit{Compass}-Strægen, og \textit{distance} (det Eenes fra det andet) \textit{conferere} og sammenholde det med Vidnernes Udsagn for Retten, som i forestaaende Grændse \textit{tabell} er indført, sætte siden eller tage derpaa et Merke, hvor Grændse-\textit{linien} skal trækkes, som de hos sig opteigne, at de det derefter, ved de Hr \textit{Jngenieurers} Ankomst igien kan finde og udviise; Ved Hiemkomsten fra denne \textit{provisionelle} Befaring, forføyer \textit{Missions} Skolemestern og de Norske BesigtelsesMænd med Vidnerne sig til nærmeste Geistlig, eller Verdzlig Kongelig Betient, som denne deres BefaringsForretning ordentlig ville i Pennen forfatte, og \textit{attestere;} Hvormed Skolemestern med de \textit{Norske} BesigtelsesMænd og de Vidner, som haves kan, begive sig til næste \textit{Norske} Ting, og for de Kongel. \textit{Norske} Betientere anmelder deres BesigtelsesForretning, som da den Kongel. \textit{Norske} Sorenskriver skriftlig og ordentlig ville forfatte \textit{in triplo:} Eet, som forbliver ved \textit{Protocollen} og \textit{Archivet;} det 2det tilstilles mig, at hæftes ved min Examinations \textit{Protocoll}, til den Kongel. Norske Grændse-\textit{Commissions} Underretning; og det 3die tilstilles \textit{Missions} Skolemestern og de \textit{Norske} Besigtelses Mænd, at de det medhave, at levere til de Kongelige \textit{Norske} Hr Jngenieurer, naar de med Vidnerne sig hos dennem paa Grændsen indfinde.\par
At \textit{Missions}Skolemesterne og de \textit{Norske} Besigtelses Mænd med Vidnerne paa disse deres Reiser baade til den \textit{provisionelle} Befaring, og til den \textit{Formelle} Grændse-Udviisning til de Norske Hr \textit{Jngenieurer}, nyde en billig Underholdning; \textit{derfore} ville den Kongel. Hr Amtmand eller Foged i \textit{Finmarken}, efter Kongel. Allernaadigste \textit{ordre}, behage at drage Omsorg. ‒\par
Om det 24de Vidne, \textit{Erik Bonjækas} ved \textit{Tana}-Elv, har dette at anmerke, at, som jeg har forstaaet, er hans Sager i \textit{Sverrig}, hvorfra han er kommet, i saadan \textit{Situation}, at han ei tør komme til \textit{Sverrig} tilbage, eller lade sig see af de Kongelige \textit{Svenske} Betientere; Hvorfor mig siunes, for ei at \textit{exponere} hannem, som nu en Kongel. \textit{Norsk} Undersaatt, til nogen Ulæmpe, at han, om fornøden findes, alleene bruges til den \textit{Provisionelle} Befaring, til at udviise de af ham bevidnede Grændse-Merker for de Norske Besigtelses Mænd, men derefter forskaanes for den \textit{Formelle} Grændsens Anviisning, som skeer til de efterkommende Kongelige \textit{Norske} Hrer \textit{Jngenieurer}, i Overværelse af de Kongelige Svenske Grændse-Maalere.\hypertarget{Schn1_123728}{}Schnitlers Protokoller V.\par
Af denne Forklaring over \textit{Finmarkens} Grændsegang har jeg ‒\par
d. 15 \textit{Martij Ao} 1745. tilstillet Hr Provst \textit{Angell} i \textit{Ost-Finmarken} og Fogden \textit{Wedege} et Ligelydende til hver, med en \textit{geographisk} Afteigning over Grændse-Merkerne; Forbeholdendes mig: Om jeg paa min Field-Reise igiennem \textit{Arisbye}, og \textit{Karasjok etc.} fleere Vidner skulle erholde, vilde jeg det dennem \textit{communicere} til nærmere Underretning og Foranstaltning, at deraf vedkommende \textit{Missions} Betiente og \textit{Norske} Besigtelses Mænd kan vorde instruerede.\par
Paa min FieldReise derefter erfaret samme \textit{Kiølens} Gang, som de forrige Vidner eenstemmig have udsagt. ‒\par
Lensmanden i \textit{Koutokeino, Rasmuss Siversen}, kom ned til mig i \textit{Masi}-Fieldbye, og berettede mig, i afvigte Høst med den \textit{Nordlandske Qvænangens} Skolemester og et par \textit{Norske} Besigtelses Mænd at have befaret og udmerket \textit{Kiølen} fra \textit{Kieldevadda} til imod \textit{Pares-oive}, sigendes, at ville i denne Sommer fuldføre den øvrige Befaring af \textit{Kiølen} fra \textit{Paresoive} til \textit{Beldovadda}, dersom han skulle faae sin rigtige Betaling for, hvis han havde gaaet, nemlig 24 s for sig selv, og 16 s for hver Reen, hvoraf han 2de har haft med sig. Efterdi man nu hans Tienneste har fornøden ej alleene i tilstundende Sommer til den øvrige \textit{provisionelle} Befaring, men og hans Vidnesbyrd og \textit{Comparition} ved de til Grændse-Maalingen \textit{Committerede} Hrer \textit{Officerers} Ankomst; Saa lovede jeg hannem sin Betaling, og skrev derfor til den Kongelige \textit{Tromsøens} Foged, at see hannem fornøyed; Toeg og Aftale med ham, og en anden \textit{Koutokeino}- Finn, Vidne \textit{Mikkel Aslaksen}, at de paa næste \textit{Bartholomæi}-Dag gammel stiil, som er 12 Dage efter vores \textit{Bartholomæi} ny Stiil, skulle være i \textit{Koutokeino}, da den \textit{Norske Missions} Skolemester med \textit{Norske} BesigtelsesMænd paa den Tid skulle møde dem der, at følge dem til \textit{Paresoive}, og udmerke de øvrige Merker alt til \textit{Beldo-vadda, Kiølens} Ende; det de 2de \textit{Kautokeino}-Mænd og udlovede.\par
Thi beder jeg Hr Amtmand (og Hr Provst \textit{Falk} i \textit{Vest-Finmarken}) behagelig at foranstalte, det en \textit{Norsk Missions} Skolemester med \textit{Norske} Besigtelses Mænd, hvoriblant man kunde tage et par af de kyndigste Vidner, som \textit{Ole Mortensen} og \textit{Niels Nielsen} af \textit{Porsangers} Fielde, eller \textit{Anders Olsen}, og \textit{Jon Mathisen, Hammerfest-Finner}, til berørte Tid og Sted \textit{præcise} sig hos disse \textit{Koutokeino}-Vidner indfinde, og var raadeligt, at den Kongelige Foged med Skolemestern sendte til hver af disse 2de \textit{Koutokeiner} et Forskud af 1 eller 2 Rdl. at \textit{animere} dem derved i Vejen, og naar de fra den \textit{provisionelle} Befaring med god forretted Ærende vare tilbage komme, da at giøre Afreigning med dennem, og efter Billighed fornøye dem; Thi Vi behøve de selvsamme baade \textit{Norske}- og \textit{Koutokeino}-Vidner til Mødning for de Hrer GrændseMaalere.\par
Om H. Amtmand skulle finde, at det for Almuen, saa og for den \textit{publiqve Cassa} vilde falde for tungt, at lade alle for \textit{specificerede Finmarkens} Vidner for de Hrer \textit{Officierer} møde og at foromtalte 2de \textit{Koutokeino}- og de \textit{Norske} Besigtelses Vidner kunde være nok til at beviise Grændsens Gang; Saa vill vel herom først \textit{confereres} med \textit{Commissions Secretairen}, Hr Reg.ts-\textit{qvart}.Mester \textit{Smidt;} dog derhos være fornøden, at nogle af de Vidner, neml. af 1. 2. 3. 4. 12. 13. 14. 15. 16. og særdeles 26de en \textit{Arisbye}Finn, som af den 1te \textit{Svenske Missionaire} i \textit{Finmarken} er døbt, tillige opsendes, at bevidne, hvad Tid den \textit{Svenske Mission} har begyndt, neml. for en 80 Aar siden, og naar \textit{Koutokeino}- og \textit{Arisbye}-Kirker først ere bygde, i de Sven\hypertarget{Schn1_123995}{}Den provisionelle Grændse-Befaring. ske \textit{Commissions} Betienteres Paahør. Hvilke Vidner og Besigtelses Mænd efter Kongel. \textit{ordre} og det høylofl. Rente\textit{Cammerets} Foranstaltning nyde af den Kongel. \textit{Cassa} for det første et Forskud til deres Underhold, siden ved Tilbagekomsten afbetales.\hspace{1em}\par
\textit{Elvbakken} ved \textit{Altens} Elv d. 12 \textit{April}. 1745. \hspace{1em}{Peter Schnitler.} [Pag. 478‒512: alfabetisk stedsnavnregister, utelates her.]\hypertarget{Schn1_124037}{}\par
\centerline{\textbf{[WIDNERS EXAMINATION OVER GRÆNDSERNE IMELLEM NORGE NORDENFIELDS OG RUSLAND. A{o}1744‒1745.]}}\begin{figure}[htbp]
\noindent\par
_______
\caption{\label{Schn1_124055}}\end{figure}
\par
\centerline{GRÆNDSE-EXAMINATIONSPROTOCOLLS 6TE VOLUMEN ANG. GRÆNDSERNE AD RUSSISK LAPMARK.}\par
\centerline{[Pag. 1‒42 er extract av 5. vol. pag. 345‒361 og utelates her.]}\begin{figure}[htbp]
\noindent\par
_______
\caption{\label{Schn1_124065}}\end{figure}
\par
Den 9de [Nov. 1744] Reiset fra \textit{Varanger}-Botten over \textit{Varangerfiord} i oster til \textit{Vadsøe}\par
d. 10de fra \textit{Vadsøe} over Land i Kieredser i Nordost til \textit{Vaardoehuus} Hvor\par
den 11te næstefter ankom, og foranstaltede Retten til\par
den 14de \textit{Decem}. paafølgende.\hspace{1em}
\DivII[1744 Des. 14.-19. Rettsmøte på Vardøhus]{1744 Des. 14.-19. Rettsmøte på Vardøhus}\label{Schn1_124116}\par
\textbf{Anno 1744. d. 14 Decembr} blev \textit{Examen} Over Grændse-Skiellet til \textit{Rusland} holden, hvor efter Fogdens Foranstaltning Lensmanden \textit{Ole Erichsen}, med 2de LougRettesMænd \textit{Jan Andersen} og \textit{Lars Pedersen}, begge Norske Boemænd mødte, medhavendes de Kyndigste Mænd:\par
(1) \textit{Siur Sevaldsen}, født i Nordland, 70 Aar Gl:, boendes paa \textit{Waardøe}\par
(2) \textit{Albricht Jonsen}, født i \textit{Wadsøe}, 34 Aar Gl:, nu boendes paa \textit{Waardøe}\par
(3) \textit{Johannes Zakariasen}, født i \textit{Anders} Bye i \textit{Waranger}fiord, 28 Aar Gammel, boendes paa \textit{Kiberg} paa \textit{Varanger} Næsses faste Land.\par
(4) \textit{Ole Jonsen}, født i \textit{Grundnæss} paa bem.te \textit{Warangernæss}, 33 Aar Gl:, boesiddendes Paa samme \textit{Kiberg}.\par
og (5) \textit{Peter Nielsen}, født i \textit{Kiberg}, 48 Aar Gl., Ved \textit{Waardøehuus} Fæstning\textit{Gvarnisons} Soldat;\par
Hvilke gave om Landets Leje og Lejlighed følgende Forklaring:\par
til det 2det Sp: Om \textit{Waardøe:} At den af 2de Vaager fra Sør i Nord indskiæres, Den Søndre Kaldes her den \textit{Østre waag}, Og gaar ind i Nord imellem \textit{Galgenæss} og \textit{Guldring} Næss; den anden kaldes her \textit{Westrevaag} og stikker i Sør imellem \textit{Skagenes} og \textit{Aselnæss} imod den Østre Vaag. Den Østre Deel af \textit{Waardøe} heder \textit{Waarberg}, og den Vestre i særdeeleshed \textit{Waardøe}; J den Øvrige Beskrivelse komme disse Over eet med \textit{Warangers} Vidner, \textit{pag.} 331 f. [i 5. Vol.]\par
Om de andre Øer og Holmer give de lige saadan Forklaring, som bem.te \textit{Warangers} Vidner, nemlig\hypertarget{Schn1_124300}{}1te til 5te Vidner i Vardøe Præstegield.\par
Om \textit{Tyvholm}, og \textit{Reenøe,}\textit{pag:} 333.\par
om \textit{Hornøe}\textit{pag:} 333.\par
om Lill og Store \textit{Ekkerøe}p: 333.\par
om Store og Lille \textit{Wadsøe}\textit{pag:} 333.\par
Om \textit{Bugøe}\textit{pag:} 334.\par
Om \textit{Kiøholmen}\textit{p.} 334.\par
Om \textit{Skoggerøe}\textit{p:} 334 hvorom her tillægges at den skal være fuld af Biørn.\par
Om \textit{Kielmøe}\textit{pag:} 335.\par
Om \textit{Reenøe}\textit{pag} 335.\par
Om \textit{Waardøe} Præstegieldz Faste Land, og især dets Nordre Støkke som ligger ud til Havs imellem \textit{Nordsøen} og \textit{Waranger}fiord, i Almindelighed kalded \textit{Waranger}Næss, give disse Mænd paa \textit{Waardøe} samme underretning, saavel om det indre Land, som om Søe- og Fiord- Bræddene, saavitt \textit{Normænd} boe, som de Vidner i \textit{Warangers}Botten fra \textit{Pag:} 335 til 340\textit{inclusive} givet have; det Øvrige af det faste Land, saavel i Vester mod \textit{Tana} Elv, som i Sør imod \textit{Ruslands} Grændser er dennem ubekiendt, undtagen fiordene og fiord Bræddene.\par
til Sp: 3 Om fisk sige de her, som Vidnerne i \textit{Waranger}botten\textit{pag} 349 Berettendes Ved denne Lejlighed at \textit{Russerne} fra \textit{Archangel} for 3de Aaerer siden have begynt at fiske Østen- og Norden for \textit{Waardøe} og norden for \textit{Warangernæss} 1-2-3 à 4 Miile fra Landet udpaa Havet Torsk, Lange, og særdeeles Hvedte; Til dette Fiskerie komme bem.te Russer det 1te Aar med en 40, det andet med omtrent 150 ‒ og det tredie nu sidst forløbne Aar med en 30 Baader, 4 Mand paa hver Baad; J Land naar de fare at Koge, eller Sove ligge de her paa \textit{Waardøe} inde hos \textit{Normænd} i deres Gammer; Herfor skal \textit{Russerne} have givet til den Norske Øvrighed i Villighed, som en Slags Tidende. ‒\par
til Sp: 4 Svar: Angaaendes Havne det samme som de i \textit{Waranger}botten\textit{pag} 350 tillæggendes, at uden eller Østen for \textit{Jarfiord}, paa dens Søndre Lands side er en Bugt, navnlig \textit{Sandhavn} eller \textit{Passvig}, som Pass-vig fælles finner indehave og bruge, J gabet af denne \textit{Sandhavn} eller \textit{Passvig} ligger 3 smaa Holmer, der giør at \textit{Sandhavn} er en tryg Havn Vel for 50 à 60. Skiibe see \textit{p:} [428]. Jnden til Vid rundagtig og Ved 5 Bøseskud dyb.\par
Sp: 5 Sv: \textit{Warangers} Hovedfiord beskrives her som i \textit{Waranger}botten\textit{pag:} 350 med det tillæg At som \textit{Warangersfiords} Nordre Næss er \textit{Kiberg} Saa Vil det Søndre Næss dertil \textit{HennØerne} Være, og ligesom \textit{Waranger}fiord fra \textit{Kiberg} Gaar i Sydvest til \textit{Ekkerøe}, saa gaar den fra \textit{HennØerne} i NordVest til \textit{Holmgraanes} paa den Søndre Landside.\par
Om \textit{Warangers} Jndfiorder der, navnl:\par
\textit{Bugøefiord} sige det samme som de i \textit{Waranger}\textit{pag:} 348. 349 og 351. Om \textit{Neidens} eller \textit{Kiøfiord} det samme som \textit{pag:} 351, undtagen det at de Vide intet at sige om \textit{Sør}- og \textit{Neidens}Elve hvorfra de komme, ligesaa er dennem de Andre Aaer \textit{Falkjok, Gardejok} og \textit{Nyelv} ubekiendte.\par
\textit{Passvig}fiord eller \textit{Bøgfiord}, stadfæstede det samme \textit{pag:} 350 udsagt er, \textit{item} Om \textit{Langfiord}, det som \textit{p:} 351 staaer, tilføyendes; Naar \textit{Passvig}fiord er Gaaen i Sør til \textit{HaakierringNæss}, stikker den Videre Frem i Sør 1/4 Miil lang, 1/16 Miil breed til Munden af \textit{ClosterElven} eller \textit{Passvig} Elven, hvilken Elvs Mund ligeledes er 1/16 M: over breed: Denne \textit{Pass-vig} Elv\hypertarget{Schn1_124705}{}Schnitlers Protokoller VI.\par
udrinder Søndenfra ud af \textit{Jndiager} Vand, men hvor lang og hvad Løb den derfra gaaer, Vidste man Ey.\par
\textit{Pass-vigfiord} faaer det Navn af \textit{ClosterFiord} saa snart den trænger sammen imellem \textit{Haakierring} Næss og det Østre faste Land, og \textit{Passvig}Elv kaldes \textit{Closter}Elv deraf, at for en 30 à 40 Aar siden, har et \textit{Russisk Closter} der staaet Ved denne \textit{Passvig}Elv, paa Vestre side, 1/4 Miil, Sønden for Elvsmunden, som er nedfalden og nu Øde: Men en Russisk Kirke staar derved endnu af træ, med en Klokke j, som for 5 à 6 Aar siden, er bleven \textit{reparered} og i fuld stand sat, og Ellers fra u-mindelig Aar der Været til. Denne \textit{Pasvigfiord} bruge \textit{Pasvigfinn:} allene til Laxfiskerie, dog see \textit{p:} 429 og den Norske \textit{Wardøhus gvarnison} at hugge hved i.\par
Russisk Præsten boe ikke stadig Ved denne Kirke, men kommer 2 gange om Aaret, nemlig om \textit{St Hans} tid til Søes, og Om Jule tider til Lands fra \textit{Cola}, og er der en 8te Dage mer og mindre, for at betiene Pasvigfinner; Herfra fare samme Russisk Præst om Sommeren igiennem \textit{Kaarsfiord} og Kiøfiord op efter \textit{Neidens} Elv ‒ Og om Vinteren over Land til \textit{Neidens} finner, hvilcke han i en Boegamme (som skal Være som et \textit{Capell} eller Skole) underviser: Dog naar \textit{Neidens} finner Vil Ægteviies, skeer \textit{Copulationen} i \textit{Passvig} eller \textit{Closter} Kirken.\par
Til denne \textit{Passvigs} Meenighed høre 8 \textit{Neid:} og 10 \textit{Passvigs FinneFamil:}\par
\textit{Pasvig}fiords Østre Næss heder \textit{Holmgraa}Næss Steenbierget og bart, flat og smalt udstupendes, et par Steenkast over bredt. Strax Østen om dette Næss er\par
\textit{HolmgraaFiord}, hvis Østre Næss heder \textit{Røberg} høyt og brat til alle 3 fiorder, steened; Jmellem disse 2de Næsse er \textit{HolmgraaFiord}, 1/4 Miil breed og ligesaa dyb, Landet Omkring fiorden er Steenet, og inde i Botten en Slette, af 1 Bøsseskudz Vidde, hvorj en Bæk fra VestSydVest Rinder 1/8 Miil Lang af \textit{HolmgraaVand}, rundt 1/8 Miil stort; denne \textit{Holmgraa}fiord bruger ingen hverken Normænd, ej heller Russisk Fælles finner, thi herinde er et ondt Land og intet at fortiene. Fiordens Østre Næss \textit{Røberg} er 1 Bøsseskud over bredt; deromkring strax Begynder\par
\textit{Jarfiord}, dens Østre Næss heder \textit{Spinspyd}, Jmellem hvilke Næsse fiorden i Gabet er 1/4 Miil vid, og gaaer ind i Sør 1 Miil, og siden i VestSydVest 1/2 Miil, i Alt 1 1/2 Miil Lang, Sidene Omkring høy og Barfielded. Paa Vestre side af denne \textit{Jarfiord} gaar en Liden Bugt, navnlig\par
\textit{Westerhavn}, 1 Bøsseskud ind i Vester dyb og et Bøsseskud Viid, som er en God Vinter Havn for 3 à 4 Jægter. ‒\par
Denne \textit{Jarfiord} Benyttes iche af Normænd, men allene af \textit{Pass-vig}finner til Laxefiskerie hvoraf den halve deel om Sommeren Ligge i \textit{Passvig}fiord, den Anden halve Deel i denne \textit{Jarfiord}, deres LaxeVarp er at opagte; Disse \textit{Passvig}finner formene de Norske undersaattere at komme i \textit{Passvig}fiord (see \textit{pag:} 348) og i denne \textit{Jarfiord} paa Laxfiskerie, det samme \textit{finner} Sig allene have tilægnet og ere i \textit{possession} af: Dog har den Kongl. Norske \textit{Gvarnison} paa \textit{Waardøehuus} alttid brugt og bruge endnu at tage sin Brænde hved, saavel af \textit{Passvig-} som denne \textit{Jarfiords} Bierkeskouge; det samme og Norske Undersaattere paa \textit{Waardøe}, naar de haft det fornøden, giort have, uden nogen Modsigelse af \textit{Passvigs} fælles finner, eller anden derimod skeed er, Laxen af denne \textit{Jar}- og \textit{Passvig}fiorder, selge \textit{Passvig}finner til den Norske handel. Fra \textit{Spinspyd} i SydVest er til \textit{Sandhavn} 1/2 Søe M: Landet derimellem er høyfieldet, flad nedhældendes bart, ubeboed.\hypertarget{Schn1_124970}{}1te til 5te Vidner i Vardøe Præstegield.\par
\textit{Sandhavn}, ellers af Finnerne kalded \textit{Passvig}, er en liden Bugt, et Bøsseskud over Vid i Gabet, 5 Bøsseskud dyb ad Botten, den Vestre Landside er bratt og barfielded, den Østre side slet af flade ned hældende fielde, med Lyng paa, Botten indentil er noget slet med Græss og noget Riis paa, uden skoug, denne \textit{Sandhavn} heder og \textit{Passvig}, fordj \textit{Passvig} finner fra Paaske til St \textit{Hans} tid der fiske torsk og anden fisk; Efter den tid fare de til \textit{JarFiord} og \textit{Passvig}- fiord og fange Lax, dog bruge og \textit{Normend} denne \textit{Sandhavn} til at fiske i naar de Vil, denne \textit{Sandhavn} er indentil Vid rundagtig. J Gabet af denne \textit{Sandhavn} ligge 3 holmer som giør af Denne \textit{Sandhavn} en tryg Skibshavn. ‒\par
Paa Vestre side af \textit{Sandhavn} staar en \textit{Russisk} Skole eller \textit{Capell}, en Liden Boe af Træ, uden Taarn og Klokke, som den Ved \textit{Neidens} Elv; Den Russiske Præst fra \textit{Cola} kommer did til \textit{Passvig}finner, Varendes deres fiskerie om Vaaren at underviise og lære dennem, men Naar en ægtevielse forrettes skal, maa den lige som for \textit{Neidens} finner skee i \textit{Closter}Kirken Ved \textit{Passvig-} eller \textit{Closter}Kirke.\par
Det Østre Næss af \textit{Sandhavn} heder\par
\textit{Kaarsnæss}, bart-berget, og slet ovenpaa, et par Bøsseskud over bredt, spidz udstikkendes ad fiorden fra \textit{Kaarsnæss} i Sydost til \textit{Kobbefiord}-Næss er 1/2 Miil. dette Kobbefiords Næss er fladsteenet, et par Bøsseskud over Vidt, spidz udgaaendes. Landet imellem \textit{Kaarsnes} og Kobbefiord Næss, bestaar af høye, steile og Skallede fielde, dog gaaer herimellem en Strøm ind i Landet med et lidet skiær i Gabet, som giør indløbet saa trang at næppe en Baad kan trænge sig derimellem, og er Strømmen paa bægge sider stridest naar Søen fløer og falder; Jnden for skiæret Vider Vaagen sig ud 1/8 Miil, og er rundagtig med en holm i, hvor Eed og Maase Værpe; denne holm heder \textit{Jelmen}.\par
Fra \textit{Kobbe}fiords Næss til det Østre Næss hvis Navn her ej Vidstes er i Sydost 1/4 Miil. Derimellem er Kobbefiord 1/4 Miil Vid og ligesaa dyb, Landet Omkring fiorden er steened og temmelig høyt med Lidet Græss og Noget VidieRiis paa fiord Bræddene; inde i fiorden ligge holmer, hvorpaa fugle om Sommeren Værpe og Kobbe om Vinteren yngler.\par
Denne Kobbefiord bruge Nordmænd fra \textit{Wadsøe} tilfælles med \textit{Passvig}finner som fra Arildz tid har Været saa holden, til at slaae Kobben, uden at den eene modsiger eller hindrer den anden. 1/16 Miil i S:ost fra Kobbefiords Østre Næss er \textit{Jakobs}Elv, løbendes 1/4 Miil af \textit{Jakobs}- ElvVand, som imellem 2de høye Berge er fra Sør i Nord 1/4 Miil langt, 2 a 3 Bøsseskud bredt; Landet imellem Kobbefiords Østre Næss og \textit{Jakobs}Elv, som falder ud i havet er paa Søebrædden hvid fin sand, men strax ovenfor steile Berge iche høye. Paa begge sider af \textit{Jakobs}Elv er temmelig god Bierkeskoug, sommestedz nogle Bøsseskud ‒ sommestedz 1/16 Miil à 1/8 Miil Vid, med mangfoldig Græs og god Lejlighed for Baader at ligge, Denne \textit{Jakobs}Elv er fiskeriig paa Lax, næsten som \textit{Neidens}Elv; hvilket Laxfiskerie icke Norske undersaattere, men Een af \textit{Carlsgams} eller \textit{Peisens} Fælles finner benytte sig af, og selger Laxen til Russerne af \textit{Cola}.\par
Ej heller bruge Normænd at hænte Bierkehved derfra. thi den Nærmeste er kun smaa, og den duelige noget langt borte. Fra denne \textit{Jakobs} Elv\textit{inclusive}, høre den Østre Lands side \textit{Carlsgams} Fællesfinner til, som om Sommeren ligge Ved Søesiden fra \textit{Jakobs}Elv i Sydost hen til \textit{CarlsGammen}, og om Vinteren sidde Ved \textit{Peisens} Elv, og heraf ogsaa kaldes \textit{Peisens} Finner, de skatte ligeledes til Norge. ‒\hypertarget{Schn1_125209}{}Schnitlers Protokoller VI.\par
Fra denne \textit{Jakobs}Elv er til \textit{Rysnæss} i Sydost 1/2 Miil. Landet bestaaer af bare og snaue, dog ikke høye Bierge, nedhældendes saa de kan bestiges, \textit{Rysnæss} er et lidet kort og lavt Næss, hvor et temmelig Got Kobbevejde er. Fra \textit{Rysnæss} i Sydost til \textit{Kaarshavn} 3/8 Miil. Landet derimellem er som imellem \textit{Jakobs}Elv og \textit{Rysnæss}. ‒ Denne \textit{Kaarshavn} er en Bugt nogle Bøsseskud i Gabet vid og ligesaa dyb, dog ligger et Skiær i Gabet hvilket naar man kommer forbj ind i Bugten er der en God Havn for Jægter og Baader. Denne \textit{Kaarshavn} er uden Skoug; thi komme Normænd der iche, ey heller fælles finner iche engang til at fiske.\par
Fra \textit{Kaarshavn} i Sydost til \textit{Falkefiord} er 1/4 Miil. \textit{Falkefiord} er i Gabet 3 à 4 Bøsseskud Vid og en knap 1/4 Miil dyb, med steile og temmelig høye Berge paa sidene og i Botten.\par
Fra \textit{FalkeFiord} i Sydost til \textit{Kobbenæss} er 1/2 Miil, Landet derimellem er paa Søebræden fladt, 1 Bøsseskud bredt, men siden stiger fladagtig op til tæmmelig høye Bierge. \textit{Kobbenæss} er et Kort og Lavt Næss, hvorpaa Kobben Ligger; og uden for \textit{Kobbenæss} er et lidet fladt Skiær, hvor Ligeledes Kobben Lægger sig paa.\par
Nær Ved Kobbenæss i Sydost Ligger \textit{Vestre Normandssæde}, en Liden Bugt, et par Bøsseskud Vid og Ligesaa dyb; det NordVestre Næss af dette Vestre Normandsæde er noget høyt og bratt. Landet imellem Vestre Normandsæde er som næstforige Vesten for \textit{Kobbenæss}.\par
Strax Sønden om Vestre Normandsæde er \textit{Østre Normandsæde} en liden Bugt, trang, 1 Bøsseskud over i Gabet, men indentil rundvoren 2 Bøsseskud Vid, Værendes en God havn for Baader; det Vestre Næss af dette Vestre Normandsæde er Lavt, men det Østre Næss stiger fra Søen strax høyt i Vejret. Landskabet omkring disse Bugter er Litt slet neer paa Søebrædden, men bliver derefter bratt og berget; er uden Skoug; J ældgamle tider, for disse Mænds og deres Forældres Tid skal i disse 2de Bugter \textit{Normænd} have boet, men af et \textit{Russisk Marauder} Partj, Være Bortskræmte; siden have de ligget Øde, og nu beboes af Jngen, hverken Norske, eller Russer, de bruges og ikke af nogen til betydelig Fiskerie, dog sige Mændene her paa \textit{Waardøe}, at de Norske om de Ville, kunde benytte sig af dennem.\par
Strax Sønden for det Østre Normandsæde er \textit{MunkeFiord}, som har sit Nafn deraf, at omtrænt 1/2 Miil fra FiordBotten i Vester Ved denne Elv, som kaldes \textit{MunkeElv} eller \textit{Peisens} Elv, paa Nordre side, staar et Russisk Munke Closter hvor en Russisk Munk holder til; af Timmer bygged, til dets Kirke og \textit{Peisens} finner paa hellige Dage søge. \textit{MunkeFiord} er i Gabet 1/8 Miil Vid, og med samme Bredde sticher først i S:V: 1/4 Miil siden i Vester 1 Miil Lang ind til Munden af MunkeElven, eller \textit{Peisens} Elv; det Jndre af Munkefiord er til deels breedere end det yttre, fiordbræden er Rundflad, med god Bierkeskoug begroed, derfra stiger Landet rundaktig op til temmelig høye fielde. ‒\par
J denne \textit{Munke}fiord fiske icke Normænd, men \textit{Carlsgams}- eller \textit{Peisens}finner alleene Lax, hvilken de til Russerne i \textit{Cola} tilhandle. Den Kongl. Norske \textit{Gvarnison} paa \textit{Waardøehuus} har Fra Gammel tid stedse brugt, og endnu bruge at hænte deres Brændehved fra denne \textit{Munke}fiord, som de hugge selv og bortføre, uden nogens Modsigelse. J botten af denne \textit{Munkef:} kommer \textit{Peisens}Elv fra SydSydvest, en Laxeriig Elv. Men hvorfra og hvor lang den er? Vidste Man ey. Fra fiord botten op efter \textit{Peisens} Elv staar Meget Bierk. Fra \textit{Kobbenæss} er i Øster til Nord til \textit{Hennøerne} 1 Miil, disse \textit{Hennøer} er 2. Den større ligger fra Sydvest i Nordost 1/4 Miil lang, halv saa breed, slet med torv og Græss paa, uden skoug, ubeboed. Den mindre \hypertarget{Schn1_125425}{}1te til 5te Vidner i Vardøe Præstegield.\textit{Hennøe} ligger Sønden for den store \textit{Henøe} 1/8 Miil, og fra \textit{Marifiord} dens Gab i N. O. 1 Miil liggendes, slet dog høyere end hiin, med Græs og Torvbanker paa, uden Skoug, langagtig, fra SydVest i Nordost 1/16 Miil, halv saa breed, ubeboed. Tilforn da Normænd Var mange i tallet i \textit{Waardøe} Præstegield, forre de til disse \textit{Hennøer}, og samlede der om Sommeren Eederduun og fugleEgg, og om Vinteren Sloge Kobbe, saaledes at hvo af Normænd eller \textit{Carlsgams} Finner der komme først, de toege Nytten af \textit{Henøerne:} men for et par Aar siden, da Normænd Mæstendeels ere uddøde, have de \textit{Carlsgams} finner begyndt at forbyde de Norske, at komme did, og at benytte sig af \textit{Henøerne:} dog de Norske agte icke det forbud. Fra \textit{Henøerne} til \textit{Carlsgammen} i Øster er 2 Miile.\par
\textit{Carlsgammen} er langt lavt Næss af Koppelsteen og Sand, sticher ud fra det Søndre faste Land, ud i Nord til osten. \textit{Carlsgams} finner fiske i Havet Østen for \textit{Henøerne} torsk og anden stor fisk, fra Paaske til \textit{St Hans} tid. Siden i \textit{Munke}fiord Lax til \textit{St Ols}tid, derefter fare de ud til \textit{Carlsgammen} og ligge der til MichelsM: paa deres fiskerie i Havet, omsider fløtte de derfra igiennem \textit{Munkefiord} op efter \textit{Peisens} Elv til deres Vinterbyer, saa at \textit{Carlsgamms} finner og \textit{Peisens} finner er Eet og det samme, adskilte alleene i Navnene efter deres adskillige Sommer og Vinter Lejer. Hvormange af disse \textit{Carlsgammens} Finner ere? Vidstes her icke i \textit{Waardøe}, dog meenes de at Være saamange, som baade i \textit{Neiden} og \textit{Passvig} ere. Fra den mindre \textit{Henøe} i Sør 1 Miil er\par
\textit{MariFiord}, som gaar ind i det Østre faste Land i Sydost, hvor viid og hvor lang? Vidste Man ey. Thi den besøges eller bruges icke, hverken af Normænd ej heller af Fælles finner, saasom ingen Elv fra Landet did indløber, og fiorden derfor er uden Lax. Om Vestre og Østre \textit{Buemands}fiorder Vidste ingen her noget, ej heller om Landskabet Østen derfor; thi de Norske komme ej længer end til \textit{Munke}fiord og til \textit{Henøerne;} ‒ Fra Søebræden op til fieldz Var Man paa \textit{Waardøe} ubekient.\par
Sp: 6. Elve, som man har Vist, ere forklarede før.\par
Sp: 7. Myrer og Dale Vistes her iche af.\par
Sp: 8. Svarede: Angaaendes Gaardene og Jordpladser see \textit{pag:} 352 i 5te \textit{Vol.}\par
Sp: 9. Sv: Om Skoug sige det samme som de Vidner i \textit{Waranger}\textit{p:} 352 og 353. Derhos forklaredes at nær ved fiordene paa dens Søndre Lands side, nemlig Ved \textit{Neidens}- og \textit{Passvig}- fiorders Botten er nogen furreskoug, som Norske undersaattere iche tage selv, men tinge det hos de fælles RussiskNorske finner, som hugge og om Vinteren Ved deres Reen nedkiøre det til fiorden, hvorfra \textit{Normænd} hænte det med Baad over \textit{Waranger}fiord, til den Nordre Landside og bygge deraf deres Stuegammer, Normænd have og god Nytte af Rækhved, som Søen driver ind til deres Land, baade til at bygge af og til at brænde. Dog er det iche til nogen Mængde at det kan forslaa til Huusbehof; Siden og Russerne i nogle [Aar] have lagt sig paa fiskerie nær \textit{Waardøe} og \textit{Waranger}land, saa opfange de meget af denne drivendes Rækhved i Havet for i Vejen, som afgaar de Norske undersaattere. ‒\par
Efter Tilspørgende: Hvorfra den Kongl. Norske \textit{Gvarnison} paa \textit{Waardøehuus} faaer deres Brændehved, siden ingen Skoug paa \textit{Waardøe}, og lidet, eller saa got som intet af Skoug er paa det Norske \textit{Warangers} faste Land? Mand svarede: paa den Søndre Landz Side, som her kaldes Russisksiden, ere 3de fiorder navnlig \textit{Passvig}fiord, \textit{Jarfiord} og \textit{MunkeFiord}, hvilk[e] \hypertarget{Schn1_125652}{} Schnitlers Protokoller VI. de fælles finners \textit{Passvigs} og \textit{Peisens} finner indehave, med God Bierkeskoug begroede, af hvilke fiorder den sidst benævnte \textit{Munke}fiord har den tætteste og ligeste Skoug; af disse Bierkeskouge har den \textit{Waardøehusk Gvarnison} Været Vannt til fra Arildz Tid, og endnu Vedholder, at hugge selv og føre deres Brændehved derfra til fæstningen, ureigned den Torv, som til deres fornødenhed paa Øerne stiches og hid føres. Om det Jndre Lands Beskaffenhed Sønden for disse opnævnte Jndfiorder, eller Om Landz \textit{Kiølen} eller Grændsegangen Vidste disse folk paa Waardøe ej at sige.\par
\centerline{\textbf{Merkværdigheder} ved de Russiske Grændser.}\par
Paa den mindre \textit{Henøe}, som deels giør det Sydostl: Næss af \textit{Waranger}fiord, har Fra umindelig Aar for disse Vidners og deres Forældres Tid staaet 3 Steen støtter af flade Steene \textit{en pyramides} opsatte, den eene Steen paa den anden lagd, dog uden at Være Murede, rundvoren nedentil Ved Jorden, 3 a 4 Favner Viid, smal opgaaendes, at de Oventil [ms.: Øentil] i toppen ere kun 2 a 3 favner Vide omkring; af 2 Karlmandz Høyde, den Vestre er indentil huul, saa romm, at Ved 4 Mand kan ligge inden i dend ner paa Jorden; disse Støtter staar nær Ved hinanden i en triangel, og kal[des] Endnu denne Dag de 3 Konger: den Vestre deraf skal bemærke \textit{Kongen} af Norge, den Østre Stor \textit{Førsten} (nu Kaiser) af \textit{Rusland}, den Søndre Kongen af Sverrig; hvo disse 3 Støtter først oprejst haver, og paa hvad tid? det Vidste Man paa \textit{Waardøe} iche; J Forjge tider, naar nogen Af Steenene Var nedfalden, lagde snart de Norske snart Russerne, hvo som af dennem først did kom, den op igien og holdte Støtterne Ved Lige; det og Vidne \textit{Siur Sevaldsen} paa \textit{Waardøe} for 30 aar da han sidst foer paa disse Øer giordt har. Widne \textit{Peter Nielsen} sammestedz sagde, at i 10 aar, han har nu faret derpaa, har ingen, hverken Normænd eller Rus bødet paa disse Støtter, men ladet de nedfaldne Steene ligge, saa at Støtterne nu ere kun 1 Mands høyde stor omtrent.\par
Disse 3 Steenstøtter holdes imellem Norske og Russer endnu denne Dag for \textit{Grændse}- Merker imellem Norske og Russer, og derfor have de Norske og Russiske undersaattere hid ind til brugt disse 2de \textit{Henøer} til Fælles til KobbeSkiøtterie og til Duun og Egg der at samle: Jhvorvel \textit{Carlsgams} eller \textit{Peisens} Finner altid imod de Norske naar de komme der have knurret, sigendes at de, nemlig Finnerne Skatte for disse \textit{Hen}Øer til Norges Crone, dog fare desuagted de Norske, naar de have Lejlighed did, og vedligeholde den Gaml: skik og deres Brug Ved og paa disse \textit{Hennøer}. ‒\par
Efter tilspørgende Vidste Widnerne her ej noget andet GrændseMærke Videre i Øster eller Sør at Norges Rige skulle stræche sig ad Rusland, ej Vidste heller hvorvit høysalig Kong \textit{Christian} den IVde høyloflig Jhukommelse med sin \textit{Esqvadre} i den tid er Gaaet, eller om han nogenstedz et \textit{Monument} eller afmindelsestegn har Fæstet eller oprettet, Vedblivendes at Grændseskiællet imell: \textit{Norge} og \textit{Russland} skulle \textit{Hen}Øerne Være de som hørt have.\par
Hvorvidt fra \textit{Henn}øerne til \textit{Bomeni, Cola, Candalax} og \textit{Archangel} ere. Vidste Man ej; dog sagde man at \textit{Bomeni} er den nærmeste Finne Bøyd til \textit{Carlsgams} Fælles Finner, at \textit{Cola} [er] den første Stad hvor Russer boer, som mæst ere fiskere, og skal den Største Deel deraf Være Soldatre, henhørendes til \textit{Cola} Fæstnings\textit{Gvarnison}.\hypertarget{Schn1_125838}{}6te Vidne i Vardøe Præstegield.\par
Om anden Nærmere Havn Østen for \textit{Henøerne} Vidste man her iche, uden ‒\par
\textit{Kørvaag}, paa Østre side af \textit{Carlsgams} odden og \textit{Keldin} Sønden for \textit{Cola;} dog Vidste Man iche ret at beskrive dem, uden at i sidste Russisk Svenske Krig, har et Russisk KrigsSkib ligget i \textit{Keldin} havn Vinteren Øver. ‒\par
Som \textit{Ordren} lyder, at \textit{informere} sig om de Russiske Grændser under Haanden, saa toeg man Vidnerne paa denne deres udsagn ej \textit{formelig} i Ed, men betydede dennem at ud sige deres Sandhed derom Som de, naar paakræves med Eed kunde bekræfte; Som Vidnerne Vedtoege og da denne deres \textit{Deposition} blev dem forelæst Vedbleve de den i alle Poster, Saaledes \textit{Examen} paa dette sted blev slutted, og af LaugRettet underskreven og forseigled.\hspace{1em}\par
\textit{Waardøe} d. 19de \textit{Decembr}. 1744.\par
\centerline{\textit{Peter Schnitler}.}Ole (L. S.) Eriksen Lensmand paa WaardøeLars Pedersen (L. S.) \textit{Waardøe}\centerline{(L. S.) Johan Andersen\textit{Svartnæss}}\hspace{1em}
\DivII[Des. 28.-29. Rettsmøte på Skattøra ved Vadsø]{Des. 28.-29. Rettsmøte på Skattøra ved Vadsø}\label{Schn1_125939}\par
\textbf{Anno 1744. d. 28 Decembr.} blev \textit{Examination} foretaget med en norsk Søefinn \textit{Mathias Samuelsen}, derhos Var tilstæde det 41 \textit{Widne Morten Pedersen} i Øverværelse af den Kongl. Norske fogd i Finmarken\textit{Johann Wedege} og Lensmand Peder Olsen med Laug Rettes Mænd, angaaendes særdeeles den deel af \textit{Warangerfiords} Søndre Landside hvor den støder til \textit{Neidens} Fælles finners \textit{District}.\par
\centerline{\textbf{6te Vidne}\textit{Mathias Samuelsen}}\par
Føedt i Jndjager, og kommet i sin Barndom her ner til den Norske \textit{Waranger}fiord, hvor han stedse har næret sig som Norsk Søefinn, over 50 Aar Gl: gift. Har 3 Børn, sidde nu som Norsk Søefinn i \textit{Bugøe}fiord Ved Øster Elv, Været sidst i Høst til Guds Bord i den Norske \textit{Wadsøe} Kirke; giver om \textit{Bugøefiord} og dens Vedliggende Land følgende underretning, som han med Corporlig Eed, naar paaæskes Vill bekræfte.\par
\textit{Bugøe} har fra gammel tid førend de Norske \textit{Warangers} Søefinner for en halv snees Aar sig der nedsatte, af bem.te Norske \textit{Warangers} Søefinner været brugt, og efter at de have siddet der bruges endnu denne Dag til fiskerie Kobbe- Otter og Ræve Skiøtterie paa bægge fiordbrædder Næmlig baade paa Østre og Vestre sider af \textit{Bugøefiord}, uden Moedsigelse af de \textit{Neidens} fælles finner; og ved man ej at disse \textit{Neidens} fælles finner have fisket eller Fiske J denne \textit{BugøeFiord}, de komme ej heller ned fra \textit{Brashavn}field som ligger tæt Østen for fiorden til \textit{Bugøe}- fiord, endten for at skyde Kobbe, eller Oter eller Ræfv paa nogen af fiord Brædene. ‒\par
Samme \textit{Neidens} finner fiske og icke i nogen af \textit{Bugøe}fiords Elve, hverken \textit{Oster- Søer}- eller \textit{Wester}Elve; Vel have de hørt at for meer end 20 Aar siden, J den tid førend de Norske \hypertarget{Schn1_126074}{}Schnitlers Protokoller VI.\textit{Warangers}finner satte sig ned at boe i \textit{Bugøefiord}, var en \textit{Neidens} finn \textit{Dimis Trophinsen} hans Qvinde neer og fisket et par uger i \textit{Sør}ElvsMunden, men for og Efter den Tid har ingen af \textit{Neidens} finner benytted sig af \textit{Bugøefiord} eller dens Elve, derimod Norske \textit{Warangerfinner}, altid alleene have brugt dem paa sider af fiord Brædene og i Botten; som det Og er alleene Norske Søefinner hvilke boe i \textit{Bugøefiord}, neml: 6 Mand Ved ØsterElven, 1 Mand Mittfiord paa østre side, og 2 Mænd paa Vestre fiordbredde, Hvor de sidde om Vinteren, men om Sommeren fløtte nogle af dem for fiskeriets skyld, ud til Øen \textit{Bugøe} og en Mand, nemlig nærværende \textit{Mathias Samuelsen} paa det Østre nemlig \textit{Brasshavn}Næss af \textit{Bugøe}fiord, hvor han nu i 5 Aar efter hinanden har siddet om Sommeren uden Moedsigelse af \textit{Neidens} finner; Bem.te \textit{Øster}Elv kommer fra Sydost af 3 Vande. Det første og østerste Vand er langagtigt, 1 1/2 Bøsseskud langt fra Sydost i NordVest 1 Bøsseskud over bredt. Det 2det er i samme Stræchning 3 Bøsseskud langt, halv saa bredt og mindre, bægge med det Navn \textit{Olvonjaure}; den 3die er i samme Løb 3 Bøsseskud langt, 1 Bøsseskud bredt, og kaldes \textit{Gosschiesvaig}, og fanges i alle 3 Øreter; det Østreste Vand henger med det Mellemste sammen Ved en Aa 1 Bøsseskud lang, det mellemste med det Vesterste eller sidste Vand Ved 1 Aae, 2 Bøsseskud lang[,] af dette sidste Vand \textit{GosschiesWaig} rinder Øster Elv i NordVest Vel 2 Bøsseskud lang i \textit{Bugøef:} 2 Bøsseskud Østen for dennes Botten eller Østen for \textit{Søer}Elv er før beskreven. \textit{Wester}-Elv er den samme, som kommer af \textit{Falkvand} Før beskreven.\par
\textit{Søer}Elv har nærværende Norske \textit{Waranger}finn \textit{Morten Olsen} benyttet sig af nu i 4 Aar da han har boet der baade Østen for Elven 2de Bøsseskud, og Vesten for samme Elv; Den Norske fieldfinn for hannem \textit{Carl Andersen} har siddet sammestedz Østen for \textit{Søer}Elv hen til de Vande \textit{Olvonjaure}, og gaaet end og Østen for samme Vande i Fælleskab med \textit{Neidens Finner}; uden Modsigelse af disse. ‒ Bægge Vidner have og hørt, at samme \textit{Carl Andersen} med 2de Andre finne \textit{Familier}, som Vare i hans Selskab nemlig \textit{Mathies} og \textit{Hendrik Mathie}Sønner fra \textit{Skoggerøe} have fløtted over \textit{Neidens}fiord til Vestre side af samme \textit{Neidens}fiord i Sør, derefter Norden for \textit{Neidens}Elv i Vester ner imod \textit{Bugøe}fiord, imidlertid deres Reen have gaaet og beetet paa \textit{Brasshavn}field, i Fælleskab med \textit{Neidens} Finner: dog er at agte, at denne Norske fieldfin Carl Andersen foruden sine eegne, fra [ms.: faa] den tid haver haft Reen for \textit{Neidens} finner at Vogte.\par
\textit{Brasshavn}field, som ligger tæt Østenfor \textit{Bugøefiord}, have de Norske \textit{Warangers} finner fra Gl: Tid benyttet sig af saaledes: De have taget deres ReenMoese derfra til deres Creaturer, og hugget Brændehved paa bægge fiordbræddene af \textit{Bugøe}fiords Bierkeskouge, og i dens Botten Ved dens Elve til huusbehov. ‒ Fræmdeeles som Om Carl Andersen er meldt, have de Norske fieldfinner brugt \textit{Brasshavn} field til at have deres tamme Reens dyr derpaa gaaendes i Fælleskab med \textit{Neidens:}, Men nu omstunder da der er kun 1 Norsk fieldfinn Ved \textit{Søer}Elv paa Søndre Landside af \textit{Waranger}fiord, saa kommer de Norske fieldfinner nu iche med deres tamme Reen paa \textit{Brasshavn} field og den ene field finn \textit{Morten Pedersen}, har Ved \textit{Sør}Elv Mosse og Græs nok, at han ey har nødig at søge \textit{Brasshavn} field: Dog have de icke hørt at de \textit{Neidens} finner, nu meere end til forn skulle for-[mene] dem at komme med deres tamme Reen paa \textit{Brasshavn}Field.\par
Disse samme 2de nærværende Vidner Vide og, at Norske finner have skudt VildReen paa \textit{Brasshavn}field, dog nægte det ikke at det er skeet hemmeligt formedelst et forbud fra \hypertarget{Schn1_126337}{}37te og 38te Vidner i Finmarken. de \textit{Neidens} Fælles finner, som Vill formene de Norske \textit{Warangers} finner at skyde VildReen paa \textit{Brasshavn}field. Men de Norske finner have iche saameget agtet de \textit{Neidens} finners Forbud, men skudt allige Vel, dog, som meldt, under haanden Widere i Øster end til dette \textit{Brasshavn}- field\textit{inclusive} paa det Søndre faste Land, ere de Norske \textit{Warangers} finner ey gaaet, og sidder endnu Een af de Norske \textit{Warangers} Søefinner yderst Ved \textit{Brasshavn}Næsset, der tillige er det Vestre Næss af \textit{Neidens} fiord.\par
\textit{Neidens} fiord have og de Norske undersaattere saavel Normænd, som Søefinner altid brugt og bruge til torsk og Seyfiskerj, til Kobbeskiøtterie, til at slaa Høe, til at tage deres Brændsel paa fiord brædene uden nogen Modsigelse af \textit{Neidens}finner: Men Laxefiskerie i \textit{Neidens} fiord og Elv har \textit{Neidens} finner alleene brugt, og ere i \textit{Possession} af, saa at [de] fra dette Laxfiskerie altid have afholdet de Norske undersaattere. Da nu det\par
\textit{41de Widne J Finmarken} nærværende \textit{Morten Pedersen} med flere Vidner i \textit{Waranger}- fiord\textit{p:} 348. 349 [i] 5te \textit{Vol.} have udsagt at \textit{Bugøefiord} skulle giøre skiellet imellem De Norske \textit{Warangers} og \textit{Neidens} Fælles finner, \textit{item} at \textit{Neidens} Finner skulle have forbudet de Norske at Komme i \textit{Neidens}-fiord; Saa efter tilspørgende forklarede han sig saaledes:\par
Hans forige udsagn i \textit{Warangers}fiords Botten Vill han have forstaaet, saaledes, som de \textit{Neidens} finner have Villet haft og \textit{prætenderet} det. Hvad tilforn er sagt som for 3 a 4 Aar skal Være \textit{passeret}, var en ting som angik en trætte imellem \textit{Normend} og Søefinnerne angaa endes StrandRætten af en Land-dreven Hval.\par
Det foromtalte forbud af \textit{Neidens}finner paa \textit{Neidens}fiord, Vill han ej anderledes have forstaaet end saavit Laxefiskeriet angaar, det \textit{Neidens} finner alleene have brugt, og forment de Norske undersottere: Men all den Brug som nærværende Vidne \textit{Mathias Samuelsen} har udsagt at de Norske Undersaattere i \textit{Neidens} fiord have haft og have, det samme siger han og at være Sandt og rigtig, som han og i alle Maader Stadfæstede denne \textit{Mathias Samuelsens} udsagn, som han om \textit{BugøeFiord} og dens Elve og fiord brædde Sampt om \textit{Brasshavns} field, og ellers her giort har.\par
Herpaa Forhøret blev slutted og af Lensmanden og LaugRettesmænd forseigled og underskreven.\par
\textit{Wadsøe} d. 29 \textit{Dec:} 1744. \centerline{\textit{Peter Schnitler}.}Peder (L. S.) Olsen Lensmand (L. S.) Hans Olsen(L. S.) Anders Pedersen.\hspace{1em}
\DivII[1745 Jan. 4.-5. Rettsmøte på Vadsø]{1745 Jan. 4.-5. Rettsmøte på Vadsø}\label{Schn1_126549}\par
\textbf{Anno 1745 d. 4 Januarij} kalded Man for \textit{Examinations} Retten i \textit{Wadsøe} de 2de forhen i \textit{Weinesfiord} til 8 \textit{dec:} sidstleden afhørte 37de [og 38te] Widne \textit{Mathies} og \textit{Salomon Mathies}- sønner til nærmere Forklari[n]g af deres i \textit{WeinæssFiord} forhen giorde udsagn:\par
Kongl: Maysts Fogd Sgr \textit{Wedege}, og paa Lensmandens Vegne \textit{Thore Hansen} med Laug-RettesMænd Var tilstæde;\par
Efter at man for disse 37 og 38te Vidner havde forelæst deres \textit{Deposition} i \textit{Weines}botten, saa og det udsagn, som 6te \textit{Widne Mathies Samuelsen}, og den forklaring, som 41de \textit{Morten \hypertarget{Schn1_126617}{}Schnitlers Protokoller VI. Pedersen} her paa \textit{Wadsøe} d. 29de \textit{Dec:} aflagt havde; Saa gik begge disse, nemlig det 37te og 38te Vidners forklaring ud paa det samme som næstbem.te \textit{Mathias Samuelsen} og \textit{Morten Pedersen expliceret} have, nemlig at \textit{Bugøefiord}, og dens Elve allene af Norske undersaattere, saaledes som forhen meldt, er bleven brugt; See \textit{pag.} 423 f.\par
\textit{Neidens}fiord ligeledes, med dens fiord bræder er af Norske undersaattere, som forhen Sagt, bleven benyttet og endnu benyttes: undtagen Laxefiskeriet, det \textit{Neidens} fælles finner sig Ene have forbeholden. See \textit{pag.} 425.\par
\textit{Brasshavn}field, som ligger imellem \textit{BugøerFiord} og \textit{Neidens} Finner, som er sagt tilforn i Henseende til deres tamme Reen: Vel formene de \textit{Neidens} finner de Norske at skyde VildReen paa \textit{Brasshavn}field, men de skyde der dog alligevæl hemmelig og under haanden, menendes dertil at Være berettiged, ligesaavel som \textit{Neidens} Finner, indtil dennem (nemlig de Norske finner) af deres den Norske Øvrighed vorden forbuden. See \textit{pag.} 424 f.\par
Og som nærværende 37te Vidne \textit{Mathis Mathisen} med den Norske Fieldfinn \textit{Carl} Andersen har Været i Følge, da de med deres tamme Reen fra \textit{Skoggerøe} over \textit{Neidens}fiord have fløttet i Vester til \textit{Brasshavn} field, saa efter tilspørgende, forklarede han, den fløtning saaledes: Carl Andersen med sit selskab, har om Sommeren fløtted fra \textit{Skoggerøe} med deres tamme Reen som de loede Svømme over \textit{Neidens}fiord nær Sønden for \textit{Riomevuødne}, som her udtales \textit{Reiomævuødne} og paa Norsk heder Vestre \textit{Leervog} til den Østre side af \textit{Brasshavn}, siden have de faret tverts over \textit{Brasshavn} i Vest Nord vest til \textit{Olvonjaure} og \textit{Bugøe}fiord, paa denne færd har disse Norske finner ligget med tamme Reen Paa \textit{Brasshavn}field fra Høsten til næste paafuldte Kyndelsmisse uden Modsigelse af \textit{Neidens} finner: Dog havde bem.te Norske finn Carl Andersen fra \textit{Neidens} finner den tid Reen at Vogte. Vild Reen skiøtterie Vill Vel \textit{Neidens} finner alleene tilægne sig paa dette \textit{Brashavn}field men de Norske skyde Vild Reen alligeVæll: dog kun under haanden.\par
Hvormed forhøret paa dette Stæd blev slutted og af den beskichede Lensmand med 2de LaugRettesmænd underskrevet og forseigled.\hspace{1em}\par
\textit{Wadsøe} d. 5te \textit{January} 1745. ‒ \centerline{\textit{Peter Schnitler}.}\hspace{1em}\centerline{(L. S.) Thoer Hansen som Lensmand}\centerline{(L. S.) Per Olsen}\centerline{(L. S.) Hans Olsen}
\DivII[Jan. 14.-19. Rettsmøte på Skattøra]{Jan. 14.-19. Rettsmøte på Skattøra}\label{Schn1_126833}\par
\centerline{Paa \textit{Skattøren}}\par
\textbf{Anno 1745: den 14de January} blev ved \textit{Wadsøe} nogle Normænd eller Norske Boemænd som kyndige Vare, til forhør fremkaldede, at sige deres Sandhed Om Landets Leje og Strækning med Videre, i Overværelse af \textit{Normands} Lænsmanden \textit{Peder Olsen}, og 2de Underteignede LaugRettesMænd navnl. \textit{Peder Olsen} den ældre og \textit{Hans Olsen} der ere Boemænd paa \textit{Wadsøe;}\hypertarget{Schn1_126874}{}7de Vidne i Finmarken.\par
\textbf{7de Vidne}\textit{Søren Olsen Sollnæss}, Norsk Buemand ‒ Født i \textit{Sandskier} Ved \textit{Waranger}fiord af Normænds forældre, 58 Aar Gl:, Gift, uden Børn [,] nærer sig af fiskerie i Havet, Ved sidste Helgemiss Tider Været til Gudz Bord i \textit{Wadsøe} Kirke\par
til 2det Spørsmaal siger her at Landet her Af \textit{Waardøe} Præstegield er deelt i Øer og fast Land.\par
Om \textit{Waardøe, Tyvholm} og \textit{Reenøe} forklarer det samme, som Vidnerne i \textit{Waardøe}. 6 \textit{Wol:}\textit{pag:} 417 her.\par
Om Lille og Store \textit{Ekkerøe} det samme som Vidnerne i \textit{Waranger}botten. 5te \textit{Wol:}\textit{pag.} 233 og 6 \textit{Wol:}\textit{p:} 417 her. tillæggendes at der er 10 \textit{Normend} som boe paa Stor \textit{Ekkerøe}. Om Stor og Lill \textit{Wadsøe} det samme som Vidnerne i 5te \textit{Wol:}\textit{p:} 333, og i 6te \textit{Wol:}\textit{pag} [417], med det tillæg at \textit{Vadsøesund} er en Vinterhavn for 2 a 3 Skibe. Om \textit{Skiaaholm} og \textit{Svinøe} det samme, som \textit{Warangers} Vidner i 5te \textit{Wolum:}\textit{p:} 333 f. Om \textit{Bugøe} det samme som \textit{Warangers} Vidner i 5te \textit{Wolum:}\textit{p:} 334 og i 6te \textit{Wol:} her \textit{pag:} 417.\par
\centerline{\textbf{Kiøholm}}\par
Beskriver, som de i \textit{Warangers} Botten \textit{pag.} 334. Forklarendes derhos, at denne \textit{Kiøholm} om Sommeren besiddes af \textit{Neidens} Fælles finner, som have nogle deres faar med sig og 6 FinneGammer derpaa bygde, roendes der Omkring paa Søefiskerie i \textit{Waranger} fiord: Dog Roe der og Sammestedz, og fiske baade \textit{Normend} og Norske Søefinner; Disse \textit{privative} Norske Undersaattere have ingen Huuse eller Gamme paa \textit{Kiøholm} nogen tid haft eller have, ej heller holde smaa Fæe derpaa, de Skiære og ingen torv der: mens naar de Roe der omtrent, ligge de inde i Gammene hos \textit{Neidens} finner, som tage Vel imod dem. ‒ Vel har en Norsk Finn for en 5 Aars Tid paa denne \textit{Kiøholm} opsatt en finne Gamme, men det skeede med forlovf af \textit{Neidens Finner;} Og saaledes have \textit{Neidens} finner Allene haft deres Gammer og faar paa \textit{Kiøholm}, dog Norske Undersaattere tilfælles med dem haft deres Roeing eller fiskerie deromkring, hvilken er holdet og endnu holdes.\par
For meer End 20 Aar siden Var en død Hval kommet i Land under \textit{Kiøholm}, den \textit{Neidens}- finner sig tileignede og til sig toeg; da den Norske \textit{Finmarkens} foged \textit{Soelgaard} Var kommet fra sin Rejse hiem, skal han have Sagt paa tinget, Om han havede Været tilstæde, skulle \textit{Neidens} finner ey have beholdet den. ‒\par
\centerline{\textbf{Skoggerøe}}\par
beskrives her ligesom i \textit{Warangers} Botten i 5te \textit{Wolum:}\textit{pag:} 334 f. og paa \textit{Waardøe} i 6te \textit{Wolum:}\textit{pag:} [417] med det tillæg, at \textit{Skoggerøe} har kostel: Græs og smaa Skoug til Brendsel; Det Søndre Næss paa Vestre side i bem.te 5te \textit{Wol:} paa finsk er nævnt \textit{Salagækie}, heder paa Norsk \textit{Skoggerøe} Næss, det Nordre Næss kaldes ligeledes Paa Norsk Som paa Finsk \textit{Kiesvadnæss}. Det Søndre Næss i Øster, som er af \textit{Leervaag} det Vestre Næss, paa finsk heder \textit{Raa aveniarg}, kaldes paa Norsk \textit{Leervaagnakken}, det Nordre Næss i Øster, det Finnerne kalde \textit{Galleniarg} siges paa Norsk \textit{Kobbevaag} Nacken; thi Strax Sønden derfor gaar en navnl: \textit{Kobbevaag} ind, 1/4 Miil lang og 1 Bøsseskud Over Breed, ‒\hypertarget{Schn1_127201}{}Schnitlers Protokoller VI.\par
Naar i 5te \textit{Wolumen}\textit{pag:} 335 er sagt; Jngen af de fælles Russisk-Norske Finner have siddet eller Sidde paa \textit{Skoggerøe}, saa forklares det her paa \textit{Skattøren} saaledes at ingen af Russisk Norske Fælles finner have haft Huuse eller Game derpaa at boe; Men ellers have disse fælles \textit{Neidens} finner hver Sommer i forige tider haft deres Reen derpaa gaaendes, saalænge de have haft et Got Boeskab og Mængde deraf, Men efter at de i en 3 a 4 Aars tiid have mistet mange af djsse deres Dyr for Ulven, have de ey trøstet sig til, at sende deres Reen, saa sicker, og saa længe om Sommeren paa denne \textit{Skoggerøe}. Da den Norske fieldfinn \textit{Carl Andersen} for en 9 Aar der sig satte, havde han baade for \textit{Neidens} og \textit{Passvigs} Fælles finner Reen At Vogte;\par
Efter hans tid have saa vel \textit{Neidens} Som og \textit{Passvigs} finner, Som Norske Søefinner og Boemænd haft deres Reen paa \textit{Skoggerøe} ukiæret og upaa anket af nogen af Deelene, saa at denne \textit{Skoggerøe} er til fælles Brug for Norske Undersaattere og fælles Russe Finner til Reendrift, som fra gammel tid er holdet.\par
\centerline{\textbf{Kielmøe} og \textbf{Reenøe,}}\par
beskrives ligesom i 5te \textit{Wolum:}335 med den forklaring, at ligesom \textit{Kielmøe} saa er og Reenøe brugt til Fælles af Norske Undersaattere og af \textit{Passvigs}Russe Fælles finner, Og at denne \textit{Reenøe} har Overflødigt Græss, som slaaes af en deel \textit{Wadsøes} Normænd. ‒\par
Det Nordre Støcke af det faste Land kalded \textit{Waranger} Næss, forklares her som i \textit{Waranger}-Botten 5te \textit{Wol:}\textit{pag:} 335 til 340. Dog Vidste Man intet om den Nordre Søebredde af dette \textit{Waranger}-Næss som Vender ud ad \textit{Nord}-Søen, og Vill Man sige at Bredden af \textit{Waranger} Næss fra \textit{Makur} til denne \textit{Skattøren}, som i 5te \textit{Wol:}\textit{Pag:} 336 er angivet for 5 Søe-Miile, skal Være 6 Søemiile.\par
Om Sp: 3 Fisk Svares her det samme som paa \textit{Waardøe}\textit{pag} 417 her Og i 5te \textit{Wol:}\textit{pag} 349.\par
Sp: 4. \textit{Havne} forklares her som paa \textit{Waardøe}\textit{pag:} 417 her, og Vill her siges at \textit{Sandhavn} eller \textit{Passvig}bugt er 1/4 Miil lang i S:S:O. J SydSydOst indentil 1/8 Miil Viid, men ud i Gabet 2de Riffelskud breed, holmene med reignet; Jmellem Holmene og det Østre faste Lands Næss \textit{Kaarsnakken} er Jndløbet for Skiibene, 1 Riffelskud over bredt, til at komme ind i denne Havn maa Skiibene have Loedz, for blinde Skiærs Skyld, baade i og uden for Gabet. J denne Havn kan Vell falde Flagvinde, men icke til skade; den er og inden til stille og frii for Søe bord formedelst Holmene der ligger i Gabet, og har skiøn Ankergrund, som den og er dyb nok.\par
Desforuden anmeldes her en Havn i \textit{Reenøesund}, inde i \textit{BøgFiord} eller \textit{Passvig}-fiord imellem \textit{Reenøe} og det Østre faste Land, vel for en 50 Skiibe om Someren ganske trygg, men om Vinteren af den Beskaffenhed, at sundet da fryser til.\par
Sp: 5 \textit{Warangers Hoved Fiord}, beskrives her som i \textit{Waardøe}\textit{p:} 417 her: dog Vill siges, at som \textit{Kiberg} er det Nordre Næss af \textit{Warangers}fiord paa det Nordre faste Land, saa maa \textit{Kobbenæss} paa det søndre faste Land, som er det Vestre Næss af \textit{MunkeFiord}, blive det søndre Næss af \textit{Waranger} fiord.\hspace{1em}\par
\textit{JndFiorder} af \textit{Warangers} Hovedfiord paa den søndre side kalded \textit{Rafte}siden\par
a. \textit{BugøeFiord}, derom stadfæste disse Mænd paa \textit{WadsøesSkattøren} det samme som næstforige sammestedz \textit{Mathies Samuelsen} og \textit{Morten Pedersen}\textit{p:} 423, og \textit{Salomon} samt \textit{Mathies \hypertarget{Schn1_127506}{}7de Vidne i Finmarken. Mathiesønner}\textit{p}: 425 forklaret have, nemlig at \textit{BugøeFiord} fra gammel tid har været brugt, og endnu denne dag bruges af Norske Søefinner til fiskerie Kobbe- Otter- og Ræve skiøtterie paa begge fiord breddene, nemlig baade paa den Vestre og Østre side af \textit{Bugøe}fiord, uden Modsigelse af \textit{Neidens} fælles finner; ligesaa have og samme Norske Søefinner benyttet sig af Elvene som løbe i \textit{Bugøe}fiord; og er det alleene Norske Søefinner 9 i tallet som sidde og boe paa begge sider af denne \textit{BugøeFiord}, med Videre der findes.\par
Om \textit{Brasshavn}field Østen for \textit{BugøeFiord} siges og det samme at Norske undersottere have brugt og bruges det til Mosse Samling og til tamme Reens Drift; Vilde-Reens Skiøtterie Vill Vel \textit{Neidens} finner forbeholde sig men Norske betiene sig dog deraf under haanden.\par
b. \textit{NeidensFiord} bekræftes her det samme om som \textit{p:} 425 see i dette 6te \textit{Wolum:} af \textit{Mathis Samuelsen} og \textit{Morten Pedersen, item}\textit{Solomon} og \textit{Mathis Mathisønner}\textit{p:} 425 f. her tilforn er forklared,\par
At samme \textit{Neidens} fiord, med dens fiord bræder af Norske Undersaattere benyttes til Torsk og Sejfiskerie, til Kobbeskiøtterie, til at slaa Høe og at Hugge Brændehved paa fiord bræddene: Men Laxefiskerie i \textit{Neidens} fiord og \textit{Neidens} Elv, have de \textit{Neidens} finner sig altid forbeholden, med udelukkelse af de Norske undersaattere.\par
Ellers beskrives her \textit{Neidens} fiord som i 5te \textit{Wol:}\textit{pag:} 351.\par
c. \textit{Passvig}fiord forklares her, som paa \textit{Waardøe} i dette 6te \textit{Wolum:}\textit{p:} 417 f. Dog det Næss imellem \textit{LangFiord} og \textit{ClosterFiord} som de Vidner i \textit{Waranger} i 5te Wolum \textit{p:} 351 og de paa \textit{Waardøe} i 6te \textit{Wolum:}\textit{p:} 417 her have navngivet. \textit{Haakierringnæss} det kalde de her paa \textit{Skattøren}\textit{Piss}ElvNæss, og det Østre Næss af \textit{ClosterFiord} imod dette \textit{Piss}ElvNæss\textit{Østre- Hælen}, høyt skallet og steilt ud ad fiorden med Mosse paa, og med Bierk neden under paa begge sider ved fiorden.\par
Saa snart \textit{Passvig}fiord kommer inden for disse 2de Næsse nemlig \textit{Piss}Elvnæss og Østre- Hælen, antager den det Navn \textit{Kløsterfiord}, og løber 1/4 Miil Vejs indtil \textit{Kløster}Elv der indfalder.\par
Denne \textit{Passvig}fiord benytte \textit{Passvigs} finner sig alleene til Laxfiskerie, dog hugge Norske undersaattere Brændsel af Bierkeskougen baade paa fiordbræddene og inde ved \textit{Kloster}fiord og \textit{Kloster}Elven, samt i \textit{Langfiord}botten saavel som furre Støcker til smaa Huuse og Baaders Bygning, hvilket saa fra Arildz tid har Været holden foruden at fælles Russefinner have imodsagt dem eller forbuden det; Naar de Norske møde Russe-finnerne ved disse Jndfiorder og sige dem de ere komne at hugge sig nogen hved eller timmer, svare Russefinnerne: Hugger i Gudz Navn! De Furrestokker som Normænd selv her hugge ere kun korte ved 6 Alen lange til smaat Brug; thi Stort timmer kan de om Sommeren ej drage igienem Elvene: Men naar de Vill have noget Stort timmer til rette Stue husers Bygning af 9 Alens Længde, tinge og Kiøbe de det af Russefinnerne, som Køre det ned med deres Reen om Vinteren over Sneen og Jisen ned til f[i]ordbotten; Denne furreskoug oven for \textit{Kløsterfiord} og \textit{LangFiord} strækker sig vidt i Sør i Sydvest og i Sydost, men jo længere Op fra fiord bottene jo kostbarer den bliver.\par
Om \textit{HolmgraaFiord} bekræftes her det samme som paa \textit{Waardøe} J dette 6te \textit{Wolum.}\textit{p:} 418.\par
Om \textit{JarFiord} ligeledes det samme som \textit{ibidem} her \textit{p:} 418 med forklaring at \textit{Passvig} finner icke formener de Norske andet end Laxfiskerie, thi Torsk og andet fiskerie maa og kan de Norske der søge naar de Vill.\hypertarget{Schn1_127801}{} Schnitlers Protokoller VI.\par
For en 30 Aar siden har og 4 Normænd fra \textit{Waardøe} der boet en 5 aars tid, ubehindrede af Russefinner: men for den lange Skydz derfra faldt til \textit{Waardøe} ere de fløttet derfra, og nu boer ingen Normænd der: Dog hugger enhver af Norske som vill Hved af Bierkeskougen der. ‒\par
Om \textit{Sandhavn} eller \textit{Passvigs} fiordBugt det samme her berettes, som i dette 6te \textit{Wolum:} paa \textit{Waardøe}\textit{pag.} 417. 419 undtagen, at den skal være 1/4 Miil lang indad som ved 4de Spørsmaal her \textit{p:} 428 sagt er.\par
Om \textit{KobbeFiord} meldes det samme her som i 6te \textit{Wol:} paa \textit{Waardøe}\textit{p:} 419 her med det tillæg at dens Østre Næss heder \textit{Steensnæss}.\label{Schn1_127866} \par 
\begin{longtable}{P{0.4948356807511737\textwidth}P{0.35516431924882624\textwidth}}
 \hline\endfoot\hline\endlastfoot Om \textit{Jakobs}Elv\tabcellsep  siges her det samme som paa \textit{Waardøe} i dette 6te \textit{Wolum:}\textit{p:} 420.\\
Om \textit{Kaarshavn}\\
Om \textit{FalkeFiord}\\
Om \textit{Normands}Sæder\\
Om \textit{MunkeFiord} og dens Elv \end{longtable} \par
 \par
Fra \textit{Kobbenæss} til \textit{Hennøerne} i Nordost er 1 Søemiil. Den Lille \textit{Hennøe} ligger fra Stor \textit{Hennøe} i Sydost, og fra \textit{MariFiord} 1 Miil i NordNordvest. Fra Store \textit{Henøe} lige i Øster til \textit{Carlsgammen} er 1 1/2 Søe-Miil. \textit{Carlsgams} eller \textit{Peisens} finner skal være 10.\par
\textit{MariFiord} som ligger fra Lill \textit{Henøe}1 Miil i SydSydOst, er 1/4 Miil i Gabet viid, og 1/4 Miil lang i SydSydost.\par
Fra \textit{MariFiord} 1 Søemil i OstSydost er til \textit{Westre BuemandsFiord;} denne er i gabet viid 1/2 Miil og gaar ind i Sør, hvor langt? Vidstes ej Egentlig, dog er det over 1 Miil at den er lang; fiskerie Vanker her icke, dog er der god Bierkeskoug paa Vestre fiordbræde; Jngen af de Norske kommer did enten at fiske, eller hugge hved; Man Ved ej heller her, om \textit{Normænd} der før boet have; og hvoraf den fiord har sit Navn. Denne Vestre \textit{BuemandsFiord} gaar ind i Landet, Vesten for \textit{Carlsgammen;} dog see her efter \textit{p:} 431.\par
Østen for \textit{Carlsgammen} gaar strax ind \textit{Østre BuemandsFiord} vendendes sig i Vester, saa at dennes Botten naar Moxen hen til Vestre \textit{BuemandsFiords} Botten, at Russefinnerne drage deres Baader over det Ejd imellem disse 2de \textit{Fiorder}.\par
J denne Østre \textit{BuemandsFiord} have \textit{Bomeni} Russefinner deres fiskeroing; Videre i Øster var man her ey bekient.\par
Sp: 6 og 7. \textit{Myrer} og Sletter ere mangfoldige paa \textit{Waranger} næss med Lyng og Myrgræss paa.\par
Om Elve siges her det samme som paa \textit{Waardøe} i 6te \textit{Wolum:}\textit{p:} 421 her; tillæges at sønden for \textit{Langfiordbotten} er \textit{FurreHvedvatten}, 1 stærk Miil fra Sør i Nord langt, 1/4 Miil og mindre bredt, hvoraf LangfiordElven 3 a 4 Bøsseskud lang rinder i Nord i Langfiordbotten. Furrehvedvatten har siig og Øreter; og omkring Vandet er en stor vid furreskoug, hvor \textit{Normend} selv hugge timmer.\par
Sp: 8 og 9 Om Gaarder og Skoug siges her det same som Paa \textit{Waardøe}\textit{p:} [421] Om det Jndre faste Land i Sør eller \textit{Kiølen} ved man her intet.\par
Om de 3 Støtter paa \textit{Henøerne}, siges her det samme som paa \textit{Waardøe} i 6te \textit{Wol:}\textit{p:} 422 undtagen at de ey kan sige om de 3 Steenstøtter skal bemerke GrændseSkiel imellem Rigerne, dog vide de at den Søndre støtte vendendes ad Sverrig, skal mest vere nedfalden, og at \textit{Bo\hypertarget{Schn1_128119}{}8de til 11te Vidner i Vardøe Præstegield. me[n]i} er den nærmeste Russefinnebye til \textit{Carlsgams} finner, og \textit{Cola} den nærmeste Russebye Østen for \textit{Bomeni}.\par
Om Sal: og Høyloflig Kong \textit{Christian} den IVde har ladet sette noget kiendeMerke ved Skibs Ringe eller andet \textit{Monument} nogenstedz Østen for \textit{Henøerne} Viste Man her icke, hvorpaa han blev \textit{Dimitered}.\par
\centerline{\textbf{8de Vidne}\textit{Arne Clasen Sandskier, Norsk Waranger Boemand}}\par
fød i \textit{Vadsøe} Sogn af \textit{Normends} forældre, 60 Aar Gl: gift, har 5 Børn, nærer sig med fiskerie af Søen. Været til Guds Bord kort for afvigte Juel, udsiger det samme som næst forige \textit{Søren Olsen}.\par
\centerline{\textbf{9de Vidne}\textit{Ole Olsen Olbæk}, paa \textit{Stor-Ekkerøe}, Norsk \textit{Waranger} Boemand}\par
Fød i \textit{Wiborg} i \textit{Storfinnland} af Bønder Forældre. Været Soldat i Sverrig i den langvarige Krig, \textit{Ao}1720. Hid til Finmarken overkommet, og har Næret sig af Søen, Gift, har 1 Søn, Ved Mickelsmiss været til Gudz bord i denne \textit{Wadsøe} Kirke, Kommer over Eens med Næste Vidner i alle Ting, tillæggendes at han har seet den Havn\par
\textit{Kørwaag} paa Østre side av \textit{Carlsgammen}, for et par smaa Skiibe paa den Nordre side af Vaagen, thi paa den søndre side er det Udgrunt. Østen for denne \textit{Kørvaag} har han faret omtrent 1 Mil og seet der paa Landsiden den vilde Strand ud ad Søen u-beboed, af smaa lave slette fielde, som siden stiger Op høyere i Vejret. ‒\par
\centerline{\textbf{10de Vidne}\textit{Hans Jermondsøn Sollnæss} Norsk \textit{Waranger Boemand}}\par
Fød i \textit{Sollnæss} paa \textit{Warangernæss} af Normænds Forældre, 28 Aar gammel, u-gift, er hiemme hos hans Moder en Enke, Nærer sig af Søen, ved Mikelsmis været til Gudz Bord, stadfæster det samme, som næste \textit{Olsøn Olbæk}.\par
\centerline{\textbf{11te Vidne}\textit{Thore Hansøn} paa \textit{Wadsøe} Norsk \textit{Waranger} Boemand}\par
Fød paa \textit{Wadsøe} af \textit{Normænds} Forældre, 57. Aar gammel. Gift, har 4 Børn, nærer sig af Søen, \textit{communiceret} ved sidste Helgemiss, bevidner det samme som næste \textit{Hans Jermondsen}.\par
Næst benævnte 5 Mand sagde, naar der skulle paakræves ville de gerne stadfæste denne deres giordte udsagn med deres Eed, hvorpaa Forhøret paa dette stæd blev Slutted, og af Lensmand med 2de LaugRettes Mænd underskrevet og forseigled.\hspace{1em}\par
\textit{WadsøesSkattøren} d. 19 \textit{January} 1745. \centerline{\textit{Peter Schnitler}}\hspace{1em}\par
Efter dette Fremkom \textit{Thore Hansen} og \textit{Arne Clasen}, erindrede sig ved Vestre \textit{BuemandsFiord} at tilføye: at i Gabet af denne Vestre \textit{BuemandsFiord}, liger en navnl. \textit{Siaaholm}, 3 Bøsseskud lang fra Sydvest i NordOst, 2 Bøsseskud breed, noget høye, Klipagtig, med Græsshouger paa nær ved den Østre Landside; Paa denne \textit{Siaaholm} og derved i Nordost liggende Skiær, have saavel Norske undersaattere, som \textit{Peisens} fælles finner været vante til at slaa Kobber til Fælles, uden at den eene den anden derj har Hindret.\hypertarget{Schn1_128366}{}Schnitlers Protokoller VI.\par
Som sagt, Søge Norske ey ind i denne \textit{Westre Buemands Fiord}, fordj der er intet Sønderligt fiskerie: Men om de skulle ville fare did ind, enten paa Fiskerie eller efter Brændehved af dens Biærkeskoug, skulle eller kunde ingen forbyde dem det. ‒\par
Samme 2de Mænd Mindes og, efter nøyere eftertanke af en gammel nu afdød Mand at have Fornomet, som han af andre gamle folk hørt havde, at baade i \textit{Normands}sæde og i \textit{BuemandsFiorden}, skal i Ældgamle Tider nogle Norske Buemænd have boet, som af de \textit{Careler} skal være bleven fordrevene: Men det er nu saa lang tid siden, at hverken disse, eller deres Forældre har kundet mindes at være skeet i deres Tid. ‒\hspace{1em}\par
\centerline{\textit{Peter Schnitler}.} (L. S.) \textit{Peder Olsen} Lensmand (L. S.) Peder Olsen den Ældre(L. S.) Hans Olsen\textit{Wadsøe}\hspace{1em}\par
Som man her siden fornam et og andet om de \textit{Kareler} og om \textit{Karlstranden, item} om \textit{Lyngis-Tuen} som skulle ligge ved disse Grændser, saa blev \textit{Examen} Anstillet, paa \textit{Skattøren} ved \textit{Wadsøe}\hspace{1em}
\DivII[Jan. 22.-23. Nytt rettsmøte på Skattøra.]{Jan. 22.-23. Nytt rettsmøte på Skattøra.}\label{Schn1_128449}\par
\textbf{Ao 1745. d. 22 Januarij.} Hvortil som Kyndige Mænd bleve fremkaldede Lensmand \textit{Peder OlsønThore Hansen} og \textit{Hans Olsen}, alle Norske Buemænd her paa \textit{Skattøren} i \textit{Wadsøe} Sogn, at sige hvad dem herom Vitterligt var:\par
Disse 3 Mænd Eenstæmmig Forklarede\par
\textit{Marifiord} paa søndre Lands side 1 Miil i SydSydost fra den lille \textit{Henøe} er før \textit{p:} 421. 430 her bleven omtalt; denne \textit{Marifiord} har til Østre Næss\par
\textit{Karel-tuen} en rund Houg med Lyng paa uden Skoug, paa sidene og oventil rundflad, ligesom en hatte Poll, af en Par Mands Høyde, et par Bøsseskud over stor, liggendes ovenpaa et slet field som hælder ned til \textit{MariFiord} i Vester og ned til Søen imod \textit{Henøerne} i Nord. Dette field hvorpaa \textit{Karltuen} ligger, er uden Skoug, dog har Lyng Mosse og Græs, med noget smaa Bierk neere ved sidene; det strækker sig i Sør hen til Botten af \textit{Marifiord}, og i Øster til Sønden langs med Søekanten 1 Søemiil lang hen til Vestre \textit{BuemandsFiord}, er slet ovenpaa, nedhældendes indtil imod Søen. 1/8 Miil Østen for \textit{Kareltuen} gaar fra dette field et Næss ud i søen med det Navn\par
\textit{Haabrandnæss;} Fra dette \textit{Haabrandnæss} i Øster begynder dette field fra Søen op at blive flauvt eller steilt et støcke op i Vejret, at ingen Mand kand bestige det, og denne flauv varer langs med Søekanten alt hen til Vestre \textit{BuemandsFiord}.\par
Paa samme field ovenfor \textit{Haabrandnes} ligger den 2den Houg eller tue, Rundagtig med Lyng paa, icke saa høy eller saa vid som den 1te Houg \textit{Kareltuen}. 1/2 Miil i Øster til sønden fra \textit{Haabrandnæss} er\par
\textit{Kierringnæss}, det Vestre Næss af Vestre \textit{BuemandsFiord}, steen-berget, fladt paa Vestre og Østre sider, icke ret høyt fra Landet spidz udstikkendes i Søen i Nordost, 1/2 Bøsseskud Over stort; Ovenpaa dette \textit{Kierringnæss} er den 3die Runde Houg eller Tue med Lyng og Mosse paa, noget mindre end de 2de Andre næstomtalte;\hypertarget{Schn1_128578}{}Om Karelerne og Karelstranden.\par
Tæt Norden for \textit{Kierringnæss} er et lidet Skiær langagtigt 3 a 4 trin stort, som ved Flodtid overskylles.\par
All denne Søekust fra \textit{Marifiord} i Øster til Sønden hen til Vestre Buemandsfiords Vestre Næss \textit{Kierringnæss} kaldes endnu denne Dag\par
\textit{Karel-Stranden}, og de 3de før omtalte Houger ovenpaa dens fielde heder\par
\textit{KarelTuer;} deraf er det og, at det store Næss Østen for Vestre \textit{BuemandsFiord} fra det faste Land udstickendes i Nordost Vel 1 Miil langt, kaldes\par
\textit{Karelsgammen}, som er det samme som\par
\textit{Karelsbye}. Dette \textit{Karels Gammen} er fra Vestre \textit{BuemandsFiord} i Øster til Østre \textit{Buemandsfiord} indentil vel 3/4 Miil over bredt, bart og slet med Lyng og noget Græss paa, ude imod Odden bliver dette Næss smalere, saaledes, at 1/4 Miil sønden fra Odden, gaar en Vaag fra Vester, og den anden Vaag fra Øster ind paa dette Næss; den Vestre Vaag er det hvor \textit{KarlsgamsFinner} om Someren holde til i deres Gammer eller Hytter; den Østre Waag heder \textit{Kørwaag} før \textit{p:} 431 ommelt.\par
Disse \textit{KarlsgamsFinner}, som og kaldes \textit{Peisens}finner siunes nu at være af de slags folk, som i ældgamle tider have haft det Navn: de \textit{Kareler;} ‒\par
\textit{Deponenterne} paa \textit{Wadsøe} have og af deres Fædre fornommet, som disse af gamle folk hørt have, at imellem de Norske og disse \textit{Karlsgams Finner}, skal i ældgamle tider altid have været Striid og u-forligelses Maal;\par
J Forige Dage have og disse \textit{Kareler} været mangfoldige og selvraadigere end de nu ere; Men nu omstunder ere de formindskede til en halv snees i Tallet, at forstaae \textit{Familier}, og skikkelige at omgaaes med, saa at naar de Norske med denem paa \textit{Hennøerne} sammenkomme, forliges de med hinanden. ‒\par
Østen for \textit{Karels-gammen} vide \textit{Deponenterne} icke ey heller have hørt at Norske undersaattere nogen tid ere komne med deres Brug. ‒\par
Ved \textit{Lynges-Tuen} mene \textit{Deponenterne}, at der maa forstaaes den 1te og største \textit{KarelTue} med Lyng bevoxen, som er det Østre Næss af \textit{MarjFiord}, og ligger i Sydost fra \textit{Waardøe}.\par
Ved Holmen vill Riimelig forstaaes den Lille \textit{Hennøe}, med de 3 \textit{Kongers} Støtter Paa, liggendes i Nord under denne lyngede \textit{Karel-tue}.\par
Ellers give \textit{Deponenterne} Berettning om en\par
\textit{Lyng-øe}, som ligger imellem \textit{Kaarsnæss} og \textit{Strømmen}, hvorom tiener til Oplysning: Paa \textit{Waardøe}\textit{p:} 419 er omrørt, at \textit{Kaarsnæss} er det Østre Næss af \textit{Sandhavn}, og fra dette \textit{Kaarsnæss} i Øster til \textit{KobbeFiord} er 1/2 Miil; Jmellem \textit{Kaarsnæss} og denne \textit{KobbeFiord} er nu \textit{Strømmen}. neml: 1/4 Miil Østen for \textit{Kaarsnæss;} Og imellem \textit{Kaarsnæss} og \textit{Strømmen} 1 Bøsseskud fra det faste Land ligger denne \textit{Lyngøe}, 4 Bøsseskud Lang fra Vester i Øster, og 2 Bøsseskud breed, klipped ud ad søen, men indentil med Græss og Lyng begroet, som er u-beboed, og bruges af Jngen hverken Norsk eller \textit{Passwigfinn}; Paa søndre side af denne Øe er en Liden Bugt, for 1 eller 2 Baader at berges. ‒\par
1/8 Miil i Vester fra \textit{Lyngøe}, og 1/8 Miil i Nordost fra \textit{Kaarsnæss} ligger et Skiær, rundt, høyt og bart, 5 à 6 Bøsseskud over stort, kaldes\hypertarget{Schn1_128832}{}Schnitlers Protokoller VI.\par
\textit{Lyng-øeKalven}, som tienner til efterrettning for de Roende, men denne \textit{Lyngøe} og \textit{LyngøeKalven} ligger fra \textit{Waardøe} i SydSydvest. ‒\par
Denne Forklaring om de \textit{Kareler, Karlstranden} og \textit{Kareltuen} give nu \textit{Deponenterne} her paa \textit{Wadsøe:} Men om denne \textit{Karelstrand} eller \textit{Kareltue} skulle Være et LandeMerke, det kunde de icke sige: Dog slutte de at de 3 Steenstøtter, navnlig \textit{Konger} der, har haft nogen Betydning af LandeMerke. Og hermed blev Forhøret sluttet, og af LaugRetten forseiglet, underskreven ‒\hspace{1em}\par
\textit{Skattøren} ved \textit{Wadsøe} d. 23de \textit{January} 1745. \centerline{\textit{Peter Schnitler}}\centerline{(L. S.) \textit{Peder Olsen} Lensmand}\centerline{(L. S.) Peder Olsen den ældre}\centerline{Endre Johansen (L. S.)}\hspace{1em}
\DivII[Febr. 12.-15. Rettsmøte i Varangers Østbotten]{Febr. 12.-15. Rettsmøte i Varangers Østbotten}\label{Schn1_128927}\par
\textbf{Ao 1745 d. 12te February.}\par
J \textit{Warangers Østbotten} ved \textit{Warangers} Markested i \textit{Peder Larsen} en Norsk Søefinn hans Vinterbye eller Vinter-Gamme, bleve til Nytt forhør fremkaldede forige \textit{Warangers} Vidner, nemlig efter 5te \textit{Wolum:}, det 35te \textit{Sabbe Minnesen}, det 36te \textit{Peder Andersen}, det 40de \textit{Peder Minnesen} og 42de \textit{Ole Olsen Mind}, for nærmere at forklare deres giorte udsagn i 5te \textit{Wolumens}\textit{pag:} 348 og 349. Wed denne Ret var Finnelensmanden \textit{Hendrich Matthisen} og med ham 2de LaugRettesMænd tilstæde.\par
Den 1te \textit{Sabbe Minnesen} blev forelæst hans \textit{Deposition} af 5te \textit{Wolumens}\textit{p:} 348 og 349 særdeeles angaaendes \textit{BugøeFiord, Brasshavnfield}, \textit{Neidens}fiord og deruden for liggende Øer, hvorefter han og fik at høre hvad de Vidner i \textit{Wadsøe}, navnlig \textit{Mathis Samuelsen, Morten Pedersen}, \textit{Mathis} og \textit{Salomon Mathisønner} i denne 6te \textit{Protocols Wolum:}\textit{p:} 423 og 425 derom havde udsagt; og bem.te \textit{Sabbe minnesen} forklarede sig saaledes:\par
Hvad han om \textit{BugøeFiord} har vidnet er at forstaa paa den Maade, som \textit{Neidens} Fælles Finner i forige gl: tid have villet haft dett. Thi da disse \textit{Neidens} finner Vare Mangfoldig flere end de Nu ere, saa ville de tilegne sig Deel i \textit{Bugøefiord} og \textit{Brasshavn}field allene til Reen Beete og VildReenskiøtterie, saaledes Mindes han at han og hans Broder \textit{Peder minnesen} det 40de Vidne for en 30 Aar siden, da de vare unge, have været i \textit{BugøeFiord} paa Kobbeskiøtterie i selskab med \textit{Neidens} finner, engang, og hvad de tilsammen skød, det blev deelt med \textit{Neidens} finner i lige Deel; Om nu \textit{Neidens} finner har været der fleere gange, og enten før eller efter brugt Kobbe- Otter eller Ræveskiøtterie eller fiskerie i denne \textit{BugøeFiord}, det kan Vidnerne ej sige: Men det Ved han, siden de Norske \textit{Warangers} Søefinner for en halv snees Aar have taget \textit{BugøeFiord} til deres Boeliger og der nedsatt sig, ere \textit{Neidens} finner ey kommet ner til \textit{BugøeFiord} hverken paa Skøtterie eller fiskerie, helst da de ere fra en 30 \textit{Familier} som de fordum vare, ere bortdøde til en 8 \textit{Familier}, som nu kun igien ere af \textit{Neidens} Finner.\par
At \textit{Brashavn}field af den Norske Finn \textit{Carl Andersen} og hans selskab som af de andre er udsagt, til Beete for tamme Reen Har været brugt, og siden den tid af de derved boende Norske finner bliver brugt til fælles med \textit{Neidens} finner, det haver han hørt. ‒\par
\textit{Neidensfiord} have Norske Undersaattere brugt og endnu bruge til torsk og Sejfiskerie \hypertarget{Schn1_129143}{}Om Karelerne og Karelstranden. i Gabet omkring \textit{Kiøholm}, til Kobbeskiøtterie inde i Fiorden alt indtil Botten og til Høeslotte, Mossesamling og Brendehugster paa Fiordbreddene.\par
Hvad for 3 a 4 Aar paa \textit{tinge} er bleven talt om \textit{Bugøe}fiord, det angik icke Neidens finner, men var en sag imellem Norske Boemænd og Norske Søefinner; Ellers om \textit{Neidens}fiord siger det samme som \textit{Søren Olsen Solnæss}\textit{p:} 427.\par
Om \textit{Kiøholm} stadfæste dette Vidne det samme som Søren Olsen \textit{Sollnæss} paa \textit{Wadsøe} i dette 6te \textit{Wolum:}\textit{p:} [427] forklaret har: dog ved intet af den Hvalvrag der er talt om, \textit{Neidens Finner} skal have til sig taget.\par
Om \textit{Skoggerøe} og \textit{Reenøe} tillægger dette Vidne det samme som bem.te \textit{Søren Sollnæss} i 6te \textit{Wolum:}\textit{p:} 427 f. sagt haver, nemlig at \textit{Skoggerøe} har Været til Fælles Brug for Norske og \textit{Neidens} finner at disse have haft deres Reen derpaa; ‒\par
\textit{Reenøe} har ligeledes Været til Fælles Brugt af Norske og \textit{Passvigs} finner, at hiine have boet og disse haft deres Reen derpaa. Dog ikke nu omstunder, som før, for Ulvenes skyld.\par
40de Vidne \textit{Peder Minnesøn} Broder til næstforige, kommer med samme \textit{Sabbe minnesøn} i Alt over eens. ‒\par
42de Widne \textit{Ole Olsen Mind}, forklarer sig paa samme Maade som Vidne \textit{Mathis Samuelsøn} i dette 6te \textit{Wol:}\textit{p:} 423 Angaaendes \textit{BugøeFiord}, at Norske finner altid have brugt den baade nu efter at den af Norske Finner Er bebygt og beboes, og for den tid, og har han ey hørt at \textit{Neidens} finner nogen tid have tileignet sig Halvdeelen af \textit{Bugøefiord:} eller at de have haft noget Brug derj, i det mindste ved han at i 12 Aar, hvoraf han i de 10 første Aar har boet i \textit{Bugøefiord}, at bruge der enten fiskerie eller Skiøtterie.\par
\textit{BrasshavnField} har han hørt at i Gammel tid \textit{Neidens} finner skal have Villet tilegne sig alleene: Men han, Vidnet, Veed at det af Norske og \textit{Neidens} finner er brugt og bruges til fælles for tamme Reen; VildReen Ville vel \textit{Neidens} forbyde de Norske at skyde, men Vidnet selv des u-agtet har skudt dem paa \textit{BrashavnField}.\par
Om \textit{Neidens}fiord siger dette Vidne det samme som \textit{Sabbe Minnesøn}\textit{p:} 434 dog veed ej om Norske slaa der Græss.\par
Om \textit{Skoggerøe} og \textit{Reenøe} stadfæste Vidnet det samme, som \textit{Sabbe Minnesen} paa næstforige side har sagt.\par
Alle Vidner stemme deri over et, at de Norske vel forliges med Russefællesfinner, og høres aldrig til nogen trætte eller u-samdrægtighed dem imellem.\par
36te Vidne \textit{Peder Andersen}, vidste intet om \textit{Bugøefiord} i sær at forklare.\par
Efter tilspørgende sagde Vidnet \textit{Peder Minnesen} at have hørt at i gl: tid for en 50 Aar siden, skal \textit{Passvigs} finner i Tallet have været vel 40 \textit{Familier} som siden af Sott ere bortdøde og formindskede til en halv snees \textit{Familier}.\par
Og hermed blev forhøret paa dette stæd slutted.\hspace{1em}\par
J \textit{Warangers Østbotten} d. 15 \textit{February}. 1745. \centerline{\textit{Peter Schnitler}}\centerline{\textit{Hendrich Mathisen} Lensmand (L. S.)}\centerline{\textit{Peder Larsen} i \textit{Ostbotten} (L. S.)}\centerline{\textit{Tude Pedersen} i \textit{Ostbotten}}
\DivI[Stedsnavnregister til Bind I]{Stedsnavnregister til Bind I}\label{Schn1_129438}
\DivI[Forord til stedsnavnregisteret]{Forord til stedsnavnregisteret}\label{Schn1_129439}\par
FORORD TIL STEDSNAVNREGISTRET\par
Til stedsnavnregisteret i bind I er brukt de samme kartene som i det tidligere utkomne bind II:\par
De utkomne blad av 
     \par \bgroup\itshape 5800001\textit{«Topografisk kart over Norge»} i målestokken 1 :100 000 \egroup\vskip6pt\par
   . De er oppført med sine signaturer, for gradteigskartene bokstav, nummer og navn, f. eks. K 13 Bodø, for rektangelkartene nummer, bokstav og navn, f. eks. 51 D Sørli. ‒ For gradteigskartene er som i bind II (se forordet der) oppført rutenummer, som går fra 1 til 12 (sjelden til 14). ‒ Både for de norske kartene og for de svenske og finske er det ofte oppgitt i hva retning på kartet et navn er å finne.\par

     \par \bgroup\itshape 5800002\textit{Kart over Troms fylke} i 4 blad i målestokken 1 : 200 000 \egroup\vskip6pt\par
   ; det nordvestre, nordaustre, sørvestre og søraustre blad av dette kartet siteres i samme rekkefølge AkNV, AkNAU, AkSV og AkSAU. (Ak = Amtskart.)\par

     \par \bgroup\itshape 5800003\textit{Generalstabens karta över Sverige, norra delen}, utgiven av Rikets allmänna kartverk i målestokken 1 : 200 000 \egroup\vskip6pt\par
   , blir oppført med nummer og navn, f. eks. 1 Råstojaure.\par

     \par \bgroup\itshape 5800004\textit{Karta öfver Storfurstendömet Finland}, utgifven af Öfverstyrelsen för landtmäteriet i målestokken 1 : 400 000 \egroup\vskip6pt\par
   , blir sitert «finsk Aa 1», «finsk Aa 2», ... «finsk A 1» osv. ‒ Dette kartet over Finnland er fordelt på 4 store blad, hvert av dem (i det brukte eksemplaret, lånt fra Universitetsbiblioteket i Oslo) inndelt i 5 x 6 = 30 rektangler (21 cm X 25 cm), slik at en får 10 nord ‒ sør- rekker og 12 vest ‒ aust-rekker av rektangler. Disse rektanglene er fra nord til sør merket med bokstaver: Aa, A, B, C, D osv., og fra vest mot aust med tall: 1, 2, 3 osv. Tallene begynner imidlertid ikke helt fra venstre. I den nordligste rekken av rektangler, Aa, er Aa 1 rektangel nr. 6 fra venstre, Aa 2 nr. 7 osv.; i alle de andre rekkene (A, B, C osv.) betyr 1-tallet 3. rektangel fra venstre, 2-tallet 4. rektangel fra venstre osv., og «finsk A 3» f. eks. betyr således 5. rektangel fra venstre i 2. rekke fra nord. Dette systemet er fastlagt med registeret i bind II. De aller fleste av de stedsnavn det her gjelder, fins i rektanglene Aa 1 ‒ 2, A 1 ‒ 5 og B 2 ‒ 5.\par
Se også rektor J. Qvigstads forord til registeret i bind II side 431.\par
Rettskrivningen i dette registeret er modernisert noe fra bind II, som er trykt for over 30 år siden, i 1929.\par
Registeret til bind I, også personnavnregisteret, er utarbeidd av Ingolf Kvamen og gjennomsett av Kr. Nissen.
\end{document}
